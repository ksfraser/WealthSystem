%
% -----------------------------------------------------------------------------
%
% A license is hereby granted to reproduce this software source code and
% to create executable versions from this source code for personal,
% non-commercial use.  The copyright notice included with the software
% must be maintained in all copies produced.
%
% THIS PROGRAM IS PROVIDED "AS IS". THE AUTHOR PROVIDES NO WARRANTIES
% WHATSOEVER, EXPRESSED OR IMPLIED, INCLUDING WARRANTIES OF
% MERCHANTABILITY, TITLE, OR FITNESS FOR ANY PARTICULAR PURPOSE.  THE
% AUTHOR DOES NOT WARRANT THAT USE OF THIS PROGRAM DOES NOT INFRINGE THE
% INTELLECTUAL PROPERTY RIGHTS OF ANY THIRD PARTY IN ANY COUNTRY.
%
% Copyright (c) 1994-2006, John Conover, All Rights Reserved.
%
% Comments and/or bug reports should be addressed to:
%
%     john@email.johncon.com (John Conover)
%
% -----------------------------------------------------------------------------
%
% Revision: \RCSRevision \\
% Revision Time: \RCSTime UMT \\
% Revision Date: \RCSDate \\
% Revision Id: \RCSId \\
% Revision File: \RCSLog \\
\RCS $Revision: 0.0 $
\RCS $Date: 2006/01/20 04:38:13 $
\RCS $Id: appa.tex,v 0.0 2006/01/20 04:38:13 john Exp $
% $Log: appa.tex,v $
% Revision 0.0  2006/01/20 04:38:13  john
% Initial version
%
%
\chapter{Tutorial on Fractal Time Series}
    \label{tutorial}

    \subidx{time series}{tutorial}
    \subidx{optimization}{betting strategies}
    \subidx{betting strategies}{optimization}
    \idx{speculative markets}
    \idx{economic theory}
    This appendix presents a remedial tutorial on the optimization of
    betting strategies in speculative markets. It is offered in
    academic perspective, and under no circumstances would it be
    appropriate to consider it financial advice. It can serve,
    however, as an introduction to the contemporary economic theory of
    speculative markets. Rigorous and sophisticated approaches that
    address the issues of investing in speculative markets are
    contained in the bibliography.

    \subidx{game}{speculative}
    \idx{coin game}
    \subidx{optimization}{betting strategies}
    \subidx{betting strategies}{optimization}
    This section begins with the analysis of a very simple speculative
    game, that of tossing coins. The analysis will then be expanded by
    permitting the use of unfair coins in the game, roughly
    following~\cite[pp. 128]{Schroeder}. An optimal betting strategy
    will be developed for the game, and this strategy will then be
    generalized and extended to include remedial betting strategies in
    certain speculative markets.

    \section{The Coin Tossing Game}
        \label{coingame}

        \idx{coin game}
        Consider a coin tossing game, where a player makes a wager,
        and then, flips a coin. If the coin comes up heads, then the
        player wins twice the original wager, (ie., makes back the
        original wager plus an amount equal to the original wager from
        the ``bank.'')  But if the coin comes up tails, then the
        player looses the wager to the bank. The game is iterated,
        many times, until the player decides not to play any more, or
        goes ``bust.''

        \subsection{Strategic Considerations in the Iterated Coin Tossing Game}

            \subidx{optimization}{betting strategies}
            \subidx{betting strategies}{optimization}
            The player has an initial cash reserve, to which are added
            the cumulative returns, and from which each wager is made,
            and the cumulative returns will increase by the amount of
            the wager each time the player wins, and, likewise the
            cumulative returns will decrease by the amount of the
            wager each time the player looses.  The objective of the
            player is, obviously, to maximize the magnitude of the
            cumulative returns, over time. Note that this is a
            speculative game, in that the player speculates on the
            likelihood that the coin will come up heads on next
            iteration of the game, and adjusts the wager accordingly,
            betting zero if the outcome of the next coin toss is
            anticipated to be tails.

            \subsubsection{Description of ``Time Series'' and ``Fractal''}

                \subidx{fractal}{time series}
                \subidx{time series}{fractal}
                \begin{description}

                    \item[Time Series:]If the player makes a list,
                    recording the time, and magnitude of cumulative
                    returns, for each iteration of the game, such a
                    table is called the {\it time series}\/ of the
                    cumulative
                    returns~\cite[pp. 223]{Schroeder},~\cite[pp. 199]{Cambel}.


                    \item[Fractal:]If the player plots a graph of the
                    time series, time on the X---axis and magnitude of
                    cumulative returns on the Y---axis, this graph
                    will exhibit {\it fractal}\footnote{Technically,
                    the term, as used here, should be {\it Random
                    Fractal}\/, or {\it Fractional Brownian
                    Motion},~\cite[pp. 170]{Feder}.} characteristics,
                    which means that it represents a system that has
                    the characteristics of a cumulative sum, (ie.,
                    integrative process,) of random events, coin
                    tosses in this
                    case~\cite[pp. 229]{Crownover},~\cite[pp. 163]{Feder}.


                \end{description}

            \subidx{iterated}{game}
            \subidx{game}{iterated}
            \idx{cumulative returns}
            \subidx{cumulative returns}{iterated game}
            \subidx{iterated game}{cumulative returns}
            It is an important concept that the magnitude of the
            cumulative returns, at any time during the iterated game,
            is the cumulative sum of all of the wins and losses of
            wagers made in the previous iterations of the game,
            starting with the initial cash reserves, ie., it is an
            integrative process. If the player does not have ``a
            priory'' knowledge of the outcome of the future coin
            tosses, then optimizing this integrative process is the
            strategic objective of playing a rational game.

            \subidx{cumulative returns}{strategic considerations}
            \subidx{strategic considerations}{cumulative returns}
            For example, it would be foolish for the player to always
            wager zero, since, although there would never be a loss,
            there would never be a win, either. Likewise, it would be
            foolish for the player to wager a large percentage of the
            cumulative returns, since a few losses in succession would
            deplete the cash reserves and cumulative returns to zero,
            and the player would ``go bust,'' thus ending the
            game. Obviously, the player could wager too much, or too
            little of the cumulative returns on a single game
            iteration. The series optimum wagers, is termed the {\it
            optimal betting
            strategy},~\cite[pp. 128]{Schroeder},~\cite[pp. 450]{Reza},~\cite[pp. 270]{Pierce}.


    \section{Optimal Betting Strategy in the Iterated Coin Tossing Game}
        \label{unfaircoingame}

        \idx{unfair coin game}
        \subidx{optimization}{betting strategies}
        \subidx{betting strategies}{optimization}
        If the coin is a fair coin, ie., it has a 50\% chance of
        coming up heads and a 50\% chance of coming up tails, then the
        player should elect not to play. The rationale for this
        statement is that, in the long run, some iterated games will
        be won, and some lost, with the amount of money won equal to
        the amount of money lost---so there is no financial incentive
        to play the game. However, suppose that there is a 60\% chance
        for the coin to come up heads, on any single iteration of the
        game, and a 40\% chance of coming up tails. It turns out that
        the fractal characteristics of the game can be exploited to
        determine the optimal betting strategy\footnote{Fortunately,
        there is a large analytical infrastructure in mathematics and
        economics available that addresses these issues. The answers
        are provided by {\it Information Theory,}\/ and the
        applications of these {\it entropic}\/ principles are a firmly
        entrenched discipline in the field of
        economics~\cite[pp. 127]{Schroeder},~\cite[pp. 450]{Reza},~\cite[pp. 270]{Pierce}.}. The
        optimal betting strategy, in this case, is for the player to
        wager 20\% of the cumulative returns, every iteration of the
        game. As it turns out, this will maximize the growth of the
        player's cumulative
        returns~\cite[pp. 128]{Schroeder},~\cite[pp. 450]{Reza},~\cite[pp. 270]{Pierce}. The
        way that this was computed was from the formula:
        \subidx{optimum strategy}{coin game, formula}

        \begin{equation}
            F = 2P - 1
            \label{bet_optimum1}
        \end{equation}

        \subidx{cumulative returns}{strategic considerations}
        \subidx{strategic considerations}{cumulative returns}
        \noindent where $F$ is the fraction of the player's cumulative
        returns that should be wagered on an iteration of a game with
        a $P$ chance of winning the
        game~\cite[pp. 151]{Schroeder},~\cite[pp. 450]{Reza},~\cite[pp. 270]{Pierce}. In
        the above case, with a 60\%, (ie., $p = 0.6$,) chance of
        winning:

        \begin{equation}
            \label{bet_optimum2}
            F = \left(2 \cdot 0.6\right) - 1
        \end{equation}

        \begin{equation}
            \label{bet_optimum3}
            F = 1.2 - 1
        \end{equation}

        \begin{equation}
            \label{bet_optimum4}
            F = 0.2
        \end{equation}

        \subidx{Shannon}{probability}
        \subidx{probability}{Shannon}
        \idx{binary symmetric channel}

        \noindent or $F$ is 20\% of the player's cumulative
        returns\footnote{This will maximize the logarithmic growth of
        the player's cumulative returns, and is the highest value that
        can be attained, as given by Shannon's {\it information
        capacity}\/, $C(P) = 1 - H(P)$ of a binary symmetric channel
        with an error probability of $P$. Here, $H(P)$ is the entropy
        function, $H(P) = -[P \ln P + (1 - P) \ln (1 -
        P)$~\cite[pp. 128,
        151]{Schroeder},~\cite[pp. 38]{Shannon},~\cite[pp. 114,
        pp. 450]{Reza},~\cite[pp. 270]{Pierce},~\cite[pp. 155]{Klir},~\cite[pp. 9]{Ash}.}. Playing
        this betting strategy, the player can expect an average of 2\%
        increase in the magnitude of cumulative returns on each toss
        of the coin\footnote{This value is computed by taking the
        logarithm to the base $2$, $H(P) = 0.97$ bits per game, and
        $2^C = 1.02$, or 2\% each
        game~\cite[pp. 128]{Schroeder},~\cite[pp. 36]{Shannon},~\cite[pp. 30]{Ash},~\cite[pp. 450]{Reza},~\cite[pp. 270]{Pierce}.}.

        For those wishing to experiment with optimal betting
        strategies, the unfair coin can be simulated with a six sided
        die.  After the wager, the die is rolled, and if the die comes
        up 1, 2, 3, or 4 the player wins. But if it comes up 5 or 6,
        the player loses. The probability of winning, $P$, in this
        case is 0.66 since the player will win 4 times out of 6, on
        average. The optimal wager will be $F = (2 \cdot 0.66) - 1 =
        0.33$, or 33\% of the cumulative returns should be wagered on
        each iteration of the game. It is interesting to play many
        iterations of the game, particularly using different betting
        strategies---for example change the wager fraction to 20\%, or
        40\% of the cumulative returns---and see how the long term
        cumulative returns change in response to the different betting
        strategies.

        \subidx{Shannon}{probability}
        \subidx{probability}{Shannon}
        \subidx{programs}{tsunfaircoin}
        \subidx{tsunfaircoin}{program}
        The program, {\it tsunfaircoin}\/, which is briefly described
        in Appendix~\ref{programs}, uses a random number generator to
        simulate the unfair coin in an iterated coin tossing game. The
        program's command line options control the wager fraction,
        $F$, and the Shannon probability, $P$, and the number of
        iterations of the game.  The cumulative returns for each
        iteration of the game are printed to the terminal, and may be
        plotted, to show that there is indeed an optimum value of
        wager fraction, $F$, for any value of Shannon probability,
        $P$, provided $P$ is greater than $\frac{1}{2}$.

    \section{Important Intuitive Concepts of Speculative Games}

        \idx{unfair coin game}
        \subidx{speculative games}{intuitive concepts}
        \subidx{games}{iterated}
        \idx{cumulative returns}
        The unfair coin tossing game is probably one of the simplest
        speculative games. It is important to develop an intuitive
        concept based on the fundamentals of this simple game.
        Speculative games have the following characteristics:

            \begin{itemize}

                \item Speculative games are iterated. A wager is made
                from the player's cumulative returns for the game, and
                depending on the outcome of the iteration of the game,
                the player either wins or looses the wager for that
                iteration. The winnings or losses, for each iteration,
                are summed to the player's cumulative returns.

                \item The outcome of a particular iteration has random
                characteristics, ie., the outcome of a particular
                iteration is not ``predictable.''

                \item The objective of the game is to maximize the
                value of the player's cumulative returns.

            \end{itemize}

        \idx{capital markets}
        As it turns out, these simple concepts have many applications,
        for example, they can be used to model and analyze the capital
        markets~\cite[pp. 81]{Peters:CAOITCM}.

    \section{An Analytical Approach to the Iterated Unfair Coin Tossing Game}
        \label{analyticalgame}

        \subidx{random mechanism}{speculative games}
        \subidx{speculative games}{random mechanism}
        In Section~\ref{unfaircoingame} it was assumed that the player
        had knowledge about the probability of a tossed coin coming up
        heads. In most speculative games, knowledge of the random
        mechanism is not available. For a simple game, like tossing an
        unfair coin, the coin could be tossed many times, and the
        probability of it coming up heads measured.  The methodology
        would be to toss the coin, say, 100 times, and count how many
        times it came up heads. Say it comes up heads 60 times out of
        the 100 tosses. Then the probability that the coin will come
        up heads on any particular iteration of the game would be
        60\%, and the player could arrange a betting strategy,
        accordingly. It turns out that this concept is very
        extensible.

        \subidx{random mechanism}{speculative games}
        \subidx{speculative games}{random mechanism}
        \idx{capital markets}
        \idx{historical data}
        \subidx{unfair}{games}
        \subidx{games}{unfair}
        \subidx{optimization}{betting strategies}
        \subidx{betting strategies}{optimization}
        In many speculative games, there is no knowledge available
        about the characteristics of the random process of the game.
        As a simple example, assume that no knowledge is available
        about the underlying random process of the unfair coin tossing
        game.  Like the capital markets, we have only historical data
        about the wagering process, ie., what was won, and what was
        lost during each iteration of the game. If we look at the
        historical time series of the game, we would observe that
        since the cumulative returns are increasing, that the game is
        unfair. It would be desirable gain some insight into the
        random process that controls the outcome of an iteration of
        the game, so a betting strategy can be formulated. Referring
        to the preceeding paragraph, when the coin was tossed a
        hundred times to count how many times it came up heads, it
        should be realized that this was a cumulative sum of number of
        times the coin came up heads over a hundred iterations.

        \subidx{unfair}{games}
        \subidx{games}{unfair}
        \subidx{optimization}{betting strategies}
        \subidx{betting strategies}{optimization}
        \subidx{analysis}{speculative games}
        \subidx{speculative games}{analysis}
        Being formal, in $n$ many tosses of the coin, it would be
        expected that the coin came up heads, $P \cdot n$ many times,
        and come up tails, $(1 - P) \cdot n$ many times, where $P$ is
        the probability of the coin coming up heads. If a counting
        process is started, tallied in $C$, by which, if the coin is
        tossed, and it comes up heads, we increase the count by one,
        and if it comes up tails, we decrease the count by one, then
        it would be expected, after $n$ many tosses\footnote{For
        computational reasons, it is advantageous to implement
        counting of the number of heads in a series of coin tosses in
        this manner, which finds the ``average'' $C$ by summing both
        heads and tails, with differing signs. In the unfair coin
        tossing game, the random mechanism can be analyzed by simply
        counting the number of times heads comes up in series of
        tosses.  However, in speculative games, in general, the random
        mechanisms are much more sophisticated, requiring an
        ``average'' to be taken. This methodology provides a means of
        extensibility to these types of systems.}:

        \begin{equation}
            C = P \cdot n - \left(1 - P\right) \cdot n
        \end{equation}

        Notice that $C$ was derived empirically, and from $C$, we can
        compute the probability, $P$, of the coin coming up heads in
        any iteration of the game. Rearranging:

        \begin{equation}
            C = P \cdot n - n  + P \cdot n
        \end{equation}

        \noindent and dividing both sides of the equation by $n$:

        \begin{equation}
            \frac{C}{n} = P - 1 + P = 2 \cdot P - 1
        \end{equation}

        \noindent and solving for $P$:

        \begin{equation}
            P = \frac{\frac{C}{n} + 1}{2}
            \label{p1}
        \end{equation}

        \noindent noting that $\frac{C}{n}$ is the ``average'' $C$.

        The same methodology can be used in general. Access to the
        unfair coin to measure the probability of it coming up heads
        on any iteration is not necessary---this information can be
        deduced from the historical files of a game where the coin was
        used.  For example, we can take the historical time series of
        a unfair coin tossing game, and for each iteration, subtract
        the value of the cumulative returns of the previous iteration
        from the value of the cumulative returns of the next
        iteration, dividing the result of the subtraction by the value
        of the cumulative returns in the previous iteration, making a
        new time series. This is a very powerful concept in the
        strategy of speculative games. The new time series contains
        the fraction of the cumulative returns that was won or lost on
        each iteration of the game.

        \subidx{random mechanism}{speculative games}
        \subidx{speculative games}{random mechanism}
        \subidx{optimization}{betting strategies}
        \subidx{betting strategies}{optimization}
        \subidx{cumulative returns}{fraction wagered}
        Using our example of the unfair coin tossing game, we would
        observe that the new time series would be a list of numbers,
        containing either $+F$, if the wager was won in an iteration,
        or $-F$, if the wager was lost, (assuming that $F$ was
        constant throughout the game.) The important concept here is
        that, given a specific iteration, the fraction of the
        cumulative returns wagered can be deduced, and whether the
        wager was won or lost. It is an important concept that we can
        reconstruct the characteristics of the random mechanism, and
        the fraction of the cumulative returns wagered from the
        historical data of a speculative game, without having
        knowledge of the random mechanism\footnote{It is not a
        complicated concept, actually, if you look at the process by
        which the historical time series was made. A wager is made,
        that is a fraction of the cumulative returns, and the wager
        was either added or subtracted from the cumulative returns for
        the game, depending on the results of a random process. When
        we subtract the value of the cumulative returns of a previous
        iteration from the value of the cumulative returns of the next
        iteration, and dividing by the value of the cumulative returns
        in the previous iteration, we are actually ``undoing'' the
        cumulative returns process of the game---kind of working
        backward to create the underlying random process and betting
        strategy.}. As before, we do a cumulative sum on the random
        game's process, only instead of it being a tossed coin, it is
        the new time series that contains the fraction of the
        cumulative returns that was won or lost in each iteration of
        the game. Formalizing, using Equation~\ref{p1}, and replacing
        $\frac{C}{n}$, the average value of $C$, with the ``average''
        value of $F$, found by summing all of the values in the new
        time series, and dividing by the number of iterations:

       \begin{equation}
           \label{p2}
           P = \frac{1}{2} + \frac{1}{2n}\sum_{i = 0}^{n} F\left(i\right)
       \end{equation}

       \noindent or more generally:

       \begin{equation}
           P = \frac{1}{2} + \frac{1}{2t}\int_0^t F\left(n\right) dn
       \end{equation}

       Interestingly, if we want to find out the fraction of the
       cumulative returns that was wagered each iteration of the game,
       the absolute value of $F(i)$ can be taken in
       Equation~\ref{p2}. In the simple case of the unfair tossed
       coin, it is simply $F$, since $F(i)$ is either $+F$ or $-F$,
       ie., we simply remove the signs, and the equation reduces
       to\footnote{The absolute value of the normalized increments,
       when averaged, is related to the root mean square of the
       increments by a constant. If the normalized increments are a
       fixed increment, the constant is unity. If the normalized
       increments have a Gaussian distribution, the constant is
       $\approx 0.8$ depending on the accuracy of of ``fit'' to a
       Gaussian distribution.}:

       \begin{equation}
           F = 2P - 1
           \label{bet_optimum5}
       \end{equation}

       \noindent which is the same as
       Equation~\ref{bet_optimum1}. Although this is a generalization,
       this derivation has not shown that this is indeed an optimal
       solution. See Section~\ref{GA} in Chapter~\ref{general} for a
       presentation on the optimal solution---it turns out that $F = 2P
       - 1$ is, indeed, the optimal solution.

% Local Variables:
% TeX-parse-self: t
% TeX-auto-save: t
% TeX-master: "fractal.tex"
% End:
