%
% -----------------------------------------------------------------------------
%
% A license is hereby granted to reproduce this software source code and
% to create executable versions from this source code for personal,
% non-commercial use.  The copyright notice included with the software
% must be maintained in all copies produced.
%
% THIS PROGRAM IS PROVIDED "AS IS". THE AUTHOR PROVIDES NO WARRANTIES
% WHATSOEVER, EXPRESSED OR IMPLIED, INCLUDING WARRANTIES OF
% MERCHANTABILITY, TITLE, OR FITNESS FOR ANY PARTICULAR PURPOSE.  THE
% AUTHOR DOES NOT WARRANT THAT USE OF THIS PROGRAM DOES NOT INFRINGE THE
% INTELLECTUAL PROPERTY RIGHTS OF ANY THIRD PARTY IN ANY COUNTRY.
%
% Copyright (c) 1994-2006, John Conover, All Rights Reserved.
%
% Comments and/or bug reports should be addressed to:
%
%     john@email.johncon.com (John Conover)
%
% -----------------------------------------------------------------------------
%
% Revision: \RCSRevision \\
% Revision Time: \RCSTime UMT \\
% Revision Date: \RCSDate \\
% Revision Id: \RCSId \\
% Revision File: \RCSLog \\
\RCS $Revision: 0.0 $
\RCS $Date: 2006/01/20 04:38:13 $
\RCS $Id: chap3.tex,v 0.0 2006/01/20 04:38:13 john Exp $
% $Log: chap3.tex,v $
% Revision 0.0  2006/01/20 04:38:13  john
% Initial version
%
%
\chapter{A Metric Methodology}
    \label{methodology}

    \subidx{increments}{see time series increments}

    This chapter outlines the methodology used in the construction of
    the data presented in appendix~\ref{markets}. The reader is
    assumed to have remedial knowledge of computing concepts,
    statistics, and manipulation of time series data sets.

    \section{General Concepts}
        \label{generalconcepts}

        \idx{time series}
        \idx{cumulative sum}
        \idx{random process}
        Consider a time series, of interest because it appears to have
        exponentially increasing rate of revenue returns, presumably
        as a result of a process similar to that discussed in
        Section~\ref{areturns} in Chapter~\ref{general}, and, further,
        appears to have characteristics of a random process as
        discussed in Chapter~\ref{general} in
        Section~\ref{analyticalgame}. Then, from
        Equation~\ref{iteration1}:

        \begin{eqnarray}
            \frac{R_{n + 1} - R_n}{R_n} & = & F_n \cdot f_n
            \label{iteration2}
        \end{eqnarray}

        \noindent where $\frac{R_{n + 1} - R_n}{R_n}$ can be
        calculated from the time series by the following algorithm:

        \subidx{incremental rate of revenue returns}{calculation}
        \subidx{rate of revenue returns}{incremental, calculation}
        \vspace{0.25in}
        \begin{bf}
            \begin{quotation}

                \vspace{0.1in}\noindent sequentially, for each value
                in the time series:

                \begin{quotation}

                    \vspace{0.1in}\noindent subtract the last value in
                    the time series from the this, the current, value
                    in the time series

                    \vspace{0.1in}\noindent divide this difference by
                    the last value in the time series

                    \vspace{0.1in}\noindent print the quotient

                \end{quotation}

            \end{quotation}
        \end{bf}
        \vspace{0.25in}

        \subidx{incremental returns}{time series}
        \subidx{rate of revenue returns}{decomposition}
        \subidx{programs}{tsfraction}
        \subidx{tsfraction}{program}
        \subidx{programs}{tslsq}
        \subidx{tslsq}{programs}
        \subidx{incremental returns}{average}
        \subidx{incremental returns}{mean}
        \subidx{incremental returns}{least squares fit}
        This will make a new time series, similar to the series shown
        schematically in Figure~\ref{WSFOCS}, and defined by
        Equations~\ref{Fequation} and~\ref{requation}. This new time
        series is the time series of the increments. It is important
        to note that this process ``decomposes'' the fractal time
        series into the time series of the underlying mechanisms that
        created the time series.  The program {\it tsfraction}\/,
        described briefly in appendix~\ref{programs}, can perform this
        function. It is important to note that the new time series
        contains the fraction of the rate of revenue returns, won or
        lost in each iteration of the game, $\frac{R_{n + 1} -
        R_n}{R_n}$. $R_{n + 1} - R_{n}$ is the amount won or lost,
        depending on whether $R_{n + 1}$ is larger, or smaller than
        $R_{n}$, respectively, and dividing this value by $R_n$
        calculates the fraction of the rate of revenue returns won or
        lost in the iteration. An initial assumption of this section
        is that the ``wins'' and ``losses'' are the result of a random
        process. Averaging all values, of $\frac{R_{n + 1} -
        R_n}{R_n}$ will give the average fraction of the rate of
        revenue returns won for all iterations in the time
        series. This can be calculated by finding the mean, perhaps
        using the program {\it tsnormal}\/, or by a least squares fit
        of the new time series, perhaps using the program {\it
        tslsq}\/, both of which are described in
        appendix~\ref{programs}. Additionally, since, from
        Equation~\ref{returnequation}:

        \begin{equation}
            \frac{R_{N}}{R_{0}} = \left(1 + f\right)^{PN} \left(1 - f\right)^{\left(1 - P\right)N}
        \end{equation}

        \noindent for $N$ many records in the time series, and, on the
        average, from Equation~\ref{requation1},

        \begin{equation}
            R\left(t\right) = \left[\left(1 + f\right)^{P} \left(1 - f\right)^{1 - P}\right]^{t}
            \label{requation2}
        \end{equation}

        \noindent where $R(t)$ can be used to find the constants in
        the general form of Equation~\ref{eequation}:

        \begin{equation}
            f\left(t\right) = e^{kt}
        \end{equation}

        \subidx{programs}{tslogreturns}
        \subidx{tslogreturns}{programs}
        \noindent which was an initial assumption in this section. The
        program {\it tslogreturns}\/, or perhaps tslsq using the
        exponential fit argument, -p, can be also be used to find the
        average exponential fit to the cumulative returns represented
        by the original time series.

        \subidx{fractional}{Brownian motion}
        \subidx{Brownian motion}{fractional}
        \subidx{fixed increments}{Brownian motion}
        \subidx{Brownian motion}{fixed increments}
        \subidx{programs}{tslsq}
        \subidx{tslsq}{program}
        \subidx{programs}{tsrms}
        \subidx{tsrms}{program}
        \subidx{incremental returns}{root mean square}
        \subidx{root mean square}{incremental returns}

        Additionally, it is important to note that since the new time
        series, derived above in this section, contains the fraction
        of rate of revenue returns won or lost in each iteration of
        the game, that the absolute value of this time series is the
        time series of the fraction of the rate of revenue returns
        wagered in each iteration of the game\footnote{The absolute
        value of the normalized increments, when averaged, is related
        to the root mean square of the increments by a constant. If
        the normalized increments are a fixed increment, the constant
        is unity. If the normalized increments have a Gaussian
        distribution, the constant is $\approx 0.8$ depending on the
        accuracy of of ``fit'' to a Gaussian distribution.}, assuming
        the original time series has characteristics of fractional
        Brownian motion, or could be ``modeled'' by Brownian motion
        with fixed increments. The absolute value, for each increment
        could be calculated by simply removing the all negative signs,
        and then averaging with, perhaps, the programs {\it
        tsnormal}\/ or {\it tslsq}\/. Alternately, the root mean
        square value of the time series may be calculated, perhaps
        using the program {\it tsrms}\/, which is described in
        appendix~\ref{programs}. The average value or root mean square
        value is the parameter $f$ in Equations~\ref{iteration2}
        and~\ref{requation2}, assuming that $f$ is constant.

        \subidx{probability}{Shannon}
        \subidx{Shannon}{probability}
        \subidx{Shannon probability}{calculating from signs of increments}
        Note, also, that if the total number of records, $N$ is
        sufficiently large, then the probability of a ``win'' in any
        iteration, $P$, can be determined by counting the number of
        positive values in the new time series of the increments, and
        dividing this number by the total number of records, $N$, in
        the time series.

        All of these values, $\frac{R_{n + 1} - R_n}{R_n}$, $f$, and
        $P$, are related by Equation~\ref{pequation}:

        \begin{equation}
            P = \frac{\ln \left(\frac{\frac{R_{n+1}}{R_{n}}}{\left(1 - f\right)}\right)}{\ln \left(\frac{\left(1 + f\right)}{\left(1 - f\right)}\right)}
        \end{equation}

        \subidx{speculative games}{modeling}
        \subidx{programs}{tsnormal}
        \subidx{tsnormal}{program}
        \subidx{histogram}{increments}
        \subidx{time series increments}{histogram}
        \noindent which can be used as a subjective evaluation of how
        accurate the ``model'' is. Additionally, the program {\it
        tsnormal}\/ can be used to plot a histogram of the increments
        for evaluation of the distribution of the increments,
        consistent with the presentation in Chapter~\ref{general},
        Section~\ref{crp}.

        \subidx{probability}{Shannon}
        \subidx{Shannon}{probability}
        \subidx{time series increments}{relation of root mean square, and average or mean}
        As an additional metric for the Shannon probability, $P$, from
        Equation~\ref{avg}:

        \begin{equation}
            avg = rms \left[P - \left(1 - P\right)\right] = rms \left(2P - 1\right)
            \label{metricvalues}
        \end{equation}

        \noindent where $avg$ and $rms$ can be measured, or:

        \begin{equation}
            \frac{avg}{rms} = 2P - 1
        \end{equation}

        \noindent or:

        \begin{equation}
            P = \frac{\frac{avg}{rms} + 1}{2}
            \label{metricvalues1}
        \end{equation}

    \section{Procedure}

        \subidx{time series analysis}{software methodology}
        \subidx{programs}{tsfraction}
        \subidx{tsfraction}{program}
        \subidx{programs}{tslsq}
        \subidx{tslsq}{program}
        \subidx{programs}{tsnormal}
        \subidx{tsnormal}{program}
        \subidx{programs}{tsrms}
        \subidx{tsrms}{program}
        \subidx{programs}{tshurst}
        \subidx{tshurst}{program}
        \subidx{programs}{tshcalc}
        \subidx{tshcalc}{program}
        \subidx{programs}{tsunfairbrownian}
        \subidx{tsunfairbrownian}{program}
        \subidx{programs}{tsshannon}
        \subidx{tsshannon}{program}
        \subidx{programs}{tsshannonmax}
        \subidx{tsshannonmax}{program}
        \subidx{programs}{tslogreturns}
        \subidx{tslogreturns}{program}
        \subidx{tsfraction}{use in analytical methodology}
        \subidx{tslsq}{use in analytical methodology}
        \subidx{tsnormal}{use in analytical methodology}
        \subidx{tsrms}{use in analytical methodology}
        \subidx{tshurst}{use in analytical methodology}
        \subidx{tshcalc}{use in analytical methodology}
        \subidx{tsunfairbrownian}{use in analytical methodology}
        \subidx{tsshannon}{use in analytical methodology}
        \subidx{tsshannonmax}{use in analytical methodology}
        \subidx{tslogreturns}{use in analytical methodology}
        The following procedure is executed in the~..x/markets
        directory from a ``Makefile'' using the Unix utility make(1).
        The programs {\it tsfraction}\/, {\it tslsq}\/, {\it
        tsnormal}\/, {\it tsrms}\/, {\it tshurst}\/, {\it tshcalc}\/,
        {\it tsunfairbrownian}, {\it tsshannonmax}\/, {\it
        tsshannon}\/, and {\it tslogreturns}\/ are described briefly
        in appendix~\ref{programs}, and in addition have online manual
        pages which can be viewed by the Unix utility man(1). In depth
        descriptions of the programs is available in the program
        sources.

        \subidx{time series increments}{root mean square}
        \subidx{time series increments}{mean}
        \subidx{time series increments}{standard deviation}
        Note that many of the parametric values in the analysis of the
        fractal time series data set are derived by different
        methodologies. This is for comparative consistency
        verification. See Section~\ref{verm}.

        \begin{enumerate}

            \item Run the program {\it tsfraction}\/ on the fractal
            time series data set to produce a time series of the
            increments.

            \item Run the program {\it tslsq}\/, with the -p option,
            on the time series of the increments to produce the least
            squares fit formula for the average of the increments in
            the time series of the increments.

            \item Run the program {\it tsnormal}\/, with the -p
            option, on the time series of the increments to produce
            the mean and standard deviation for the average of the
            increments in the time series of the increments.

            \item Run the program {\it tsrms}\/ on the time series of
            the increments to produce a time series of the root mean
            square of the time series of the increments.

            \item Run the program {\it tsrms}\/, with the -p option,
            on the time series of the increments to produce the root
            mean square of the time series of the increments.

            \item Using the Unix utility sed(1), remove any negative
            signs from the time series of the increments to produce a
            time series of the absolute value of the time series of
            the increments.

            \item Run the program {\it tslsq}\/, with the -p option,
            on the time series of the absolute value of the increments
            to produce the least squares fit formula for the absolute
            value of the increments.

            \item Run the program {\it tsnormal}\/, with the -p
            option, on the time series of the absolute value of the
            increments to produce the mean and standard deviation for
            the average of the time series of the absolute value of
            the increments.

            \item Run the program {\it tsnormal}\/, with the options
            -t -s 30, on the time series of the increments to produce
            a time series graph of the bell curve of the distribution
            of the increments in the time series of the increments.

            \item Run the program {\it tsnormal}\/, with the options
            -t -s 30 -f, on the time series of the increments to
            produce a time series graph of the distribution of the
            increments in the time series of the increments.

            \item Run the program {\it tsXsquared}\/ on the
            distribution of the increments to produce a $\chi^2$
            confidence level that the distribution of the increments
            does have a Gaussian distribution.

            \item Run the program {\it tsstatest}\/ on the
            distribution of the increments to produce an estimation of
            the size of the required data set for reasonable accuracy.

            \item Run the program {\it tsderivative}\/ on the time
            series of the increments to produce the first derivative
            of the time series of the increments. Additionally, run
            the program {\it tsnormal}\/, with the options -t -s 30,
            and -t -s 30 -f to produce a time series graph of the
            distribution of the first derivative of the increments.

            \item Run the program {\it tsderivative}\/ on the time
            series of the increments to produce the second derivative
            of the time series of the increments.  Additionally, run
            the program {\it tsnormal}\/, with the options -t -s 30,
            and -t -s 30 -f to produce a time series graph of the
            distribution of the second derivative of the increments.

            \item Run the program {\it tshurst}\/ on the time series
            to produce a graph of the Hurst coefficient of the time
            series.

            \item Run the program {\it tslsq}\/, with the -p option,
            on the graph of the Hurst coefficient of the time series
            to produce the least squares fit formula for the Hurst
            coefficient of the time series.

            \item Run the program {\it tshcalc}\/ on the time series
            to produce a graph of the H parameter of the time series
            of the increments.

            \item Run the program {\it tslsq}\/, with the -p option,
            on the graph of the H parameter of the time series of the
            increments to produce the least squares fit formula for
            the H parameter of the time series of the increments.

            \item Run the program {\it tsunfairbrownian}\/, with the
            -f option and the root mean square value of the time
            series of the increments, on the fractal time series data
            set to produce a simulation of the fractal time series
            data set.

            \item Run the program {\it tsfraction}\/ on the simulation
            of the fractal time series data set to produce a time
            series of the increments of the simulation of the fractal
            time series data set.

            \item Run the program {\it tsnormal}\/, with the -p
            option, on the simulation of the time series of the
            increments to produce the mean and standard deviation for
            the average of the increments in the simulation of the
            time series of the increments.

            \item Run the program {\it tsnormal}\/, with the options
            -t -s 30, on the simulation of the time series of the
            increments to produce a time series graph of the bell
            curve of the distribution of the increments in the
            simulation of the time series of the increments.

            \item Run the program {\it tsnormal}\/, with the options
            -t -s 30 -f, on the simulation of the time series of the
            increments to produce a time series graph of the
            distribution of the increments in the simulation of the
            time series of the increments.

            \item Run the program {\it tsshannonmax}\/ on the fractal
            time series data set to produce a graph of the maximum
            Shannon probability for the fractal time series data set.

            \item Run the program {\it tsshannonmax}\/, with the -p
            option, on the fractal time series data set to produce the
            value of the maximum Shannon probability for the fractal
            time series data set.

            \item Run the program {\it tslogreturns}\/, with the -p
            option, on the fractal time series data set to produce the
            value of the logarithmic returns of the fractal time
            series data set.

            \item Run the program {\it tsshannon}\/ with the value of
            the logarithmic returns of the fractal time series data
            set to produce the value of the Shannon probability for
            the fractal time series data set.

            \item Run the program {\it tslsq}\/, with the -e -p
            options, on the fractal time series data set to produce
            the value of the coefficient of the exponential returns
            for the fractal time series data set.

            \item Run the program {\it tsshannon}\/ with the value of
            the coefficient of the exponential returns of the fractal
            time series data set to produce the value of the Shannon
            probability for the fractal time series data set.

            \item Use the Unix utility egrep(1) with the argument ``-e
            -'' on the time series of the increments to ``filter''
            records containing a negative sign. Pipe this time series
            to the Unix utility wc(1) to produce a count of the
            records in the time series of the increments with negative
            signs.

            \item Use the Unix utility wc(1) on the time series of the
            increments to produce a count of the records in the time
            series of the increments.

            \item Use the Unix utility awk(1) divide the count of the
            records in the time series of the increments with negative
            signs, by the count of the records in the time series of
            the increments, and subtracting from unity, to produce the
            value of the maximum Shannon probability for the time
            series of the increments.

        \end{enumerate}

        In addition, the Unix utility awk(1) is used to parse and
        reformat data from this procedure into {\LaTeX} macros for
        direct import into this manuscript. Appendix~\ref{markets} is
        machine generated.

    \section{Description of the Usage of the Programs in the Procedure}

        \subidx{programs}{tsfraction}
        \subidx{tsfraction}{program}
        The time series of the increments was made using the program
        {\it tsfraction}\/, which essentially implements the formula
        of Equation~\ref{iteration2}, for the fractal time series data
        set.

        \subidx{time series increments}{root mean square}
        The root mean square of the time series of the increments is
        calculated using the program {\it tsrms}\/. The program, also,
        is used to produce a running graph of the root mean square
        process.

        \subidx{time series increments}{mean}
        \subidx{time series increments}{standard deviation}
        The average, ie., mean, and standard deviation of the
        increments of the time series of the increments is calculated
        using the program {\it tsnormal}\/.

        \subidx{time series increments}{root mean square}
        \subidx{time series increments}{mean}
        \subidx{time series increments}{standard deviation}
        \subidx{Shannon}{probability}
        \subidx{probability}{Shannon}
        \subidx{programs}{tsshannonmax}
        \subidx{tsshannonmax}{program}
        \subidx{time series increments}{root mean square, mean, standard deviation, alternative calculations}
        \subidx{time series increments}{mean, calculation}
        \subidx{time series increments}{root mean square, calculation}
        \subidx{time series increments}{standard deviation, calculation}
        The root mean square, mean, and standard deviation of the time
        series of the increments are the metric values of
        Equation~\ref{metricvalues}. The Shannon probability was
        calculated by three different methods. One, using the program
        {\it shannonmax}\/, which produces the maximum Shannon
        probability from the original fractal time series data set;
        another, by counting the number of records with negative signs
        in the time series of the increments; and last, by using the
        program {\it tsshannon}\/, which calculates the Shannon
        probability from the logarithmic returns of the fractal time
        series data set.

        \subidx{programs}{tsnormal}
        \subidx{tsnormal}{program}
        \subidx{Gaussian}{increment property}
        \subidx{increment property}{Gaussian}
        \subidx{time series increments}{normalized histogram, calculation}
        The normalized histogram and bell curve of the time series
        graphs, in each analysis, was made by first using the program
        {\it tsfraction}\/ to find the increments of the time
        series. Then the program {\it tsnormal}\/ was run on the
        increment time series, with 30 intervals, ie., -s 30 argument,
        to provide the data for the bell curve. The program {\it
        tsnormal}\/ was executed again, with -f -s 30 arguments, to
        provide the histogram. See~\cite[pp. 250]{Crownover} for the
        rationale. If the bell curve and the histogram are
        approximately the same, then Gaussian increment property,
        ``property 2'' as described in~\cite[pp. 245]{Crownover}, is
        validated, and the time series probably represents a
        fractional Brownian motion. Additionally, the program is used
        with the options -t -s 30, and -t -s 30 -f to produce a time
        series graph of the distribution of the first and second
        derivative of the increments---these are useful for comparison
        of the distributions with the standard ``white noise'' for a
        ``qualitative'' verification of the cumulative sum process in
        empirical data.

        \subidx{programs}{thshurst}
        \subidx{thshurst}{program}
        \subidx{programs}{tslsq}
        \subidx{tslsq}{program}
        \subidx{Hurst coefficient}{calculation}
        \subidx{time series increments}{Hurst coefficient, calculation}
        The Hurst coefficient graph, in each analysis, was made by
        running the program {\it tshurst}\/ on the time series.
        See~\cite[pp. 153]{Feder},~\cite[pp. 253]{Casti:C},~\cite[pp. 493]{Peitgen},~\cite[pp. 172]{Cambel},~\cite[pp. 62]{Peters:CAOITCM},
        or,~\cite[pp. 129]{Schroeder}, for the rationale. The program
        {\it tslsq}\/ was used on this Hurst coefficient data, with
        the -p argument, to calculate the least squares approximation
        to the Hurst coefficient.

        \subidx{programs}{tshcalc}
        \subidx{thshcalc}{program}
        \subidx{programs}{tslsq}
        \subidx{tslsq}{program}
        \subidx{H parameter}{calculation}
        \subidx{time series increments}{H parameter, calculation}
        The H parameter graph, in each analysis, was made by running
        the program {\it tshcalc}\/ on the time series.
        See~\cite[pp. 249]{Crownover} for the rationale. The program
        {\it tslsq}\/ was used on the H parameter data, with the -p
        argument, to calculate the least squares approximation to the
        H parameter data.

        \subidx{programs}{tslogreturns}
        \subidx{tslogreturns}{program}
        \subidx{programs}{tslsq}
        \subidx{tslsq}{program}
        \subidx{fiscal strategy}{optimum}
        For the optimum fiscal strategy, the program {\it
        tslogreturns}\/ was run on the fractal time series data set
        with the -p option to print the formulas for the logarithmic
        returns. As an alternative, the program {\it tslsq}\/ was used
        with the -e and -p options to print the formulas for a least
        squares exponential fit to the fractal time series data
        set. This renders a slightly more accurate set of formulas,
        but was not used in the analysis to be consistent
        with~\cite[pp. 81]{Peters:CAOITCM}

        \subidx{programs}{tsshannon}
        \subidx{tsshannon}{program}
        \subidx{Shannon}{probability}
        \subidx{probability}{Shannon}
        For the calculation of the Shannon probability, the program
        {\it tsshannon}\/ was run with the formulas derived from the
        {\it tslogreturns}\/ program, above. The formulas were parsed
        with a Bourne shell script, using the Unix stream editor,
        sed(1), and presented to the program {\it tsshannon}\/ via the
        command line.

        \subidx{Shannon}{probability}
        \subidx{probability}{Shannon}
        \subidx{programs}{tsunfairbrownian}
        \subidx{tsunfairbrownian}{program}
        \subidx{fiscal strategy}{simulation}
        \subidx{time series}{simulation}
        \subidx{strategy}{simulation}
        For the simulations, the program {\it tsunfairbrownian}\/ was
        used. This program performs the inverse function of {\it
        tsfraction}\/. Given a Shannon probability, or alternatively,
        the fraction of the cumulative sum to be ``wagered'' on each
        element in the time series of the increments of the fractal
        time series data set, a simulated fractal time series data set
        can be produced that has the ``wager'' altered to the metric
        values calculated above. This simulation can be analyzed using
        the procedure outlined herein, and the characteristics of the
        simulation compared against the original.

        \subidx{Hurst coefficient}{relation to spectral exponent}
        \subidx{spectral exponent}{relation to Hurst coefficient}
        \idx{Fourier analysis}
        In this way, the data analysis and reduction were largely
        automated for each of the individual markets studied. It
        should be reiterated that the data analysis methodology
        presented here is remedial by contemporary standards for such
        issues. A more formal approach has been suggested
        by~\cite[pp. 259]{Crownover} using Fourier analysis do derive
        the spectral exponent of a time series. The program {\it
        tsdft}\/, described briefly in appendix~\ref{programs}, can
        perform this function.  The Hurst coefficient is related to
        the spectral exponent, by the relation:

        \begin{equation} \beta = 2H + 1 \end{equation}

        \noindent where $\beta$ is the Fourier spectral exponent, and
        $H$ is the Hurst
        coefficient,~\cite[pp. 130]{Schroeder},~\cite[pp. 262]{Crownover},~\cite[pp. 207]{Cambel}. This
        methodology is difficult to implement without manual
        intervention, but produces superior accuracies.

        Additionally, the Hurst coefficient is related to the Shannon
        probability of a time series as derived in
        Chapter~\ref{general}. A Shannon probability of 0.5 should
        give a far term Hurst coefficient of 0.5. Other values of
        Shannon probability and Hurst coefficient are related,
        however, there are known accuracy issues with the methodology
        of deriving the Hurst
        coefficient. See~\cite[pp. 156]{Feder},~\cite[pp. 27]{Brock}.

        \idx{fractal analysis}
        \idx{fractal dimension}
        \subidx{fractal dimension}{relation to Hurst coefficient}
        \subidx{Hurst coefficient}{Relation to fractal dimension}
        Additionally, there are methods of fractal analysis that
        address concepts of fractal dimension. The fractal dimension
        of a time series is related to the Hurst coefficient by the
        following
        relationship,~\cite[pp. 196]{Feder},~\cite[pp. 495]{Peitgen}:

        \begin{equation} D = 2 - H \end{equation}

        \noindent where $D$ is the fractal dimension, and $H$ is the
        Hurst coefficient.

    \section{Verification Methodology}
        \label{verm}

        As a cursory verification methodology:

        \begin{enumerate}

            \item Using the mean and root mean square values of the
            normalized increments of the time series data, and the
            Shannon probability as calculated by counting the total
            number of records that the market movement was positive,
            in relation to the total number of records in the data
            set, verify the accuracy of the equality in
            Equation~\ref{metricvalues1}.

            \item Compare the Shannon probability, as found by the
            {\it tsshannonmax}\/ program to the value of the Shannon
            probability as calculated by counting the total number of
            records that the market movement was positive, in relation
            to the total number of records in the data set

            \item Compare the four methods of calculating the
            logarithmic returns:

            \begin{itemize}

                \item By calculation based on the mean of the
                normalized increments.

                \item By the calculation of the constant in the least
                squares approximation to the normalized increments.

                \item By the calculation of the exponential least
                squares fit to the original time series data set, with
                the program {\it tslsq}\/.

                \item By the calculation of the logarithmic returns,
                with the program {\it tslogreturns}\/.

            \end{itemize}

            \item Using the mean, standard deviation, and the root
            mean square of the normalized increments, and the Shannon
            probability as calculated by counting the total number of
            records that the market movement was positive, in relation
            to the total number of records in the data set, verify the
            accuracy of the equality of Equation~\ref{stddev9}.

            \item Compare the accuracy of the equality of the absolute
            value and root mean square of the normalized
            increments\footnote{The absolute value of the normalized
            increments, when averaged, is related to the root mean
            square of the increments by a constant. If the normalized
            increments are a fixed increment, the constant is
            unity. If the normalized increments have a Gaussian
            distribution, the constant is $\approx 0.8$ depending on
            the accuracy of of ``fit'' to a Gaussian distribution.}.

        \end{enumerate}

        Note that the numerical manipulations are relatively simple,
        and can be implemented with simple awk(1) scripts.

% Local Variables:
% TeX-parse-self: t
% TeX-auto-save: t
% TeX-master: "fractal.tex"
% End:
