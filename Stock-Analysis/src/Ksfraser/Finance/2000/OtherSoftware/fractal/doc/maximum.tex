%
% -----------------------------------------------------------------------------
%
% A license is hereby granted to reproduce this software source code and
% to create executable versions from this source code for personal,
% non-commercial use.  The copyright notice included with the software
% must be maintained in all copies produced.
%
% THIS PROGRAM IS PROVIDED "AS IS". THE AUTHOR PROVIDES NO WARRANTIES
% WHATSOEVER, EXPRESSED OR IMPLIED, INCLUDING WARRANTIES OF
% MERCHANTABILITY, TITLE, OR FITNESS FOR ANY PARTICULAR PURPOSE.  THE
% AUTHOR DOES NOT WARRANT THAT USE OF THIS PROGRAM DOES NOT INFRINGE THE
% INTELLECTUAL PROPERTY RIGHTS OF ANY THIRD PARTY IN ANY COUNTRY.
%
% Copyright (c) 1994-2006, John Conover, All Rights Reserved.
%
% Comments and/or bug reports should be addressed to:
%
%     john@email.johncon.com (John Conover)
%
% -----------------------------------------------------------------------------
%
% Revision: \RCSRevision \\
% Revision Time: \RCSTime UMT \\
% Revision Date: \RCSDate \\
% Revision Id: \RCSId \\
% Revision File: \RCSLog \\
\RCS $Revision: 0.0 $
\RCS $Date: 2006/01/20 04:38:13 $
\RCS $Id: maximum.tex,v 0.0 2006/01/20 04:38:13 john Exp $
% $Log: maximum.tex,v $
% Revision 0.0  2006/01/20 04:38:13  john
% Initial version
%
%
    \subsection{Simulation of Fixed Increment Approximation for Optimally Maximal Fiscal Strategy}
        \label{\SETLABEL:MAXSHANNON}
        \subidx{\market}{fiscal strategy, simulation}
        \subidx{\market}{maximum Shannon probability}
        \subidx{markets}{simulation}
        \subidx{simulation}{markets}
        \subidx{strategy}{optimum fiscal, simulation}
        \subidx{fiscal}{optimum strategy, simulation}
        \subidx{programs}{tsunfairbrownian}
        \subidx{tsunfairbrownian}{program}
        \subidx{Shannon}{probability}
        \subidx{probability}{Shannon}

        \subidx{strategy}{exploitable}
        \subidx{exploitable}{strategy}
        \subidx{programs}{tsshannonmax}
        \subidx{tsshannonmax}{program}
        \subidx{programs}{tsunfairbrownian}
        \subidx{tsunfairbrownian}{program}
        \subidx{strategy}{fiscal}
        \subidx{fiscal}{strategy}
        The data in this section is presented in tabular form in
        Section~\ref{\SETLABELREF:MAXSHANNON}. One of the issues of
        analysis, as mentioned in Section~\ref{\SETLABEL:OPS}, is to
        determine the maximum Shannon probability for the time series
        presented in Figure~\ref{\SETLABEL:TS}. Potentially, this
        could be exploited with an aggressive fiscal
        strategy. Figure~\ref{\SETLABEL:SHANNONMAX0} is a graph of the
        output of the {\it tsshannonmax}\/ program, which is described
        briefly in appendix~\ref{programs}. The maximum of this
        function is the maximum Shannon probability for the time
        series data presented in Figure~\ref{\SETLABEL:TS}.
        Figure~\ref{\SETLABEL:SHANNONMAX1} was constructed using {\it
        tsunfairbrownian}\/ program, which is also described in
        appendix~\ref{programs}, with the maximum Shannon probability,
        and the time series data presented in
        Figure~\ref{\SETLABEL:TS}. This represents a ``what if'' the
        investment strategy was changed from a Shannon probability of
        {\shannonlogreturns}, as derived in Section~\ref{\SETLABEL:SP}
        to {\shannonmax}. This process, essentially, extracts the
        random statistical data from the time series presented in
        Figure~\ref{\SETLABEL:TS}, and constructs a new time series,
        using the random statistical data, with a different investment
        strategy.  The program, {\it tsunfairbrownian}\/, essentially,
        constructs the new time series as a Brownian fractal with
        fixed increments.  The ``quality'' of such a reconstruction
        should be subject to adequate scepticism and scrutiny since,
        in all probability, the increments in the original data
        represent a relatively complex process, that may not be
        ``modeled'' with such a simple methodology.

        \begin{figure}[ht]
            \begin{center}
                \begin{minipage}[t]{0.45\textwidth}
                    \epsfxsize=1.0\linewidth
                    \epsffile{\directory/data.tsshannonmax.eps}
                    \caption[{\market}, maximum rate of revenue
                        returns] {{\market}, maximum rate of revenue
                        returns, per {\timescale}, vs. Shannon
                        probability. The maximum rate of revenue
                        returns, per {\timescale}, occurs at a Shannon
                        probability of {\shannonmax}.}
                    \label{\SETLABEL:SHANNONMAX0}
                \end{minipage}
                \hfill
                \begin{minipage}[t]{0.45\textwidth}
                    \epsfxsize=1.0\linewidth
                    \epsffile{\directory/data.tsshannonmax-p.tsunfairbrownian-p.eps}
                    \caption[{\market}, maximum rate of revenue
                        returns] {{\market}, maximum rate of revenue
                        returns, per {\timescale}, at a Shannon
                        probability, of {\shannonmax}, corresponding
                        to a ``wager'' fraction of {\twoponemax}.}
                    \label{\SETLABEL:SHANNONMAX1}
                \end{minipage}
            \end{center}
        \end{figure}

        \subidx{fractional}{Brownian motion}
        \subidx{Brownian motion}{fractional}
        \subidx{Shannon}{probability}
        \subidx{probability}{Shannon}
        \subidx{programs}{tsshannonmax}
        \subidx{tsshannonmax}{program}
        If it is assumed that the time series data set, presented in
        Figure~\ref{\SETLABEL:TS}, constitutes classical Brownian
        motion, then the Shannon probability can be calculated by
        counting the total number of {\timescale}s that the {\market}
        movement was positive, and dividing by the total number of
        {timescale}s represented in the time series. This quotient is
        {\pmax}, as compared with the predicted value from the program
        {\it tsshannonmax}\/ of {\shannonmax}.

% Local Variables:
% TeX-parse-self: t
% TeX-auto-save: t
% TeX-master: "fractal.tex"
% End:
