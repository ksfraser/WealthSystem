%
% -----------------------------------------------------------------------------
%
% A license is hereby granted to reproduce this software source code and
% to create executable versions from this source code for personal,
% non-commercial use.  The copyright notice included with the software
% must be maintained in all copies produced.
%
% THIS PROGRAM IS PROVIDED "AS IS". THE AUTHOR PROVIDES NO WARRANTIES
% WHATSOEVER, EXPRESSED OR IMPLIED, INCLUDING WARRANTIES OF
% MERCHANTABILITY, TITLE, OR FITNESS FOR ANY PARTICULAR PURPOSE.  THE
% AUTHOR DOES NOT WARRANT THAT USE OF THIS PROGRAM DOES NOT INFRINGE THE
% INTELLECTUAL PROPERTY RIGHTS OF ANY THIRD PARTY IN ANY COUNTRY.
%
% Copyright (c) 1994-2006, John Conover, All Rights Reserved.
%
% Comments and/or bug reports should be addressed to:
%
%     john@email.johncon.com (John Conover)
%
% -----------------------------------------------------------------------------
%
% Revision: \RCSRevision \\
% Revision Time: \RCSTime UMT \\
% Revision Date: \RCSDate \\
% Revision Id: \RCSId \\
% Revision File: \RCSLog \\
\RCS $Revision: 0.0 $
\RCS $Date: 2006/01/09 04:38:13 $
\RCS $Id: appc.tex,v 0.0 2006/01/09 04:38:13 john Exp $
% $Log: appc.tex,v $
% Revision 0.0  2006/01/09 04:38:13  john
% Initial version
%
%
% The following are numerical data macros-the data is inserted by the
% the input of the parameters.tex files-these files were made during the
% processing of the data for each market place in ../markets, with the
% execution of the maketex.awk awk script, which is as follows:
%
% -----------------------------------------------------------------------------
%
% Make the LaTeX parameters for this market
%
% Input must be successive records that contain:
%
%     0) the mean of the file data.tsfraction, from file
%         "data.tsfraction.tsnormal-p.mean"
%     1) the standard deviation of the file data.tsfraction, from the file,
%         "data.tsfraction.tsnormal-p.stddev"
%     2) the root mean square of the file data.tsfraction, from file
%         "data.tsfraction.tsrms-p"
%     3) the fraction of cumulative returns wagered, from file
%         "data.tsfraction.abs.tsnormal-p.mean"
%     4) the standard deviation of cumulative returns wagered, from the
%         the file "data.tsfraction.abs.tsnormal-p.stddev"
%     5) the constant in the least squares approximation of the file
%         data.tsfraction, from the file "data.tsfraction.tslsq-p.constant"
%     6) the slope in the least squares approximation of the file
%         data.tsfraction, from the file "data.tsfraction.tslsq-p.slope"
%     7) the constant in the least squares approximation of the file
%         data.tsfraction.abs , from the file
%         "data.tsfraction.abs.tslsq-p.constant"
%     8) the slope in the least squares approximation of the file
%         data.tsfraction.abs, from the file
%         "data.tsfraction.abs.tslsq-p.slope"
%     9) the hurst coefficient, from file
%        "data.tsfraction.tshurst-d.tslsq-p.hurstall"
%     10) the hurst coefficient for the lower end of the graph, from file
%         "data.tsfraction.tshurst-d.tslsq-p.low.hurstlow"
%     11) the h parameter, from file
%         "data.tsfraction.tshcalc-d.tslsq-p.hcalcall"
%     12) the h parameter for the lower end end of the graph, from file
%         "data.tsfraction.tshcalc-d.tslsq-p.low.hcalclow"
%     13) the maximum value of Shannon probability, from file
%         "data.tsshannonmax-p.max"
%     14) the logrithmic returns using tslogreturns, from file
%         "data.tslogreturns-p.tsshannon.returns"
%     15) the count of records with negative signes in data.tsfraction, from
%         the file "data.tsfraction.pmaxnumerator"
%     16) the count of records in data.tsfraction, from the file
%         "data.tsfraction.pmaxdenominator"
%     17) the mean of the file tsunfairbrownian-f.tsfraction, from file
%         "tsunfairbrownian-f.fraction.mean"
%     18) the standard deviation of the file tsunfairbrownian-f.tsfraction,
%         from file "tsunfairbrownian-f.fraction.mean"
%     19) the Shannon probability using tslogreturns, from file
%         "data.tslogreturns-p.tsshannon.probability"
%     20) the logrithmic returns in bits from the file
%         "data.tslsq-e-p.bits"
%     21) the traditional hurst coefficient, from file
%        "data.tshurst.tslsq-p.hurstall"
%     22) the traditional hurst coefficient for the lower end of the graph,
%         from file "data.tshurst.tslsq-p.low.hurstlow"
%     23) the traditional h parameter, from file
%         "data.tshcalc.tslsq-p.hcalcall"
%     24) the traditional h parameter for the lower end end of the graph,
%         from file "data.tshcalc.tslsq-p.low.hcalclow"
%     25) the chi-squared value, from file "chisquared"
%     26) the critical value for the chi-squared value, from file
%         "critical"
%{
%
%    if (linectr == 0)
%    {
%        datafractionmean = $0
%        printf ("\\renewcommand{\\datafractionmean}{%f}\n", datafractionmean)
%        datafractionmeanbits = log (datafractionmean + 1) / log (2.0)
%        printf ("\\renewcommand{\\datafractionmeanbits}{%f}\n", datafractionmeanbits)
%        datafractionmeanq = datafractionmean / 3.0
%        printf ("\\renewcommand{\\datafractionmeanq}{%f}\n", datafractionmeanq)
%        datafractionmeanbitsq = log (datafractionmeanq + 1) / log (2.0)
%        printf ("\\renewcommand{\\datafractionmeanbitsq}{%f}\n", datafractionmeanbitsq)
%    }
%
%    if (linectr == 1)
%    {
%        datafractionstddev = $0
%        printf ("\\renewcommand{\\datafractionstddev}{%f}\n", datafractionstddev)
%    }
%
%    if (linectr == 2)
%    {
%        datafractionrms = $0
%        printf ("\\renewcommand{\\datafractionrms}{%f}\n", datafractionrms)
%        avgrms = ((datafractionmean / datafractionrms) + 1.0) / 2
%        printf ("\\renewcommand{\\avgrms}{%f}\n", avgrms)
%        ncompanies = datafractionmean / (datafractionrms * datafractionrms)
%        printf ("\\renewcommand{\\ncompanies}{%f}\n", ncompanies)
%        pncompanies = ((datafractionmean / (sqrt (ncompanies) * datafractionrms)) + 1.0) / 2.0
%        printf ("\\renewcommand{\\pncompanies}{%f}\n", pncompanies)
%    }
%
%    if (linectr == 3)
%    {
%        datafractionabsmean = $0
%        printf ("\\renewcommand{\\datafractionabsmean}{%f}\n", datafractionabsmean)
%    }
%
%    if (linectr == 4)
%    {
%        datafractionabsstddev = $0
%        printf ("\\renewcommand{\\datafractionabsstddev}{%f}\n", datafractionabsstddev)
%    }
%
%    if (linectr == 5)
%    {
%        datafractionconstant = $0
%        printf ("\\renewcommand{\\datafractionconstant}{%f}\n", datafractionconstant)
%        datafractionconstantbits = log (datafractionconstant + 1) / log (2.0)
%        printf ("\\renewcommand{\\datafractionconstantbits}{%f}\n", datafractionconstantbits)
%        datafractionconstantq = datafractionconstant / 3.0
%        printf ("\\renewcommand{\\datafractionconstantq}{%f}\n", datafractionconstantq)
%        datafractionconstantbitsq = log (datafractionconstantq + 1) / log (2.0)
%        printf ("\\renewcommand{\\datafractionconstantbitsq}{%f}\n", datafractionconstantbitsq)
%    }
%
%    if (linectr == 6)
%    {
%        datafractionslope = $0
%        printf ("\\renewcommand{\\datafractionslope}{%f}\n", datafractionslope)
%    }
%
%    if (linectr == 7)
%    {
%        datafractionabsconstant = $0
%        printf ("\\renewcommand{\\datafractionabsconstant}{%f}\n", datafractionabsconstant)
%    }
%
%    if (linectr == 8)
%    {
%        datafractionabsslope = $0
%        printf ("\\renewcommand{\\datafractionabsslope}{%f}\n", datafractionabsslope)
%    }
%
%    if (linectr == 9)
%    {
%        hurstall = $0
%        printf ("\\renewcommand{\\hurstall}{%f}\n", hurstall)
%    }
%
%    if (linectr == 10)
%    {
%        hurstlow = $0
%        printf ("\\renewcommand{\\hurstlow}{%f}\n", hurstlow)
%        hurstlowtwo = hurstlow * 2.0
%        printf ("\\renewcommand{\\hurstlowtwo}{%f}\n", hurstlowtwo)
%        hurstlowhundred = hurstlow * 100.0
%        printf ("\\renewcommand{\\hurstlowhundred}{%f}\n", hurstlowhundred)
%    }
%
%    if (linectr == 11)
%    {
%        hcalcall = $0
%        printf ("\\renewcommand{\\hcalcall}{%f}\n", hcalcall)
%    }
%
%    if (linectr == 12)
%    {
%        hcalclow = $0
%        printf ("\\renewcommand{\\hcalclow}{%f}\n", hcalclow)
%    }
%
%    if (linectr == 13)
%    {
%        shannonmax = $0
%        printf ("\\renewcommand{\\shannonmax}{%f}\n", shannonmax)
%        twoponemax = 2.0 * shannonmax - 1
%        printf ("\\renewcommand{\\twoponemax}{%f}\n", twoponemax)
%    }
%
%    if (linectr == 14)
%    {
%        logreturns = $0
%        printf ("\\renewcommand{\\logreturns}{%f}\n", logreturns)
%        twologreturns = exp (logreturns * log (2.0))
%        printf ("\\renewcommand{\\twologreturns}{%f}\n", twologreturns)
%        twologreturnshundred = (twologreturns - 1.0) * 100.0
%        printf ("\\renewcommand{\\twologreturnshundred}{%f}\n", twologreturnshundred)
%        oneoverlogreturns = 1.0 / logreturns
%        printf ("\\renewcommand{\\oneoverlogreturns}{%f}\n", oneoverlogreturns)
%    }
%
%    if (linectr == 15)
%    {
%        pmaxnumerator = $0
%    }
%
%    if (linectr == 16)
%    {
%        pmaxdenominator = $0
%        pmax = (pmaxdenominator - pmaxnumerator) / pmaxdenominator
%        if (pmax == 1)
%        {
%            pmax =0.99999
%        }
%        printf ("\\renewcommand{\\pmax}{%f}\n", pmax)
%        twopminusone = (2 * pmax) - 1
%        printf ("\\renewcommand{\\twopminusone}{%f}\n", twopminusone)
%        rmsp = datafractionrms * twopminusone
%        printf ("\\renewcommand{\\rmsp}{%f}\n", rmsp)
%        twopx = ((2 * pmax) - 1) / (2 * sqrt (pmax * (1 - pmax)))
%        printf ("\\renewcommand{\\twopx}{%f}\n", twopx)
%        sigmap = datafractionstddev * twopx
%        printf ("\\renewcommand{\\sigmap}{%f}\n", sigmap)
%
%    }
%
%    if (linectr == 17)
%    {
%        tsunfairbrownianfractionmean = $0
%        printf ("\\renewcommand{\\tsunfairbrownianfractionmean}{%f}\n", tsunfairbrownianfractionmean)
%    }
%
%    if (linectr == 18)
%    {
%        tsunfairbrownianfractionstddev = $0
%        printf ("\\renewcommand{\\tsunfairbrownianfractionstddev}{%f}\n", tsunfairbrownianfractionstddev)
%    }
%
%    if (linectr == 19)
%    {
%        shannonlogreturns = $0
%        printf ("\\renewcommand{\\shannonlogreturns}{%f}\n", shannonlogreturns)
%        shannonlogreturnshundred = shannonlogreturns * 100.0
%        printf ("\\renewcommand{\\shannonlogreturnshundred}{%f}\n", shannonlogreturnshundred)
%        twopone =  (2.0 * shannonlogreturns) - 1.0
%        printf ("\\renewcommand{\\twopone}{%f}\n", twopone)
%        twoponehundred = twopone * 100.0
%        printf ("\\renewcommand{\\twoponehundred}{%f}\n", twoponehundred)
%        hundredtwoponehundred = 100.0 - twoponehundred
%        printf ("\\renewcommand{\\hundredtwoponehundred}{%f}\n", hundredtwoponehundred)
%        hundredshannonlogreturnshundred = 100.0 - shannonlogreturnshundred
%        printf ("\\renewcommand{\\hundredshannonlogreturnshundred}{%f}\n", hundredshannonlogreturnshundred)
%    }
%
%    if (linectr == 20)
%    {
%        datatslsqepbits = $0
%        printf ("\\renewcommand{\\datatslsqepbits}{%f}\n", datatslsqepbits)
%    }
%
%    if (linectr == 21)
%    {
%        thurstall = $0
%        printf ("\\renewcommand{\\thurstall}{%f}\n", thurstall)
%    }
%
%    if (linectr == 22)
%    {
%        thurstlow = $0
%        printf ("\\renewcommand{\\thurstlow}{%f}\n", thurstlow)
%        thurstlowtwo = thurstlow * 2.0
%        printf ("\\renewcommand{\\thurstlowtwo}{%f}\n", thurstlowtwo)
%        thurstlowhundred = thurstlow * 100.0
%        printf ("\\renewcommand{\\thurstlowhundred}{%f}\n", thurstlowhundred)
%    }
%
%    if (linectr == 23)
%    {
%        thcalcall = $0
%        printf ("\\renewcommand{\\thcalcall}{%f}\n", thcalcall)
%    }
%
%    if (linectr == 24)
%    {
%        thcalclow = $0
%        printf ("\\renewcommand{\\thcalclow}{%f}\n", thcalclow)
%    }
%
%    if (linectr == 25)
%    {
%        chisquared = $0
%        printf ("\\renewcommand{\\chisquared}{%f}\n", chisquared)
%    }
%
%    if (linectr == 26)
%    {
%        critical = $0
%        printf ("\\renewcommand{\\critical}{%f}\n", critical)
%    }
%
%    linectr++
%}
\newcommand{\LABPRE}{}
\newcommand{\LABPREREF}{}
\newcommand{\market}{}
\newcommand{\directory}{}
\newcommand{\timescale}{}
\newcommand{\SETLABEL}{}
\newcommand{\SETLABELQ}{}
\newcommand{\SETLABELREF}{}
\newcommand{\datafractionmean}{0.0}
\newcommand{\datafractionmeanbits}{0.0}
\newcommand{\datafractionmeanq}{0.0}
\newcommand{\datafractionmeanbitsq}{0.0}
\newcommand{\datafractionstddev}{0.0}
\newcommand{\datafractionrms}{0.0}
\newcommand{\avgrms}{0.0}
\newcommand{\ncompanies}{0.0}
\newcommand{\pncompanies}{0.0}
\newcommand{\datafractionabsmean}{0.0}
\newcommand{\datafractionabsstddev}{0.0}
\newcommand{\datafractionconstant}{0.0}
\newcommand{\datafractionconstantbits}{0.0}
\newcommand{\datafractionconstantq}{0.0}
\newcommand{\datafractionconstantbitsq}{0.0}
\newcommand{\datafractionslope}{0.0}
\newcommand{\datafractionabsconstant}{0.0}
\newcommand{\datafractionabsslope}{0.0}
\newcommand{\hurstall}{0.0}
\newcommand{\hurstlow}{0.0}
\newcommand{\hurstlowtwo}{0.0}
\newcommand{\hurstlowhundred}{0.0}
\newcommand{\hcalcall}{0.0}
\newcommand{\hcalclow}{0.0}
\newcommand{\shannonlogreturns}{0.0}
\newcommand{\shannonlogreturnshundred}{0.0}
\newcommand{\hundredshannonlogreturnshundred}{0.0}
\newcommand{\datatslsqepbits}{0.0}
\newcommand{\twopone}{0.0}
\newcommand{\twoponehundred}{0.0}
\newcommand{\hundredtwoponehundred}{0.0}
\newcommand{\logreturns}{0.0}
\newcommand{\twologreturns}{0.0}
\newcommand{\twologreturnshundred}{0.0}
\newcommand{\oneoverlogreturns}{0.0}
\newcommand{\shannonmax}{0.0}
\newcommand{\twoponemax}{0.0}
\newcommand{\pmax}{0.0}
\newcommand{\twopminusone}{0.0}
\newcommand{\rmsp}{0.0}
\newcommand{\twopx}{0.0}
\newcommand{\sigmap}{0.0}
\newcommand{\tsunfairbrownianfractionmean}{0.0}
\newcommand{\tsunfairbrownianfractionstddev}{0.0}
\newcommand{\thurstall}{0.0}
\newcommand{\thurstlow}{0.0}
\newcommand{\thurstlowtwo}{0.0}
\newcommand{\thurstlowhundred}{0.0}
\newcommand{\thcalcall}{0.0}
\newcommand{\thcalclow}{0.0}
\newcommand{\chisquared}{0.0}
\newcommand{\critical}{0.0}
%
% Following is a work around for an issue with lacheck(1). The program
% will not {\footnotesize\begin{verbatim}
For a mean of 0.039960, with a confidence level of 0.900000
    that the error did not exceed 0.003996, 6778 samples would be required.
    (With 5000 samples, the estimated error is 0.004652 = 11.642514 percent.)
For a standard deviation of 0.200000, with a confidence level of 0.900000
    that the error did not exceed 0.020000, 136 samples would be required.
    (With 5000 samples, the estimated error is 0.003290 = 1.644854 percent.)
\end{verbatim}}
, so it
% is made into a macro. The idea is that since lacheck(1) does not
% expand macros, it won't see the \input{..}. Unfortunately, this
% means that the file will not be checked. To remove it, search for
% XXX, and replace \XXX with \input.
%
\newcommand{\XXX}{\input}
%
\chapter{Fractal Analysis of Various Market Segments in the North American Electronics Industry}
    \label{markets}

    \subidx{markets}{analysis}
    \subidx{analysis}{markets}
    \subidx{strategy}{optimum fiscal}
    \subidx{fiscal}{optimum strategy}
    \subidx{industry}{electronics}
    \subidx{electronics}{industry}
    \subidx{tsunfairbrownian}{program}
    \subidx{programs}{tsunfairbrownian}
    This appendix presents a remedial analysis on the optimization of
    fiscal strategies in various market segments in the North American
    electronics industry. It is offered in academic perspective, and
    under no circumstances would it be appropriate to consider it
    financial advice. It can serve, however, as an illustrative method
    for comparative analysis of various market segments.  Rigorous and
    sophisticated approaches that address the issues of financial
    strategies in industrial markets are contained in the
    bibliography. The analysis of the Dow Jones Average, United States
    Gross Domestic Product, United States M2, United States Leading
    Economic Indicators, and United States Employment Figures, and
    United States Treasury Bill Returns, are presented for comparative
    purposes\footnote{One of the reasons that these are included in
    the analysis is for reasons of scientific induction. The reasoning
    is as follows. Since the electronics industry is one of the major
    industries in the United States, fluctuations in the rate of
    revenue returns of the industry should have correlations in the
    total production of the United States, flow of money, which can be
    related by the GNP and M2, leading indicators, and employment
    figures.  Of course, bonds should have an
    anti-correlation. Additionally, it would seem that a company's
    equity value, represented by its stock evaluation would rise
    exponentially as the industry's rate of revenue returns increased
    exponentially---and this should be reflected in the aggregate
    industry stock index. The intent was to investigate the
    correlations in the normalized increments in the decomposition of
    the time series for each of the macro economic entities. Whether
    such a correlation can be induced remains conjecture.}---although
    the optimum fiscal strategies were derived, these optimums may
    have no real meaning, interpretation, or significance for other
    than comparative purposes with the rather large research already
    done by others on these time series.  The coin tossing games are
    presented for ``theoretical'' comparison of the characteristics of
    Brownian motion, and regression testing, as are the constructions
    using the program {\it tsunfairbrownian}\/, etc., and are useful
    in evaluating software system correctness.  Additionally, note
    that the fiscal strategies that are derived in each case, are the
    financial strategies that will do at least as well as the rest of
    the industry, in the long run, and may not, necessarily, be the
    maximal strategy if the rest of the industry is not maximally
    optimum---ie., it is commensurate with the industry as a
    whole. Additionally, it should be noted that the amount of data,
    from various sources, that was analyzed in each market section was
    very sparce,
    see~\cite[pp. 179]{Feder},~\cite[pp. 83]{Peters:CAOITCM}. The
    reader is urged to use caution when judging the accuracy of these
    presentations.

    For the analysis, the data for the various market segments was in
    the directory~../market, the simulation programs where in the
    directory~../simulation, and the utility programs in the
    directory~../utilities. A brief description of the programs
    appears in Appendix~\ref{programs}, and the methodology used is
    described in Chapter~\ref{methodology}. To add a new market
    segment to the analysis, make a new directory in~../market, and
    copy all of the files from any other directory into the new
    directory. The file, named ``data,'' should contain the market
    time series, with a syntax that is consistent with the program
    {\it tsfraction}\/, which is described briefly in
    appendix~\ref{programs}. Several simulation files are created
    during the analysis, for example
    ``data\-.tsshannonmax-p\-.tsunfairbrownian-p,'' which may be
    re-analyzed by the same method.

    The data presented in this appendix is presented in in condensed
    tabular form in appendix~\ref{tables}.

    \renewcommand{\LABPRE}{C} % if this is changed, change \newcommand{\LABPRETWO}{C} in chap2.tex, also.
    \renewcommand{\LABPREREF}{D}

    \renewcommand{\market}{North American Integrated Circuit Market}
    \renewcommand{\directory}{../markets/ic.namerica}
    \renewcommand{\datafractionmean}{0.008052}
\renewcommand{\datafractionmeanbits}{0.011570}
\renewcommand{\datafractionmeanq}{0.002684}
\renewcommand{\datafractionmeanbitsq}{0.003867}
\renewcommand{\datafractionstddev}{0.038579}
\renewcommand{\datafractionrms}{0.039311}
\renewcommand{\avgrms}{0.602414}
\renewcommand{\ncompanies}{5.210454}
\renewcommand{\pncompanies}{0.544866}
\renewcommand{\datafractionabsmean}{0.029745}
\renewcommand{\datafractionabsstddev}{0.025769}
\renewcommand{\datafractionconstant}{0.010041}
\renewcommand{\datafractionconstantbits}{0.014414}
\renewcommand{\datafractionconstantq}{0.003347}
\renewcommand{\datafractionconstantbitsq}{0.004821}
\renewcommand{\datafractionslope}{-0.000021}
\renewcommand{\datafractionabsconstant}{0.035145}
\renewcommand{\datafractionabsslope}{-0.000057}
\renewcommand{\hurstall}{0.659558}
\renewcommand{\hurstlow}{0.707509}
\renewcommand{\hurstlowtwo}{1.415018}
\renewcommand{\hurstlowhundred}{70.750900}
\renewcommand{\hcalcall}{0.184942}
\renewcommand{\hcalclow}{0.102042}
\renewcommand{\shannonmax}{0.604167}
\renewcommand{\twoponemax}{0.208334}
\renewcommand{\logreturns}{0.010456}
\renewcommand{\twologreturns}{1.007274}
\renewcommand{\twologreturnshundred}{0.727387}
\renewcommand{\oneoverlogreturns}{95.638868}
\renewcommand{\pmax}{0.602094}
\renewcommand{\twopminusone}{0.204188}
\renewcommand{\rmsp}{0.008027}
\renewcommand{\twopx}{0.208583}
\renewcommand{\sigmap}{0.008047}
\renewcommand{\tsunfairbrownianfractionmean}{0.007862}
\renewcommand{\tsunfairbrownianfractionstddev}{0.038619}
\renewcommand{\shannonlogreturns}{0.560125}
\renewcommand{\shannonlogreturnshundred}{56.012500}
\renewcommand{\twopone}{0.120250}
\renewcommand{\twoponehundred}{12.025000}
\renewcommand{\hundredtwoponehundred}{87.975000}
\renewcommand{\hundredshannonlogreturnshundred}{43.987500}
\renewcommand{\datatslsqepbits}{0.007623}
\renewcommand{\thurstall}{0.633980}
\renewcommand{\thurstlow}{0.710108}
\renewcommand{\thurstlowtwo}{1.420216}
\renewcommand{\thurstlowhundred}{71.010800}
\renewcommand{\thcalcall}{0.247886}
\renewcommand{\thcalclow}{0.171737}
\renewcommand{\chisquared}{2.862000}
\renewcommand{\critical}{42.557000}

    \renewcommand{\timescale}{quarter}
    \subidx{market}{\market}
    \idx{\market}

    \section{\market}

        \renewcommand{\SETLABEL}{\LABPRE:NAICM}
        \renewcommand{\SETLABELQ}{\LABPRE:NAICMQ}
        \label{\SETLABEL}
        \renewcommand{\SETLABELREF}{\LABPREREF:NAICM}

        \idx{Semiconductor Industry Association}
        For the analysis, the data was in the directory
        {\directory}\footnote{Data from the Semiconductor Industry
        Association, 1979---1994, by {\timescale}s, in millions of
        dollars, US.}.

        The data in this section is presented in tabular form in
        Section~\ref{\SETLABELREF}.

        %
% -----------------------------------------------------------------------------
%
% A license is hereby granted to reproduce this software source code and
% to create executable versions from this source code for personal,
% non-commercial use.  The copyright notice included with the software
% must be maintained in all copies produced.
%
% THIS PROGRAM IS PROVIDED "AS IS". THE AUTHOR PROVIDES NO WARRANTIES
% WHATSOEVER, EXPRESSED OR IMPLIED, INCLUDING WARRANTIES OF
% MERCHANTABILITY, TITLE, OR FITNESS FOR ANY PARTICULAR PURPOSE.  THE
% AUTHOR DOES NOT WARRANT THAT USE OF THIS PROGRAM DOES NOT INFRINGE THE
% INTELLECTUAL PROPERTY RIGHTS OF ANY THIRD PARTY IN ANY COUNTRY.
%
% Copyright (c) 1994-2006, John Conover, All Rights Reserved.
%
% Comments and/or bug reports should be addressed to:
%
%     john@email.johncon.com (John Conover)
%
% -----------------------------------------------------------------------------
%
% Revision: \RCSRevision \\
% Revision Time: \RCSTime UMT \\
% Revision Date: \RCSDate \\
% Revision Id: \RCSId \\
% Revision File: \RCSLog \\
\RCS $Revision: 0.0 $
\RCS $Date: 2006/01/20 04:38:13 $
\RCS $Id: fraction.tex,v 0.0 2006/01/20 04:38:13 john Exp $
% $Log: fraction.tex,v $
% Revision 0.0  2006/01/20 04:38:13  john
% Initial version
%
%
    \subsection{Time Series Increments Analysis}
        \label{\SETLABEL:TSA}

        \subidx{\market}{Time series analysis}
        \subidx{time series}{increments}
        \subidx{time series}{analysis}
        \subidx{cumulative sum}{analysis}
        \subidx{analysis}{cumulative sum}
        \subidx{analysis}{random process}
        \subidx{random process}{analysis}
        \subidx{Gaussian}{increments}
        \subidx{increments}{Gaussian}
        \subidx{Brownian}{motion, fractional}
        \subidx{fractional}{Brownian motion}
        \subidx{fractal}{Brownian motion}
        The data in this section is presented in tabular form in
        Section~\ref{\SETLABELREF:TSA}.  Figure~\ref{\SETLABEL:TS} is
        a graph of the time series data for the {\market}.

        \subidx{increments}{normalized}
        \subidx{normalized}{increments}
        \subidx{programs}{tsfraction}
        \subidx{tsfraction}{program}
        Figure~\ref{\SETLABEL:TF} is a graph of the normalized
        increments of the time series data presented in
        Figure~\ref{\SETLABEL:TS}. The data presented was made by
        running the program {\it tsfraction}\/ on the time series
        data. The program {\it tsfraction}\/ is described briefly in
        Appendix~\ref{programs}, and subtracts the previous value from
        the next value, dividing this difference by the previous
        value, for each element in the time series data. The new time
        series contains the instantaneous change in the rate of
        revenue returns, divided by the magnitude of the instantaneous
        rate of revenue returns.

        \subidx{mean}{standard deviation}
        \subidx{standard deviation}{mean}
        \idx{root mean square}
        \idx{least squares approximation}
        \begin{figure}[ht]
            \begin{center}
                \begin{minipage}[t]{0.45\textwidth}
                    \epsfxsize=1.0\linewidth
                    \epsffile{\directory/data.eps}
                    \caption{{\market}, time series data.}
                    \label{\SETLABEL:TS}
                    \label{\SETLABELQ:TS}
                \end{minipage}
                \hfill
                \begin{minipage}[t]{0.45\textwidth}
                    \epsfxsize=1.0\linewidth
                    \epsffile{\directory/data.tsfraction.eps}
                    \caption[{\market}, normalized
                        increments]{{\market}, normalized increments
                        of the time series data presented in
                        Figure~\ref{\SETLABEL:TS}. The mean is
                        {\datafractionmean} with a standard deviation
                        of {\datafractionstddev}. The formula for the
                        least squares approximation is
                        ${\datafractionconstant} +
                        {\datafractionslope}t$, and the root mean
                        squared value is {\datafractionrms}. The
                        graph, labeled ``data\-.tsfraction\-.tsrms,''
                        is the running root mean square, and
                        ``data\-.tsfraction\-.tsavg'' is the running
                        average of the normalized increments.  This
                        graph is the fraction of change in the time
                        series, as a function of time. Note that the
                        slope of the mean, {\datafractionslope}, is
                        the coefficient of the nonlinearity term in
                        the normalized increments. See
                        Chapter~\ref{general}, Section~\ref{nlextend}
                        for a possible application of the logistic
                        function to this data set.}
                    \label{\SETLABEL:TF}
                    \label{\SETLABELQ:TF}
                \end{minipage}
            \end{center}
        \end{figure}

        \subidx{absolute value}{increments}
        \subidx{increments}{absolute value}

        Figure~\ref{\SETLABEL:TFA} is a graph of the absolute value of
        the normalized increments of the time series data presented in
        Figure~\ref{\SETLABEL:TF}. The data presented was made by
        running the Unix utility sed(1) on the normalized increments
        time series data to remove the negative signs. This is an
        absolute value procedure.  The resulting time series contains
        the absolute value of the instantaneous change in the rate of
        revenue returns, divided by the magnitude of the instantaneous
        rate of revenue returns\footnote{The absolute value of the
        normalized increments, when averaged, is related to the root
        mean square of the increments by a constant. If the normalized
        increments are a fixed increment, the constant is unity. If
        the normalized increments have a Gaussian distribution, the
        constant is $\approx 0.8$ depending on the accuracy of of
        ``fit'' to a Gaussian distribution.}.

        \subidx{histogram}{normalized}
        \subidx{normalized}{histogram}
        \subidx{programs}{tsnormal}
        \subidx{tsnormal}{program}
        \subidx{mean}{standard deviation}
        \subidx{standard deviation}{mean}
        \idx{root mean square}
        \idx{least squares approximation}
        \subidx{\market}{analysis of increments}
        Figure~\ref{\SETLABEL:NH} is the normalized histogram of the
        normalized increments of the time series data shown in
        Figure~\ref{\SETLABEL:TF}. The abscissa is 3 $\sigma$ limits,
        and the area under the two curves is identical. The data for
        this figure was produced by the program {\it tsnormal}\/,
        which is described briefly in Appendix~\ref{programs}.

        \begin{figure}[ht]
            \begin{center}
                \begin{minipage}[t]{0.45\textwidth}
                    \epsfxsize=1.0\linewidth
                    \epsffile{\directory/data.tsfraction.abs.eps}
                    \caption[{\market}, absolute value of the
                        normalized increments]{{\market}, absolute
                        value of the normalized increments of the time
                        series data presented in
                        Figure~\ref{\SETLABEL:TF}.  The mean is
                        {\datafractionabsmean} with a standard
                        deviation of {\datafractionabsstddev}. The
                        formula for the least squares approximation is
                        ${\datafractionabsconstant} +
                        {\datafractionabsslope}t$, and the root mean
                        square value, from Figure~\ref{\SETLABEL:TF},
                        is {\datafractionrms}.  The graph, labeled
                        ``data\-.tsfraction\-.tsrms,'' is the running
                        root mean square, and
                        ``data\-.tsfraction\-.tsavg'' is the running
                        average of the normalized increments presented
                        in Figure~\ref{\SETLABEL:TF}, superimposed
                        here for convenience. This graph is the
                        absolute value of the fraction of change in
                        the time series, as a function of time.}
                    \label{\SETLABEL:TFA}
                    \label{\SETLABELQ:TFA}
                \end{minipage}
                \hfill
                \begin{minipage}[t]{0.45\textwidth}
                    \epsfxsize=1.0\linewidth
                    \epsffile{\directory/data.tsfraction.tsnormal-s30.eps}
                    \caption[{\market}, normalized histogram of the
                        normalized increments]{{\market}, normalized
                        histogram of the normalized increments of the
                        time series data shown in
                        Figure~\ref{\SETLABEL:TF}.  The data has a
                        mean of {\datafractionmean}, with a standard
                        deviation of {\datafractionstddev}.  The area
                        under the two curves is identical. The
                        $\chi^2$ value of the observed and expected
                        values of the two curves is {\chisquared},
                        with a critical value of {\critical}.}
                    \label{\SETLABEL:NH}
                \end{minipage}
            \end{center}
        \end{figure}

        \subidx{programs}{tsXsquared}
        \subidx{tsXsquared}{program}
        \subidx{\market}{chi-squared values of increments}
        The program {\it tsXsquared}\/, which is briefly described in
        appendix~\ref{programs}, was used to derive the $\chi^2$
        statistics for the data presented in
        Figure~\ref{\SETLABEL:NH}.

        \subidx{programs}{tsstatest}
        \subidx{tsstatest}{program}
        \subidx{\market}{statistical estimates}

        Figure~\ref{\SETLABEL:SE} is the statistical estimate for the
        data presented in Figure~\ref{\SETLABEL:TF}, as derived by the
        program {\it tsstatest}\/, which is briefly described in
        appendix~\ref{programs}.

        \begin{figure}[ht]
            \begin{center}
                \begin{minipage}[t]{\textwidth}
                    \center{\fbox{\parbox{0.9\textwidth}{\XXX{\directory/data.tsstatest-f0.1-c0.9-i.tex}}}}
                    \caption[{\market}, statistical estimates of the
                        normalized increments]{{\market}, statistical
                        estimates of the normalized increments of the
                        time series shown in Figure~\ref{\SETLABEL:TF}.
                        The table was produced with the {\it
                        tsstatest}\/ program, and illustrates the
                        size of the data set required for a confidence
                        level of 90\%, with an error estimate of $\pm$
                        10\%, or alternately, the error estimate on
                        the time series shown in Figure~\ref{\SETLABEL:TF}.}
                    \label{\SETLABEL:SE}
                \end{minipage}
            \end{center}
        \end{figure}

        Note that the data set size estimations, as produced by the
        {\it tsstatest}\/ program, are probably very conservative,
        depending on the magnitude of the Shannon probability, $P =
        \shannonlogreturns$, as derived in
        Section~\ref{\SETLABEL:SP}. See Chapter~\ref{general},
        Section~\ref{serdss} for possible alternative methodologies
        for addressing the analysis of fractal time series with
        limited data set sizes. Depending on the magnitude of the
        Shannon probability, $P$, these estimates can be several
        orders of magnitude too high.

        \subidx{derivative of increments}{normalized}
        \subidx{normalized}{derivative of increments}
        \subidx{programs}{tsderivative}
        \subidx{tsderivative}{program}
        Figure~\ref{\SETLABEL:TF1} is the normalized histogram of the
        first derivative of the normalized increments of the time
        series data shown in Figure~\ref{\SETLABEL:TF}. In principle,
        if the distribution of the normalized increments presented in
        Figure~\ref{\SETLABEL:NH} is Gaussian in nature, this
        distribution would be similar to ``white noise,'' as presented
        in appendix~\ref{programs}, Figure~\ref{whiteexample}. The
        data was generated by the {\it tsderivative}\/ program, which
        is briefly described in
        appendix~\ref{programs}. Figure~\ref{\SETLABEL:TF2} is the
        normalized histogram of the second derivative of the
        normalized increments of the time series data shown in
        Figure~\ref{\SETLABEL:TF}. In principle, if the distribution
        of the normalized increments presented in
        Figure~\ref{\SETLABEL:NH} is an integrated Gaussian
        distribution in nature, this distribution would be similar to
        ``white noise,'' as presented in appendix~\ref{programs},
        Figure~\ref{whiteexample}.

        \begin{figure}[ht]
            \begin{center}
                \begin{minipage}[t]{0.45\textwidth}
                    \epsfxsize=1.0\linewidth
                    \epsffile{\directory/data.tsfraction.tsderivative.tsnormal-s30.eps}
                    \caption[{\market}, histogram of the first
                        derivative of the increments]{{\market},
                        normalized histogram of the first derivative
                        of the normalized increments of the time
                        series data shown in
                        Figure~\ref{\SETLABEL:TF}.}
                    \label{\SETLABEL:TF1}
                \end{minipage}
                \hfill
                \begin{minipage}[t]{0.45\textwidth}
                    \epsfxsize=1.0\linewidth
                    \epsffile{\directory/data.tsfraction.2tsderivative.tsnormal-s30.eps}
                    \caption[{\market}, histogram of the second
                        derivative of the increments]{{\market},
                        normalized histogram of second derivative of
                        the the normalized increments of the time
                        series data shown in
                        Figure~\ref{\SETLABEL:TF}.}
                    \label{\SETLABEL:TF2}
                \end{minipage}
            \end{center}
        \end{figure}

        \subidx{fractal}{range}
        \subidx{fractal}{R/S analysis}
        \subidx{\market}{rate of revenue returns, range}
        \subidx{\market}{deterministic mechanism}
        \subidx{deterministic}{mechanism}
        \subidx{mechanism}{deterministic}
        Figure~\ref{\SETLABEL:TR} is the range of values of the time
        series shown in Figure~\ref{\SETLABEL:TS}. The horizontal axis
        is time into the future. In principle, if the time series was
        characterized as fractional Brownian motion the graph in
        Figure~\ref{\SETLABEL:TR} would be a square root
        function\footnote{Note that the ``roughness,'' or ``sawtooth''
        characteristics of the graph in Figure~\ref{\SETLABEL:TR} are
        a computational artifact---caused by not using the -m option
        to the program {\it tshurst}\/, which is computationally
        inefficient.}. Figure~\ref{\SETLABEL:TD} is the deterministic
        map of the normalized increments of the time series data shown
        in Figure~\ref{\SETLABEL:TF}. The deterministic map is useful
        for determining if a time series was created by a
        deterministic mechanism. This, essentially, maps each element
        in the time series with the previous element in the time
        series.  See,~\cite[pp. 745]{Peitgen}.

        \begin{figure}[ht]
            \begin{center}
                \begin{minipage}[t]{0.45\textwidth}
                    \epsfxsize=1.0\linewidth
                    \epsffile{\directory/data.tshurst-f.eps}
                    \caption[{\market}, range]{{\market}, range of the
                        time series data shown in
                        Figure~\ref{\SETLABEL:TS}.}
                    \label{\SETLABEL:TR}
                \end{minipage}
                \hfill
                \begin{minipage}[t]{0.45\textwidth}
                    \epsfxsize=1.0\linewidth
                    \epsffile{\directory/data.tsfraction.tsdeterministic.eps}
                    \caption[{\market}, deterministic map]{{\market},
                        deterministic map of the normalized increments
                        of the time series data shown in
                        Figure~\ref{\SETLABEL:TF}.}
                    \label{\SETLABEL:TD}
                \end{minipage}
            \end{center}
        \end{figure}

% Local Variables:
% TeX-parse-self: t
% TeX-auto-save: t
% TeX-master: "fractal.tex"
% End:


        \subsubsection{Observations on the Time Series Increments Analysis}

            Figure~\ref{\SETLABEL:NH} would seem to indicate that the
            time series data for the {\market} represents a cumulative
            sum/integration of a random process that has a Gaussian
            distribution, (ie., satisfies the Gaussian increments
            property of fractional Brownian
            motion~\cite[pp. 250]{Crownover},) tending to justify the
            assumption that the time series data represents fractional
            Brownian motion.

        %
% -----------------------------------------------------------------------------
%
% A license is hereby granted to reproduce this software source code and
% to create executable versions from this source code for personal,
% non-commercial use.  The copyright notice included with the software
% must be maintained in all copies produced.
%
% THIS PROGRAM IS PROVIDED "AS IS". THE AUTHOR PROVIDES NO WARRANTIES
% WHATSOEVER, EXPRESSED OR IMPLIED, INCLUDING WARRANTIES OF
% MERCHANTABILITY, TITLE, OR FITNESS FOR ANY PARTICULAR PURPOSE.  THE
% AUTHOR DOES NOT WARRANT THAT USE OF THIS PROGRAM DOES NOT INFRINGE THE
% INTELLECTUAL PROPERTY RIGHTS OF ANY THIRD PARTY IN ANY COUNTRY.
%
% Copyright (c) 1994-2006, John Conover, All Rights Reserved.
%
% Comments and/or bug reports should be addressed to:
%
%     john@email.johncon.com (John Conover)
%
% -----------------------------------------------------------------------------
%
% Revision: \RCSRevision \\
% Revision Time: \RCSTime UMT \\
% Revision Date: \RCSDate \\
% Revision Id: \RCSId \\
% Revision File: \RCSLog \\
\RCS $Revision: 0.0 $
\RCS $Date: 2006/01/20 04:38:13 $
\RCS $Id: instant.tex,v 0.0 2006/01/20 04:38:13 john Exp $
% $Log: instant.tex,v $
% Revision 0.0  2006/01/20 04:38:13  john
% Initial version
%
%
    \subsection{Instantaneous Analysis of Normalized Increments}
        \label{\SETLABEL:IA}

        \subidx{\market}{instantaneous analysis of normalized increments}
        \idx{average of normalized increments}
        \idx{root mean square of normalized increments}
        \subidx{Shannon probability}{instantaneous computation of}
        \subidx{average of normalized increments}{instantaneous computation of}
        \subidx{root mean square of normalized increments}{instantaneous computation of}
        \subidx{instantaneous computation}{Shannon probability}
        \subidx{instantaneous computation}{average of normalized increments}
        \subidx{instantaneous computation}{root mean square of normalized increments}
        \idx{time series}
        \subidx{time series}{instantaneous analysis}
        \subidx{instantaneous analysis}{time series}
        \subidx{time series}{increments}
        \subidx{time series}{analysis}
        \subidx{Shannon}{probability}
        \subidx{probability}{Shannon}
        \subidx{normalized}{increments}
        \subidx{increments}{normalized}

        The program {\it tsinstant}\/, which is briefly described in
        Appendix~\ref{programs}, is for finding the instantaneous
        fraction of change in a time series. The value of a sample in
        the time series is subtracted from the previous sample in the
        time series, and divided by the value of the previous sample.
        As explained in Chapter~\ref{general},
        Sections~\ref{derivation},~\ref{GA},~\ref{abmfi},~\ref{aftsma}
        and,~\ref{ompl} for Brownian motion, random walk fractals, the
        absolute value of the instantaneous fraction of change is also
        the root mean square of the instantaneous fraction of
        change\footnote{The absolute value of the normalized
        increments, when averaged, is related to the root mean square
        of the increments by a constant. If the normalized increments
        are a fixed increment, the constant is unity. If the
        normalized increments have a Gaussian distribution, the
        constant is $\approx 0.8$ depending on the accuracy of of
        ``fit'' to a Gaussian distribution.}. Squaring this value is
        the average of the instantaneous fraction of change, and
        adding unity to the absolute value of the instantaneous
        fraction of change, and dividing by two, is the Shannon
        probability of the instantaneous fraction of change.

        Figure~\ref{\SETLABEL:IA1} is the instantaneous value of the
        root mean square of the normalized increments for the
        {\market}, and Figure~\ref{\SETLABEL:IA2} is the instantaneous
        Shannon probability for the normalized increments.

        \begin{figure}[ht]
            \begin{center}
                \begin{minipage}[t]{0.45\textwidth}
                    \epsfxsize=1.0\linewidth
                    \epsffile{\directory/data.tsinstant-r.eps}
                    \caption[{\market}, instantaneous value of
                        rms.]{{\market}, instantaneous value of the
                        root mean square of the normalized increments,
                        provided by running the program {\it
                        tsinstant}\/ with the -r option on the data
                        presented in Figure~\ref{\SETLABEL:TS}.}
                    \label{\SETLABEL:IA1}
                    \label{\SETLABELQ:IA1}
                \end{minipage}
                \hfill
                \begin{minipage}[t]{0.45\textwidth}
                    \epsfxsize=1.0\linewidth
                    \epsffile{\directory/data.tsinstant-s.eps}
                    \caption[{\market}, instantaneous value of
                        Shannon probability.]{{\market}, instantaneous
                        value of the Shannon probability of the
                        normalized increments, provided by running the
                        program {\it tsinstant}\/ with the -s option
                        on the data presented in
                        Figure~\ref{\SETLABEL:TS}.}
                    \label{\SETLABEL:IA2}
                    \label{\SETLABELQ:IA2}
                \end{minipage}
            \end{center}
        \end{figure}

% Local Variables:
% TeX-parse-self: t
% TeX-auto-save: t
% TeX-master: "fractal.tex"
% End:


        %
% -----------------------------------------------------------------------------
%
% A license is hereby granted to reproduce this software source code and
% to create executable versions from this source code for personal,
% non-commercial use.  The copyright notice included with the software
% must be maintained in all copies produced.
%
% THIS PROGRAM IS PROVIDED "AS IS". THE AUTHOR PROVIDES NO WARRANTIES
% WHATSOEVER, EXPRESSED OR IMPLIED, INCLUDING WARRANTIES OF
% MERCHANTABILITY, TITLE, OR FITNESS FOR ANY PARTICULAR PURPOSE.  THE
% AUTHOR DOES NOT WARRANT THAT USE OF THIS PROGRAM DOES NOT INFRINGE THE
% INTELLECTUAL PROPERTY RIGHTS OF ANY THIRD PARTY IN ANY COUNTRY.
%
% Copyright (c) 1994-2006, John Conover, All Rights Reserved.
%
% Comments and/or bug reports should be addressed to:
%
%     john@email.johncon.com (John Conover)
%
% -----------------------------------------------------------------------------
%
% Revision: \RCSRevision \\
% Revision Time: \RCSTime UMT \\
% Revision Date: \RCSDate \\
% Revision Id: \RCSId \\
% Revision File: \RCSLog \\
\RCS $Revision: 0.0 $
\RCS $Date: 2006/01/20 04:38:13 $
\RCS $Id: logistic.tex,v 0.0 2006/01/20 04:38:13 john Exp $
% $Log: logistic.tex,v $
% Revision 0.0  2006/01/20 04:38:13  john
% Initial version
%
%
    \subsection{Logistic Analysis}
        \label{\SETLABEL:LA}

        \subidx{\market}{Logistic function analysis}
        \subidx{time series}{logistic function}
        \subidx{logistic function}{time series}
        \subidx{time series}{increments}
        \subidx{time series}{analysis}
        \subidx{cumulative sum}{analysis}
        \subidx{analysis}{cumulative sum}
        \subidx{analysis}{random process}
        \subidx{random process}{analysis}
        The data in this section is presented in tabular form in
        Section~\ref{\SETLABELREF:LAA}.  Figure~\ref{\SETLABEL:LA1} is
        a graph of the logistic function estimates of the time series
        data for the {\market}. The reader is cautioned that these
        graphs are constructed using the method suggested in
        Chapter~\ref{general}, Section~\ref{nlextend} and enormous
        precision is required for adequate prediction of the logistic
        function,~\cite{Modis}. Particularly, the non-linear term will
        usually require intervention to produce a practical fit to the
        data. In addition, there are numerical stability issues with
        logistic function methodologies\footnote{For example, in
        Figures~\ref{\SETLABEL:LA1} and~\ref{\SETLABEL:LA2}, if the
        non-linear term, $b$, was greater than zero, it was set to
        zero to produce the graphs. See Section~\ref{\SETLABELREF:LAA}
        for the actual derived values. In other cases, the magnitude
        of $b$ was too large, resulting in a graph that was decreasing
        as a function of time}.  The methodology should be regarded as
        ``fragile.'' It is included for completeness.

        \idx{least squares approximation}
        Figure~\ref{\SETLABEL:LA1} is a graph of the logistic function
        for the time series data presented in
        Figure~\ref{\SETLABEL:TS}. The data presented was made by
        running the program {\it tsdlogistic}\/, which is described
        briefly in Appendix~\ref{programs}, on the parameters
        extracted from the time series data as suggested in
        Figure~\ref{\SETLABEL:TF}. The program {\it tslsq}\/ was used
        to derive the constant and the slope of the normalized
        increments of the data presented in Figure~\ref{\SETLABEL:TF}.
        Figure~\ref{\SETLABEL:LA2} is the same graph, but with the
        time scale expanded by a factor of two.

        \begin{figure}[ht]
            \begin{center}
                \begin{minipage}[t]{0.45\textwidth}
                    \epsfxsize=1.0\linewidth
                    \epsffile{\directory/data.tsfraction.tslsq-p.tsdlogistic.eps}
                    \caption[{\market}, logistic function
                        estimates.]{{\market}, logistic function
                        estimates, provided by running the {\it
                        tslsq}\/ program on the normalized increments
                        presented in Figure~\ref{\SETLABEL:TF} with
                        the -p option. These parameters were used as
                        arguments to the {\it tsdlogistic}\/ program.}
                    \label{\SETLABEL:LA1}
                    \label{\SETLABELQ:LA1}
                \end{minipage}
                \hfill
                \begin{minipage}[t]{0.45\textwidth}
                    \epsfxsize=1.0\linewidth
                    \epsffile{\directory/data.tsfraction.tslsq-p.tsdlogistic2.eps}
                    \caption[{\market}, logistic function
                        estimates.]{{\market}, logistic function
                        estimates of Figure~\ref{\SETLABEL:LA1} with
                        the time scale expanded by a factor of two.}
                    \label{\SETLABEL:LA2}
                    \label{\SETLABELQ:LA2}
                \end{minipage}
            \end{center}
        \end{figure}

% Local Variables:
% TeX-parse-self: t
% TeX-auto-save: t
% TeX-master: "fractal.tex"
% End:


        %
% -----------------------------------------------------------------------------
%
% A license is hereby granted to reproduce this software source code and
% to create executable versions from this source code for personal,
% non-commercial use.  The copyright notice included with the software
% must be maintained in all copies produced.
%
% THIS PROGRAM IS PROVIDED "AS IS". THE AUTHOR PROVIDES NO WARRANTIES
% WHATSOEVER, EXPRESSED OR IMPLIED, INCLUDING WARRANTIES OF
% MERCHANTABILITY, TITLE, OR FITNESS FOR ANY PARTICULAR PURPOSE.  THE
% AUTHOR DOES NOT WARRANT THAT USE OF THIS PROGRAM DOES NOT INFRINGE THE
% INTELLECTUAL PROPERTY RIGHTS OF ANY THIRD PARTY IN ANY COUNTRY.
%
% Copyright (c) 1994-2006, John Conover, All Rights Reserved.
%
% Comments and/or bug reports should be addressed to:
%
%     john@email.johncon.com (John Conover)
%
% -----------------------------------------------------------------------------
%
% Revision: \RCSRevision \\
% Revision Time: \RCSTime UMT \\
% Revision Date: \RCSDate \\
% Revision Id: \RCSId \\
% Revision File: \RCSLog \\
\RCS $Revision: 0.0 $
\RCS $Date: 2006/01/20 04:38:13 $
\RCS $Id: hurst.tex,v 0.0 2006/01/20 04:38:13 john Exp $
% $Log: hurst.tex,v $
% Revision 0.0  2006/01/20 04:38:13  john
% Initial version
%
%
    \subsection{Hurst Coefficient Analysis}
        \label{\SETLABEL:H}

        \subidx{\market}{Hurst coefficient analysis}
        \subidx{Hurst coefficient}{analysis}
        \subidx{increments}{normalized}
        \subidx{normalized}{increments}
        \subidx{programs}{tshurst}
        \subidx{tshurst}{program}
        The data in this section is presented in tabular form in
        Section~\ref{\SETLABELREF:HCHP}. Figure~\ref{\SETLABEL:HC} is
        a graph of the Hurst coefficient data time series data shown
        in Figure~\ref{\SETLABEL:TS}. The slope of the graph is the
        Hurst coefficient.  The data for this figure was produced by
        the program {\it tshurst}\/, which is described briefly in
        Appendix~\ref{programs}.

        \subidx{\market}{H parameter analysis}
        \subidx{H parameter}{analysis}
        \subidx{programs}{tshcalc}
        \subidx{tshcalc}{program}
        Figure~\ref{\SETLABEL:HP} is a graph of the H parameter data
        for the normalized increments of the time series data shown in
        Figure~\ref{\SETLABEL:TF}. The data for this figure was
        produced by the program {\it tshcalc}\/, which is described
        briefly in Appendix~\ref{programs}.

        \begin{figure}[ht]
            \begin{center}
                \begin{minipage}[t]{0.45\textwidth}
                    \epsfxsize=1.0\linewidth
                    \epsffile{\directory/data.tshurst.eps}
                    \caption[{\market}, Hurst coefficient data]{{\market},
                        Hurst coefficient data for the normalized
                        increments of the time series data shown in
                        Figure~\ref{\SETLABEL:TF}.  The slope of the graph
                        is the Hurst coefficient.}
                    \label{\SETLABEL:HC}
                \end{minipage}
                \hfill
                \begin{minipage}[t]{0.45\textwidth}
                    \epsfxsize=1.0\linewidth
                    \epsffile{\directory/data.tshcalc.eps}
                    \caption[{\market}, H parameter data]{{\market}, H
                        parameter data for the normalized increments of
                        the time series data shown in
                        Figure~\ref{\SETLABEL:TF} The slope of the graph
                        is the H parameter.}
                    \label{\SETLABEL:HP}
                \end{minipage}
            \end{center}
        \end{figure}

        \subidx{revenue}{See, rate of revenue returns}
        \subidx{returns}{See, rate of revenue returns}
        \subidx{\market}{revenues}
        \subidx{Hurst coefficient}{analysis}
        \subidx{\market}{Hurst coefficient analysis}
        \subidx{\market}{rate of change}
        \subidx{\market}{windows of opportunity}
        \subidx{rate of revenue returns}{forecast}
        \subidx{forecast}{rate of revenue returns}
        \idx{windows of opportunity}
        \subidx{programs}{tslsq}
        \subidx{tslsq}{program}

        The approximately linear slope of the graph in
        Figure~\ref{\SETLABEL:HC} implies that the variance of the
        rate of revenue returns, (per {\timescale},) in the {\market},
        $V(t_2 - t_1)$, over a period of time is proportional to the
        period of time raised to twice the Hurst
        coefficient~\cite[pp. 180]{Feder},~\cite[pp. 246]{Crownover}.
        This seems to be a quantitative statement concerning how fast,
        and to what degree, the rate of revenue returns' state of
        affairs can change over a period of time.  An additional
        implication, for Hurst coefficients sufficiently close to 0.5,
        is that the probability of the state of affairs repeating
        sometime in the future goes down with increasing
        time\footnote{It can be shown that the number of expected
        market ``high'' and ``low'' transitions, $N$, scales with the
        square root of time, or $N \propto \sqrt {t}$, meaning that
        the cumulative distribution of the probability, $P$, of the
        duration of a market's ``high'' or ``low'' exceeding a given
        time interval, $t$, is proportional to the reciprocal of the
        square root of the time interval, $P \propto 1 / \sqrt {t}$,
        (or, conversely, that the probability of the duration of a
        market's ``high'' or ``low'' exceeding a given time interval
        is proportional to the reciprocal of the time interval raised
        to the power $3 / 2$, ie., $P \propto 1 / t^{3 /
        2}$,~\cite[pp. 153]{Schroeder}. What this means is that a
        histogram of the ``zero free'' run-lengths of a market being
        ``high'' or ``low,'' over a long time, would have a $1 / t^{3
        / 2}$ characteristic.)}, $t$, $p(t) = erf (1/\sqrt{2t})$ which
        is approximately $1/\sqrt{t}$ for $t \gg
        1$~\cite[pp. 160]{Schroeder}. Figures~\ref{\SETLABEL:FN},
        and,~\ref{\SETLABEL:FF} compare methods of approximation of
        the ``forecastability'' of the rate of revenue returns in the
        {\market} for the near term and far term,
        respectively~\cite[pp. 83-84]{Peters:CAOITCM}\footnote{The
        author is not comfortable with Peters' interpretation. For
        example, if the algorithm explained
        in~\cite[pp. 82]{Peters:CAOITCM} is used on ``white noise''
        which, by definition, never has any correlations, the short
        term Hurst coefficient, and thus the ``forecastability,'' is
        still near unity---a bit of an enigma. This can be verified
        with the {\it tswhite}\/ and {\it tshurst}\/ programs, which
        are briefly described in Appendix~\ref{programs}.}.  This
        seems to be a quantitative statement concerning ``windows of
        opportunity'' in the rate of revenue returns, (per
        {\timescale}.)  The program {\it tslsq}\/ was used on the
        Hurst coefficient data, presented in
        Figure~\ref{\SETLABEL:HC}, to provide a least squares
        approximation to the Hurst coefficient. The superimposed least
        squares approximation with on original Hurst coefficient data
        is presented.  The time series data has a Hurst coefficient of
        {\thurstlow}, so that:

        \subidx{\market}{Hurst coefficient analysis}
        \begin{eqnarray}
            V\left(t_2 - t_1\right) & \propto & \left(t_2 - t_1\right)^{2 \cdot H}\\
            V\left(t_2 - t_1\right) & \propto & \left(t_2 - t_1\right)^{2 \cdot {\thurstlow}}\\
                                    & \propto & \left(t_2 - t_1\right)^{\thurstlowtwo}
            \label{\SETLABEL:V}
        \end{eqnarray}

        \subidx{fractional}{Brownian motion}
        \subidx{Brownian motion}{fractional}
        \idx{fractal}
        \noindent where $V(t_2 - t_1)$ is the variance of the
        increments of the rate of revenue returns, (per {\timescale},)
        over the time interval $t_2 -
        t_1$,~\cite[pp. 177]{Feder},~\cite[pp. 494]{Peitgen}. If $H >
        \frac{1}{2}$, then the time series is termed as being
        characterized by ``fractional Brownian
        motion~\cite[pp. 170]{Feder}.''

        \subidx{rate of revenue returns}{predictability}
        \subidx{rate of revenue returns}{forecastability}
        \subidx{rate of revenue returns}{consistency}
        \subidx{predictability}{rate of revenue returns}
        \subidx{forecastability}{rate of revenue returns}
        \subidx{consistency}{rate of revenue returns}
        \subidx{\market}{rate of revenue returns, predictability}
        \subidx{\market}{rate of revenue returns, forecastability}
        \subidx{\market}{rate of revenue returns, consistency}
        \subidx{Hurst coefficient}{analysis}
        \subidx{\market}{Hurst coefficient analysis}
        \subidx{\market}{rate of change}

        In some sense, the Hurst coefficient is a quantitative
        expression of the ``forecastability'' of the future based on
        the past\footnote{Actually, in general, when summing fractal
        entities, the method used should be a root mean square
        process, dependent on the Hurst Coefficient, $H$, where
        $P_{total}^H = P_1^H + P_2^H + \cdots$, where $P_n$ is the
        fractal entities. For a Brownian motion, or random walk type
        of fractal the Hurst Coefficient is a function of time into
        the future. For the ``near term,'' the Hurst coefficient is
        very near unity, meaning the summation process is linear. For
        the ``long term,'' $H \approx 0.5$, or a standard root mean
        square summation process should be used. If $H$ is $0.5$ then
        the market is termed a Brownian motion, or random walk
        process. If it is larger than 0.5, it is termed fractional
        Brownian motion process. For a random walk process, ``near
        term'' and ``far term'' are quantitatively differentiated on
        the Hurst Coefficient graph where $1 - \ln (t) = 0.5 \cdot \ln
        (t)$, or when $\ln (t) = 2$, or $t = 7.389\ldots$ See
        Section~\ref{\SETLABEL:FS} for the particulars on using Hurst
        Coefficient to sum fractal process' for the {\market}. See
        also~\cite[pp. 67, 83-84]{Peters:CAOITCM} and~\cite[pp. 129,
        159]{Schroeder} for particulars on the implications of the
        Hurst Coefficient and root mean square summation issues.}.  A
        Hurst coefficient of {\thurstlow}, (for the near future, and
        {\thurstall} for the distant future.) implies that the
        likelihood of the rate of revenue returns, (per {\timescale},)
        for any two consecutive {\timescale}s being the same is
        {\thurstlowhundred}\%~\cite[pp. 66]{Peters:CAOITCM} for the
        near future, and {\thurstall} for the distant
        future. Likewise, there is a {\thurstlowhundred}\% chance of
        the rate of revenue returns, (per {\timescale},) movements
        being the same in consecutive time periods---ie., if, in a
        given {\timescale}, the rate of revenue returns, (per
        {\timescale},) is increasing, there is a {\thurstlowhundred}\%
        that the rate of revenue returns, (per {\timescale},) will
        increase in the following period, also. In some sense, this is
        a quantitative statement on how ``predictable,'' or
        ``forecastable'' the rate of revenue returns, (per
        {\timescale},) for the {\market} are over time, since the
        probability of having $n$ many consecutive {\timescale}s of
        the same agenda is $H^n$ where $H$ is the Hurst coefficient,
        or, letting the short term probability of having $n$ many
        {\timescale}s of the same market agenda, $p_a$, is:

        \begin{eqnarray}
            p_a\left(n\right) & = & H^{n}\\
                              & = & {\thurstlow}^{n}
            \label{\SETLABEL:MA}
        \end{eqnarray}

        \subidx{rate of revenue returns}{predictability}
        \subidx{rate of revenue returns}{forecastability}
        \subidx{rate of revenue returns}{consistency}
        \subidx{predictability}{rate of revenue returns}
        \subidx{forecastability}{rate of revenue returns}
        \subidx{consistency}{rate of revenue returns}
        As an interesting interpretation of the normalized increments
        of the time series data presented in
        Figure~\ref{\SETLABEL:TF}, if the vertical axis is multiplied
        by 100, to convert to percent, then the graph represents the
        error, in percent, that would be made by forecasting, month by
        month, that the next {\timescale}'s rate of revenue returns
        would be the same as the current {\timescale}'s revenue
        rate. Interestingly, it is $\datafractionmean \cdot 100$
        percent, on the average, with a standard deviation of
        $\datafractionstddev \cdot 100$ percent, and a root mean
        square error value of $\datafractionrms \cdot 100$
        percent---small values for such a simple forecasting
        mechanism.

        \subidx{\market}{rate of revenue returns, range}
        \subidx{Hurst coefficient}{analysis}
        \subidx{\market}{Hurst coefficient analysis}
        \subidx{\market}{rate of change}

        This is, essentially, a statement of the range of values, in
        the increments of the rate of revenue returns, (per
        {\timescale},) that is to be expected over the time interval,
        $t_2 - t_1$,
        $R_v$,~\cite[pp. 178]{Feder},~\cite[pp. 172]{Cambel}:

        \begin{eqnarray}
            R_v\left(t_2 - t_1\right) & \propto & \left(t_2 - t_1\right)^{H}\\
                                      & \propto & \left(t_2 - t_1\right)^{\thurstlow}
            \label{\SETLABEL:R}
        \end{eqnarray}

        \subidx{\market}{rate of revenue returns, range}
        \subidx{Hurst coefficient}{analysis}
        \subidx{\market}{Hurst coefficient analysis}
        \subidx{\market}{rate of change}
        \subidx{Markov}{statistics}
        \subidx{statistics}{Markov}
        \noindent where $R$ is the range of values in the increments
        of the rate of revenue returns, (per {\timescale}.) A Hurst
        coefficient, $H$, that is much larger than $\frac{1}{2}$, (but
        less than 1,) implies a strongly non-Gaussian distribution in
        the increments of the rate of revenue returns, (per
        {\timescale},)~\cite[pp. 152, 194]{Feder}, and a Hurst
        coefficient near $\frac{1}{2}$ implies that the increments of
        the rate of revenue returns, (per {\timescale}) is
        characteristic of an independent
        process~\cite[pp. 195]{Feder}. Extreme caution should be
        exercised in using Markov statistics in any analysis where the
        Hurst coefficient is not
        $\frac{1}{2}$,~\cite[pp. 124]{Crownover},~\cite[pp. 106]{Peters:CAOITCM}.


        As a useful approximation, if $H$, is approximately
        $\frac{1}{2}$, Equation~\ref{\SETLABEL:R} reduces
        to,~\cite[pp. 129]{Schroeder}:

        \begin{eqnarray}
            R\left(t_2 - t_1\right) & \propto & (t_2 - t_1)^{\frac{1}{2}}\\
                                    & \propto & \sqrt{\left(t_2 - t_1\right)}
        \end{eqnarray}

        \subidx{\market}{rate of revenue returns, range}
        \subidx{\market}{rate of revenue returns, increase and decrease}
        \subidx{Hurst coefficient}{analysis}
        \subidx{\market}{Hurst coefficient analysis}
        \subidx{\market}{rate of change}
        \subidx{Markov}{statistics}
        \subidx{statistics}{Markov}

        In the case where the Hurst coefficient, $H$, is
        $\frac{1}{2}$, the range of values in the increments of the
        rate of revenue returns, (per {\timescale},) divided by the
        standard deviation of these values, $S$, can be anticipated to
        increase over time according to the following
        relation,~\cite[pp. 154]{Feder},~\cite[pp. 129]{Schroeder}:

        \begin{equation}
            \frac{R\left(t_2 - t_1\right)}{S} \propto \left(t_2 - t_1\right)^{\frac{1}{2}}
        \end{equation}

        \subidx{\market}{rate of revenue returns, range}
        \subidx{\market}{rate of revenue returns, increase and decrease}
        \subidx{Hurst coefficient}{analysis}
        \subidx{\market}{Hurst coefficient analysis}
        \subidx{\market}{rate of change}
        \noindent which is a useful conceptual approximation, since it
        involves only the square root function---if the range and the
        standard deviation of the increments of the rate of revenue
        returns, (per {\timescale},) are known, (and $H \approx
        \frac{1}{2}$,) then the expected change in $\frac{R}{S}$, will
        increase with the square root of time\footnote{To be precise,
        it is actually asymptotically proportional to
        $\tau^{\frac{1}{2}}$}.

        Another useful approximation when rescaling processes that are
        characterize by Brownian motion, (ie., when $H \approx
        \frac{1}{2}$,) is that:

        \begin{eqnarray}
            X\left(t\right) & \propto & \frac{X\left(rt\right)}{r^{H}}\\
                            & \propto & \frac{X\left(rt\right)}{r^{\thurstlow}}
        \end{eqnarray}

        \idx{Brownian motion}
        \idx{fractal}
        Where $X(t)$ is the process characterized by Brownian motion,
        and $r$ is a scaling factor,~\cite[pp. 494]{Peitgen}.

        \subidx{programs}{tslsq}
        \subidx{tslsq}{program}
        The program {\it tslsq}\/ was used on the H parameter data,
        presented in Figure~\ref{\SETLABEL:HP}, to provide a least
        squares approximation to the H parameter for the
        {\market}. The superimposed least squares approximation on the
        original H parameter data is presented.  By contrast, the H
        parameter, as derived by the methodology outlined
        in~\cite[pp. 249]{Crownover}, is {\thcalclow} for the near
        future, and {\thcalcall} for the distant future.

        \subidx{\market}{Hurst coefficient analysis}
        \subidx{Hurst coefficient}{analysis}
        \subidx{increments}{normalized}
        \subidx{normalized}{increments}
        \subidx{programs}{tshurst}
        \subidx{tshurst}{program}
        \subidx{\market}{H parameter analysis}
        \subidx{H parameter}{analysis}
        \subidx{programs}{tshcalc}
        \subidx{tshcalc}{program}
        Figures~\ref{\SETLABEL:HC} and~\ref{\SETLABEL:HP} represent
        Hurst coefficient and H parameter data that are derived from
        the normalized increments, shown in
        Figure~\ref{\SETLABEL:TF}. In this case, the data is
        considered a normalized derivative of the time series data
        presented in Figure~\ref{\SETLABEL:TF}, instead of a
        cumulative sum.  The program, {\it tshurst}\/, is described
        briefly in appendix~\ref{programs}, and the data for
        figures~\ref{\SETLABEL:THC} and~\ref{\SETLABEL:THP} was made
        using the -d option.

        \begin{figure}[ht]
            \begin{center}
                \begin{minipage}[t]{0.45\textwidth}
                    \epsfxsize=1.0\linewidth
                    \epsffile{\directory/data.tsfraction.tshurst-d.eps}
                    \caption[{\market}, traditional Hurst coefficient
                        data]{{\market}, traditional Hurst coefficient
                        data for the time series data shown in
                        Figure~\ref{\SETLABEL:TS}.  The slope of the
                        graph is the Hurst coefficient, and is
                        {\hurstlow} for the near term, and
                        {\hurstall} for the far term.}
                    \label{\SETLABEL:THC}
                \end{minipage}
                \hfill
                \begin{minipage}[t]{0.45\textwidth}
                    \epsfxsize=1.0\linewidth
                    \epsffile{\directory/data.tsfraction.tshcalc-d.eps}
                    \caption[{\market}, traditional H parameter
                        data]{{\market}, traditional H parameter data
                        for the time series data shown in
                        Figure~\ref{\SETLABEL:TS} The slope of the
                        graph is the H parameter, and is {\hcalclow}
                        for the near term, and {\hcalcall} for the
                        far term.}
                    \label{\SETLABEL:THP}
                \end{minipage}
            \end{center}
        \end{figure}

% Local Variables:
% TeX-parse-self: t
% TeX-auto-save: t
% TeX-master: "fractal.tex"
% End:


        \subsubsection{Observations on the Hurst Coefficient Analysis}

            Many {\market} industry analyst speculate that there is
            ``periodic'' behavior in the market place, at
            approximately 5 year intervals. Both the Hurst coefficient
            and H parameter graphs would tend to support the
            intuition. Notice that the slope of the graphs, in
            figures~\ref{\SETLABEL:HC} and~\ref{\SETLABEL:HP}, tend to
            decrease abruptly at $t \approx \ln(3) \approx 20$
            {\timescale}s, which is approximately 60 months, or 5
            years~\cite[pp. 96]{Peters:CAOITCM}. Whether this is
            ``periodic'' behavior, or an indication of more complex
            system dynamics, perhaps ``chaotic,'' remains to be
            seen. If that is the case, it could provide an exploitive
            venue.

        %
% -----------------------------------------------------------------------------
%
% A license is hereby granted to reproduce this software source code and
% to create executable versions from this source code for personal,
% non-commercial use.  The copyright notice included with the software
% must be maintained in all copies produced.
%
% THIS PROGRAM IS PROVIDED "AS IS". THE AUTHOR PROVIDES NO WARRANTIES
% WHATSOEVER, EXPRESSED OR IMPLIED, INCLUDING WARRANTIES OF
% MERCHANTABILITY, TITLE, OR FITNESS FOR ANY PARTICULAR PURPOSE.  THE
% AUTHOR DOES NOT WARRANT THAT USE OF THIS PROGRAM DOES NOT INFRINGE THE
% INTELLECTUAL PROPERTY RIGHTS OF ANY THIRD PARTY IN ANY COUNTRY.
%
% Copyright (c) 1994-2006, John Conover, All Rights Reserved.
%
% Comments and/or bug reports should be addressed to:
%
%     john@email.johncon.com (John Conover)
%
% -----------------------------------------------------------------------------
%
% Revision: \RCSRevision \\
% Revision Time: \RCSTime UMT \\
% Revision Date: \RCSDate \\
% Revision Id: \RCSId \\
% Revision File: \RCSLog \\
\RCS $Revision: 0.0 $
\RCS $Date: 2006/01/20 04:38:13 $
\RCS $Id: fiscal.tex,v 0.0 2006/01/20 04:38:13 john Exp $
% $Log: fiscal.tex,v $
% Revision 0.0  2006/01/20 04:38:13  john
% Initial version
%
%
    \subsection{Fixed Increment Approximation for Fiscal Strategy}
        \label{\SETLABEL:FS}

        \subidx{\market}{fiscal strategy}
        \subidx{markets}{analysis}
        \subidx{analysis}{markets}
        \subidx{strategy}{fiscal}
        \subidx{fiscal}{strategy}
        The data in this section is presented in tabular form in
        Section~\ref{\SETLABELREF:LR}. This section derives various
        values based on the ``average'' of the normalized increments
        presented in Figure~\ref{\SETLABEL:TFA}. These values are an
        approximation to a, probably, complex process with a
        distribution shown in Figure~\ref{\SETLABEL:TF}. These values
        will be used in a fixed increment Brownian fractal analysis
        and simulation of the {\market}, and may, or may not, provide
        adequate accuracy for projections.

        For an organization operating in the {\market}, the fiscal
        strategy, commensurate with the aggregate environment, can be
        derived as follows~\cite[pp. 128, pp
        151]{Schroeder},~\cite[pp. 450]{Reza},~\cite[pp. 270]{Pierce}:
        \vspace{0.15in}

        \subsubsection{Logarithmic Returns}
            \label{\SETLABEL:LR}

            \subidx{logarithmic}{returns}
            \subidx{returns}{logarithmic}
            \subidx{\market}{logarithmic returns}
            The logarithmic returns can be calculated by various
            means. Four will be presented here, for comparison.

            \subidx{programs}{tsnormal}
            \subidx{tsnormal}{program}
            \subidx{logarithmic}{returns}
            \subidx{returns}{logarithmic}
            The logarithmic returns, in bits, $bits$, as computed from
            the mean, by the program {\it tsnormal}\/, which is
            described in Chapter~\ref{programs}, and is presented in
            Figure~\ref{\SETLABEL:TF}, and Equation~\ref{abits} from
            Section~\ref{ereturns} in Chapter~\ref{general}:

            \begin{equation}
                bits = \frac{\ln \left({\datafractionmean} + 1\right)}{\ln \left(2\right)} = \datafractionmeanbits
            \end{equation}

            \subidx{programs}{tslsq}
            \subidx{tslsq}{program}
            \subidx{logarithmic}{returns}
            \subidx{returns}{logarithmic}
            \noindent By comparison, the logarithmic returns, in bits,
            $bits$, as computed from the constant in the least squares
            approximation, using the program {\it tslsq}\/, which is briefly
            described in Chapter~\ref{programs}, as presented in
            Figure~\ref{\SETLABEL:TF}, and Equation~\ref{abits} from
            Section~\ref{ereturns} in Chapter~\ref{general}:

            \begin{equation}
                bits = \frac{\ln \left({\datafractionconstant} + 1\right)}{\ln \left(2\right)} = \datafractionconstantbits
            \end{equation}

            Note that if the mean is not constant in
            Figure~\ref{\SETLABEL:TF}, this method will not provide
            accurate results.

            \subidx{programs}{tslsq}
            \subidx{tslsq}{program}
            \subidx{logarithmic}{returns}
            \subidx{returns}{logarithmic}
            \noindent And by yet another comparison, using the program
            {\it tslsq}\/, which is briefly described in
            Chapter~\ref{programs}, with the -e -p options, to provide
            a formula for the least squares exponential fit to the
            time series data set presented in
            Figure~\ref{\SETLABEL:TS}:

            \begin{equation}
                bits = {\datatslsqepbits}
            \end{equation}

            \subidx{programs}{tslogreturns}
            \subidx{tslogreturns}{program}
            \subidx{logarithmic}{returns}
            \subidx{returns}{logarithmic}
            \noindent And finally, by comparison, from the
            {\it tslogreturns}\/ program, which is briefly described
            in Chapter~\ref{programs}, with the -p option, to provide
            a formula for the logarithmic returns of the time series
            data set presented in Figure~\ref{\SETLABEL:TS}:

            \begin{equation}
                bits = {\logreturns}
            \end{equation}

        \subsubsection{Calculation of Shannon Probability}
            \label{\SETLABEL:SP}

            \subidx{\market}{Shannon probability}
            Ideally, all of the values presented in
            Section~\ref{\SETLABEL:LR} would be equal. Using the
            logarithmic returns provided by the {\it tslogreturns}\/
            program, to be consistent
            with~\cite[pp. 81]{Peters:CAOITCM}

            \subidx{programs}{tslogreturns}
            \subidx{tslogreturns}{program}
            \begin{equation}
                2^{{\logreturns}t}
            \end{equation}

            \noindent therefore:
            \begin{equation}
                C\left(p\right) = {\logreturns}
            \end{equation}
            \subidx{programs}{tsshannon}
            \subidx{tsshannon}{program}
            \subidx{Shannon}{probability}
            \subidx{probability}{Shannon}
            \noindent and, {\it tsshannon}\/ {\logreturns} gives:
            \begin{equation}
                \label{\SETLABEL:F0}
                C\left({\shannonlogreturns}\right) = {\logreturns}
            \end{equation}
            \noindent therefore:
            \begin{eqnarray}
                2^{C\left({\shannonlogreturns}\right)} & = & 2^{\logreturns}\\
                                                       & = & {\twologreturns}\\
                                                       & = & {\twologreturnshundred}\%
            \end{eqnarray}
            \noindent and:
            \begin{eqnarray}
                2p - 1 & = & \left(2 \cdot {\shannonlogreturns}\right) - 1\\
                       & = & {\twopone}\\
                       \label{\SETLABEL:F1}
                       & = & {\twoponehundred}\%
            \end{eqnarray}

            \subidx{\market}{fiscal strategy}
            \subidx{markets}{analysis}
            \subidx{analysis}{markets}
            \subidx{strategy}{fiscal}
            \subidx{fiscal}{strategy}
            \subidx{\market}{fiscal strategy}
            \subidx{\market}{growth rate}
            Presuming the simplified assumptions outlined in
            Section~\ref{assumptions}, the ``typical'' organization
            operating in the {\market} executes a long term fiscal
            strategy, commensurate with the aggregate environment,
            that is to invest, every {\timescale}, in sufficient
            additional resources and infrastructure, to increase the
            manufacturing of goods and services by {\twoponehundred}\%
            of its rate of revenue returns, (per {\timescale}.) As a
            conceptual model, the remaining {\hundredtwoponehundred}\%
            will be held in ``reserve'' with a
            {\shannonlogreturnshundred}\% chance of making twice the
            {\twoponehundred}\% back, (and a
            {\hundredshannonlogreturnshundred}\% chance of making
            0.0,) in one {\timescale}, on the average, for an average
            growth in its rate of revenue returns, (per {\timescale},)
            of {\twologreturnshundred}\%, or a doubling of its rate of
            revenue returns, (per {\timescale},) in
            {\oneoverlogreturns} {\timescale}s.

        \subsubsection{Example Fixed Increment Approximation Fiscal Strategies}

            \subidx{\market}{fiscal strategy}
            \subidx{markets}{analysis}
            \subidx{analysis}{markets}
            \subidx{strategy}{fiscal}
            \subidx{fiscal}{strategy}
            \subidx{\market}{fiscal strategy}
            \subidx{\market}{growth rate}
            \subidx{\market}{management metric}
            \idx{management metric}
            A possible metric on the effectiveness of long term fiscal
            management could possibly be that if an investment of
            {\twoponehundred}\% per {\timescale} of the rate of
            revenue returns, (per {\timescale},) is made in resources
            and infrastructure, then the rate of revenue returns would
            be expected to increase by {\twologreturnshundred}\%, per
            {\timescale}, on average.

            Note that the metrics presented in this section are
            representative of the {\market} as an aggregate whole, and
            may or may not be accurate representations for any
            particular participant in the environment. Of interest to
            the participants in the environment would be a similar
            analysis of each product or service rendered in the
            marketplace.

            \subidx{\market}{fiscal strategy}
            \subidx{markets}{analysis}
            \subidx{analysis}{markets}
            \subidx{strategy}{fiscal}
            \subidx{fiscal}{strategy}
            \subidx{\market}{fiscal strategy}
            As a simple illustrative example, a company operating in
            this environment might obtain a credit line from a bank
            that is equal to {\twoponehundred}\% of its rate of
            revenue returns, (per {\timescale},) to finance additional
            operations. In this simple scenario, the company would use
            its revenue base as collateral for the loan. Some
            {\timescale}s, depending on the {\market}'s environment,
            the company's rate of revenue returns exceeds what was
            borrowed from the bank, and the loan is repaid in
            full. Other {\timescale}s, the company must default, and
            the bank seizes a portion of the company's revenue base to
            pay the delinquent loan. However, on the average, the
            company will expand its rate of revenue returns at
            {\twologreturnshundred}\% per {\timescale}.

            \subidx{\market}{fiscal strategy}
            \subidx{markets}{analysis}
            \subidx{analysis}{markets}
            \subidx{strategy}{fiscal}
            \subidx{fiscal}{strategy}
            \subidx{\market}{fiscal strategy}
            As another simple example, a company re-invests
            {\twoponehundred}\% of its rate of revenue returns, (per
            {\timescale},) in development, marketing, sales, and
            distribution of new products.  Although some products will
            be successful and the return on the investment will exceed
            the {\twoponehundred}\% per {\timescale} investment,
            others will not. However, on the average, the company will
            expand it gross rate of revenue returns at
            {\twologreturnshundred}\% per {\timescale}.

            \subidx{\market}{fiscal strategy}
            \subidx{markets}{analysis}
            \subidx{analysis}{markets}
            \subidx{strategy}{fiscal}
            \subidx{fiscal}{strategy}
            \subidx{\market}{fiscal strategy}
            \subidx{\market}{product portfolio}
            \subidx{\market}{product diversity}
            \subidx{\market}{product mix}
            \subidx{\market}{optimum number of products}
            \idx{product portfolio}
            \idx{product diversity}
            \idx{optimum number of products}
            \idx{product mix}

            As an example of ``product portfolio'' management, suppose
            a company re-invests {\twoponehundred}\% of its rate of
            revenue returns, (per {\timescale},) in development,
            marketing, sales, and distribution of new products.
            Further suppose that the company has two products, and a
            fractal analysis of the individual product rate of revenue
            return time series indicates that one product has a
            Shannon probability of 0.65, and the other has a Shannon
            probability of 0.55. Then the percentage of re-investment
            in the first product would be $(2 \cdot 0.65 - 1) \cdot
            {\twoponehundred}$, percent of the rate of revenue
            returns, and $(2 \cdot 0.55 - 1) \cdot {\twoponehundred}$
            percent for the second product, implying that the company
            should diversify its product line\footnote{The astute
            reader would note that the linear addition was used to add
            the contribution to development of each product. This is a
            ``near term'' interpretation. Actually, in general, the
            method used should be a root mean square process,
            dependent on the Hurst Coefficient, $H$, where
            $P_{total}^H = P_1^H + P_2^H + \cdots$, where $P_n$ is the
            contribution to each individual product. For a Brownian
            motion, or random walk type of fractal the Hurst
            Coefficient is a function of time into the future. For the
            ``near term,'' the Hurst coefficient is very near unity,
            meaning the summation process is linear. For the ``long
            term,'' $H \approx 0.5$, or a standard root mean square
            summation process should be used. If $H$ is $0.5$ then the
            market is termed a Brownian motion, or random walk
            process. If it is larger than 0.5, it is termed fractional
            Brownian motion process. For a random walk process, ``near
            term'' and ``far term'' are quantitatively differentiated
            on the Hurst Coefficient graph where $1 - \ln (t) = 0.5
            \cdot \ln (t)$, or when $\ln (t) = 2$, or $t =
            7.389\ldots$ See~\cite[pp. 67, 83-84]{Peters:CAOITCM}
            and~\cite[pp. 129, 159]{Schroeder} for particulars on the
            implications of the Hurst Coefficient and root mean square
            summation issues.}.  Note that this is a ``bet hedging''
            metric methodology, and assumes that the products have
            uncorrelated revenue return rates. If this re-investment
            methodology is not feasible, perhaps for strategic
            financial reasons, then the re-investment in both products
            should total the ${\twoponehundred}$\%, and the investment
            in each product should be made at a ratio of $\frac{(2
            \cdot 0.65 - 1)}{(2 \cdot 0.55 - 1)} = 3 : 1$,
            respectively. Note that this ``bet hedging'' can be used
            to define the optimal number of products that can be
            supported on the rate of revenue returns. If it assumed
            that all products are ``typical'' for the {\market}, as a
            standard bench mark, then the optimal number will be
            $\frac{1}{{\twopone}}$. Note that this is a
            ``theoretical'' value, since not all products are
            ``typical,'' and there may be strategic reasons, for
            example product leveraging, that may increase the number
            of products above the optimum. However, most of the
            revenue should come from the optimal number of products,
            since having more products will decrease the amount of the
            potential investment in each product, and having less than
            the optimum number of products will increase the risk that
            many of the products could suffer a ``down market''
            concurrently, impacting the rate of revenue returns.  As
            another interesting interpretation of the optimal
            ``hedging of bets,'' in product portfolio strategy, and
            considering the graph of the normalized increments
            presented in Figure~\ref{\SETLABEL:TF}, if the
            organization is running optimally, then these products
            will generate, at least in principle, one standard
            deviation, approximately $0.8413 = 84.13$\% of the future
            growth in rate of revenue returns. Naturally, these are
            approximations, and the values are an approximation to a,
            probably, complex process, and appropriate scrutiny should
            be exercised before making specific projections.  As yet
            another example of ``product portfolio'' management,
            consider the issue of product mix. In this interpretation,
            {\twoponehundred}\% of the product manufactured should be
            ``proprietary,'' while the rest is ``industry standard.''
            As yet another possibility, {\twoponehundred}\% of the
            product manufactured should be predatory into new markets,
            and the remainder in markets that are ``traditional'' for
            the company.

% Local Variables:
% TeX-parse-self: t
% TeX-auto-save: t
% TeX-master: "fractal.tex"
% End:


        \subsubsection{Observations on the Fixed Increment Approximation for Fiscal Strategy}

            A re-investment of {\twoponehundred} of the rate of
            revenue returns per {\timescale} does not seem
            inconsistent with the industry averages, since it includes
            investments in research and development, additional
            manufacturing infrastructure, advertising,
            etc. Additionally, a product mix of {\twoponehundred}\%
            ``proprietary'' and the remainder ``industry standard''
            products seems consistent with the industry analyst
            ``20/80'' rule. The value of one standard deviation,
            $84.13$\%, of the revenue return rate being generated by
            $\frac{1}{{\twopone}}$ products seems consistent with the
            industry, also.

        %
% -----------------------------------------------------------------------------
%
% A license is hereby granted to reproduce this software source code and
% to create executable versions from this source code for personal,
% non-commercial use.  The copyright notice included with the software
% must be maintained in all copies produced.
%
% THIS PROGRAM IS PROVIDED "AS IS". THE AUTHOR PROVIDES NO WARRANTIES
% WHATSOEVER, EXPRESSED OR IMPLIED, INCLUDING WARRANTIES OF
% MERCHANTABILITY, TITLE, OR FITNESS FOR ANY PARTICULAR PURPOSE.  THE
% AUTHOR DOES NOT WARRANT THAT USE OF THIS PROGRAM DOES NOT INFRINGE THE
% INTELLECTUAL PROPERTY RIGHTS OF ANY THIRD PARTY IN ANY COUNTRY.
%
% Copyright (c) 1994-2006, John Conover, All Rights Reserved.
%
% Comments and/or bug reports should be addressed to:
%
%     john@email.johncon.com (John Conover)
%
% -----------------------------------------------------------------------------
%
% Revision: \RCSRevision \\
% Revision Time: \RCSTime UMT \\
% Revision Date: \RCSDate \\
% Revision Id: \RCSId \\
% Revision File: \RCSLog \\
\RCS $Revision: 0.0 $
\RCS $Date: 2006/01/20 04:38:13 $
\RCS $Id: companies.tex,v 0.0 2006/01/20 04:38:13 john Exp $
% $Log: companies.tex,v $
% Revision 0.0  2006/01/20 04:38:13  john
% Initial version
%
%
    \subsection{Number of Companies}
        \label{\SETLABEL:QNC}

        \subidx{\market}{number of companies}
        \subidx{number of companies}{analysis}
        \subidx{analysis}{number of companies}
        \subidx{Shannon}{probability}
        \subidx{probability}{Shannon}
        This section evaluates the approximate, or ``average,'' number
        of companies in the {\market}, and uses the method outlined in
        Chapter~\ref{general}, Section~\ref{aftsma}. Since the
        average, $avg_{ind}$, and the root mean square, $rms_{ind}$,
        of the normalized increments of the {\market} time series is
        \datafractionmean, and \datafractionrms respectively, the
        number of companies participating in the market can be
        calculated by Equation~\ref{ncompanies} to be {\ncompanies}.

        If this value seems consistent number of companies in the
        {\market}, within the assumptions outlined in
        Chapter~\ref{general}, Section~\ref{aftsma}, then it would
        seem that there is some circumstantial or indirect evidence
        that the companies participating in the {\market} are
        operating optimally, and the ``average'' Shannon probability,
        $P$ for each participating company would be, using
        Equation~\ref{pncompanies}, {\pncompanies}, which would be the
        value which should be used in Section~\ref{\SETLABEL:FS} for
        each participating company if market expansion was to be
        consistent with the rest of the industry. However, if the
        Shannon probability derived in Section~\ref{\SETLABEL:FS} is
        greater than the average Shannon probability for the companies
        participating in the {\market}, as derived in this section,
        then the market would, possibly, be exploitable with the
        fiscal strategy outlined in Section~\ref{\SETLABEL:FS}. The
        maximum exploitability for the {\market} is derived in
        Section~\ref{\SETLABEL:MAXSHANNON}, but it is probably of
        doubtful practicality.

        Note that these optimizations would maximize a company's
        market growth. Since there are probably many companies
        competing in the market place, this would not necessarily
        maximize a company's P\&L, as described in
        Chapter~\ref{general}, Section~\ref{ompl}. The Shannon
        probability that maximizes market share in the {\market} is
        \pncompanies, with several alternative solutions listed in the
        previous paragraph. However, these should be contrasted to the
        Shannon probability that maximizes a company's P\&L which is
        \avgrms~in the {\market}. In all cases, the fraction of the
        P\&L that should be ``wagered'' on the future, $f$, should be:

        \begin{equation}
            f = 2P - 1
        \end{equation}

        \noindent where $P$ is the particular Shannon probability
        chosen optimize a particular fiscal strategy. Interestingly,
        the measured Shannon probability of the {\market} would tend
        to indicate that the companies participating in the market
        have chosen a fiscal strategy that optimizes market growth, as
        opposed to capital growth.

        \subidx{\market}{increasing returns}
        \subidx{economic increasing returns}{\market}
        As interesting interpretation of these exploitive issues,
        since all three fiscal strategies will result in exponential
        market growth for every company participating in the market,
        is that they may represent, perhaps, an example of
        ``increasing returns.''

% Local Variables:
% TeX-parse-self: t
% TeX-auto-save: t
% TeX-master: "fractal.tex"
% End:


        %
% -----------------------------------------------------------------------------
%
% A license is hereby granted to reproduce this software source code and
% to create executable versions from this source code for personal,
% non-commercial use.  The copyright notice included with the software
% must be maintained in all copies produced.
%
% THIS PROGRAM IS PROVIDED "AS IS". THE AUTHOR PROVIDES NO WARRANTIES
% WHATSOEVER, EXPRESSED OR IMPLIED, INCLUDING WARRANTIES OF
% MERCHANTABILITY, TITLE, OR FITNESS FOR ANY PARTICULAR PURPOSE.  THE
% AUTHOR DOES NOT WARRANT THAT USE OF THIS PROGRAM DOES NOT INFRINGE THE
% INTELLECTUAL PROPERTY RIGHTS OF ANY THIRD PARTY IN ANY COUNTRY.
%
% Copyright (c) 1994-2006, John Conover, All Rights Reserved.
%
% Comments and/or bug reports should be addressed to:
%
%     john@email.johncon.com (John Conover)
%
% -----------------------------------------------------------------------------
%
% Revision: \RCSRevision \\
% Revision Time: \RCSTime UMT \\
% Revision Date: \RCSDate \\
% Revision Id: \RCSId \\
% Revision File: \RCSLog \\
\RCS $Revision: 0.0 $
\RCS $Date: 2006/01/20 04:38:13 $
\RCS $Id: operations.tex,v 0.0 2006/01/20 04:38:13 john Exp $
% $Log: operations.tex,v $
% Revision 0.0  2006/01/20 04:38:13  john
% Initial version
%
%
    \subsection{Fixed Increment Approximation for Operational Strategy}
        \label{\SETLABEL:OPS}.

        This section derives various values based on the ``average''
        of the normalized increments presented in
        Figure~\ref{\SETLABEL:TFA}. These values are an approximation
        to a, probably, complex process with a distribution shown in
        Figure~\ref{\SETLABEL:TF}. These values will be used in a
        fixed increment Brownian fractal analysis and simulation of
        the {\market}, and may, or may not, provide adequate accuracy
        for projections.

        \subidx{\market}{fiscal strategy}
        \subidx{\market}{Shannon probability}
        \subidx{strategy}{fiscal}
        \subidx{fiscal}{strategy}
        \subidx{Shannon}{probability}
        \subidx{probability}{Shannon}
        It should be noted that the analysis of fiscal strategy,
        presented in Section~\ref{\SETLABEL:FS}, is derived from the
        {\market} metrics and may, or may not, be maximally
        optimal. For the optimal fiscal strategy, which may be
        exploitable, see Section~\ref{\SETLABEL:MAXSHANNON}.

        \subidx{strategy}{exploitable}
        \subidx{exploitable}{strategy}
        \subidx{\market}{windows of opportunity}
        \idx{windows of opportunity}
        \subidx{decision}{obsolete}
        \subidx{obsolete}{decision}
        \subidx{decision}{timeliness}
        \subidx{timeliness}{decision}
        \subidx{rate of revenue returns}{forecast}
        \subidx{forecast}{rate of revenue returns}
        An additional exploitable strategy may be time itself.
        Equations~\ref{\SETLABEL:V},~\ref{\SETLABEL:R},
        and,~\ref{\SETLABEL:MA}, are, essentially, metrics on how fast
        a decision, which is based on information concerning the
        current status of the {\market}, becomes obsolete. Obviously,
        how long a decision is expected to remain relevant should be
        addressed as an operational necessity in strategic planning
        and project management. Figures~\ref{\SETLABEL:FN},
        and,~\ref{\SETLABEL:FF} compare methods of approximation of
        the ``forecastability'' of rate of revenue returns in the
        {\market} for the near term and far
        term~\cite[pp. 83-84]{Peters:CAOITCM}, respectively. As a
        general rule, caution must be exercised when making decisions
        that will span a time interval larger than the time interval
        where the ``forecastability'' of rate of revenue returns drops
        below 50\%. Beyond this time interval, the chances increase
        that the competitive and market forces will alter the market
        environment in a possibly detrimental unanticipated
        fashion. Obviously, there is significant advantage in
        ``timeliness'' of development, manufacturing, and distribution
        of products and services that are consistent with this
        temporal agenda. Automation of these processes, if executed
        consistently with this agenda, should be considered a
        competitive advantage.

        \subidx{strategy}{exploitable}
        \subidx{exploitable}{strategy}
        \subidx{rate of revenue returns}{forecast}
        \subidx{forecast}{rate of revenue returns}
        \idx{product life cycle}
        \idx{life cycle, product}
        In some sense, this temporal agenda defines the ``average''
        product or service life cycle in the {\market}. When the
        ``forecastability'' of rate of revenue returns drops below
        50\%, there is an even chance that the rate of revenue returns
        for the product or service will change in a detrimental
        fashion. If it is assumed that a product or service life cycle
        consists of a ramp up, a maintenence interval, and a ramp
        down, then, if all three life cycle intervals are equal, the
        product life cycle will be, approximately, three times the
        time interval where the ``forecastability'' of rate of revenue
        returns drops below 50\%. Although probably not an accurate
        prediction of product or service life cycle, the technique may
        be used as a conceptual approximation to the dynamics of
        ``market windows.\footnote{For example, consider the market
        for table salt. Since it has inelastic supply and demand
        curves, and is a necessary requirement for life, it would be
        expected that the Hurst coefficient would be very near
        unity---ignoring competitive pressures in the market. The
        predictability of the table salt market would, therefore, be
        expected to be relatively good, over time.}''  The conceptual
        approximation will probably predict a ``conservative'' or
        ``pessimistic'' value in relation to actual markets.

        \begin{figure}[ht]
            \begin{center}
                \begin{minipage}[t]{0.45\textwidth}
                    \epsfxsize=1.0\linewidth
                    \epsffile{\directory/datahurstlownear.eps}
                    \caption[{\market}, ``forecastability'' of near
                        term rate of revenue returns]{{\market},
                        ``forecastability'' of near term rate of
                        revenue returns. Although the error function
                        is the most accurate, for the near term,
                        $H^{t} = \thurstlow^{t}$ may be used as a
                        reliable metric of ``forecastability'' of the
                        rate of revenue returns.}
                    \label{\SETLABEL:FN}
                \end{minipage}
                \hfill
                \begin{minipage}[t]{0.45\textwidth}
                    \epsfxsize=1.0\linewidth
                    \epsffile{\directory/datahurstlowfar.eps}
                    \caption[{\market}, ``forecastability'' of far
                        term rate of revenue returns]{{\market},
                        ``forecastability'' of far term rate of
                        revenue returns. Although the error function
                        is the most accurate, for the far term,
                        $\frac{1}{\sqrt{t}}$ may be used as a reliable
                        metric of ``forecastability'' of the rate of
                        revenue returns.}
                    \label{\SETLABEL:FF}
                \end{minipage}
            \end{center}
        \end{figure}

        \idx{operations research}
        As an interesting interpretation of the data presented in
        Figure~\ref{\SETLABEL:FN}, there may be, perhaps, some
        applicability to such operational agendas as inventory
        control. Maintaining too little inventory, obviously, will
        create a situation where the organization can not exploit
        market expansion, and maintaining too much inventory,
        likewise, would over extend the company, creating unnecessary
        losses when the market contracts. The company should maintain
        inventory levels that do not exceed, from
        Equation~\ref{\SETLABEL:MA}, ${\thurstlow}^{n} = 0.5$
        {\timescale}s of operations. Since the optimal amount of
        inventory and, from Equation~\ref{\SETLABEL:V}, the variance
        of change in the rate of revenue returns in the future can be
        calculated, there may, perhaps, be some applicability to a
        forecasting methodology that can be incorporated into other
        areas of operations research, for example the linear algebras
        using simplex methodologies for optimization of manufacturing
        processes. Traditionally, these forecasts are made by the
        sales department, and are subject to various subjective
        biases.

% Local Variables:
% TeX-parse-self: t
% TeX-auto-save: t
% TeX-master: "fractal.tex"
% End:


        \subsubsection{Observations on the Fixed Increment Approximation for Operational Strategy}

            As an interesting interpretation of
            Figure~\ref{\SETLABEL:FF}, and evaluating the
            approximation $\frac{1}{\sqrt{t}}$ at 60 months gives a
            probability that the market will still have the same
            agenda of about $0.12909945$, or about 1 in 8. This is
            commensurate with numbers from the venture
            community\footnote{For example, see ``IEEE Engineering
            Management Review,'' Volume 23 Number 3, Fall 1995,
            pp. 83}. Of course new venture backed companies fail for
            many reasons, but market appropriateness to product
            portfolio 60 months in the future may be a major
            contributor. Additionally, the success rate of development
            projects of 8 month duration, which have a market success
            rate of about 1 in 3, seems consistent with
            $\frac{1}{\sqrt{3}} = 0.353553391$. Naturally, projects
            fail in the market for many reasons, but market
            appropriateness, in a dynamic market environment may be a
            major contributor to failure.

            As mentioned in Section~\ref{\SETLABEL:H},
            Equation~\ref{\SETLABEL:MA}, and the preceeding section,
            approximately 3 times the value where ${\thurstlow}^{n} =
            0.5$ could be interpreted as an approximation to the
            ``average'' product life cycle. This seems consistent with
            the 6 to 12 month life cycles quoted by many industry
            analyst. In addition, maintaining inventory levels that do
            not exceed the anticipated requirements of
            $\frac{\ln{0.5}}{\ln{\thurstlow}}$ many {\timescale}s
            seems consistent with the author's experience in the
            industry.

        For convenience of comparison, converting from quarters to
        months by dividing the logarithmic returns by 3:

        \renewcommand{\timescale}{month}
        \renewcommand{\datafractionmean}{0.044437}
\renewcommand{\datafractionmeanbits}{0.062725}
\renewcommand{\datafractionmeanq}{0.014812}
\renewcommand{\datafractionmeanbitsq}{0.021213}
\renewcommand{\datafractionstddev}{0.064421}
\renewcommand{\datafractionrms}{0.025913}
\renewcommand{\avgrms}{1.357427}
\renewcommand{\ncompanies}{66.177345}
\renewcommand{\pncompanies}{0.605400}
\renewcommand{\datafractionabsmean}{0.061981}
\renewcommand{\datafractionabsstddev}{0.047389}
\renewcommand{\datafractionconstant}{0.039513}
\renewcommand{\datafractionconstantbits}{0.055908}
\renewcommand{\datafractionconstantq}{0.013171}
\renewcommand{\datafractionconstantbitsq}{0.018878}
\renewcommand{\datafractionslope}{0.000197}
\renewcommand{\datafractionabsconstant}{0.078868}
\renewcommand{\datafractionabsslope}{-0.000675}
\renewcommand{\hurstall}{0.644727}
\renewcommand{\hurstlow}{1.028920}
\renewcommand{\hurstlowtwo}{2.057840}
\renewcommand{\hurstlowhundred}{102.892000}
\renewcommand{\hcalcall}{0.712999}
\renewcommand{\hcalclow}{0.745601}
\renewcommand{\shannonmax}{0.826923}
\renewcommand{\twoponemax}{0.653846}
\renewcommand{\logreturns}{0.019605}
\renewcommand{\twologreturns}{1.013682}
\renewcommand{\twologreturnshundred}{1.368190}
\renewcommand{\oneoverlogreturns}{51.007396}
\renewcommand{\pmax}{0.823529}
\renewcommand{\twopminusone}{0.647059}
\renewcommand{\rmsp}{0.016767}
\renewcommand{\twopx}{0.848668}
\renewcommand{\sigmap}{0.054672}
\renewcommand{\tsunfairbrownianfractionmean}{0.049753}
\renewcommand{\tsunfairbrownianfractionstddev}{0.060339}
\renewcommand{\shannonlogreturns}{0.582242}
\renewcommand{\shannonlogreturnshundred}{58.224200}
\renewcommand{\twopone}{0.164484}
\renewcommand{\twoponehundred}{16.448400}
\renewcommand{\hundredtwoponehundred}{83.551600}
\renewcommand{\hundredshannonlogreturnshundred}{41.775800}
\renewcommand{\datatslsqepbits}{0.017926}
\renewcommand{\thurstall}{0.725956}
\renewcommand{\thurstlow}{1.025249}
\renewcommand{\thurstlowtwo}{2.050498}
\renewcommand{\thurstlowhundred}{102.524900}
\renewcommand{\thcalcall}{0.885411}
\renewcommand{\thcalclow}{0.871338}
\renewcommand{\chisquared}{9.194000}
\renewcommand{\critical}{42.557000}

        \renewcommand{\SETLABEL}{\LABPRE:NAICMQ}
        \renewcommand{\datafractionmean}{\datafractionmeanq}
        \renewcommand{\datafractionconstant}{\datafractionconstantq}
        \renewcommand{\datafractionmeanbits}{\datafractionmeanbitsq}
        \renewcommand{\datafractionconstantbits}{\datafractionconstantbitsq}

        %
% -----------------------------------------------------------------------------
%
% A license is hereby granted to reproduce this software source code and
% to create executable versions from this source code for personal,
% non-commercial use.  The copyright notice included with the software
% must be maintained in all copies produced.
%
% THIS PROGRAM IS PROVIDED "AS IS". THE AUTHOR PROVIDES NO WARRANTIES
% WHATSOEVER, EXPRESSED OR IMPLIED, INCLUDING WARRANTIES OF
% MERCHANTABILITY, TITLE, OR FITNESS FOR ANY PARTICULAR PURPOSE.  THE
% AUTHOR DOES NOT WARRANT THAT USE OF THIS PROGRAM DOES NOT INFRINGE THE
% INTELLECTUAL PROPERTY RIGHTS OF ANY THIRD PARTY IN ANY COUNTRY.
%
% Copyright (c) 1994-2006, John Conover, All Rights Reserved.
%
% Comments and/or bug reports should be addressed to:
%
%     john@email.johncon.com (John Conover)
%
% -----------------------------------------------------------------------------
%
% Revision: \RCSRevision \\
% Revision Time: \RCSTime UMT \\
% Revision Date: \RCSDate \\
% Revision Id: \RCSId \\
% Revision File: \RCSLog \\
\RCS $Revision: 0.0 $
\RCS $Date: 2006/01/20 04:38:13 $
\RCS $Id: fiscal.tex,v 0.0 2006/01/20 04:38:13 john Exp $
% $Log: fiscal.tex,v $
% Revision 0.0  2006/01/20 04:38:13  john
% Initial version
%
%
    \subsection{Fixed Increment Approximation for Fiscal Strategy}
        \label{\SETLABEL:FS}

        \subidx{\market}{fiscal strategy}
        \subidx{markets}{analysis}
        \subidx{analysis}{markets}
        \subidx{strategy}{fiscal}
        \subidx{fiscal}{strategy}
        The data in this section is presented in tabular form in
        Section~\ref{\SETLABELREF:LR}. This section derives various
        values based on the ``average'' of the normalized increments
        presented in Figure~\ref{\SETLABEL:TFA}. These values are an
        approximation to a, probably, complex process with a
        distribution shown in Figure~\ref{\SETLABEL:TF}. These values
        will be used in a fixed increment Brownian fractal analysis
        and simulation of the {\market}, and may, or may not, provide
        adequate accuracy for projections.

        For an organization operating in the {\market}, the fiscal
        strategy, commensurate with the aggregate environment, can be
        derived as follows~\cite[pp. 128, pp
        151]{Schroeder},~\cite[pp. 450]{Reza},~\cite[pp. 270]{Pierce}:
        \vspace{0.15in}

        \subsubsection{Logarithmic Returns}
            \label{\SETLABEL:LR}

            \subidx{logarithmic}{returns}
            \subidx{returns}{logarithmic}
            \subidx{\market}{logarithmic returns}
            The logarithmic returns can be calculated by various
            means. Four will be presented here, for comparison.

            \subidx{programs}{tsnormal}
            \subidx{tsnormal}{program}
            \subidx{logarithmic}{returns}
            \subidx{returns}{logarithmic}
            The logarithmic returns, in bits, $bits$, as computed from
            the mean, by the program {\it tsnormal}\/, which is
            described in Chapter~\ref{programs}, and is presented in
            Figure~\ref{\SETLABEL:TF}, and Equation~\ref{abits} from
            Section~\ref{ereturns} in Chapter~\ref{general}:

            \begin{equation}
                bits = \frac{\ln \left({\datafractionmean} + 1\right)}{\ln \left(2\right)} = \datafractionmeanbits
            \end{equation}

            \subidx{programs}{tslsq}
            \subidx{tslsq}{program}
            \subidx{logarithmic}{returns}
            \subidx{returns}{logarithmic}
            \noindent By comparison, the logarithmic returns, in bits,
            $bits$, as computed from the constant in the least squares
            approximation, using the program {\it tslsq}\/, which is briefly
            described in Chapter~\ref{programs}, as presented in
            Figure~\ref{\SETLABEL:TF}, and Equation~\ref{abits} from
            Section~\ref{ereturns} in Chapter~\ref{general}:

            \begin{equation}
                bits = \frac{\ln \left({\datafractionconstant} + 1\right)}{\ln \left(2\right)} = \datafractionconstantbits
            \end{equation}

            Note that if the mean is not constant in
            Figure~\ref{\SETLABEL:TF}, this method will not provide
            accurate results.

            \subidx{programs}{tslsq}
            \subidx{tslsq}{program}
            \subidx{logarithmic}{returns}
            \subidx{returns}{logarithmic}
            \noindent And by yet another comparison, using the program
            {\it tslsq}\/, which is briefly described in
            Chapter~\ref{programs}, with the -e -p options, to provide
            a formula for the least squares exponential fit to the
            time series data set presented in
            Figure~\ref{\SETLABEL:TS}:

            \begin{equation}
                bits = {\datatslsqepbits}
            \end{equation}

            \subidx{programs}{tslogreturns}
            \subidx{tslogreturns}{program}
            \subidx{logarithmic}{returns}
            \subidx{returns}{logarithmic}
            \noindent And finally, by comparison, from the
            {\it tslogreturns}\/ program, which is briefly described
            in Chapter~\ref{programs}, with the -p option, to provide
            a formula for the logarithmic returns of the time series
            data set presented in Figure~\ref{\SETLABEL:TS}:

            \begin{equation}
                bits = {\logreturns}
            \end{equation}

        \subsubsection{Calculation of Shannon Probability}
            \label{\SETLABEL:SP}

            \subidx{\market}{Shannon probability}
            Ideally, all of the values presented in
            Section~\ref{\SETLABEL:LR} would be equal. Using the
            logarithmic returns provided by the {\it tslogreturns}\/
            program, to be consistent
            with~\cite[pp. 81]{Peters:CAOITCM}

            \subidx{programs}{tslogreturns}
            \subidx{tslogreturns}{program}
            \begin{equation}
                2^{{\logreturns}t}
            \end{equation}

            \noindent therefore:
            \begin{equation}
                C\left(p\right) = {\logreturns}
            \end{equation}
            \subidx{programs}{tsshannon}
            \subidx{tsshannon}{program}
            \subidx{Shannon}{probability}
            \subidx{probability}{Shannon}
            \noindent and, {\it tsshannon}\/ {\logreturns} gives:
            \begin{equation}
                \label{\SETLABEL:F0}
                C\left({\shannonlogreturns}\right) = {\logreturns}
            \end{equation}
            \noindent therefore:
            \begin{eqnarray}
                2^{C\left({\shannonlogreturns}\right)} & = & 2^{\logreturns}\\
                                                       & = & {\twologreturns}\\
                                                       & = & {\twologreturnshundred}\%
            \end{eqnarray}
            \noindent and:
            \begin{eqnarray}
                2p - 1 & = & \left(2 \cdot {\shannonlogreturns}\right) - 1\\
                       & = & {\twopone}\\
                       \label{\SETLABEL:F1}
                       & = & {\twoponehundred}\%
            \end{eqnarray}

            \subidx{\market}{fiscal strategy}
            \subidx{markets}{analysis}
            \subidx{analysis}{markets}
            \subidx{strategy}{fiscal}
            \subidx{fiscal}{strategy}
            \subidx{\market}{fiscal strategy}
            \subidx{\market}{growth rate}
            Presuming the simplified assumptions outlined in
            Section~\ref{assumptions}, the ``typical'' organization
            operating in the {\market} executes a long term fiscal
            strategy, commensurate with the aggregate environment,
            that is to invest, every {\timescale}, in sufficient
            additional resources and infrastructure, to increase the
            manufacturing of goods and services by {\twoponehundred}\%
            of its rate of revenue returns, (per {\timescale}.) As a
            conceptual model, the remaining {\hundredtwoponehundred}\%
            will be held in ``reserve'' with a
            {\shannonlogreturnshundred}\% chance of making twice the
            {\twoponehundred}\% back, (and a
            {\hundredshannonlogreturnshundred}\% chance of making
            0.0,) in one {\timescale}, on the average, for an average
            growth in its rate of revenue returns, (per {\timescale},)
            of {\twologreturnshundred}\%, or a doubling of its rate of
            revenue returns, (per {\timescale},) in
            {\oneoverlogreturns} {\timescale}s.

        \subsubsection{Example Fixed Increment Approximation Fiscal Strategies}

            \subidx{\market}{fiscal strategy}
            \subidx{markets}{analysis}
            \subidx{analysis}{markets}
            \subidx{strategy}{fiscal}
            \subidx{fiscal}{strategy}
            \subidx{\market}{fiscal strategy}
            \subidx{\market}{growth rate}
            \subidx{\market}{management metric}
            \idx{management metric}
            A possible metric on the effectiveness of long term fiscal
            management could possibly be that if an investment of
            {\twoponehundred}\% per {\timescale} of the rate of
            revenue returns, (per {\timescale},) is made in resources
            and infrastructure, then the rate of revenue returns would
            be expected to increase by {\twologreturnshundred}\%, per
            {\timescale}, on average.

            Note that the metrics presented in this section are
            representative of the {\market} as an aggregate whole, and
            may or may not be accurate representations for any
            particular participant in the environment. Of interest to
            the participants in the environment would be a similar
            analysis of each product or service rendered in the
            marketplace.

            \subidx{\market}{fiscal strategy}
            \subidx{markets}{analysis}
            \subidx{analysis}{markets}
            \subidx{strategy}{fiscal}
            \subidx{fiscal}{strategy}
            \subidx{\market}{fiscal strategy}
            As a simple illustrative example, a company operating in
            this environment might obtain a credit line from a bank
            that is equal to {\twoponehundred}\% of its rate of
            revenue returns, (per {\timescale},) to finance additional
            operations. In this simple scenario, the company would use
            its revenue base as collateral for the loan. Some
            {\timescale}s, depending on the {\market}'s environment,
            the company's rate of revenue returns exceeds what was
            borrowed from the bank, and the loan is repaid in
            full. Other {\timescale}s, the company must default, and
            the bank seizes a portion of the company's revenue base to
            pay the delinquent loan. However, on the average, the
            company will expand its rate of revenue returns at
            {\twologreturnshundred}\% per {\timescale}.

            \subidx{\market}{fiscal strategy}
            \subidx{markets}{analysis}
            \subidx{analysis}{markets}
            \subidx{strategy}{fiscal}
            \subidx{fiscal}{strategy}
            \subidx{\market}{fiscal strategy}
            As another simple example, a company re-invests
            {\twoponehundred}\% of its rate of revenue returns, (per
            {\timescale},) in development, marketing, sales, and
            distribution of new products.  Although some products will
            be successful and the return on the investment will exceed
            the {\twoponehundred}\% per {\timescale} investment,
            others will not. However, on the average, the company will
            expand it gross rate of revenue returns at
            {\twologreturnshundred}\% per {\timescale}.

            \subidx{\market}{fiscal strategy}
            \subidx{markets}{analysis}
            \subidx{analysis}{markets}
            \subidx{strategy}{fiscal}
            \subidx{fiscal}{strategy}
            \subidx{\market}{fiscal strategy}
            \subidx{\market}{product portfolio}
            \subidx{\market}{product diversity}
            \subidx{\market}{product mix}
            \subidx{\market}{optimum number of products}
            \idx{product portfolio}
            \idx{product diversity}
            \idx{optimum number of products}
            \idx{product mix}

            As an example of ``product portfolio'' management, suppose
            a company re-invests {\twoponehundred}\% of its rate of
            revenue returns, (per {\timescale},) in development,
            marketing, sales, and distribution of new products.
            Further suppose that the company has two products, and a
            fractal analysis of the individual product rate of revenue
            return time series indicates that one product has a
            Shannon probability of 0.65, and the other has a Shannon
            probability of 0.55. Then the percentage of re-investment
            in the first product would be $(2 \cdot 0.65 - 1) \cdot
            {\twoponehundred}$, percent of the rate of revenue
            returns, and $(2 \cdot 0.55 - 1) \cdot {\twoponehundred}$
            percent for the second product, implying that the company
            should diversify its product line\footnote{The astute
            reader would note that the linear addition was used to add
            the contribution to development of each product. This is a
            ``near term'' interpretation. Actually, in general, the
            method used should be a root mean square process,
            dependent on the Hurst Coefficient, $H$, where
            $P_{total}^H = P_1^H + P_2^H + \cdots$, where $P_n$ is the
            contribution to each individual product. For a Brownian
            motion, or random walk type of fractal the Hurst
            Coefficient is a function of time into the future. For the
            ``near term,'' the Hurst coefficient is very near unity,
            meaning the summation process is linear. For the ``long
            term,'' $H \approx 0.5$, or a standard root mean square
            summation process should be used. If $H$ is $0.5$ then the
            market is termed a Brownian motion, or random walk
            process. If it is larger than 0.5, it is termed fractional
            Brownian motion process. For a random walk process, ``near
            term'' and ``far term'' are quantitatively differentiated
            on the Hurst Coefficient graph where $1 - \ln (t) = 0.5
            \cdot \ln (t)$, or when $\ln (t) = 2$, or $t =
            7.389\ldots$ See~\cite[pp. 67, 83-84]{Peters:CAOITCM}
            and~\cite[pp. 129, 159]{Schroeder} for particulars on the
            implications of the Hurst Coefficient and root mean square
            summation issues.}.  Note that this is a ``bet hedging''
            metric methodology, and assumes that the products have
            uncorrelated revenue return rates. If this re-investment
            methodology is not feasible, perhaps for strategic
            financial reasons, then the re-investment in both products
            should total the ${\twoponehundred}$\%, and the investment
            in each product should be made at a ratio of $\frac{(2
            \cdot 0.65 - 1)}{(2 \cdot 0.55 - 1)} = 3 : 1$,
            respectively. Note that this ``bet hedging'' can be used
            to define the optimal number of products that can be
            supported on the rate of revenue returns. If it assumed
            that all products are ``typical'' for the {\market}, as a
            standard bench mark, then the optimal number will be
            $\frac{1}{{\twopone}}$. Note that this is a
            ``theoretical'' value, since not all products are
            ``typical,'' and there may be strategic reasons, for
            example product leveraging, that may increase the number
            of products above the optimum. However, most of the
            revenue should come from the optimal number of products,
            since having more products will decrease the amount of the
            potential investment in each product, and having less than
            the optimum number of products will increase the risk that
            many of the products could suffer a ``down market''
            concurrently, impacting the rate of revenue returns.  As
            another interesting interpretation of the optimal
            ``hedging of bets,'' in product portfolio strategy, and
            considering the graph of the normalized increments
            presented in Figure~\ref{\SETLABEL:TF}, if the
            organization is running optimally, then these products
            will generate, at least in principle, one standard
            deviation, approximately $0.8413 = 84.13$\% of the future
            growth in rate of revenue returns. Naturally, these are
            approximations, and the values are an approximation to a,
            probably, complex process, and appropriate scrutiny should
            be exercised before making specific projections.  As yet
            another example of ``product portfolio'' management,
            consider the issue of product mix. In this interpretation,
            {\twoponehundred}\% of the product manufactured should be
            ``proprietary,'' while the rest is ``industry standard.''
            As yet another possibility, {\twoponehundred}\% of the
            product manufactured should be predatory into new markets,
            and the remainder in markets that are ``traditional'' for
            the company.

% Local Variables:
% TeX-parse-self: t
% TeX-auto-save: t
% TeX-master: "fractal.tex"
% End:


        \renewcommand{\SETLABEL}{\LABPRE:NAICM}
        \renewcommand{\datafractionmean}{0.008052}
\renewcommand{\datafractionmeanbits}{0.011570}
\renewcommand{\datafractionmeanq}{0.002684}
\renewcommand{\datafractionmeanbitsq}{0.003867}
\renewcommand{\datafractionstddev}{0.038579}
\renewcommand{\datafractionrms}{0.039311}
\renewcommand{\avgrms}{0.602414}
\renewcommand{\ncompanies}{5.210454}
\renewcommand{\pncompanies}{0.544866}
\renewcommand{\datafractionabsmean}{0.029745}
\renewcommand{\datafractionabsstddev}{0.025769}
\renewcommand{\datafractionconstant}{0.010041}
\renewcommand{\datafractionconstantbits}{0.014414}
\renewcommand{\datafractionconstantq}{0.003347}
\renewcommand{\datafractionconstantbitsq}{0.004821}
\renewcommand{\datafractionslope}{-0.000021}
\renewcommand{\datafractionabsconstant}{0.035145}
\renewcommand{\datafractionabsslope}{-0.000057}
\renewcommand{\hurstall}{0.659558}
\renewcommand{\hurstlow}{0.707509}
\renewcommand{\hurstlowtwo}{1.415018}
\renewcommand{\hurstlowhundred}{70.750900}
\renewcommand{\hcalcall}{0.184942}
\renewcommand{\hcalclow}{0.102042}
\renewcommand{\shannonmax}{0.604167}
\renewcommand{\twoponemax}{0.208334}
\renewcommand{\logreturns}{0.010456}
\renewcommand{\twologreturns}{1.007274}
\renewcommand{\twologreturnshundred}{0.727387}
\renewcommand{\oneoverlogreturns}{95.638868}
\renewcommand{\pmax}{0.602094}
\renewcommand{\twopminusone}{0.204188}
\renewcommand{\rmsp}{0.008027}
\renewcommand{\twopx}{0.208583}
\renewcommand{\sigmap}{0.008047}
\renewcommand{\tsunfairbrownianfractionmean}{0.007862}
\renewcommand{\tsunfairbrownianfractionstddev}{0.038619}
\renewcommand{\shannonlogreturns}{0.560125}
\renewcommand{\shannonlogreturnshundred}{56.012500}
\renewcommand{\twopone}{0.120250}
\renewcommand{\twoponehundred}{12.025000}
\renewcommand{\hundredtwoponehundred}{87.975000}
\renewcommand{\hundredshannonlogreturnshundred}{43.987500}
\renewcommand{\datatslsqepbits}{0.007623}
\renewcommand{\thurstall}{0.633980}
\renewcommand{\thurstlow}{0.710108}
\renewcommand{\thurstlowtwo}{1.420216}
\renewcommand{\thurstlowhundred}{71.010800}
\renewcommand{\thcalcall}{0.247886}
\renewcommand{\thcalclow}{0.171737}
\renewcommand{\chisquared}{2.862000}
\renewcommand{\critical}{42.557000}

        \renewcommand{\timescale}{quarter}

        %
% -----------------------------------------------------------------------------
%
% A license is hereby granted to reproduce this software source code and
% to create executable versions from this source code for personal,
% non-commercial use.  The copyright notice included with the software
% must be maintained in all copies produced.
%
% THIS PROGRAM IS PROVIDED "AS IS". THE AUTHOR PROVIDES NO WARRANTIES
% WHATSOEVER, EXPRESSED OR IMPLIED, INCLUDING WARRANTIES OF
% MERCHANTABILITY, TITLE, OR FITNESS FOR ANY PARTICULAR PURPOSE.  THE
% AUTHOR DOES NOT WARRANT THAT USE OF THIS PROGRAM DOES NOT INFRINGE THE
% INTELLECTUAL PROPERTY RIGHTS OF ANY THIRD PARTY IN ANY COUNTRY.
%
% Copyright (c) 1994-2006, John Conover, All Rights Reserved.
%
% Comments and/or bug reports should be addressed to:
%
%     john@email.johncon.com (John Conover)
%
% -----------------------------------------------------------------------------
%
% Revision: \RCSRevision \\
% Revision Time: \RCSTime UMT \\
% Revision Date: \RCSDate \\
% Revision Id: \RCSId \\
% Revision File: \RCSLog \\
\RCS $Revision: 0.0 $
\RCS $Date: 2006/01/20 04:38:13 $
\RCS $Id: simulation.tex,v 0.0 2006/01/20 04:38:13 john Exp $
% $Log: simulation.tex,v $
% Revision 0.0  2006/01/20 04:38:13  john
% Initial version
%
%
    \subsection{Simulation of Fixed Increment Approximation for Fiscal Strategy}
        \label{\SETLABEL:TSUNFAIRBROWNIAN}

        \subidx{\market}{market simulation}
        The data in this section is presented in tabular form in
        Section~\ref{\SETLABELREF:SIM}.
        Figure~\ref{\SETLABEL:TSUNFAIRBROWNIAN0} represents a
        constructional simulation of the time series data presented in
        Figure~\ref{\SETLABEL:TS}. The program {\it
        tsunfairbrownian}\/, which is briefly described in
        appendix~\ref{programs}, was used in the reconstruction. The
        reconstructed data is superimposed on the original time series
        data.  The program, {\it tsunfairbrownian}\/, essentially,
        constructs the new time series as a Brownian fractal with
        fixed increments---the value of the fixed increment is derived
        from the root mean square average of the normalized increments
        presented in Figure~\ref{\SETLABEL:TF}. The ``quality'' of
        such a reconstruction should be subject to adequate scepticism
        and scrutiny since, in all probability, the normalized
        increments presented in Figure~\ref{\SETLABEL:TF} represent a
        relatively complex process, that may not be ``modeled'' with
        such a simple methodology.

        As a further comparison of the the constructional simulation
        with the original time series data,
        Figure~\ref{\SETLABEL:TSUNFAIRBROWNIAN1} presents a normalized
        histogram of the normalized increments of the reconstructed
        time series, superimposed on the normalized histogram
        presented in Figure~\ref{\SETLABEL:NH}.

        \subidx{\market}{fiscal strategy, simulation}
        \subidx{markets}{simulation}
        \subidx{simulation}{markets}
        \subidx{strategy}{fiscal, simulation}
        \subidx{fiscal}{strategy, simulation}
        \subidx{programs}{tsunfairbrownian}
        \subidx{tsunfairbrownian}{program}
        \begin{figure}[ht]
            \begin{center}
                \begin{minipage}[t]{0.45\textwidth}
                    \epsfxsize=1.0\linewidth
                    \epsffile{\directory/tsunfairbrownian-f.eps}
                    \caption[{\market}, Time series data, empirical and
                        simulated]{{\market}, Time series data, empirical
                        and simulated, using the program {\it tsunfairbrownian}\/
                        with f = {\datafractionrms}. This data is
                        superimposed on the data presented in
                        Figure~\ref{\SETLABEL:TS}.}
                    \label{\SETLABEL:TSUNFAIRBROWNIAN0}
                \end{minipage}
                \hfill
                \begin{minipage}[t]{0.45\textwidth}
                    \epsfxsize=1.0\linewidth
                    \epsffile{\directory/tsunfairbrownian-f.tsfraction.tsnormal-s30.eps}
                    \caption[{\market}, normalized histogram,
                        empirical and simulated]{{\market}, normalized
                        histogram of the normalized increments of the
                        time series data shown in
                        Figure~\ref{\SETLABEL:TSUNFAIRBROWNIAN0},
                        empirical and simulated.  The empirical data
                        has a mean of {\datafractionmean}, with a
                        standard deviation of {\datafractionstddev}.
                        By comparison, the simulated data has a mean
                        of {\tsunfairbrownianfractionmean} with a
                        standard deviation of
                        {\tsunfairbrownianfractionstddev}. This data
                        is superimposed on the data presented in
                        Figure~\ref{\SETLABEL:NH}. The area under the
                        four curves is identical.}
                    \label{\SETLABEL:TSUNFAIRBROWNIAN1}
                \end{minipage}
            \end{center}
        \end{figure}

% Local Variables:
% TeX-parse-self: t
% TeX-auto-save: t
% TeX-master: "fractal.tex"
% End:


        %
% -----------------------------------------------------------------------------
%
% A license is hereby granted to reproduce this software source code and
% to create executable versions from this source code for personal,
% non-commercial use.  The copyright notice included with the software
% must be maintained in all copies produced.
%
% THIS PROGRAM IS PROVIDED "AS IS". THE AUTHOR PROVIDES NO WARRANTIES
% WHATSOEVER, EXPRESSED OR IMPLIED, INCLUDING WARRANTIES OF
% MERCHANTABILITY, TITLE, OR FITNESS FOR ANY PARTICULAR PURPOSE.  THE
% AUTHOR DOES NOT WARRANT THAT USE OF THIS PROGRAM DOES NOT INFRINGE THE
% INTELLECTUAL PROPERTY RIGHTS OF ANY THIRD PARTY IN ANY COUNTRY.
%
% Copyright (c) 1994-2006, John Conover, All Rights Reserved.
%
% Comments and/or bug reports should be addressed to:
%
%     john@email.johncon.com (John Conover)
%
% -----------------------------------------------------------------------------
%
% Revision: \RCSRevision \\
% Revision Time: \RCSTime UMT \\
% Revision Date: \RCSDate \\
% Revision Id: \RCSId \\
% Revision File: \RCSLog \\
\RCS $Revision: 0.0 $
\RCS $Date: 2006/01/20 04:38:13 $
\RCS $Id: maximum.tex,v 0.0 2006/01/20 04:38:13 john Exp $
% $Log: maximum.tex,v $
% Revision 0.0  2006/01/20 04:38:13  john
% Initial version
%
%
    \subsection{Simulation of Fixed Increment Approximation for Optimally Maximal Fiscal Strategy}
        \label{\SETLABEL:MAXSHANNON}
        \subidx{\market}{fiscal strategy, simulation}
        \subidx{\market}{maximum Shannon probability}
        \subidx{markets}{simulation}
        \subidx{simulation}{markets}
        \subidx{strategy}{optimum fiscal, simulation}
        \subidx{fiscal}{optimum strategy, simulation}
        \subidx{programs}{tsunfairbrownian}
        \subidx{tsunfairbrownian}{program}
        \subidx{Shannon}{probability}
        \subidx{probability}{Shannon}

        \subidx{strategy}{exploitable}
        \subidx{exploitable}{strategy}
        \subidx{programs}{tsshannonmax}
        \subidx{tsshannonmax}{program}
        \subidx{programs}{tsunfairbrownian}
        \subidx{tsunfairbrownian}{program}
        \subidx{strategy}{fiscal}
        \subidx{fiscal}{strategy}
        The data in this section is presented in tabular form in
        Section~\ref{\SETLABELREF:MAXSHANNON}. One of the issues of
        analysis, as mentioned in Section~\ref{\SETLABEL:OPS}, is to
        determine the maximum Shannon probability for the time series
        presented in Figure~\ref{\SETLABEL:TS}. Potentially, this
        could be exploited with an aggressive fiscal
        strategy. Figure~\ref{\SETLABEL:SHANNONMAX0} is a graph of the
        output of the {\it tsshannonmax}\/ program, which is described
        briefly in appendix~\ref{programs}. The maximum of this
        function is the maximum Shannon probability for the time
        series data presented in Figure~\ref{\SETLABEL:TS}.
        Figure~\ref{\SETLABEL:SHANNONMAX1} was constructed using {\it
        tsunfairbrownian}\/ program, which is also described in
        appendix~\ref{programs}, with the maximum Shannon probability,
        and the time series data presented in
        Figure~\ref{\SETLABEL:TS}. This represents a ``what if'' the
        investment strategy was changed from a Shannon probability of
        {\shannonlogreturns}, as derived in Section~\ref{\SETLABEL:SP}
        to {\shannonmax}. This process, essentially, extracts the
        random statistical data from the time series presented in
        Figure~\ref{\SETLABEL:TS}, and constructs a new time series,
        using the random statistical data, with a different investment
        strategy.  The program, {\it tsunfairbrownian}\/, essentially,
        constructs the new time series as a Brownian fractal with
        fixed increments.  The ``quality'' of such a reconstruction
        should be subject to adequate scepticism and scrutiny since,
        in all probability, the increments in the original data
        represent a relatively complex process, that may not be
        ``modeled'' with such a simple methodology.

        \begin{figure}[ht]
            \begin{center}
                \begin{minipage}[t]{0.45\textwidth}
                    \epsfxsize=1.0\linewidth
                    \epsffile{\directory/data.tsshannonmax.eps}
                    \caption[{\market}, maximum rate of revenue
                        returns] {{\market}, maximum rate of revenue
                        returns, per {\timescale}, vs. Shannon
                        probability. The maximum rate of revenue
                        returns, per {\timescale}, occurs at a Shannon
                        probability of {\shannonmax}.}
                    \label{\SETLABEL:SHANNONMAX0}
                \end{minipage}
                \hfill
                \begin{minipage}[t]{0.45\textwidth}
                    \epsfxsize=1.0\linewidth
                    \epsffile{\directory/data.tsshannonmax-p.tsunfairbrownian-p.eps}
                    \caption[{\market}, maximum rate of revenue
                        returns] {{\market}, maximum rate of revenue
                        returns, per {\timescale}, at a Shannon
                        probability, of {\shannonmax}, corresponding
                        to a ``wager'' fraction of {\twoponemax}.}
                    \label{\SETLABEL:SHANNONMAX1}
                \end{minipage}
            \end{center}
        \end{figure}

        \subidx{fractional}{Brownian motion}
        \subidx{Brownian motion}{fractional}
        \subidx{Shannon}{probability}
        \subidx{probability}{Shannon}
        \subidx{programs}{tsshannonmax}
        \subidx{tsshannonmax}{program}
        If it is assumed that the time series data set, presented in
        Figure~\ref{\SETLABEL:TS}, constitutes classical Brownian
        motion, then the Shannon probability can be calculated by
        counting the total number of {\timescale}s that the {\market}
        movement was positive, and dividing by the total number of
        {timescale}s represented in the time series. This quotient is
        {\pmax}, as compared with the predicted value from the program
        {\it tsshannonmax}\/ of {\shannonmax}.

% Local Variables:
% TeX-parse-self: t
% TeX-auto-save: t
% TeX-master: "fractal.tex"
% End:


        \subsubsection{Observations on the Simulation of Fixed Increment Approximation for Optimally Maximal Fiscal Strategy}

            Note that these simulations are base on a very, perhaps
            overly, simplified model. For example, from
            Section~\ref{\SETLABEL:TSA}, Figure~\ref{\SETLABEL:NH}, it
            would appear that the {\market}'s normalized increments
            are characterized by fractional Brownian motion---but the
            simulations used classical Brownian motion as the
            model. One consequence of this is that a re-investment
            strategy that is to ``wager'' a fraction of {\twoponemax}
            of the rate of returns every {\timescale} is overly
            aggressive, since in the classical Brownian scenario, the
            maximum loss, in any {\timescale}, was no more that what
            was ``wagered.'' However, in the fractional Brownian
            scenario, much more can be lost. From
            Equation~\ref{fopt2},

            \begin{equation}
                \frac{avg}{rms^2} = \frac{f_{opt}}{rms} = K
            \end{equation}

            \noindent where, under the optimum classical Brownian
            scenario, $K$ is unity, or $avg = rms^2$. Notice that,
            since $f = rms$, whether the scenario is optimal or not,
            that the operational ``wager'' fraction, from
            Figure~\ref{\SETLABEL:TF} of {\datafractionrms}, vs.\ an
            ``theoretical optimal'' value of {\twoponemax} seems
            overly conservative. Additionally, notice that, at least
            in principle, the chance of failure in the fractional
            Brownian scenario, which is more accurate, would
            correspond to 1 standard deviation, or about 15.865\% per
            {\timescale}, which is unacceptably high. However, it is
            not clear why the {\market} is running at a value of
            {\datafractionrms}, which seems very
            conservative. However, a re-investment strategy of
            {\datafractionrms} per {\timescale} does not seem
            inconsistent with a failure rate, on the Fortune 500 list,
            which it is inferred that the {\market} is similar to, of
            about 50\% in ten years, which corresponds to $(1 -
            p_f)^{120} \approx 0.5$, or $p_f$, the probability of
            failure, is $0.005759576$, which is, approximately, 2.5
            standard deviations, meaning that to be consistent with
            the large companies in the Fortune 500, the re-investment
            rate should be, approximately, $\frac{\twoponemax}{2.5}$,
            compared with an operational value, from
            Figure~\ref{\SETLABEL:NH} of {\datafractionrms}.

            An interesting, and intriguing, interpretation and
            discussion of the maximum Shannon probability, is an
            explanation as to why the companies in the {\market} are
            not running near the optimal re-investment strategy. This
            seems enigmatic, since those companies that run, on a long
            term average, far below the optimally maximal value would
            seem to be eclipsed by those that didn't. And those that
            run too close, or even above, the optimally maximal value
            would be over extended, and become financially destitute
            during market down turns, which is inevitable in a fractal
            time series as presented in Figure~\ref{\SETLABEL:TS}.  It
            would seem that the natural selection process of the
            competitive environment would allow only those companies
            that run sufficiently near the optimally maximal value to
            survive, in the long run. One possible explanation,
            foremost, is that the analytical methodology presented
            herein is inappropriate.  Another explanation is that the
            gross margins are less than the fraction {\shannonmax} of
            the rate of revenue returns, and thus could not
            accommodate such an aggressive re-investment strategy. If
            this is the case, then it presents an intriguing
            issue. If, in a capitalistic market, the natural outcome
            of the competitive situation, according to game-theoretic
            analysis, is that there will be many competitors, each
            making minimal gross margins, then how do the companies
            grow their markets?  Naturally, those that run the most
            efficient will have lower costs, making larger percentage
            of rate of revenue returns re-investment possible. Yet
            another interpretation is that the number of competitors
            would grow at an exponential rate, but all of them would
            make minimal returns. However, an operational Shannon
            probability of {\shannonlogreturns} is not just marginally
            lower than the maximum Shannon probability of
            {\shannonmax}. There is a significant disparity which is
            difficult to explain. It would seem that the
            game-theoretic eventual outcome of a competitive market
            place would be a solution that hinders growth, wealth and
            jobs creation, etc., which does not seem consistent with
            capitalistic theory. On the other hand, is there an
            optimum number of competitors in a market place, where the
            gross margins can be higher, permitting wealth and job
            creation, and also a competitive situation? If this
            analysis is correct, and that should be subject to
            scrutiny, then it would appear that this is the case. But
            this brings up another issue---that of taxation, and other
            contributions to the social welfare function. If there is
            an optimum number of competitors in the market place, that
            maximizes wealth and job creation, then, perhaps by lemma,
            there is also an optimal value of taxation rate, and other
            contributions to the social welfare function, that will
            permit maximal industrial growth, and thus maximal growth
            in the tax base. But this would seem to be inconsistent
            with the work of Kenneth Arrow and the so called
            Impossibility Theorem, which states that such
            optimizations can not be determined because the ordering
            of priorities are intransitive.  All very perplexing,
            since the simulation of the maximum Shannon probability in
            the next section seems to indicate that such an aggressive
            re-investment strategy is, indeed, feasible.

            Yet another possibility for the industry not running at
            maximum Shannon probability is the high cost of expansion
            of operations. Some of these industries require very
            sophisticated manufacturing processes, which have high
            barrier costs.

            Additionally, as mentioned in both~\cite[pp. 29]{Brock},
            and~\cite[pp. 8]{Arthur:CTIRALIBHE}, optimal efficiency
            may not be attainable in increasing-return economic
            scenarios.

        %
% -----------------------------------------------------------------------------
%
% A license is hereby granted to reproduce this software source code and
% to create executable versions from this source code for personal,
% non-commercial use.  The copyright notice included with the software
% must be maintained in all copies produced.
%
% THIS PROGRAM IS PROVIDED "AS IS". THE AUTHOR PROVIDES NO WARRANTIES
% WHATSOEVER, EXPRESSED OR IMPLIED, INCLUDING WARRANTIES OF
% MERCHANTABILITY, TITLE, OR FITNESS FOR ANY PARTICULAR PURPOSE.  THE
% AUTHOR DOES NOT WARRANT THAT USE OF THIS PROGRAM DOES NOT INFRINGE THE
% INTELLECTUAL PROPERTY RIGHTS OF ANY THIRD PARTY IN ANY COUNTRY.
%
% Copyright (c) 1994-2006, John Conover, All Rights Reserved.
%
% Comments and/or bug reports should be addressed to:
%
%     john@email.johncon.com (John Conover)
%
% -----------------------------------------------------------------------------
%
% Revision: \RCSRevision \\
% Revision Time: \RCSTime UMT \\
% Revision Date: \RCSDate \\
% Revision Id: \RCSId \\
% Revision File: \RCSLog \\
\RCS $Revision: 0.0 $
\RCS $Date: 2006/01/20 04:38:13 $
\RCS $Id: verification.tex,v 0.0 2006/01/20 04:38:13 john Exp $
% $Log: verification.tex,v $
% Revision 0.0  2006/01/20 04:38:13  john
% Initial version
%
%
    \subsection{Qualitative Verification of Fixed Increment Approximation Analysis}
        \label{\SETLABEL:QVA}

        \subidx{\market}{verification of analysis}
        \subidx{verification}{analysis}
        \subidx{analysis}{verification}
        \subidx{quality}{of analysis}
        \subidx{verification}{of methodology}
        \subidx{methodology}{verification of}
        \subidx{Shannon}{probability}
        \subidx{probability}{Shannon}

        This section evaluates various values based on the ``average''
        of the normalized increments presented in
        Figure~\ref{\SETLABEL:TFA}. These values are an approximation
        to a, probably, complex process with a distribution shown in
        Figure~\ref{\SETLABEL:TF}. These values will be used in a
        fixed increment Brownian fractal analysis of the {\market},
        and may, or may not, provide adequate accuracy for
        projections.

        The data in this section is presented in tabular form in
        sections~\ref{\SETLABELREF:VI1} and~\ref{\SETLABELREF:VI2}.
        As a subjective evaluation of the ``quality'' of the analysis
        of the {\market}, from Chapter~\ref{methodology},
        Equation~\ref{metricvalues1}, and using the mean and root mean
        square values of the normalized increments of the time series
        data presented in Figure~\ref{\SETLABEL:TS} from
        Figure~\ref{\SETLABEL:TF}, and the Shannon probability as
        calculated by counting the total number of {\timescale}s that
        the {\market} movement was positive, as presented in
        Section~\ref{\SETLABEL:MAXSHANNON}:

        \begin{eqnarray}
                  P & \approx & \frac{\frac{avg}{rms} + 1}{2}\\
            {\pmax} & \approx & \frac{\frac{\datafractionmean}{\datafractionrms} + 1}{2}\\
            {\pmax} & \approx & {\avgrms}
            \label{\SETLABEL:AVGS}
        \end{eqnarray}

        \subidx{Shannon}{probability}
        \subidx{probability}{Shannon}
        \noindent and comparing these values to the Shannon
        probability, as found by the {\it tsshannonmax}\/ program, which
        iterates for a maximum:

        \begin{eqnarray}
            {\pmax} \approx {\avgrms} \approx {\shannonmax}
        \end{eqnarray}

        \subidx{logarithmic}{returns}
        \subidx{returns}{logarithmic}
        In addition, the different methods of calculating the
        logarithmic returns, presented in Section~\ref{\SETLABEL:FS},
        should be compared. The four methods used were the mean of
        Figure~\ref{\SETLABEL:TF}, the constant in the least squares
        approximation to Figure~\ref{\SETLABEL:TF}, the least squares
        exponential approximation to Figure~\ref{\SETLABEL:TS}, and
        the logarithmic returns of Figure~\ref{\SETLABEL:TS}, derived
        as the mean of the logarithms of the quotients of the
        increments. The values for each of the methods are,
        respectively:

        \begin{equation}
            \datafractionmeanbits \approx \datafractionconstantbits \approx \datatslsqepbits \approx \logreturns
        \end{equation}

        It is implied in Section~\ref{\SETLABEL:FS},
        Subsection~\ref{\SETLABEL:SP} and in
        Section~\ref{\SETLABEL:TSUNFAIRBROWNIAN} that, a Brownian
        motion with fixed increments fractal may ``model'' the
        {\market}. Using Equation~\ref{stddev9} from
        Chapter~\ref{general}, Section~\ref{abmfi}:

        \begin{eqnarray}
                                    rms \left(2P - 1\right) & \approx & \frac{\sigma \left(2P - 1\right)}{2 \sqrt{P\left(1 - P\right)}}\\
            \datafractionrms \left(2 \cdot \pmax - 1\right) & \approx & \frac{\datafractionstddev \left(2 \cdot \pmax - 1\right)}{2\sqrt{\pmax \left(1 - \pmax\right)}}\\
                       \datafractionrms \cdot \twopminusone & \approx & \datafractionstddev \cdot \twopx\\
                                                      \rmsp & \approx & \sigmap
        \end{eqnarray}

        \noindent and, equating to the mean:

        \begin{equation}
            \datafractionmean \approx \rmsp \approx \sigmap
        \end{equation}

        \subidx{Shannon}{probability}
        \subidx{probability}{Shannon}
        \noindent where, as in Equation~\ref{\SETLABEL:AVGS} using the
        mean, root mean square, and standard deviation values of the
        normalized increments of the time series data presented in
        Figure~\ref{\SETLABEL:TS} from Figure~\ref{\SETLABEL:TF}, and
        the Shannon probability as calculated by counting the total
        number of {\timescale}s that the {\market} movement was
        positive, as presented in Section~\ref{\SETLABEL:MAXSHANNON}.

        As a final qualitative comparison, the absolute value of the
        normalized increments should be the same as the root mean
        square value\footnote{The absolute value of the normalized
        increments, when averaged, is related to the root mean square
        of the increments by a constant. If the normalized increments
        are a fixed increment, the constant is unity. If the
        normalized increments have a Gaussian distribution, the
        constant is $\approx 0.8$ depending on the accuracy of of
        ``fit'' to a Gaussian distribution.}, where the absolute value
        is presented in Figure~\ref{\SETLABEL:TFA}, and the root mean
        square value is presented in Figure~\ref{\SETLABEL:TF}:

        \begin{equation}
            \datafractionabsmean \approx \datafractionrms
        \end{equation}

        Note, that if the {\market} could be ``modeled'' as a Brownian
        motion with fixed increments fractal, then the standard
        deviation of the absolute value of the normalized increments
        of the time series data presented in Figure~\ref{\SETLABEL:TS}
        from Figure~\ref{\SETLABEL:TF} should be zero. It is
        $\datafractionabsstddev$.

% Local Variables:
% TeX-parse-self: t
% TeX-auto-save: t
% TeX-master: "fractal.tex"
% End:


    \renewcommand{\market}{World Semiconductor Market}
    \renewcommand{\directory}{../markets/semiconductors.world}
    \renewcommand{\datafractionmean}{0.008052}
\renewcommand{\datafractionmeanbits}{0.011570}
\renewcommand{\datafractionmeanq}{0.002684}
\renewcommand{\datafractionmeanbitsq}{0.003867}
\renewcommand{\datafractionstddev}{0.038579}
\renewcommand{\datafractionrms}{0.039311}
\renewcommand{\avgrms}{0.602414}
\renewcommand{\ncompanies}{5.210454}
\renewcommand{\pncompanies}{0.544866}
\renewcommand{\datafractionabsmean}{0.029745}
\renewcommand{\datafractionabsstddev}{0.025769}
\renewcommand{\datafractionconstant}{0.010041}
\renewcommand{\datafractionconstantbits}{0.014414}
\renewcommand{\datafractionconstantq}{0.003347}
\renewcommand{\datafractionconstantbitsq}{0.004821}
\renewcommand{\datafractionslope}{-0.000021}
\renewcommand{\datafractionabsconstant}{0.035145}
\renewcommand{\datafractionabsslope}{-0.000057}
\renewcommand{\hurstall}{0.659558}
\renewcommand{\hurstlow}{0.707509}
\renewcommand{\hurstlowtwo}{1.415018}
\renewcommand{\hurstlowhundred}{70.750900}
\renewcommand{\hcalcall}{0.184942}
\renewcommand{\hcalclow}{0.102042}
\renewcommand{\shannonmax}{0.604167}
\renewcommand{\twoponemax}{0.208334}
\renewcommand{\logreturns}{0.010456}
\renewcommand{\twologreturns}{1.007274}
\renewcommand{\twologreturnshundred}{0.727387}
\renewcommand{\oneoverlogreturns}{95.638868}
\renewcommand{\pmax}{0.602094}
\renewcommand{\twopminusone}{0.204188}
\renewcommand{\rmsp}{0.008027}
\renewcommand{\twopx}{0.208583}
\renewcommand{\sigmap}{0.008047}
\renewcommand{\tsunfairbrownianfractionmean}{0.007862}
\renewcommand{\tsunfairbrownianfractionstddev}{0.038619}
\renewcommand{\shannonlogreturns}{0.560125}
\renewcommand{\shannonlogreturnshundred}{56.012500}
\renewcommand{\twopone}{0.120250}
\renewcommand{\twoponehundred}{12.025000}
\renewcommand{\hundredtwoponehundred}{87.975000}
\renewcommand{\hundredshannonlogreturnshundred}{43.987500}
\renewcommand{\datatslsqepbits}{0.007623}
\renewcommand{\thurstall}{0.633980}
\renewcommand{\thurstlow}{0.710108}
\renewcommand{\thurstlowtwo}{1.420216}
\renewcommand{\thurstlowhundred}{71.010800}
\renewcommand{\thcalcall}{0.247886}
\renewcommand{\thcalclow}{0.171737}
\renewcommand{\chisquared}{2.862000}
\renewcommand{\critical}{42.557000}

    \renewcommand{\timescale}{quarter}
    \subidx{market}{\market}
    \idx{\market}

    \section{\market}

        \renewcommand{\SETLABEL}{\LABPRE:WSM}
        \renewcommand{\SETLABELQ}{\LABPRE:WSMQ}
        \label{\SETLABEL}
        \renewcommand{\SETLABELREF}{\LABPREREF:WSM}

        \idx{Semiconductor Industry Association}
        For the analysis, the data was in the directory
        {\directory}\footnote{Data from the Semiconductor Industry
        Association, 1982---1994, by {\timescale}s, in millions of
        dollars, US.}.

        The data in this section is presented in tabular form in
        Section~\ref{\SETLABELREF}.

        %
% -----------------------------------------------------------------------------
%
% A license is hereby granted to reproduce this software source code and
% to create executable versions from this source code for personal,
% non-commercial use.  The copyright notice included with the software
% must be maintained in all copies produced.
%
% THIS PROGRAM IS PROVIDED "AS IS". THE AUTHOR PROVIDES NO WARRANTIES
% WHATSOEVER, EXPRESSED OR IMPLIED, INCLUDING WARRANTIES OF
% MERCHANTABILITY, TITLE, OR FITNESS FOR ANY PARTICULAR PURPOSE.  THE
% AUTHOR DOES NOT WARRANT THAT USE OF THIS PROGRAM DOES NOT INFRINGE THE
% INTELLECTUAL PROPERTY RIGHTS OF ANY THIRD PARTY IN ANY COUNTRY.
%
% Copyright (c) 1994-2006, John Conover, All Rights Reserved.
%
% Comments and/or bug reports should be addressed to:
%
%     john@email.johncon.com (John Conover)
%
% -----------------------------------------------------------------------------
%
% Revision: \RCSRevision \\
% Revision Time: \RCSTime UMT \\
% Revision Date: \RCSDate \\
% Revision Id: \RCSId \\
% Revision File: \RCSLog \\
\RCS $Revision: 0.0 $
\RCS $Date: 2006/01/20 04:38:13 $
\RCS $Id: fraction.tex,v 0.0 2006/01/20 04:38:13 john Exp $
% $Log: fraction.tex,v $
% Revision 0.0  2006/01/20 04:38:13  john
% Initial version
%
%
    \subsection{Time Series Increments Analysis}
        \label{\SETLABEL:TSA}

        \subidx{\market}{Time series analysis}
        \subidx{time series}{increments}
        \subidx{time series}{analysis}
        \subidx{cumulative sum}{analysis}
        \subidx{analysis}{cumulative sum}
        \subidx{analysis}{random process}
        \subidx{random process}{analysis}
        \subidx{Gaussian}{increments}
        \subidx{increments}{Gaussian}
        \subidx{Brownian}{motion, fractional}
        \subidx{fractional}{Brownian motion}
        \subidx{fractal}{Brownian motion}
        The data in this section is presented in tabular form in
        Section~\ref{\SETLABELREF:TSA}.  Figure~\ref{\SETLABEL:TS} is
        a graph of the time series data for the {\market}.

        \subidx{increments}{normalized}
        \subidx{normalized}{increments}
        \subidx{programs}{tsfraction}
        \subidx{tsfraction}{program}
        Figure~\ref{\SETLABEL:TF} is a graph of the normalized
        increments of the time series data presented in
        Figure~\ref{\SETLABEL:TS}. The data presented was made by
        running the program {\it tsfraction}\/ on the time series
        data. The program {\it tsfraction}\/ is described briefly in
        Appendix~\ref{programs}, and subtracts the previous value from
        the next value, dividing this difference by the previous
        value, for each element in the time series data. The new time
        series contains the instantaneous change in the rate of
        revenue returns, divided by the magnitude of the instantaneous
        rate of revenue returns.

        \subidx{mean}{standard deviation}
        \subidx{standard deviation}{mean}
        \idx{root mean square}
        \idx{least squares approximation}
        \begin{figure}[ht]
            \begin{center}
                \begin{minipage}[t]{0.45\textwidth}
                    \epsfxsize=1.0\linewidth
                    \epsffile{\directory/data.eps}
                    \caption{{\market}, time series data.}
                    \label{\SETLABEL:TS}
                    \label{\SETLABELQ:TS}
                \end{minipage}
                \hfill
                \begin{minipage}[t]{0.45\textwidth}
                    \epsfxsize=1.0\linewidth
                    \epsffile{\directory/data.tsfraction.eps}
                    \caption[{\market}, normalized
                        increments]{{\market}, normalized increments
                        of the time series data presented in
                        Figure~\ref{\SETLABEL:TS}. The mean is
                        {\datafractionmean} with a standard deviation
                        of {\datafractionstddev}. The formula for the
                        least squares approximation is
                        ${\datafractionconstant} +
                        {\datafractionslope}t$, and the root mean
                        squared value is {\datafractionrms}. The
                        graph, labeled ``data\-.tsfraction\-.tsrms,''
                        is the running root mean square, and
                        ``data\-.tsfraction\-.tsavg'' is the running
                        average of the normalized increments.  This
                        graph is the fraction of change in the time
                        series, as a function of time. Note that the
                        slope of the mean, {\datafractionslope}, is
                        the coefficient of the nonlinearity term in
                        the normalized increments. See
                        Chapter~\ref{general}, Section~\ref{nlextend}
                        for a possible application of the logistic
                        function to this data set.}
                    \label{\SETLABEL:TF}
                    \label{\SETLABELQ:TF}
                \end{minipage}
            \end{center}
        \end{figure}

        \subidx{absolute value}{increments}
        \subidx{increments}{absolute value}

        Figure~\ref{\SETLABEL:TFA} is a graph of the absolute value of
        the normalized increments of the time series data presented in
        Figure~\ref{\SETLABEL:TF}. The data presented was made by
        running the Unix utility sed(1) on the normalized increments
        time series data to remove the negative signs. This is an
        absolute value procedure.  The resulting time series contains
        the absolute value of the instantaneous change in the rate of
        revenue returns, divided by the magnitude of the instantaneous
        rate of revenue returns\footnote{The absolute value of the
        normalized increments, when averaged, is related to the root
        mean square of the increments by a constant. If the normalized
        increments are a fixed increment, the constant is unity. If
        the normalized increments have a Gaussian distribution, the
        constant is $\approx 0.8$ depending on the accuracy of of
        ``fit'' to a Gaussian distribution.}.

        \subidx{histogram}{normalized}
        \subidx{normalized}{histogram}
        \subidx{programs}{tsnormal}
        \subidx{tsnormal}{program}
        \subidx{mean}{standard deviation}
        \subidx{standard deviation}{mean}
        \idx{root mean square}
        \idx{least squares approximation}
        \subidx{\market}{analysis of increments}
        Figure~\ref{\SETLABEL:NH} is the normalized histogram of the
        normalized increments of the time series data shown in
        Figure~\ref{\SETLABEL:TF}. The abscissa is 3 $\sigma$ limits,
        and the area under the two curves is identical. The data for
        this figure was produced by the program {\it tsnormal}\/,
        which is described briefly in Appendix~\ref{programs}.

        \begin{figure}[ht]
            \begin{center}
                \begin{minipage}[t]{0.45\textwidth}
                    \epsfxsize=1.0\linewidth
                    \epsffile{\directory/data.tsfraction.abs.eps}
                    \caption[{\market}, absolute value of the
                        normalized increments]{{\market}, absolute
                        value of the normalized increments of the time
                        series data presented in
                        Figure~\ref{\SETLABEL:TF}.  The mean is
                        {\datafractionabsmean} with a standard
                        deviation of {\datafractionabsstddev}. The
                        formula for the least squares approximation is
                        ${\datafractionabsconstant} +
                        {\datafractionabsslope}t$, and the root mean
                        square value, from Figure~\ref{\SETLABEL:TF},
                        is {\datafractionrms}.  The graph, labeled
                        ``data\-.tsfraction\-.tsrms,'' is the running
                        root mean square, and
                        ``data\-.tsfraction\-.tsavg'' is the running
                        average of the normalized increments presented
                        in Figure~\ref{\SETLABEL:TF}, superimposed
                        here for convenience. This graph is the
                        absolute value of the fraction of change in
                        the time series, as a function of time.}
                    \label{\SETLABEL:TFA}
                    \label{\SETLABELQ:TFA}
                \end{minipage}
                \hfill
                \begin{minipage}[t]{0.45\textwidth}
                    \epsfxsize=1.0\linewidth
                    \epsffile{\directory/data.tsfraction.tsnormal-s30.eps}
                    \caption[{\market}, normalized histogram of the
                        normalized increments]{{\market}, normalized
                        histogram of the normalized increments of the
                        time series data shown in
                        Figure~\ref{\SETLABEL:TF}.  The data has a
                        mean of {\datafractionmean}, with a standard
                        deviation of {\datafractionstddev}.  The area
                        under the two curves is identical. The
                        $\chi^2$ value of the observed and expected
                        values of the two curves is {\chisquared},
                        with a critical value of {\critical}.}
                    \label{\SETLABEL:NH}
                \end{minipage}
            \end{center}
        \end{figure}

        \subidx{programs}{tsXsquared}
        \subidx{tsXsquared}{program}
        \subidx{\market}{chi-squared values of increments}
        The program {\it tsXsquared}\/, which is briefly described in
        appendix~\ref{programs}, was used to derive the $\chi^2$
        statistics for the data presented in
        Figure~\ref{\SETLABEL:NH}.

        \subidx{programs}{tsstatest}
        \subidx{tsstatest}{program}
        \subidx{\market}{statistical estimates}

        Figure~\ref{\SETLABEL:SE} is the statistical estimate for the
        data presented in Figure~\ref{\SETLABEL:TF}, as derived by the
        program {\it tsstatest}\/, which is briefly described in
        appendix~\ref{programs}.

        \begin{figure}[ht]
            \begin{center}
                \begin{minipage}[t]{\textwidth}
                    \center{\fbox{\parbox{0.9\textwidth}{\XXX{\directory/data.tsstatest-f0.1-c0.9-i.tex}}}}
                    \caption[{\market}, statistical estimates of the
                        normalized increments]{{\market}, statistical
                        estimates of the normalized increments of the
                        time series shown in Figure~\ref{\SETLABEL:TF}.
                        The table was produced with the {\it
                        tsstatest}\/ program, and illustrates the
                        size of the data set required for a confidence
                        level of 90\%, with an error estimate of $\pm$
                        10\%, or alternately, the error estimate on
                        the time series shown in Figure~\ref{\SETLABEL:TF}.}
                    \label{\SETLABEL:SE}
                \end{minipage}
            \end{center}
        \end{figure}

        Note that the data set size estimations, as produced by the
        {\it tsstatest}\/ program, are probably very conservative,
        depending on the magnitude of the Shannon probability, $P =
        \shannonlogreturns$, as derived in
        Section~\ref{\SETLABEL:SP}. See Chapter~\ref{general},
        Section~\ref{serdss} for possible alternative methodologies
        for addressing the analysis of fractal time series with
        limited data set sizes. Depending on the magnitude of the
        Shannon probability, $P$, these estimates can be several
        orders of magnitude too high.

        \subidx{derivative of increments}{normalized}
        \subidx{normalized}{derivative of increments}
        \subidx{programs}{tsderivative}
        \subidx{tsderivative}{program}
        Figure~\ref{\SETLABEL:TF1} is the normalized histogram of the
        first derivative of the normalized increments of the time
        series data shown in Figure~\ref{\SETLABEL:TF}. In principle,
        if the distribution of the normalized increments presented in
        Figure~\ref{\SETLABEL:NH} is Gaussian in nature, this
        distribution would be similar to ``white noise,'' as presented
        in appendix~\ref{programs}, Figure~\ref{whiteexample}. The
        data was generated by the {\it tsderivative}\/ program, which
        is briefly described in
        appendix~\ref{programs}. Figure~\ref{\SETLABEL:TF2} is the
        normalized histogram of the second derivative of the
        normalized increments of the time series data shown in
        Figure~\ref{\SETLABEL:TF}. In principle, if the distribution
        of the normalized increments presented in
        Figure~\ref{\SETLABEL:NH} is an integrated Gaussian
        distribution in nature, this distribution would be similar to
        ``white noise,'' as presented in appendix~\ref{programs},
        Figure~\ref{whiteexample}.

        \begin{figure}[ht]
            \begin{center}
                \begin{minipage}[t]{0.45\textwidth}
                    \epsfxsize=1.0\linewidth
                    \epsffile{\directory/data.tsfraction.tsderivative.tsnormal-s30.eps}
                    \caption[{\market}, histogram of the first
                        derivative of the increments]{{\market},
                        normalized histogram of the first derivative
                        of the normalized increments of the time
                        series data shown in
                        Figure~\ref{\SETLABEL:TF}.}
                    \label{\SETLABEL:TF1}
                \end{minipage}
                \hfill
                \begin{minipage}[t]{0.45\textwidth}
                    \epsfxsize=1.0\linewidth
                    \epsffile{\directory/data.tsfraction.2tsderivative.tsnormal-s30.eps}
                    \caption[{\market}, histogram of the second
                        derivative of the increments]{{\market},
                        normalized histogram of second derivative of
                        the the normalized increments of the time
                        series data shown in
                        Figure~\ref{\SETLABEL:TF}.}
                    \label{\SETLABEL:TF2}
                \end{minipage}
            \end{center}
        \end{figure}

        \subidx{fractal}{range}
        \subidx{fractal}{R/S analysis}
        \subidx{\market}{rate of revenue returns, range}
        \subidx{\market}{deterministic mechanism}
        \subidx{deterministic}{mechanism}
        \subidx{mechanism}{deterministic}
        Figure~\ref{\SETLABEL:TR} is the range of values of the time
        series shown in Figure~\ref{\SETLABEL:TS}. The horizontal axis
        is time into the future. In principle, if the time series was
        characterized as fractional Brownian motion the graph in
        Figure~\ref{\SETLABEL:TR} would be a square root
        function\footnote{Note that the ``roughness,'' or ``sawtooth''
        characteristics of the graph in Figure~\ref{\SETLABEL:TR} are
        a computational artifact---caused by not using the -m option
        to the program {\it tshurst}\/, which is computationally
        inefficient.}. Figure~\ref{\SETLABEL:TD} is the deterministic
        map of the normalized increments of the time series data shown
        in Figure~\ref{\SETLABEL:TF}. The deterministic map is useful
        for determining if a time series was created by a
        deterministic mechanism. This, essentially, maps each element
        in the time series with the previous element in the time
        series.  See,~\cite[pp. 745]{Peitgen}.

        \begin{figure}[ht]
            \begin{center}
                \begin{minipage}[t]{0.45\textwidth}
                    \epsfxsize=1.0\linewidth
                    \epsffile{\directory/data.tshurst-f.eps}
                    \caption[{\market}, range]{{\market}, range of the
                        time series data shown in
                        Figure~\ref{\SETLABEL:TS}.}
                    \label{\SETLABEL:TR}
                \end{minipage}
                \hfill
                \begin{minipage}[t]{0.45\textwidth}
                    \epsfxsize=1.0\linewidth
                    \epsffile{\directory/data.tsfraction.tsdeterministic.eps}
                    \caption[{\market}, deterministic map]{{\market},
                        deterministic map of the normalized increments
                        of the time series data shown in
                        Figure~\ref{\SETLABEL:TF}.}
                    \label{\SETLABEL:TD}
                \end{minipage}
            \end{center}
        \end{figure}

% Local Variables:
% TeX-parse-self: t
% TeX-auto-save: t
% TeX-master: "fractal.tex"
% End:


        \subsubsection{Observations on the Time Series Increments Analysis}

            Figure~\ref{\SETLABEL:NH} would seem to indicate that the
            time series data for the {\market} represents a cumulative
            sum/integration of a random process that has a Gaussian
            distribution, (ie., satisfies the Gaussian increments
            property of fractional Brownian
            motion~\cite[pp. 250]{Crownover},) tending to justify the
            assumption that the time series data represents fractional
            Brownian motion.

        %
% -----------------------------------------------------------------------------
%
% A license is hereby granted to reproduce this software source code and
% to create executable versions from this source code for personal,
% non-commercial use.  The copyright notice included with the software
% must be maintained in all copies produced.
%
% THIS PROGRAM IS PROVIDED "AS IS". THE AUTHOR PROVIDES NO WARRANTIES
% WHATSOEVER, EXPRESSED OR IMPLIED, INCLUDING WARRANTIES OF
% MERCHANTABILITY, TITLE, OR FITNESS FOR ANY PARTICULAR PURPOSE.  THE
% AUTHOR DOES NOT WARRANT THAT USE OF THIS PROGRAM DOES NOT INFRINGE THE
% INTELLECTUAL PROPERTY RIGHTS OF ANY THIRD PARTY IN ANY COUNTRY.
%
% Copyright (c) 1994-2006, John Conover, All Rights Reserved.
%
% Comments and/or bug reports should be addressed to:
%
%     john@email.johncon.com (John Conover)
%
% -----------------------------------------------------------------------------
%
% Revision: \RCSRevision \\
% Revision Time: \RCSTime UMT \\
% Revision Date: \RCSDate \\
% Revision Id: \RCSId \\
% Revision File: \RCSLog \\
\RCS $Revision: 0.0 $
\RCS $Date: 2006/01/20 04:38:13 $
\RCS $Id: instant.tex,v 0.0 2006/01/20 04:38:13 john Exp $
% $Log: instant.tex,v $
% Revision 0.0  2006/01/20 04:38:13  john
% Initial version
%
%
    \subsection{Instantaneous Analysis of Normalized Increments}
        \label{\SETLABEL:IA}

        \subidx{\market}{instantaneous analysis of normalized increments}
        \idx{average of normalized increments}
        \idx{root mean square of normalized increments}
        \subidx{Shannon probability}{instantaneous computation of}
        \subidx{average of normalized increments}{instantaneous computation of}
        \subidx{root mean square of normalized increments}{instantaneous computation of}
        \subidx{instantaneous computation}{Shannon probability}
        \subidx{instantaneous computation}{average of normalized increments}
        \subidx{instantaneous computation}{root mean square of normalized increments}
        \idx{time series}
        \subidx{time series}{instantaneous analysis}
        \subidx{instantaneous analysis}{time series}
        \subidx{time series}{increments}
        \subidx{time series}{analysis}
        \subidx{Shannon}{probability}
        \subidx{probability}{Shannon}
        \subidx{normalized}{increments}
        \subidx{increments}{normalized}

        The program {\it tsinstant}\/, which is briefly described in
        Appendix~\ref{programs}, is for finding the instantaneous
        fraction of change in a time series. The value of a sample in
        the time series is subtracted from the previous sample in the
        time series, and divided by the value of the previous sample.
        As explained in Chapter~\ref{general},
        Sections~\ref{derivation},~\ref{GA},~\ref{abmfi},~\ref{aftsma}
        and,~\ref{ompl} for Brownian motion, random walk fractals, the
        absolute value of the instantaneous fraction of change is also
        the root mean square of the instantaneous fraction of
        change\footnote{The absolute value of the normalized
        increments, when averaged, is related to the root mean square
        of the increments by a constant. If the normalized increments
        are a fixed increment, the constant is unity. If the
        normalized increments have a Gaussian distribution, the
        constant is $\approx 0.8$ depending on the accuracy of of
        ``fit'' to a Gaussian distribution.}. Squaring this value is
        the average of the instantaneous fraction of change, and
        adding unity to the absolute value of the instantaneous
        fraction of change, and dividing by two, is the Shannon
        probability of the instantaneous fraction of change.

        Figure~\ref{\SETLABEL:IA1} is the instantaneous value of the
        root mean square of the normalized increments for the
        {\market}, and Figure~\ref{\SETLABEL:IA2} is the instantaneous
        Shannon probability for the normalized increments.

        \begin{figure}[ht]
            \begin{center}
                \begin{minipage}[t]{0.45\textwidth}
                    \epsfxsize=1.0\linewidth
                    \epsffile{\directory/data.tsinstant-r.eps}
                    \caption[{\market}, instantaneous value of
                        rms.]{{\market}, instantaneous value of the
                        root mean square of the normalized increments,
                        provided by running the program {\it
                        tsinstant}\/ with the -r option on the data
                        presented in Figure~\ref{\SETLABEL:TS}.}
                    \label{\SETLABEL:IA1}
                    \label{\SETLABELQ:IA1}
                \end{minipage}
                \hfill
                \begin{minipage}[t]{0.45\textwidth}
                    \epsfxsize=1.0\linewidth
                    \epsffile{\directory/data.tsinstant-s.eps}
                    \caption[{\market}, instantaneous value of
                        Shannon probability.]{{\market}, instantaneous
                        value of the Shannon probability of the
                        normalized increments, provided by running the
                        program {\it tsinstant}\/ with the -s option
                        on the data presented in
                        Figure~\ref{\SETLABEL:TS}.}
                    \label{\SETLABEL:IA2}
                    \label{\SETLABELQ:IA2}
                \end{minipage}
            \end{center}
        \end{figure}

% Local Variables:
% TeX-parse-self: t
% TeX-auto-save: t
% TeX-master: "fractal.tex"
% End:


        %
% -----------------------------------------------------------------------------
%
% A license is hereby granted to reproduce this software source code and
% to create executable versions from this source code for personal,
% non-commercial use.  The copyright notice included with the software
% must be maintained in all copies produced.
%
% THIS PROGRAM IS PROVIDED "AS IS". THE AUTHOR PROVIDES NO WARRANTIES
% WHATSOEVER, EXPRESSED OR IMPLIED, INCLUDING WARRANTIES OF
% MERCHANTABILITY, TITLE, OR FITNESS FOR ANY PARTICULAR PURPOSE.  THE
% AUTHOR DOES NOT WARRANT THAT USE OF THIS PROGRAM DOES NOT INFRINGE THE
% INTELLECTUAL PROPERTY RIGHTS OF ANY THIRD PARTY IN ANY COUNTRY.
%
% Copyright (c) 1994-2006, John Conover, All Rights Reserved.
%
% Comments and/or bug reports should be addressed to:
%
%     john@email.johncon.com (John Conover)
%
% -----------------------------------------------------------------------------
%
% Revision: \RCSRevision \\
% Revision Time: \RCSTime UMT \\
% Revision Date: \RCSDate \\
% Revision Id: \RCSId \\
% Revision File: \RCSLog \\
\RCS $Revision: 0.0 $
\RCS $Date: 2006/01/20 04:38:13 $
\RCS $Id: logistic.tex,v 0.0 2006/01/20 04:38:13 john Exp $
% $Log: logistic.tex,v $
% Revision 0.0  2006/01/20 04:38:13  john
% Initial version
%
%
    \subsection{Logistic Analysis}
        \label{\SETLABEL:LA}

        \subidx{\market}{Logistic function analysis}
        \subidx{time series}{logistic function}
        \subidx{logistic function}{time series}
        \subidx{time series}{increments}
        \subidx{time series}{analysis}
        \subidx{cumulative sum}{analysis}
        \subidx{analysis}{cumulative sum}
        \subidx{analysis}{random process}
        \subidx{random process}{analysis}
        The data in this section is presented in tabular form in
        Section~\ref{\SETLABELREF:LAA}.  Figure~\ref{\SETLABEL:LA1} is
        a graph of the logistic function estimates of the time series
        data for the {\market}. The reader is cautioned that these
        graphs are constructed using the method suggested in
        Chapter~\ref{general}, Section~\ref{nlextend} and enormous
        precision is required for adequate prediction of the logistic
        function,~\cite{Modis}. Particularly, the non-linear term will
        usually require intervention to produce a practical fit to the
        data. In addition, there are numerical stability issues with
        logistic function methodologies\footnote{For example, in
        Figures~\ref{\SETLABEL:LA1} and~\ref{\SETLABEL:LA2}, if the
        non-linear term, $b$, was greater than zero, it was set to
        zero to produce the graphs. See Section~\ref{\SETLABELREF:LAA}
        for the actual derived values. In other cases, the magnitude
        of $b$ was too large, resulting in a graph that was decreasing
        as a function of time}.  The methodology should be regarded as
        ``fragile.'' It is included for completeness.

        \idx{least squares approximation}
        Figure~\ref{\SETLABEL:LA1} is a graph of the logistic function
        for the time series data presented in
        Figure~\ref{\SETLABEL:TS}. The data presented was made by
        running the program {\it tsdlogistic}\/, which is described
        briefly in Appendix~\ref{programs}, on the parameters
        extracted from the time series data as suggested in
        Figure~\ref{\SETLABEL:TF}. The program {\it tslsq}\/ was used
        to derive the constant and the slope of the normalized
        increments of the data presented in Figure~\ref{\SETLABEL:TF}.
        Figure~\ref{\SETLABEL:LA2} is the same graph, but with the
        time scale expanded by a factor of two.

        \begin{figure}[ht]
            \begin{center}
                \begin{minipage}[t]{0.45\textwidth}
                    \epsfxsize=1.0\linewidth
                    \epsffile{\directory/data.tsfraction.tslsq-p.tsdlogistic.eps}
                    \caption[{\market}, logistic function
                        estimates.]{{\market}, logistic function
                        estimates, provided by running the {\it
                        tslsq}\/ program on the normalized increments
                        presented in Figure~\ref{\SETLABEL:TF} with
                        the -p option. These parameters were used as
                        arguments to the {\it tsdlogistic}\/ program.}
                    \label{\SETLABEL:LA1}
                    \label{\SETLABELQ:LA1}
                \end{minipage}
                \hfill
                \begin{minipage}[t]{0.45\textwidth}
                    \epsfxsize=1.0\linewidth
                    \epsffile{\directory/data.tsfraction.tslsq-p.tsdlogistic2.eps}
                    \caption[{\market}, logistic function
                        estimates.]{{\market}, logistic function
                        estimates of Figure~\ref{\SETLABEL:LA1} with
                        the time scale expanded by a factor of two.}
                    \label{\SETLABEL:LA2}
                    \label{\SETLABELQ:LA2}
                \end{minipage}
            \end{center}
        \end{figure}

% Local Variables:
% TeX-parse-self: t
% TeX-auto-save: t
% TeX-master: "fractal.tex"
% End:


        %
% -----------------------------------------------------------------------------
%
% A license is hereby granted to reproduce this software source code and
% to create executable versions from this source code for personal,
% non-commercial use.  The copyright notice included with the software
% must be maintained in all copies produced.
%
% THIS PROGRAM IS PROVIDED "AS IS". THE AUTHOR PROVIDES NO WARRANTIES
% WHATSOEVER, EXPRESSED OR IMPLIED, INCLUDING WARRANTIES OF
% MERCHANTABILITY, TITLE, OR FITNESS FOR ANY PARTICULAR PURPOSE.  THE
% AUTHOR DOES NOT WARRANT THAT USE OF THIS PROGRAM DOES NOT INFRINGE THE
% INTELLECTUAL PROPERTY RIGHTS OF ANY THIRD PARTY IN ANY COUNTRY.
%
% Copyright (c) 1994-2006, John Conover, All Rights Reserved.
%
% Comments and/or bug reports should be addressed to:
%
%     john@email.johncon.com (John Conover)
%
% -----------------------------------------------------------------------------
%
% Revision: \RCSRevision \\
% Revision Time: \RCSTime UMT \\
% Revision Date: \RCSDate \\
% Revision Id: \RCSId \\
% Revision File: \RCSLog \\
\RCS $Revision: 0.0 $
\RCS $Date: 2006/01/20 04:38:13 $
\RCS $Id: hurst.tex,v 0.0 2006/01/20 04:38:13 john Exp $
% $Log: hurst.tex,v $
% Revision 0.0  2006/01/20 04:38:13  john
% Initial version
%
%
    \subsection{Hurst Coefficient Analysis}
        \label{\SETLABEL:H}

        \subidx{\market}{Hurst coefficient analysis}
        \subidx{Hurst coefficient}{analysis}
        \subidx{increments}{normalized}
        \subidx{normalized}{increments}
        \subidx{programs}{tshurst}
        \subidx{tshurst}{program}
        The data in this section is presented in tabular form in
        Section~\ref{\SETLABELREF:HCHP}. Figure~\ref{\SETLABEL:HC} is
        a graph of the Hurst coefficient data time series data shown
        in Figure~\ref{\SETLABEL:TS}. The slope of the graph is the
        Hurst coefficient.  The data for this figure was produced by
        the program {\it tshurst}\/, which is described briefly in
        Appendix~\ref{programs}.

        \subidx{\market}{H parameter analysis}
        \subidx{H parameter}{analysis}
        \subidx{programs}{tshcalc}
        \subidx{tshcalc}{program}
        Figure~\ref{\SETLABEL:HP} is a graph of the H parameter data
        for the normalized increments of the time series data shown in
        Figure~\ref{\SETLABEL:TF}. The data for this figure was
        produced by the program {\it tshcalc}\/, which is described
        briefly in Appendix~\ref{programs}.

        \begin{figure}[ht]
            \begin{center}
                \begin{minipage}[t]{0.45\textwidth}
                    \epsfxsize=1.0\linewidth
                    \epsffile{\directory/data.tshurst.eps}
                    \caption[{\market}, Hurst coefficient data]{{\market},
                        Hurst coefficient data for the normalized
                        increments of the time series data shown in
                        Figure~\ref{\SETLABEL:TF}.  The slope of the graph
                        is the Hurst coefficient.}
                    \label{\SETLABEL:HC}
                \end{minipage}
                \hfill
                \begin{minipage}[t]{0.45\textwidth}
                    \epsfxsize=1.0\linewidth
                    \epsffile{\directory/data.tshcalc.eps}
                    \caption[{\market}, H parameter data]{{\market}, H
                        parameter data for the normalized increments of
                        the time series data shown in
                        Figure~\ref{\SETLABEL:TF} The slope of the graph
                        is the H parameter.}
                    \label{\SETLABEL:HP}
                \end{minipage}
            \end{center}
        \end{figure}

        \subidx{revenue}{See, rate of revenue returns}
        \subidx{returns}{See, rate of revenue returns}
        \subidx{\market}{revenues}
        \subidx{Hurst coefficient}{analysis}
        \subidx{\market}{Hurst coefficient analysis}
        \subidx{\market}{rate of change}
        \subidx{\market}{windows of opportunity}
        \subidx{rate of revenue returns}{forecast}
        \subidx{forecast}{rate of revenue returns}
        \idx{windows of opportunity}
        \subidx{programs}{tslsq}
        \subidx{tslsq}{program}

        The approximately linear slope of the graph in
        Figure~\ref{\SETLABEL:HC} implies that the variance of the
        rate of revenue returns, (per {\timescale},) in the {\market},
        $V(t_2 - t_1)$, over a period of time is proportional to the
        period of time raised to twice the Hurst
        coefficient~\cite[pp. 180]{Feder},~\cite[pp. 246]{Crownover}.
        This seems to be a quantitative statement concerning how fast,
        and to what degree, the rate of revenue returns' state of
        affairs can change over a period of time.  An additional
        implication, for Hurst coefficients sufficiently close to 0.5,
        is that the probability of the state of affairs repeating
        sometime in the future goes down with increasing
        time\footnote{It can be shown that the number of expected
        market ``high'' and ``low'' transitions, $N$, scales with the
        square root of time, or $N \propto \sqrt {t}$, meaning that
        the cumulative distribution of the probability, $P$, of the
        duration of a market's ``high'' or ``low'' exceeding a given
        time interval, $t$, is proportional to the reciprocal of the
        square root of the time interval, $P \propto 1 / \sqrt {t}$,
        (or, conversely, that the probability of the duration of a
        market's ``high'' or ``low'' exceeding a given time interval
        is proportional to the reciprocal of the time interval raised
        to the power $3 / 2$, ie., $P \propto 1 / t^{3 /
        2}$,~\cite[pp. 153]{Schroeder}. What this means is that a
        histogram of the ``zero free'' run-lengths of a market being
        ``high'' or ``low,'' over a long time, would have a $1 / t^{3
        / 2}$ characteristic.)}, $t$, $p(t) = erf (1/\sqrt{2t})$ which
        is approximately $1/\sqrt{t}$ for $t \gg
        1$~\cite[pp. 160]{Schroeder}. Figures~\ref{\SETLABEL:FN},
        and,~\ref{\SETLABEL:FF} compare methods of approximation of
        the ``forecastability'' of the rate of revenue returns in the
        {\market} for the near term and far term,
        respectively~\cite[pp. 83-84]{Peters:CAOITCM}\footnote{The
        author is not comfortable with Peters' interpretation. For
        example, if the algorithm explained
        in~\cite[pp. 82]{Peters:CAOITCM} is used on ``white noise''
        which, by definition, never has any correlations, the short
        term Hurst coefficient, and thus the ``forecastability,'' is
        still near unity---a bit of an enigma. This can be verified
        with the {\it tswhite}\/ and {\it tshurst}\/ programs, which
        are briefly described in Appendix~\ref{programs}.}.  This
        seems to be a quantitative statement concerning ``windows of
        opportunity'' in the rate of revenue returns, (per
        {\timescale}.)  The program {\it tslsq}\/ was used on the
        Hurst coefficient data, presented in
        Figure~\ref{\SETLABEL:HC}, to provide a least squares
        approximation to the Hurst coefficient. The superimposed least
        squares approximation with on original Hurst coefficient data
        is presented.  The time series data has a Hurst coefficient of
        {\thurstlow}, so that:

        \subidx{\market}{Hurst coefficient analysis}
        \begin{eqnarray}
            V\left(t_2 - t_1\right) & \propto & \left(t_2 - t_1\right)^{2 \cdot H}\\
            V\left(t_2 - t_1\right) & \propto & \left(t_2 - t_1\right)^{2 \cdot {\thurstlow}}\\
                                    & \propto & \left(t_2 - t_1\right)^{\thurstlowtwo}
            \label{\SETLABEL:V}
        \end{eqnarray}

        \subidx{fractional}{Brownian motion}
        \subidx{Brownian motion}{fractional}
        \idx{fractal}
        \noindent where $V(t_2 - t_1)$ is the variance of the
        increments of the rate of revenue returns, (per {\timescale},)
        over the time interval $t_2 -
        t_1$,~\cite[pp. 177]{Feder},~\cite[pp. 494]{Peitgen}. If $H >
        \frac{1}{2}$, then the time series is termed as being
        characterized by ``fractional Brownian
        motion~\cite[pp. 170]{Feder}.''

        \subidx{rate of revenue returns}{predictability}
        \subidx{rate of revenue returns}{forecastability}
        \subidx{rate of revenue returns}{consistency}
        \subidx{predictability}{rate of revenue returns}
        \subidx{forecastability}{rate of revenue returns}
        \subidx{consistency}{rate of revenue returns}
        \subidx{\market}{rate of revenue returns, predictability}
        \subidx{\market}{rate of revenue returns, forecastability}
        \subidx{\market}{rate of revenue returns, consistency}
        \subidx{Hurst coefficient}{analysis}
        \subidx{\market}{Hurst coefficient analysis}
        \subidx{\market}{rate of change}

        In some sense, the Hurst coefficient is a quantitative
        expression of the ``forecastability'' of the future based on
        the past\footnote{Actually, in general, when summing fractal
        entities, the method used should be a root mean square
        process, dependent on the Hurst Coefficient, $H$, where
        $P_{total}^H = P_1^H + P_2^H + \cdots$, where $P_n$ is the
        fractal entities. For a Brownian motion, or random walk type
        of fractal the Hurst Coefficient is a function of time into
        the future. For the ``near term,'' the Hurst coefficient is
        very near unity, meaning the summation process is linear. For
        the ``long term,'' $H \approx 0.5$, or a standard root mean
        square summation process should be used. If $H$ is $0.5$ then
        the market is termed a Brownian motion, or random walk
        process. If it is larger than 0.5, it is termed fractional
        Brownian motion process. For a random walk process, ``near
        term'' and ``far term'' are quantitatively differentiated on
        the Hurst Coefficient graph where $1 - \ln (t) = 0.5 \cdot \ln
        (t)$, or when $\ln (t) = 2$, or $t = 7.389\ldots$ See
        Section~\ref{\SETLABEL:FS} for the particulars on using Hurst
        Coefficient to sum fractal process' for the {\market}. See
        also~\cite[pp. 67, 83-84]{Peters:CAOITCM} and~\cite[pp. 129,
        159]{Schroeder} for particulars on the implications of the
        Hurst Coefficient and root mean square summation issues.}.  A
        Hurst coefficient of {\thurstlow}, (for the near future, and
        {\thurstall} for the distant future.) implies that the
        likelihood of the rate of revenue returns, (per {\timescale},)
        for any two consecutive {\timescale}s being the same is
        {\thurstlowhundred}\%~\cite[pp. 66]{Peters:CAOITCM} for the
        near future, and {\thurstall} for the distant
        future. Likewise, there is a {\thurstlowhundred}\% chance of
        the rate of revenue returns, (per {\timescale},) movements
        being the same in consecutive time periods---ie., if, in a
        given {\timescale}, the rate of revenue returns, (per
        {\timescale},) is increasing, there is a {\thurstlowhundred}\%
        that the rate of revenue returns, (per {\timescale},) will
        increase in the following period, also. In some sense, this is
        a quantitative statement on how ``predictable,'' or
        ``forecastable'' the rate of revenue returns, (per
        {\timescale},) for the {\market} are over time, since the
        probability of having $n$ many consecutive {\timescale}s of
        the same agenda is $H^n$ where $H$ is the Hurst coefficient,
        or, letting the short term probability of having $n$ many
        {\timescale}s of the same market agenda, $p_a$, is:

        \begin{eqnarray}
            p_a\left(n\right) & = & H^{n}\\
                              & = & {\thurstlow}^{n}
            \label{\SETLABEL:MA}
        \end{eqnarray}

        \subidx{rate of revenue returns}{predictability}
        \subidx{rate of revenue returns}{forecastability}
        \subidx{rate of revenue returns}{consistency}
        \subidx{predictability}{rate of revenue returns}
        \subidx{forecastability}{rate of revenue returns}
        \subidx{consistency}{rate of revenue returns}
        As an interesting interpretation of the normalized increments
        of the time series data presented in
        Figure~\ref{\SETLABEL:TF}, if the vertical axis is multiplied
        by 100, to convert to percent, then the graph represents the
        error, in percent, that would be made by forecasting, month by
        month, that the next {\timescale}'s rate of revenue returns
        would be the same as the current {\timescale}'s revenue
        rate. Interestingly, it is $\datafractionmean \cdot 100$
        percent, on the average, with a standard deviation of
        $\datafractionstddev \cdot 100$ percent, and a root mean
        square error value of $\datafractionrms \cdot 100$
        percent---small values for such a simple forecasting
        mechanism.

        \subidx{\market}{rate of revenue returns, range}
        \subidx{Hurst coefficient}{analysis}
        \subidx{\market}{Hurst coefficient analysis}
        \subidx{\market}{rate of change}

        This is, essentially, a statement of the range of values, in
        the increments of the rate of revenue returns, (per
        {\timescale},) that is to be expected over the time interval,
        $t_2 - t_1$,
        $R_v$,~\cite[pp. 178]{Feder},~\cite[pp. 172]{Cambel}:

        \begin{eqnarray}
            R_v\left(t_2 - t_1\right) & \propto & \left(t_2 - t_1\right)^{H}\\
                                      & \propto & \left(t_2 - t_1\right)^{\thurstlow}
            \label{\SETLABEL:R}
        \end{eqnarray}

        \subidx{\market}{rate of revenue returns, range}
        \subidx{Hurst coefficient}{analysis}
        \subidx{\market}{Hurst coefficient analysis}
        \subidx{\market}{rate of change}
        \subidx{Markov}{statistics}
        \subidx{statistics}{Markov}
        \noindent where $R$ is the range of values in the increments
        of the rate of revenue returns, (per {\timescale}.) A Hurst
        coefficient, $H$, that is much larger than $\frac{1}{2}$, (but
        less than 1,) implies a strongly non-Gaussian distribution in
        the increments of the rate of revenue returns, (per
        {\timescale},)~\cite[pp. 152, 194]{Feder}, and a Hurst
        coefficient near $\frac{1}{2}$ implies that the increments of
        the rate of revenue returns, (per {\timescale}) is
        characteristic of an independent
        process~\cite[pp. 195]{Feder}. Extreme caution should be
        exercised in using Markov statistics in any analysis where the
        Hurst coefficient is not
        $\frac{1}{2}$,~\cite[pp. 124]{Crownover},~\cite[pp. 106]{Peters:CAOITCM}.


        As a useful approximation, if $H$, is approximately
        $\frac{1}{2}$, Equation~\ref{\SETLABEL:R} reduces
        to,~\cite[pp. 129]{Schroeder}:

        \begin{eqnarray}
            R\left(t_2 - t_1\right) & \propto & (t_2 - t_1)^{\frac{1}{2}}\\
                                    & \propto & \sqrt{\left(t_2 - t_1\right)}
        \end{eqnarray}

        \subidx{\market}{rate of revenue returns, range}
        \subidx{\market}{rate of revenue returns, increase and decrease}
        \subidx{Hurst coefficient}{analysis}
        \subidx{\market}{Hurst coefficient analysis}
        \subidx{\market}{rate of change}
        \subidx{Markov}{statistics}
        \subidx{statistics}{Markov}

        In the case where the Hurst coefficient, $H$, is
        $\frac{1}{2}$, the range of values in the increments of the
        rate of revenue returns, (per {\timescale},) divided by the
        standard deviation of these values, $S$, can be anticipated to
        increase over time according to the following
        relation,~\cite[pp. 154]{Feder},~\cite[pp. 129]{Schroeder}:

        \begin{equation}
            \frac{R\left(t_2 - t_1\right)}{S} \propto \left(t_2 - t_1\right)^{\frac{1}{2}}
        \end{equation}

        \subidx{\market}{rate of revenue returns, range}
        \subidx{\market}{rate of revenue returns, increase and decrease}
        \subidx{Hurst coefficient}{analysis}
        \subidx{\market}{Hurst coefficient analysis}
        \subidx{\market}{rate of change}
        \noindent which is a useful conceptual approximation, since it
        involves only the square root function---if the range and the
        standard deviation of the increments of the rate of revenue
        returns, (per {\timescale},) are known, (and $H \approx
        \frac{1}{2}$,) then the expected change in $\frac{R}{S}$, will
        increase with the square root of time\footnote{To be precise,
        it is actually asymptotically proportional to
        $\tau^{\frac{1}{2}}$}.

        Another useful approximation when rescaling processes that are
        characterize by Brownian motion, (ie., when $H \approx
        \frac{1}{2}$,) is that:

        \begin{eqnarray}
            X\left(t\right) & \propto & \frac{X\left(rt\right)}{r^{H}}\\
                            & \propto & \frac{X\left(rt\right)}{r^{\thurstlow}}
        \end{eqnarray}

        \idx{Brownian motion}
        \idx{fractal}
        Where $X(t)$ is the process characterized by Brownian motion,
        and $r$ is a scaling factor,~\cite[pp. 494]{Peitgen}.

        \subidx{programs}{tslsq}
        \subidx{tslsq}{program}
        The program {\it tslsq}\/ was used on the H parameter data,
        presented in Figure~\ref{\SETLABEL:HP}, to provide a least
        squares approximation to the H parameter for the
        {\market}. The superimposed least squares approximation on the
        original H parameter data is presented.  By contrast, the H
        parameter, as derived by the methodology outlined
        in~\cite[pp. 249]{Crownover}, is {\thcalclow} for the near
        future, and {\thcalcall} for the distant future.

        \subidx{\market}{Hurst coefficient analysis}
        \subidx{Hurst coefficient}{analysis}
        \subidx{increments}{normalized}
        \subidx{normalized}{increments}
        \subidx{programs}{tshurst}
        \subidx{tshurst}{program}
        \subidx{\market}{H parameter analysis}
        \subidx{H parameter}{analysis}
        \subidx{programs}{tshcalc}
        \subidx{tshcalc}{program}
        Figures~\ref{\SETLABEL:HC} and~\ref{\SETLABEL:HP} represent
        Hurst coefficient and H parameter data that are derived from
        the normalized increments, shown in
        Figure~\ref{\SETLABEL:TF}. In this case, the data is
        considered a normalized derivative of the time series data
        presented in Figure~\ref{\SETLABEL:TF}, instead of a
        cumulative sum.  The program, {\it tshurst}\/, is described
        briefly in appendix~\ref{programs}, and the data for
        figures~\ref{\SETLABEL:THC} and~\ref{\SETLABEL:THP} was made
        using the -d option.

        \begin{figure}[ht]
            \begin{center}
                \begin{minipage}[t]{0.45\textwidth}
                    \epsfxsize=1.0\linewidth
                    \epsffile{\directory/data.tsfraction.tshurst-d.eps}
                    \caption[{\market}, traditional Hurst coefficient
                        data]{{\market}, traditional Hurst coefficient
                        data for the time series data shown in
                        Figure~\ref{\SETLABEL:TS}.  The slope of the
                        graph is the Hurst coefficient, and is
                        {\hurstlow} for the near term, and
                        {\hurstall} for the far term.}
                    \label{\SETLABEL:THC}
                \end{minipage}
                \hfill
                \begin{minipage}[t]{0.45\textwidth}
                    \epsfxsize=1.0\linewidth
                    \epsffile{\directory/data.tsfraction.tshcalc-d.eps}
                    \caption[{\market}, traditional H parameter
                        data]{{\market}, traditional H parameter data
                        for the time series data shown in
                        Figure~\ref{\SETLABEL:TS} The slope of the
                        graph is the H parameter, and is {\hcalclow}
                        for the near term, and {\hcalcall} for the
                        far term.}
                    \label{\SETLABEL:THP}
                \end{minipage}
            \end{center}
        \end{figure}

% Local Variables:
% TeX-parse-self: t
% TeX-auto-save: t
% TeX-master: "fractal.tex"
% End:


        \subsubsection{Observations on the Hurst Coefficient Analysis}

            Many {\market} industry analyst speculate that there is
            ``periodic'' behavior in the market place, at
            approximately 5 year intervals. Both the Hurst coefficient
            and H parameter graphs would tend to support the
            intuition. Notice that the slope of the graphs, in
            figures~\ref{\SETLABEL:HC} and~\ref{\SETLABEL:HP}, tend to
            decrease abruptly at $t \approx \ln(3) \approx 20$
            {\timescale}s, which is approximately 60 months, or 5
            years~\cite[pp. 96]{Peters:CAOITCM}. Whether this is
            ``periodic'' behavior, or an indication of more complex
            system dynamics, perhaps ``chaotic,'' remains to be
            seen. If that is the case, it could provide an exploitive
            venue.

        %
% -----------------------------------------------------------------------------
%
% A license is hereby granted to reproduce this software source code and
% to create executable versions from this source code for personal,
% non-commercial use.  The copyright notice included with the software
% must be maintained in all copies produced.
%
% THIS PROGRAM IS PROVIDED "AS IS". THE AUTHOR PROVIDES NO WARRANTIES
% WHATSOEVER, EXPRESSED OR IMPLIED, INCLUDING WARRANTIES OF
% MERCHANTABILITY, TITLE, OR FITNESS FOR ANY PARTICULAR PURPOSE.  THE
% AUTHOR DOES NOT WARRANT THAT USE OF THIS PROGRAM DOES NOT INFRINGE THE
% INTELLECTUAL PROPERTY RIGHTS OF ANY THIRD PARTY IN ANY COUNTRY.
%
% Copyright (c) 1994-2006, John Conover, All Rights Reserved.
%
% Comments and/or bug reports should be addressed to:
%
%     john@email.johncon.com (John Conover)
%
% -----------------------------------------------------------------------------
%
% Revision: \RCSRevision \\
% Revision Time: \RCSTime UMT \\
% Revision Date: \RCSDate \\
% Revision Id: \RCSId \\
% Revision File: \RCSLog \\
\RCS $Revision: 0.0 $
\RCS $Date: 2006/01/20 04:38:13 $
\RCS $Id: fiscal.tex,v 0.0 2006/01/20 04:38:13 john Exp $
% $Log: fiscal.tex,v $
% Revision 0.0  2006/01/20 04:38:13  john
% Initial version
%
%
    \subsection{Fixed Increment Approximation for Fiscal Strategy}
        \label{\SETLABEL:FS}

        \subidx{\market}{fiscal strategy}
        \subidx{markets}{analysis}
        \subidx{analysis}{markets}
        \subidx{strategy}{fiscal}
        \subidx{fiscal}{strategy}
        The data in this section is presented in tabular form in
        Section~\ref{\SETLABELREF:LR}. This section derives various
        values based on the ``average'' of the normalized increments
        presented in Figure~\ref{\SETLABEL:TFA}. These values are an
        approximation to a, probably, complex process with a
        distribution shown in Figure~\ref{\SETLABEL:TF}. These values
        will be used in a fixed increment Brownian fractal analysis
        and simulation of the {\market}, and may, or may not, provide
        adequate accuracy for projections.

        For an organization operating in the {\market}, the fiscal
        strategy, commensurate with the aggregate environment, can be
        derived as follows~\cite[pp. 128, pp
        151]{Schroeder},~\cite[pp. 450]{Reza},~\cite[pp. 270]{Pierce}:
        \vspace{0.15in}

        \subsubsection{Logarithmic Returns}
            \label{\SETLABEL:LR}

            \subidx{logarithmic}{returns}
            \subidx{returns}{logarithmic}
            \subidx{\market}{logarithmic returns}
            The logarithmic returns can be calculated by various
            means. Four will be presented here, for comparison.

            \subidx{programs}{tsnormal}
            \subidx{tsnormal}{program}
            \subidx{logarithmic}{returns}
            \subidx{returns}{logarithmic}
            The logarithmic returns, in bits, $bits$, as computed from
            the mean, by the program {\it tsnormal}\/, which is
            described in Chapter~\ref{programs}, and is presented in
            Figure~\ref{\SETLABEL:TF}, and Equation~\ref{abits} from
            Section~\ref{ereturns} in Chapter~\ref{general}:

            \begin{equation}
                bits = \frac{\ln \left({\datafractionmean} + 1\right)}{\ln \left(2\right)} = \datafractionmeanbits
            \end{equation}

            \subidx{programs}{tslsq}
            \subidx{tslsq}{program}
            \subidx{logarithmic}{returns}
            \subidx{returns}{logarithmic}
            \noindent By comparison, the logarithmic returns, in bits,
            $bits$, as computed from the constant in the least squares
            approximation, using the program {\it tslsq}\/, which is briefly
            described in Chapter~\ref{programs}, as presented in
            Figure~\ref{\SETLABEL:TF}, and Equation~\ref{abits} from
            Section~\ref{ereturns} in Chapter~\ref{general}:

            \begin{equation}
                bits = \frac{\ln \left({\datafractionconstant} + 1\right)}{\ln \left(2\right)} = \datafractionconstantbits
            \end{equation}

            Note that if the mean is not constant in
            Figure~\ref{\SETLABEL:TF}, this method will not provide
            accurate results.

            \subidx{programs}{tslsq}
            \subidx{tslsq}{program}
            \subidx{logarithmic}{returns}
            \subidx{returns}{logarithmic}
            \noindent And by yet another comparison, using the program
            {\it tslsq}\/, which is briefly described in
            Chapter~\ref{programs}, with the -e -p options, to provide
            a formula for the least squares exponential fit to the
            time series data set presented in
            Figure~\ref{\SETLABEL:TS}:

            \begin{equation}
                bits = {\datatslsqepbits}
            \end{equation}

            \subidx{programs}{tslogreturns}
            \subidx{tslogreturns}{program}
            \subidx{logarithmic}{returns}
            \subidx{returns}{logarithmic}
            \noindent And finally, by comparison, from the
            {\it tslogreturns}\/ program, which is briefly described
            in Chapter~\ref{programs}, with the -p option, to provide
            a formula for the logarithmic returns of the time series
            data set presented in Figure~\ref{\SETLABEL:TS}:

            \begin{equation}
                bits = {\logreturns}
            \end{equation}

        \subsubsection{Calculation of Shannon Probability}
            \label{\SETLABEL:SP}

            \subidx{\market}{Shannon probability}
            Ideally, all of the values presented in
            Section~\ref{\SETLABEL:LR} would be equal. Using the
            logarithmic returns provided by the {\it tslogreturns}\/
            program, to be consistent
            with~\cite[pp. 81]{Peters:CAOITCM}

            \subidx{programs}{tslogreturns}
            \subidx{tslogreturns}{program}
            \begin{equation}
                2^{{\logreturns}t}
            \end{equation}

            \noindent therefore:
            \begin{equation}
                C\left(p\right) = {\logreturns}
            \end{equation}
            \subidx{programs}{tsshannon}
            \subidx{tsshannon}{program}
            \subidx{Shannon}{probability}
            \subidx{probability}{Shannon}
            \noindent and, {\it tsshannon}\/ {\logreturns} gives:
            \begin{equation}
                \label{\SETLABEL:F0}
                C\left({\shannonlogreturns}\right) = {\logreturns}
            \end{equation}
            \noindent therefore:
            \begin{eqnarray}
                2^{C\left({\shannonlogreturns}\right)} & = & 2^{\logreturns}\\
                                                       & = & {\twologreturns}\\
                                                       & = & {\twologreturnshundred}\%
            \end{eqnarray}
            \noindent and:
            \begin{eqnarray}
                2p - 1 & = & \left(2 \cdot {\shannonlogreturns}\right) - 1\\
                       & = & {\twopone}\\
                       \label{\SETLABEL:F1}
                       & = & {\twoponehundred}\%
            \end{eqnarray}

            \subidx{\market}{fiscal strategy}
            \subidx{markets}{analysis}
            \subidx{analysis}{markets}
            \subidx{strategy}{fiscal}
            \subidx{fiscal}{strategy}
            \subidx{\market}{fiscal strategy}
            \subidx{\market}{growth rate}
            Presuming the simplified assumptions outlined in
            Section~\ref{assumptions}, the ``typical'' organization
            operating in the {\market} executes a long term fiscal
            strategy, commensurate with the aggregate environment,
            that is to invest, every {\timescale}, in sufficient
            additional resources and infrastructure, to increase the
            manufacturing of goods and services by {\twoponehundred}\%
            of its rate of revenue returns, (per {\timescale}.) As a
            conceptual model, the remaining {\hundredtwoponehundred}\%
            will be held in ``reserve'' with a
            {\shannonlogreturnshundred}\% chance of making twice the
            {\twoponehundred}\% back, (and a
            {\hundredshannonlogreturnshundred}\% chance of making
            0.0,) in one {\timescale}, on the average, for an average
            growth in its rate of revenue returns, (per {\timescale},)
            of {\twologreturnshundred}\%, or a doubling of its rate of
            revenue returns, (per {\timescale},) in
            {\oneoverlogreturns} {\timescale}s.

        \subsubsection{Example Fixed Increment Approximation Fiscal Strategies}

            \subidx{\market}{fiscal strategy}
            \subidx{markets}{analysis}
            \subidx{analysis}{markets}
            \subidx{strategy}{fiscal}
            \subidx{fiscal}{strategy}
            \subidx{\market}{fiscal strategy}
            \subidx{\market}{growth rate}
            \subidx{\market}{management metric}
            \idx{management metric}
            A possible metric on the effectiveness of long term fiscal
            management could possibly be that if an investment of
            {\twoponehundred}\% per {\timescale} of the rate of
            revenue returns, (per {\timescale},) is made in resources
            and infrastructure, then the rate of revenue returns would
            be expected to increase by {\twologreturnshundred}\%, per
            {\timescale}, on average.

            Note that the metrics presented in this section are
            representative of the {\market} as an aggregate whole, and
            may or may not be accurate representations for any
            particular participant in the environment. Of interest to
            the participants in the environment would be a similar
            analysis of each product or service rendered in the
            marketplace.

            \subidx{\market}{fiscal strategy}
            \subidx{markets}{analysis}
            \subidx{analysis}{markets}
            \subidx{strategy}{fiscal}
            \subidx{fiscal}{strategy}
            \subidx{\market}{fiscal strategy}
            As a simple illustrative example, a company operating in
            this environment might obtain a credit line from a bank
            that is equal to {\twoponehundred}\% of its rate of
            revenue returns, (per {\timescale},) to finance additional
            operations. In this simple scenario, the company would use
            its revenue base as collateral for the loan. Some
            {\timescale}s, depending on the {\market}'s environment,
            the company's rate of revenue returns exceeds what was
            borrowed from the bank, and the loan is repaid in
            full. Other {\timescale}s, the company must default, and
            the bank seizes a portion of the company's revenue base to
            pay the delinquent loan. However, on the average, the
            company will expand its rate of revenue returns at
            {\twologreturnshundred}\% per {\timescale}.

            \subidx{\market}{fiscal strategy}
            \subidx{markets}{analysis}
            \subidx{analysis}{markets}
            \subidx{strategy}{fiscal}
            \subidx{fiscal}{strategy}
            \subidx{\market}{fiscal strategy}
            As another simple example, a company re-invests
            {\twoponehundred}\% of its rate of revenue returns, (per
            {\timescale},) in development, marketing, sales, and
            distribution of new products.  Although some products will
            be successful and the return on the investment will exceed
            the {\twoponehundred}\% per {\timescale} investment,
            others will not. However, on the average, the company will
            expand it gross rate of revenue returns at
            {\twologreturnshundred}\% per {\timescale}.

            \subidx{\market}{fiscal strategy}
            \subidx{markets}{analysis}
            \subidx{analysis}{markets}
            \subidx{strategy}{fiscal}
            \subidx{fiscal}{strategy}
            \subidx{\market}{fiscal strategy}
            \subidx{\market}{product portfolio}
            \subidx{\market}{product diversity}
            \subidx{\market}{product mix}
            \subidx{\market}{optimum number of products}
            \idx{product portfolio}
            \idx{product diversity}
            \idx{optimum number of products}
            \idx{product mix}

            As an example of ``product portfolio'' management, suppose
            a company re-invests {\twoponehundred}\% of its rate of
            revenue returns, (per {\timescale},) in development,
            marketing, sales, and distribution of new products.
            Further suppose that the company has two products, and a
            fractal analysis of the individual product rate of revenue
            return time series indicates that one product has a
            Shannon probability of 0.65, and the other has a Shannon
            probability of 0.55. Then the percentage of re-investment
            in the first product would be $(2 \cdot 0.65 - 1) \cdot
            {\twoponehundred}$, percent of the rate of revenue
            returns, and $(2 \cdot 0.55 - 1) \cdot {\twoponehundred}$
            percent for the second product, implying that the company
            should diversify its product line\footnote{The astute
            reader would note that the linear addition was used to add
            the contribution to development of each product. This is a
            ``near term'' interpretation. Actually, in general, the
            method used should be a root mean square process,
            dependent on the Hurst Coefficient, $H$, where
            $P_{total}^H = P_1^H + P_2^H + \cdots$, where $P_n$ is the
            contribution to each individual product. For a Brownian
            motion, or random walk type of fractal the Hurst
            Coefficient is a function of time into the future. For the
            ``near term,'' the Hurst coefficient is very near unity,
            meaning the summation process is linear. For the ``long
            term,'' $H \approx 0.5$, or a standard root mean square
            summation process should be used. If $H$ is $0.5$ then the
            market is termed a Brownian motion, or random walk
            process. If it is larger than 0.5, it is termed fractional
            Brownian motion process. For a random walk process, ``near
            term'' and ``far term'' are quantitatively differentiated
            on the Hurst Coefficient graph where $1 - \ln (t) = 0.5
            \cdot \ln (t)$, or when $\ln (t) = 2$, or $t =
            7.389\ldots$ See~\cite[pp. 67, 83-84]{Peters:CAOITCM}
            and~\cite[pp. 129, 159]{Schroeder} for particulars on the
            implications of the Hurst Coefficient and root mean square
            summation issues.}.  Note that this is a ``bet hedging''
            metric methodology, and assumes that the products have
            uncorrelated revenue return rates. If this re-investment
            methodology is not feasible, perhaps for strategic
            financial reasons, then the re-investment in both products
            should total the ${\twoponehundred}$\%, and the investment
            in each product should be made at a ratio of $\frac{(2
            \cdot 0.65 - 1)}{(2 \cdot 0.55 - 1)} = 3 : 1$,
            respectively. Note that this ``bet hedging'' can be used
            to define the optimal number of products that can be
            supported on the rate of revenue returns. If it assumed
            that all products are ``typical'' for the {\market}, as a
            standard bench mark, then the optimal number will be
            $\frac{1}{{\twopone}}$. Note that this is a
            ``theoretical'' value, since not all products are
            ``typical,'' and there may be strategic reasons, for
            example product leveraging, that may increase the number
            of products above the optimum. However, most of the
            revenue should come from the optimal number of products,
            since having more products will decrease the amount of the
            potential investment in each product, and having less than
            the optimum number of products will increase the risk that
            many of the products could suffer a ``down market''
            concurrently, impacting the rate of revenue returns.  As
            another interesting interpretation of the optimal
            ``hedging of bets,'' in product portfolio strategy, and
            considering the graph of the normalized increments
            presented in Figure~\ref{\SETLABEL:TF}, if the
            organization is running optimally, then these products
            will generate, at least in principle, one standard
            deviation, approximately $0.8413 = 84.13$\% of the future
            growth in rate of revenue returns. Naturally, these are
            approximations, and the values are an approximation to a,
            probably, complex process, and appropriate scrutiny should
            be exercised before making specific projections.  As yet
            another example of ``product portfolio'' management,
            consider the issue of product mix. In this interpretation,
            {\twoponehundred}\% of the product manufactured should be
            ``proprietary,'' while the rest is ``industry standard.''
            As yet another possibility, {\twoponehundred}\% of the
            product manufactured should be predatory into new markets,
            and the remainder in markets that are ``traditional'' for
            the company.

% Local Variables:
% TeX-parse-self: t
% TeX-auto-save: t
% TeX-master: "fractal.tex"
% End:


        \subsubsection{Observations on the Fixed Increment Approximation for Fiscal Strategy}

            A re-investment of {\twoponehundred} of the rate of
            revenue returns per {\timescale} does not seem
            inconsistent with the industry averages, since it includes
            investments in research and development, additional
            manufacturing infrastructure, advertising,
            etc. Additionally, a product mix of {\twoponehundred}\%
            ``proprietary'' and the remainder ``industry standard''
            products seems consistent with the industry analyst
            ``20/80'' rule. The value of one standard deviation,
            $84.13$\%, of the revenue return rate being generated by
            $\frac{1}{{\twopone}}$ products seems consistent with the
            industry, also.

        %
% -----------------------------------------------------------------------------
%
% A license is hereby granted to reproduce this software source code and
% to create executable versions from this source code for personal,
% non-commercial use.  The copyright notice included with the software
% must be maintained in all copies produced.
%
% THIS PROGRAM IS PROVIDED "AS IS". THE AUTHOR PROVIDES NO WARRANTIES
% WHATSOEVER, EXPRESSED OR IMPLIED, INCLUDING WARRANTIES OF
% MERCHANTABILITY, TITLE, OR FITNESS FOR ANY PARTICULAR PURPOSE.  THE
% AUTHOR DOES NOT WARRANT THAT USE OF THIS PROGRAM DOES NOT INFRINGE THE
% INTELLECTUAL PROPERTY RIGHTS OF ANY THIRD PARTY IN ANY COUNTRY.
%
% Copyright (c) 1994-2006, John Conover, All Rights Reserved.
%
% Comments and/or bug reports should be addressed to:
%
%     john@email.johncon.com (John Conover)
%
% -----------------------------------------------------------------------------
%
% Revision: \RCSRevision \\
% Revision Time: \RCSTime UMT \\
% Revision Date: \RCSDate \\
% Revision Id: \RCSId \\
% Revision File: \RCSLog \\
\RCS $Revision: 0.0 $
\RCS $Date: 2006/01/20 04:38:13 $
\RCS $Id: companies.tex,v 0.0 2006/01/20 04:38:13 john Exp $
% $Log: companies.tex,v $
% Revision 0.0  2006/01/20 04:38:13  john
% Initial version
%
%
    \subsection{Number of Companies}
        \label{\SETLABEL:QNC}

        \subidx{\market}{number of companies}
        \subidx{number of companies}{analysis}
        \subidx{analysis}{number of companies}
        \subidx{Shannon}{probability}
        \subidx{probability}{Shannon}
        This section evaluates the approximate, or ``average,'' number
        of companies in the {\market}, and uses the method outlined in
        Chapter~\ref{general}, Section~\ref{aftsma}. Since the
        average, $avg_{ind}$, and the root mean square, $rms_{ind}$,
        of the normalized increments of the {\market} time series is
        \datafractionmean, and \datafractionrms respectively, the
        number of companies participating in the market can be
        calculated by Equation~\ref{ncompanies} to be {\ncompanies}.

        If this value seems consistent number of companies in the
        {\market}, within the assumptions outlined in
        Chapter~\ref{general}, Section~\ref{aftsma}, then it would
        seem that there is some circumstantial or indirect evidence
        that the companies participating in the {\market} are
        operating optimally, and the ``average'' Shannon probability,
        $P$ for each participating company would be, using
        Equation~\ref{pncompanies}, {\pncompanies}, which would be the
        value which should be used in Section~\ref{\SETLABEL:FS} for
        each participating company if market expansion was to be
        consistent with the rest of the industry. However, if the
        Shannon probability derived in Section~\ref{\SETLABEL:FS} is
        greater than the average Shannon probability for the companies
        participating in the {\market}, as derived in this section,
        then the market would, possibly, be exploitable with the
        fiscal strategy outlined in Section~\ref{\SETLABEL:FS}. The
        maximum exploitability for the {\market} is derived in
        Section~\ref{\SETLABEL:MAXSHANNON}, but it is probably of
        doubtful practicality.

        Note that these optimizations would maximize a company's
        market growth. Since there are probably many companies
        competing in the market place, this would not necessarily
        maximize a company's P\&L, as described in
        Chapter~\ref{general}, Section~\ref{ompl}. The Shannon
        probability that maximizes market share in the {\market} is
        \pncompanies, with several alternative solutions listed in the
        previous paragraph. However, these should be contrasted to the
        Shannon probability that maximizes a company's P\&L which is
        \avgrms~in the {\market}. In all cases, the fraction of the
        P\&L that should be ``wagered'' on the future, $f$, should be:

        \begin{equation}
            f = 2P - 1
        \end{equation}

        \noindent where $P$ is the particular Shannon probability
        chosen optimize a particular fiscal strategy. Interestingly,
        the measured Shannon probability of the {\market} would tend
        to indicate that the companies participating in the market
        have chosen a fiscal strategy that optimizes market growth, as
        opposed to capital growth.

        \subidx{\market}{increasing returns}
        \subidx{economic increasing returns}{\market}
        As interesting interpretation of these exploitive issues,
        since all three fiscal strategies will result in exponential
        market growth for every company participating in the market,
        is that they may represent, perhaps, an example of
        ``increasing returns.''

% Local Variables:
% TeX-parse-self: t
% TeX-auto-save: t
% TeX-master: "fractal.tex"
% End:


        %
% -----------------------------------------------------------------------------
%
% A license is hereby granted to reproduce this software source code and
% to create executable versions from this source code for personal,
% non-commercial use.  The copyright notice included with the software
% must be maintained in all copies produced.
%
% THIS PROGRAM IS PROVIDED "AS IS". THE AUTHOR PROVIDES NO WARRANTIES
% WHATSOEVER, EXPRESSED OR IMPLIED, INCLUDING WARRANTIES OF
% MERCHANTABILITY, TITLE, OR FITNESS FOR ANY PARTICULAR PURPOSE.  THE
% AUTHOR DOES NOT WARRANT THAT USE OF THIS PROGRAM DOES NOT INFRINGE THE
% INTELLECTUAL PROPERTY RIGHTS OF ANY THIRD PARTY IN ANY COUNTRY.
%
% Copyright (c) 1994-2006, John Conover, All Rights Reserved.
%
% Comments and/or bug reports should be addressed to:
%
%     john@email.johncon.com (John Conover)
%
% -----------------------------------------------------------------------------
%
% Revision: \RCSRevision \\
% Revision Time: \RCSTime UMT \\
% Revision Date: \RCSDate \\
% Revision Id: \RCSId \\
% Revision File: \RCSLog \\
\RCS $Revision: 0.0 $
\RCS $Date: 2006/01/20 04:38:13 $
\RCS $Id: operations.tex,v 0.0 2006/01/20 04:38:13 john Exp $
% $Log: operations.tex,v $
% Revision 0.0  2006/01/20 04:38:13  john
% Initial version
%
%
    \subsection{Fixed Increment Approximation for Operational Strategy}
        \label{\SETLABEL:OPS}.

        This section derives various values based on the ``average''
        of the normalized increments presented in
        Figure~\ref{\SETLABEL:TFA}. These values are an approximation
        to a, probably, complex process with a distribution shown in
        Figure~\ref{\SETLABEL:TF}. These values will be used in a
        fixed increment Brownian fractal analysis and simulation of
        the {\market}, and may, or may not, provide adequate accuracy
        for projections.

        \subidx{\market}{fiscal strategy}
        \subidx{\market}{Shannon probability}
        \subidx{strategy}{fiscal}
        \subidx{fiscal}{strategy}
        \subidx{Shannon}{probability}
        \subidx{probability}{Shannon}
        It should be noted that the analysis of fiscal strategy,
        presented in Section~\ref{\SETLABEL:FS}, is derived from the
        {\market} metrics and may, or may not, be maximally
        optimal. For the optimal fiscal strategy, which may be
        exploitable, see Section~\ref{\SETLABEL:MAXSHANNON}.

        \subidx{strategy}{exploitable}
        \subidx{exploitable}{strategy}
        \subidx{\market}{windows of opportunity}
        \idx{windows of opportunity}
        \subidx{decision}{obsolete}
        \subidx{obsolete}{decision}
        \subidx{decision}{timeliness}
        \subidx{timeliness}{decision}
        \subidx{rate of revenue returns}{forecast}
        \subidx{forecast}{rate of revenue returns}
        An additional exploitable strategy may be time itself.
        Equations~\ref{\SETLABEL:V},~\ref{\SETLABEL:R},
        and,~\ref{\SETLABEL:MA}, are, essentially, metrics on how fast
        a decision, which is based on information concerning the
        current status of the {\market}, becomes obsolete. Obviously,
        how long a decision is expected to remain relevant should be
        addressed as an operational necessity in strategic planning
        and project management. Figures~\ref{\SETLABEL:FN},
        and,~\ref{\SETLABEL:FF} compare methods of approximation of
        the ``forecastability'' of rate of revenue returns in the
        {\market} for the near term and far
        term~\cite[pp. 83-84]{Peters:CAOITCM}, respectively. As a
        general rule, caution must be exercised when making decisions
        that will span a time interval larger than the time interval
        where the ``forecastability'' of rate of revenue returns drops
        below 50\%. Beyond this time interval, the chances increase
        that the competitive and market forces will alter the market
        environment in a possibly detrimental unanticipated
        fashion. Obviously, there is significant advantage in
        ``timeliness'' of development, manufacturing, and distribution
        of products and services that are consistent with this
        temporal agenda. Automation of these processes, if executed
        consistently with this agenda, should be considered a
        competitive advantage.

        \subidx{strategy}{exploitable}
        \subidx{exploitable}{strategy}
        \subidx{rate of revenue returns}{forecast}
        \subidx{forecast}{rate of revenue returns}
        \idx{product life cycle}
        \idx{life cycle, product}
        In some sense, this temporal agenda defines the ``average''
        product or service life cycle in the {\market}. When the
        ``forecastability'' of rate of revenue returns drops below
        50\%, there is an even chance that the rate of revenue returns
        for the product or service will change in a detrimental
        fashion. If it is assumed that a product or service life cycle
        consists of a ramp up, a maintenence interval, and a ramp
        down, then, if all three life cycle intervals are equal, the
        product life cycle will be, approximately, three times the
        time interval where the ``forecastability'' of rate of revenue
        returns drops below 50\%. Although probably not an accurate
        prediction of product or service life cycle, the technique may
        be used as a conceptual approximation to the dynamics of
        ``market windows.\footnote{For example, consider the market
        for table salt. Since it has inelastic supply and demand
        curves, and is a necessary requirement for life, it would be
        expected that the Hurst coefficient would be very near
        unity---ignoring competitive pressures in the market. The
        predictability of the table salt market would, therefore, be
        expected to be relatively good, over time.}''  The conceptual
        approximation will probably predict a ``conservative'' or
        ``pessimistic'' value in relation to actual markets.

        \begin{figure}[ht]
            \begin{center}
                \begin{minipage}[t]{0.45\textwidth}
                    \epsfxsize=1.0\linewidth
                    \epsffile{\directory/datahurstlownear.eps}
                    \caption[{\market}, ``forecastability'' of near
                        term rate of revenue returns]{{\market},
                        ``forecastability'' of near term rate of
                        revenue returns. Although the error function
                        is the most accurate, for the near term,
                        $H^{t} = \thurstlow^{t}$ may be used as a
                        reliable metric of ``forecastability'' of the
                        rate of revenue returns.}
                    \label{\SETLABEL:FN}
                \end{minipage}
                \hfill
                \begin{minipage}[t]{0.45\textwidth}
                    \epsfxsize=1.0\linewidth
                    \epsffile{\directory/datahurstlowfar.eps}
                    \caption[{\market}, ``forecastability'' of far
                        term rate of revenue returns]{{\market},
                        ``forecastability'' of far term rate of
                        revenue returns. Although the error function
                        is the most accurate, for the far term,
                        $\frac{1}{\sqrt{t}}$ may be used as a reliable
                        metric of ``forecastability'' of the rate of
                        revenue returns.}
                    \label{\SETLABEL:FF}
                \end{minipage}
            \end{center}
        \end{figure}

        \idx{operations research}
        As an interesting interpretation of the data presented in
        Figure~\ref{\SETLABEL:FN}, there may be, perhaps, some
        applicability to such operational agendas as inventory
        control. Maintaining too little inventory, obviously, will
        create a situation where the organization can not exploit
        market expansion, and maintaining too much inventory,
        likewise, would over extend the company, creating unnecessary
        losses when the market contracts. The company should maintain
        inventory levels that do not exceed, from
        Equation~\ref{\SETLABEL:MA}, ${\thurstlow}^{n} = 0.5$
        {\timescale}s of operations. Since the optimal amount of
        inventory and, from Equation~\ref{\SETLABEL:V}, the variance
        of change in the rate of revenue returns in the future can be
        calculated, there may, perhaps, be some applicability to a
        forecasting methodology that can be incorporated into other
        areas of operations research, for example the linear algebras
        using simplex methodologies for optimization of manufacturing
        processes. Traditionally, these forecasts are made by the
        sales department, and are subject to various subjective
        biases.

% Local Variables:
% TeX-parse-self: t
% TeX-auto-save: t
% TeX-master: "fractal.tex"
% End:


        \subsubsection{Observations on the Fixed Increment Approximation for Operational Strategy}

            As an interesting interpretation of
            Figure~\ref{\SETLABEL:FF}, and evaluating the
            approximation $\frac{1}{\sqrt{t}}$ at 60 months gives a
            probability that the market will still have the same
            agenda of about $0.12909945$, or about 1 in 8. This is
            commensurate with numbers from the venture
            community\footnote{For example, see ``IEEE Engineering
            Management Review,'' Volume 23 Number 3, Fall 1995,
            pp. 83}. Of course new venture backed companies fail for
            many reasons, but market appropriateness to product
            portfolio 60 months in the future may be a major
            contributor. Additionally, the success rate of development
            projects of 8 month duration, which have a market success
            rate of about 1 in 3, seems consistent with
            $\frac{1}{\sqrt{3}} = 0.353553391$. Naturally, projects
            fail in the market for many reasons, but market
            appropriateness, in a dynamic market environment may be a
            major contributor to failure.

            As mentioned in Section~\ref{\SETLABEL:H},
            Equation~\ref{\SETLABEL:MA}, and the preceeding section,
            approximately 3 times the value where ${\thurstlow}^{n} =
            0.5$ could be interpreted as an approximation to the
            ``average'' product life cycle. This seems consistent with
            the 6 to 12 month life cycles quoted by many industry
            analyst. In addition, maintaining inventory levels that do
            not exceed the anticipated requirements of
            $\frac{\ln{0.5}}{\ln{\thurstlow}}$ many {\timescale}s
            seems consistent with the author's experience in the
            industry.

        For convenience of comparison, converting from quarters to
        months by dividing the logarithmic returns by 3:

        \renewcommand{\timescale}{month}
        \renewcommand{\datafractionmean}{0.044437}
\renewcommand{\datafractionmeanbits}{0.062725}
\renewcommand{\datafractionmeanq}{0.014812}
\renewcommand{\datafractionmeanbitsq}{0.021213}
\renewcommand{\datafractionstddev}{0.064421}
\renewcommand{\datafractionrms}{0.025913}
\renewcommand{\avgrms}{1.357427}
\renewcommand{\ncompanies}{66.177345}
\renewcommand{\pncompanies}{0.605400}
\renewcommand{\datafractionabsmean}{0.061981}
\renewcommand{\datafractionabsstddev}{0.047389}
\renewcommand{\datafractionconstant}{0.039513}
\renewcommand{\datafractionconstantbits}{0.055908}
\renewcommand{\datafractionconstantq}{0.013171}
\renewcommand{\datafractionconstantbitsq}{0.018878}
\renewcommand{\datafractionslope}{0.000197}
\renewcommand{\datafractionabsconstant}{0.078868}
\renewcommand{\datafractionabsslope}{-0.000675}
\renewcommand{\hurstall}{0.644727}
\renewcommand{\hurstlow}{1.028920}
\renewcommand{\hurstlowtwo}{2.057840}
\renewcommand{\hurstlowhundred}{102.892000}
\renewcommand{\hcalcall}{0.712999}
\renewcommand{\hcalclow}{0.745601}
\renewcommand{\shannonmax}{0.826923}
\renewcommand{\twoponemax}{0.653846}
\renewcommand{\logreturns}{0.019605}
\renewcommand{\twologreturns}{1.013682}
\renewcommand{\twologreturnshundred}{1.368190}
\renewcommand{\oneoverlogreturns}{51.007396}
\renewcommand{\pmax}{0.823529}
\renewcommand{\twopminusone}{0.647059}
\renewcommand{\rmsp}{0.016767}
\renewcommand{\twopx}{0.848668}
\renewcommand{\sigmap}{0.054672}
\renewcommand{\tsunfairbrownianfractionmean}{0.049753}
\renewcommand{\tsunfairbrownianfractionstddev}{0.060339}
\renewcommand{\shannonlogreturns}{0.582242}
\renewcommand{\shannonlogreturnshundred}{58.224200}
\renewcommand{\twopone}{0.164484}
\renewcommand{\twoponehundred}{16.448400}
\renewcommand{\hundredtwoponehundred}{83.551600}
\renewcommand{\hundredshannonlogreturnshundred}{41.775800}
\renewcommand{\datatslsqepbits}{0.017926}
\renewcommand{\thurstall}{0.725956}
\renewcommand{\thurstlow}{1.025249}
\renewcommand{\thurstlowtwo}{2.050498}
\renewcommand{\thurstlowhundred}{102.524900}
\renewcommand{\thcalcall}{0.885411}
\renewcommand{\thcalclow}{0.871338}
\renewcommand{\chisquared}{9.194000}
\renewcommand{\critical}{42.557000}

        \renewcommand{\SETLABEL}{\LABPRE:WSMQ}
        \renewcommand{\datafractionmean}{\datafractionmeanq}
        \renewcommand{\datafractionconstant}{\datafractionconstantq}
        \renewcommand{\datafractionmeanbits}{\datafractionmeanbitsq}
        \renewcommand{\datafractionconstantbits}{\datafractionconstantbitsq}

        %
% -----------------------------------------------------------------------------
%
% A license is hereby granted to reproduce this software source code and
% to create executable versions from this source code for personal,
% non-commercial use.  The copyright notice included with the software
% must be maintained in all copies produced.
%
% THIS PROGRAM IS PROVIDED "AS IS". THE AUTHOR PROVIDES NO WARRANTIES
% WHATSOEVER, EXPRESSED OR IMPLIED, INCLUDING WARRANTIES OF
% MERCHANTABILITY, TITLE, OR FITNESS FOR ANY PARTICULAR PURPOSE.  THE
% AUTHOR DOES NOT WARRANT THAT USE OF THIS PROGRAM DOES NOT INFRINGE THE
% INTELLECTUAL PROPERTY RIGHTS OF ANY THIRD PARTY IN ANY COUNTRY.
%
% Copyright (c) 1994-2006, John Conover, All Rights Reserved.
%
% Comments and/or bug reports should be addressed to:
%
%     john@email.johncon.com (John Conover)
%
% -----------------------------------------------------------------------------
%
% Revision: \RCSRevision \\
% Revision Time: \RCSTime UMT \\
% Revision Date: \RCSDate \\
% Revision Id: \RCSId \\
% Revision File: \RCSLog \\
\RCS $Revision: 0.0 $
\RCS $Date: 2006/01/20 04:38:13 $
\RCS $Id: fiscal.tex,v 0.0 2006/01/20 04:38:13 john Exp $
% $Log: fiscal.tex,v $
% Revision 0.0  2006/01/20 04:38:13  john
% Initial version
%
%
    \subsection{Fixed Increment Approximation for Fiscal Strategy}
        \label{\SETLABEL:FS}

        \subidx{\market}{fiscal strategy}
        \subidx{markets}{analysis}
        \subidx{analysis}{markets}
        \subidx{strategy}{fiscal}
        \subidx{fiscal}{strategy}
        The data in this section is presented in tabular form in
        Section~\ref{\SETLABELREF:LR}. This section derives various
        values based on the ``average'' of the normalized increments
        presented in Figure~\ref{\SETLABEL:TFA}. These values are an
        approximation to a, probably, complex process with a
        distribution shown in Figure~\ref{\SETLABEL:TF}. These values
        will be used in a fixed increment Brownian fractal analysis
        and simulation of the {\market}, and may, or may not, provide
        adequate accuracy for projections.

        For an organization operating in the {\market}, the fiscal
        strategy, commensurate with the aggregate environment, can be
        derived as follows~\cite[pp. 128, pp
        151]{Schroeder},~\cite[pp. 450]{Reza},~\cite[pp. 270]{Pierce}:
        \vspace{0.15in}

        \subsubsection{Logarithmic Returns}
            \label{\SETLABEL:LR}

            \subidx{logarithmic}{returns}
            \subidx{returns}{logarithmic}
            \subidx{\market}{logarithmic returns}
            The logarithmic returns can be calculated by various
            means. Four will be presented here, for comparison.

            \subidx{programs}{tsnormal}
            \subidx{tsnormal}{program}
            \subidx{logarithmic}{returns}
            \subidx{returns}{logarithmic}
            The logarithmic returns, in bits, $bits$, as computed from
            the mean, by the program {\it tsnormal}\/, which is
            described in Chapter~\ref{programs}, and is presented in
            Figure~\ref{\SETLABEL:TF}, and Equation~\ref{abits} from
            Section~\ref{ereturns} in Chapter~\ref{general}:

            \begin{equation}
                bits = \frac{\ln \left({\datafractionmean} + 1\right)}{\ln \left(2\right)} = \datafractionmeanbits
            \end{equation}

            \subidx{programs}{tslsq}
            \subidx{tslsq}{program}
            \subidx{logarithmic}{returns}
            \subidx{returns}{logarithmic}
            \noindent By comparison, the logarithmic returns, in bits,
            $bits$, as computed from the constant in the least squares
            approximation, using the program {\it tslsq}\/, which is briefly
            described in Chapter~\ref{programs}, as presented in
            Figure~\ref{\SETLABEL:TF}, and Equation~\ref{abits} from
            Section~\ref{ereturns} in Chapter~\ref{general}:

            \begin{equation}
                bits = \frac{\ln \left({\datafractionconstant} + 1\right)}{\ln \left(2\right)} = \datafractionconstantbits
            \end{equation}

            Note that if the mean is not constant in
            Figure~\ref{\SETLABEL:TF}, this method will not provide
            accurate results.

            \subidx{programs}{tslsq}
            \subidx{tslsq}{program}
            \subidx{logarithmic}{returns}
            \subidx{returns}{logarithmic}
            \noindent And by yet another comparison, using the program
            {\it tslsq}\/, which is briefly described in
            Chapter~\ref{programs}, with the -e -p options, to provide
            a formula for the least squares exponential fit to the
            time series data set presented in
            Figure~\ref{\SETLABEL:TS}:

            \begin{equation}
                bits = {\datatslsqepbits}
            \end{equation}

            \subidx{programs}{tslogreturns}
            \subidx{tslogreturns}{program}
            \subidx{logarithmic}{returns}
            \subidx{returns}{logarithmic}
            \noindent And finally, by comparison, from the
            {\it tslogreturns}\/ program, which is briefly described
            in Chapter~\ref{programs}, with the -p option, to provide
            a formula for the logarithmic returns of the time series
            data set presented in Figure~\ref{\SETLABEL:TS}:

            \begin{equation}
                bits = {\logreturns}
            \end{equation}

        \subsubsection{Calculation of Shannon Probability}
            \label{\SETLABEL:SP}

            \subidx{\market}{Shannon probability}
            Ideally, all of the values presented in
            Section~\ref{\SETLABEL:LR} would be equal. Using the
            logarithmic returns provided by the {\it tslogreturns}\/
            program, to be consistent
            with~\cite[pp. 81]{Peters:CAOITCM}

            \subidx{programs}{tslogreturns}
            \subidx{tslogreturns}{program}
            \begin{equation}
                2^{{\logreturns}t}
            \end{equation}

            \noindent therefore:
            \begin{equation}
                C\left(p\right) = {\logreturns}
            \end{equation}
            \subidx{programs}{tsshannon}
            \subidx{tsshannon}{program}
            \subidx{Shannon}{probability}
            \subidx{probability}{Shannon}
            \noindent and, {\it tsshannon}\/ {\logreturns} gives:
            \begin{equation}
                \label{\SETLABEL:F0}
                C\left({\shannonlogreturns}\right) = {\logreturns}
            \end{equation}
            \noindent therefore:
            \begin{eqnarray}
                2^{C\left({\shannonlogreturns}\right)} & = & 2^{\logreturns}\\
                                                       & = & {\twologreturns}\\
                                                       & = & {\twologreturnshundred}\%
            \end{eqnarray}
            \noindent and:
            \begin{eqnarray}
                2p - 1 & = & \left(2 \cdot {\shannonlogreturns}\right) - 1\\
                       & = & {\twopone}\\
                       \label{\SETLABEL:F1}
                       & = & {\twoponehundred}\%
            \end{eqnarray}

            \subidx{\market}{fiscal strategy}
            \subidx{markets}{analysis}
            \subidx{analysis}{markets}
            \subidx{strategy}{fiscal}
            \subidx{fiscal}{strategy}
            \subidx{\market}{fiscal strategy}
            \subidx{\market}{growth rate}
            Presuming the simplified assumptions outlined in
            Section~\ref{assumptions}, the ``typical'' organization
            operating in the {\market} executes a long term fiscal
            strategy, commensurate with the aggregate environment,
            that is to invest, every {\timescale}, in sufficient
            additional resources and infrastructure, to increase the
            manufacturing of goods and services by {\twoponehundred}\%
            of its rate of revenue returns, (per {\timescale}.) As a
            conceptual model, the remaining {\hundredtwoponehundred}\%
            will be held in ``reserve'' with a
            {\shannonlogreturnshundred}\% chance of making twice the
            {\twoponehundred}\% back, (and a
            {\hundredshannonlogreturnshundred}\% chance of making
            0.0,) in one {\timescale}, on the average, for an average
            growth in its rate of revenue returns, (per {\timescale},)
            of {\twologreturnshundred}\%, or a doubling of its rate of
            revenue returns, (per {\timescale},) in
            {\oneoverlogreturns} {\timescale}s.

        \subsubsection{Example Fixed Increment Approximation Fiscal Strategies}

            \subidx{\market}{fiscal strategy}
            \subidx{markets}{analysis}
            \subidx{analysis}{markets}
            \subidx{strategy}{fiscal}
            \subidx{fiscal}{strategy}
            \subidx{\market}{fiscal strategy}
            \subidx{\market}{growth rate}
            \subidx{\market}{management metric}
            \idx{management metric}
            A possible metric on the effectiveness of long term fiscal
            management could possibly be that if an investment of
            {\twoponehundred}\% per {\timescale} of the rate of
            revenue returns, (per {\timescale},) is made in resources
            and infrastructure, then the rate of revenue returns would
            be expected to increase by {\twologreturnshundred}\%, per
            {\timescale}, on average.

            Note that the metrics presented in this section are
            representative of the {\market} as an aggregate whole, and
            may or may not be accurate representations for any
            particular participant in the environment. Of interest to
            the participants in the environment would be a similar
            analysis of each product or service rendered in the
            marketplace.

            \subidx{\market}{fiscal strategy}
            \subidx{markets}{analysis}
            \subidx{analysis}{markets}
            \subidx{strategy}{fiscal}
            \subidx{fiscal}{strategy}
            \subidx{\market}{fiscal strategy}
            As a simple illustrative example, a company operating in
            this environment might obtain a credit line from a bank
            that is equal to {\twoponehundred}\% of its rate of
            revenue returns, (per {\timescale},) to finance additional
            operations. In this simple scenario, the company would use
            its revenue base as collateral for the loan. Some
            {\timescale}s, depending on the {\market}'s environment,
            the company's rate of revenue returns exceeds what was
            borrowed from the bank, and the loan is repaid in
            full. Other {\timescale}s, the company must default, and
            the bank seizes a portion of the company's revenue base to
            pay the delinquent loan. However, on the average, the
            company will expand its rate of revenue returns at
            {\twologreturnshundred}\% per {\timescale}.

            \subidx{\market}{fiscal strategy}
            \subidx{markets}{analysis}
            \subidx{analysis}{markets}
            \subidx{strategy}{fiscal}
            \subidx{fiscal}{strategy}
            \subidx{\market}{fiscal strategy}
            As another simple example, a company re-invests
            {\twoponehundred}\% of its rate of revenue returns, (per
            {\timescale},) in development, marketing, sales, and
            distribution of new products.  Although some products will
            be successful and the return on the investment will exceed
            the {\twoponehundred}\% per {\timescale} investment,
            others will not. However, on the average, the company will
            expand it gross rate of revenue returns at
            {\twologreturnshundred}\% per {\timescale}.

            \subidx{\market}{fiscal strategy}
            \subidx{markets}{analysis}
            \subidx{analysis}{markets}
            \subidx{strategy}{fiscal}
            \subidx{fiscal}{strategy}
            \subidx{\market}{fiscal strategy}
            \subidx{\market}{product portfolio}
            \subidx{\market}{product diversity}
            \subidx{\market}{product mix}
            \subidx{\market}{optimum number of products}
            \idx{product portfolio}
            \idx{product diversity}
            \idx{optimum number of products}
            \idx{product mix}

            As an example of ``product portfolio'' management, suppose
            a company re-invests {\twoponehundred}\% of its rate of
            revenue returns, (per {\timescale},) in development,
            marketing, sales, and distribution of new products.
            Further suppose that the company has two products, and a
            fractal analysis of the individual product rate of revenue
            return time series indicates that one product has a
            Shannon probability of 0.65, and the other has a Shannon
            probability of 0.55. Then the percentage of re-investment
            in the first product would be $(2 \cdot 0.65 - 1) \cdot
            {\twoponehundred}$, percent of the rate of revenue
            returns, and $(2 \cdot 0.55 - 1) \cdot {\twoponehundred}$
            percent for the second product, implying that the company
            should diversify its product line\footnote{The astute
            reader would note that the linear addition was used to add
            the contribution to development of each product. This is a
            ``near term'' interpretation. Actually, in general, the
            method used should be a root mean square process,
            dependent on the Hurst Coefficient, $H$, where
            $P_{total}^H = P_1^H + P_2^H + \cdots$, where $P_n$ is the
            contribution to each individual product. For a Brownian
            motion, or random walk type of fractal the Hurst
            Coefficient is a function of time into the future. For the
            ``near term,'' the Hurst coefficient is very near unity,
            meaning the summation process is linear. For the ``long
            term,'' $H \approx 0.5$, or a standard root mean square
            summation process should be used. If $H$ is $0.5$ then the
            market is termed a Brownian motion, or random walk
            process. If it is larger than 0.5, it is termed fractional
            Brownian motion process. For a random walk process, ``near
            term'' and ``far term'' are quantitatively differentiated
            on the Hurst Coefficient graph where $1 - \ln (t) = 0.5
            \cdot \ln (t)$, or when $\ln (t) = 2$, or $t =
            7.389\ldots$ See~\cite[pp. 67, 83-84]{Peters:CAOITCM}
            and~\cite[pp. 129, 159]{Schroeder} for particulars on the
            implications of the Hurst Coefficient and root mean square
            summation issues.}.  Note that this is a ``bet hedging''
            metric methodology, and assumes that the products have
            uncorrelated revenue return rates. If this re-investment
            methodology is not feasible, perhaps for strategic
            financial reasons, then the re-investment in both products
            should total the ${\twoponehundred}$\%, and the investment
            in each product should be made at a ratio of $\frac{(2
            \cdot 0.65 - 1)}{(2 \cdot 0.55 - 1)} = 3 : 1$,
            respectively. Note that this ``bet hedging'' can be used
            to define the optimal number of products that can be
            supported on the rate of revenue returns. If it assumed
            that all products are ``typical'' for the {\market}, as a
            standard bench mark, then the optimal number will be
            $\frac{1}{{\twopone}}$. Note that this is a
            ``theoretical'' value, since not all products are
            ``typical,'' and there may be strategic reasons, for
            example product leveraging, that may increase the number
            of products above the optimum. However, most of the
            revenue should come from the optimal number of products,
            since having more products will decrease the amount of the
            potential investment in each product, and having less than
            the optimum number of products will increase the risk that
            many of the products could suffer a ``down market''
            concurrently, impacting the rate of revenue returns.  As
            another interesting interpretation of the optimal
            ``hedging of bets,'' in product portfolio strategy, and
            considering the graph of the normalized increments
            presented in Figure~\ref{\SETLABEL:TF}, if the
            organization is running optimally, then these products
            will generate, at least in principle, one standard
            deviation, approximately $0.8413 = 84.13$\% of the future
            growth in rate of revenue returns. Naturally, these are
            approximations, and the values are an approximation to a,
            probably, complex process, and appropriate scrutiny should
            be exercised before making specific projections.  As yet
            another example of ``product portfolio'' management,
            consider the issue of product mix. In this interpretation,
            {\twoponehundred}\% of the product manufactured should be
            ``proprietary,'' while the rest is ``industry standard.''
            As yet another possibility, {\twoponehundred}\% of the
            product manufactured should be predatory into new markets,
            and the remainder in markets that are ``traditional'' for
            the company.

% Local Variables:
% TeX-parse-self: t
% TeX-auto-save: t
% TeX-master: "fractal.tex"
% End:


        \renewcommand{\SETLABEL}{\LABPRE:WSM}
        \renewcommand{\datafractionmean}{0.008052}
\renewcommand{\datafractionmeanbits}{0.011570}
\renewcommand{\datafractionmeanq}{0.002684}
\renewcommand{\datafractionmeanbitsq}{0.003867}
\renewcommand{\datafractionstddev}{0.038579}
\renewcommand{\datafractionrms}{0.039311}
\renewcommand{\avgrms}{0.602414}
\renewcommand{\ncompanies}{5.210454}
\renewcommand{\pncompanies}{0.544866}
\renewcommand{\datafractionabsmean}{0.029745}
\renewcommand{\datafractionabsstddev}{0.025769}
\renewcommand{\datafractionconstant}{0.010041}
\renewcommand{\datafractionconstantbits}{0.014414}
\renewcommand{\datafractionconstantq}{0.003347}
\renewcommand{\datafractionconstantbitsq}{0.004821}
\renewcommand{\datafractionslope}{-0.000021}
\renewcommand{\datafractionabsconstant}{0.035145}
\renewcommand{\datafractionabsslope}{-0.000057}
\renewcommand{\hurstall}{0.659558}
\renewcommand{\hurstlow}{0.707509}
\renewcommand{\hurstlowtwo}{1.415018}
\renewcommand{\hurstlowhundred}{70.750900}
\renewcommand{\hcalcall}{0.184942}
\renewcommand{\hcalclow}{0.102042}
\renewcommand{\shannonmax}{0.604167}
\renewcommand{\twoponemax}{0.208334}
\renewcommand{\logreturns}{0.010456}
\renewcommand{\twologreturns}{1.007274}
\renewcommand{\twologreturnshundred}{0.727387}
\renewcommand{\oneoverlogreturns}{95.638868}
\renewcommand{\pmax}{0.602094}
\renewcommand{\twopminusone}{0.204188}
\renewcommand{\rmsp}{0.008027}
\renewcommand{\twopx}{0.208583}
\renewcommand{\sigmap}{0.008047}
\renewcommand{\tsunfairbrownianfractionmean}{0.007862}
\renewcommand{\tsunfairbrownianfractionstddev}{0.038619}
\renewcommand{\shannonlogreturns}{0.560125}
\renewcommand{\shannonlogreturnshundred}{56.012500}
\renewcommand{\twopone}{0.120250}
\renewcommand{\twoponehundred}{12.025000}
\renewcommand{\hundredtwoponehundred}{87.975000}
\renewcommand{\hundredshannonlogreturnshundred}{43.987500}
\renewcommand{\datatslsqepbits}{0.007623}
\renewcommand{\thurstall}{0.633980}
\renewcommand{\thurstlow}{0.710108}
\renewcommand{\thurstlowtwo}{1.420216}
\renewcommand{\thurstlowhundred}{71.010800}
\renewcommand{\thcalcall}{0.247886}
\renewcommand{\thcalclow}{0.171737}
\renewcommand{\chisquared}{2.862000}
\renewcommand{\critical}{42.557000}

        \renewcommand{\timescale}{quarter}

        %
% -----------------------------------------------------------------------------
%
% A license is hereby granted to reproduce this software source code and
% to create executable versions from this source code for personal,
% non-commercial use.  The copyright notice included with the software
% must be maintained in all copies produced.
%
% THIS PROGRAM IS PROVIDED "AS IS". THE AUTHOR PROVIDES NO WARRANTIES
% WHATSOEVER, EXPRESSED OR IMPLIED, INCLUDING WARRANTIES OF
% MERCHANTABILITY, TITLE, OR FITNESS FOR ANY PARTICULAR PURPOSE.  THE
% AUTHOR DOES NOT WARRANT THAT USE OF THIS PROGRAM DOES NOT INFRINGE THE
% INTELLECTUAL PROPERTY RIGHTS OF ANY THIRD PARTY IN ANY COUNTRY.
%
% Copyright (c) 1994-2006, John Conover, All Rights Reserved.
%
% Comments and/or bug reports should be addressed to:
%
%     john@email.johncon.com (John Conover)
%
% -----------------------------------------------------------------------------
%
% Revision: \RCSRevision \\
% Revision Time: \RCSTime UMT \\
% Revision Date: \RCSDate \\
% Revision Id: \RCSId \\
% Revision File: \RCSLog \\
\RCS $Revision: 0.0 $
\RCS $Date: 2006/01/20 04:38:13 $
\RCS $Id: simulation.tex,v 0.0 2006/01/20 04:38:13 john Exp $
% $Log: simulation.tex,v $
% Revision 0.0  2006/01/20 04:38:13  john
% Initial version
%
%
    \subsection{Simulation of Fixed Increment Approximation for Fiscal Strategy}
        \label{\SETLABEL:TSUNFAIRBROWNIAN}

        \subidx{\market}{market simulation}
        The data in this section is presented in tabular form in
        Section~\ref{\SETLABELREF:SIM}.
        Figure~\ref{\SETLABEL:TSUNFAIRBROWNIAN0} represents a
        constructional simulation of the time series data presented in
        Figure~\ref{\SETLABEL:TS}. The program {\it
        tsunfairbrownian}\/, which is briefly described in
        appendix~\ref{programs}, was used in the reconstruction. The
        reconstructed data is superimposed on the original time series
        data.  The program, {\it tsunfairbrownian}\/, essentially,
        constructs the new time series as a Brownian fractal with
        fixed increments---the value of the fixed increment is derived
        from the root mean square average of the normalized increments
        presented in Figure~\ref{\SETLABEL:TF}. The ``quality'' of
        such a reconstruction should be subject to adequate scepticism
        and scrutiny since, in all probability, the normalized
        increments presented in Figure~\ref{\SETLABEL:TF} represent a
        relatively complex process, that may not be ``modeled'' with
        such a simple methodology.

        As a further comparison of the the constructional simulation
        with the original time series data,
        Figure~\ref{\SETLABEL:TSUNFAIRBROWNIAN1} presents a normalized
        histogram of the normalized increments of the reconstructed
        time series, superimposed on the normalized histogram
        presented in Figure~\ref{\SETLABEL:NH}.

        \subidx{\market}{fiscal strategy, simulation}
        \subidx{markets}{simulation}
        \subidx{simulation}{markets}
        \subidx{strategy}{fiscal, simulation}
        \subidx{fiscal}{strategy, simulation}
        \subidx{programs}{tsunfairbrownian}
        \subidx{tsunfairbrownian}{program}
        \begin{figure}[ht]
            \begin{center}
                \begin{minipage}[t]{0.45\textwidth}
                    \epsfxsize=1.0\linewidth
                    \epsffile{\directory/tsunfairbrownian-f.eps}
                    \caption[{\market}, Time series data, empirical and
                        simulated]{{\market}, Time series data, empirical
                        and simulated, using the program {\it tsunfairbrownian}\/
                        with f = {\datafractionrms}. This data is
                        superimposed on the data presented in
                        Figure~\ref{\SETLABEL:TS}.}
                    \label{\SETLABEL:TSUNFAIRBROWNIAN0}
                \end{minipage}
                \hfill
                \begin{minipage}[t]{0.45\textwidth}
                    \epsfxsize=1.0\linewidth
                    \epsffile{\directory/tsunfairbrownian-f.tsfraction.tsnormal-s30.eps}
                    \caption[{\market}, normalized histogram,
                        empirical and simulated]{{\market}, normalized
                        histogram of the normalized increments of the
                        time series data shown in
                        Figure~\ref{\SETLABEL:TSUNFAIRBROWNIAN0},
                        empirical and simulated.  The empirical data
                        has a mean of {\datafractionmean}, with a
                        standard deviation of {\datafractionstddev}.
                        By comparison, the simulated data has a mean
                        of {\tsunfairbrownianfractionmean} with a
                        standard deviation of
                        {\tsunfairbrownianfractionstddev}. This data
                        is superimposed on the data presented in
                        Figure~\ref{\SETLABEL:NH}. The area under the
                        four curves is identical.}
                    \label{\SETLABEL:TSUNFAIRBROWNIAN1}
                \end{minipage}
            \end{center}
        \end{figure}

% Local Variables:
% TeX-parse-self: t
% TeX-auto-save: t
% TeX-master: "fractal.tex"
% End:


        %
% -----------------------------------------------------------------------------
%
% A license is hereby granted to reproduce this software source code and
% to create executable versions from this source code for personal,
% non-commercial use.  The copyright notice included with the software
% must be maintained in all copies produced.
%
% THIS PROGRAM IS PROVIDED "AS IS". THE AUTHOR PROVIDES NO WARRANTIES
% WHATSOEVER, EXPRESSED OR IMPLIED, INCLUDING WARRANTIES OF
% MERCHANTABILITY, TITLE, OR FITNESS FOR ANY PARTICULAR PURPOSE.  THE
% AUTHOR DOES NOT WARRANT THAT USE OF THIS PROGRAM DOES NOT INFRINGE THE
% INTELLECTUAL PROPERTY RIGHTS OF ANY THIRD PARTY IN ANY COUNTRY.
%
% Copyright (c) 1994-2006, John Conover, All Rights Reserved.
%
% Comments and/or bug reports should be addressed to:
%
%     john@email.johncon.com (John Conover)
%
% -----------------------------------------------------------------------------
%
% Revision: \RCSRevision \\
% Revision Time: \RCSTime UMT \\
% Revision Date: \RCSDate \\
% Revision Id: \RCSId \\
% Revision File: \RCSLog \\
\RCS $Revision: 0.0 $
\RCS $Date: 2006/01/20 04:38:13 $
\RCS $Id: maximum.tex,v 0.0 2006/01/20 04:38:13 john Exp $
% $Log: maximum.tex,v $
% Revision 0.0  2006/01/20 04:38:13  john
% Initial version
%
%
    \subsection{Simulation of Fixed Increment Approximation for Optimally Maximal Fiscal Strategy}
        \label{\SETLABEL:MAXSHANNON}
        \subidx{\market}{fiscal strategy, simulation}
        \subidx{\market}{maximum Shannon probability}
        \subidx{markets}{simulation}
        \subidx{simulation}{markets}
        \subidx{strategy}{optimum fiscal, simulation}
        \subidx{fiscal}{optimum strategy, simulation}
        \subidx{programs}{tsunfairbrownian}
        \subidx{tsunfairbrownian}{program}
        \subidx{Shannon}{probability}
        \subidx{probability}{Shannon}

        \subidx{strategy}{exploitable}
        \subidx{exploitable}{strategy}
        \subidx{programs}{tsshannonmax}
        \subidx{tsshannonmax}{program}
        \subidx{programs}{tsunfairbrownian}
        \subidx{tsunfairbrownian}{program}
        \subidx{strategy}{fiscal}
        \subidx{fiscal}{strategy}
        The data in this section is presented in tabular form in
        Section~\ref{\SETLABELREF:MAXSHANNON}. One of the issues of
        analysis, as mentioned in Section~\ref{\SETLABEL:OPS}, is to
        determine the maximum Shannon probability for the time series
        presented in Figure~\ref{\SETLABEL:TS}. Potentially, this
        could be exploited with an aggressive fiscal
        strategy. Figure~\ref{\SETLABEL:SHANNONMAX0} is a graph of the
        output of the {\it tsshannonmax}\/ program, which is described
        briefly in appendix~\ref{programs}. The maximum of this
        function is the maximum Shannon probability for the time
        series data presented in Figure~\ref{\SETLABEL:TS}.
        Figure~\ref{\SETLABEL:SHANNONMAX1} was constructed using {\it
        tsunfairbrownian}\/ program, which is also described in
        appendix~\ref{programs}, with the maximum Shannon probability,
        and the time series data presented in
        Figure~\ref{\SETLABEL:TS}. This represents a ``what if'' the
        investment strategy was changed from a Shannon probability of
        {\shannonlogreturns}, as derived in Section~\ref{\SETLABEL:SP}
        to {\shannonmax}. This process, essentially, extracts the
        random statistical data from the time series presented in
        Figure~\ref{\SETLABEL:TS}, and constructs a new time series,
        using the random statistical data, with a different investment
        strategy.  The program, {\it tsunfairbrownian}\/, essentially,
        constructs the new time series as a Brownian fractal with
        fixed increments.  The ``quality'' of such a reconstruction
        should be subject to adequate scepticism and scrutiny since,
        in all probability, the increments in the original data
        represent a relatively complex process, that may not be
        ``modeled'' with such a simple methodology.

        \begin{figure}[ht]
            \begin{center}
                \begin{minipage}[t]{0.45\textwidth}
                    \epsfxsize=1.0\linewidth
                    \epsffile{\directory/data.tsshannonmax.eps}
                    \caption[{\market}, maximum rate of revenue
                        returns] {{\market}, maximum rate of revenue
                        returns, per {\timescale}, vs. Shannon
                        probability. The maximum rate of revenue
                        returns, per {\timescale}, occurs at a Shannon
                        probability of {\shannonmax}.}
                    \label{\SETLABEL:SHANNONMAX0}
                \end{minipage}
                \hfill
                \begin{minipage}[t]{0.45\textwidth}
                    \epsfxsize=1.0\linewidth
                    \epsffile{\directory/data.tsshannonmax-p.tsunfairbrownian-p.eps}
                    \caption[{\market}, maximum rate of revenue
                        returns] {{\market}, maximum rate of revenue
                        returns, per {\timescale}, at a Shannon
                        probability, of {\shannonmax}, corresponding
                        to a ``wager'' fraction of {\twoponemax}.}
                    \label{\SETLABEL:SHANNONMAX1}
                \end{minipage}
            \end{center}
        \end{figure}

        \subidx{fractional}{Brownian motion}
        \subidx{Brownian motion}{fractional}
        \subidx{Shannon}{probability}
        \subidx{probability}{Shannon}
        \subidx{programs}{tsshannonmax}
        \subidx{tsshannonmax}{program}
        If it is assumed that the time series data set, presented in
        Figure~\ref{\SETLABEL:TS}, constitutes classical Brownian
        motion, then the Shannon probability can be calculated by
        counting the total number of {\timescale}s that the {\market}
        movement was positive, and dividing by the total number of
        {timescale}s represented in the time series. This quotient is
        {\pmax}, as compared with the predicted value from the program
        {\it tsshannonmax}\/ of {\shannonmax}.

% Local Variables:
% TeX-parse-self: t
% TeX-auto-save: t
% TeX-master: "fractal.tex"
% End:


        \subsubsection{Observations on the Simulation of Fixed Increment Approximation for Optimally Maximal Fiscal Strategy}

            Note that these simulations are base on a very, perhaps
            overly, simplified model. For example, from
            Section~\ref{\SETLABEL:TSA}, Figure~\ref{\SETLABEL:NH}, it
            would appear that the {\market}'s normalized increments
            are characterized by fractional Brownian motion---but the
            simulations used classical Brownian motion as the
            model. One consequence of this is that a re-investment
            strategy that is to ``wager'' a fraction of {\twoponemax}
            of the rate of returns every {\timescale} is overly
            aggressive, since in the classical Brownian scenario, the
            maximum loss, in any {\timescale}, was no more that what
            was ``wagered.'' However, in the fractional Brownian
            scenario, much more can be lost. From
            Equation~\ref{fopt2},

            \begin{equation}
                \frac{avg}{rms^2} = \frac{f_{opt}}{rms} = K
            \end{equation}

            \noindent where, under the optimum classical Brownian
            scenario, $K$ is unity, or $avg = rms^2$. Notice that,
            since $f = rms$, whether the scenario is optimal or not,
            that the operational ``wager'' fraction, from
            Figure~\ref{\SETLABEL:TF} of {\datafractionrms}, vs.\ an
            ``theoretical optimal'' value of {\twoponemax} seems
            overly conservative. Additionally, notice that, at least
            in principle, the chance of failure in the fractional
            Brownian scenario, which is more accurate, would
            correspond to 1 standard deviation, or about 15.865\% per
            {\timescale}, which is unacceptably high. However, it is
            not clear why the {\market} is running at a value of
            {\datafractionrms}, which seems very
            conservative. However, a re-investment strategy of
            {\datafractionrms} per {\timescale} does not seem
            inconsistent with a failure rate, on the Fortune 500 list,
            which it is inferred that the {\market} is similar to, of
            about 50\% in ten years, which corresponds to $(1 -
            p_f)^{120} \approx 0.5$, or $p_f$, the probability of
            failure, is $0.005759576$, which is, approximately, 2.5
            standard deviations, meaning that to be consistent with
            the large companies in the Fortune 500, the re-investment
            rate should be, approximately, $\frac{\twoponemax}{2.5}$,
            compared with an operational value, from
            Figure~\ref{\SETLABEL:NH} of {\datafractionrms}.

            An interesting, and intriguing, interpretation and
            discussion of the maximum Shannon probability, is an
            explanation as to why the companies in the {\market} are
            not running an optimal re-investment strategy. This seems
            enigmatic, since those companies that run, on a long term
            average, below the optimally maximal value would seem to
            be eclipsed by those that didn't. And those that run above
            the optimally maximal value would be over extended, and
            become financially destitute during market down turns,
            which is inevitable in a fractal time series as presented
            in Figure~\ref{\SETLABEL:TS}.  It would seem that the
            natural selection process of the competitive environment
            would allow only those companies that run near the
            optimally maximal value to survive, in the long run. One
            possible explanation, foremost, is that the analytical
            methodology presented herein is inappropriate.  Another
            explanation is that the gross margins are less than the
            fraction {\shannonmax} of the rate of revenue returns, and
            thus could not accommodate such an aggressive
            re-investment strategy. If this is the case, then it
            presents an intriguing issue. If, in a capitalistic
            market, the natural outcome of the competitive situation,
            according to game-theoretic analysis, is that there will
            be many competitors, each making minimal gross margins,
            then how do the companies grow their markets?  Naturally,
            those that run the most efficient will have lower costs,
            making larger percentage of rate of revenue returns
            re-investment possible. Yet another interpretation is that
            the number of competitors would grow at an exponential
            rate, but all of them would make minimal returns. However,
            an operational Shannon probability of {\shannonlogreturns}
            is not just marginally lower than the maximum Shannon
            probability of {\shannonmax}. There is a significant
            disparity which is difficult to explain. It would seem
            that the game-theoretic eventual outcome of a competitive
            market place would be a solution that hinders growth,
            wealth and jobs creation, etc., which does not seem
            consistent with capitalistic theory. On the other hand, is
            there an optimum number of competitors in a market place,
            where the gross margins can be higher, permitting wealth
            and job creation, and also a competitive situation? If
            this analysis is correct, and that should be subject to
            scrutiny, then it would appear that this is the case. But
            this brings up another issue---that of taxation, and other
            contributions to the social welfare function. If there is
            an optimum number of competitors in the market place, that
            maximizes wealth and job creation, then, perhaps by lemma,
            there is also an optimal value of taxation rate, and other
            contributions to the social welfare function, that will
            permit maximal industrial growth, and thus maximal growth
            in the tax base. But this would seem to be inconsistent
            with the work of Kenneth Arrow and the so called
            Impossibility Theorem, which states that such
            optimizations can not be determined because the ordering
            of priorities are intransitive.  All very perplexing,
            since the simulation of the maximum Shannon probability in
            the next section seems to indicate that such an aggressive
            re-investment strategy is, indeed, feasible.

            Yet another possibility for the industry not running at
            maximum Shannon probability is the high cost of expansion
            of operations. Some of these industries require very
            sophisticated manufacturing processes, which have high
            barrier costs.

            Additionally, as mentioned in both~\cite[pp. 29]{Brock},
            and~\cite[pp. 8]{Arthur:CTIRALIBHE}, optimal efficiency
            may not be attainable in increasing-return economic
            scenarios.

        %
% -----------------------------------------------------------------------------
%
% A license is hereby granted to reproduce this software source code and
% to create executable versions from this source code for personal,
% non-commercial use.  The copyright notice included with the software
% must be maintained in all copies produced.
%
% THIS PROGRAM IS PROVIDED "AS IS". THE AUTHOR PROVIDES NO WARRANTIES
% WHATSOEVER, EXPRESSED OR IMPLIED, INCLUDING WARRANTIES OF
% MERCHANTABILITY, TITLE, OR FITNESS FOR ANY PARTICULAR PURPOSE.  THE
% AUTHOR DOES NOT WARRANT THAT USE OF THIS PROGRAM DOES NOT INFRINGE THE
% INTELLECTUAL PROPERTY RIGHTS OF ANY THIRD PARTY IN ANY COUNTRY.
%
% Copyright (c) 1994-2006, John Conover, All Rights Reserved.
%
% Comments and/or bug reports should be addressed to:
%
%     john@email.johncon.com (John Conover)
%
% -----------------------------------------------------------------------------
%
% Revision: \RCSRevision \\
% Revision Time: \RCSTime UMT \\
% Revision Date: \RCSDate \\
% Revision Id: \RCSId \\
% Revision File: \RCSLog \\
\RCS $Revision: 0.0 $
\RCS $Date: 2006/01/20 04:38:13 $
\RCS $Id: verification.tex,v 0.0 2006/01/20 04:38:13 john Exp $
% $Log: verification.tex,v $
% Revision 0.0  2006/01/20 04:38:13  john
% Initial version
%
%
    \subsection{Qualitative Verification of Fixed Increment Approximation Analysis}
        \label{\SETLABEL:QVA}

        \subidx{\market}{verification of analysis}
        \subidx{verification}{analysis}
        \subidx{analysis}{verification}
        \subidx{quality}{of analysis}
        \subidx{verification}{of methodology}
        \subidx{methodology}{verification of}
        \subidx{Shannon}{probability}
        \subidx{probability}{Shannon}

        This section evaluates various values based on the ``average''
        of the normalized increments presented in
        Figure~\ref{\SETLABEL:TFA}. These values are an approximation
        to a, probably, complex process with a distribution shown in
        Figure~\ref{\SETLABEL:TF}. These values will be used in a
        fixed increment Brownian fractal analysis of the {\market},
        and may, or may not, provide adequate accuracy for
        projections.

        The data in this section is presented in tabular form in
        sections~\ref{\SETLABELREF:VI1} and~\ref{\SETLABELREF:VI2}.
        As a subjective evaluation of the ``quality'' of the analysis
        of the {\market}, from Chapter~\ref{methodology},
        Equation~\ref{metricvalues1}, and using the mean and root mean
        square values of the normalized increments of the time series
        data presented in Figure~\ref{\SETLABEL:TS} from
        Figure~\ref{\SETLABEL:TF}, and the Shannon probability as
        calculated by counting the total number of {\timescale}s that
        the {\market} movement was positive, as presented in
        Section~\ref{\SETLABEL:MAXSHANNON}:

        \begin{eqnarray}
                  P & \approx & \frac{\frac{avg}{rms} + 1}{2}\\
            {\pmax} & \approx & \frac{\frac{\datafractionmean}{\datafractionrms} + 1}{2}\\
            {\pmax} & \approx & {\avgrms}
            \label{\SETLABEL:AVGS}
        \end{eqnarray}

        \subidx{Shannon}{probability}
        \subidx{probability}{Shannon}
        \noindent and comparing these values to the Shannon
        probability, as found by the {\it tsshannonmax}\/ program, which
        iterates for a maximum:

        \begin{eqnarray}
            {\pmax} \approx {\avgrms} \approx {\shannonmax}
        \end{eqnarray}

        \subidx{logarithmic}{returns}
        \subidx{returns}{logarithmic}
        In addition, the different methods of calculating the
        logarithmic returns, presented in Section~\ref{\SETLABEL:FS},
        should be compared. The four methods used were the mean of
        Figure~\ref{\SETLABEL:TF}, the constant in the least squares
        approximation to Figure~\ref{\SETLABEL:TF}, the least squares
        exponential approximation to Figure~\ref{\SETLABEL:TS}, and
        the logarithmic returns of Figure~\ref{\SETLABEL:TS}, derived
        as the mean of the logarithms of the quotients of the
        increments. The values for each of the methods are,
        respectively:

        \begin{equation}
            \datafractionmeanbits \approx \datafractionconstantbits \approx \datatslsqepbits \approx \logreturns
        \end{equation}

        It is implied in Section~\ref{\SETLABEL:FS},
        Subsection~\ref{\SETLABEL:SP} and in
        Section~\ref{\SETLABEL:TSUNFAIRBROWNIAN} that, a Brownian
        motion with fixed increments fractal may ``model'' the
        {\market}. Using Equation~\ref{stddev9} from
        Chapter~\ref{general}, Section~\ref{abmfi}:

        \begin{eqnarray}
                                    rms \left(2P - 1\right) & \approx & \frac{\sigma \left(2P - 1\right)}{2 \sqrt{P\left(1 - P\right)}}\\
            \datafractionrms \left(2 \cdot \pmax - 1\right) & \approx & \frac{\datafractionstddev \left(2 \cdot \pmax - 1\right)}{2\sqrt{\pmax \left(1 - \pmax\right)}}\\
                       \datafractionrms \cdot \twopminusone & \approx & \datafractionstddev \cdot \twopx\\
                                                      \rmsp & \approx & \sigmap
        \end{eqnarray}

        \noindent and, equating to the mean:

        \begin{equation}
            \datafractionmean \approx \rmsp \approx \sigmap
        \end{equation}

        \subidx{Shannon}{probability}
        \subidx{probability}{Shannon}
        \noindent where, as in Equation~\ref{\SETLABEL:AVGS} using the
        mean, root mean square, and standard deviation values of the
        normalized increments of the time series data presented in
        Figure~\ref{\SETLABEL:TS} from Figure~\ref{\SETLABEL:TF}, and
        the Shannon probability as calculated by counting the total
        number of {\timescale}s that the {\market} movement was
        positive, as presented in Section~\ref{\SETLABEL:MAXSHANNON}.

        As a final qualitative comparison, the absolute value of the
        normalized increments should be the same as the root mean
        square value\footnote{The absolute value of the normalized
        increments, when averaged, is related to the root mean square
        of the increments by a constant. If the normalized increments
        are a fixed increment, the constant is unity. If the
        normalized increments have a Gaussian distribution, the
        constant is $\approx 0.8$ depending on the accuracy of of
        ``fit'' to a Gaussian distribution.}, where the absolute value
        is presented in Figure~\ref{\SETLABEL:TFA}, and the root mean
        square value is presented in Figure~\ref{\SETLABEL:TF}:

        \begin{equation}
            \datafractionabsmean \approx \datafractionrms
        \end{equation}

        Note, that if the {\market} could be ``modeled'' as a Brownian
        motion with fixed increments fractal, then the standard
        deviation of the absolute value of the normalized increments
        of the time series data presented in Figure~\ref{\SETLABEL:TS}
        from Figure~\ref{\SETLABEL:TF} should be zero. It is
        $\datafractionabsstddev$.

% Local Variables:
% TeX-parse-self: t
% TeX-auto-save: t
% TeX-master: "fractal.tex"
% End:


    \renewcommand{\market}{North American Semiconductor Market}
    \renewcommand{\directory}{../markets/semiconductors.namerica}
    \renewcommand{\datafractionmean}{0.008052}
\renewcommand{\datafractionmeanbits}{0.011570}
\renewcommand{\datafractionmeanq}{0.002684}
\renewcommand{\datafractionmeanbitsq}{0.003867}
\renewcommand{\datafractionstddev}{0.038579}
\renewcommand{\datafractionrms}{0.039311}
\renewcommand{\avgrms}{0.602414}
\renewcommand{\ncompanies}{5.210454}
\renewcommand{\pncompanies}{0.544866}
\renewcommand{\datafractionabsmean}{0.029745}
\renewcommand{\datafractionabsstddev}{0.025769}
\renewcommand{\datafractionconstant}{0.010041}
\renewcommand{\datafractionconstantbits}{0.014414}
\renewcommand{\datafractionconstantq}{0.003347}
\renewcommand{\datafractionconstantbitsq}{0.004821}
\renewcommand{\datafractionslope}{-0.000021}
\renewcommand{\datafractionabsconstant}{0.035145}
\renewcommand{\datafractionabsslope}{-0.000057}
\renewcommand{\hurstall}{0.659558}
\renewcommand{\hurstlow}{0.707509}
\renewcommand{\hurstlowtwo}{1.415018}
\renewcommand{\hurstlowhundred}{70.750900}
\renewcommand{\hcalcall}{0.184942}
\renewcommand{\hcalclow}{0.102042}
\renewcommand{\shannonmax}{0.604167}
\renewcommand{\twoponemax}{0.208334}
\renewcommand{\logreturns}{0.010456}
\renewcommand{\twologreturns}{1.007274}
\renewcommand{\twologreturnshundred}{0.727387}
\renewcommand{\oneoverlogreturns}{95.638868}
\renewcommand{\pmax}{0.602094}
\renewcommand{\twopminusone}{0.204188}
\renewcommand{\rmsp}{0.008027}
\renewcommand{\twopx}{0.208583}
\renewcommand{\sigmap}{0.008047}
\renewcommand{\tsunfairbrownianfractionmean}{0.007862}
\renewcommand{\tsunfairbrownianfractionstddev}{0.038619}
\renewcommand{\shannonlogreturns}{0.560125}
\renewcommand{\shannonlogreturnshundred}{56.012500}
\renewcommand{\twopone}{0.120250}
\renewcommand{\twoponehundred}{12.025000}
\renewcommand{\hundredtwoponehundred}{87.975000}
\renewcommand{\hundredshannonlogreturnshundred}{43.987500}
\renewcommand{\datatslsqepbits}{0.007623}
\renewcommand{\thurstall}{0.633980}
\renewcommand{\thurstlow}{0.710108}
\renewcommand{\thurstlowtwo}{1.420216}
\renewcommand{\thurstlowhundred}{71.010800}
\renewcommand{\thcalcall}{0.247886}
\renewcommand{\thcalclow}{0.171737}
\renewcommand{\chisquared}{2.862000}
\renewcommand{\critical}{42.557000}

    \renewcommand{\timescale}{quarter}
    \subidx{market}{\market}
    \idx{\market}

    \section{\market}

        \renewcommand{\SETLABEL}{\LABPRE:NASM}
        \renewcommand{\SETLABELQ}{\LABPRE:NASMQ}
        \label{\SETLABEL}
        \renewcommand{\SETLABELREF}{\LABPREREF:NASM}

        \idx{Semiconductor Industry Association}
        For the analysis, the data was in the directory
        {\directory}\footnote{Data from the Semiconductor Industry
        Association, 1979---1994, by {\timescale}s, in millions of
        dollars, US.}.

        The data in this section is presented in tabular form in
        Section~\ref{\SETLABELREF}.

        %
% -----------------------------------------------------------------------------
%
% A license is hereby granted to reproduce this software source code and
% to create executable versions from this source code for personal,
% non-commercial use.  The copyright notice included with the software
% must be maintained in all copies produced.
%
% THIS PROGRAM IS PROVIDED "AS IS". THE AUTHOR PROVIDES NO WARRANTIES
% WHATSOEVER, EXPRESSED OR IMPLIED, INCLUDING WARRANTIES OF
% MERCHANTABILITY, TITLE, OR FITNESS FOR ANY PARTICULAR PURPOSE.  THE
% AUTHOR DOES NOT WARRANT THAT USE OF THIS PROGRAM DOES NOT INFRINGE THE
% INTELLECTUAL PROPERTY RIGHTS OF ANY THIRD PARTY IN ANY COUNTRY.
%
% Copyright (c) 1994-2006, John Conover, All Rights Reserved.
%
% Comments and/or bug reports should be addressed to:
%
%     john@email.johncon.com (John Conover)
%
% -----------------------------------------------------------------------------
%
% Revision: \RCSRevision \\
% Revision Time: \RCSTime UMT \\
% Revision Date: \RCSDate \\
% Revision Id: \RCSId \\
% Revision File: \RCSLog \\
\RCS $Revision: 0.0 $
\RCS $Date: 2006/01/20 04:38:13 $
\RCS $Id: fraction.tex,v 0.0 2006/01/20 04:38:13 john Exp $
% $Log: fraction.tex,v $
% Revision 0.0  2006/01/20 04:38:13  john
% Initial version
%
%
    \subsection{Time Series Increments Analysis}
        \label{\SETLABEL:TSA}

        \subidx{\market}{Time series analysis}
        \subidx{time series}{increments}
        \subidx{time series}{analysis}
        \subidx{cumulative sum}{analysis}
        \subidx{analysis}{cumulative sum}
        \subidx{analysis}{random process}
        \subidx{random process}{analysis}
        \subidx{Gaussian}{increments}
        \subidx{increments}{Gaussian}
        \subidx{Brownian}{motion, fractional}
        \subidx{fractional}{Brownian motion}
        \subidx{fractal}{Brownian motion}
        The data in this section is presented in tabular form in
        Section~\ref{\SETLABELREF:TSA}.  Figure~\ref{\SETLABEL:TS} is
        a graph of the time series data for the {\market}.

        \subidx{increments}{normalized}
        \subidx{normalized}{increments}
        \subidx{programs}{tsfraction}
        \subidx{tsfraction}{program}
        Figure~\ref{\SETLABEL:TF} is a graph of the normalized
        increments of the time series data presented in
        Figure~\ref{\SETLABEL:TS}. The data presented was made by
        running the program {\it tsfraction}\/ on the time series
        data. The program {\it tsfraction}\/ is described briefly in
        Appendix~\ref{programs}, and subtracts the previous value from
        the next value, dividing this difference by the previous
        value, for each element in the time series data. The new time
        series contains the instantaneous change in the rate of
        revenue returns, divided by the magnitude of the instantaneous
        rate of revenue returns.

        \subidx{mean}{standard deviation}
        \subidx{standard deviation}{mean}
        \idx{root mean square}
        \idx{least squares approximation}
        \begin{figure}[ht]
            \begin{center}
                \begin{minipage}[t]{0.45\textwidth}
                    \epsfxsize=1.0\linewidth
                    \epsffile{\directory/data.eps}
                    \caption{{\market}, time series data.}
                    \label{\SETLABEL:TS}
                    \label{\SETLABELQ:TS}
                \end{minipage}
                \hfill
                \begin{minipage}[t]{0.45\textwidth}
                    \epsfxsize=1.0\linewidth
                    \epsffile{\directory/data.tsfraction.eps}
                    \caption[{\market}, normalized
                        increments]{{\market}, normalized increments
                        of the time series data presented in
                        Figure~\ref{\SETLABEL:TS}. The mean is
                        {\datafractionmean} with a standard deviation
                        of {\datafractionstddev}. The formula for the
                        least squares approximation is
                        ${\datafractionconstant} +
                        {\datafractionslope}t$, and the root mean
                        squared value is {\datafractionrms}. The
                        graph, labeled ``data\-.tsfraction\-.tsrms,''
                        is the running root mean square, and
                        ``data\-.tsfraction\-.tsavg'' is the running
                        average of the normalized increments.  This
                        graph is the fraction of change in the time
                        series, as a function of time. Note that the
                        slope of the mean, {\datafractionslope}, is
                        the coefficient of the nonlinearity term in
                        the normalized increments. See
                        Chapter~\ref{general}, Section~\ref{nlextend}
                        for a possible application of the logistic
                        function to this data set.}
                    \label{\SETLABEL:TF}
                    \label{\SETLABELQ:TF}
                \end{minipage}
            \end{center}
        \end{figure}

        \subidx{absolute value}{increments}
        \subidx{increments}{absolute value}

        Figure~\ref{\SETLABEL:TFA} is a graph of the absolute value of
        the normalized increments of the time series data presented in
        Figure~\ref{\SETLABEL:TF}. The data presented was made by
        running the Unix utility sed(1) on the normalized increments
        time series data to remove the negative signs. This is an
        absolute value procedure.  The resulting time series contains
        the absolute value of the instantaneous change in the rate of
        revenue returns, divided by the magnitude of the instantaneous
        rate of revenue returns\footnote{The absolute value of the
        normalized increments, when averaged, is related to the root
        mean square of the increments by a constant. If the normalized
        increments are a fixed increment, the constant is unity. If
        the normalized increments have a Gaussian distribution, the
        constant is $\approx 0.8$ depending on the accuracy of of
        ``fit'' to a Gaussian distribution.}.

        \subidx{histogram}{normalized}
        \subidx{normalized}{histogram}
        \subidx{programs}{tsnormal}
        \subidx{tsnormal}{program}
        \subidx{mean}{standard deviation}
        \subidx{standard deviation}{mean}
        \idx{root mean square}
        \idx{least squares approximation}
        \subidx{\market}{analysis of increments}
        Figure~\ref{\SETLABEL:NH} is the normalized histogram of the
        normalized increments of the time series data shown in
        Figure~\ref{\SETLABEL:TF}. The abscissa is 3 $\sigma$ limits,
        and the area under the two curves is identical. The data for
        this figure was produced by the program {\it tsnormal}\/,
        which is described briefly in Appendix~\ref{programs}.

        \begin{figure}[ht]
            \begin{center}
                \begin{minipage}[t]{0.45\textwidth}
                    \epsfxsize=1.0\linewidth
                    \epsffile{\directory/data.tsfraction.abs.eps}
                    \caption[{\market}, absolute value of the
                        normalized increments]{{\market}, absolute
                        value of the normalized increments of the time
                        series data presented in
                        Figure~\ref{\SETLABEL:TF}.  The mean is
                        {\datafractionabsmean} with a standard
                        deviation of {\datafractionabsstddev}. The
                        formula for the least squares approximation is
                        ${\datafractionabsconstant} +
                        {\datafractionabsslope}t$, and the root mean
                        square value, from Figure~\ref{\SETLABEL:TF},
                        is {\datafractionrms}.  The graph, labeled
                        ``data\-.tsfraction\-.tsrms,'' is the running
                        root mean square, and
                        ``data\-.tsfraction\-.tsavg'' is the running
                        average of the normalized increments presented
                        in Figure~\ref{\SETLABEL:TF}, superimposed
                        here for convenience. This graph is the
                        absolute value of the fraction of change in
                        the time series, as a function of time.}
                    \label{\SETLABEL:TFA}
                    \label{\SETLABELQ:TFA}
                \end{minipage}
                \hfill
                \begin{minipage}[t]{0.45\textwidth}
                    \epsfxsize=1.0\linewidth
                    \epsffile{\directory/data.tsfraction.tsnormal-s30.eps}
                    \caption[{\market}, normalized histogram of the
                        normalized increments]{{\market}, normalized
                        histogram of the normalized increments of the
                        time series data shown in
                        Figure~\ref{\SETLABEL:TF}.  The data has a
                        mean of {\datafractionmean}, with a standard
                        deviation of {\datafractionstddev}.  The area
                        under the two curves is identical. The
                        $\chi^2$ value of the observed and expected
                        values of the two curves is {\chisquared},
                        with a critical value of {\critical}.}
                    \label{\SETLABEL:NH}
                \end{minipage}
            \end{center}
        \end{figure}

        \subidx{programs}{tsXsquared}
        \subidx{tsXsquared}{program}
        \subidx{\market}{chi-squared values of increments}
        The program {\it tsXsquared}\/, which is briefly described in
        appendix~\ref{programs}, was used to derive the $\chi^2$
        statistics for the data presented in
        Figure~\ref{\SETLABEL:NH}.

        \subidx{programs}{tsstatest}
        \subidx{tsstatest}{program}
        \subidx{\market}{statistical estimates}

        Figure~\ref{\SETLABEL:SE} is the statistical estimate for the
        data presented in Figure~\ref{\SETLABEL:TF}, as derived by the
        program {\it tsstatest}\/, which is briefly described in
        appendix~\ref{programs}.

        \begin{figure}[ht]
            \begin{center}
                \begin{minipage}[t]{\textwidth}
                    \center{\fbox{\parbox{0.9\textwidth}{\XXX{\directory/data.tsstatest-f0.1-c0.9-i.tex}}}}
                    \caption[{\market}, statistical estimates of the
                        normalized increments]{{\market}, statistical
                        estimates of the normalized increments of the
                        time series shown in Figure~\ref{\SETLABEL:TF}.
                        The table was produced with the {\it
                        tsstatest}\/ program, and illustrates the
                        size of the data set required for a confidence
                        level of 90\%, with an error estimate of $\pm$
                        10\%, or alternately, the error estimate on
                        the time series shown in Figure~\ref{\SETLABEL:TF}.}
                    \label{\SETLABEL:SE}
                \end{minipage}
            \end{center}
        \end{figure}

        Note that the data set size estimations, as produced by the
        {\it tsstatest}\/ program, are probably very conservative,
        depending on the magnitude of the Shannon probability, $P =
        \shannonlogreturns$, as derived in
        Section~\ref{\SETLABEL:SP}. See Chapter~\ref{general},
        Section~\ref{serdss} for possible alternative methodologies
        for addressing the analysis of fractal time series with
        limited data set sizes. Depending on the magnitude of the
        Shannon probability, $P$, these estimates can be several
        orders of magnitude too high.

        \subidx{derivative of increments}{normalized}
        \subidx{normalized}{derivative of increments}
        \subidx{programs}{tsderivative}
        \subidx{tsderivative}{program}
        Figure~\ref{\SETLABEL:TF1} is the normalized histogram of the
        first derivative of the normalized increments of the time
        series data shown in Figure~\ref{\SETLABEL:TF}. In principle,
        if the distribution of the normalized increments presented in
        Figure~\ref{\SETLABEL:NH} is Gaussian in nature, this
        distribution would be similar to ``white noise,'' as presented
        in appendix~\ref{programs}, Figure~\ref{whiteexample}. The
        data was generated by the {\it tsderivative}\/ program, which
        is briefly described in
        appendix~\ref{programs}. Figure~\ref{\SETLABEL:TF2} is the
        normalized histogram of the second derivative of the
        normalized increments of the time series data shown in
        Figure~\ref{\SETLABEL:TF}. In principle, if the distribution
        of the normalized increments presented in
        Figure~\ref{\SETLABEL:NH} is an integrated Gaussian
        distribution in nature, this distribution would be similar to
        ``white noise,'' as presented in appendix~\ref{programs},
        Figure~\ref{whiteexample}.

        \begin{figure}[ht]
            \begin{center}
                \begin{minipage}[t]{0.45\textwidth}
                    \epsfxsize=1.0\linewidth
                    \epsffile{\directory/data.tsfraction.tsderivative.tsnormal-s30.eps}
                    \caption[{\market}, histogram of the first
                        derivative of the increments]{{\market},
                        normalized histogram of the first derivative
                        of the normalized increments of the time
                        series data shown in
                        Figure~\ref{\SETLABEL:TF}.}
                    \label{\SETLABEL:TF1}
                \end{minipage}
                \hfill
                \begin{minipage}[t]{0.45\textwidth}
                    \epsfxsize=1.0\linewidth
                    \epsffile{\directory/data.tsfraction.2tsderivative.tsnormal-s30.eps}
                    \caption[{\market}, histogram of the second
                        derivative of the increments]{{\market},
                        normalized histogram of second derivative of
                        the the normalized increments of the time
                        series data shown in
                        Figure~\ref{\SETLABEL:TF}.}
                    \label{\SETLABEL:TF2}
                \end{minipage}
            \end{center}
        \end{figure}

        \subidx{fractal}{range}
        \subidx{fractal}{R/S analysis}
        \subidx{\market}{rate of revenue returns, range}
        \subidx{\market}{deterministic mechanism}
        \subidx{deterministic}{mechanism}
        \subidx{mechanism}{deterministic}
        Figure~\ref{\SETLABEL:TR} is the range of values of the time
        series shown in Figure~\ref{\SETLABEL:TS}. The horizontal axis
        is time into the future. In principle, if the time series was
        characterized as fractional Brownian motion the graph in
        Figure~\ref{\SETLABEL:TR} would be a square root
        function\footnote{Note that the ``roughness,'' or ``sawtooth''
        characteristics of the graph in Figure~\ref{\SETLABEL:TR} are
        a computational artifact---caused by not using the -m option
        to the program {\it tshurst}\/, which is computationally
        inefficient.}. Figure~\ref{\SETLABEL:TD} is the deterministic
        map of the normalized increments of the time series data shown
        in Figure~\ref{\SETLABEL:TF}. The deterministic map is useful
        for determining if a time series was created by a
        deterministic mechanism. This, essentially, maps each element
        in the time series with the previous element in the time
        series.  See,~\cite[pp. 745]{Peitgen}.

        \begin{figure}[ht]
            \begin{center}
                \begin{minipage}[t]{0.45\textwidth}
                    \epsfxsize=1.0\linewidth
                    \epsffile{\directory/data.tshurst-f.eps}
                    \caption[{\market}, range]{{\market}, range of the
                        time series data shown in
                        Figure~\ref{\SETLABEL:TS}.}
                    \label{\SETLABEL:TR}
                \end{minipage}
                \hfill
                \begin{minipage}[t]{0.45\textwidth}
                    \epsfxsize=1.0\linewidth
                    \epsffile{\directory/data.tsfraction.tsdeterministic.eps}
                    \caption[{\market}, deterministic map]{{\market},
                        deterministic map of the normalized increments
                        of the time series data shown in
                        Figure~\ref{\SETLABEL:TF}.}
                    \label{\SETLABEL:TD}
                \end{minipage}
            \end{center}
        \end{figure}

% Local Variables:
% TeX-parse-self: t
% TeX-auto-save: t
% TeX-master: "fractal.tex"
% End:


        \subsubsection{Observations on the Time Series Increments Analysis}

            Figure~\ref{\SETLABEL:NH} would seem to indicate that the
            time series data for the {\market} represents a cumulative
            sum/integration of a random process that has a Gaussian
            distribution, (ie., satisfies the Gaussian increments
            property of fractional Brownian
            motion~\cite[pp. 250]{Crownover},) tending to justify the
            assumption that the time series data represents fractional
            Brownian motion.

        %
% -----------------------------------------------------------------------------
%
% A license is hereby granted to reproduce this software source code and
% to create executable versions from this source code for personal,
% non-commercial use.  The copyright notice included with the software
% must be maintained in all copies produced.
%
% THIS PROGRAM IS PROVIDED "AS IS". THE AUTHOR PROVIDES NO WARRANTIES
% WHATSOEVER, EXPRESSED OR IMPLIED, INCLUDING WARRANTIES OF
% MERCHANTABILITY, TITLE, OR FITNESS FOR ANY PARTICULAR PURPOSE.  THE
% AUTHOR DOES NOT WARRANT THAT USE OF THIS PROGRAM DOES NOT INFRINGE THE
% INTELLECTUAL PROPERTY RIGHTS OF ANY THIRD PARTY IN ANY COUNTRY.
%
% Copyright (c) 1994-2006, John Conover, All Rights Reserved.
%
% Comments and/or bug reports should be addressed to:
%
%     john@email.johncon.com (John Conover)
%
% -----------------------------------------------------------------------------
%
% Revision: \RCSRevision \\
% Revision Time: \RCSTime UMT \\
% Revision Date: \RCSDate \\
% Revision Id: \RCSId \\
% Revision File: \RCSLog \\
\RCS $Revision: 0.0 $
\RCS $Date: 2006/01/20 04:38:13 $
\RCS $Id: instant.tex,v 0.0 2006/01/20 04:38:13 john Exp $
% $Log: instant.tex,v $
% Revision 0.0  2006/01/20 04:38:13  john
% Initial version
%
%
    \subsection{Instantaneous Analysis of Normalized Increments}
        \label{\SETLABEL:IA}

        \subidx{\market}{instantaneous analysis of normalized increments}
        \idx{average of normalized increments}
        \idx{root mean square of normalized increments}
        \subidx{Shannon probability}{instantaneous computation of}
        \subidx{average of normalized increments}{instantaneous computation of}
        \subidx{root mean square of normalized increments}{instantaneous computation of}
        \subidx{instantaneous computation}{Shannon probability}
        \subidx{instantaneous computation}{average of normalized increments}
        \subidx{instantaneous computation}{root mean square of normalized increments}
        \idx{time series}
        \subidx{time series}{instantaneous analysis}
        \subidx{instantaneous analysis}{time series}
        \subidx{time series}{increments}
        \subidx{time series}{analysis}
        \subidx{Shannon}{probability}
        \subidx{probability}{Shannon}
        \subidx{normalized}{increments}
        \subidx{increments}{normalized}

        The program {\it tsinstant}\/, which is briefly described in
        Appendix~\ref{programs}, is for finding the instantaneous
        fraction of change in a time series. The value of a sample in
        the time series is subtracted from the previous sample in the
        time series, and divided by the value of the previous sample.
        As explained in Chapter~\ref{general},
        Sections~\ref{derivation},~\ref{GA},~\ref{abmfi},~\ref{aftsma}
        and,~\ref{ompl} for Brownian motion, random walk fractals, the
        absolute value of the instantaneous fraction of change is also
        the root mean square of the instantaneous fraction of
        change\footnote{The absolute value of the normalized
        increments, when averaged, is related to the root mean square
        of the increments by a constant. If the normalized increments
        are a fixed increment, the constant is unity. If the
        normalized increments have a Gaussian distribution, the
        constant is $\approx 0.8$ depending on the accuracy of of
        ``fit'' to a Gaussian distribution.}. Squaring this value is
        the average of the instantaneous fraction of change, and
        adding unity to the absolute value of the instantaneous
        fraction of change, and dividing by two, is the Shannon
        probability of the instantaneous fraction of change.

        Figure~\ref{\SETLABEL:IA1} is the instantaneous value of the
        root mean square of the normalized increments for the
        {\market}, and Figure~\ref{\SETLABEL:IA2} is the instantaneous
        Shannon probability for the normalized increments.

        \begin{figure}[ht]
            \begin{center}
                \begin{minipage}[t]{0.45\textwidth}
                    \epsfxsize=1.0\linewidth
                    \epsffile{\directory/data.tsinstant-r.eps}
                    \caption[{\market}, instantaneous value of
                        rms.]{{\market}, instantaneous value of the
                        root mean square of the normalized increments,
                        provided by running the program {\it
                        tsinstant}\/ with the -r option on the data
                        presented in Figure~\ref{\SETLABEL:TS}.}
                    \label{\SETLABEL:IA1}
                    \label{\SETLABELQ:IA1}
                \end{minipage}
                \hfill
                \begin{minipage}[t]{0.45\textwidth}
                    \epsfxsize=1.0\linewidth
                    \epsffile{\directory/data.tsinstant-s.eps}
                    \caption[{\market}, instantaneous value of
                        Shannon probability.]{{\market}, instantaneous
                        value of the Shannon probability of the
                        normalized increments, provided by running the
                        program {\it tsinstant}\/ with the -s option
                        on the data presented in
                        Figure~\ref{\SETLABEL:TS}.}
                    \label{\SETLABEL:IA2}
                    \label{\SETLABELQ:IA2}
                \end{minipage}
            \end{center}
        \end{figure}

% Local Variables:
% TeX-parse-self: t
% TeX-auto-save: t
% TeX-master: "fractal.tex"
% End:


        %
% -----------------------------------------------------------------------------
%
% A license is hereby granted to reproduce this software source code and
% to create executable versions from this source code for personal,
% non-commercial use.  The copyright notice included with the software
% must be maintained in all copies produced.
%
% THIS PROGRAM IS PROVIDED "AS IS". THE AUTHOR PROVIDES NO WARRANTIES
% WHATSOEVER, EXPRESSED OR IMPLIED, INCLUDING WARRANTIES OF
% MERCHANTABILITY, TITLE, OR FITNESS FOR ANY PARTICULAR PURPOSE.  THE
% AUTHOR DOES NOT WARRANT THAT USE OF THIS PROGRAM DOES NOT INFRINGE THE
% INTELLECTUAL PROPERTY RIGHTS OF ANY THIRD PARTY IN ANY COUNTRY.
%
% Copyright (c) 1994-2006, John Conover, All Rights Reserved.
%
% Comments and/or bug reports should be addressed to:
%
%     john@email.johncon.com (John Conover)
%
% -----------------------------------------------------------------------------
%
% Revision: \RCSRevision \\
% Revision Time: \RCSTime UMT \\
% Revision Date: \RCSDate \\
% Revision Id: \RCSId \\
% Revision File: \RCSLog \\
\RCS $Revision: 0.0 $
\RCS $Date: 2006/01/20 04:38:13 $
\RCS $Id: logistic.tex,v 0.0 2006/01/20 04:38:13 john Exp $
% $Log: logistic.tex,v $
% Revision 0.0  2006/01/20 04:38:13  john
% Initial version
%
%
    \subsection{Logistic Analysis}
        \label{\SETLABEL:LA}

        \subidx{\market}{Logistic function analysis}
        \subidx{time series}{logistic function}
        \subidx{logistic function}{time series}
        \subidx{time series}{increments}
        \subidx{time series}{analysis}
        \subidx{cumulative sum}{analysis}
        \subidx{analysis}{cumulative sum}
        \subidx{analysis}{random process}
        \subidx{random process}{analysis}
        The data in this section is presented in tabular form in
        Section~\ref{\SETLABELREF:LAA}.  Figure~\ref{\SETLABEL:LA1} is
        a graph of the logistic function estimates of the time series
        data for the {\market}. The reader is cautioned that these
        graphs are constructed using the method suggested in
        Chapter~\ref{general}, Section~\ref{nlextend} and enormous
        precision is required for adequate prediction of the logistic
        function,~\cite{Modis}. Particularly, the non-linear term will
        usually require intervention to produce a practical fit to the
        data. In addition, there are numerical stability issues with
        logistic function methodologies\footnote{For example, in
        Figures~\ref{\SETLABEL:LA1} and~\ref{\SETLABEL:LA2}, if the
        non-linear term, $b$, was greater than zero, it was set to
        zero to produce the graphs. See Section~\ref{\SETLABELREF:LAA}
        for the actual derived values. In other cases, the magnitude
        of $b$ was too large, resulting in a graph that was decreasing
        as a function of time}.  The methodology should be regarded as
        ``fragile.'' It is included for completeness.

        \idx{least squares approximation}
        Figure~\ref{\SETLABEL:LA1} is a graph of the logistic function
        for the time series data presented in
        Figure~\ref{\SETLABEL:TS}. The data presented was made by
        running the program {\it tsdlogistic}\/, which is described
        briefly in Appendix~\ref{programs}, on the parameters
        extracted from the time series data as suggested in
        Figure~\ref{\SETLABEL:TF}. The program {\it tslsq}\/ was used
        to derive the constant and the slope of the normalized
        increments of the data presented in Figure~\ref{\SETLABEL:TF}.
        Figure~\ref{\SETLABEL:LA2} is the same graph, but with the
        time scale expanded by a factor of two.

        \begin{figure}[ht]
            \begin{center}
                \begin{minipage}[t]{0.45\textwidth}
                    \epsfxsize=1.0\linewidth
                    \epsffile{\directory/data.tsfraction.tslsq-p.tsdlogistic.eps}
                    \caption[{\market}, logistic function
                        estimates.]{{\market}, logistic function
                        estimates, provided by running the {\it
                        tslsq}\/ program on the normalized increments
                        presented in Figure~\ref{\SETLABEL:TF} with
                        the -p option. These parameters were used as
                        arguments to the {\it tsdlogistic}\/ program.}
                    \label{\SETLABEL:LA1}
                    \label{\SETLABELQ:LA1}
                \end{minipage}
                \hfill
                \begin{minipage}[t]{0.45\textwidth}
                    \epsfxsize=1.0\linewidth
                    \epsffile{\directory/data.tsfraction.tslsq-p.tsdlogistic2.eps}
                    \caption[{\market}, logistic function
                        estimates.]{{\market}, logistic function
                        estimates of Figure~\ref{\SETLABEL:LA1} with
                        the time scale expanded by a factor of two.}
                    \label{\SETLABEL:LA2}
                    \label{\SETLABELQ:LA2}
                \end{minipage}
            \end{center}
        \end{figure}

% Local Variables:
% TeX-parse-self: t
% TeX-auto-save: t
% TeX-master: "fractal.tex"
% End:


        %
% -----------------------------------------------------------------------------
%
% A license is hereby granted to reproduce this software source code and
% to create executable versions from this source code for personal,
% non-commercial use.  The copyright notice included with the software
% must be maintained in all copies produced.
%
% THIS PROGRAM IS PROVIDED "AS IS". THE AUTHOR PROVIDES NO WARRANTIES
% WHATSOEVER, EXPRESSED OR IMPLIED, INCLUDING WARRANTIES OF
% MERCHANTABILITY, TITLE, OR FITNESS FOR ANY PARTICULAR PURPOSE.  THE
% AUTHOR DOES NOT WARRANT THAT USE OF THIS PROGRAM DOES NOT INFRINGE THE
% INTELLECTUAL PROPERTY RIGHTS OF ANY THIRD PARTY IN ANY COUNTRY.
%
% Copyright (c) 1994-2006, John Conover, All Rights Reserved.
%
% Comments and/or bug reports should be addressed to:
%
%     john@email.johncon.com (John Conover)
%
% -----------------------------------------------------------------------------
%
% Revision: \RCSRevision \\
% Revision Time: \RCSTime UMT \\
% Revision Date: \RCSDate \\
% Revision Id: \RCSId \\
% Revision File: \RCSLog \\
\RCS $Revision: 0.0 $
\RCS $Date: 2006/01/20 04:38:13 $
\RCS $Id: hurst.tex,v 0.0 2006/01/20 04:38:13 john Exp $
% $Log: hurst.tex,v $
% Revision 0.0  2006/01/20 04:38:13  john
% Initial version
%
%
    \subsection{Hurst Coefficient Analysis}
        \label{\SETLABEL:H}

        \subidx{\market}{Hurst coefficient analysis}
        \subidx{Hurst coefficient}{analysis}
        \subidx{increments}{normalized}
        \subidx{normalized}{increments}
        \subidx{programs}{tshurst}
        \subidx{tshurst}{program}
        The data in this section is presented in tabular form in
        Section~\ref{\SETLABELREF:HCHP}. Figure~\ref{\SETLABEL:HC} is
        a graph of the Hurst coefficient data time series data shown
        in Figure~\ref{\SETLABEL:TS}. The slope of the graph is the
        Hurst coefficient.  The data for this figure was produced by
        the program {\it tshurst}\/, which is described briefly in
        Appendix~\ref{programs}.

        \subidx{\market}{H parameter analysis}
        \subidx{H parameter}{analysis}
        \subidx{programs}{tshcalc}
        \subidx{tshcalc}{program}
        Figure~\ref{\SETLABEL:HP} is a graph of the H parameter data
        for the normalized increments of the time series data shown in
        Figure~\ref{\SETLABEL:TF}. The data for this figure was
        produced by the program {\it tshcalc}\/, which is described
        briefly in Appendix~\ref{programs}.

        \begin{figure}[ht]
            \begin{center}
                \begin{minipage}[t]{0.45\textwidth}
                    \epsfxsize=1.0\linewidth
                    \epsffile{\directory/data.tshurst.eps}
                    \caption[{\market}, Hurst coefficient data]{{\market},
                        Hurst coefficient data for the normalized
                        increments of the time series data shown in
                        Figure~\ref{\SETLABEL:TF}.  The slope of the graph
                        is the Hurst coefficient.}
                    \label{\SETLABEL:HC}
                \end{minipage}
                \hfill
                \begin{minipage}[t]{0.45\textwidth}
                    \epsfxsize=1.0\linewidth
                    \epsffile{\directory/data.tshcalc.eps}
                    \caption[{\market}, H parameter data]{{\market}, H
                        parameter data for the normalized increments of
                        the time series data shown in
                        Figure~\ref{\SETLABEL:TF} The slope of the graph
                        is the H parameter.}
                    \label{\SETLABEL:HP}
                \end{minipage}
            \end{center}
        \end{figure}

        \subidx{revenue}{See, rate of revenue returns}
        \subidx{returns}{See, rate of revenue returns}
        \subidx{\market}{revenues}
        \subidx{Hurst coefficient}{analysis}
        \subidx{\market}{Hurst coefficient analysis}
        \subidx{\market}{rate of change}
        \subidx{\market}{windows of opportunity}
        \subidx{rate of revenue returns}{forecast}
        \subidx{forecast}{rate of revenue returns}
        \idx{windows of opportunity}
        \subidx{programs}{tslsq}
        \subidx{tslsq}{program}

        The approximately linear slope of the graph in
        Figure~\ref{\SETLABEL:HC} implies that the variance of the
        rate of revenue returns, (per {\timescale},) in the {\market},
        $V(t_2 - t_1)$, over a period of time is proportional to the
        period of time raised to twice the Hurst
        coefficient~\cite[pp. 180]{Feder},~\cite[pp. 246]{Crownover}.
        This seems to be a quantitative statement concerning how fast,
        and to what degree, the rate of revenue returns' state of
        affairs can change over a period of time.  An additional
        implication, for Hurst coefficients sufficiently close to 0.5,
        is that the probability of the state of affairs repeating
        sometime in the future goes down with increasing
        time\footnote{It can be shown that the number of expected
        market ``high'' and ``low'' transitions, $N$, scales with the
        square root of time, or $N \propto \sqrt {t}$, meaning that
        the cumulative distribution of the probability, $P$, of the
        duration of a market's ``high'' or ``low'' exceeding a given
        time interval, $t$, is proportional to the reciprocal of the
        square root of the time interval, $P \propto 1 / \sqrt {t}$,
        (or, conversely, that the probability of the duration of a
        market's ``high'' or ``low'' exceeding a given time interval
        is proportional to the reciprocal of the time interval raised
        to the power $3 / 2$, ie., $P \propto 1 / t^{3 /
        2}$,~\cite[pp. 153]{Schroeder}. What this means is that a
        histogram of the ``zero free'' run-lengths of a market being
        ``high'' or ``low,'' over a long time, would have a $1 / t^{3
        / 2}$ characteristic.)}, $t$, $p(t) = erf (1/\sqrt{2t})$ which
        is approximately $1/\sqrt{t}$ for $t \gg
        1$~\cite[pp. 160]{Schroeder}. Figures~\ref{\SETLABEL:FN},
        and,~\ref{\SETLABEL:FF} compare methods of approximation of
        the ``forecastability'' of the rate of revenue returns in the
        {\market} for the near term and far term,
        respectively~\cite[pp. 83-84]{Peters:CAOITCM}\footnote{The
        author is not comfortable with Peters' interpretation. For
        example, if the algorithm explained
        in~\cite[pp. 82]{Peters:CAOITCM} is used on ``white noise''
        which, by definition, never has any correlations, the short
        term Hurst coefficient, and thus the ``forecastability,'' is
        still near unity---a bit of an enigma. This can be verified
        with the {\it tswhite}\/ and {\it tshurst}\/ programs, which
        are briefly described in Appendix~\ref{programs}.}.  This
        seems to be a quantitative statement concerning ``windows of
        opportunity'' in the rate of revenue returns, (per
        {\timescale}.)  The program {\it tslsq}\/ was used on the
        Hurst coefficient data, presented in
        Figure~\ref{\SETLABEL:HC}, to provide a least squares
        approximation to the Hurst coefficient. The superimposed least
        squares approximation with on original Hurst coefficient data
        is presented.  The time series data has a Hurst coefficient of
        {\thurstlow}, so that:

        \subidx{\market}{Hurst coefficient analysis}
        \begin{eqnarray}
            V\left(t_2 - t_1\right) & \propto & \left(t_2 - t_1\right)^{2 \cdot H}\\
            V\left(t_2 - t_1\right) & \propto & \left(t_2 - t_1\right)^{2 \cdot {\thurstlow}}\\
                                    & \propto & \left(t_2 - t_1\right)^{\thurstlowtwo}
            \label{\SETLABEL:V}
        \end{eqnarray}

        \subidx{fractional}{Brownian motion}
        \subidx{Brownian motion}{fractional}
        \idx{fractal}
        \noindent where $V(t_2 - t_1)$ is the variance of the
        increments of the rate of revenue returns, (per {\timescale},)
        over the time interval $t_2 -
        t_1$,~\cite[pp. 177]{Feder},~\cite[pp. 494]{Peitgen}. If $H >
        \frac{1}{2}$, then the time series is termed as being
        characterized by ``fractional Brownian
        motion~\cite[pp. 170]{Feder}.''

        \subidx{rate of revenue returns}{predictability}
        \subidx{rate of revenue returns}{forecastability}
        \subidx{rate of revenue returns}{consistency}
        \subidx{predictability}{rate of revenue returns}
        \subidx{forecastability}{rate of revenue returns}
        \subidx{consistency}{rate of revenue returns}
        \subidx{\market}{rate of revenue returns, predictability}
        \subidx{\market}{rate of revenue returns, forecastability}
        \subidx{\market}{rate of revenue returns, consistency}
        \subidx{Hurst coefficient}{analysis}
        \subidx{\market}{Hurst coefficient analysis}
        \subidx{\market}{rate of change}

        In some sense, the Hurst coefficient is a quantitative
        expression of the ``forecastability'' of the future based on
        the past\footnote{Actually, in general, when summing fractal
        entities, the method used should be a root mean square
        process, dependent on the Hurst Coefficient, $H$, where
        $P_{total}^H = P_1^H + P_2^H + \cdots$, where $P_n$ is the
        fractal entities. For a Brownian motion, or random walk type
        of fractal the Hurst Coefficient is a function of time into
        the future. For the ``near term,'' the Hurst coefficient is
        very near unity, meaning the summation process is linear. For
        the ``long term,'' $H \approx 0.5$, or a standard root mean
        square summation process should be used. If $H$ is $0.5$ then
        the market is termed a Brownian motion, or random walk
        process. If it is larger than 0.5, it is termed fractional
        Brownian motion process. For a random walk process, ``near
        term'' and ``far term'' are quantitatively differentiated on
        the Hurst Coefficient graph where $1 - \ln (t) = 0.5 \cdot \ln
        (t)$, or when $\ln (t) = 2$, or $t = 7.389\ldots$ See
        Section~\ref{\SETLABEL:FS} for the particulars on using Hurst
        Coefficient to sum fractal process' for the {\market}. See
        also~\cite[pp. 67, 83-84]{Peters:CAOITCM} and~\cite[pp. 129,
        159]{Schroeder} for particulars on the implications of the
        Hurst Coefficient and root mean square summation issues.}.  A
        Hurst coefficient of {\thurstlow}, (for the near future, and
        {\thurstall} for the distant future.) implies that the
        likelihood of the rate of revenue returns, (per {\timescale},)
        for any two consecutive {\timescale}s being the same is
        {\thurstlowhundred}\%~\cite[pp. 66]{Peters:CAOITCM} for the
        near future, and {\thurstall} for the distant
        future. Likewise, there is a {\thurstlowhundred}\% chance of
        the rate of revenue returns, (per {\timescale},) movements
        being the same in consecutive time periods---ie., if, in a
        given {\timescale}, the rate of revenue returns, (per
        {\timescale},) is increasing, there is a {\thurstlowhundred}\%
        that the rate of revenue returns, (per {\timescale},) will
        increase in the following period, also. In some sense, this is
        a quantitative statement on how ``predictable,'' or
        ``forecastable'' the rate of revenue returns, (per
        {\timescale},) for the {\market} are over time, since the
        probability of having $n$ many consecutive {\timescale}s of
        the same agenda is $H^n$ where $H$ is the Hurst coefficient,
        or, letting the short term probability of having $n$ many
        {\timescale}s of the same market agenda, $p_a$, is:

        \begin{eqnarray}
            p_a\left(n\right) & = & H^{n}\\
                              & = & {\thurstlow}^{n}
            \label{\SETLABEL:MA}
        \end{eqnarray}

        \subidx{rate of revenue returns}{predictability}
        \subidx{rate of revenue returns}{forecastability}
        \subidx{rate of revenue returns}{consistency}
        \subidx{predictability}{rate of revenue returns}
        \subidx{forecastability}{rate of revenue returns}
        \subidx{consistency}{rate of revenue returns}
        As an interesting interpretation of the normalized increments
        of the time series data presented in
        Figure~\ref{\SETLABEL:TF}, if the vertical axis is multiplied
        by 100, to convert to percent, then the graph represents the
        error, in percent, that would be made by forecasting, month by
        month, that the next {\timescale}'s rate of revenue returns
        would be the same as the current {\timescale}'s revenue
        rate. Interestingly, it is $\datafractionmean \cdot 100$
        percent, on the average, with a standard deviation of
        $\datafractionstddev \cdot 100$ percent, and a root mean
        square error value of $\datafractionrms \cdot 100$
        percent---small values for such a simple forecasting
        mechanism.

        \subidx{\market}{rate of revenue returns, range}
        \subidx{Hurst coefficient}{analysis}
        \subidx{\market}{Hurst coefficient analysis}
        \subidx{\market}{rate of change}

        This is, essentially, a statement of the range of values, in
        the increments of the rate of revenue returns, (per
        {\timescale},) that is to be expected over the time interval,
        $t_2 - t_1$,
        $R_v$,~\cite[pp. 178]{Feder},~\cite[pp. 172]{Cambel}:

        \begin{eqnarray}
            R_v\left(t_2 - t_1\right) & \propto & \left(t_2 - t_1\right)^{H}\\
                                      & \propto & \left(t_2 - t_1\right)^{\thurstlow}
            \label{\SETLABEL:R}
        \end{eqnarray}

        \subidx{\market}{rate of revenue returns, range}
        \subidx{Hurst coefficient}{analysis}
        \subidx{\market}{Hurst coefficient analysis}
        \subidx{\market}{rate of change}
        \subidx{Markov}{statistics}
        \subidx{statistics}{Markov}
        \noindent where $R$ is the range of values in the increments
        of the rate of revenue returns, (per {\timescale}.) A Hurst
        coefficient, $H$, that is much larger than $\frac{1}{2}$, (but
        less than 1,) implies a strongly non-Gaussian distribution in
        the increments of the rate of revenue returns, (per
        {\timescale},)~\cite[pp. 152, 194]{Feder}, and a Hurst
        coefficient near $\frac{1}{2}$ implies that the increments of
        the rate of revenue returns, (per {\timescale}) is
        characteristic of an independent
        process~\cite[pp. 195]{Feder}. Extreme caution should be
        exercised in using Markov statistics in any analysis where the
        Hurst coefficient is not
        $\frac{1}{2}$,~\cite[pp. 124]{Crownover},~\cite[pp. 106]{Peters:CAOITCM}.


        As a useful approximation, if $H$, is approximately
        $\frac{1}{2}$, Equation~\ref{\SETLABEL:R} reduces
        to,~\cite[pp. 129]{Schroeder}:

        \begin{eqnarray}
            R\left(t_2 - t_1\right) & \propto & (t_2 - t_1)^{\frac{1}{2}}\\
                                    & \propto & \sqrt{\left(t_2 - t_1\right)}
        \end{eqnarray}

        \subidx{\market}{rate of revenue returns, range}
        \subidx{\market}{rate of revenue returns, increase and decrease}
        \subidx{Hurst coefficient}{analysis}
        \subidx{\market}{Hurst coefficient analysis}
        \subidx{\market}{rate of change}
        \subidx{Markov}{statistics}
        \subidx{statistics}{Markov}

        In the case where the Hurst coefficient, $H$, is
        $\frac{1}{2}$, the range of values in the increments of the
        rate of revenue returns, (per {\timescale},) divided by the
        standard deviation of these values, $S$, can be anticipated to
        increase over time according to the following
        relation,~\cite[pp. 154]{Feder},~\cite[pp. 129]{Schroeder}:

        \begin{equation}
            \frac{R\left(t_2 - t_1\right)}{S} \propto \left(t_2 - t_1\right)^{\frac{1}{2}}
        \end{equation}

        \subidx{\market}{rate of revenue returns, range}
        \subidx{\market}{rate of revenue returns, increase and decrease}
        \subidx{Hurst coefficient}{analysis}
        \subidx{\market}{Hurst coefficient analysis}
        \subidx{\market}{rate of change}
        \noindent which is a useful conceptual approximation, since it
        involves only the square root function---if the range and the
        standard deviation of the increments of the rate of revenue
        returns, (per {\timescale},) are known, (and $H \approx
        \frac{1}{2}$,) then the expected change in $\frac{R}{S}$, will
        increase with the square root of time\footnote{To be precise,
        it is actually asymptotically proportional to
        $\tau^{\frac{1}{2}}$}.

        Another useful approximation when rescaling processes that are
        characterize by Brownian motion, (ie., when $H \approx
        \frac{1}{2}$,) is that:

        \begin{eqnarray}
            X\left(t\right) & \propto & \frac{X\left(rt\right)}{r^{H}}\\
                            & \propto & \frac{X\left(rt\right)}{r^{\thurstlow}}
        \end{eqnarray}

        \idx{Brownian motion}
        \idx{fractal}
        Where $X(t)$ is the process characterized by Brownian motion,
        and $r$ is a scaling factor,~\cite[pp. 494]{Peitgen}.

        \subidx{programs}{tslsq}
        \subidx{tslsq}{program}
        The program {\it tslsq}\/ was used on the H parameter data,
        presented in Figure~\ref{\SETLABEL:HP}, to provide a least
        squares approximation to the H parameter for the
        {\market}. The superimposed least squares approximation on the
        original H parameter data is presented.  By contrast, the H
        parameter, as derived by the methodology outlined
        in~\cite[pp. 249]{Crownover}, is {\thcalclow} for the near
        future, and {\thcalcall} for the distant future.

        \subidx{\market}{Hurst coefficient analysis}
        \subidx{Hurst coefficient}{analysis}
        \subidx{increments}{normalized}
        \subidx{normalized}{increments}
        \subidx{programs}{tshurst}
        \subidx{tshurst}{program}
        \subidx{\market}{H parameter analysis}
        \subidx{H parameter}{analysis}
        \subidx{programs}{tshcalc}
        \subidx{tshcalc}{program}
        Figures~\ref{\SETLABEL:HC} and~\ref{\SETLABEL:HP} represent
        Hurst coefficient and H parameter data that are derived from
        the normalized increments, shown in
        Figure~\ref{\SETLABEL:TF}. In this case, the data is
        considered a normalized derivative of the time series data
        presented in Figure~\ref{\SETLABEL:TF}, instead of a
        cumulative sum.  The program, {\it tshurst}\/, is described
        briefly in appendix~\ref{programs}, and the data for
        figures~\ref{\SETLABEL:THC} and~\ref{\SETLABEL:THP} was made
        using the -d option.

        \begin{figure}[ht]
            \begin{center}
                \begin{minipage}[t]{0.45\textwidth}
                    \epsfxsize=1.0\linewidth
                    \epsffile{\directory/data.tsfraction.tshurst-d.eps}
                    \caption[{\market}, traditional Hurst coefficient
                        data]{{\market}, traditional Hurst coefficient
                        data for the time series data shown in
                        Figure~\ref{\SETLABEL:TS}.  The slope of the
                        graph is the Hurst coefficient, and is
                        {\hurstlow} for the near term, and
                        {\hurstall} for the far term.}
                    \label{\SETLABEL:THC}
                \end{minipage}
                \hfill
                \begin{minipage}[t]{0.45\textwidth}
                    \epsfxsize=1.0\linewidth
                    \epsffile{\directory/data.tsfraction.tshcalc-d.eps}
                    \caption[{\market}, traditional H parameter
                        data]{{\market}, traditional H parameter data
                        for the time series data shown in
                        Figure~\ref{\SETLABEL:TS} The slope of the
                        graph is the H parameter, and is {\hcalclow}
                        for the near term, and {\hcalcall} for the
                        far term.}
                    \label{\SETLABEL:THP}
                \end{minipage}
            \end{center}
        \end{figure}

% Local Variables:
% TeX-parse-self: t
% TeX-auto-save: t
% TeX-master: "fractal.tex"
% End:


        \subsubsection{Observations on the Hurst Coefficient Analysis}

            Many {\market} industry analyst speculate that there is
            ``periodic'' behavior in the market place, at
            approximately 5 year intervals. Both the Hurst coefficient
            and H parameter graphs would tend to support the
            intuition. Notice that the slope of the graphs, in
            figures~\ref{\SETLABEL:HC} and~\ref{\SETLABEL:HP}, tend to
            decrease abruptly at $t \approx \ln(3) \approx 20$
            {\timescale}s, which is approximately 60 months, or 5
            years~\cite[pp. 96]{Peters:CAOITCM}. Whether this is
            ``periodic'' behavior, or an indication of more complex
            system dynamics, perhaps ``chaotic,'' remains to be
            seen. If that is the case, it could provide an exploitive
            venue.

        %
% -----------------------------------------------------------------------------
%
% A license is hereby granted to reproduce this software source code and
% to create executable versions from this source code for personal,
% non-commercial use.  The copyright notice included with the software
% must be maintained in all copies produced.
%
% THIS PROGRAM IS PROVIDED "AS IS". THE AUTHOR PROVIDES NO WARRANTIES
% WHATSOEVER, EXPRESSED OR IMPLIED, INCLUDING WARRANTIES OF
% MERCHANTABILITY, TITLE, OR FITNESS FOR ANY PARTICULAR PURPOSE.  THE
% AUTHOR DOES NOT WARRANT THAT USE OF THIS PROGRAM DOES NOT INFRINGE THE
% INTELLECTUAL PROPERTY RIGHTS OF ANY THIRD PARTY IN ANY COUNTRY.
%
% Copyright (c) 1994-2006, John Conover, All Rights Reserved.
%
% Comments and/or bug reports should be addressed to:
%
%     john@email.johncon.com (John Conover)
%
% -----------------------------------------------------------------------------
%
% Revision: \RCSRevision \\
% Revision Time: \RCSTime UMT \\
% Revision Date: \RCSDate \\
% Revision Id: \RCSId \\
% Revision File: \RCSLog \\
\RCS $Revision: 0.0 $
\RCS $Date: 2006/01/20 04:38:13 $
\RCS $Id: fiscal.tex,v 0.0 2006/01/20 04:38:13 john Exp $
% $Log: fiscal.tex,v $
% Revision 0.0  2006/01/20 04:38:13  john
% Initial version
%
%
    \subsection{Fixed Increment Approximation for Fiscal Strategy}
        \label{\SETLABEL:FS}

        \subidx{\market}{fiscal strategy}
        \subidx{markets}{analysis}
        \subidx{analysis}{markets}
        \subidx{strategy}{fiscal}
        \subidx{fiscal}{strategy}
        The data in this section is presented in tabular form in
        Section~\ref{\SETLABELREF:LR}. This section derives various
        values based on the ``average'' of the normalized increments
        presented in Figure~\ref{\SETLABEL:TFA}. These values are an
        approximation to a, probably, complex process with a
        distribution shown in Figure~\ref{\SETLABEL:TF}. These values
        will be used in a fixed increment Brownian fractal analysis
        and simulation of the {\market}, and may, or may not, provide
        adequate accuracy for projections.

        For an organization operating in the {\market}, the fiscal
        strategy, commensurate with the aggregate environment, can be
        derived as follows~\cite[pp. 128, pp
        151]{Schroeder},~\cite[pp. 450]{Reza},~\cite[pp. 270]{Pierce}:
        \vspace{0.15in}

        \subsubsection{Logarithmic Returns}
            \label{\SETLABEL:LR}

            \subidx{logarithmic}{returns}
            \subidx{returns}{logarithmic}
            \subidx{\market}{logarithmic returns}
            The logarithmic returns can be calculated by various
            means. Four will be presented here, for comparison.

            \subidx{programs}{tsnormal}
            \subidx{tsnormal}{program}
            \subidx{logarithmic}{returns}
            \subidx{returns}{logarithmic}
            The logarithmic returns, in bits, $bits$, as computed from
            the mean, by the program {\it tsnormal}\/, which is
            described in Chapter~\ref{programs}, and is presented in
            Figure~\ref{\SETLABEL:TF}, and Equation~\ref{abits} from
            Section~\ref{ereturns} in Chapter~\ref{general}:

            \begin{equation}
                bits = \frac{\ln \left({\datafractionmean} + 1\right)}{\ln \left(2\right)} = \datafractionmeanbits
            \end{equation}

            \subidx{programs}{tslsq}
            \subidx{tslsq}{program}
            \subidx{logarithmic}{returns}
            \subidx{returns}{logarithmic}
            \noindent By comparison, the logarithmic returns, in bits,
            $bits$, as computed from the constant in the least squares
            approximation, using the program {\it tslsq}\/, which is briefly
            described in Chapter~\ref{programs}, as presented in
            Figure~\ref{\SETLABEL:TF}, and Equation~\ref{abits} from
            Section~\ref{ereturns} in Chapter~\ref{general}:

            \begin{equation}
                bits = \frac{\ln \left({\datafractionconstant} + 1\right)}{\ln \left(2\right)} = \datafractionconstantbits
            \end{equation}

            Note that if the mean is not constant in
            Figure~\ref{\SETLABEL:TF}, this method will not provide
            accurate results.

            \subidx{programs}{tslsq}
            \subidx{tslsq}{program}
            \subidx{logarithmic}{returns}
            \subidx{returns}{logarithmic}
            \noindent And by yet another comparison, using the program
            {\it tslsq}\/, which is briefly described in
            Chapter~\ref{programs}, with the -e -p options, to provide
            a formula for the least squares exponential fit to the
            time series data set presented in
            Figure~\ref{\SETLABEL:TS}:

            \begin{equation}
                bits = {\datatslsqepbits}
            \end{equation}

            \subidx{programs}{tslogreturns}
            \subidx{tslogreturns}{program}
            \subidx{logarithmic}{returns}
            \subidx{returns}{logarithmic}
            \noindent And finally, by comparison, from the
            {\it tslogreturns}\/ program, which is briefly described
            in Chapter~\ref{programs}, with the -p option, to provide
            a formula for the logarithmic returns of the time series
            data set presented in Figure~\ref{\SETLABEL:TS}:

            \begin{equation}
                bits = {\logreturns}
            \end{equation}

        \subsubsection{Calculation of Shannon Probability}
            \label{\SETLABEL:SP}

            \subidx{\market}{Shannon probability}
            Ideally, all of the values presented in
            Section~\ref{\SETLABEL:LR} would be equal. Using the
            logarithmic returns provided by the {\it tslogreturns}\/
            program, to be consistent
            with~\cite[pp. 81]{Peters:CAOITCM}

            \subidx{programs}{tslogreturns}
            \subidx{tslogreturns}{program}
            \begin{equation}
                2^{{\logreturns}t}
            \end{equation}

            \noindent therefore:
            \begin{equation}
                C\left(p\right) = {\logreturns}
            \end{equation}
            \subidx{programs}{tsshannon}
            \subidx{tsshannon}{program}
            \subidx{Shannon}{probability}
            \subidx{probability}{Shannon}
            \noindent and, {\it tsshannon}\/ {\logreturns} gives:
            \begin{equation}
                \label{\SETLABEL:F0}
                C\left({\shannonlogreturns}\right) = {\logreturns}
            \end{equation}
            \noindent therefore:
            \begin{eqnarray}
                2^{C\left({\shannonlogreturns}\right)} & = & 2^{\logreturns}\\
                                                       & = & {\twologreturns}\\
                                                       & = & {\twologreturnshundred}\%
            \end{eqnarray}
            \noindent and:
            \begin{eqnarray}
                2p - 1 & = & \left(2 \cdot {\shannonlogreturns}\right) - 1\\
                       & = & {\twopone}\\
                       \label{\SETLABEL:F1}
                       & = & {\twoponehundred}\%
            \end{eqnarray}

            \subidx{\market}{fiscal strategy}
            \subidx{markets}{analysis}
            \subidx{analysis}{markets}
            \subidx{strategy}{fiscal}
            \subidx{fiscal}{strategy}
            \subidx{\market}{fiscal strategy}
            \subidx{\market}{growth rate}
            Presuming the simplified assumptions outlined in
            Section~\ref{assumptions}, the ``typical'' organization
            operating in the {\market} executes a long term fiscal
            strategy, commensurate with the aggregate environment,
            that is to invest, every {\timescale}, in sufficient
            additional resources and infrastructure, to increase the
            manufacturing of goods and services by {\twoponehundred}\%
            of its rate of revenue returns, (per {\timescale}.) As a
            conceptual model, the remaining {\hundredtwoponehundred}\%
            will be held in ``reserve'' with a
            {\shannonlogreturnshundred}\% chance of making twice the
            {\twoponehundred}\% back, (and a
            {\hundredshannonlogreturnshundred}\% chance of making
            0.0,) in one {\timescale}, on the average, for an average
            growth in its rate of revenue returns, (per {\timescale},)
            of {\twologreturnshundred}\%, or a doubling of its rate of
            revenue returns, (per {\timescale},) in
            {\oneoverlogreturns} {\timescale}s.

        \subsubsection{Example Fixed Increment Approximation Fiscal Strategies}

            \subidx{\market}{fiscal strategy}
            \subidx{markets}{analysis}
            \subidx{analysis}{markets}
            \subidx{strategy}{fiscal}
            \subidx{fiscal}{strategy}
            \subidx{\market}{fiscal strategy}
            \subidx{\market}{growth rate}
            \subidx{\market}{management metric}
            \idx{management metric}
            A possible metric on the effectiveness of long term fiscal
            management could possibly be that if an investment of
            {\twoponehundred}\% per {\timescale} of the rate of
            revenue returns, (per {\timescale},) is made in resources
            and infrastructure, then the rate of revenue returns would
            be expected to increase by {\twologreturnshundred}\%, per
            {\timescale}, on average.

            Note that the metrics presented in this section are
            representative of the {\market} as an aggregate whole, and
            may or may not be accurate representations for any
            particular participant in the environment. Of interest to
            the participants in the environment would be a similar
            analysis of each product or service rendered in the
            marketplace.

            \subidx{\market}{fiscal strategy}
            \subidx{markets}{analysis}
            \subidx{analysis}{markets}
            \subidx{strategy}{fiscal}
            \subidx{fiscal}{strategy}
            \subidx{\market}{fiscal strategy}
            As a simple illustrative example, a company operating in
            this environment might obtain a credit line from a bank
            that is equal to {\twoponehundred}\% of its rate of
            revenue returns, (per {\timescale},) to finance additional
            operations. In this simple scenario, the company would use
            its revenue base as collateral for the loan. Some
            {\timescale}s, depending on the {\market}'s environment,
            the company's rate of revenue returns exceeds what was
            borrowed from the bank, and the loan is repaid in
            full. Other {\timescale}s, the company must default, and
            the bank seizes a portion of the company's revenue base to
            pay the delinquent loan. However, on the average, the
            company will expand its rate of revenue returns at
            {\twologreturnshundred}\% per {\timescale}.

            \subidx{\market}{fiscal strategy}
            \subidx{markets}{analysis}
            \subidx{analysis}{markets}
            \subidx{strategy}{fiscal}
            \subidx{fiscal}{strategy}
            \subidx{\market}{fiscal strategy}
            As another simple example, a company re-invests
            {\twoponehundred}\% of its rate of revenue returns, (per
            {\timescale},) in development, marketing, sales, and
            distribution of new products.  Although some products will
            be successful and the return on the investment will exceed
            the {\twoponehundred}\% per {\timescale} investment,
            others will not. However, on the average, the company will
            expand it gross rate of revenue returns at
            {\twologreturnshundred}\% per {\timescale}.

            \subidx{\market}{fiscal strategy}
            \subidx{markets}{analysis}
            \subidx{analysis}{markets}
            \subidx{strategy}{fiscal}
            \subidx{fiscal}{strategy}
            \subidx{\market}{fiscal strategy}
            \subidx{\market}{product portfolio}
            \subidx{\market}{product diversity}
            \subidx{\market}{product mix}
            \subidx{\market}{optimum number of products}
            \idx{product portfolio}
            \idx{product diversity}
            \idx{optimum number of products}
            \idx{product mix}

            As an example of ``product portfolio'' management, suppose
            a company re-invests {\twoponehundred}\% of its rate of
            revenue returns, (per {\timescale},) in development,
            marketing, sales, and distribution of new products.
            Further suppose that the company has two products, and a
            fractal analysis of the individual product rate of revenue
            return time series indicates that one product has a
            Shannon probability of 0.65, and the other has a Shannon
            probability of 0.55. Then the percentage of re-investment
            in the first product would be $(2 \cdot 0.65 - 1) \cdot
            {\twoponehundred}$, percent of the rate of revenue
            returns, and $(2 \cdot 0.55 - 1) \cdot {\twoponehundred}$
            percent for the second product, implying that the company
            should diversify its product line\footnote{The astute
            reader would note that the linear addition was used to add
            the contribution to development of each product. This is a
            ``near term'' interpretation. Actually, in general, the
            method used should be a root mean square process,
            dependent on the Hurst Coefficient, $H$, where
            $P_{total}^H = P_1^H + P_2^H + \cdots$, where $P_n$ is the
            contribution to each individual product. For a Brownian
            motion, or random walk type of fractal the Hurst
            Coefficient is a function of time into the future. For the
            ``near term,'' the Hurst coefficient is very near unity,
            meaning the summation process is linear. For the ``long
            term,'' $H \approx 0.5$, or a standard root mean square
            summation process should be used. If $H$ is $0.5$ then the
            market is termed a Brownian motion, or random walk
            process. If it is larger than 0.5, it is termed fractional
            Brownian motion process. For a random walk process, ``near
            term'' and ``far term'' are quantitatively differentiated
            on the Hurst Coefficient graph where $1 - \ln (t) = 0.5
            \cdot \ln (t)$, or when $\ln (t) = 2$, or $t =
            7.389\ldots$ See~\cite[pp. 67, 83-84]{Peters:CAOITCM}
            and~\cite[pp. 129, 159]{Schroeder} for particulars on the
            implications of the Hurst Coefficient and root mean square
            summation issues.}.  Note that this is a ``bet hedging''
            metric methodology, and assumes that the products have
            uncorrelated revenue return rates. If this re-investment
            methodology is not feasible, perhaps for strategic
            financial reasons, then the re-investment in both products
            should total the ${\twoponehundred}$\%, and the investment
            in each product should be made at a ratio of $\frac{(2
            \cdot 0.65 - 1)}{(2 \cdot 0.55 - 1)} = 3 : 1$,
            respectively. Note that this ``bet hedging'' can be used
            to define the optimal number of products that can be
            supported on the rate of revenue returns. If it assumed
            that all products are ``typical'' for the {\market}, as a
            standard bench mark, then the optimal number will be
            $\frac{1}{{\twopone}}$. Note that this is a
            ``theoretical'' value, since not all products are
            ``typical,'' and there may be strategic reasons, for
            example product leveraging, that may increase the number
            of products above the optimum. However, most of the
            revenue should come from the optimal number of products,
            since having more products will decrease the amount of the
            potential investment in each product, and having less than
            the optimum number of products will increase the risk that
            many of the products could suffer a ``down market''
            concurrently, impacting the rate of revenue returns.  As
            another interesting interpretation of the optimal
            ``hedging of bets,'' in product portfolio strategy, and
            considering the graph of the normalized increments
            presented in Figure~\ref{\SETLABEL:TF}, if the
            organization is running optimally, then these products
            will generate, at least in principle, one standard
            deviation, approximately $0.8413 = 84.13$\% of the future
            growth in rate of revenue returns. Naturally, these are
            approximations, and the values are an approximation to a,
            probably, complex process, and appropriate scrutiny should
            be exercised before making specific projections.  As yet
            another example of ``product portfolio'' management,
            consider the issue of product mix. In this interpretation,
            {\twoponehundred}\% of the product manufactured should be
            ``proprietary,'' while the rest is ``industry standard.''
            As yet another possibility, {\twoponehundred}\% of the
            product manufactured should be predatory into new markets,
            and the remainder in markets that are ``traditional'' for
            the company.

% Local Variables:
% TeX-parse-self: t
% TeX-auto-save: t
% TeX-master: "fractal.tex"
% End:


        \subsubsection{Observations on the Fixed Increment Approximation for Fiscal Strategy}

            A re-investment of {\twoponehundred} of the rate of
            revenue returns per {\timescale} does not seem
            inconsistent with the industry averages, since it includes
            investments in research and development, additional
            manufacturing infrastructure, advertising,
            etc. Additionally, a product mix of {\twoponehundred}\%
            ``proprietary'' and the remainder ``industry standard''
            products seems consistent with the industry analyst
            ``20/80'' rule. The value of one standard deviation,
            $84.13$\%, of the revenue return rate being generated by
            $\frac{1}{{\twopone}}$ products seems consistent with the
            industry, also.

        %
% -----------------------------------------------------------------------------
%
% A license is hereby granted to reproduce this software source code and
% to create executable versions from this source code for personal,
% non-commercial use.  The copyright notice included with the software
% must be maintained in all copies produced.
%
% THIS PROGRAM IS PROVIDED "AS IS". THE AUTHOR PROVIDES NO WARRANTIES
% WHATSOEVER, EXPRESSED OR IMPLIED, INCLUDING WARRANTIES OF
% MERCHANTABILITY, TITLE, OR FITNESS FOR ANY PARTICULAR PURPOSE.  THE
% AUTHOR DOES NOT WARRANT THAT USE OF THIS PROGRAM DOES NOT INFRINGE THE
% INTELLECTUAL PROPERTY RIGHTS OF ANY THIRD PARTY IN ANY COUNTRY.
%
% Copyright (c) 1994-2006, John Conover, All Rights Reserved.
%
% Comments and/or bug reports should be addressed to:
%
%     john@email.johncon.com (John Conover)
%
% -----------------------------------------------------------------------------
%
% Revision: \RCSRevision \\
% Revision Time: \RCSTime UMT \\
% Revision Date: \RCSDate \\
% Revision Id: \RCSId \\
% Revision File: \RCSLog \\
\RCS $Revision: 0.0 $
\RCS $Date: 2006/01/20 04:38:13 $
\RCS $Id: companies.tex,v 0.0 2006/01/20 04:38:13 john Exp $
% $Log: companies.tex,v $
% Revision 0.0  2006/01/20 04:38:13  john
% Initial version
%
%
    \subsection{Number of Companies}
        \label{\SETLABEL:QNC}

        \subidx{\market}{number of companies}
        \subidx{number of companies}{analysis}
        \subidx{analysis}{number of companies}
        \subidx{Shannon}{probability}
        \subidx{probability}{Shannon}
        This section evaluates the approximate, or ``average,'' number
        of companies in the {\market}, and uses the method outlined in
        Chapter~\ref{general}, Section~\ref{aftsma}. Since the
        average, $avg_{ind}$, and the root mean square, $rms_{ind}$,
        of the normalized increments of the {\market} time series is
        \datafractionmean, and \datafractionrms respectively, the
        number of companies participating in the market can be
        calculated by Equation~\ref{ncompanies} to be {\ncompanies}.

        If this value seems consistent number of companies in the
        {\market}, within the assumptions outlined in
        Chapter~\ref{general}, Section~\ref{aftsma}, then it would
        seem that there is some circumstantial or indirect evidence
        that the companies participating in the {\market} are
        operating optimally, and the ``average'' Shannon probability,
        $P$ for each participating company would be, using
        Equation~\ref{pncompanies}, {\pncompanies}, which would be the
        value which should be used in Section~\ref{\SETLABEL:FS} for
        each participating company if market expansion was to be
        consistent with the rest of the industry. However, if the
        Shannon probability derived in Section~\ref{\SETLABEL:FS} is
        greater than the average Shannon probability for the companies
        participating in the {\market}, as derived in this section,
        then the market would, possibly, be exploitable with the
        fiscal strategy outlined in Section~\ref{\SETLABEL:FS}. The
        maximum exploitability for the {\market} is derived in
        Section~\ref{\SETLABEL:MAXSHANNON}, but it is probably of
        doubtful practicality.

        Note that these optimizations would maximize a company's
        market growth. Since there are probably many companies
        competing in the market place, this would not necessarily
        maximize a company's P\&L, as described in
        Chapter~\ref{general}, Section~\ref{ompl}. The Shannon
        probability that maximizes market share in the {\market} is
        \pncompanies, with several alternative solutions listed in the
        previous paragraph. However, these should be contrasted to the
        Shannon probability that maximizes a company's P\&L which is
        \avgrms~in the {\market}. In all cases, the fraction of the
        P\&L that should be ``wagered'' on the future, $f$, should be:

        \begin{equation}
            f = 2P - 1
        \end{equation}

        \noindent where $P$ is the particular Shannon probability
        chosen optimize a particular fiscal strategy. Interestingly,
        the measured Shannon probability of the {\market} would tend
        to indicate that the companies participating in the market
        have chosen a fiscal strategy that optimizes market growth, as
        opposed to capital growth.

        \subidx{\market}{increasing returns}
        \subidx{economic increasing returns}{\market}
        As interesting interpretation of these exploitive issues,
        since all three fiscal strategies will result in exponential
        market growth for every company participating in the market,
        is that they may represent, perhaps, an example of
        ``increasing returns.''

% Local Variables:
% TeX-parse-self: t
% TeX-auto-save: t
% TeX-master: "fractal.tex"
% End:


        %
% -----------------------------------------------------------------------------
%
% A license is hereby granted to reproduce this software source code and
% to create executable versions from this source code for personal,
% non-commercial use.  The copyright notice included with the software
% must be maintained in all copies produced.
%
% THIS PROGRAM IS PROVIDED "AS IS". THE AUTHOR PROVIDES NO WARRANTIES
% WHATSOEVER, EXPRESSED OR IMPLIED, INCLUDING WARRANTIES OF
% MERCHANTABILITY, TITLE, OR FITNESS FOR ANY PARTICULAR PURPOSE.  THE
% AUTHOR DOES NOT WARRANT THAT USE OF THIS PROGRAM DOES NOT INFRINGE THE
% INTELLECTUAL PROPERTY RIGHTS OF ANY THIRD PARTY IN ANY COUNTRY.
%
% Copyright (c) 1994-2006, John Conover, All Rights Reserved.
%
% Comments and/or bug reports should be addressed to:
%
%     john@email.johncon.com (John Conover)
%
% -----------------------------------------------------------------------------
%
% Revision: \RCSRevision \\
% Revision Time: \RCSTime UMT \\
% Revision Date: \RCSDate \\
% Revision Id: \RCSId \\
% Revision File: \RCSLog \\
\RCS $Revision: 0.0 $
\RCS $Date: 2006/01/20 04:38:13 $
\RCS $Id: operations.tex,v 0.0 2006/01/20 04:38:13 john Exp $
% $Log: operations.tex,v $
% Revision 0.0  2006/01/20 04:38:13  john
% Initial version
%
%
    \subsection{Fixed Increment Approximation for Operational Strategy}
        \label{\SETLABEL:OPS}.

        This section derives various values based on the ``average''
        of the normalized increments presented in
        Figure~\ref{\SETLABEL:TFA}. These values are an approximation
        to a, probably, complex process with a distribution shown in
        Figure~\ref{\SETLABEL:TF}. These values will be used in a
        fixed increment Brownian fractal analysis and simulation of
        the {\market}, and may, or may not, provide adequate accuracy
        for projections.

        \subidx{\market}{fiscal strategy}
        \subidx{\market}{Shannon probability}
        \subidx{strategy}{fiscal}
        \subidx{fiscal}{strategy}
        \subidx{Shannon}{probability}
        \subidx{probability}{Shannon}
        It should be noted that the analysis of fiscal strategy,
        presented in Section~\ref{\SETLABEL:FS}, is derived from the
        {\market} metrics and may, or may not, be maximally
        optimal. For the optimal fiscal strategy, which may be
        exploitable, see Section~\ref{\SETLABEL:MAXSHANNON}.

        \subidx{strategy}{exploitable}
        \subidx{exploitable}{strategy}
        \subidx{\market}{windows of opportunity}
        \idx{windows of opportunity}
        \subidx{decision}{obsolete}
        \subidx{obsolete}{decision}
        \subidx{decision}{timeliness}
        \subidx{timeliness}{decision}
        \subidx{rate of revenue returns}{forecast}
        \subidx{forecast}{rate of revenue returns}
        An additional exploitable strategy may be time itself.
        Equations~\ref{\SETLABEL:V},~\ref{\SETLABEL:R},
        and,~\ref{\SETLABEL:MA}, are, essentially, metrics on how fast
        a decision, which is based on information concerning the
        current status of the {\market}, becomes obsolete. Obviously,
        how long a decision is expected to remain relevant should be
        addressed as an operational necessity in strategic planning
        and project management. Figures~\ref{\SETLABEL:FN},
        and,~\ref{\SETLABEL:FF} compare methods of approximation of
        the ``forecastability'' of rate of revenue returns in the
        {\market} for the near term and far
        term~\cite[pp. 83-84]{Peters:CAOITCM}, respectively. As a
        general rule, caution must be exercised when making decisions
        that will span a time interval larger than the time interval
        where the ``forecastability'' of rate of revenue returns drops
        below 50\%. Beyond this time interval, the chances increase
        that the competitive and market forces will alter the market
        environment in a possibly detrimental unanticipated
        fashion. Obviously, there is significant advantage in
        ``timeliness'' of development, manufacturing, and distribution
        of products and services that are consistent with this
        temporal agenda. Automation of these processes, if executed
        consistently with this agenda, should be considered a
        competitive advantage.

        \subidx{strategy}{exploitable}
        \subidx{exploitable}{strategy}
        \subidx{rate of revenue returns}{forecast}
        \subidx{forecast}{rate of revenue returns}
        \idx{product life cycle}
        \idx{life cycle, product}
        In some sense, this temporal agenda defines the ``average''
        product or service life cycle in the {\market}. When the
        ``forecastability'' of rate of revenue returns drops below
        50\%, there is an even chance that the rate of revenue returns
        for the product or service will change in a detrimental
        fashion. If it is assumed that a product or service life cycle
        consists of a ramp up, a maintenence interval, and a ramp
        down, then, if all three life cycle intervals are equal, the
        product life cycle will be, approximately, three times the
        time interval where the ``forecastability'' of rate of revenue
        returns drops below 50\%. Although probably not an accurate
        prediction of product or service life cycle, the technique may
        be used as a conceptual approximation to the dynamics of
        ``market windows.\footnote{For example, consider the market
        for table salt. Since it has inelastic supply and demand
        curves, and is a necessary requirement for life, it would be
        expected that the Hurst coefficient would be very near
        unity---ignoring competitive pressures in the market. The
        predictability of the table salt market would, therefore, be
        expected to be relatively good, over time.}''  The conceptual
        approximation will probably predict a ``conservative'' or
        ``pessimistic'' value in relation to actual markets.

        \begin{figure}[ht]
            \begin{center}
                \begin{minipage}[t]{0.45\textwidth}
                    \epsfxsize=1.0\linewidth
                    \epsffile{\directory/datahurstlownear.eps}
                    \caption[{\market}, ``forecastability'' of near
                        term rate of revenue returns]{{\market},
                        ``forecastability'' of near term rate of
                        revenue returns. Although the error function
                        is the most accurate, for the near term,
                        $H^{t} = \thurstlow^{t}$ may be used as a
                        reliable metric of ``forecastability'' of the
                        rate of revenue returns.}
                    \label{\SETLABEL:FN}
                \end{minipage}
                \hfill
                \begin{minipage}[t]{0.45\textwidth}
                    \epsfxsize=1.0\linewidth
                    \epsffile{\directory/datahurstlowfar.eps}
                    \caption[{\market}, ``forecastability'' of far
                        term rate of revenue returns]{{\market},
                        ``forecastability'' of far term rate of
                        revenue returns. Although the error function
                        is the most accurate, for the far term,
                        $\frac{1}{\sqrt{t}}$ may be used as a reliable
                        metric of ``forecastability'' of the rate of
                        revenue returns.}
                    \label{\SETLABEL:FF}
                \end{minipage}
            \end{center}
        \end{figure}

        \idx{operations research}
        As an interesting interpretation of the data presented in
        Figure~\ref{\SETLABEL:FN}, there may be, perhaps, some
        applicability to such operational agendas as inventory
        control. Maintaining too little inventory, obviously, will
        create a situation where the organization can not exploit
        market expansion, and maintaining too much inventory,
        likewise, would over extend the company, creating unnecessary
        losses when the market contracts. The company should maintain
        inventory levels that do not exceed, from
        Equation~\ref{\SETLABEL:MA}, ${\thurstlow}^{n} = 0.5$
        {\timescale}s of operations. Since the optimal amount of
        inventory and, from Equation~\ref{\SETLABEL:V}, the variance
        of change in the rate of revenue returns in the future can be
        calculated, there may, perhaps, be some applicability to a
        forecasting methodology that can be incorporated into other
        areas of operations research, for example the linear algebras
        using simplex methodologies for optimization of manufacturing
        processes. Traditionally, these forecasts are made by the
        sales department, and are subject to various subjective
        biases.

% Local Variables:
% TeX-parse-self: t
% TeX-auto-save: t
% TeX-master: "fractal.tex"
% End:


        \subsubsection{Observations on the Fixed Increment Approximation for Operational Strategy}

            As an interesting interpretation of
            Figure~\ref{\SETLABEL:FF}, and evaluating the
            approximation $\frac{1}{\sqrt{t}}$ at 60 months gives a
            probability that the market will still have the same
            agenda of about $0.12909945$, or about 1 in 8. This is
            commensurate with numbers from the venture
            community\footnote{For example, see ``IEEE Engineering
            Management Review,'' Volume 23 Number 3, Fall 1995,
            pp. 83}. Of course new venture backed companies fail for
            many reasons, but market appropriateness to product
            portfolio 60 months in the future may be a major
            contributor. Additionally, the success rate of development
            projects of 8 month duration, which have a market success
            rate of about 1 in 3, seems consistent with
            $\frac{1}{\sqrt{3}} = 0.353553391$. Naturally, projects
            fail in the market for many reasons, but market
            appropriateness, in a dynamic market environment may be a
            major contributor to failure.

            As mentioned in Section~\ref{\SETLABEL:H},
            Equation~\ref{\SETLABEL:MA}, and the preceeding section,
            approximately 3 times the value where ${\thurstlow}^{n} =
            0.5$ could be interpreted as an approximation to the
            ``average'' product life cycle. This seems consistent with
            the 6 to 12 month life cycles quoted by many industry
            analyst. In addition, maintaining inventory levels that do
            not exceed the anticipated requirements of
            $\frac{\ln{0.5}}{\ln{\thurstlow}}$ many {\timescale}s
            seems consistent with the author's experience in the
            industry.

        For convenience of comparison, converting from quarters to
        months by dividing the logarithmic returns by 3:

        \renewcommand{\timescale}{month}
        \renewcommand{\datafractionmean}{0.044437}
\renewcommand{\datafractionmeanbits}{0.062725}
\renewcommand{\datafractionmeanq}{0.014812}
\renewcommand{\datafractionmeanbitsq}{0.021213}
\renewcommand{\datafractionstddev}{0.064421}
\renewcommand{\datafractionrms}{0.025913}
\renewcommand{\avgrms}{1.357427}
\renewcommand{\ncompanies}{66.177345}
\renewcommand{\pncompanies}{0.605400}
\renewcommand{\datafractionabsmean}{0.061981}
\renewcommand{\datafractionabsstddev}{0.047389}
\renewcommand{\datafractionconstant}{0.039513}
\renewcommand{\datafractionconstantbits}{0.055908}
\renewcommand{\datafractionconstantq}{0.013171}
\renewcommand{\datafractionconstantbitsq}{0.018878}
\renewcommand{\datafractionslope}{0.000197}
\renewcommand{\datafractionabsconstant}{0.078868}
\renewcommand{\datafractionabsslope}{-0.000675}
\renewcommand{\hurstall}{0.644727}
\renewcommand{\hurstlow}{1.028920}
\renewcommand{\hurstlowtwo}{2.057840}
\renewcommand{\hurstlowhundred}{102.892000}
\renewcommand{\hcalcall}{0.712999}
\renewcommand{\hcalclow}{0.745601}
\renewcommand{\shannonmax}{0.826923}
\renewcommand{\twoponemax}{0.653846}
\renewcommand{\logreturns}{0.019605}
\renewcommand{\twologreturns}{1.013682}
\renewcommand{\twologreturnshundred}{1.368190}
\renewcommand{\oneoverlogreturns}{51.007396}
\renewcommand{\pmax}{0.823529}
\renewcommand{\twopminusone}{0.647059}
\renewcommand{\rmsp}{0.016767}
\renewcommand{\twopx}{0.848668}
\renewcommand{\sigmap}{0.054672}
\renewcommand{\tsunfairbrownianfractionmean}{0.049753}
\renewcommand{\tsunfairbrownianfractionstddev}{0.060339}
\renewcommand{\shannonlogreturns}{0.582242}
\renewcommand{\shannonlogreturnshundred}{58.224200}
\renewcommand{\twopone}{0.164484}
\renewcommand{\twoponehundred}{16.448400}
\renewcommand{\hundredtwoponehundred}{83.551600}
\renewcommand{\hundredshannonlogreturnshundred}{41.775800}
\renewcommand{\datatslsqepbits}{0.017926}
\renewcommand{\thurstall}{0.725956}
\renewcommand{\thurstlow}{1.025249}
\renewcommand{\thurstlowtwo}{2.050498}
\renewcommand{\thurstlowhundred}{102.524900}
\renewcommand{\thcalcall}{0.885411}
\renewcommand{\thcalclow}{0.871338}
\renewcommand{\chisquared}{9.194000}
\renewcommand{\critical}{42.557000}

        \renewcommand{\SETLABEL}{\LABPRE:NASMQ}
        \renewcommand{\datafractionmean}{\datafractionmeanq}
        \renewcommand{\datafractionconstant}{\datafractionconstantq}
        \renewcommand{\datafractionmeanbits}{\datafractionmeanbitsq}
        \renewcommand{\datafractionconstantbits}{\datafractionconstantbitsq}

        %
% -----------------------------------------------------------------------------
%
% A license is hereby granted to reproduce this software source code and
% to create executable versions from this source code for personal,
% non-commercial use.  The copyright notice included with the software
% must be maintained in all copies produced.
%
% THIS PROGRAM IS PROVIDED "AS IS". THE AUTHOR PROVIDES NO WARRANTIES
% WHATSOEVER, EXPRESSED OR IMPLIED, INCLUDING WARRANTIES OF
% MERCHANTABILITY, TITLE, OR FITNESS FOR ANY PARTICULAR PURPOSE.  THE
% AUTHOR DOES NOT WARRANT THAT USE OF THIS PROGRAM DOES NOT INFRINGE THE
% INTELLECTUAL PROPERTY RIGHTS OF ANY THIRD PARTY IN ANY COUNTRY.
%
% Copyright (c) 1994-2006, John Conover, All Rights Reserved.
%
% Comments and/or bug reports should be addressed to:
%
%     john@email.johncon.com (John Conover)
%
% -----------------------------------------------------------------------------
%
% Revision: \RCSRevision \\
% Revision Time: \RCSTime UMT \\
% Revision Date: \RCSDate \\
% Revision Id: \RCSId \\
% Revision File: \RCSLog \\
\RCS $Revision: 0.0 $
\RCS $Date: 2006/01/20 04:38:13 $
\RCS $Id: fiscal.tex,v 0.0 2006/01/20 04:38:13 john Exp $
% $Log: fiscal.tex,v $
% Revision 0.0  2006/01/20 04:38:13  john
% Initial version
%
%
    \subsection{Fixed Increment Approximation for Fiscal Strategy}
        \label{\SETLABEL:FS}

        \subidx{\market}{fiscal strategy}
        \subidx{markets}{analysis}
        \subidx{analysis}{markets}
        \subidx{strategy}{fiscal}
        \subidx{fiscal}{strategy}
        The data in this section is presented in tabular form in
        Section~\ref{\SETLABELREF:LR}. This section derives various
        values based on the ``average'' of the normalized increments
        presented in Figure~\ref{\SETLABEL:TFA}. These values are an
        approximation to a, probably, complex process with a
        distribution shown in Figure~\ref{\SETLABEL:TF}. These values
        will be used in a fixed increment Brownian fractal analysis
        and simulation of the {\market}, and may, or may not, provide
        adequate accuracy for projections.

        For an organization operating in the {\market}, the fiscal
        strategy, commensurate with the aggregate environment, can be
        derived as follows~\cite[pp. 128, pp
        151]{Schroeder},~\cite[pp. 450]{Reza},~\cite[pp. 270]{Pierce}:
        \vspace{0.15in}

        \subsubsection{Logarithmic Returns}
            \label{\SETLABEL:LR}

            \subidx{logarithmic}{returns}
            \subidx{returns}{logarithmic}
            \subidx{\market}{logarithmic returns}
            The logarithmic returns can be calculated by various
            means. Four will be presented here, for comparison.

            \subidx{programs}{tsnormal}
            \subidx{tsnormal}{program}
            \subidx{logarithmic}{returns}
            \subidx{returns}{logarithmic}
            The logarithmic returns, in bits, $bits$, as computed from
            the mean, by the program {\it tsnormal}\/, which is
            described in Chapter~\ref{programs}, and is presented in
            Figure~\ref{\SETLABEL:TF}, and Equation~\ref{abits} from
            Section~\ref{ereturns} in Chapter~\ref{general}:

            \begin{equation}
                bits = \frac{\ln \left({\datafractionmean} + 1\right)}{\ln \left(2\right)} = \datafractionmeanbits
            \end{equation}

            \subidx{programs}{tslsq}
            \subidx{tslsq}{program}
            \subidx{logarithmic}{returns}
            \subidx{returns}{logarithmic}
            \noindent By comparison, the logarithmic returns, in bits,
            $bits$, as computed from the constant in the least squares
            approximation, using the program {\it tslsq}\/, which is briefly
            described in Chapter~\ref{programs}, as presented in
            Figure~\ref{\SETLABEL:TF}, and Equation~\ref{abits} from
            Section~\ref{ereturns} in Chapter~\ref{general}:

            \begin{equation}
                bits = \frac{\ln \left({\datafractionconstant} + 1\right)}{\ln \left(2\right)} = \datafractionconstantbits
            \end{equation}

            Note that if the mean is not constant in
            Figure~\ref{\SETLABEL:TF}, this method will not provide
            accurate results.

            \subidx{programs}{tslsq}
            \subidx{tslsq}{program}
            \subidx{logarithmic}{returns}
            \subidx{returns}{logarithmic}
            \noindent And by yet another comparison, using the program
            {\it tslsq}\/, which is briefly described in
            Chapter~\ref{programs}, with the -e -p options, to provide
            a formula for the least squares exponential fit to the
            time series data set presented in
            Figure~\ref{\SETLABEL:TS}:

            \begin{equation}
                bits = {\datatslsqepbits}
            \end{equation}

            \subidx{programs}{tslogreturns}
            \subidx{tslogreturns}{program}
            \subidx{logarithmic}{returns}
            \subidx{returns}{logarithmic}
            \noindent And finally, by comparison, from the
            {\it tslogreturns}\/ program, which is briefly described
            in Chapter~\ref{programs}, with the -p option, to provide
            a formula for the logarithmic returns of the time series
            data set presented in Figure~\ref{\SETLABEL:TS}:

            \begin{equation}
                bits = {\logreturns}
            \end{equation}

        \subsubsection{Calculation of Shannon Probability}
            \label{\SETLABEL:SP}

            \subidx{\market}{Shannon probability}
            Ideally, all of the values presented in
            Section~\ref{\SETLABEL:LR} would be equal. Using the
            logarithmic returns provided by the {\it tslogreturns}\/
            program, to be consistent
            with~\cite[pp. 81]{Peters:CAOITCM}

            \subidx{programs}{tslogreturns}
            \subidx{tslogreturns}{program}
            \begin{equation}
                2^{{\logreturns}t}
            \end{equation}

            \noindent therefore:
            \begin{equation}
                C\left(p\right) = {\logreturns}
            \end{equation}
            \subidx{programs}{tsshannon}
            \subidx{tsshannon}{program}
            \subidx{Shannon}{probability}
            \subidx{probability}{Shannon}
            \noindent and, {\it tsshannon}\/ {\logreturns} gives:
            \begin{equation}
                \label{\SETLABEL:F0}
                C\left({\shannonlogreturns}\right) = {\logreturns}
            \end{equation}
            \noindent therefore:
            \begin{eqnarray}
                2^{C\left({\shannonlogreturns}\right)} & = & 2^{\logreturns}\\
                                                       & = & {\twologreturns}\\
                                                       & = & {\twologreturnshundred}\%
            \end{eqnarray}
            \noindent and:
            \begin{eqnarray}
                2p - 1 & = & \left(2 \cdot {\shannonlogreturns}\right) - 1\\
                       & = & {\twopone}\\
                       \label{\SETLABEL:F1}
                       & = & {\twoponehundred}\%
            \end{eqnarray}

            \subidx{\market}{fiscal strategy}
            \subidx{markets}{analysis}
            \subidx{analysis}{markets}
            \subidx{strategy}{fiscal}
            \subidx{fiscal}{strategy}
            \subidx{\market}{fiscal strategy}
            \subidx{\market}{growth rate}
            Presuming the simplified assumptions outlined in
            Section~\ref{assumptions}, the ``typical'' organization
            operating in the {\market} executes a long term fiscal
            strategy, commensurate with the aggregate environment,
            that is to invest, every {\timescale}, in sufficient
            additional resources and infrastructure, to increase the
            manufacturing of goods and services by {\twoponehundred}\%
            of its rate of revenue returns, (per {\timescale}.) As a
            conceptual model, the remaining {\hundredtwoponehundred}\%
            will be held in ``reserve'' with a
            {\shannonlogreturnshundred}\% chance of making twice the
            {\twoponehundred}\% back, (and a
            {\hundredshannonlogreturnshundred}\% chance of making
            0.0,) in one {\timescale}, on the average, for an average
            growth in its rate of revenue returns, (per {\timescale},)
            of {\twologreturnshundred}\%, or a doubling of its rate of
            revenue returns, (per {\timescale},) in
            {\oneoverlogreturns} {\timescale}s.

        \subsubsection{Example Fixed Increment Approximation Fiscal Strategies}

            \subidx{\market}{fiscal strategy}
            \subidx{markets}{analysis}
            \subidx{analysis}{markets}
            \subidx{strategy}{fiscal}
            \subidx{fiscal}{strategy}
            \subidx{\market}{fiscal strategy}
            \subidx{\market}{growth rate}
            \subidx{\market}{management metric}
            \idx{management metric}
            A possible metric on the effectiveness of long term fiscal
            management could possibly be that if an investment of
            {\twoponehundred}\% per {\timescale} of the rate of
            revenue returns, (per {\timescale},) is made in resources
            and infrastructure, then the rate of revenue returns would
            be expected to increase by {\twologreturnshundred}\%, per
            {\timescale}, on average.

            Note that the metrics presented in this section are
            representative of the {\market} as an aggregate whole, and
            may or may not be accurate representations for any
            particular participant in the environment. Of interest to
            the participants in the environment would be a similar
            analysis of each product or service rendered in the
            marketplace.

            \subidx{\market}{fiscal strategy}
            \subidx{markets}{analysis}
            \subidx{analysis}{markets}
            \subidx{strategy}{fiscal}
            \subidx{fiscal}{strategy}
            \subidx{\market}{fiscal strategy}
            As a simple illustrative example, a company operating in
            this environment might obtain a credit line from a bank
            that is equal to {\twoponehundred}\% of its rate of
            revenue returns, (per {\timescale},) to finance additional
            operations. In this simple scenario, the company would use
            its revenue base as collateral for the loan. Some
            {\timescale}s, depending on the {\market}'s environment,
            the company's rate of revenue returns exceeds what was
            borrowed from the bank, and the loan is repaid in
            full. Other {\timescale}s, the company must default, and
            the bank seizes a portion of the company's revenue base to
            pay the delinquent loan. However, on the average, the
            company will expand its rate of revenue returns at
            {\twologreturnshundred}\% per {\timescale}.

            \subidx{\market}{fiscal strategy}
            \subidx{markets}{analysis}
            \subidx{analysis}{markets}
            \subidx{strategy}{fiscal}
            \subidx{fiscal}{strategy}
            \subidx{\market}{fiscal strategy}
            As another simple example, a company re-invests
            {\twoponehundred}\% of its rate of revenue returns, (per
            {\timescale},) in development, marketing, sales, and
            distribution of new products.  Although some products will
            be successful and the return on the investment will exceed
            the {\twoponehundred}\% per {\timescale} investment,
            others will not. However, on the average, the company will
            expand it gross rate of revenue returns at
            {\twologreturnshundred}\% per {\timescale}.

            \subidx{\market}{fiscal strategy}
            \subidx{markets}{analysis}
            \subidx{analysis}{markets}
            \subidx{strategy}{fiscal}
            \subidx{fiscal}{strategy}
            \subidx{\market}{fiscal strategy}
            \subidx{\market}{product portfolio}
            \subidx{\market}{product diversity}
            \subidx{\market}{product mix}
            \subidx{\market}{optimum number of products}
            \idx{product portfolio}
            \idx{product diversity}
            \idx{optimum number of products}
            \idx{product mix}

            As an example of ``product portfolio'' management, suppose
            a company re-invests {\twoponehundred}\% of its rate of
            revenue returns, (per {\timescale},) in development,
            marketing, sales, and distribution of new products.
            Further suppose that the company has two products, and a
            fractal analysis of the individual product rate of revenue
            return time series indicates that one product has a
            Shannon probability of 0.65, and the other has a Shannon
            probability of 0.55. Then the percentage of re-investment
            in the first product would be $(2 \cdot 0.65 - 1) \cdot
            {\twoponehundred}$, percent of the rate of revenue
            returns, and $(2 \cdot 0.55 - 1) \cdot {\twoponehundred}$
            percent for the second product, implying that the company
            should diversify its product line\footnote{The astute
            reader would note that the linear addition was used to add
            the contribution to development of each product. This is a
            ``near term'' interpretation. Actually, in general, the
            method used should be a root mean square process,
            dependent on the Hurst Coefficient, $H$, where
            $P_{total}^H = P_1^H + P_2^H + \cdots$, where $P_n$ is the
            contribution to each individual product. For a Brownian
            motion, or random walk type of fractal the Hurst
            Coefficient is a function of time into the future. For the
            ``near term,'' the Hurst coefficient is very near unity,
            meaning the summation process is linear. For the ``long
            term,'' $H \approx 0.5$, or a standard root mean square
            summation process should be used. If $H$ is $0.5$ then the
            market is termed a Brownian motion, or random walk
            process. If it is larger than 0.5, it is termed fractional
            Brownian motion process. For a random walk process, ``near
            term'' and ``far term'' are quantitatively differentiated
            on the Hurst Coefficient graph where $1 - \ln (t) = 0.5
            \cdot \ln (t)$, or when $\ln (t) = 2$, or $t =
            7.389\ldots$ See~\cite[pp. 67, 83-84]{Peters:CAOITCM}
            and~\cite[pp. 129, 159]{Schroeder} for particulars on the
            implications of the Hurst Coefficient and root mean square
            summation issues.}.  Note that this is a ``bet hedging''
            metric methodology, and assumes that the products have
            uncorrelated revenue return rates. If this re-investment
            methodology is not feasible, perhaps for strategic
            financial reasons, then the re-investment in both products
            should total the ${\twoponehundred}$\%, and the investment
            in each product should be made at a ratio of $\frac{(2
            \cdot 0.65 - 1)}{(2 \cdot 0.55 - 1)} = 3 : 1$,
            respectively. Note that this ``bet hedging'' can be used
            to define the optimal number of products that can be
            supported on the rate of revenue returns. If it assumed
            that all products are ``typical'' for the {\market}, as a
            standard bench mark, then the optimal number will be
            $\frac{1}{{\twopone}}$. Note that this is a
            ``theoretical'' value, since not all products are
            ``typical,'' and there may be strategic reasons, for
            example product leveraging, that may increase the number
            of products above the optimum. However, most of the
            revenue should come from the optimal number of products,
            since having more products will decrease the amount of the
            potential investment in each product, and having less than
            the optimum number of products will increase the risk that
            many of the products could suffer a ``down market''
            concurrently, impacting the rate of revenue returns.  As
            another interesting interpretation of the optimal
            ``hedging of bets,'' in product portfolio strategy, and
            considering the graph of the normalized increments
            presented in Figure~\ref{\SETLABEL:TF}, if the
            organization is running optimally, then these products
            will generate, at least in principle, one standard
            deviation, approximately $0.8413 = 84.13$\% of the future
            growth in rate of revenue returns. Naturally, these are
            approximations, and the values are an approximation to a,
            probably, complex process, and appropriate scrutiny should
            be exercised before making specific projections.  As yet
            another example of ``product portfolio'' management,
            consider the issue of product mix. In this interpretation,
            {\twoponehundred}\% of the product manufactured should be
            ``proprietary,'' while the rest is ``industry standard.''
            As yet another possibility, {\twoponehundred}\% of the
            product manufactured should be predatory into new markets,
            and the remainder in markets that are ``traditional'' for
            the company.

% Local Variables:
% TeX-parse-self: t
% TeX-auto-save: t
% TeX-master: "fractal.tex"
% End:


        \renewcommand{\SETLABEL}{\LABPRE:NASM}
        \renewcommand{\datafractionmean}{0.008052}
\renewcommand{\datafractionmeanbits}{0.011570}
\renewcommand{\datafractionmeanq}{0.002684}
\renewcommand{\datafractionmeanbitsq}{0.003867}
\renewcommand{\datafractionstddev}{0.038579}
\renewcommand{\datafractionrms}{0.039311}
\renewcommand{\avgrms}{0.602414}
\renewcommand{\ncompanies}{5.210454}
\renewcommand{\pncompanies}{0.544866}
\renewcommand{\datafractionabsmean}{0.029745}
\renewcommand{\datafractionabsstddev}{0.025769}
\renewcommand{\datafractionconstant}{0.010041}
\renewcommand{\datafractionconstantbits}{0.014414}
\renewcommand{\datafractionconstantq}{0.003347}
\renewcommand{\datafractionconstantbitsq}{0.004821}
\renewcommand{\datafractionslope}{-0.000021}
\renewcommand{\datafractionabsconstant}{0.035145}
\renewcommand{\datafractionabsslope}{-0.000057}
\renewcommand{\hurstall}{0.659558}
\renewcommand{\hurstlow}{0.707509}
\renewcommand{\hurstlowtwo}{1.415018}
\renewcommand{\hurstlowhundred}{70.750900}
\renewcommand{\hcalcall}{0.184942}
\renewcommand{\hcalclow}{0.102042}
\renewcommand{\shannonmax}{0.604167}
\renewcommand{\twoponemax}{0.208334}
\renewcommand{\logreturns}{0.010456}
\renewcommand{\twologreturns}{1.007274}
\renewcommand{\twologreturnshundred}{0.727387}
\renewcommand{\oneoverlogreturns}{95.638868}
\renewcommand{\pmax}{0.602094}
\renewcommand{\twopminusone}{0.204188}
\renewcommand{\rmsp}{0.008027}
\renewcommand{\twopx}{0.208583}
\renewcommand{\sigmap}{0.008047}
\renewcommand{\tsunfairbrownianfractionmean}{0.007862}
\renewcommand{\tsunfairbrownianfractionstddev}{0.038619}
\renewcommand{\shannonlogreturns}{0.560125}
\renewcommand{\shannonlogreturnshundred}{56.012500}
\renewcommand{\twopone}{0.120250}
\renewcommand{\twoponehundred}{12.025000}
\renewcommand{\hundredtwoponehundred}{87.975000}
\renewcommand{\hundredshannonlogreturnshundred}{43.987500}
\renewcommand{\datatslsqepbits}{0.007623}
\renewcommand{\thurstall}{0.633980}
\renewcommand{\thurstlow}{0.710108}
\renewcommand{\thurstlowtwo}{1.420216}
\renewcommand{\thurstlowhundred}{71.010800}
\renewcommand{\thcalcall}{0.247886}
\renewcommand{\thcalclow}{0.171737}
\renewcommand{\chisquared}{2.862000}
\renewcommand{\critical}{42.557000}

        \renewcommand{\timescale}{quarter}

        %
% -----------------------------------------------------------------------------
%
% A license is hereby granted to reproduce this software source code and
% to create executable versions from this source code for personal,
% non-commercial use.  The copyright notice included with the software
% must be maintained in all copies produced.
%
% THIS PROGRAM IS PROVIDED "AS IS". THE AUTHOR PROVIDES NO WARRANTIES
% WHATSOEVER, EXPRESSED OR IMPLIED, INCLUDING WARRANTIES OF
% MERCHANTABILITY, TITLE, OR FITNESS FOR ANY PARTICULAR PURPOSE.  THE
% AUTHOR DOES NOT WARRANT THAT USE OF THIS PROGRAM DOES NOT INFRINGE THE
% INTELLECTUAL PROPERTY RIGHTS OF ANY THIRD PARTY IN ANY COUNTRY.
%
% Copyright (c) 1994-2006, John Conover, All Rights Reserved.
%
% Comments and/or bug reports should be addressed to:
%
%     john@email.johncon.com (John Conover)
%
% -----------------------------------------------------------------------------
%
% Revision: \RCSRevision \\
% Revision Time: \RCSTime UMT \\
% Revision Date: \RCSDate \\
% Revision Id: \RCSId \\
% Revision File: \RCSLog \\
\RCS $Revision: 0.0 $
\RCS $Date: 2006/01/20 04:38:13 $
\RCS $Id: simulation.tex,v 0.0 2006/01/20 04:38:13 john Exp $
% $Log: simulation.tex,v $
% Revision 0.0  2006/01/20 04:38:13  john
% Initial version
%
%
    \subsection{Simulation of Fixed Increment Approximation for Fiscal Strategy}
        \label{\SETLABEL:TSUNFAIRBROWNIAN}

        \subidx{\market}{market simulation}
        The data in this section is presented in tabular form in
        Section~\ref{\SETLABELREF:SIM}.
        Figure~\ref{\SETLABEL:TSUNFAIRBROWNIAN0} represents a
        constructional simulation of the time series data presented in
        Figure~\ref{\SETLABEL:TS}. The program {\it
        tsunfairbrownian}\/, which is briefly described in
        appendix~\ref{programs}, was used in the reconstruction. The
        reconstructed data is superimposed on the original time series
        data.  The program, {\it tsunfairbrownian}\/, essentially,
        constructs the new time series as a Brownian fractal with
        fixed increments---the value of the fixed increment is derived
        from the root mean square average of the normalized increments
        presented in Figure~\ref{\SETLABEL:TF}. The ``quality'' of
        such a reconstruction should be subject to adequate scepticism
        and scrutiny since, in all probability, the normalized
        increments presented in Figure~\ref{\SETLABEL:TF} represent a
        relatively complex process, that may not be ``modeled'' with
        such a simple methodology.

        As a further comparison of the the constructional simulation
        with the original time series data,
        Figure~\ref{\SETLABEL:TSUNFAIRBROWNIAN1} presents a normalized
        histogram of the normalized increments of the reconstructed
        time series, superimposed on the normalized histogram
        presented in Figure~\ref{\SETLABEL:NH}.

        \subidx{\market}{fiscal strategy, simulation}
        \subidx{markets}{simulation}
        \subidx{simulation}{markets}
        \subidx{strategy}{fiscal, simulation}
        \subidx{fiscal}{strategy, simulation}
        \subidx{programs}{tsunfairbrownian}
        \subidx{tsunfairbrownian}{program}
        \begin{figure}[ht]
            \begin{center}
                \begin{minipage}[t]{0.45\textwidth}
                    \epsfxsize=1.0\linewidth
                    \epsffile{\directory/tsunfairbrownian-f.eps}
                    \caption[{\market}, Time series data, empirical and
                        simulated]{{\market}, Time series data, empirical
                        and simulated, using the program {\it tsunfairbrownian}\/
                        with f = {\datafractionrms}. This data is
                        superimposed on the data presented in
                        Figure~\ref{\SETLABEL:TS}.}
                    \label{\SETLABEL:TSUNFAIRBROWNIAN0}
                \end{minipage}
                \hfill
                \begin{minipage}[t]{0.45\textwidth}
                    \epsfxsize=1.0\linewidth
                    \epsffile{\directory/tsunfairbrownian-f.tsfraction.tsnormal-s30.eps}
                    \caption[{\market}, normalized histogram,
                        empirical and simulated]{{\market}, normalized
                        histogram of the normalized increments of the
                        time series data shown in
                        Figure~\ref{\SETLABEL:TSUNFAIRBROWNIAN0},
                        empirical and simulated.  The empirical data
                        has a mean of {\datafractionmean}, with a
                        standard deviation of {\datafractionstddev}.
                        By comparison, the simulated data has a mean
                        of {\tsunfairbrownianfractionmean} with a
                        standard deviation of
                        {\tsunfairbrownianfractionstddev}. This data
                        is superimposed on the data presented in
                        Figure~\ref{\SETLABEL:NH}. The area under the
                        four curves is identical.}
                    \label{\SETLABEL:TSUNFAIRBROWNIAN1}
                \end{minipage}
            \end{center}
        \end{figure}

% Local Variables:
% TeX-parse-self: t
% TeX-auto-save: t
% TeX-master: "fractal.tex"
% End:


        \subsubsection{Observations on the Simulation of Fixed Increment Approximation for Optimally Maximal Fiscal Strategy}

            Note that these simulations are base on a very, perhaps
            overly, simplified model. For example, from
            Section~\ref{\SETLABEL:TSA}, Figure~\ref{\SETLABEL:NH}, it
            would appear that the {\market}'s normalized increments
            are characterized by fractional Brownian motion---but the
            simulations used classical Brownian motion as the
            model. One consequence of this is that a re-investment
            strategy that is to ``wager'' a fraction of {\twoponemax}
            of the rate of returns every {\timescale} is overly
            aggressive, since in the classical Brownian scenario, the
            maximum loss, in any {\timescale}, was no more that what
            was ``wagered.'' However, in the fractional Brownian
            scenario, much more can be lost. From
            Equation~\ref{fopt2},

            \begin{equation}
                \frac{avg}{rms^2} = \frac{f_{opt}}{rms} = K
            \end{equation}

            \noindent where, under the optimum classical Brownian
            scenario, $K$ is unity, or $avg = rms^2$. Notice that,
            since $f = rms$, whether the scenario is optimal or not,
            that the operational ``wager'' fraction, from
            Figure~\ref{\SETLABEL:TF} of {\datafractionrms}, vs.\ an
            ``theoretical optimal'' value of {\twoponemax} seems
            overly conservative. Additionally, notice that, at least
            in principle, the chance of failure in the fractional
            Brownian scenario, which is more accurate, would
            correspond to 1 standard deviation, or about 15.865\% per
            {\timescale}, which is unacceptably high. However, it is
            not clear why the {\market} is running at a value of
            {\datafractionrms}, which seems very
            conservative. However, a re-investment strategy of
            {\datafractionrms} per {\timescale} does not seem
            inconsistent with a failure rate, on the Fortune 500 list,
            which it is inferred that the {\market} is similar to, of
            about 50\% in ten years, which corresponds to $(1 -
            p_f)^{120} \approx 0.5$, or $p_f$, the probability of
            failure, is $0.005759576$, which is, approximately, 2.5
            standard deviations, meaning that to be consistent with
            the large companies in the Fortune 500, the re-investment
            rate should be, approximately, $\frac{\twoponemax}{2.5}$,
            compared with an operational value, from
            Figure~\ref{\SETLABEL:NH} of {\datafractionrms}.

            An interesting, and intriguing, interpretation and
            discussion of the maximum Shannon probability, is an
            explanation as to why the companies in the {\market} are
            not running an optimal re-investment strategy. This seems
            enigmatic, since those companies that run, on a long term
            average, below the optimally maximal value would seem to
            be eclipsed by those that didn't. And those that run above
            the optimally maximal value would be over extended, and
            become financially destitute during market down turns,
            which is inevitable in a fractal time series as presented
            in Figure~\ref{\SETLABEL:TS}.  It would seem that the
            natural selection process of the competitive environment
            would allow only those companies that run near the
            optimally maximal value to survive, in the long run. One
            possible explanation, foremost, is that the analytical
            methodology presented herein is inappropriate.  Another
            explanation is that the gross margins are less than the
            fraction {\shannonmax} of the rate of revenue returns, and
            thus could not accommodate such an aggressive
            re-investment strategy. If this is the case, then it
            presents an intriguing issue. If, in a capitalistic
            market, the natural outcome of the competitive situation,
            according to game-theoretic analysis, is that there will
            be many competitors, each making minimal gross margins,
            then how do the companies grow their markets?  Naturally,
            those that run the most efficient will have lower costs,
            making larger percentage of rate of revenue returns
            re-investment possible. But an operational Shannon
            probability of {\shannonlogreturns} is not just marginally
            lower than the maximum Shannon probability of
            {\shannonmax}. There is a significant disparity. It would
            seem that the game-theoretic eventual outcome of a
            competitive market place would be a solution that hinders
            growth, wealth and jobs creation, etc., which does not
            seem consistent with capitalistic theory. On the other
            hand, is there an optimum number of competitors in a
            market place, where the gross margins can be higher,
            permitting wealth and job creation, and also a competitive
            situation? If this analysis is correct, and that should be
            subject to scrutiny, then it would appear that this is the
            case. But this brings up another issue---that of taxation,
            and other contributions to the social welfare function. If
            there is an optimum number of competitors in the market
            place, that maximizes wealth and job creation, then,
            perhaps by lemma, there is also an optimal value of
            taxation rate, and other contributions to the social
            welfare function, that will permit maximal industrial
            growth, and thus maximal growth in the tax base. But this
            would seem to be inconsistent with the work of Kenneth
            Arrow and the so called Impossibility Theorem, which
            states that such optimizations can not be optimized
            because the ordering of priorities is intransitive.  All
            very perplexing, since the simulation of the maximum
            Shannon probability in the next section seems to indicate
            that such an aggressive re-investment strategy is, indeed,
            feasible.

        %
% -----------------------------------------------------------------------------
%
% A license is hereby granted to reproduce this software source code and
% to create executable versions from this source code for personal,
% non-commercial use.  The copyright notice included with the software
% must be maintained in all copies produced.
%
% THIS PROGRAM IS PROVIDED "AS IS". THE AUTHOR PROVIDES NO WARRANTIES
% WHATSOEVER, EXPRESSED OR IMPLIED, INCLUDING WARRANTIES OF
% MERCHANTABILITY, TITLE, OR FITNESS FOR ANY PARTICULAR PURPOSE.  THE
% AUTHOR DOES NOT WARRANT THAT USE OF THIS PROGRAM DOES NOT INFRINGE THE
% INTELLECTUAL PROPERTY RIGHTS OF ANY THIRD PARTY IN ANY COUNTRY.
%
% Copyright (c) 1994-2006, John Conover, All Rights Reserved.
%
% Comments and/or bug reports should be addressed to:
%
%     john@email.johncon.com (John Conover)
%
% -----------------------------------------------------------------------------
%
% Revision: \RCSRevision \\
% Revision Time: \RCSTime UMT \\
% Revision Date: \RCSDate \\
% Revision Id: \RCSId \\
% Revision File: \RCSLog \\
\RCS $Revision: 0.0 $
\RCS $Date: 2006/01/20 04:38:13 $
\RCS $Id: maximum.tex,v 0.0 2006/01/20 04:38:13 john Exp $
% $Log: maximum.tex,v $
% Revision 0.0  2006/01/20 04:38:13  john
% Initial version
%
%
    \subsection{Simulation of Fixed Increment Approximation for Optimally Maximal Fiscal Strategy}
        \label{\SETLABEL:MAXSHANNON}
        \subidx{\market}{fiscal strategy, simulation}
        \subidx{\market}{maximum Shannon probability}
        \subidx{markets}{simulation}
        \subidx{simulation}{markets}
        \subidx{strategy}{optimum fiscal, simulation}
        \subidx{fiscal}{optimum strategy, simulation}
        \subidx{programs}{tsunfairbrownian}
        \subidx{tsunfairbrownian}{program}
        \subidx{Shannon}{probability}
        \subidx{probability}{Shannon}

        \subidx{strategy}{exploitable}
        \subidx{exploitable}{strategy}
        \subidx{programs}{tsshannonmax}
        \subidx{tsshannonmax}{program}
        \subidx{programs}{tsunfairbrownian}
        \subidx{tsunfairbrownian}{program}
        \subidx{strategy}{fiscal}
        \subidx{fiscal}{strategy}
        The data in this section is presented in tabular form in
        Section~\ref{\SETLABELREF:MAXSHANNON}. One of the issues of
        analysis, as mentioned in Section~\ref{\SETLABEL:OPS}, is to
        determine the maximum Shannon probability for the time series
        presented in Figure~\ref{\SETLABEL:TS}. Potentially, this
        could be exploited with an aggressive fiscal
        strategy. Figure~\ref{\SETLABEL:SHANNONMAX0} is a graph of the
        output of the {\it tsshannonmax}\/ program, which is described
        briefly in appendix~\ref{programs}. The maximum of this
        function is the maximum Shannon probability for the time
        series data presented in Figure~\ref{\SETLABEL:TS}.
        Figure~\ref{\SETLABEL:SHANNONMAX1} was constructed using {\it
        tsunfairbrownian}\/ program, which is also described in
        appendix~\ref{programs}, with the maximum Shannon probability,
        and the time series data presented in
        Figure~\ref{\SETLABEL:TS}. This represents a ``what if'' the
        investment strategy was changed from a Shannon probability of
        {\shannonlogreturns}, as derived in Section~\ref{\SETLABEL:SP}
        to {\shannonmax}. This process, essentially, extracts the
        random statistical data from the time series presented in
        Figure~\ref{\SETLABEL:TS}, and constructs a new time series,
        using the random statistical data, with a different investment
        strategy.  The program, {\it tsunfairbrownian}\/, essentially,
        constructs the new time series as a Brownian fractal with
        fixed increments.  The ``quality'' of such a reconstruction
        should be subject to adequate scepticism and scrutiny since,
        in all probability, the increments in the original data
        represent a relatively complex process, that may not be
        ``modeled'' with such a simple methodology.

        \begin{figure}[ht]
            \begin{center}
                \begin{minipage}[t]{0.45\textwidth}
                    \epsfxsize=1.0\linewidth
                    \epsffile{\directory/data.tsshannonmax.eps}
                    \caption[{\market}, maximum rate of revenue
                        returns] {{\market}, maximum rate of revenue
                        returns, per {\timescale}, vs. Shannon
                        probability. The maximum rate of revenue
                        returns, per {\timescale}, occurs at a Shannon
                        probability of {\shannonmax}.}
                    \label{\SETLABEL:SHANNONMAX0}
                \end{minipage}
                \hfill
                \begin{minipage}[t]{0.45\textwidth}
                    \epsfxsize=1.0\linewidth
                    \epsffile{\directory/data.tsshannonmax-p.tsunfairbrownian-p.eps}
                    \caption[{\market}, maximum rate of revenue
                        returns] {{\market}, maximum rate of revenue
                        returns, per {\timescale}, at a Shannon
                        probability, of {\shannonmax}, corresponding
                        to a ``wager'' fraction of {\twoponemax}.}
                    \label{\SETLABEL:SHANNONMAX1}
                \end{minipage}
            \end{center}
        \end{figure}

        \subidx{fractional}{Brownian motion}
        \subidx{Brownian motion}{fractional}
        \subidx{Shannon}{probability}
        \subidx{probability}{Shannon}
        \subidx{programs}{tsshannonmax}
        \subidx{tsshannonmax}{program}
        If it is assumed that the time series data set, presented in
        Figure~\ref{\SETLABEL:TS}, constitutes classical Brownian
        motion, then the Shannon probability can be calculated by
        counting the total number of {\timescale}s that the {\market}
        movement was positive, and dividing by the total number of
        {timescale}s represented in the time series. This quotient is
        {\pmax}, as compared with the predicted value from the program
        {\it tsshannonmax}\/ of {\shannonmax}.

% Local Variables:
% TeX-parse-self: t
% TeX-auto-save: t
% TeX-master: "fractal.tex"
% End:


        \subsubsection{Observations on the Simulation of Fixed Increment Approximation for Optimally Maximal Fiscal Strategy}

            Note that these simulations are base on a very, perhaps
            overly, simplified model. For example, from
            Section~\ref{\SETLABEL:TSA}, Figure~\ref{\SETLABEL:NH}, it
            would appear that the {\market}'s normalized increments
            are characterized by fractional Brownian motion---but the
            simulations used classical Brownian motion as the
            model. One consequence of this is that a re-investment
            strategy that is to ``wager'' a fraction of {\twoponemax}
            of the rate of returns every {\timescale} is overly
            aggressive, since in the classical Brownian scenario, the
            maximum loss, in any {\timescale}, was no more that what
            was ``wagered.'' However, in the fractional Brownian
            scenario, much more can be lost. From
            Equation~\ref{fopt2},

            \begin{equation}
                \frac{avg}{rms^2} = \frac{f_{opt}}{rms} = K
            \end{equation}

            \noindent where, under the optimum classical Brownian
            scenario, $K$ is unity, or $avg = rms^2$. Notice that,
            since $f = rms$, whether the scenario is optimal or not,
            that the operational ``wager'' fraction, from
            Figure~\ref{\SETLABEL:TF} of {\datafractionrms}, vs.\ an
            ``theoretical optimal'' value of {\twoponemax} seems
            overly conservative. Additionally, notice that, at least
            in principle, the chance of failure in the fractional
            Brownian scenario, which is more accurate, would
            correspond to 1 standard deviation, or about 15.865\% per
            {\timescale}, which is unacceptably high. However, it is
            not clear why the {\market} is running at a value of
            {\datafractionrms}, which seems very
            conservative. However, a re-investment strategy of
            {\datafractionrms} per {\timescale} does not seem
            inconsistent with a failure rate, on the Fortune 500 list,
            which it is inferred that the {\market} is similar to, of
            about 50\% in ten years, which corresponds to $(1 -
            p_f)^{120} \approx 0.5$, or $p_f$, the probability of
            failure, is $0.005759576$, which is, approximately, 2.5
            standard deviations, meaning that to be consistent with
            the large companies in the Fortune 500, the re-investment
            rate should be, approximately, $\frac{\twoponemax}{2.5}$,
            compared with an operational value, from
            Figure~\ref{\SETLABEL:NH} of {\datafractionrms}.

            An interesting, and intriguing, interpretation and
            discussion of the maximum Shannon probability, is an
            explanation as to why the companies in the {\market} are
            not running an optimal re-investment strategy. This seems
            enigmatic, since those companies that run, on a long term
            average, below the optimally maximal value would seem to
            be eclipsed by those that didn't. And those that run above
            the optimally maximal value would be over extended, and
            become financially destitute during market down turns,
            which is inevitable in a fractal time series as presented
            in Figure~\ref{\SETLABEL:TS}.  It would seem that the
            natural selection process of the competitive environment
            would allow only those companies that run near the
            optimally maximal value to survive, in the long run. One
            possible explanation, foremost, is that the analytical
            methodology presented herein is inappropriate.  Another
            explanation is that the gross margins are less than the
            fraction {\shannonmax} of the rate of revenue returns, and
            thus could not accommodate such an aggressive
            re-investment strategy. If this is the case, then it
            presents an intriguing issue. If, in a capitalistic
            market, the natural outcome of the competitive situation,
            according to game-theoretic analysis, is that there will
            be many competitors, each making minimal gross margins,
            then how do the companies grow their markets?  Naturally,
            those that run the most efficient will have lower costs,
            making larger percentage of rate of revenue returns
            re-investment possible. Yet another interpretation is that
            the number of competitors would grow at an exponential
            rate, but all of them would make minimal returns. However,
            an operational Shannon probability of {\shannonlogreturns}
            is not just marginally lower than the maximum Shannon
            probability of {\shannonmax}. There is a significant
            disparity which is difficult to explain. It would seem
            that the game-theoretic eventual outcome of a competitive
            market place would be a solution that hinders growth,
            wealth and jobs creation, etc., which does not seem
            consistent with capitalistic theory. On the other hand, is
            there an optimum number of competitors in a market place,
            where the gross margins can be higher, permitting wealth
            and job creation, and also a competitive situation? If
            this analysis is correct, and that should be subject to
            scrutiny, then it would appear that this is the case. But
            this brings up another issue---that of taxation, and other
            contributions to the social welfare function. If there is
            an optimum number of competitors in the market place, that
            maximizes wealth and job creation, then, perhaps by lemma,
            there is also an optimal value of taxation rate, and other
            contributions to the social welfare function, that will
            permit maximal industrial growth, and thus maximal growth
            in the tax base. But this would seem to be inconsistent
            with the work of Kenneth Arrow and the so called
            Impossibility Theorem, which states that such
            optimizations can not be determined because the ordering
            of priorities are intransitive.  All very perplexing,
            since the simulation of the maximum Shannon probability in
            the next section seems to indicate that such an aggressive
            re-investment strategy is, indeed, feasible.

            Yet another possibility for the industry not running at
            maximum Shannon probability is the high cost of expansion
            of operations. Some of these industries require very
            sophisticated manufacturing processes, which have high
            barrier costs.

            Additionally, as mentioned in both~\cite[pp. 29]{Brock},
            and~\cite[pp. 8]{Arthur:CTIRALIBHE}, optimal efficiency
            may not be attainable in increasing-return economic
            scenarios.

        %
% -----------------------------------------------------------------------------
%
% A license is hereby granted to reproduce this software source code and
% to create executable versions from this source code for personal,
% non-commercial use.  The copyright notice included with the software
% must be maintained in all copies produced.
%
% THIS PROGRAM IS PROVIDED "AS IS". THE AUTHOR PROVIDES NO WARRANTIES
% WHATSOEVER, EXPRESSED OR IMPLIED, INCLUDING WARRANTIES OF
% MERCHANTABILITY, TITLE, OR FITNESS FOR ANY PARTICULAR PURPOSE.  THE
% AUTHOR DOES NOT WARRANT THAT USE OF THIS PROGRAM DOES NOT INFRINGE THE
% INTELLECTUAL PROPERTY RIGHTS OF ANY THIRD PARTY IN ANY COUNTRY.
%
% Copyright (c) 1994-2006, John Conover, All Rights Reserved.
%
% Comments and/or bug reports should be addressed to:
%
%     john@email.johncon.com (John Conover)
%
% -----------------------------------------------------------------------------
%
% Revision: \RCSRevision \\
% Revision Time: \RCSTime UMT \\
% Revision Date: \RCSDate \\
% Revision Id: \RCSId \\
% Revision File: \RCSLog \\
\RCS $Revision: 0.0 $
\RCS $Date: 2006/01/20 04:38:13 $
\RCS $Id: verification.tex,v 0.0 2006/01/20 04:38:13 john Exp $
% $Log: verification.tex,v $
% Revision 0.0  2006/01/20 04:38:13  john
% Initial version
%
%
    \subsection{Qualitative Verification of Fixed Increment Approximation Analysis}
        \label{\SETLABEL:QVA}

        \subidx{\market}{verification of analysis}
        \subidx{verification}{analysis}
        \subidx{analysis}{verification}
        \subidx{quality}{of analysis}
        \subidx{verification}{of methodology}
        \subidx{methodology}{verification of}
        \subidx{Shannon}{probability}
        \subidx{probability}{Shannon}

        This section evaluates various values based on the ``average''
        of the normalized increments presented in
        Figure~\ref{\SETLABEL:TFA}. These values are an approximation
        to a, probably, complex process with a distribution shown in
        Figure~\ref{\SETLABEL:TF}. These values will be used in a
        fixed increment Brownian fractal analysis of the {\market},
        and may, or may not, provide adequate accuracy for
        projections.

        The data in this section is presented in tabular form in
        sections~\ref{\SETLABELREF:VI1} and~\ref{\SETLABELREF:VI2}.
        As a subjective evaluation of the ``quality'' of the analysis
        of the {\market}, from Chapter~\ref{methodology},
        Equation~\ref{metricvalues1}, and using the mean and root mean
        square values of the normalized increments of the time series
        data presented in Figure~\ref{\SETLABEL:TS} from
        Figure~\ref{\SETLABEL:TF}, and the Shannon probability as
        calculated by counting the total number of {\timescale}s that
        the {\market} movement was positive, as presented in
        Section~\ref{\SETLABEL:MAXSHANNON}:

        \begin{eqnarray}
                  P & \approx & \frac{\frac{avg}{rms} + 1}{2}\\
            {\pmax} & \approx & \frac{\frac{\datafractionmean}{\datafractionrms} + 1}{2}\\
            {\pmax} & \approx & {\avgrms}
            \label{\SETLABEL:AVGS}
        \end{eqnarray}

        \subidx{Shannon}{probability}
        \subidx{probability}{Shannon}
        \noindent and comparing these values to the Shannon
        probability, as found by the {\it tsshannonmax}\/ program, which
        iterates for a maximum:

        \begin{eqnarray}
            {\pmax} \approx {\avgrms} \approx {\shannonmax}
        \end{eqnarray}

        \subidx{logarithmic}{returns}
        \subidx{returns}{logarithmic}
        In addition, the different methods of calculating the
        logarithmic returns, presented in Section~\ref{\SETLABEL:FS},
        should be compared. The four methods used were the mean of
        Figure~\ref{\SETLABEL:TF}, the constant in the least squares
        approximation to Figure~\ref{\SETLABEL:TF}, the least squares
        exponential approximation to Figure~\ref{\SETLABEL:TS}, and
        the logarithmic returns of Figure~\ref{\SETLABEL:TS}, derived
        as the mean of the logarithms of the quotients of the
        increments. The values for each of the methods are,
        respectively:

        \begin{equation}
            \datafractionmeanbits \approx \datafractionconstantbits \approx \datatslsqepbits \approx \logreturns
        \end{equation}

        It is implied in Section~\ref{\SETLABEL:FS},
        Subsection~\ref{\SETLABEL:SP} and in
        Section~\ref{\SETLABEL:TSUNFAIRBROWNIAN} that, a Brownian
        motion with fixed increments fractal may ``model'' the
        {\market}. Using Equation~\ref{stddev9} from
        Chapter~\ref{general}, Section~\ref{abmfi}:

        \begin{eqnarray}
                                    rms \left(2P - 1\right) & \approx & \frac{\sigma \left(2P - 1\right)}{2 \sqrt{P\left(1 - P\right)}}\\
            \datafractionrms \left(2 \cdot \pmax - 1\right) & \approx & \frac{\datafractionstddev \left(2 \cdot \pmax - 1\right)}{2\sqrt{\pmax \left(1 - \pmax\right)}}\\
                       \datafractionrms \cdot \twopminusone & \approx & \datafractionstddev \cdot \twopx\\
                                                      \rmsp & \approx & \sigmap
        \end{eqnarray}

        \noindent and, equating to the mean:

        \begin{equation}
            \datafractionmean \approx \rmsp \approx \sigmap
        \end{equation}

        \subidx{Shannon}{probability}
        \subidx{probability}{Shannon}
        \noindent where, as in Equation~\ref{\SETLABEL:AVGS} using the
        mean, root mean square, and standard deviation values of the
        normalized increments of the time series data presented in
        Figure~\ref{\SETLABEL:TS} from Figure~\ref{\SETLABEL:TF}, and
        the Shannon probability as calculated by counting the total
        number of {\timescale}s that the {\market} movement was
        positive, as presented in Section~\ref{\SETLABEL:MAXSHANNON}.

        As a final qualitative comparison, the absolute value of the
        normalized increments should be the same as the root mean
        square value\footnote{The absolute value of the normalized
        increments, when averaged, is related to the root mean square
        of the increments by a constant. If the normalized increments
        are a fixed increment, the constant is unity. If the
        normalized increments have a Gaussian distribution, the
        constant is $\approx 0.8$ depending on the accuracy of of
        ``fit'' to a Gaussian distribution.}, where the absolute value
        is presented in Figure~\ref{\SETLABEL:TFA}, and the root mean
        square value is presented in Figure~\ref{\SETLABEL:TF}:

        \begin{equation}
            \datafractionabsmean \approx \datafractionrms
        \end{equation}

        Note, that if the {\market} could be ``modeled'' as a Brownian
        motion with fixed increments fractal, then the standard
        deviation of the absolute value of the normalized increments
        of the time series data presented in Figure~\ref{\SETLABEL:TS}
        from Figure~\ref{\SETLABEL:TF} should be zero. It is
        $\datafractionabsstddev$.

% Local Variables:
% TeX-parse-self: t
% TeX-auto-save: t
% TeX-master: "fractal.tex"
% End:


    \renewcommand{\market}{United States Electronic Component Shipments}
    \renewcommand{\directory}{../markets/electronic.components.shipments}
    \renewcommand{\datafractionmean}{0.008052}
\renewcommand{\datafractionmeanbits}{0.011570}
\renewcommand{\datafractionmeanq}{0.002684}
\renewcommand{\datafractionmeanbitsq}{0.003867}
\renewcommand{\datafractionstddev}{0.038579}
\renewcommand{\datafractionrms}{0.039311}
\renewcommand{\avgrms}{0.602414}
\renewcommand{\ncompanies}{5.210454}
\renewcommand{\pncompanies}{0.544866}
\renewcommand{\datafractionabsmean}{0.029745}
\renewcommand{\datafractionabsstddev}{0.025769}
\renewcommand{\datafractionconstant}{0.010041}
\renewcommand{\datafractionconstantbits}{0.014414}
\renewcommand{\datafractionconstantq}{0.003347}
\renewcommand{\datafractionconstantbitsq}{0.004821}
\renewcommand{\datafractionslope}{-0.000021}
\renewcommand{\datafractionabsconstant}{0.035145}
\renewcommand{\datafractionabsslope}{-0.000057}
\renewcommand{\hurstall}{0.659558}
\renewcommand{\hurstlow}{0.707509}
\renewcommand{\hurstlowtwo}{1.415018}
\renewcommand{\hurstlowhundred}{70.750900}
\renewcommand{\hcalcall}{0.184942}
\renewcommand{\hcalclow}{0.102042}
\renewcommand{\shannonmax}{0.604167}
\renewcommand{\twoponemax}{0.208334}
\renewcommand{\logreturns}{0.010456}
\renewcommand{\twologreturns}{1.007274}
\renewcommand{\twologreturnshundred}{0.727387}
\renewcommand{\oneoverlogreturns}{95.638868}
\renewcommand{\pmax}{0.602094}
\renewcommand{\twopminusone}{0.204188}
\renewcommand{\rmsp}{0.008027}
\renewcommand{\twopx}{0.208583}
\renewcommand{\sigmap}{0.008047}
\renewcommand{\tsunfairbrownianfractionmean}{0.007862}
\renewcommand{\tsunfairbrownianfractionstddev}{0.038619}
\renewcommand{\shannonlogreturns}{0.560125}
\renewcommand{\shannonlogreturnshundred}{56.012500}
\renewcommand{\twopone}{0.120250}
\renewcommand{\twoponehundred}{12.025000}
\renewcommand{\hundredtwoponehundred}{87.975000}
\renewcommand{\hundredshannonlogreturnshundred}{43.987500}
\renewcommand{\datatslsqepbits}{0.007623}
\renewcommand{\thurstall}{0.633980}
\renewcommand{\thurstlow}{0.710108}
\renewcommand{\thurstlowtwo}{1.420216}
\renewcommand{\thurstlowhundred}{71.010800}
\renewcommand{\thcalcall}{0.247886}
\renewcommand{\thcalclow}{0.171737}
\renewcommand{\chisquared}{2.862000}
\renewcommand{\critical}{42.557000}

    \renewcommand{\timescale}{month}
    \subidx{market}{\market}
    \idx{\market}

    \section{\market}

        \renewcommand{\SETLABEL}{\LABPRE:NAECS}
        \renewcommand{\SETLABELQ}{\LABPRE:NAECSQ}
        \label{\SETLABEL}
        \renewcommand{\SETLABELREF}{\LABPREREF:NAECS}

        \idx{United States Department of Commerce}
        For the analysis, the data was in the directory
        {\directory}\footnote{Data from the United States Department
        of Commerce, 1979---1994, by {\timescale}s, in millions of
        dollars, US.}.

        The data in this section is presented in tabular form in
        Section~\ref{\SETLABELREF}.

        %
% -----------------------------------------------------------------------------
%
% A license is hereby granted to reproduce this software source code and
% to create executable versions from this source code for personal,
% non-commercial use.  The copyright notice included with the software
% must be maintained in all copies produced.
%
% THIS PROGRAM IS PROVIDED "AS IS". THE AUTHOR PROVIDES NO WARRANTIES
% WHATSOEVER, EXPRESSED OR IMPLIED, INCLUDING WARRANTIES OF
% MERCHANTABILITY, TITLE, OR FITNESS FOR ANY PARTICULAR PURPOSE.  THE
% AUTHOR DOES NOT WARRANT THAT USE OF THIS PROGRAM DOES NOT INFRINGE THE
% INTELLECTUAL PROPERTY RIGHTS OF ANY THIRD PARTY IN ANY COUNTRY.
%
% Copyright (c) 1994-2006, John Conover, All Rights Reserved.
%
% Comments and/or bug reports should be addressed to:
%
%     john@email.johncon.com (John Conover)
%
% -----------------------------------------------------------------------------
%
% Revision: \RCSRevision \\
% Revision Time: \RCSTime UMT \\
% Revision Date: \RCSDate \\
% Revision Id: \RCSId \\
% Revision File: \RCSLog \\
\RCS $Revision: 0.0 $
\RCS $Date: 2006/01/20 04:38:13 $
\RCS $Id: fraction.tex,v 0.0 2006/01/20 04:38:13 john Exp $
% $Log: fraction.tex,v $
% Revision 0.0  2006/01/20 04:38:13  john
% Initial version
%
%
    \subsection{Time Series Increments Analysis}
        \label{\SETLABEL:TSA}

        \subidx{\market}{Time series analysis}
        \subidx{time series}{increments}
        \subidx{time series}{analysis}
        \subidx{cumulative sum}{analysis}
        \subidx{analysis}{cumulative sum}
        \subidx{analysis}{random process}
        \subidx{random process}{analysis}
        \subidx{Gaussian}{increments}
        \subidx{increments}{Gaussian}
        \subidx{Brownian}{motion, fractional}
        \subidx{fractional}{Brownian motion}
        \subidx{fractal}{Brownian motion}
        The data in this section is presented in tabular form in
        Section~\ref{\SETLABELREF:TSA}.  Figure~\ref{\SETLABEL:TS} is
        a graph of the time series data for the {\market}.

        \subidx{increments}{normalized}
        \subidx{normalized}{increments}
        \subidx{programs}{tsfraction}
        \subidx{tsfraction}{program}
        Figure~\ref{\SETLABEL:TF} is a graph of the normalized
        increments of the time series data presented in
        Figure~\ref{\SETLABEL:TS}. The data presented was made by
        running the program {\it tsfraction}\/ on the time series
        data. The program {\it tsfraction}\/ is described briefly in
        Appendix~\ref{programs}, and subtracts the previous value from
        the next value, dividing this difference by the previous
        value, for each element in the time series data. The new time
        series contains the instantaneous change in the rate of
        revenue returns, divided by the magnitude of the instantaneous
        rate of revenue returns.

        \subidx{mean}{standard deviation}
        \subidx{standard deviation}{mean}
        \idx{root mean square}
        \idx{least squares approximation}
        \begin{figure}[ht]
            \begin{center}
                \begin{minipage}[t]{0.45\textwidth}
                    \epsfxsize=1.0\linewidth
                    \epsffile{\directory/data.eps}
                    \caption{{\market}, time series data.}
                    \label{\SETLABEL:TS}
                    \label{\SETLABELQ:TS}
                \end{minipage}
                \hfill
                \begin{minipage}[t]{0.45\textwidth}
                    \epsfxsize=1.0\linewidth
                    \epsffile{\directory/data.tsfraction.eps}
                    \caption[{\market}, normalized
                        increments]{{\market}, normalized increments
                        of the time series data presented in
                        Figure~\ref{\SETLABEL:TS}. The mean is
                        {\datafractionmean} with a standard deviation
                        of {\datafractionstddev}. The formula for the
                        least squares approximation is
                        ${\datafractionconstant} +
                        {\datafractionslope}t$, and the root mean
                        squared value is {\datafractionrms}. The
                        graph, labeled ``data\-.tsfraction\-.tsrms,''
                        is the running root mean square, and
                        ``data\-.tsfraction\-.tsavg'' is the running
                        average of the normalized increments.  This
                        graph is the fraction of change in the time
                        series, as a function of time. Note that the
                        slope of the mean, {\datafractionslope}, is
                        the coefficient of the nonlinearity term in
                        the normalized increments. See
                        Chapter~\ref{general}, Section~\ref{nlextend}
                        for a possible application of the logistic
                        function to this data set.}
                    \label{\SETLABEL:TF}
                    \label{\SETLABELQ:TF}
                \end{minipage}
            \end{center}
        \end{figure}

        \subidx{absolute value}{increments}
        \subidx{increments}{absolute value}

        Figure~\ref{\SETLABEL:TFA} is a graph of the absolute value of
        the normalized increments of the time series data presented in
        Figure~\ref{\SETLABEL:TF}. The data presented was made by
        running the Unix utility sed(1) on the normalized increments
        time series data to remove the negative signs. This is an
        absolute value procedure.  The resulting time series contains
        the absolute value of the instantaneous change in the rate of
        revenue returns, divided by the magnitude of the instantaneous
        rate of revenue returns\footnote{The absolute value of the
        normalized increments, when averaged, is related to the root
        mean square of the increments by a constant. If the normalized
        increments are a fixed increment, the constant is unity. If
        the normalized increments have a Gaussian distribution, the
        constant is $\approx 0.8$ depending on the accuracy of of
        ``fit'' to a Gaussian distribution.}.

        \subidx{histogram}{normalized}
        \subidx{normalized}{histogram}
        \subidx{programs}{tsnormal}
        \subidx{tsnormal}{program}
        \subidx{mean}{standard deviation}
        \subidx{standard deviation}{mean}
        \idx{root mean square}
        \idx{least squares approximation}
        \subidx{\market}{analysis of increments}
        Figure~\ref{\SETLABEL:NH} is the normalized histogram of the
        normalized increments of the time series data shown in
        Figure~\ref{\SETLABEL:TF}. The abscissa is 3 $\sigma$ limits,
        and the area under the two curves is identical. The data for
        this figure was produced by the program {\it tsnormal}\/,
        which is described briefly in Appendix~\ref{programs}.

        \begin{figure}[ht]
            \begin{center}
                \begin{minipage}[t]{0.45\textwidth}
                    \epsfxsize=1.0\linewidth
                    \epsffile{\directory/data.tsfraction.abs.eps}
                    \caption[{\market}, absolute value of the
                        normalized increments]{{\market}, absolute
                        value of the normalized increments of the time
                        series data presented in
                        Figure~\ref{\SETLABEL:TF}.  The mean is
                        {\datafractionabsmean} with a standard
                        deviation of {\datafractionabsstddev}. The
                        formula for the least squares approximation is
                        ${\datafractionabsconstant} +
                        {\datafractionabsslope}t$, and the root mean
                        square value, from Figure~\ref{\SETLABEL:TF},
                        is {\datafractionrms}.  The graph, labeled
                        ``data\-.tsfraction\-.tsrms,'' is the running
                        root mean square, and
                        ``data\-.tsfraction\-.tsavg'' is the running
                        average of the normalized increments presented
                        in Figure~\ref{\SETLABEL:TF}, superimposed
                        here for convenience. This graph is the
                        absolute value of the fraction of change in
                        the time series, as a function of time.}
                    \label{\SETLABEL:TFA}
                    \label{\SETLABELQ:TFA}
                \end{minipage}
                \hfill
                \begin{minipage}[t]{0.45\textwidth}
                    \epsfxsize=1.0\linewidth
                    \epsffile{\directory/data.tsfraction.tsnormal-s30.eps}
                    \caption[{\market}, normalized histogram of the
                        normalized increments]{{\market}, normalized
                        histogram of the normalized increments of the
                        time series data shown in
                        Figure~\ref{\SETLABEL:TF}.  The data has a
                        mean of {\datafractionmean}, with a standard
                        deviation of {\datafractionstddev}.  The area
                        under the two curves is identical. The
                        $\chi^2$ value of the observed and expected
                        values of the two curves is {\chisquared},
                        with a critical value of {\critical}.}
                    \label{\SETLABEL:NH}
                \end{minipage}
            \end{center}
        \end{figure}

        \subidx{programs}{tsXsquared}
        \subidx{tsXsquared}{program}
        \subidx{\market}{chi-squared values of increments}
        The program {\it tsXsquared}\/, which is briefly described in
        appendix~\ref{programs}, was used to derive the $\chi^2$
        statistics for the data presented in
        Figure~\ref{\SETLABEL:NH}.

        \subidx{programs}{tsstatest}
        \subidx{tsstatest}{program}
        \subidx{\market}{statistical estimates}

        Figure~\ref{\SETLABEL:SE} is the statistical estimate for the
        data presented in Figure~\ref{\SETLABEL:TF}, as derived by the
        program {\it tsstatest}\/, which is briefly described in
        appendix~\ref{programs}.

        \begin{figure}[ht]
            \begin{center}
                \begin{minipage}[t]{\textwidth}
                    \center{\fbox{\parbox{0.9\textwidth}{\XXX{\directory/data.tsstatest-f0.1-c0.9-i.tex}}}}
                    \caption[{\market}, statistical estimates of the
                        normalized increments]{{\market}, statistical
                        estimates of the normalized increments of the
                        time series shown in Figure~\ref{\SETLABEL:TF}.
                        The table was produced with the {\it
                        tsstatest}\/ program, and illustrates the
                        size of the data set required for a confidence
                        level of 90\%, with an error estimate of $\pm$
                        10\%, or alternately, the error estimate on
                        the time series shown in Figure~\ref{\SETLABEL:TF}.}
                    \label{\SETLABEL:SE}
                \end{minipage}
            \end{center}
        \end{figure}

        Note that the data set size estimations, as produced by the
        {\it tsstatest}\/ program, are probably very conservative,
        depending on the magnitude of the Shannon probability, $P =
        \shannonlogreturns$, as derived in
        Section~\ref{\SETLABEL:SP}. See Chapter~\ref{general},
        Section~\ref{serdss} for possible alternative methodologies
        for addressing the analysis of fractal time series with
        limited data set sizes. Depending on the magnitude of the
        Shannon probability, $P$, these estimates can be several
        orders of magnitude too high.

        \subidx{derivative of increments}{normalized}
        \subidx{normalized}{derivative of increments}
        \subidx{programs}{tsderivative}
        \subidx{tsderivative}{program}
        Figure~\ref{\SETLABEL:TF1} is the normalized histogram of the
        first derivative of the normalized increments of the time
        series data shown in Figure~\ref{\SETLABEL:TF}. In principle,
        if the distribution of the normalized increments presented in
        Figure~\ref{\SETLABEL:NH} is Gaussian in nature, this
        distribution would be similar to ``white noise,'' as presented
        in appendix~\ref{programs}, Figure~\ref{whiteexample}. The
        data was generated by the {\it tsderivative}\/ program, which
        is briefly described in
        appendix~\ref{programs}. Figure~\ref{\SETLABEL:TF2} is the
        normalized histogram of the second derivative of the
        normalized increments of the time series data shown in
        Figure~\ref{\SETLABEL:TF}. In principle, if the distribution
        of the normalized increments presented in
        Figure~\ref{\SETLABEL:NH} is an integrated Gaussian
        distribution in nature, this distribution would be similar to
        ``white noise,'' as presented in appendix~\ref{programs},
        Figure~\ref{whiteexample}.

        \begin{figure}[ht]
            \begin{center}
                \begin{minipage}[t]{0.45\textwidth}
                    \epsfxsize=1.0\linewidth
                    \epsffile{\directory/data.tsfraction.tsderivative.tsnormal-s30.eps}
                    \caption[{\market}, histogram of the first
                        derivative of the increments]{{\market},
                        normalized histogram of the first derivative
                        of the normalized increments of the time
                        series data shown in
                        Figure~\ref{\SETLABEL:TF}.}
                    \label{\SETLABEL:TF1}
                \end{minipage}
                \hfill
                \begin{minipage}[t]{0.45\textwidth}
                    \epsfxsize=1.0\linewidth
                    \epsffile{\directory/data.tsfraction.2tsderivative.tsnormal-s30.eps}
                    \caption[{\market}, histogram of the second
                        derivative of the increments]{{\market},
                        normalized histogram of second derivative of
                        the the normalized increments of the time
                        series data shown in
                        Figure~\ref{\SETLABEL:TF}.}
                    \label{\SETLABEL:TF2}
                \end{minipage}
            \end{center}
        \end{figure}

        \subidx{fractal}{range}
        \subidx{fractal}{R/S analysis}
        \subidx{\market}{rate of revenue returns, range}
        \subidx{\market}{deterministic mechanism}
        \subidx{deterministic}{mechanism}
        \subidx{mechanism}{deterministic}
        Figure~\ref{\SETLABEL:TR} is the range of values of the time
        series shown in Figure~\ref{\SETLABEL:TS}. The horizontal axis
        is time into the future. In principle, if the time series was
        characterized as fractional Brownian motion the graph in
        Figure~\ref{\SETLABEL:TR} would be a square root
        function\footnote{Note that the ``roughness,'' or ``sawtooth''
        characteristics of the graph in Figure~\ref{\SETLABEL:TR} are
        a computational artifact---caused by not using the -m option
        to the program {\it tshurst}\/, which is computationally
        inefficient.}. Figure~\ref{\SETLABEL:TD} is the deterministic
        map of the normalized increments of the time series data shown
        in Figure~\ref{\SETLABEL:TF}. The deterministic map is useful
        for determining if a time series was created by a
        deterministic mechanism. This, essentially, maps each element
        in the time series with the previous element in the time
        series.  See,~\cite[pp. 745]{Peitgen}.

        \begin{figure}[ht]
            \begin{center}
                \begin{minipage}[t]{0.45\textwidth}
                    \epsfxsize=1.0\linewidth
                    \epsffile{\directory/data.tshurst-f.eps}
                    \caption[{\market}, range]{{\market}, range of the
                        time series data shown in
                        Figure~\ref{\SETLABEL:TS}.}
                    \label{\SETLABEL:TR}
                \end{minipage}
                \hfill
                \begin{minipage}[t]{0.45\textwidth}
                    \epsfxsize=1.0\linewidth
                    \epsffile{\directory/data.tsfraction.tsdeterministic.eps}
                    \caption[{\market}, deterministic map]{{\market},
                        deterministic map of the normalized increments
                        of the time series data shown in
                        Figure~\ref{\SETLABEL:TF}.}
                    \label{\SETLABEL:TD}
                \end{minipage}
            \end{center}
        \end{figure}

% Local Variables:
% TeX-parse-self: t
% TeX-auto-save: t
% TeX-master: "fractal.tex"
% End:


        \subsubsection{Observations on the Time Series Increments Analysis}

            Figure~\ref{\SETLABEL:NH} would seem to indicate that the
            time series data for the {\market} represents a cumulative
            sum/integration of a random process that has a Gaussian
            distribution, (ie., satisfies the Gaussian increments
            property of fractional Brownian
            motion~\cite[pp. 250]{Crownover},) tending to justify the
            assumption that the time series data represents fractional
            Brownian motion.

        %
% -----------------------------------------------------------------------------
%
% A license is hereby granted to reproduce this software source code and
% to create executable versions from this source code for personal,
% non-commercial use.  The copyright notice included with the software
% must be maintained in all copies produced.
%
% THIS PROGRAM IS PROVIDED "AS IS". THE AUTHOR PROVIDES NO WARRANTIES
% WHATSOEVER, EXPRESSED OR IMPLIED, INCLUDING WARRANTIES OF
% MERCHANTABILITY, TITLE, OR FITNESS FOR ANY PARTICULAR PURPOSE.  THE
% AUTHOR DOES NOT WARRANT THAT USE OF THIS PROGRAM DOES NOT INFRINGE THE
% INTELLECTUAL PROPERTY RIGHTS OF ANY THIRD PARTY IN ANY COUNTRY.
%
% Copyright (c) 1994-2006, John Conover, All Rights Reserved.
%
% Comments and/or bug reports should be addressed to:
%
%     john@email.johncon.com (John Conover)
%
% -----------------------------------------------------------------------------
%
% Revision: \RCSRevision \\
% Revision Time: \RCSTime UMT \\
% Revision Date: \RCSDate \\
% Revision Id: \RCSId \\
% Revision File: \RCSLog \\
\RCS $Revision: 0.0 $
\RCS $Date: 2006/01/20 04:38:13 $
\RCS $Id: instant.tex,v 0.0 2006/01/20 04:38:13 john Exp $
% $Log: instant.tex,v $
% Revision 0.0  2006/01/20 04:38:13  john
% Initial version
%
%
    \subsection{Instantaneous Analysis of Normalized Increments}
        \label{\SETLABEL:IA}

        \subidx{\market}{instantaneous analysis of normalized increments}
        \idx{average of normalized increments}
        \idx{root mean square of normalized increments}
        \subidx{Shannon probability}{instantaneous computation of}
        \subidx{average of normalized increments}{instantaneous computation of}
        \subidx{root mean square of normalized increments}{instantaneous computation of}
        \subidx{instantaneous computation}{Shannon probability}
        \subidx{instantaneous computation}{average of normalized increments}
        \subidx{instantaneous computation}{root mean square of normalized increments}
        \idx{time series}
        \subidx{time series}{instantaneous analysis}
        \subidx{instantaneous analysis}{time series}
        \subidx{time series}{increments}
        \subidx{time series}{analysis}
        \subidx{Shannon}{probability}
        \subidx{probability}{Shannon}
        \subidx{normalized}{increments}
        \subidx{increments}{normalized}

        The program {\it tsinstant}\/, which is briefly described in
        Appendix~\ref{programs}, is for finding the instantaneous
        fraction of change in a time series. The value of a sample in
        the time series is subtracted from the previous sample in the
        time series, and divided by the value of the previous sample.
        As explained in Chapter~\ref{general},
        Sections~\ref{derivation},~\ref{GA},~\ref{abmfi},~\ref{aftsma}
        and,~\ref{ompl} for Brownian motion, random walk fractals, the
        absolute value of the instantaneous fraction of change is also
        the root mean square of the instantaneous fraction of
        change\footnote{The absolute value of the normalized
        increments, when averaged, is related to the root mean square
        of the increments by a constant. If the normalized increments
        are a fixed increment, the constant is unity. If the
        normalized increments have a Gaussian distribution, the
        constant is $\approx 0.8$ depending on the accuracy of of
        ``fit'' to a Gaussian distribution.}. Squaring this value is
        the average of the instantaneous fraction of change, and
        adding unity to the absolute value of the instantaneous
        fraction of change, and dividing by two, is the Shannon
        probability of the instantaneous fraction of change.

        Figure~\ref{\SETLABEL:IA1} is the instantaneous value of the
        root mean square of the normalized increments for the
        {\market}, and Figure~\ref{\SETLABEL:IA2} is the instantaneous
        Shannon probability for the normalized increments.

        \begin{figure}[ht]
            \begin{center}
                \begin{minipage}[t]{0.45\textwidth}
                    \epsfxsize=1.0\linewidth
                    \epsffile{\directory/data.tsinstant-r.eps}
                    \caption[{\market}, instantaneous value of
                        rms.]{{\market}, instantaneous value of the
                        root mean square of the normalized increments,
                        provided by running the program {\it
                        tsinstant}\/ with the -r option on the data
                        presented in Figure~\ref{\SETLABEL:TS}.}
                    \label{\SETLABEL:IA1}
                    \label{\SETLABELQ:IA1}
                \end{minipage}
                \hfill
                \begin{minipage}[t]{0.45\textwidth}
                    \epsfxsize=1.0\linewidth
                    \epsffile{\directory/data.tsinstant-s.eps}
                    \caption[{\market}, instantaneous value of
                        Shannon probability.]{{\market}, instantaneous
                        value of the Shannon probability of the
                        normalized increments, provided by running the
                        program {\it tsinstant}\/ with the -s option
                        on the data presented in
                        Figure~\ref{\SETLABEL:TS}.}
                    \label{\SETLABEL:IA2}
                    \label{\SETLABELQ:IA2}
                \end{minipage}
            \end{center}
        \end{figure}

% Local Variables:
% TeX-parse-self: t
% TeX-auto-save: t
% TeX-master: "fractal.tex"
% End:


        %
% -----------------------------------------------------------------------------
%
% A license is hereby granted to reproduce this software source code and
% to create executable versions from this source code for personal,
% non-commercial use.  The copyright notice included with the software
% must be maintained in all copies produced.
%
% THIS PROGRAM IS PROVIDED "AS IS". THE AUTHOR PROVIDES NO WARRANTIES
% WHATSOEVER, EXPRESSED OR IMPLIED, INCLUDING WARRANTIES OF
% MERCHANTABILITY, TITLE, OR FITNESS FOR ANY PARTICULAR PURPOSE.  THE
% AUTHOR DOES NOT WARRANT THAT USE OF THIS PROGRAM DOES NOT INFRINGE THE
% INTELLECTUAL PROPERTY RIGHTS OF ANY THIRD PARTY IN ANY COUNTRY.
%
% Copyright (c) 1994-2006, John Conover, All Rights Reserved.
%
% Comments and/or bug reports should be addressed to:
%
%     john@email.johncon.com (John Conover)
%
% -----------------------------------------------------------------------------
%
% Revision: \RCSRevision \\
% Revision Time: \RCSTime UMT \\
% Revision Date: \RCSDate \\
% Revision Id: \RCSId \\
% Revision File: \RCSLog \\
\RCS $Revision: 0.0 $
\RCS $Date: 2006/01/20 04:38:13 $
\RCS $Id: logistic.tex,v 0.0 2006/01/20 04:38:13 john Exp $
% $Log: logistic.tex,v $
% Revision 0.0  2006/01/20 04:38:13  john
% Initial version
%
%
    \subsection{Logistic Analysis}
        \label{\SETLABEL:LA}

        \subidx{\market}{Logistic function analysis}
        \subidx{time series}{logistic function}
        \subidx{logistic function}{time series}
        \subidx{time series}{increments}
        \subidx{time series}{analysis}
        \subidx{cumulative sum}{analysis}
        \subidx{analysis}{cumulative sum}
        \subidx{analysis}{random process}
        \subidx{random process}{analysis}
        The data in this section is presented in tabular form in
        Section~\ref{\SETLABELREF:LAA}.  Figure~\ref{\SETLABEL:LA1} is
        a graph of the logistic function estimates of the time series
        data for the {\market}. The reader is cautioned that these
        graphs are constructed using the method suggested in
        Chapter~\ref{general}, Section~\ref{nlextend} and enormous
        precision is required for adequate prediction of the logistic
        function,~\cite{Modis}. Particularly, the non-linear term will
        usually require intervention to produce a practical fit to the
        data. In addition, there are numerical stability issues with
        logistic function methodologies\footnote{For example, in
        Figures~\ref{\SETLABEL:LA1} and~\ref{\SETLABEL:LA2}, if the
        non-linear term, $b$, was greater than zero, it was set to
        zero to produce the graphs. See Section~\ref{\SETLABELREF:LAA}
        for the actual derived values. In other cases, the magnitude
        of $b$ was too large, resulting in a graph that was decreasing
        as a function of time}.  The methodology should be regarded as
        ``fragile.'' It is included for completeness.

        \idx{least squares approximation}
        Figure~\ref{\SETLABEL:LA1} is a graph of the logistic function
        for the time series data presented in
        Figure~\ref{\SETLABEL:TS}. The data presented was made by
        running the program {\it tsdlogistic}\/, which is described
        briefly in Appendix~\ref{programs}, on the parameters
        extracted from the time series data as suggested in
        Figure~\ref{\SETLABEL:TF}. The program {\it tslsq}\/ was used
        to derive the constant and the slope of the normalized
        increments of the data presented in Figure~\ref{\SETLABEL:TF}.
        Figure~\ref{\SETLABEL:LA2} is the same graph, but with the
        time scale expanded by a factor of two.

        \begin{figure}[ht]
            \begin{center}
                \begin{minipage}[t]{0.45\textwidth}
                    \epsfxsize=1.0\linewidth
                    \epsffile{\directory/data.tsfraction.tslsq-p.tsdlogistic.eps}
                    \caption[{\market}, logistic function
                        estimates.]{{\market}, logistic function
                        estimates, provided by running the {\it
                        tslsq}\/ program on the normalized increments
                        presented in Figure~\ref{\SETLABEL:TF} with
                        the -p option. These parameters were used as
                        arguments to the {\it tsdlogistic}\/ program.}
                    \label{\SETLABEL:LA1}
                    \label{\SETLABELQ:LA1}
                \end{minipage}
                \hfill
                \begin{minipage}[t]{0.45\textwidth}
                    \epsfxsize=1.0\linewidth
                    \epsffile{\directory/data.tsfraction.tslsq-p.tsdlogistic2.eps}
                    \caption[{\market}, logistic function
                        estimates.]{{\market}, logistic function
                        estimates of Figure~\ref{\SETLABEL:LA1} with
                        the time scale expanded by a factor of two.}
                    \label{\SETLABEL:LA2}
                    \label{\SETLABELQ:LA2}
                \end{minipage}
            \end{center}
        \end{figure}

% Local Variables:
% TeX-parse-self: t
% TeX-auto-save: t
% TeX-master: "fractal.tex"
% End:


        %
% -----------------------------------------------------------------------------
%
% A license is hereby granted to reproduce this software source code and
% to create executable versions from this source code for personal,
% non-commercial use.  The copyright notice included with the software
% must be maintained in all copies produced.
%
% THIS PROGRAM IS PROVIDED "AS IS". THE AUTHOR PROVIDES NO WARRANTIES
% WHATSOEVER, EXPRESSED OR IMPLIED, INCLUDING WARRANTIES OF
% MERCHANTABILITY, TITLE, OR FITNESS FOR ANY PARTICULAR PURPOSE.  THE
% AUTHOR DOES NOT WARRANT THAT USE OF THIS PROGRAM DOES NOT INFRINGE THE
% INTELLECTUAL PROPERTY RIGHTS OF ANY THIRD PARTY IN ANY COUNTRY.
%
% Copyright (c) 1994-2006, John Conover, All Rights Reserved.
%
% Comments and/or bug reports should be addressed to:
%
%     john@email.johncon.com (John Conover)
%
% -----------------------------------------------------------------------------
%
% Revision: \RCSRevision \\
% Revision Time: \RCSTime UMT \\
% Revision Date: \RCSDate \\
% Revision Id: \RCSId \\
% Revision File: \RCSLog \\
\RCS $Revision: 0.0 $
\RCS $Date: 2006/01/20 04:38:13 $
\RCS $Id: hurst.tex,v 0.0 2006/01/20 04:38:13 john Exp $
% $Log: hurst.tex,v $
% Revision 0.0  2006/01/20 04:38:13  john
% Initial version
%
%
    \subsection{Hurst Coefficient Analysis}
        \label{\SETLABEL:H}

        \subidx{\market}{Hurst coefficient analysis}
        \subidx{Hurst coefficient}{analysis}
        \subidx{increments}{normalized}
        \subidx{normalized}{increments}
        \subidx{programs}{tshurst}
        \subidx{tshurst}{program}
        The data in this section is presented in tabular form in
        Section~\ref{\SETLABELREF:HCHP}. Figure~\ref{\SETLABEL:HC} is
        a graph of the Hurst coefficient data time series data shown
        in Figure~\ref{\SETLABEL:TS}. The slope of the graph is the
        Hurst coefficient.  The data for this figure was produced by
        the program {\it tshurst}\/, which is described briefly in
        Appendix~\ref{programs}.

        \subidx{\market}{H parameter analysis}
        \subidx{H parameter}{analysis}
        \subidx{programs}{tshcalc}
        \subidx{tshcalc}{program}
        Figure~\ref{\SETLABEL:HP} is a graph of the H parameter data
        for the normalized increments of the time series data shown in
        Figure~\ref{\SETLABEL:TF}. The data for this figure was
        produced by the program {\it tshcalc}\/, which is described
        briefly in Appendix~\ref{programs}.

        \begin{figure}[ht]
            \begin{center}
                \begin{minipage}[t]{0.45\textwidth}
                    \epsfxsize=1.0\linewidth
                    \epsffile{\directory/data.tshurst.eps}
                    \caption[{\market}, Hurst coefficient data]{{\market},
                        Hurst coefficient data for the normalized
                        increments of the time series data shown in
                        Figure~\ref{\SETLABEL:TF}.  The slope of the graph
                        is the Hurst coefficient.}
                    \label{\SETLABEL:HC}
                \end{minipage}
                \hfill
                \begin{minipage}[t]{0.45\textwidth}
                    \epsfxsize=1.0\linewidth
                    \epsffile{\directory/data.tshcalc.eps}
                    \caption[{\market}, H parameter data]{{\market}, H
                        parameter data for the normalized increments of
                        the time series data shown in
                        Figure~\ref{\SETLABEL:TF} The slope of the graph
                        is the H parameter.}
                    \label{\SETLABEL:HP}
                \end{minipage}
            \end{center}
        \end{figure}

        \subidx{revenue}{See, rate of revenue returns}
        \subidx{returns}{See, rate of revenue returns}
        \subidx{\market}{revenues}
        \subidx{Hurst coefficient}{analysis}
        \subidx{\market}{Hurst coefficient analysis}
        \subidx{\market}{rate of change}
        \subidx{\market}{windows of opportunity}
        \subidx{rate of revenue returns}{forecast}
        \subidx{forecast}{rate of revenue returns}
        \idx{windows of opportunity}
        \subidx{programs}{tslsq}
        \subidx{tslsq}{program}

        The approximately linear slope of the graph in
        Figure~\ref{\SETLABEL:HC} implies that the variance of the
        rate of revenue returns, (per {\timescale},) in the {\market},
        $V(t_2 - t_1)$, over a period of time is proportional to the
        period of time raised to twice the Hurst
        coefficient~\cite[pp. 180]{Feder},~\cite[pp. 246]{Crownover}.
        This seems to be a quantitative statement concerning how fast,
        and to what degree, the rate of revenue returns' state of
        affairs can change over a period of time.  An additional
        implication, for Hurst coefficients sufficiently close to 0.5,
        is that the probability of the state of affairs repeating
        sometime in the future goes down with increasing
        time\footnote{It can be shown that the number of expected
        market ``high'' and ``low'' transitions, $N$, scales with the
        square root of time, or $N \propto \sqrt {t}$, meaning that
        the cumulative distribution of the probability, $P$, of the
        duration of a market's ``high'' or ``low'' exceeding a given
        time interval, $t$, is proportional to the reciprocal of the
        square root of the time interval, $P \propto 1 / \sqrt {t}$,
        (or, conversely, that the probability of the duration of a
        market's ``high'' or ``low'' exceeding a given time interval
        is proportional to the reciprocal of the time interval raised
        to the power $3 / 2$, ie., $P \propto 1 / t^{3 /
        2}$,~\cite[pp. 153]{Schroeder}. What this means is that a
        histogram of the ``zero free'' run-lengths of a market being
        ``high'' or ``low,'' over a long time, would have a $1 / t^{3
        / 2}$ characteristic.)}, $t$, $p(t) = erf (1/\sqrt{2t})$ which
        is approximately $1/\sqrt{t}$ for $t \gg
        1$~\cite[pp. 160]{Schroeder}. Figures~\ref{\SETLABEL:FN},
        and,~\ref{\SETLABEL:FF} compare methods of approximation of
        the ``forecastability'' of the rate of revenue returns in the
        {\market} for the near term and far term,
        respectively~\cite[pp. 83-84]{Peters:CAOITCM}\footnote{The
        author is not comfortable with Peters' interpretation. For
        example, if the algorithm explained
        in~\cite[pp. 82]{Peters:CAOITCM} is used on ``white noise''
        which, by definition, never has any correlations, the short
        term Hurst coefficient, and thus the ``forecastability,'' is
        still near unity---a bit of an enigma. This can be verified
        with the {\it tswhite}\/ and {\it tshurst}\/ programs, which
        are briefly described in Appendix~\ref{programs}.}.  This
        seems to be a quantitative statement concerning ``windows of
        opportunity'' in the rate of revenue returns, (per
        {\timescale}.)  The program {\it tslsq}\/ was used on the
        Hurst coefficient data, presented in
        Figure~\ref{\SETLABEL:HC}, to provide a least squares
        approximation to the Hurst coefficient. The superimposed least
        squares approximation with on original Hurst coefficient data
        is presented.  The time series data has a Hurst coefficient of
        {\thurstlow}, so that:

        \subidx{\market}{Hurst coefficient analysis}
        \begin{eqnarray}
            V\left(t_2 - t_1\right) & \propto & \left(t_2 - t_1\right)^{2 \cdot H}\\
            V\left(t_2 - t_1\right) & \propto & \left(t_2 - t_1\right)^{2 \cdot {\thurstlow}}\\
                                    & \propto & \left(t_2 - t_1\right)^{\thurstlowtwo}
            \label{\SETLABEL:V}
        \end{eqnarray}

        \subidx{fractional}{Brownian motion}
        \subidx{Brownian motion}{fractional}
        \idx{fractal}
        \noindent where $V(t_2 - t_1)$ is the variance of the
        increments of the rate of revenue returns, (per {\timescale},)
        over the time interval $t_2 -
        t_1$,~\cite[pp. 177]{Feder},~\cite[pp. 494]{Peitgen}. If $H >
        \frac{1}{2}$, then the time series is termed as being
        characterized by ``fractional Brownian
        motion~\cite[pp. 170]{Feder}.''

        \subidx{rate of revenue returns}{predictability}
        \subidx{rate of revenue returns}{forecastability}
        \subidx{rate of revenue returns}{consistency}
        \subidx{predictability}{rate of revenue returns}
        \subidx{forecastability}{rate of revenue returns}
        \subidx{consistency}{rate of revenue returns}
        \subidx{\market}{rate of revenue returns, predictability}
        \subidx{\market}{rate of revenue returns, forecastability}
        \subidx{\market}{rate of revenue returns, consistency}
        \subidx{Hurst coefficient}{analysis}
        \subidx{\market}{Hurst coefficient analysis}
        \subidx{\market}{rate of change}

        In some sense, the Hurst coefficient is a quantitative
        expression of the ``forecastability'' of the future based on
        the past\footnote{Actually, in general, when summing fractal
        entities, the method used should be a root mean square
        process, dependent on the Hurst Coefficient, $H$, where
        $P_{total}^H = P_1^H + P_2^H + \cdots$, where $P_n$ is the
        fractal entities. For a Brownian motion, or random walk type
        of fractal the Hurst Coefficient is a function of time into
        the future. For the ``near term,'' the Hurst coefficient is
        very near unity, meaning the summation process is linear. For
        the ``long term,'' $H \approx 0.5$, or a standard root mean
        square summation process should be used. If $H$ is $0.5$ then
        the market is termed a Brownian motion, or random walk
        process. If it is larger than 0.5, it is termed fractional
        Brownian motion process. For a random walk process, ``near
        term'' and ``far term'' are quantitatively differentiated on
        the Hurst Coefficient graph where $1 - \ln (t) = 0.5 \cdot \ln
        (t)$, or when $\ln (t) = 2$, or $t = 7.389\ldots$ See
        Section~\ref{\SETLABEL:FS} for the particulars on using Hurst
        Coefficient to sum fractal process' for the {\market}. See
        also~\cite[pp. 67, 83-84]{Peters:CAOITCM} and~\cite[pp. 129,
        159]{Schroeder} for particulars on the implications of the
        Hurst Coefficient and root mean square summation issues.}.  A
        Hurst coefficient of {\thurstlow}, (for the near future, and
        {\thurstall} for the distant future.) implies that the
        likelihood of the rate of revenue returns, (per {\timescale},)
        for any two consecutive {\timescale}s being the same is
        {\thurstlowhundred}\%~\cite[pp. 66]{Peters:CAOITCM} for the
        near future, and {\thurstall} for the distant
        future. Likewise, there is a {\thurstlowhundred}\% chance of
        the rate of revenue returns, (per {\timescale},) movements
        being the same in consecutive time periods---ie., if, in a
        given {\timescale}, the rate of revenue returns, (per
        {\timescale},) is increasing, there is a {\thurstlowhundred}\%
        that the rate of revenue returns, (per {\timescale},) will
        increase in the following period, also. In some sense, this is
        a quantitative statement on how ``predictable,'' or
        ``forecastable'' the rate of revenue returns, (per
        {\timescale},) for the {\market} are over time, since the
        probability of having $n$ many consecutive {\timescale}s of
        the same agenda is $H^n$ where $H$ is the Hurst coefficient,
        or, letting the short term probability of having $n$ many
        {\timescale}s of the same market agenda, $p_a$, is:

        \begin{eqnarray}
            p_a\left(n\right) & = & H^{n}\\
                              & = & {\thurstlow}^{n}
            \label{\SETLABEL:MA}
        \end{eqnarray}

        \subidx{rate of revenue returns}{predictability}
        \subidx{rate of revenue returns}{forecastability}
        \subidx{rate of revenue returns}{consistency}
        \subidx{predictability}{rate of revenue returns}
        \subidx{forecastability}{rate of revenue returns}
        \subidx{consistency}{rate of revenue returns}
        As an interesting interpretation of the normalized increments
        of the time series data presented in
        Figure~\ref{\SETLABEL:TF}, if the vertical axis is multiplied
        by 100, to convert to percent, then the graph represents the
        error, in percent, that would be made by forecasting, month by
        month, that the next {\timescale}'s rate of revenue returns
        would be the same as the current {\timescale}'s revenue
        rate. Interestingly, it is $\datafractionmean \cdot 100$
        percent, on the average, with a standard deviation of
        $\datafractionstddev \cdot 100$ percent, and a root mean
        square error value of $\datafractionrms \cdot 100$
        percent---small values for such a simple forecasting
        mechanism.

        \subidx{\market}{rate of revenue returns, range}
        \subidx{Hurst coefficient}{analysis}
        \subidx{\market}{Hurst coefficient analysis}
        \subidx{\market}{rate of change}

        This is, essentially, a statement of the range of values, in
        the increments of the rate of revenue returns, (per
        {\timescale},) that is to be expected over the time interval,
        $t_2 - t_1$,
        $R_v$,~\cite[pp. 178]{Feder},~\cite[pp. 172]{Cambel}:

        \begin{eqnarray}
            R_v\left(t_2 - t_1\right) & \propto & \left(t_2 - t_1\right)^{H}\\
                                      & \propto & \left(t_2 - t_1\right)^{\thurstlow}
            \label{\SETLABEL:R}
        \end{eqnarray}

        \subidx{\market}{rate of revenue returns, range}
        \subidx{Hurst coefficient}{analysis}
        \subidx{\market}{Hurst coefficient analysis}
        \subidx{\market}{rate of change}
        \subidx{Markov}{statistics}
        \subidx{statistics}{Markov}
        \noindent where $R$ is the range of values in the increments
        of the rate of revenue returns, (per {\timescale}.) A Hurst
        coefficient, $H$, that is much larger than $\frac{1}{2}$, (but
        less than 1,) implies a strongly non-Gaussian distribution in
        the increments of the rate of revenue returns, (per
        {\timescale},)~\cite[pp. 152, 194]{Feder}, and a Hurst
        coefficient near $\frac{1}{2}$ implies that the increments of
        the rate of revenue returns, (per {\timescale}) is
        characteristic of an independent
        process~\cite[pp. 195]{Feder}. Extreme caution should be
        exercised in using Markov statistics in any analysis where the
        Hurst coefficient is not
        $\frac{1}{2}$,~\cite[pp. 124]{Crownover},~\cite[pp. 106]{Peters:CAOITCM}.


        As a useful approximation, if $H$, is approximately
        $\frac{1}{2}$, Equation~\ref{\SETLABEL:R} reduces
        to,~\cite[pp. 129]{Schroeder}:

        \begin{eqnarray}
            R\left(t_2 - t_1\right) & \propto & (t_2 - t_1)^{\frac{1}{2}}\\
                                    & \propto & \sqrt{\left(t_2 - t_1\right)}
        \end{eqnarray}

        \subidx{\market}{rate of revenue returns, range}
        \subidx{\market}{rate of revenue returns, increase and decrease}
        \subidx{Hurst coefficient}{analysis}
        \subidx{\market}{Hurst coefficient analysis}
        \subidx{\market}{rate of change}
        \subidx{Markov}{statistics}
        \subidx{statistics}{Markov}

        In the case where the Hurst coefficient, $H$, is
        $\frac{1}{2}$, the range of values in the increments of the
        rate of revenue returns, (per {\timescale},) divided by the
        standard deviation of these values, $S$, can be anticipated to
        increase over time according to the following
        relation,~\cite[pp. 154]{Feder},~\cite[pp. 129]{Schroeder}:

        \begin{equation}
            \frac{R\left(t_2 - t_1\right)}{S} \propto \left(t_2 - t_1\right)^{\frac{1}{2}}
        \end{equation}

        \subidx{\market}{rate of revenue returns, range}
        \subidx{\market}{rate of revenue returns, increase and decrease}
        \subidx{Hurst coefficient}{analysis}
        \subidx{\market}{Hurst coefficient analysis}
        \subidx{\market}{rate of change}
        \noindent which is a useful conceptual approximation, since it
        involves only the square root function---if the range and the
        standard deviation of the increments of the rate of revenue
        returns, (per {\timescale},) are known, (and $H \approx
        \frac{1}{2}$,) then the expected change in $\frac{R}{S}$, will
        increase with the square root of time\footnote{To be precise,
        it is actually asymptotically proportional to
        $\tau^{\frac{1}{2}}$}.

        Another useful approximation when rescaling processes that are
        characterize by Brownian motion, (ie., when $H \approx
        \frac{1}{2}$,) is that:

        \begin{eqnarray}
            X\left(t\right) & \propto & \frac{X\left(rt\right)}{r^{H}}\\
                            & \propto & \frac{X\left(rt\right)}{r^{\thurstlow}}
        \end{eqnarray}

        \idx{Brownian motion}
        \idx{fractal}
        Where $X(t)$ is the process characterized by Brownian motion,
        and $r$ is a scaling factor,~\cite[pp. 494]{Peitgen}.

        \subidx{programs}{tslsq}
        \subidx{tslsq}{program}
        The program {\it tslsq}\/ was used on the H parameter data,
        presented in Figure~\ref{\SETLABEL:HP}, to provide a least
        squares approximation to the H parameter for the
        {\market}. The superimposed least squares approximation on the
        original H parameter data is presented.  By contrast, the H
        parameter, as derived by the methodology outlined
        in~\cite[pp. 249]{Crownover}, is {\thcalclow} for the near
        future, and {\thcalcall} for the distant future.

        \subidx{\market}{Hurst coefficient analysis}
        \subidx{Hurst coefficient}{analysis}
        \subidx{increments}{normalized}
        \subidx{normalized}{increments}
        \subidx{programs}{tshurst}
        \subidx{tshurst}{program}
        \subidx{\market}{H parameter analysis}
        \subidx{H parameter}{analysis}
        \subidx{programs}{tshcalc}
        \subidx{tshcalc}{program}
        Figures~\ref{\SETLABEL:HC} and~\ref{\SETLABEL:HP} represent
        Hurst coefficient and H parameter data that are derived from
        the normalized increments, shown in
        Figure~\ref{\SETLABEL:TF}. In this case, the data is
        considered a normalized derivative of the time series data
        presented in Figure~\ref{\SETLABEL:TF}, instead of a
        cumulative sum.  The program, {\it tshurst}\/, is described
        briefly in appendix~\ref{programs}, and the data for
        figures~\ref{\SETLABEL:THC} and~\ref{\SETLABEL:THP} was made
        using the -d option.

        \begin{figure}[ht]
            \begin{center}
                \begin{minipage}[t]{0.45\textwidth}
                    \epsfxsize=1.0\linewidth
                    \epsffile{\directory/data.tsfraction.tshurst-d.eps}
                    \caption[{\market}, traditional Hurst coefficient
                        data]{{\market}, traditional Hurst coefficient
                        data for the time series data shown in
                        Figure~\ref{\SETLABEL:TS}.  The slope of the
                        graph is the Hurst coefficient, and is
                        {\hurstlow} for the near term, and
                        {\hurstall} for the far term.}
                    \label{\SETLABEL:THC}
                \end{minipage}
                \hfill
                \begin{minipage}[t]{0.45\textwidth}
                    \epsfxsize=1.0\linewidth
                    \epsffile{\directory/data.tsfraction.tshcalc-d.eps}
                    \caption[{\market}, traditional H parameter
                        data]{{\market}, traditional H parameter data
                        for the time series data shown in
                        Figure~\ref{\SETLABEL:TS} The slope of the
                        graph is the H parameter, and is {\hcalclow}
                        for the near term, and {\hcalcall} for the
                        far term.}
                    \label{\SETLABEL:THP}
                \end{minipage}
            \end{center}
        \end{figure}

% Local Variables:
% TeX-parse-self: t
% TeX-auto-save: t
% TeX-master: "fractal.tex"
% End:


        \subsubsection{Observations on the Hurst Coefficient Analysis}

            Note that the H parameter data is not linear, and the long
            term predictability is better than the short term
            predictability, indicating that the least squares
            approximation is low.

        %
% -----------------------------------------------------------------------------
%
% A license is hereby granted to reproduce this software source code and
% to create executable versions from this source code for personal,
% non-commercial use.  The copyright notice included with the software
% must be maintained in all copies produced.
%
% THIS PROGRAM IS PROVIDED "AS IS". THE AUTHOR PROVIDES NO WARRANTIES
% WHATSOEVER, EXPRESSED OR IMPLIED, INCLUDING WARRANTIES OF
% MERCHANTABILITY, TITLE, OR FITNESS FOR ANY PARTICULAR PURPOSE.  THE
% AUTHOR DOES NOT WARRANT THAT USE OF THIS PROGRAM DOES NOT INFRINGE THE
% INTELLECTUAL PROPERTY RIGHTS OF ANY THIRD PARTY IN ANY COUNTRY.
%
% Copyright (c) 1994-2006, John Conover, All Rights Reserved.
%
% Comments and/or bug reports should be addressed to:
%
%     john@email.johncon.com (John Conover)
%
% -----------------------------------------------------------------------------
%
% Revision: \RCSRevision \\
% Revision Time: \RCSTime UMT \\
% Revision Date: \RCSDate \\
% Revision Id: \RCSId \\
% Revision File: \RCSLog \\
\RCS $Revision: 0.0 $
\RCS $Date: 2006/01/20 04:38:13 $
\RCS $Id: fiscal.tex,v 0.0 2006/01/20 04:38:13 john Exp $
% $Log: fiscal.tex,v $
% Revision 0.0  2006/01/20 04:38:13  john
% Initial version
%
%
    \subsection{Fixed Increment Approximation for Fiscal Strategy}
        \label{\SETLABEL:FS}

        \subidx{\market}{fiscal strategy}
        \subidx{markets}{analysis}
        \subidx{analysis}{markets}
        \subidx{strategy}{fiscal}
        \subidx{fiscal}{strategy}
        The data in this section is presented in tabular form in
        Section~\ref{\SETLABELREF:LR}. This section derives various
        values based on the ``average'' of the normalized increments
        presented in Figure~\ref{\SETLABEL:TFA}. These values are an
        approximation to a, probably, complex process with a
        distribution shown in Figure~\ref{\SETLABEL:TF}. These values
        will be used in a fixed increment Brownian fractal analysis
        and simulation of the {\market}, and may, or may not, provide
        adequate accuracy for projections.

        For an organization operating in the {\market}, the fiscal
        strategy, commensurate with the aggregate environment, can be
        derived as follows~\cite[pp. 128, pp
        151]{Schroeder},~\cite[pp. 450]{Reza},~\cite[pp. 270]{Pierce}:
        \vspace{0.15in}

        \subsubsection{Logarithmic Returns}
            \label{\SETLABEL:LR}

            \subidx{logarithmic}{returns}
            \subidx{returns}{logarithmic}
            \subidx{\market}{logarithmic returns}
            The logarithmic returns can be calculated by various
            means. Four will be presented here, for comparison.

            \subidx{programs}{tsnormal}
            \subidx{tsnormal}{program}
            \subidx{logarithmic}{returns}
            \subidx{returns}{logarithmic}
            The logarithmic returns, in bits, $bits$, as computed from
            the mean, by the program {\it tsnormal}\/, which is
            described in Chapter~\ref{programs}, and is presented in
            Figure~\ref{\SETLABEL:TF}, and Equation~\ref{abits} from
            Section~\ref{ereturns} in Chapter~\ref{general}:

            \begin{equation}
                bits = \frac{\ln \left({\datafractionmean} + 1\right)}{\ln \left(2\right)} = \datafractionmeanbits
            \end{equation}

            \subidx{programs}{tslsq}
            \subidx{tslsq}{program}
            \subidx{logarithmic}{returns}
            \subidx{returns}{logarithmic}
            \noindent By comparison, the logarithmic returns, in bits,
            $bits$, as computed from the constant in the least squares
            approximation, using the program {\it tslsq}\/, which is briefly
            described in Chapter~\ref{programs}, as presented in
            Figure~\ref{\SETLABEL:TF}, and Equation~\ref{abits} from
            Section~\ref{ereturns} in Chapter~\ref{general}:

            \begin{equation}
                bits = \frac{\ln \left({\datafractionconstant} + 1\right)}{\ln \left(2\right)} = \datafractionconstantbits
            \end{equation}

            Note that if the mean is not constant in
            Figure~\ref{\SETLABEL:TF}, this method will not provide
            accurate results.

            \subidx{programs}{tslsq}
            \subidx{tslsq}{program}
            \subidx{logarithmic}{returns}
            \subidx{returns}{logarithmic}
            \noindent And by yet another comparison, using the program
            {\it tslsq}\/, which is briefly described in
            Chapter~\ref{programs}, with the -e -p options, to provide
            a formula for the least squares exponential fit to the
            time series data set presented in
            Figure~\ref{\SETLABEL:TS}:

            \begin{equation}
                bits = {\datatslsqepbits}
            \end{equation}

            \subidx{programs}{tslogreturns}
            \subidx{tslogreturns}{program}
            \subidx{logarithmic}{returns}
            \subidx{returns}{logarithmic}
            \noindent And finally, by comparison, from the
            {\it tslogreturns}\/ program, which is briefly described
            in Chapter~\ref{programs}, with the -p option, to provide
            a formula for the logarithmic returns of the time series
            data set presented in Figure~\ref{\SETLABEL:TS}:

            \begin{equation}
                bits = {\logreturns}
            \end{equation}

        \subsubsection{Calculation of Shannon Probability}
            \label{\SETLABEL:SP}

            \subidx{\market}{Shannon probability}
            Ideally, all of the values presented in
            Section~\ref{\SETLABEL:LR} would be equal. Using the
            logarithmic returns provided by the {\it tslogreturns}\/
            program, to be consistent
            with~\cite[pp. 81]{Peters:CAOITCM}

            \subidx{programs}{tslogreturns}
            \subidx{tslogreturns}{program}
            \begin{equation}
                2^{{\logreturns}t}
            \end{equation}

            \noindent therefore:
            \begin{equation}
                C\left(p\right) = {\logreturns}
            \end{equation}
            \subidx{programs}{tsshannon}
            \subidx{tsshannon}{program}
            \subidx{Shannon}{probability}
            \subidx{probability}{Shannon}
            \noindent and, {\it tsshannon}\/ {\logreturns} gives:
            \begin{equation}
                \label{\SETLABEL:F0}
                C\left({\shannonlogreturns}\right) = {\logreturns}
            \end{equation}
            \noindent therefore:
            \begin{eqnarray}
                2^{C\left({\shannonlogreturns}\right)} & = & 2^{\logreturns}\\
                                                       & = & {\twologreturns}\\
                                                       & = & {\twologreturnshundred}\%
            \end{eqnarray}
            \noindent and:
            \begin{eqnarray}
                2p - 1 & = & \left(2 \cdot {\shannonlogreturns}\right) - 1\\
                       & = & {\twopone}\\
                       \label{\SETLABEL:F1}
                       & = & {\twoponehundred}\%
            \end{eqnarray}

            \subidx{\market}{fiscal strategy}
            \subidx{markets}{analysis}
            \subidx{analysis}{markets}
            \subidx{strategy}{fiscal}
            \subidx{fiscal}{strategy}
            \subidx{\market}{fiscal strategy}
            \subidx{\market}{growth rate}
            Presuming the simplified assumptions outlined in
            Section~\ref{assumptions}, the ``typical'' organization
            operating in the {\market} executes a long term fiscal
            strategy, commensurate with the aggregate environment,
            that is to invest, every {\timescale}, in sufficient
            additional resources and infrastructure, to increase the
            manufacturing of goods and services by {\twoponehundred}\%
            of its rate of revenue returns, (per {\timescale}.) As a
            conceptual model, the remaining {\hundredtwoponehundred}\%
            will be held in ``reserve'' with a
            {\shannonlogreturnshundred}\% chance of making twice the
            {\twoponehundred}\% back, (and a
            {\hundredshannonlogreturnshundred}\% chance of making
            0.0,) in one {\timescale}, on the average, for an average
            growth in its rate of revenue returns, (per {\timescale},)
            of {\twologreturnshundred}\%, or a doubling of its rate of
            revenue returns, (per {\timescale},) in
            {\oneoverlogreturns} {\timescale}s.

        \subsubsection{Example Fixed Increment Approximation Fiscal Strategies}

            \subidx{\market}{fiscal strategy}
            \subidx{markets}{analysis}
            \subidx{analysis}{markets}
            \subidx{strategy}{fiscal}
            \subidx{fiscal}{strategy}
            \subidx{\market}{fiscal strategy}
            \subidx{\market}{growth rate}
            \subidx{\market}{management metric}
            \idx{management metric}
            A possible metric on the effectiveness of long term fiscal
            management could possibly be that if an investment of
            {\twoponehundred}\% per {\timescale} of the rate of
            revenue returns, (per {\timescale},) is made in resources
            and infrastructure, then the rate of revenue returns would
            be expected to increase by {\twologreturnshundred}\%, per
            {\timescale}, on average.

            Note that the metrics presented in this section are
            representative of the {\market} as an aggregate whole, and
            may or may not be accurate representations for any
            particular participant in the environment. Of interest to
            the participants in the environment would be a similar
            analysis of each product or service rendered in the
            marketplace.

            \subidx{\market}{fiscal strategy}
            \subidx{markets}{analysis}
            \subidx{analysis}{markets}
            \subidx{strategy}{fiscal}
            \subidx{fiscal}{strategy}
            \subidx{\market}{fiscal strategy}
            As a simple illustrative example, a company operating in
            this environment might obtain a credit line from a bank
            that is equal to {\twoponehundred}\% of its rate of
            revenue returns, (per {\timescale},) to finance additional
            operations. In this simple scenario, the company would use
            its revenue base as collateral for the loan. Some
            {\timescale}s, depending on the {\market}'s environment,
            the company's rate of revenue returns exceeds what was
            borrowed from the bank, and the loan is repaid in
            full. Other {\timescale}s, the company must default, and
            the bank seizes a portion of the company's revenue base to
            pay the delinquent loan. However, on the average, the
            company will expand its rate of revenue returns at
            {\twologreturnshundred}\% per {\timescale}.

            \subidx{\market}{fiscal strategy}
            \subidx{markets}{analysis}
            \subidx{analysis}{markets}
            \subidx{strategy}{fiscal}
            \subidx{fiscal}{strategy}
            \subidx{\market}{fiscal strategy}
            As another simple example, a company re-invests
            {\twoponehundred}\% of its rate of revenue returns, (per
            {\timescale},) in development, marketing, sales, and
            distribution of new products.  Although some products will
            be successful and the return on the investment will exceed
            the {\twoponehundred}\% per {\timescale} investment,
            others will not. However, on the average, the company will
            expand it gross rate of revenue returns at
            {\twologreturnshundred}\% per {\timescale}.

            \subidx{\market}{fiscal strategy}
            \subidx{markets}{analysis}
            \subidx{analysis}{markets}
            \subidx{strategy}{fiscal}
            \subidx{fiscal}{strategy}
            \subidx{\market}{fiscal strategy}
            \subidx{\market}{product portfolio}
            \subidx{\market}{product diversity}
            \subidx{\market}{product mix}
            \subidx{\market}{optimum number of products}
            \idx{product portfolio}
            \idx{product diversity}
            \idx{optimum number of products}
            \idx{product mix}

            As an example of ``product portfolio'' management, suppose
            a company re-invests {\twoponehundred}\% of its rate of
            revenue returns, (per {\timescale},) in development,
            marketing, sales, and distribution of new products.
            Further suppose that the company has two products, and a
            fractal analysis of the individual product rate of revenue
            return time series indicates that one product has a
            Shannon probability of 0.65, and the other has a Shannon
            probability of 0.55. Then the percentage of re-investment
            in the first product would be $(2 \cdot 0.65 - 1) \cdot
            {\twoponehundred}$, percent of the rate of revenue
            returns, and $(2 \cdot 0.55 - 1) \cdot {\twoponehundred}$
            percent for the second product, implying that the company
            should diversify its product line\footnote{The astute
            reader would note that the linear addition was used to add
            the contribution to development of each product. This is a
            ``near term'' interpretation. Actually, in general, the
            method used should be a root mean square process,
            dependent on the Hurst Coefficient, $H$, where
            $P_{total}^H = P_1^H + P_2^H + \cdots$, where $P_n$ is the
            contribution to each individual product. For a Brownian
            motion, or random walk type of fractal the Hurst
            Coefficient is a function of time into the future. For the
            ``near term,'' the Hurst coefficient is very near unity,
            meaning the summation process is linear. For the ``long
            term,'' $H \approx 0.5$, or a standard root mean square
            summation process should be used. If $H$ is $0.5$ then the
            market is termed a Brownian motion, or random walk
            process. If it is larger than 0.5, it is termed fractional
            Brownian motion process. For a random walk process, ``near
            term'' and ``far term'' are quantitatively differentiated
            on the Hurst Coefficient graph where $1 - \ln (t) = 0.5
            \cdot \ln (t)$, or when $\ln (t) = 2$, or $t =
            7.389\ldots$ See~\cite[pp. 67, 83-84]{Peters:CAOITCM}
            and~\cite[pp. 129, 159]{Schroeder} for particulars on the
            implications of the Hurst Coefficient and root mean square
            summation issues.}.  Note that this is a ``bet hedging''
            metric methodology, and assumes that the products have
            uncorrelated revenue return rates. If this re-investment
            methodology is not feasible, perhaps for strategic
            financial reasons, then the re-investment in both products
            should total the ${\twoponehundred}$\%, and the investment
            in each product should be made at a ratio of $\frac{(2
            \cdot 0.65 - 1)}{(2 \cdot 0.55 - 1)} = 3 : 1$,
            respectively. Note that this ``bet hedging'' can be used
            to define the optimal number of products that can be
            supported on the rate of revenue returns. If it assumed
            that all products are ``typical'' for the {\market}, as a
            standard bench mark, then the optimal number will be
            $\frac{1}{{\twopone}}$. Note that this is a
            ``theoretical'' value, since not all products are
            ``typical,'' and there may be strategic reasons, for
            example product leveraging, that may increase the number
            of products above the optimum. However, most of the
            revenue should come from the optimal number of products,
            since having more products will decrease the amount of the
            potential investment in each product, and having less than
            the optimum number of products will increase the risk that
            many of the products could suffer a ``down market''
            concurrently, impacting the rate of revenue returns.  As
            another interesting interpretation of the optimal
            ``hedging of bets,'' in product portfolio strategy, and
            considering the graph of the normalized increments
            presented in Figure~\ref{\SETLABEL:TF}, if the
            organization is running optimally, then these products
            will generate, at least in principle, one standard
            deviation, approximately $0.8413 = 84.13$\% of the future
            growth in rate of revenue returns. Naturally, these are
            approximations, and the values are an approximation to a,
            probably, complex process, and appropriate scrutiny should
            be exercised before making specific projections.  As yet
            another example of ``product portfolio'' management,
            consider the issue of product mix. In this interpretation,
            {\twoponehundred}\% of the product manufactured should be
            ``proprietary,'' while the rest is ``industry standard.''
            As yet another possibility, {\twoponehundred}\% of the
            product manufactured should be predatory into new markets,
            and the remainder in markets that are ``traditional'' for
            the company.

% Local Variables:
% TeX-parse-self: t
% TeX-auto-save: t
% TeX-master: "fractal.tex"
% End:


        \subsubsection{Observations on the Fixed Increment Approximation for Fiscal Strategy}

            A re-investment of {\twoponehundred} of the rate of
            revenue returns per {\timescale} does not seem
            inconsistent with the industry averages, since it includes
            investments in research and development, additional
            manufacturing infrastructure, advertising,
            etc. Additionally, a product mix of {\twoponehundred}\%
            ``proprietary'' and the remainder ``industry standard''
            products seems consistent with the industry analyst
            ``20/80'' rule. The value of one standard deviation,
            $84.13$\%, of the revenue return rate being generated by
            $\frac{1}{{\twopone}}$ products seems consistent with the
            industry, also.

        %
% -----------------------------------------------------------------------------
%
% A license is hereby granted to reproduce this software source code and
% to create executable versions from this source code for personal,
% non-commercial use.  The copyright notice included with the software
% must be maintained in all copies produced.
%
% THIS PROGRAM IS PROVIDED "AS IS". THE AUTHOR PROVIDES NO WARRANTIES
% WHATSOEVER, EXPRESSED OR IMPLIED, INCLUDING WARRANTIES OF
% MERCHANTABILITY, TITLE, OR FITNESS FOR ANY PARTICULAR PURPOSE.  THE
% AUTHOR DOES NOT WARRANT THAT USE OF THIS PROGRAM DOES NOT INFRINGE THE
% INTELLECTUAL PROPERTY RIGHTS OF ANY THIRD PARTY IN ANY COUNTRY.
%
% Copyright (c) 1994-2006, John Conover, All Rights Reserved.
%
% Comments and/or bug reports should be addressed to:
%
%     john@email.johncon.com (John Conover)
%
% -----------------------------------------------------------------------------
%
% Revision: \RCSRevision \\
% Revision Time: \RCSTime UMT \\
% Revision Date: \RCSDate \\
% Revision Id: \RCSId \\
% Revision File: \RCSLog \\
\RCS $Revision: 0.0 $
\RCS $Date: 2006/01/20 04:38:13 $
\RCS $Id: companies.tex,v 0.0 2006/01/20 04:38:13 john Exp $
% $Log: companies.tex,v $
% Revision 0.0  2006/01/20 04:38:13  john
% Initial version
%
%
    \subsection{Number of Companies}
        \label{\SETLABEL:QNC}

        \subidx{\market}{number of companies}
        \subidx{number of companies}{analysis}
        \subidx{analysis}{number of companies}
        \subidx{Shannon}{probability}
        \subidx{probability}{Shannon}
        This section evaluates the approximate, or ``average,'' number
        of companies in the {\market}, and uses the method outlined in
        Chapter~\ref{general}, Section~\ref{aftsma}. Since the
        average, $avg_{ind}$, and the root mean square, $rms_{ind}$,
        of the normalized increments of the {\market} time series is
        \datafractionmean, and \datafractionrms respectively, the
        number of companies participating in the market can be
        calculated by Equation~\ref{ncompanies} to be {\ncompanies}.

        If this value seems consistent number of companies in the
        {\market}, within the assumptions outlined in
        Chapter~\ref{general}, Section~\ref{aftsma}, then it would
        seem that there is some circumstantial or indirect evidence
        that the companies participating in the {\market} are
        operating optimally, and the ``average'' Shannon probability,
        $P$ for each participating company would be, using
        Equation~\ref{pncompanies}, {\pncompanies}, which would be the
        value which should be used in Section~\ref{\SETLABEL:FS} for
        each participating company if market expansion was to be
        consistent with the rest of the industry. However, if the
        Shannon probability derived in Section~\ref{\SETLABEL:FS} is
        greater than the average Shannon probability for the companies
        participating in the {\market}, as derived in this section,
        then the market would, possibly, be exploitable with the
        fiscal strategy outlined in Section~\ref{\SETLABEL:FS}. The
        maximum exploitability for the {\market} is derived in
        Section~\ref{\SETLABEL:MAXSHANNON}, but it is probably of
        doubtful practicality.

        Note that these optimizations would maximize a company's
        market growth. Since there are probably many companies
        competing in the market place, this would not necessarily
        maximize a company's P\&L, as described in
        Chapter~\ref{general}, Section~\ref{ompl}. The Shannon
        probability that maximizes market share in the {\market} is
        \pncompanies, with several alternative solutions listed in the
        previous paragraph. However, these should be contrasted to the
        Shannon probability that maximizes a company's P\&L which is
        \avgrms~in the {\market}. In all cases, the fraction of the
        P\&L that should be ``wagered'' on the future, $f$, should be:

        \begin{equation}
            f = 2P - 1
        \end{equation}

        \noindent where $P$ is the particular Shannon probability
        chosen optimize a particular fiscal strategy. Interestingly,
        the measured Shannon probability of the {\market} would tend
        to indicate that the companies participating in the market
        have chosen a fiscal strategy that optimizes market growth, as
        opposed to capital growth.

        \subidx{\market}{increasing returns}
        \subidx{economic increasing returns}{\market}
        As interesting interpretation of these exploitive issues,
        since all three fiscal strategies will result in exponential
        market growth for every company participating in the market,
        is that they may represent, perhaps, an example of
        ``increasing returns.''

% Local Variables:
% TeX-parse-self: t
% TeX-auto-save: t
% TeX-master: "fractal.tex"
% End:


        %
% -----------------------------------------------------------------------------
%
% A license is hereby granted to reproduce this software source code and
% to create executable versions from this source code for personal,
% non-commercial use.  The copyright notice included with the software
% must be maintained in all copies produced.
%
% THIS PROGRAM IS PROVIDED "AS IS". THE AUTHOR PROVIDES NO WARRANTIES
% WHATSOEVER, EXPRESSED OR IMPLIED, INCLUDING WARRANTIES OF
% MERCHANTABILITY, TITLE, OR FITNESS FOR ANY PARTICULAR PURPOSE.  THE
% AUTHOR DOES NOT WARRANT THAT USE OF THIS PROGRAM DOES NOT INFRINGE THE
% INTELLECTUAL PROPERTY RIGHTS OF ANY THIRD PARTY IN ANY COUNTRY.
%
% Copyright (c) 1994-2006, John Conover, All Rights Reserved.
%
% Comments and/or bug reports should be addressed to:
%
%     john@email.johncon.com (John Conover)
%
% -----------------------------------------------------------------------------
%
% Revision: \RCSRevision \\
% Revision Time: \RCSTime UMT \\
% Revision Date: \RCSDate \\
% Revision Id: \RCSId \\
% Revision File: \RCSLog \\
\RCS $Revision: 0.0 $
\RCS $Date: 2006/01/20 04:38:13 $
\RCS $Id: operations.tex,v 0.0 2006/01/20 04:38:13 john Exp $
% $Log: operations.tex,v $
% Revision 0.0  2006/01/20 04:38:13  john
% Initial version
%
%
    \subsection{Fixed Increment Approximation for Operational Strategy}
        \label{\SETLABEL:OPS}.

        This section derives various values based on the ``average''
        of the normalized increments presented in
        Figure~\ref{\SETLABEL:TFA}. These values are an approximation
        to a, probably, complex process with a distribution shown in
        Figure~\ref{\SETLABEL:TF}. These values will be used in a
        fixed increment Brownian fractal analysis and simulation of
        the {\market}, and may, or may not, provide adequate accuracy
        for projections.

        \subidx{\market}{fiscal strategy}
        \subidx{\market}{Shannon probability}
        \subidx{strategy}{fiscal}
        \subidx{fiscal}{strategy}
        \subidx{Shannon}{probability}
        \subidx{probability}{Shannon}
        It should be noted that the analysis of fiscal strategy,
        presented in Section~\ref{\SETLABEL:FS}, is derived from the
        {\market} metrics and may, or may not, be maximally
        optimal. For the optimal fiscal strategy, which may be
        exploitable, see Section~\ref{\SETLABEL:MAXSHANNON}.

        \subidx{strategy}{exploitable}
        \subidx{exploitable}{strategy}
        \subidx{\market}{windows of opportunity}
        \idx{windows of opportunity}
        \subidx{decision}{obsolete}
        \subidx{obsolete}{decision}
        \subidx{decision}{timeliness}
        \subidx{timeliness}{decision}
        \subidx{rate of revenue returns}{forecast}
        \subidx{forecast}{rate of revenue returns}
        An additional exploitable strategy may be time itself.
        Equations~\ref{\SETLABEL:V},~\ref{\SETLABEL:R},
        and,~\ref{\SETLABEL:MA}, are, essentially, metrics on how fast
        a decision, which is based on information concerning the
        current status of the {\market}, becomes obsolete. Obviously,
        how long a decision is expected to remain relevant should be
        addressed as an operational necessity in strategic planning
        and project management. Figures~\ref{\SETLABEL:FN},
        and,~\ref{\SETLABEL:FF} compare methods of approximation of
        the ``forecastability'' of rate of revenue returns in the
        {\market} for the near term and far
        term~\cite[pp. 83-84]{Peters:CAOITCM}, respectively. As a
        general rule, caution must be exercised when making decisions
        that will span a time interval larger than the time interval
        where the ``forecastability'' of rate of revenue returns drops
        below 50\%. Beyond this time interval, the chances increase
        that the competitive and market forces will alter the market
        environment in a possibly detrimental unanticipated
        fashion. Obviously, there is significant advantage in
        ``timeliness'' of development, manufacturing, and distribution
        of products and services that are consistent with this
        temporal agenda. Automation of these processes, if executed
        consistently with this agenda, should be considered a
        competitive advantage.

        \subidx{strategy}{exploitable}
        \subidx{exploitable}{strategy}
        \subidx{rate of revenue returns}{forecast}
        \subidx{forecast}{rate of revenue returns}
        \idx{product life cycle}
        \idx{life cycle, product}
        In some sense, this temporal agenda defines the ``average''
        product or service life cycle in the {\market}. When the
        ``forecastability'' of rate of revenue returns drops below
        50\%, there is an even chance that the rate of revenue returns
        for the product or service will change in a detrimental
        fashion. If it is assumed that a product or service life cycle
        consists of a ramp up, a maintenence interval, and a ramp
        down, then, if all three life cycle intervals are equal, the
        product life cycle will be, approximately, three times the
        time interval where the ``forecastability'' of rate of revenue
        returns drops below 50\%. Although probably not an accurate
        prediction of product or service life cycle, the technique may
        be used as a conceptual approximation to the dynamics of
        ``market windows.\footnote{For example, consider the market
        for table salt. Since it has inelastic supply and demand
        curves, and is a necessary requirement for life, it would be
        expected that the Hurst coefficient would be very near
        unity---ignoring competitive pressures in the market. The
        predictability of the table salt market would, therefore, be
        expected to be relatively good, over time.}''  The conceptual
        approximation will probably predict a ``conservative'' or
        ``pessimistic'' value in relation to actual markets.

        \begin{figure}[ht]
            \begin{center}
                \begin{minipage}[t]{0.45\textwidth}
                    \epsfxsize=1.0\linewidth
                    \epsffile{\directory/datahurstlownear.eps}
                    \caption[{\market}, ``forecastability'' of near
                        term rate of revenue returns]{{\market},
                        ``forecastability'' of near term rate of
                        revenue returns. Although the error function
                        is the most accurate, for the near term,
                        $H^{t} = \thurstlow^{t}$ may be used as a
                        reliable metric of ``forecastability'' of the
                        rate of revenue returns.}
                    \label{\SETLABEL:FN}
                \end{minipage}
                \hfill
                \begin{minipage}[t]{0.45\textwidth}
                    \epsfxsize=1.0\linewidth
                    \epsffile{\directory/datahurstlowfar.eps}
                    \caption[{\market}, ``forecastability'' of far
                        term rate of revenue returns]{{\market},
                        ``forecastability'' of far term rate of
                        revenue returns. Although the error function
                        is the most accurate, for the far term,
                        $\frac{1}{\sqrt{t}}$ may be used as a reliable
                        metric of ``forecastability'' of the rate of
                        revenue returns.}
                    \label{\SETLABEL:FF}
                \end{minipage}
            \end{center}
        \end{figure}

        \idx{operations research}
        As an interesting interpretation of the data presented in
        Figure~\ref{\SETLABEL:FN}, there may be, perhaps, some
        applicability to such operational agendas as inventory
        control. Maintaining too little inventory, obviously, will
        create a situation where the organization can not exploit
        market expansion, and maintaining too much inventory,
        likewise, would over extend the company, creating unnecessary
        losses when the market contracts. The company should maintain
        inventory levels that do not exceed, from
        Equation~\ref{\SETLABEL:MA}, ${\thurstlow}^{n} = 0.5$
        {\timescale}s of operations. Since the optimal amount of
        inventory and, from Equation~\ref{\SETLABEL:V}, the variance
        of change in the rate of revenue returns in the future can be
        calculated, there may, perhaps, be some applicability to a
        forecasting methodology that can be incorporated into other
        areas of operations research, for example the linear algebras
        using simplex methodologies for optimization of manufacturing
        processes. Traditionally, these forecasts are made by the
        sales department, and are subject to various subjective
        biases.

% Local Variables:
% TeX-parse-self: t
% TeX-auto-save: t
% TeX-master: "fractal.tex"
% End:


        \subsubsection{Observations on the Fixed Increment Approximation for Operational Strategy}

            As an interesting interpretation of
            Figure~\ref{\SETLABEL:FF}, and evaluating the
            approximation $\frac{1}{\sqrt{t}}$ at 60 months gives a
            probability that the market will still have the same
            agenda of about $0.12909945$, or about 1 in 8. This is
            commensurate with numbers from the venture
            community\footnote{For example, see ``IEEE Engineering
            Management Review,'' Volume 23 Number 3, Fall 1995,
            pp. 83}. Of course new venture backed companies fail for
            many reasons, but market appropriateness to product
            portfolio 60 months in the future may be a major
            contributor. Additionally, the success rate of development
            projects of 8 month duration, which have a market success
            rate of about 1 in 3, seems consistent with
            $\frac{1}{\sqrt{3}} = 0.353553391$. Naturally, projects
            fail in the market for many reasons, but market
            appropriateness, in a dynamic market environment may be a
            major contributor to failure.

            As mentioned in Section~\ref{\SETLABEL:H},
            Equation~\ref{\SETLABEL:MA}, and the preceeding section,
            approximately 3 times the value where ${\thurstlow}^{n} =
            0.5$ could be interpreted as an approximation to the
            ``average'' product life cycle. This seems consistent with
            the 6 to 12 month life cycles quoted by many industry
            analyst. In addition, maintaining inventory levels that do
            not exceed the anticipated requirements of
            $\frac{\ln{0.5}}{\ln{\thurstlow}}$ many {\timescale}s
            seems consistent with the author's experience in the
            industry.

        %
% -----------------------------------------------------------------------------
%
% A license is hereby granted to reproduce this software source code and
% to create executable versions from this source code for personal,
% non-commercial use.  The copyright notice included with the software
% must be maintained in all copies produced.
%
% THIS PROGRAM IS PROVIDED "AS IS". THE AUTHOR PROVIDES NO WARRANTIES
% WHATSOEVER, EXPRESSED OR IMPLIED, INCLUDING WARRANTIES OF
% MERCHANTABILITY, TITLE, OR FITNESS FOR ANY PARTICULAR PURPOSE.  THE
% AUTHOR DOES NOT WARRANT THAT USE OF THIS PROGRAM DOES NOT INFRINGE THE
% INTELLECTUAL PROPERTY RIGHTS OF ANY THIRD PARTY IN ANY COUNTRY.
%
% Copyright (c) 1994-2006, John Conover, All Rights Reserved.
%
% Comments and/or bug reports should be addressed to:
%
%     john@email.johncon.com (John Conover)
%
% -----------------------------------------------------------------------------
%
% Revision: \RCSRevision \\
% Revision Time: \RCSTime UMT \\
% Revision Date: \RCSDate \\
% Revision Id: \RCSId \\
% Revision File: \RCSLog \\
\RCS $Revision: 0.0 $
\RCS $Date: 2006/01/20 04:38:13 $
\RCS $Id: simulation.tex,v 0.0 2006/01/20 04:38:13 john Exp $
% $Log: simulation.tex,v $
% Revision 0.0  2006/01/20 04:38:13  john
% Initial version
%
%
    \subsection{Simulation of Fixed Increment Approximation for Fiscal Strategy}
        \label{\SETLABEL:TSUNFAIRBROWNIAN}

        \subidx{\market}{market simulation}
        The data in this section is presented in tabular form in
        Section~\ref{\SETLABELREF:SIM}.
        Figure~\ref{\SETLABEL:TSUNFAIRBROWNIAN0} represents a
        constructional simulation of the time series data presented in
        Figure~\ref{\SETLABEL:TS}. The program {\it
        tsunfairbrownian}\/, which is briefly described in
        appendix~\ref{programs}, was used in the reconstruction. The
        reconstructed data is superimposed on the original time series
        data.  The program, {\it tsunfairbrownian}\/, essentially,
        constructs the new time series as a Brownian fractal with
        fixed increments---the value of the fixed increment is derived
        from the root mean square average of the normalized increments
        presented in Figure~\ref{\SETLABEL:TF}. The ``quality'' of
        such a reconstruction should be subject to adequate scepticism
        and scrutiny since, in all probability, the normalized
        increments presented in Figure~\ref{\SETLABEL:TF} represent a
        relatively complex process, that may not be ``modeled'' with
        such a simple methodology.

        As a further comparison of the the constructional simulation
        with the original time series data,
        Figure~\ref{\SETLABEL:TSUNFAIRBROWNIAN1} presents a normalized
        histogram of the normalized increments of the reconstructed
        time series, superimposed on the normalized histogram
        presented in Figure~\ref{\SETLABEL:NH}.

        \subidx{\market}{fiscal strategy, simulation}
        \subidx{markets}{simulation}
        \subidx{simulation}{markets}
        \subidx{strategy}{fiscal, simulation}
        \subidx{fiscal}{strategy, simulation}
        \subidx{programs}{tsunfairbrownian}
        \subidx{tsunfairbrownian}{program}
        \begin{figure}[ht]
            \begin{center}
                \begin{minipage}[t]{0.45\textwidth}
                    \epsfxsize=1.0\linewidth
                    \epsffile{\directory/tsunfairbrownian-f.eps}
                    \caption[{\market}, Time series data, empirical and
                        simulated]{{\market}, Time series data, empirical
                        and simulated, using the program {\it tsunfairbrownian}\/
                        with f = {\datafractionrms}. This data is
                        superimposed on the data presented in
                        Figure~\ref{\SETLABEL:TS}.}
                    \label{\SETLABEL:TSUNFAIRBROWNIAN0}
                \end{minipage}
                \hfill
                \begin{minipage}[t]{0.45\textwidth}
                    \epsfxsize=1.0\linewidth
                    \epsffile{\directory/tsunfairbrownian-f.tsfraction.tsnormal-s30.eps}
                    \caption[{\market}, normalized histogram,
                        empirical and simulated]{{\market}, normalized
                        histogram of the normalized increments of the
                        time series data shown in
                        Figure~\ref{\SETLABEL:TSUNFAIRBROWNIAN0},
                        empirical and simulated.  The empirical data
                        has a mean of {\datafractionmean}, with a
                        standard deviation of {\datafractionstddev}.
                        By comparison, the simulated data has a mean
                        of {\tsunfairbrownianfractionmean} with a
                        standard deviation of
                        {\tsunfairbrownianfractionstddev}. This data
                        is superimposed on the data presented in
                        Figure~\ref{\SETLABEL:NH}. The area under the
                        four curves is identical.}
                    \label{\SETLABEL:TSUNFAIRBROWNIAN1}
                \end{minipage}
            \end{center}
        \end{figure}

% Local Variables:
% TeX-parse-self: t
% TeX-auto-save: t
% TeX-master: "fractal.tex"
% End:


        %
% -----------------------------------------------------------------------------
%
% A license is hereby granted to reproduce this software source code and
% to create executable versions from this source code for personal,
% non-commercial use.  The copyright notice included with the software
% must be maintained in all copies produced.
%
% THIS PROGRAM IS PROVIDED "AS IS". THE AUTHOR PROVIDES NO WARRANTIES
% WHATSOEVER, EXPRESSED OR IMPLIED, INCLUDING WARRANTIES OF
% MERCHANTABILITY, TITLE, OR FITNESS FOR ANY PARTICULAR PURPOSE.  THE
% AUTHOR DOES NOT WARRANT THAT USE OF THIS PROGRAM DOES NOT INFRINGE THE
% INTELLECTUAL PROPERTY RIGHTS OF ANY THIRD PARTY IN ANY COUNTRY.
%
% Copyright (c) 1994-2006, John Conover, All Rights Reserved.
%
% Comments and/or bug reports should be addressed to:
%
%     john@email.johncon.com (John Conover)
%
% -----------------------------------------------------------------------------
%
% Revision: \RCSRevision \\
% Revision Time: \RCSTime UMT \\
% Revision Date: \RCSDate \\
% Revision Id: \RCSId \\
% Revision File: \RCSLog \\
\RCS $Revision: 0.0 $
\RCS $Date: 2006/01/20 04:38:13 $
\RCS $Id: maximum.tex,v 0.0 2006/01/20 04:38:13 john Exp $
% $Log: maximum.tex,v $
% Revision 0.0  2006/01/20 04:38:13  john
% Initial version
%
%
    \subsection{Simulation of Fixed Increment Approximation for Optimally Maximal Fiscal Strategy}
        \label{\SETLABEL:MAXSHANNON}
        \subidx{\market}{fiscal strategy, simulation}
        \subidx{\market}{maximum Shannon probability}
        \subidx{markets}{simulation}
        \subidx{simulation}{markets}
        \subidx{strategy}{optimum fiscal, simulation}
        \subidx{fiscal}{optimum strategy, simulation}
        \subidx{programs}{tsunfairbrownian}
        \subidx{tsunfairbrownian}{program}
        \subidx{Shannon}{probability}
        \subidx{probability}{Shannon}

        \subidx{strategy}{exploitable}
        \subidx{exploitable}{strategy}
        \subidx{programs}{tsshannonmax}
        \subidx{tsshannonmax}{program}
        \subidx{programs}{tsunfairbrownian}
        \subidx{tsunfairbrownian}{program}
        \subidx{strategy}{fiscal}
        \subidx{fiscal}{strategy}
        The data in this section is presented in tabular form in
        Section~\ref{\SETLABELREF:MAXSHANNON}. One of the issues of
        analysis, as mentioned in Section~\ref{\SETLABEL:OPS}, is to
        determine the maximum Shannon probability for the time series
        presented in Figure~\ref{\SETLABEL:TS}. Potentially, this
        could be exploited with an aggressive fiscal
        strategy. Figure~\ref{\SETLABEL:SHANNONMAX0} is a graph of the
        output of the {\it tsshannonmax}\/ program, which is described
        briefly in appendix~\ref{programs}. The maximum of this
        function is the maximum Shannon probability for the time
        series data presented in Figure~\ref{\SETLABEL:TS}.
        Figure~\ref{\SETLABEL:SHANNONMAX1} was constructed using {\it
        tsunfairbrownian}\/ program, which is also described in
        appendix~\ref{programs}, with the maximum Shannon probability,
        and the time series data presented in
        Figure~\ref{\SETLABEL:TS}. This represents a ``what if'' the
        investment strategy was changed from a Shannon probability of
        {\shannonlogreturns}, as derived in Section~\ref{\SETLABEL:SP}
        to {\shannonmax}. This process, essentially, extracts the
        random statistical data from the time series presented in
        Figure~\ref{\SETLABEL:TS}, and constructs a new time series,
        using the random statistical data, with a different investment
        strategy.  The program, {\it tsunfairbrownian}\/, essentially,
        constructs the new time series as a Brownian fractal with
        fixed increments.  The ``quality'' of such a reconstruction
        should be subject to adequate scepticism and scrutiny since,
        in all probability, the increments in the original data
        represent a relatively complex process, that may not be
        ``modeled'' with such a simple methodology.

        \begin{figure}[ht]
            \begin{center}
                \begin{minipage}[t]{0.45\textwidth}
                    \epsfxsize=1.0\linewidth
                    \epsffile{\directory/data.tsshannonmax.eps}
                    \caption[{\market}, maximum rate of revenue
                        returns] {{\market}, maximum rate of revenue
                        returns, per {\timescale}, vs. Shannon
                        probability. The maximum rate of revenue
                        returns, per {\timescale}, occurs at a Shannon
                        probability of {\shannonmax}.}
                    \label{\SETLABEL:SHANNONMAX0}
                \end{minipage}
                \hfill
                \begin{minipage}[t]{0.45\textwidth}
                    \epsfxsize=1.0\linewidth
                    \epsffile{\directory/data.tsshannonmax-p.tsunfairbrownian-p.eps}
                    \caption[{\market}, maximum rate of revenue
                        returns] {{\market}, maximum rate of revenue
                        returns, per {\timescale}, at a Shannon
                        probability, of {\shannonmax}, corresponding
                        to a ``wager'' fraction of {\twoponemax}.}
                    \label{\SETLABEL:SHANNONMAX1}
                \end{minipage}
            \end{center}
        \end{figure}

        \subidx{fractional}{Brownian motion}
        \subidx{Brownian motion}{fractional}
        \subidx{Shannon}{probability}
        \subidx{probability}{Shannon}
        \subidx{programs}{tsshannonmax}
        \subidx{tsshannonmax}{program}
        If it is assumed that the time series data set, presented in
        Figure~\ref{\SETLABEL:TS}, constitutes classical Brownian
        motion, then the Shannon probability can be calculated by
        counting the total number of {\timescale}s that the {\market}
        movement was positive, and dividing by the total number of
        {timescale}s represented in the time series. This quotient is
        {\pmax}, as compared with the predicted value from the program
        {\it tsshannonmax}\/ of {\shannonmax}.

% Local Variables:
% TeX-parse-self: t
% TeX-auto-save: t
% TeX-master: "fractal.tex"
% End:


        \subsubsection{Observations on the Simulation of Fixed Increment Approximation for Optimally Maximal Fiscal Strategy}

            Note that these simulations are base on a very, perhaps
            overly, simplified model. For example, from
            Section~\ref{\SETLABEL:TSA}, Figure~\ref{\SETLABEL:NH}, it
            would appear that the {\market}'s normalized increments
            are characterized by fractional Brownian motion---but the
            simulations used classical Brownian motion as the
            model. One consequence of this is that a re-investment
            strategy that is to ``wager'' a fraction of {\twoponemax}
            of the rate of returns every {\timescale} is overly
            aggressive, since in the classical Brownian scenario, the
            maximum loss, in any {\timescale}, was no more that what
            was ``wagered.'' However, in the fractional Brownian
            scenario, much more can be lost. From
            Equation~\ref{fopt2},

            \begin{equation}
                \frac{avg}{rms^2} = \frac{f_{opt}}{rms} = K
            \end{equation}

            \noindent where, under the optimum classical Brownian
            scenario, $K$ is unity, or $avg = rms^2$. Notice that,
            since $f = rms$, whether the scenario is optimal or not,
            that the operational ``wager'' fraction, from
            Figure~\ref{\SETLABEL:TF} of {\datafractionrms}, vs.\ an
            ``theoretical optimal'' value of {\twoponemax} seems
            overly conservative. Additionally, notice that, at least
            in principle, the chance of failure in the fractional
            Brownian scenario, which is more accurate, would
            correspond to 1 standard deviation, or about 15.865\% per
            {\timescale}, which is unacceptably high. However, it is
            not clear why the {\market} is running at a value of
            {\datafractionrms}, which seems very
            conservative. However, a re-investment strategy of
            {\datafractionrms} per {\timescale} does not seem
            inconsistent with a failure rate, on the Fortune 500 list,
            which it is inferred that the {\market} is similar to, of
            about 50\% in ten years, which corresponds to $(1 -
            p_f)^{120} \approx 0.5$, or $p_f$, the probability of
            failure, is $0.005759576$, which is, approximately, 2.5
            standard deviations, meaning that to be consistent with
            the large companies in the Fortune 500, the re-investment
            rate should be, approximately, $\frac{\twoponemax}{2.5}$,
            compared with an operational value, from
            Figure~\ref{\SETLABEL:NH} of {\datafractionrms}.

            An interesting, and intriguing, interpretation and
            discussion of the maximum Shannon probability, is an
            explanation as to why the companies in the {\market} are
            not running an optimal re-investment strategy. This seems
            enigmatic, since those companies that run, on a long term
            average, below the optimally maximal value would seem to
            be eclipsed by those that didn't. And those that run above
            the optimally maximal value would be over extended, and
            become financially destitute during market down turns,
            which is inevitable in a fractal time series as presented
            in Figure~\ref{\SETLABEL:TS}.  It would seem that the
            natural selection process of the competitive environment
            would allow only those companies that run near the
            optimally maximal value to survive, in the long run. One
            possible explanation, foremost, is that the analytical
            methodology presented herein is inappropriate.  Another
            explanation is that the gross margins are less than the
            fraction {\shannonmax} of the rate of revenue returns, and
            thus could not accommodate such an aggressive
            re-investment strategy. If this is the case, then it
            presents an intriguing issue. If, in a capitalistic
            market, the natural outcome of the competitive situation,
            according to game-theoretic analysis, is that there will
            be many competitors, each making minimal gross margins,
            then how do the companies grow their markets?  Naturally,
            those that run the most efficient will have lower costs,
            making larger percentage of rate of revenue returns
            re-investment possible. Yet another interpretation is that
            the number of competitors would grow at an exponential
            rate, but all of them would make minimal returns. However,
            an operational Shannon probability of {\shannonlogreturns}
            is not just marginally lower than the maximum Shannon
            probability of {\shannonmax}. There is a significant
            disparity which is difficult to explain. It would seem
            that the game-theoretic eventual outcome of a competitive
            market place would be a solution that hinders growth,
            wealth and jobs creation, etc., which does not seem
            consistent with capitalistic theory. On the other hand, is
            there an optimum number of competitors in a market place,
            where the gross margins can be higher, permitting wealth
            and job creation, and also a competitive situation? If
            this analysis is correct, and that should be subject to
            scrutiny, then it would appear that this is the case. But
            this brings up another issue---that of taxation, and other
            contributions to the social welfare function. If there is
            an optimum number of competitors in the market place, that
            maximizes wealth and job creation, then, perhaps by lemma,
            there is also an optimal value of taxation rate, and other
            contributions to the social welfare function, that will
            permit maximal industrial growth, and thus maximal growth
            in the tax base. But this would seem to be inconsistent
            with the work of Kenneth Arrow and the so called
            Impossibility Theorem, which states that such
            optimizations can not be determined because the ordering
            of priorities are intransitive.  All very perplexing,
            since the simulation of the maximum Shannon probability in
            the next section seems to indicate that such an aggressive
            re-investment strategy is, indeed, feasible.

            Yet another possibility for the industry not running at
            maximum Shannon probability is the high cost of expansion
            of operations. Some of these industries require very
            sophisticated manufacturing processes, which have high
            barrier costs.

            Additionally, as mentioned in both~\cite[pp. 29]{Brock},
            and~\cite[pp. 8]{Arthur:CTIRALIBHE}, optimal efficiency
            may not be attainable in increasing-return economic
            scenarios.

        %
% -----------------------------------------------------------------------------
%
% A license is hereby granted to reproduce this software source code and
% to create executable versions from this source code for personal,
% non-commercial use.  The copyright notice included with the software
% must be maintained in all copies produced.
%
% THIS PROGRAM IS PROVIDED "AS IS". THE AUTHOR PROVIDES NO WARRANTIES
% WHATSOEVER, EXPRESSED OR IMPLIED, INCLUDING WARRANTIES OF
% MERCHANTABILITY, TITLE, OR FITNESS FOR ANY PARTICULAR PURPOSE.  THE
% AUTHOR DOES NOT WARRANT THAT USE OF THIS PROGRAM DOES NOT INFRINGE THE
% INTELLECTUAL PROPERTY RIGHTS OF ANY THIRD PARTY IN ANY COUNTRY.
%
% Copyright (c) 1994-2006, John Conover, All Rights Reserved.
%
% Comments and/or bug reports should be addressed to:
%
%     john@email.johncon.com (John Conover)
%
% -----------------------------------------------------------------------------
%
% Revision: \RCSRevision \\
% Revision Time: \RCSTime UMT \\
% Revision Date: \RCSDate \\
% Revision Id: \RCSId \\
% Revision File: \RCSLog \\
\RCS $Revision: 0.0 $
\RCS $Date: 2006/01/20 04:38:13 $
\RCS $Id: verification.tex,v 0.0 2006/01/20 04:38:13 john Exp $
% $Log: verification.tex,v $
% Revision 0.0  2006/01/20 04:38:13  john
% Initial version
%
%
    \subsection{Qualitative Verification of Fixed Increment Approximation Analysis}
        \label{\SETLABEL:QVA}

        \subidx{\market}{verification of analysis}
        \subidx{verification}{analysis}
        \subidx{analysis}{verification}
        \subidx{quality}{of analysis}
        \subidx{verification}{of methodology}
        \subidx{methodology}{verification of}
        \subidx{Shannon}{probability}
        \subidx{probability}{Shannon}

        This section evaluates various values based on the ``average''
        of the normalized increments presented in
        Figure~\ref{\SETLABEL:TFA}. These values are an approximation
        to a, probably, complex process with a distribution shown in
        Figure~\ref{\SETLABEL:TF}. These values will be used in a
        fixed increment Brownian fractal analysis of the {\market},
        and may, or may not, provide adequate accuracy for
        projections.

        The data in this section is presented in tabular form in
        sections~\ref{\SETLABELREF:VI1} and~\ref{\SETLABELREF:VI2}.
        As a subjective evaluation of the ``quality'' of the analysis
        of the {\market}, from Chapter~\ref{methodology},
        Equation~\ref{metricvalues1}, and using the mean and root mean
        square values of the normalized increments of the time series
        data presented in Figure~\ref{\SETLABEL:TS} from
        Figure~\ref{\SETLABEL:TF}, and the Shannon probability as
        calculated by counting the total number of {\timescale}s that
        the {\market} movement was positive, as presented in
        Section~\ref{\SETLABEL:MAXSHANNON}:

        \begin{eqnarray}
                  P & \approx & \frac{\frac{avg}{rms} + 1}{2}\\
            {\pmax} & \approx & \frac{\frac{\datafractionmean}{\datafractionrms} + 1}{2}\\
            {\pmax} & \approx & {\avgrms}
            \label{\SETLABEL:AVGS}
        \end{eqnarray}

        \subidx{Shannon}{probability}
        \subidx{probability}{Shannon}
        \noindent and comparing these values to the Shannon
        probability, as found by the {\it tsshannonmax}\/ program, which
        iterates for a maximum:

        \begin{eqnarray}
            {\pmax} \approx {\avgrms} \approx {\shannonmax}
        \end{eqnarray}

        \subidx{logarithmic}{returns}
        \subidx{returns}{logarithmic}
        In addition, the different methods of calculating the
        logarithmic returns, presented in Section~\ref{\SETLABEL:FS},
        should be compared. The four methods used were the mean of
        Figure~\ref{\SETLABEL:TF}, the constant in the least squares
        approximation to Figure~\ref{\SETLABEL:TF}, the least squares
        exponential approximation to Figure~\ref{\SETLABEL:TS}, and
        the logarithmic returns of Figure~\ref{\SETLABEL:TS}, derived
        as the mean of the logarithms of the quotients of the
        increments. The values for each of the methods are,
        respectively:

        \begin{equation}
            \datafractionmeanbits \approx \datafractionconstantbits \approx \datatslsqepbits \approx \logreturns
        \end{equation}

        It is implied in Section~\ref{\SETLABEL:FS},
        Subsection~\ref{\SETLABEL:SP} and in
        Section~\ref{\SETLABEL:TSUNFAIRBROWNIAN} that, a Brownian
        motion with fixed increments fractal may ``model'' the
        {\market}. Using Equation~\ref{stddev9} from
        Chapter~\ref{general}, Section~\ref{abmfi}:

        \begin{eqnarray}
                                    rms \left(2P - 1\right) & \approx & \frac{\sigma \left(2P - 1\right)}{2 \sqrt{P\left(1 - P\right)}}\\
            \datafractionrms \left(2 \cdot \pmax - 1\right) & \approx & \frac{\datafractionstddev \left(2 \cdot \pmax - 1\right)}{2\sqrt{\pmax \left(1 - \pmax\right)}}\\
                       \datafractionrms \cdot \twopminusone & \approx & \datafractionstddev \cdot \twopx\\
                                                      \rmsp & \approx & \sigmap
        \end{eqnarray}

        \noindent and, equating to the mean:

        \begin{equation}
            \datafractionmean \approx \rmsp \approx \sigmap
        \end{equation}

        \subidx{Shannon}{probability}
        \subidx{probability}{Shannon}
        \noindent where, as in Equation~\ref{\SETLABEL:AVGS} using the
        mean, root mean square, and standard deviation values of the
        normalized increments of the time series data presented in
        Figure~\ref{\SETLABEL:TS} from Figure~\ref{\SETLABEL:TF}, and
        the Shannon probability as calculated by counting the total
        number of {\timescale}s that the {\market} movement was
        positive, as presented in Section~\ref{\SETLABEL:MAXSHANNON}.

        As a final qualitative comparison, the absolute value of the
        normalized increments should be the same as the root mean
        square value\footnote{The absolute value of the normalized
        increments, when averaged, is related to the root mean square
        of the increments by a constant. If the normalized increments
        are a fixed increment, the constant is unity. If the
        normalized increments have a Gaussian distribution, the
        constant is $\approx 0.8$ depending on the accuracy of of
        ``fit'' to a Gaussian distribution.}, where the absolute value
        is presented in Figure~\ref{\SETLABEL:TFA}, and the root mean
        square value is presented in Figure~\ref{\SETLABEL:TF}:

        \begin{equation}
            \datafractionabsmean \approx \datafractionrms
        \end{equation}

        Note, that if the {\market} could be ``modeled'' as a Brownian
        motion with fixed increments fractal, then the standard
        deviation of the absolute value of the normalized increments
        of the time series data presented in Figure~\ref{\SETLABEL:TS}
        from Figure~\ref{\SETLABEL:TF} should be zero. It is
        $\datafractionabsstddev$.

% Local Variables:
% TeX-parse-self: t
% TeX-auto-save: t
% TeX-master: "fractal.tex"
% End:


    \renewcommand{\market}{United States Electronic Component Production}
    \renewcommand{\directory}{../markets/electronic.components.production}
    \renewcommand{\datafractionmean}{0.008052}
\renewcommand{\datafractionmeanbits}{0.011570}
\renewcommand{\datafractionmeanq}{0.002684}
\renewcommand{\datafractionmeanbitsq}{0.003867}
\renewcommand{\datafractionstddev}{0.038579}
\renewcommand{\datafractionrms}{0.039311}
\renewcommand{\avgrms}{0.602414}
\renewcommand{\ncompanies}{5.210454}
\renewcommand{\pncompanies}{0.544866}
\renewcommand{\datafractionabsmean}{0.029745}
\renewcommand{\datafractionabsstddev}{0.025769}
\renewcommand{\datafractionconstant}{0.010041}
\renewcommand{\datafractionconstantbits}{0.014414}
\renewcommand{\datafractionconstantq}{0.003347}
\renewcommand{\datafractionconstantbitsq}{0.004821}
\renewcommand{\datafractionslope}{-0.000021}
\renewcommand{\datafractionabsconstant}{0.035145}
\renewcommand{\datafractionabsslope}{-0.000057}
\renewcommand{\hurstall}{0.659558}
\renewcommand{\hurstlow}{0.707509}
\renewcommand{\hurstlowtwo}{1.415018}
\renewcommand{\hurstlowhundred}{70.750900}
\renewcommand{\hcalcall}{0.184942}
\renewcommand{\hcalclow}{0.102042}
\renewcommand{\shannonmax}{0.604167}
\renewcommand{\twoponemax}{0.208334}
\renewcommand{\logreturns}{0.010456}
\renewcommand{\twologreturns}{1.007274}
\renewcommand{\twologreturnshundred}{0.727387}
\renewcommand{\oneoverlogreturns}{95.638868}
\renewcommand{\pmax}{0.602094}
\renewcommand{\twopminusone}{0.204188}
\renewcommand{\rmsp}{0.008027}
\renewcommand{\twopx}{0.208583}
\renewcommand{\sigmap}{0.008047}
\renewcommand{\tsunfairbrownianfractionmean}{0.007862}
\renewcommand{\tsunfairbrownianfractionstddev}{0.038619}
\renewcommand{\shannonlogreturns}{0.560125}
\renewcommand{\shannonlogreturnshundred}{56.012500}
\renewcommand{\twopone}{0.120250}
\renewcommand{\twoponehundred}{12.025000}
\renewcommand{\hundredtwoponehundred}{87.975000}
\renewcommand{\hundredshannonlogreturnshundred}{43.987500}
\renewcommand{\datatslsqepbits}{0.007623}
\renewcommand{\thurstall}{0.633980}
\renewcommand{\thurstlow}{0.710108}
\renewcommand{\thurstlowtwo}{1.420216}
\renewcommand{\thurstlowhundred}{71.010800}
\renewcommand{\thcalcall}{0.247886}
\renewcommand{\thcalclow}{0.171737}
\renewcommand{\chisquared}{2.862000}
\renewcommand{\critical}{42.557000}

    \renewcommand{\timescale}{month}
    \subidx{market}{\market}
    \idx{\market}

    \section{\market}

        \renewcommand{\SETLABEL}{\LABPRE:NAECP}
        \renewcommand{\SETLABELQ}{\LABPRE:NAECPQ}
        \label{\SETLABEL}
        \renewcommand{\SETLABELREF}{\LABPREREF:NAECP}

        \idx{United States Department of Commerce}
        For the analysis, the data was in the directory
        {\directory}\footnote{Data from the United States Department
        of Commerce, 1980---1994, by {\timescale}s, as an index, 1987
        = 100.}.

        The data in this section is presented in tabular form in
        Section~\ref{\SETLABELREF}.

        %
% -----------------------------------------------------------------------------
%
% A license is hereby granted to reproduce this software source code and
% to create executable versions from this source code for personal,
% non-commercial use.  The copyright notice included with the software
% must be maintained in all copies produced.
%
% THIS PROGRAM IS PROVIDED "AS IS". THE AUTHOR PROVIDES NO WARRANTIES
% WHATSOEVER, EXPRESSED OR IMPLIED, INCLUDING WARRANTIES OF
% MERCHANTABILITY, TITLE, OR FITNESS FOR ANY PARTICULAR PURPOSE.  THE
% AUTHOR DOES NOT WARRANT THAT USE OF THIS PROGRAM DOES NOT INFRINGE THE
% INTELLECTUAL PROPERTY RIGHTS OF ANY THIRD PARTY IN ANY COUNTRY.
%
% Copyright (c) 1994-2006, John Conover, All Rights Reserved.
%
% Comments and/or bug reports should be addressed to:
%
%     john@email.johncon.com (John Conover)
%
% -----------------------------------------------------------------------------
%
% Revision: \RCSRevision \\
% Revision Time: \RCSTime UMT \\
% Revision Date: \RCSDate \\
% Revision Id: \RCSId \\
% Revision File: \RCSLog \\
\RCS $Revision: 0.0 $
\RCS $Date: 2006/01/20 04:38:13 $
\RCS $Id: fraction.tex,v 0.0 2006/01/20 04:38:13 john Exp $
% $Log: fraction.tex,v $
% Revision 0.0  2006/01/20 04:38:13  john
% Initial version
%
%
    \subsection{Time Series Increments Analysis}
        \label{\SETLABEL:TSA}

        \subidx{\market}{Time series analysis}
        \subidx{time series}{increments}
        \subidx{time series}{analysis}
        \subidx{cumulative sum}{analysis}
        \subidx{analysis}{cumulative sum}
        \subidx{analysis}{random process}
        \subidx{random process}{analysis}
        \subidx{Gaussian}{increments}
        \subidx{increments}{Gaussian}
        \subidx{Brownian}{motion, fractional}
        \subidx{fractional}{Brownian motion}
        \subidx{fractal}{Brownian motion}
        The data in this section is presented in tabular form in
        Section~\ref{\SETLABELREF:TSA}.  Figure~\ref{\SETLABEL:TS} is
        a graph of the time series data for the {\market}.

        \subidx{increments}{normalized}
        \subidx{normalized}{increments}
        \subidx{programs}{tsfraction}
        \subidx{tsfraction}{program}
        Figure~\ref{\SETLABEL:TF} is a graph of the normalized
        increments of the time series data presented in
        Figure~\ref{\SETLABEL:TS}. The data presented was made by
        running the program {\it tsfraction}\/ on the time series
        data. The program {\it tsfraction}\/ is described briefly in
        Appendix~\ref{programs}, and subtracts the previous value from
        the next value, dividing this difference by the previous
        value, for each element in the time series data. The new time
        series contains the instantaneous change in the rate of
        revenue returns, divided by the magnitude of the instantaneous
        rate of revenue returns.

        \subidx{mean}{standard deviation}
        \subidx{standard deviation}{mean}
        \idx{root mean square}
        \idx{least squares approximation}
        \begin{figure}[ht]
            \begin{center}
                \begin{minipage}[t]{0.45\textwidth}
                    \epsfxsize=1.0\linewidth
                    \epsffile{\directory/data.eps}
                    \caption{{\market}, time series data.}
                    \label{\SETLABEL:TS}
                    \label{\SETLABELQ:TS}
                \end{minipage}
                \hfill
                \begin{minipage}[t]{0.45\textwidth}
                    \epsfxsize=1.0\linewidth
                    \epsffile{\directory/data.tsfraction.eps}
                    \caption[{\market}, normalized
                        increments]{{\market}, normalized increments
                        of the time series data presented in
                        Figure~\ref{\SETLABEL:TS}. The mean is
                        {\datafractionmean} with a standard deviation
                        of {\datafractionstddev}. The formula for the
                        least squares approximation is
                        ${\datafractionconstant} +
                        {\datafractionslope}t$, and the root mean
                        squared value is {\datafractionrms}. The
                        graph, labeled ``data\-.tsfraction\-.tsrms,''
                        is the running root mean square, and
                        ``data\-.tsfraction\-.tsavg'' is the running
                        average of the normalized increments.  This
                        graph is the fraction of change in the time
                        series, as a function of time. Note that the
                        slope of the mean, {\datafractionslope}, is
                        the coefficient of the nonlinearity term in
                        the normalized increments. See
                        Chapter~\ref{general}, Section~\ref{nlextend}
                        for a possible application of the logistic
                        function to this data set.}
                    \label{\SETLABEL:TF}
                    \label{\SETLABELQ:TF}
                \end{minipage}
            \end{center}
        \end{figure}

        \subidx{absolute value}{increments}
        \subidx{increments}{absolute value}

        Figure~\ref{\SETLABEL:TFA} is a graph of the absolute value of
        the normalized increments of the time series data presented in
        Figure~\ref{\SETLABEL:TF}. The data presented was made by
        running the Unix utility sed(1) on the normalized increments
        time series data to remove the negative signs. This is an
        absolute value procedure.  The resulting time series contains
        the absolute value of the instantaneous change in the rate of
        revenue returns, divided by the magnitude of the instantaneous
        rate of revenue returns\footnote{The absolute value of the
        normalized increments, when averaged, is related to the root
        mean square of the increments by a constant. If the normalized
        increments are a fixed increment, the constant is unity. If
        the normalized increments have a Gaussian distribution, the
        constant is $\approx 0.8$ depending on the accuracy of of
        ``fit'' to a Gaussian distribution.}.

        \subidx{histogram}{normalized}
        \subidx{normalized}{histogram}
        \subidx{programs}{tsnormal}
        \subidx{tsnormal}{program}
        \subidx{mean}{standard deviation}
        \subidx{standard deviation}{mean}
        \idx{root mean square}
        \idx{least squares approximation}
        \subidx{\market}{analysis of increments}
        Figure~\ref{\SETLABEL:NH} is the normalized histogram of the
        normalized increments of the time series data shown in
        Figure~\ref{\SETLABEL:TF}. The abscissa is 3 $\sigma$ limits,
        and the area under the two curves is identical. The data for
        this figure was produced by the program {\it tsnormal}\/,
        which is described briefly in Appendix~\ref{programs}.

        \begin{figure}[ht]
            \begin{center}
                \begin{minipage}[t]{0.45\textwidth}
                    \epsfxsize=1.0\linewidth
                    \epsffile{\directory/data.tsfraction.abs.eps}
                    \caption[{\market}, absolute value of the
                        normalized increments]{{\market}, absolute
                        value of the normalized increments of the time
                        series data presented in
                        Figure~\ref{\SETLABEL:TF}.  The mean is
                        {\datafractionabsmean} with a standard
                        deviation of {\datafractionabsstddev}. The
                        formula for the least squares approximation is
                        ${\datafractionabsconstant} +
                        {\datafractionabsslope}t$, and the root mean
                        square value, from Figure~\ref{\SETLABEL:TF},
                        is {\datafractionrms}.  The graph, labeled
                        ``data\-.tsfraction\-.tsrms,'' is the running
                        root mean square, and
                        ``data\-.tsfraction\-.tsavg'' is the running
                        average of the normalized increments presented
                        in Figure~\ref{\SETLABEL:TF}, superimposed
                        here for convenience. This graph is the
                        absolute value of the fraction of change in
                        the time series, as a function of time.}
                    \label{\SETLABEL:TFA}
                    \label{\SETLABELQ:TFA}
                \end{minipage}
                \hfill
                \begin{minipage}[t]{0.45\textwidth}
                    \epsfxsize=1.0\linewidth
                    \epsffile{\directory/data.tsfraction.tsnormal-s30.eps}
                    \caption[{\market}, normalized histogram of the
                        normalized increments]{{\market}, normalized
                        histogram of the normalized increments of the
                        time series data shown in
                        Figure~\ref{\SETLABEL:TF}.  The data has a
                        mean of {\datafractionmean}, with a standard
                        deviation of {\datafractionstddev}.  The area
                        under the two curves is identical. The
                        $\chi^2$ value of the observed and expected
                        values of the two curves is {\chisquared},
                        with a critical value of {\critical}.}
                    \label{\SETLABEL:NH}
                \end{minipage}
            \end{center}
        \end{figure}

        \subidx{programs}{tsXsquared}
        \subidx{tsXsquared}{program}
        \subidx{\market}{chi-squared values of increments}
        The program {\it tsXsquared}\/, which is briefly described in
        appendix~\ref{programs}, was used to derive the $\chi^2$
        statistics for the data presented in
        Figure~\ref{\SETLABEL:NH}.

        \subidx{programs}{tsstatest}
        \subidx{tsstatest}{program}
        \subidx{\market}{statistical estimates}

        Figure~\ref{\SETLABEL:SE} is the statistical estimate for the
        data presented in Figure~\ref{\SETLABEL:TF}, as derived by the
        program {\it tsstatest}\/, which is briefly described in
        appendix~\ref{programs}.

        \begin{figure}[ht]
            \begin{center}
                \begin{minipage}[t]{\textwidth}
                    \center{\fbox{\parbox{0.9\textwidth}{\XXX{\directory/data.tsstatest-f0.1-c0.9-i.tex}}}}
                    \caption[{\market}, statistical estimates of the
                        normalized increments]{{\market}, statistical
                        estimates of the normalized increments of the
                        time series shown in Figure~\ref{\SETLABEL:TF}.
                        The table was produced with the {\it
                        tsstatest}\/ program, and illustrates the
                        size of the data set required for a confidence
                        level of 90\%, with an error estimate of $\pm$
                        10\%, or alternately, the error estimate on
                        the time series shown in Figure~\ref{\SETLABEL:TF}.}
                    \label{\SETLABEL:SE}
                \end{minipage}
            \end{center}
        \end{figure}

        Note that the data set size estimations, as produced by the
        {\it tsstatest}\/ program, are probably very conservative,
        depending on the magnitude of the Shannon probability, $P =
        \shannonlogreturns$, as derived in
        Section~\ref{\SETLABEL:SP}. See Chapter~\ref{general},
        Section~\ref{serdss} for possible alternative methodologies
        for addressing the analysis of fractal time series with
        limited data set sizes. Depending on the magnitude of the
        Shannon probability, $P$, these estimates can be several
        orders of magnitude too high.

        \subidx{derivative of increments}{normalized}
        \subidx{normalized}{derivative of increments}
        \subidx{programs}{tsderivative}
        \subidx{tsderivative}{program}
        Figure~\ref{\SETLABEL:TF1} is the normalized histogram of the
        first derivative of the normalized increments of the time
        series data shown in Figure~\ref{\SETLABEL:TF}. In principle,
        if the distribution of the normalized increments presented in
        Figure~\ref{\SETLABEL:NH} is Gaussian in nature, this
        distribution would be similar to ``white noise,'' as presented
        in appendix~\ref{programs}, Figure~\ref{whiteexample}. The
        data was generated by the {\it tsderivative}\/ program, which
        is briefly described in
        appendix~\ref{programs}. Figure~\ref{\SETLABEL:TF2} is the
        normalized histogram of the second derivative of the
        normalized increments of the time series data shown in
        Figure~\ref{\SETLABEL:TF}. In principle, if the distribution
        of the normalized increments presented in
        Figure~\ref{\SETLABEL:NH} is an integrated Gaussian
        distribution in nature, this distribution would be similar to
        ``white noise,'' as presented in appendix~\ref{programs},
        Figure~\ref{whiteexample}.

        \begin{figure}[ht]
            \begin{center}
                \begin{minipage}[t]{0.45\textwidth}
                    \epsfxsize=1.0\linewidth
                    \epsffile{\directory/data.tsfraction.tsderivative.tsnormal-s30.eps}
                    \caption[{\market}, histogram of the first
                        derivative of the increments]{{\market},
                        normalized histogram of the first derivative
                        of the normalized increments of the time
                        series data shown in
                        Figure~\ref{\SETLABEL:TF}.}
                    \label{\SETLABEL:TF1}
                \end{minipage}
                \hfill
                \begin{minipage}[t]{0.45\textwidth}
                    \epsfxsize=1.0\linewidth
                    \epsffile{\directory/data.tsfraction.2tsderivative.tsnormal-s30.eps}
                    \caption[{\market}, histogram of the second
                        derivative of the increments]{{\market},
                        normalized histogram of second derivative of
                        the the normalized increments of the time
                        series data shown in
                        Figure~\ref{\SETLABEL:TF}.}
                    \label{\SETLABEL:TF2}
                \end{minipage}
            \end{center}
        \end{figure}

        \subidx{fractal}{range}
        \subidx{fractal}{R/S analysis}
        \subidx{\market}{rate of revenue returns, range}
        \subidx{\market}{deterministic mechanism}
        \subidx{deterministic}{mechanism}
        \subidx{mechanism}{deterministic}
        Figure~\ref{\SETLABEL:TR} is the range of values of the time
        series shown in Figure~\ref{\SETLABEL:TS}. The horizontal axis
        is time into the future. In principle, if the time series was
        characterized as fractional Brownian motion the graph in
        Figure~\ref{\SETLABEL:TR} would be a square root
        function\footnote{Note that the ``roughness,'' or ``sawtooth''
        characteristics of the graph in Figure~\ref{\SETLABEL:TR} are
        a computational artifact---caused by not using the -m option
        to the program {\it tshurst}\/, which is computationally
        inefficient.}. Figure~\ref{\SETLABEL:TD} is the deterministic
        map of the normalized increments of the time series data shown
        in Figure~\ref{\SETLABEL:TF}. The deterministic map is useful
        for determining if a time series was created by a
        deterministic mechanism. This, essentially, maps each element
        in the time series with the previous element in the time
        series.  See,~\cite[pp. 745]{Peitgen}.

        \begin{figure}[ht]
            \begin{center}
                \begin{minipage}[t]{0.45\textwidth}
                    \epsfxsize=1.0\linewidth
                    \epsffile{\directory/data.tshurst-f.eps}
                    \caption[{\market}, range]{{\market}, range of the
                        time series data shown in
                        Figure~\ref{\SETLABEL:TS}.}
                    \label{\SETLABEL:TR}
                \end{minipage}
                \hfill
                \begin{minipage}[t]{0.45\textwidth}
                    \epsfxsize=1.0\linewidth
                    \epsffile{\directory/data.tsfraction.tsdeterministic.eps}
                    \caption[{\market}, deterministic map]{{\market},
                        deterministic map of the normalized increments
                        of the time series data shown in
                        Figure~\ref{\SETLABEL:TF}.}
                    \label{\SETLABEL:TD}
                \end{minipage}
            \end{center}
        \end{figure}

% Local Variables:
% TeX-parse-self: t
% TeX-auto-save: t
% TeX-master: "fractal.tex"
% End:


        \subsubsection{Observations on the Time Series Increments Analysis}

            Figure~\ref{\SETLABEL:NH} would seem to indicate that the
            time series data for the {\market} represents a cumulative
            sum/integration of a random process that has a Gaussian
            distribution, (ie., satisfies the Gaussian increments
            property of fractional Brownian
            motion~\cite[pp. 250]{Crownover},) tending to justify the
            assumption that the time series data represents fractional
            Brownian motion.

        %
% -----------------------------------------------------------------------------
%
% A license is hereby granted to reproduce this software source code and
% to create executable versions from this source code for personal,
% non-commercial use.  The copyright notice included with the software
% must be maintained in all copies produced.
%
% THIS PROGRAM IS PROVIDED "AS IS". THE AUTHOR PROVIDES NO WARRANTIES
% WHATSOEVER, EXPRESSED OR IMPLIED, INCLUDING WARRANTIES OF
% MERCHANTABILITY, TITLE, OR FITNESS FOR ANY PARTICULAR PURPOSE.  THE
% AUTHOR DOES NOT WARRANT THAT USE OF THIS PROGRAM DOES NOT INFRINGE THE
% INTELLECTUAL PROPERTY RIGHTS OF ANY THIRD PARTY IN ANY COUNTRY.
%
% Copyright (c) 1994-2006, John Conover, All Rights Reserved.
%
% Comments and/or bug reports should be addressed to:
%
%     john@email.johncon.com (John Conover)
%
% -----------------------------------------------------------------------------
%
% Revision: \RCSRevision \\
% Revision Time: \RCSTime UMT \\
% Revision Date: \RCSDate \\
% Revision Id: \RCSId \\
% Revision File: \RCSLog \\
\RCS $Revision: 0.0 $
\RCS $Date: 2006/01/20 04:38:13 $
\RCS $Id: instant.tex,v 0.0 2006/01/20 04:38:13 john Exp $
% $Log: instant.tex,v $
% Revision 0.0  2006/01/20 04:38:13  john
% Initial version
%
%
    \subsection{Instantaneous Analysis of Normalized Increments}
        \label{\SETLABEL:IA}

        \subidx{\market}{instantaneous analysis of normalized increments}
        \idx{average of normalized increments}
        \idx{root mean square of normalized increments}
        \subidx{Shannon probability}{instantaneous computation of}
        \subidx{average of normalized increments}{instantaneous computation of}
        \subidx{root mean square of normalized increments}{instantaneous computation of}
        \subidx{instantaneous computation}{Shannon probability}
        \subidx{instantaneous computation}{average of normalized increments}
        \subidx{instantaneous computation}{root mean square of normalized increments}
        \idx{time series}
        \subidx{time series}{instantaneous analysis}
        \subidx{instantaneous analysis}{time series}
        \subidx{time series}{increments}
        \subidx{time series}{analysis}
        \subidx{Shannon}{probability}
        \subidx{probability}{Shannon}
        \subidx{normalized}{increments}
        \subidx{increments}{normalized}

        The program {\it tsinstant}\/, which is briefly described in
        Appendix~\ref{programs}, is for finding the instantaneous
        fraction of change in a time series. The value of a sample in
        the time series is subtracted from the previous sample in the
        time series, and divided by the value of the previous sample.
        As explained in Chapter~\ref{general},
        Sections~\ref{derivation},~\ref{GA},~\ref{abmfi},~\ref{aftsma}
        and,~\ref{ompl} for Brownian motion, random walk fractals, the
        absolute value of the instantaneous fraction of change is also
        the root mean square of the instantaneous fraction of
        change\footnote{The absolute value of the normalized
        increments, when averaged, is related to the root mean square
        of the increments by a constant. If the normalized increments
        are a fixed increment, the constant is unity. If the
        normalized increments have a Gaussian distribution, the
        constant is $\approx 0.8$ depending on the accuracy of of
        ``fit'' to a Gaussian distribution.}. Squaring this value is
        the average of the instantaneous fraction of change, and
        adding unity to the absolute value of the instantaneous
        fraction of change, and dividing by two, is the Shannon
        probability of the instantaneous fraction of change.

        Figure~\ref{\SETLABEL:IA1} is the instantaneous value of the
        root mean square of the normalized increments for the
        {\market}, and Figure~\ref{\SETLABEL:IA2} is the instantaneous
        Shannon probability for the normalized increments.

        \begin{figure}[ht]
            \begin{center}
                \begin{minipage}[t]{0.45\textwidth}
                    \epsfxsize=1.0\linewidth
                    \epsffile{\directory/data.tsinstant-r.eps}
                    \caption[{\market}, instantaneous value of
                        rms.]{{\market}, instantaneous value of the
                        root mean square of the normalized increments,
                        provided by running the program {\it
                        tsinstant}\/ with the -r option on the data
                        presented in Figure~\ref{\SETLABEL:TS}.}
                    \label{\SETLABEL:IA1}
                    \label{\SETLABELQ:IA1}
                \end{minipage}
                \hfill
                \begin{minipage}[t]{0.45\textwidth}
                    \epsfxsize=1.0\linewidth
                    \epsffile{\directory/data.tsinstant-s.eps}
                    \caption[{\market}, instantaneous value of
                        Shannon probability.]{{\market}, instantaneous
                        value of the Shannon probability of the
                        normalized increments, provided by running the
                        program {\it tsinstant}\/ with the -s option
                        on the data presented in
                        Figure~\ref{\SETLABEL:TS}.}
                    \label{\SETLABEL:IA2}
                    \label{\SETLABELQ:IA2}
                \end{minipage}
            \end{center}
        \end{figure}

% Local Variables:
% TeX-parse-self: t
% TeX-auto-save: t
% TeX-master: "fractal.tex"
% End:


        %
% -----------------------------------------------------------------------------
%
% A license is hereby granted to reproduce this software source code and
% to create executable versions from this source code for personal,
% non-commercial use.  The copyright notice included with the software
% must be maintained in all copies produced.
%
% THIS PROGRAM IS PROVIDED "AS IS". THE AUTHOR PROVIDES NO WARRANTIES
% WHATSOEVER, EXPRESSED OR IMPLIED, INCLUDING WARRANTIES OF
% MERCHANTABILITY, TITLE, OR FITNESS FOR ANY PARTICULAR PURPOSE.  THE
% AUTHOR DOES NOT WARRANT THAT USE OF THIS PROGRAM DOES NOT INFRINGE THE
% INTELLECTUAL PROPERTY RIGHTS OF ANY THIRD PARTY IN ANY COUNTRY.
%
% Copyright (c) 1994-2006, John Conover, All Rights Reserved.
%
% Comments and/or bug reports should be addressed to:
%
%     john@email.johncon.com (John Conover)
%
% -----------------------------------------------------------------------------
%
% Revision: \RCSRevision \\
% Revision Time: \RCSTime UMT \\
% Revision Date: \RCSDate \\
% Revision Id: \RCSId \\
% Revision File: \RCSLog \\
\RCS $Revision: 0.0 $
\RCS $Date: 2006/01/20 04:38:13 $
\RCS $Id: logistic.tex,v 0.0 2006/01/20 04:38:13 john Exp $
% $Log: logistic.tex,v $
% Revision 0.0  2006/01/20 04:38:13  john
% Initial version
%
%
    \subsection{Logistic Analysis}
        \label{\SETLABEL:LA}

        \subidx{\market}{Logistic function analysis}
        \subidx{time series}{logistic function}
        \subidx{logistic function}{time series}
        \subidx{time series}{increments}
        \subidx{time series}{analysis}
        \subidx{cumulative sum}{analysis}
        \subidx{analysis}{cumulative sum}
        \subidx{analysis}{random process}
        \subidx{random process}{analysis}
        The data in this section is presented in tabular form in
        Section~\ref{\SETLABELREF:LAA}.  Figure~\ref{\SETLABEL:LA1} is
        a graph of the logistic function estimates of the time series
        data for the {\market}. The reader is cautioned that these
        graphs are constructed using the method suggested in
        Chapter~\ref{general}, Section~\ref{nlextend} and enormous
        precision is required for adequate prediction of the logistic
        function,~\cite{Modis}. Particularly, the non-linear term will
        usually require intervention to produce a practical fit to the
        data. In addition, there are numerical stability issues with
        logistic function methodologies\footnote{For example, in
        Figures~\ref{\SETLABEL:LA1} and~\ref{\SETLABEL:LA2}, if the
        non-linear term, $b$, was greater than zero, it was set to
        zero to produce the graphs. See Section~\ref{\SETLABELREF:LAA}
        for the actual derived values. In other cases, the magnitude
        of $b$ was too large, resulting in a graph that was decreasing
        as a function of time}.  The methodology should be regarded as
        ``fragile.'' It is included for completeness.

        \idx{least squares approximation}
        Figure~\ref{\SETLABEL:LA1} is a graph of the logistic function
        for the time series data presented in
        Figure~\ref{\SETLABEL:TS}. The data presented was made by
        running the program {\it tsdlogistic}\/, which is described
        briefly in Appendix~\ref{programs}, on the parameters
        extracted from the time series data as suggested in
        Figure~\ref{\SETLABEL:TF}. The program {\it tslsq}\/ was used
        to derive the constant and the slope of the normalized
        increments of the data presented in Figure~\ref{\SETLABEL:TF}.
        Figure~\ref{\SETLABEL:LA2} is the same graph, but with the
        time scale expanded by a factor of two.

        \begin{figure}[ht]
            \begin{center}
                \begin{minipage}[t]{0.45\textwidth}
                    \epsfxsize=1.0\linewidth
                    \epsffile{\directory/data.tsfraction.tslsq-p.tsdlogistic.eps}
                    \caption[{\market}, logistic function
                        estimates.]{{\market}, logistic function
                        estimates, provided by running the {\it
                        tslsq}\/ program on the normalized increments
                        presented in Figure~\ref{\SETLABEL:TF} with
                        the -p option. These parameters were used as
                        arguments to the {\it tsdlogistic}\/ program.}
                    \label{\SETLABEL:LA1}
                    \label{\SETLABELQ:LA1}
                \end{minipage}
                \hfill
                \begin{minipage}[t]{0.45\textwidth}
                    \epsfxsize=1.0\linewidth
                    \epsffile{\directory/data.tsfraction.tslsq-p.tsdlogistic2.eps}
                    \caption[{\market}, logistic function
                        estimates.]{{\market}, logistic function
                        estimates of Figure~\ref{\SETLABEL:LA1} with
                        the time scale expanded by a factor of two.}
                    \label{\SETLABEL:LA2}
                    \label{\SETLABELQ:LA2}
                \end{minipage}
            \end{center}
        \end{figure}

% Local Variables:
% TeX-parse-self: t
% TeX-auto-save: t
% TeX-master: "fractal.tex"
% End:


        %
% -----------------------------------------------------------------------------
%
% A license is hereby granted to reproduce this software source code and
% to create executable versions from this source code for personal,
% non-commercial use.  The copyright notice included with the software
% must be maintained in all copies produced.
%
% THIS PROGRAM IS PROVIDED "AS IS". THE AUTHOR PROVIDES NO WARRANTIES
% WHATSOEVER, EXPRESSED OR IMPLIED, INCLUDING WARRANTIES OF
% MERCHANTABILITY, TITLE, OR FITNESS FOR ANY PARTICULAR PURPOSE.  THE
% AUTHOR DOES NOT WARRANT THAT USE OF THIS PROGRAM DOES NOT INFRINGE THE
% INTELLECTUAL PROPERTY RIGHTS OF ANY THIRD PARTY IN ANY COUNTRY.
%
% Copyright (c) 1994-2006, John Conover, All Rights Reserved.
%
% Comments and/or bug reports should be addressed to:
%
%     john@email.johncon.com (John Conover)
%
% -----------------------------------------------------------------------------
%
% Revision: \RCSRevision \\
% Revision Time: \RCSTime UMT \\
% Revision Date: \RCSDate \\
% Revision Id: \RCSId \\
% Revision File: \RCSLog \\
\RCS $Revision: 0.0 $
\RCS $Date: 2006/01/20 04:38:13 $
\RCS $Id: hurst.tex,v 0.0 2006/01/20 04:38:13 john Exp $
% $Log: hurst.tex,v $
% Revision 0.0  2006/01/20 04:38:13  john
% Initial version
%
%
    \subsection{Hurst Coefficient Analysis}
        \label{\SETLABEL:H}

        \subidx{\market}{Hurst coefficient analysis}
        \subidx{Hurst coefficient}{analysis}
        \subidx{increments}{normalized}
        \subidx{normalized}{increments}
        \subidx{programs}{tshurst}
        \subidx{tshurst}{program}
        The data in this section is presented in tabular form in
        Section~\ref{\SETLABELREF:HCHP}. Figure~\ref{\SETLABEL:HC} is
        a graph of the Hurst coefficient data time series data shown
        in Figure~\ref{\SETLABEL:TS}. The slope of the graph is the
        Hurst coefficient.  The data for this figure was produced by
        the program {\it tshurst}\/, which is described briefly in
        Appendix~\ref{programs}.

        \subidx{\market}{H parameter analysis}
        \subidx{H parameter}{analysis}
        \subidx{programs}{tshcalc}
        \subidx{tshcalc}{program}
        Figure~\ref{\SETLABEL:HP} is a graph of the H parameter data
        for the normalized increments of the time series data shown in
        Figure~\ref{\SETLABEL:TF}. The data for this figure was
        produced by the program {\it tshcalc}\/, which is described
        briefly in Appendix~\ref{programs}.

        \begin{figure}[ht]
            \begin{center}
                \begin{minipage}[t]{0.45\textwidth}
                    \epsfxsize=1.0\linewidth
                    \epsffile{\directory/data.tshurst.eps}
                    \caption[{\market}, Hurst coefficient data]{{\market},
                        Hurst coefficient data for the normalized
                        increments of the time series data shown in
                        Figure~\ref{\SETLABEL:TF}.  The slope of the graph
                        is the Hurst coefficient.}
                    \label{\SETLABEL:HC}
                \end{minipage}
                \hfill
                \begin{minipage}[t]{0.45\textwidth}
                    \epsfxsize=1.0\linewidth
                    \epsffile{\directory/data.tshcalc.eps}
                    \caption[{\market}, H parameter data]{{\market}, H
                        parameter data for the normalized increments of
                        the time series data shown in
                        Figure~\ref{\SETLABEL:TF} The slope of the graph
                        is the H parameter.}
                    \label{\SETLABEL:HP}
                \end{minipage}
            \end{center}
        \end{figure}

        \subidx{revenue}{See, rate of revenue returns}
        \subidx{returns}{See, rate of revenue returns}
        \subidx{\market}{revenues}
        \subidx{Hurst coefficient}{analysis}
        \subidx{\market}{Hurst coefficient analysis}
        \subidx{\market}{rate of change}
        \subidx{\market}{windows of opportunity}
        \subidx{rate of revenue returns}{forecast}
        \subidx{forecast}{rate of revenue returns}
        \idx{windows of opportunity}
        \subidx{programs}{tslsq}
        \subidx{tslsq}{program}

        The approximately linear slope of the graph in
        Figure~\ref{\SETLABEL:HC} implies that the variance of the
        rate of revenue returns, (per {\timescale},) in the {\market},
        $V(t_2 - t_1)$, over a period of time is proportional to the
        period of time raised to twice the Hurst
        coefficient~\cite[pp. 180]{Feder},~\cite[pp. 246]{Crownover}.
        This seems to be a quantitative statement concerning how fast,
        and to what degree, the rate of revenue returns' state of
        affairs can change over a period of time.  An additional
        implication, for Hurst coefficients sufficiently close to 0.5,
        is that the probability of the state of affairs repeating
        sometime in the future goes down with increasing
        time\footnote{It can be shown that the number of expected
        market ``high'' and ``low'' transitions, $N$, scales with the
        square root of time, or $N \propto \sqrt {t}$, meaning that
        the cumulative distribution of the probability, $P$, of the
        duration of a market's ``high'' or ``low'' exceeding a given
        time interval, $t$, is proportional to the reciprocal of the
        square root of the time interval, $P \propto 1 / \sqrt {t}$,
        (or, conversely, that the probability of the duration of a
        market's ``high'' or ``low'' exceeding a given time interval
        is proportional to the reciprocal of the time interval raised
        to the power $3 / 2$, ie., $P \propto 1 / t^{3 /
        2}$,~\cite[pp. 153]{Schroeder}. What this means is that a
        histogram of the ``zero free'' run-lengths of a market being
        ``high'' or ``low,'' over a long time, would have a $1 / t^{3
        / 2}$ characteristic.)}, $t$, $p(t) = erf (1/\sqrt{2t})$ which
        is approximately $1/\sqrt{t}$ for $t \gg
        1$~\cite[pp. 160]{Schroeder}. Figures~\ref{\SETLABEL:FN},
        and,~\ref{\SETLABEL:FF} compare methods of approximation of
        the ``forecastability'' of the rate of revenue returns in the
        {\market} for the near term and far term,
        respectively~\cite[pp. 83-84]{Peters:CAOITCM}\footnote{The
        author is not comfortable with Peters' interpretation. For
        example, if the algorithm explained
        in~\cite[pp. 82]{Peters:CAOITCM} is used on ``white noise''
        which, by definition, never has any correlations, the short
        term Hurst coefficient, and thus the ``forecastability,'' is
        still near unity---a bit of an enigma. This can be verified
        with the {\it tswhite}\/ and {\it tshurst}\/ programs, which
        are briefly described in Appendix~\ref{programs}.}.  This
        seems to be a quantitative statement concerning ``windows of
        opportunity'' in the rate of revenue returns, (per
        {\timescale}.)  The program {\it tslsq}\/ was used on the
        Hurst coefficient data, presented in
        Figure~\ref{\SETLABEL:HC}, to provide a least squares
        approximation to the Hurst coefficient. The superimposed least
        squares approximation with on original Hurst coefficient data
        is presented.  The time series data has a Hurst coefficient of
        {\thurstlow}, so that:

        \subidx{\market}{Hurst coefficient analysis}
        \begin{eqnarray}
            V\left(t_2 - t_1\right) & \propto & \left(t_2 - t_1\right)^{2 \cdot H}\\
            V\left(t_2 - t_1\right) & \propto & \left(t_2 - t_1\right)^{2 \cdot {\thurstlow}}\\
                                    & \propto & \left(t_2 - t_1\right)^{\thurstlowtwo}
            \label{\SETLABEL:V}
        \end{eqnarray}

        \subidx{fractional}{Brownian motion}
        \subidx{Brownian motion}{fractional}
        \idx{fractal}
        \noindent where $V(t_2 - t_1)$ is the variance of the
        increments of the rate of revenue returns, (per {\timescale},)
        over the time interval $t_2 -
        t_1$,~\cite[pp. 177]{Feder},~\cite[pp. 494]{Peitgen}. If $H >
        \frac{1}{2}$, then the time series is termed as being
        characterized by ``fractional Brownian
        motion~\cite[pp. 170]{Feder}.''

        \subidx{rate of revenue returns}{predictability}
        \subidx{rate of revenue returns}{forecastability}
        \subidx{rate of revenue returns}{consistency}
        \subidx{predictability}{rate of revenue returns}
        \subidx{forecastability}{rate of revenue returns}
        \subidx{consistency}{rate of revenue returns}
        \subidx{\market}{rate of revenue returns, predictability}
        \subidx{\market}{rate of revenue returns, forecastability}
        \subidx{\market}{rate of revenue returns, consistency}
        \subidx{Hurst coefficient}{analysis}
        \subidx{\market}{Hurst coefficient analysis}
        \subidx{\market}{rate of change}

        In some sense, the Hurst coefficient is a quantitative
        expression of the ``forecastability'' of the future based on
        the past\footnote{Actually, in general, when summing fractal
        entities, the method used should be a root mean square
        process, dependent on the Hurst Coefficient, $H$, where
        $P_{total}^H = P_1^H + P_2^H + \cdots$, where $P_n$ is the
        fractal entities. For a Brownian motion, or random walk type
        of fractal the Hurst Coefficient is a function of time into
        the future. For the ``near term,'' the Hurst coefficient is
        very near unity, meaning the summation process is linear. For
        the ``long term,'' $H \approx 0.5$, or a standard root mean
        square summation process should be used. If $H$ is $0.5$ then
        the market is termed a Brownian motion, or random walk
        process. If it is larger than 0.5, it is termed fractional
        Brownian motion process. For a random walk process, ``near
        term'' and ``far term'' are quantitatively differentiated on
        the Hurst Coefficient graph where $1 - \ln (t) = 0.5 \cdot \ln
        (t)$, or when $\ln (t) = 2$, or $t = 7.389\ldots$ See
        Section~\ref{\SETLABEL:FS} for the particulars on using Hurst
        Coefficient to sum fractal process' for the {\market}. See
        also~\cite[pp. 67, 83-84]{Peters:CAOITCM} and~\cite[pp. 129,
        159]{Schroeder} for particulars on the implications of the
        Hurst Coefficient and root mean square summation issues.}.  A
        Hurst coefficient of {\thurstlow}, (for the near future, and
        {\thurstall} for the distant future.) implies that the
        likelihood of the rate of revenue returns, (per {\timescale},)
        for any two consecutive {\timescale}s being the same is
        {\thurstlowhundred}\%~\cite[pp. 66]{Peters:CAOITCM} for the
        near future, and {\thurstall} for the distant
        future. Likewise, there is a {\thurstlowhundred}\% chance of
        the rate of revenue returns, (per {\timescale},) movements
        being the same in consecutive time periods---ie., if, in a
        given {\timescale}, the rate of revenue returns, (per
        {\timescale},) is increasing, there is a {\thurstlowhundred}\%
        that the rate of revenue returns, (per {\timescale},) will
        increase in the following period, also. In some sense, this is
        a quantitative statement on how ``predictable,'' or
        ``forecastable'' the rate of revenue returns, (per
        {\timescale},) for the {\market} are over time, since the
        probability of having $n$ many consecutive {\timescale}s of
        the same agenda is $H^n$ where $H$ is the Hurst coefficient,
        or, letting the short term probability of having $n$ many
        {\timescale}s of the same market agenda, $p_a$, is:

        \begin{eqnarray}
            p_a\left(n\right) & = & H^{n}\\
                              & = & {\thurstlow}^{n}
            \label{\SETLABEL:MA}
        \end{eqnarray}

        \subidx{rate of revenue returns}{predictability}
        \subidx{rate of revenue returns}{forecastability}
        \subidx{rate of revenue returns}{consistency}
        \subidx{predictability}{rate of revenue returns}
        \subidx{forecastability}{rate of revenue returns}
        \subidx{consistency}{rate of revenue returns}
        As an interesting interpretation of the normalized increments
        of the time series data presented in
        Figure~\ref{\SETLABEL:TF}, if the vertical axis is multiplied
        by 100, to convert to percent, then the graph represents the
        error, in percent, that would be made by forecasting, month by
        month, that the next {\timescale}'s rate of revenue returns
        would be the same as the current {\timescale}'s revenue
        rate. Interestingly, it is $\datafractionmean \cdot 100$
        percent, on the average, with a standard deviation of
        $\datafractionstddev \cdot 100$ percent, and a root mean
        square error value of $\datafractionrms \cdot 100$
        percent---small values for such a simple forecasting
        mechanism.

        \subidx{\market}{rate of revenue returns, range}
        \subidx{Hurst coefficient}{analysis}
        \subidx{\market}{Hurst coefficient analysis}
        \subidx{\market}{rate of change}

        This is, essentially, a statement of the range of values, in
        the increments of the rate of revenue returns, (per
        {\timescale},) that is to be expected over the time interval,
        $t_2 - t_1$,
        $R_v$,~\cite[pp. 178]{Feder},~\cite[pp. 172]{Cambel}:

        \begin{eqnarray}
            R_v\left(t_2 - t_1\right) & \propto & \left(t_2 - t_1\right)^{H}\\
                                      & \propto & \left(t_2 - t_1\right)^{\thurstlow}
            \label{\SETLABEL:R}
        \end{eqnarray}

        \subidx{\market}{rate of revenue returns, range}
        \subidx{Hurst coefficient}{analysis}
        \subidx{\market}{Hurst coefficient analysis}
        \subidx{\market}{rate of change}
        \subidx{Markov}{statistics}
        \subidx{statistics}{Markov}
        \noindent where $R$ is the range of values in the increments
        of the rate of revenue returns, (per {\timescale}.) A Hurst
        coefficient, $H$, that is much larger than $\frac{1}{2}$, (but
        less than 1,) implies a strongly non-Gaussian distribution in
        the increments of the rate of revenue returns, (per
        {\timescale},)~\cite[pp. 152, 194]{Feder}, and a Hurst
        coefficient near $\frac{1}{2}$ implies that the increments of
        the rate of revenue returns, (per {\timescale}) is
        characteristic of an independent
        process~\cite[pp. 195]{Feder}. Extreme caution should be
        exercised in using Markov statistics in any analysis where the
        Hurst coefficient is not
        $\frac{1}{2}$,~\cite[pp. 124]{Crownover},~\cite[pp. 106]{Peters:CAOITCM}.


        As a useful approximation, if $H$, is approximately
        $\frac{1}{2}$, Equation~\ref{\SETLABEL:R} reduces
        to,~\cite[pp. 129]{Schroeder}:

        \begin{eqnarray}
            R\left(t_2 - t_1\right) & \propto & (t_2 - t_1)^{\frac{1}{2}}\\
                                    & \propto & \sqrt{\left(t_2 - t_1\right)}
        \end{eqnarray}

        \subidx{\market}{rate of revenue returns, range}
        \subidx{\market}{rate of revenue returns, increase and decrease}
        \subidx{Hurst coefficient}{analysis}
        \subidx{\market}{Hurst coefficient analysis}
        \subidx{\market}{rate of change}
        \subidx{Markov}{statistics}
        \subidx{statistics}{Markov}

        In the case where the Hurst coefficient, $H$, is
        $\frac{1}{2}$, the range of values in the increments of the
        rate of revenue returns, (per {\timescale},) divided by the
        standard deviation of these values, $S$, can be anticipated to
        increase over time according to the following
        relation,~\cite[pp. 154]{Feder},~\cite[pp. 129]{Schroeder}:

        \begin{equation}
            \frac{R\left(t_2 - t_1\right)}{S} \propto \left(t_2 - t_1\right)^{\frac{1}{2}}
        \end{equation}

        \subidx{\market}{rate of revenue returns, range}
        \subidx{\market}{rate of revenue returns, increase and decrease}
        \subidx{Hurst coefficient}{analysis}
        \subidx{\market}{Hurst coefficient analysis}
        \subidx{\market}{rate of change}
        \noindent which is a useful conceptual approximation, since it
        involves only the square root function---if the range and the
        standard deviation of the increments of the rate of revenue
        returns, (per {\timescale},) are known, (and $H \approx
        \frac{1}{2}$,) then the expected change in $\frac{R}{S}$, will
        increase with the square root of time\footnote{To be precise,
        it is actually asymptotically proportional to
        $\tau^{\frac{1}{2}}$}.

        Another useful approximation when rescaling processes that are
        characterize by Brownian motion, (ie., when $H \approx
        \frac{1}{2}$,) is that:

        \begin{eqnarray}
            X\left(t\right) & \propto & \frac{X\left(rt\right)}{r^{H}}\\
                            & \propto & \frac{X\left(rt\right)}{r^{\thurstlow}}
        \end{eqnarray}

        \idx{Brownian motion}
        \idx{fractal}
        Where $X(t)$ is the process characterized by Brownian motion,
        and $r$ is a scaling factor,~\cite[pp. 494]{Peitgen}.

        \subidx{programs}{tslsq}
        \subidx{tslsq}{program}
        The program {\it tslsq}\/ was used on the H parameter data,
        presented in Figure~\ref{\SETLABEL:HP}, to provide a least
        squares approximation to the H parameter for the
        {\market}. The superimposed least squares approximation on the
        original H parameter data is presented.  By contrast, the H
        parameter, as derived by the methodology outlined
        in~\cite[pp. 249]{Crownover}, is {\thcalclow} for the near
        future, and {\thcalcall} for the distant future.

        \subidx{\market}{Hurst coefficient analysis}
        \subidx{Hurst coefficient}{analysis}
        \subidx{increments}{normalized}
        \subidx{normalized}{increments}
        \subidx{programs}{tshurst}
        \subidx{tshurst}{program}
        \subidx{\market}{H parameter analysis}
        \subidx{H parameter}{analysis}
        \subidx{programs}{tshcalc}
        \subidx{tshcalc}{program}
        Figures~\ref{\SETLABEL:HC} and~\ref{\SETLABEL:HP} represent
        Hurst coefficient and H parameter data that are derived from
        the normalized increments, shown in
        Figure~\ref{\SETLABEL:TF}. In this case, the data is
        considered a normalized derivative of the time series data
        presented in Figure~\ref{\SETLABEL:TF}, instead of a
        cumulative sum.  The program, {\it tshurst}\/, is described
        briefly in appendix~\ref{programs}, and the data for
        figures~\ref{\SETLABEL:THC} and~\ref{\SETLABEL:THP} was made
        using the -d option.

        \begin{figure}[ht]
            \begin{center}
                \begin{minipage}[t]{0.45\textwidth}
                    \epsfxsize=1.0\linewidth
                    \epsffile{\directory/data.tsfraction.tshurst-d.eps}
                    \caption[{\market}, traditional Hurst coefficient
                        data]{{\market}, traditional Hurst coefficient
                        data for the time series data shown in
                        Figure~\ref{\SETLABEL:TS}.  The slope of the
                        graph is the Hurst coefficient, and is
                        {\hurstlow} for the near term, and
                        {\hurstall} for the far term.}
                    \label{\SETLABEL:THC}
                \end{minipage}
                \hfill
                \begin{minipage}[t]{0.45\textwidth}
                    \epsfxsize=1.0\linewidth
                    \epsffile{\directory/data.tsfraction.tshcalc-d.eps}
                    \caption[{\market}, traditional H parameter
                        data]{{\market}, traditional H parameter data
                        for the time series data shown in
                        Figure~\ref{\SETLABEL:TS} The slope of the
                        graph is the H parameter, and is {\hcalclow}
                        for the near term, and {\hcalcall} for the
                        far term.}
                    \label{\SETLABEL:THP}
                \end{minipage}
            \end{center}
        \end{figure}

% Local Variables:
% TeX-parse-self: t
% TeX-auto-save: t
% TeX-master: "fractal.tex"
% End:


        %
% -----------------------------------------------------------------------------
%
% A license is hereby granted to reproduce this software source code and
% to create executable versions from this source code for personal,
% non-commercial use.  The copyright notice included with the software
% must be maintained in all copies produced.
%
% THIS PROGRAM IS PROVIDED "AS IS". THE AUTHOR PROVIDES NO WARRANTIES
% WHATSOEVER, EXPRESSED OR IMPLIED, INCLUDING WARRANTIES OF
% MERCHANTABILITY, TITLE, OR FITNESS FOR ANY PARTICULAR PURPOSE.  THE
% AUTHOR DOES NOT WARRANT THAT USE OF THIS PROGRAM DOES NOT INFRINGE THE
% INTELLECTUAL PROPERTY RIGHTS OF ANY THIRD PARTY IN ANY COUNTRY.
%
% Copyright (c) 1994-2006, John Conover, All Rights Reserved.
%
% Comments and/or bug reports should be addressed to:
%
%     john@email.johncon.com (John Conover)
%
% -----------------------------------------------------------------------------
%
% Revision: \RCSRevision \\
% Revision Time: \RCSTime UMT \\
% Revision Date: \RCSDate \\
% Revision Id: \RCSId \\
% Revision File: \RCSLog \\
\RCS $Revision: 0.0 $
\RCS $Date: 2006/01/20 04:38:13 $
\RCS $Id: fiscal.tex,v 0.0 2006/01/20 04:38:13 john Exp $
% $Log: fiscal.tex,v $
% Revision 0.0  2006/01/20 04:38:13  john
% Initial version
%
%
    \subsection{Fixed Increment Approximation for Fiscal Strategy}
        \label{\SETLABEL:FS}

        \subidx{\market}{fiscal strategy}
        \subidx{markets}{analysis}
        \subidx{analysis}{markets}
        \subidx{strategy}{fiscal}
        \subidx{fiscal}{strategy}
        The data in this section is presented in tabular form in
        Section~\ref{\SETLABELREF:LR}. This section derives various
        values based on the ``average'' of the normalized increments
        presented in Figure~\ref{\SETLABEL:TFA}. These values are an
        approximation to a, probably, complex process with a
        distribution shown in Figure~\ref{\SETLABEL:TF}. These values
        will be used in a fixed increment Brownian fractal analysis
        and simulation of the {\market}, and may, or may not, provide
        adequate accuracy for projections.

        For an organization operating in the {\market}, the fiscal
        strategy, commensurate with the aggregate environment, can be
        derived as follows~\cite[pp. 128, pp
        151]{Schroeder},~\cite[pp. 450]{Reza},~\cite[pp. 270]{Pierce}:
        \vspace{0.15in}

        \subsubsection{Logarithmic Returns}
            \label{\SETLABEL:LR}

            \subidx{logarithmic}{returns}
            \subidx{returns}{logarithmic}
            \subidx{\market}{logarithmic returns}
            The logarithmic returns can be calculated by various
            means. Four will be presented here, for comparison.

            \subidx{programs}{tsnormal}
            \subidx{tsnormal}{program}
            \subidx{logarithmic}{returns}
            \subidx{returns}{logarithmic}
            The logarithmic returns, in bits, $bits$, as computed from
            the mean, by the program {\it tsnormal}\/, which is
            described in Chapter~\ref{programs}, and is presented in
            Figure~\ref{\SETLABEL:TF}, and Equation~\ref{abits} from
            Section~\ref{ereturns} in Chapter~\ref{general}:

            \begin{equation}
                bits = \frac{\ln \left({\datafractionmean} + 1\right)}{\ln \left(2\right)} = \datafractionmeanbits
            \end{equation}

            \subidx{programs}{tslsq}
            \subidx{tslsq}{program}
            \subidx{logarithmic}{returns}
            \subidx{returns}{logarithmic}
            \noindent By comparison, the logarithmic returns, in bits,
            $bits$, as computed from the constant in the least squares
            approximation, using the program {\it tslsq}\/, which is briefly
            described in Chapter~\ref{programs}, as presented in
            Figure~\ref{\SETLABEL:TF}, and Equation~\ref{abits} from
            Section~\ref{ereturns} in Chapter~\ref{general}:

            \begin{equation}
                bits = \frac{\ln \left({\datafractionconstant} + 1\right)}{\ln \left(2\right)} = \datafractionconstantbits
            \end{equation}

            Note that if the mean is not constant in
            Figure~\ref{\SETLABEL:TF}, this method will not provide
            accurate results.

            \subidx{programs}{tslsq}
            \subidx{tslsq}{program}
            \subidx{logarithmic}{returns}
            \subidx{returns}{logarithmic}
            \noindent And by yet another comparison, using the program
            {\it tslsq}\/, which is briefly described in
            Chapter~\ref{programs}, with the -e -p options, to provide
            a formula for the least squares exponential fit to the
            time series data set presented in
            Figure~\ref{\SETLABEL:TS}:

            \begin{equation}
                bits = {\datatslsqepbits}
            \end{equation}

            \subidx{programs}{tslogreturns}
            \subidx{tslogreturns}{program}
            \subidx{logarithmic}{returns}
            \subidx{returns}{logarithmic}
            \noindent And finally, by comparison, from the
            {\it tslogreturns}\/ program, which is briefly described
            in Chapter~\ref{programs}, with the -p option, to provide
            a formula for the logarithmic returns of the time series
            data set presented in Figure~\ref{\SETLABEL:TS}:

            \begin{equation}
                bits = {\logreturns}
            \end{equation}

        \subsubsection{Calculation of Shannon Probability}
            \label{\SETLABEL:SP}

            \subidx{\market}{Shannon probability}
            Ideally, all of the values presented in
            Section~\ref{\SETLABEL:LR} would be equal. Using the
            logarithmic returns provided by the {\it tslogreturns}\/
            program, to be consistent
            with~\cite[pp. 81]{Peters:CAOITCM}

            \subidx{programs}{tslogreturns}
            \subidx{tslogreturns}{program}
            \begin{equation}
                2^{{\logreturns}t}
            \end{equation}

            \noindent therefore:
            \begin{equation}
                C\left(p\right) = {\logreturns}
            \end{equation}
            \subidx{programs}{tsshannon}
            \subidx{tsshannon}{program}
            \subidx{Shannon}{probability}
            \subidx{probability}{Shannon}
            \noindent and, {\it tsshannon}\/ {\logreturns} gives:
            \begin{equation}
                \label{\SETLABEL:F0}
                C\left({\shannonlogreturns}\right) = {\logreturns}
            \end{equation}
            \noindent therefore:
            \begin{eqnarray}
                2^{C\left({\shannonlogreturns}\right)} & = & 2^{\logreturns}\\
                                                       & = & {\twologreturns}\\
                                                       & = & {\twologreturnshundred}\%
            \end{eqnarray}
            \noindent and:
            \begin{eqnarray}
                2p - 1 & = & \left(2 \cdot {\shannonlogreturns}\right) - 1\\
                       & = & {\twopone}\\
                       \label{\SETLABEL:F1}
                       & = & {\twoponehundred}\%
            \end{eqnarray}

            \subidx{\market}{fiscal strategy}
            \subidx{markets}{analysis}
            \subidx{analysis}{markets}
            \subidx{strategy}{fiscal}
            \subidx{fiscal}{strategy}
            \subidx{\market}{fiscal strategy}
            \subidx{\market}{growth rate}
            Presuming the simplified assumptions outlined in
            Section~\ref{assumptions}, the ``typical'' organization
            operating in the {\market} executes a long term fiscal
            strategy, commensurate with the aggregate environment,
            that is to invest, every {\timescale}, in sufficient
            additional resources and infrastructure, to increase the
            manufacturing of goods and services by {\twoponehundred}\%
            of its rate of revenue returns, (per {\timescale}.) As a
            conceptual model, the remaining {\hundredtwoponehundred}\%
            will be held in ``reserve'' with a
            {\shannonlogreturnshundred}\% chance of making twice the
            {\twoponehundred}\% back, (and a
            {\hundredshannonlogreturnshundred}\% chance of making
            0.0,) in one {\timescale}, on the average, for an average
            growth in its rate of revenue returns, (per {\timescale},)
            of {\twologreturnshundred}\%, or a doubling of its rate of
            revenue returns, (per {\timescale},) in
            {\oneoverlogreturns} {\timescale}s.

        \subsubsection{Example Fixed Increment Approximation Fiscal Strategies}

            \subidx{\market}{fiscal strategy}
            \subidx{markets}{analysis}
            \subidx{analysis}{markets}
            \subidx{strategy}{fiscal}
            \subidx{fiscal}{strategy}
            \subidx{\market}{fiscal strategy}
            \subidx{\market}{growth rate}
            \subidx{\market}{management metric}
            \idx{management metric}
            A possible metric on the effectiveness of long term fiscal
            management could possibly be that if an investment of
            {\twoponehundred}\% per {\timescale} of the rate of
            revenue returns, (per {\timescale},) is made in resources
            and infrastructure, then the rate of revenue returns would
            be expected to increase by {\twologreturnshundred}\%, per
            {\timescale}, on average.

            Note that the metrics presented in this section are
            representative of the {\market} as an aggregate whole, and
            may or may not be accurate representations for any
            particular participant in the environment. Of interest to
            the participants in the environment would be a similar
            analysis of each product or service rendered in the
            marketplace.

            \subidx{\market}{fiscal strategy}
            \subidx{markets}{analysis}
            \subidx{analysis}{markets}
            \subidx{strategy}{fiscal}
            \subidx{fiscal}{strategy}
            \subidx{\market}{fiscal strategy}
            As a simple illustrative example, a company operating in
            this environment might obtain a credit line from a bank
            that is equal to {\twoponehundred}\% of its rate of
            revenue returns, (per {\timescale},) to finance additional
            operations. In this simple scenario, the company would use
            its revenue base as collateral for the loan. Some
            {\timescale}s, depending on the {\market}'s environment,
            the company's rate of revenue returns exceeds what was
            borrowed from the bank, and the loan is repaid in
            full. Other {\timescale}s, the company must default, and
            the bank seizes a portion of the company's revenue base to
            pay the delinquent loan. However, on the average, the
            company will expand its rate of revenue returns at
            {\twologreturnshundred}\% per {\timescale}.

            \subidx{\market}{fiscal strategy}
            \subidx{markets}{analysis}
            \subidx{analysis}{markets}
            \subidx{strategy}{fiscal}
            \subidx{fiscal}{strategy}
            \subidx{\market}{fiscal strategy}
            As another simple example, a company re-invests
            {\twoponehundred}\% of its rate of revenue returns, (per
            {\timescale},) in development, marketing, sales, and
            distribution of new products.  Although some products will
            be successful and the return on the investment will exceed
            the {\twoponehundred}\% per {\timescale} investment,
            others will not. However, on the average, the company will
            expand it gross rate of revenue returns at
            {\twologreturnshundred}\% per {\timescale}.

            \subidx{\market}{fiscal strategy}
            \subidx{markets}{analysis}
            \subidx{analysis}{markets}
            \subidx{strategy}{fiscal}
            \subidx{fiscal}{strategy}
            \subidx{\market}{fiscal strategy}
            \subidx{\market}{product portfolio}
            \subidx{\market}{product diversity}
            \subidx{\market}{product mix}
            \subidx{\market}{optimum number of products}
            \idx{product portfolio}
            \idx{product diversity}
            \idx{optimum number of products}
            \idx{product mix}

            As an example of ``product portfolio'' management, suppose
            a company re-invests {\twoponehundred}\% of its rate of
            revenue returns, (per {\timescale},) in development,
            marketing, sales, and distribution of new products.
            Further suppose that the company has two products, and a
            fractal analysis of the individual product rate of revenue
            return time series indicates that one product has a
            Shannon probability of 0.65, and the other has a Shannon
            probability of 0.55. Then the percentage of re-investment
            in the first product would be $(2 \cdot 0.65 - 1) \cdot
            {\twoponehundred}$, percent of the rate of revenue
            returns, and $(2 \cdot 0.55 - 1) \cdot {\twoponehundred}$
            percent for the second product, implying that the company
            should diversify its product line\footnote{The astute
            reader would note that the linear addition was used to add
            the contribution to development of each product. This is a
            ``near term'' interpretation. Actually, in general, the
            method used should be a root mean square process,
            dependent on the Hurst Coefficient, $H$, where
            $P_{total}^H = P_1^H + P_2^H + \cdots$, where $P_n$ is the
            contribution to each individual product. For a Brownian
            motion, or random walk type of fractal the Hurst
            Coefficient is a function of time into the future. For the
            ``near term,'' the Hurst coefficient is very near unity,
            meaning the summation process is linear. For the ``long
            term,'' $H \approx 0.5$, or a standard root mean square
            summation process should be used. If $H$ is $0.5$ then the
            market is termed a Brownian motion, or random walk
            process. If it is larger than 0.5, it is termed fractional
            Brownian motion process. For a random walk process, ``near
            term'' and ``far term'' are quantitatively differentiated
            on the Hurst Coefficient graph where $1 - \ln (t) = 0.5
            \cdot \ln (t)$, or when $\ln (t) = 2$, or $t =
            7.389\ldots$ See~\cite[pp. 67, 83-84]{Peters:CAOITCM}
            and~\cite[pp. 129, 159]{Schroeder} for particulars on the
            implications of the Hurst Coefficient and root mean square
            summation issues.}.  Note that this is a ``bet hedging''
            metric methodology, and assumes that the products have
            uncorrelated revenue return rates. If this re-investment
            methodology is not feasible, perhaps for strategic
            financial reasons, then the re-investment in both products
            should total the ${\twoponehundred}$\%, and the investment
            in each product should be made at a ratio of $\frac{(2
            \cdot 0.65 - 1)}{(2 \cdot 0.55 - 1)} = 3 : 1$,
            respectively. Note that this ``bet hedging'' can be used
            to define the optimal number of products that can be
            supported on the rate of revenue returns. If it assumed
            that all products are ``typical'' for the {\market}, as a
            standard bench mark, then the optimal number will be
            $\frac{1}{{\twopone}}$. Note that this is a
            ``theoretical'' value, since not all products are
            ``typical,'' and there may be strategic reasons, for
            example product leveraging, that may increase the number
            of products above the optimum. However, most of the
            revenue should come from the optimal number of products,
            since having more products will decrease the amount of the
            potential investment in each product, and having less than
            the optimum number of products will increase the risk that
            many of the products could suffer a ``down market''
            concurrently, impacting the rate of revenue returns.  As
            another interesting interpretation of the optimal
            ``hedging of bets,'' in product portfolio strategy, and
            considering the graph of the normalized increments
            presented in Figure~\ref{\SETLABEL:TF}, if the
            organization is running optimally, then these products
            will generate, at least in principle, one standard
            deviation, approximately $0.8413 = 84.13$\% of the future
            growth in rate of revenue returns. Naturally, these are
            approximations, and the values are an approximation to a,
            probably, complex process, and appropriate scrutiny should
            be exercised before making specific projections.  As yet
            another example of ``product portfolio'' management,
            consider the issue of product mix. In this interpretation,
            {\twoponehundred}\% of the product manufactured should be
            ``proprietary,'' while the rest is ``industry standard.''
            As yet another possibility, {\twoponehundred}\% of the
            product manufactured should be predatory into new markets,
            and the remainder in markets that are ``traditional'' for
            the company.

% Local Variables:
% TeX-parse-self: t
% TeX-auto-save: t
% TeX-master: "fractal.tex"
% End:


        \subsubsection{Observations on the Fixed Increment Approximation for Fiscal Strategy}

            A re-investment of {\twoponehundred} of the rate of
            revenue returns per {\timescale} does not seem
            inconsistent with the industry averages, since it includes
            investments in research and development, additional
            manufacturing infrastructure, advertising,
            etc. Additionally, a product mix of {\twoponehundred}\%
            ``proprietary'' and the remainder ``industry standard''
            products seems consistent with the industry analyst
            ``20/80'' rule. The value of one standard deviation,
            $84.13$\%, of the revenue return rate being generated by
            $\frac{1}{{\twopone}}$ products seems consistent with the
            industry, also.

        %
% -----------------------------------------------------------------------------
%
% A license is hereby granted to reproduce this software source code and
% to create executable versions from this source code for personal,
% non-commercial use.  The copyright notice included with the software
% must be maintained in all copies produced.
%
% THIS PROGRAM IS PROVIDED "AS IS". THE AUTHOR PROVIDES NO WARRANTIES
% WHATSOEVER, EXPRESSED OR IMPLIED, INCLUDING WARRANTIES OF
% MERCHANTABILITY, TITLE, OR FITNESS FOR ANY PARTICULAR PURPOSE.  THE
% AUTHOR DOES NOT WARRANT THAT USE OF THIS PROGRAM DOES NOT INFRINGE THE
% INTELLECTUAL PROPERTY RIGHTS OF ANY THIRD PARTY IN ANY COUNTRY.
%
% Copyright (c) 1994-2006, John Conover, All Rights Reserved.
%
% Comments and/or bug reports should be addressed to:
%
%     john@email.johncon.com (John Conover)
%
% -----------------------------------------------------------------------------
%
% Revision: \RCSRevision \\
% Revision Time: \RCSTime UMT \\
% Revision Date: \RCSDate \\
% Revision Id: \RCSId \\
% Revision File: \RCSLog \\
\RCS $Revision: 0.0 $
\RCS $Date: 2006/01/20 04:38:13 $
\RCS $Id: companies.tex,v 0.0 2006/01/20 04:38:13 john Exp $
% $Log: companies.tex,v $
% Revision 0.0  2006/01/20 04:38:13  john
% Initial version
%
%
    \subsection{Number of Companies}
        \label{\SETLABEL:QNC}

        \subidx{\market}{number of companies}
        \subidx{number of companies}{analysis}
        \subidx{analysis}{number of companies}
        \subidx{Shannon}{probability}
        \subidx{probability}{Shannon}
        This section evaluates the approximate, or ``average,'' number
        of companies in the {\market}, and uses the method outlined in
        Chapter~\ref{general}, Section~\ref{aftsma}. Since the
        average, $avg_{ind}$, and the root mean square, $rms_{ind}$,
        of the normalized increments of the {\market} time series is
        \datafractionmean, and \datafractionrms respectively, the
        number of companies participating in the market can be
        calculated by Equation~\ref{ncompanies} to be {\ncompanies}.

        If this value seems consistent number of companies in the
        {\market}, within the assumptions outlined in
        Chapter~\ref{general}, Section~\ref{aftsma}, then it would
        seem that there is some circumstantial or indirect evidence
        that the companies participating in the {\market} are
        operating optimally, and the ``average'' Shannon probability,
        $P$ for each participating company would be, using
        Equation~\ref{pncompanies}, {\pncompanies}, which would be the
        value which should be used in Section~\ref{\SETLABEL:FS} for
        each participating company if market expansion was to be
        consistent with the rest of the industry. However, if the
        Shannon probability derived in Section~\ref{\SETLABEL:FS} is
        greater than the average Shannon probability for the companies
        participating in the {\market}, as derived in this section,
        then the market would, possibly, be exploitable with the
        fiscal strategy outlined in Section~\ref{\SETLABEL:FS}. The
        maximum exploitability for the {\market} is derived in
        Section~\ref{\SETLABEL:MAXSHANNON}, but it is probably of
        doubtful practicality.

        Note that these optimizations would maximize a company's
        market growth. Since there are probably many companies
        competing in the market place, this would not necessarily
        maximize a company's P\&L, as described in
        Chapter~\ref{general}, Section~\ref{ompl}. The Shannon
        probability that maximizes market share in the {\market} is
        \pncompanies, with several alternative solutions listed in the
        previous paragraph. However, these should be contrasted to the
        Shannon probability that maximizes a company's P\&L which is
        \avgrms~in the {\market}. In all cases, the fraction of the
        P\&L that should be ``wagered'' on the future, $f$, should be:

        \begin{equation}
            f = 2P - 1
        \end{equation}

        \noindent where $P$ is the particular Shannon probability
        chosen optimize a particular fiscal strategy. Interestingly,
        the measured Shannon probability of the {\market} would tend
        to indicate that the companies participating in the market
        have chosen a fiscal strategy that optimizes market growth, as
        opposed to capital growth.

        \subidx{\market}{increasing returns}
        \subidx{economic increasing returns}{\market}
        As interesting interpretation of these exploitive issues,
        since all three fiscal strategies will result in exponential
        market growth for every company participating in the market,
        is that they may represent, perhaps, an example of
        ``increasing returns.''

% Local Variables:
% TeX-parse-self: t
% TeX-auto-save: t
% TeX-master: "fractal.tex"
% End:


        %
% -----------------------------------------------------------------------------
%
% A license is hereby granted to reproduce this software source code and
% to create executable versions from this source code for personal,
% non-commercial use.  The copyright notice included with the software
% must be maintained in all copies produced.
%
% THIS PROGRAM IS PROVIDED "AS IS". THE AUTHOR PROVIDES NO WARRANTIES
% WHATSOEVER, EXPRESSED OR IMPLIED, INCLUDING WARRANTIES OF
% MERCHANTABILITY, TITLE, OR FITNESS FOR ANY PARTICULAR PURPOSE.  THE
% AUTHOR DOES NOT WARRANT THAT USE OF THIS PROGRAM DOES NOT INFRINGE THE
% INTELLECTUAL PROPERTY RIGHTS OF ANY THIRD PARTY IN ANY COUNTRY.
%
% Copyright (c) 1994-2006, John Conover, All Rights Reserved.
%
% Comments and/or bug reports should be addressed to:
%
%     john@email.johncon.com (John Conover)
%
% -----------------------------------------------------------------------------
%
% Revision: \RCSRevision \\
% Revision Time: \RCSTime UMT \\
% Revision Date: \RCSDate \\
% Revision Id: \RCSId \\
% Revision File: \RCSLog \\
\RCS $Revision: 0.0 $
\RCS $Date: 2006/01/20 04:38:13 $
\RCS $Id: operations.tex,v 0.0 2006/01/20 04:38:13 john Exp $
% $Log: operations.tex,v $
% Revision 0.0  2006/01/20 04:38:13  john
% Initial version
%
%
    \subsection{Fixed Increment Approximation for Operational Strategy}
        \label{\SETLABEL:OPS}.

        This section derives various values based on the ``average''
        of the normalized increments presented in
        Figure~\ref{\SETLABEL:TFA}. These values are an approximation
        to a, probably, complex process with a distribution shown in
        Figure~\ref{\SETLABEL:TF}. These values will be used in a
        fixed increment Brownian fractal analysis and simulation of
        the {\market}, and may, or may not, provide adequate accuracy
        for projections.

        \subidx{\market}{fiscal strategy}
        \subidx{\market}{Shannon probability}
        \subidx{strategy}{fiscal}
        \subidx{fiscal}{strategy}
        \subidx{Shannon}{probability}
        \subidx{probability}{Shannon}
        It should be noted that the analysis of fiscal strategy,
        presented in Section~\ref{\SETLABEL:FS}, is derived from the
        {\market} metrics and may, or may not, be maximally
        optimal. For the optimal fiscal strategy, which may be
        exploitable, see Section~\ref{\SETLABEL:MAXSHANNON}.

        \subidx{strategy}{exploitable}
        \subidx{exploitable}{strategy}
        \subidx{\market}{windows of opportunity}
        \idx{windows of opportunity}
        \subidx{decision}{obsolete}
        \subidx{obsolete}{decision}
        \subidx{decision}{timeliness}
        \subidx{timeliness}{decision}
        \subidx{rate of revenue returns}{forecast}
        \subidx{forecast}{rate of revenue returns}
        An additional exploitable strategy may be time itself.
        Equations~\ref{\SETLABEL:V},~\ref{\SETLABEL:R},
        and,~\ref{\SETLABEL:MA}, are, essentially, metrics on how fast
        a decision, which is based on information concerning the
        current status of the {\market}, becomes obsolete. Obviously,
        how long a decision is expected to remain relevant should be
        addressed as an operational necessity in strategic planning
        and project management. Figures~\ref{\SETLABEL:FN},
        and,~\ref{\SETLABEL:FF} compare methods of approximation of
        the ``forecastability'' of rate of revenue returns in the
        {\market} for the near term and far
        term~\cite[pp. 83-84]{Peters:CAOITCM}, respectively. As a
        general rule, caution must be exercised when making decisions
        that will span a time interval larger than the time interval
        where the ``forecastability'' of rate of revenue returns drops
        below 50\%. Beyond this time interval, the chances increase
        that the competitive and market forces will alter the market
        environment in a possibly detrimental unanticipated
        fashion. Obviously, there is significant advantage in
        ``timeliness'' of development, manufacturing, and distribution
        of products and services that are consistent with this
        temporal agenda. Automation of these processes, if executed
        consistently with this agenda, should be considered a
        competitive advantage.

        \subidx{strategy}{exploitable}
        \subidx{exploitable}{strategy}
        \subidx{rate of revenue returns}{forecast}
        \subidx{forecast}{rate of revenue returns}
        \idx{product life cycle}
        \idx{life cycle, product}
        In some sense, this temporal agenda defines the ``average''
        product or service life cycle in the {\market}. When the
        ``forecastability'' of rate of revenue returns drops below
        50\%, there is an even chance that the rate of revenue returns
        for the product or service will change in a detrimental
        fashion. If it is assumed that a product or service life cycle
        consists of a ramp up, a maintenence interval, and a ramp
        down, then, if all three life cycle intervals are equal, the
        product life cycle will be, approximately, three times the
        time interval where the ``forecastability'' of rate of revenue
        returns drops below 50\%. Although probably not an accurate
        prediction of product or service life cycle, the technique may
        be used as a conceptual approximation to the dynamics of
        ``market windows.\footnote{For example, consider the market
        for table salt. Since it has inelastic supply and demand
        curves, and is a necessary requirement for life, it would be
        expected that the Hurst coefficient would be very near
        unity---ignoring competitive pressures in the market. The
        predictability of the table salt market would, therefore, be
        expected to be relatively good, over time.}''  The conceptual
        approximation will probably predict a ``conservative'' or
        ``pessimistic'' value in relation to actual markets.

        \begin{figure}[ht]
            \begin{center}
                \begin{minipage}[t]{0.45\textwidth}
                    \epsfxsize=1.0\linewidth
                    \epsffile{\directory/datahurstlownear.eps}
                    \caption[{\market}, ``forecastability'' of near
                        term rate of revenue returns]{{\market},
                        ``forecastability'' of near term rate of
                        revenue returns. Although the error function
                        is the most accurate, for the near term,
                        $H^{t} = \thurstlow^{t}$ may be used as a
                        reliable metric of ``forecastability'' of the
                        rate of revenue returns.}
                    \label{\SETLABEL:FN}
                \end{minipage}
                \hfill
                \begin{minipage}[t]{0.45\textwidth}
                    \epsfxsize=1.0\linewidth
                    \epsffile{\directory/datahurstlowfar.eps}
                    \caption[{\market}, ``forecastability'' of far
                        term rate of revenue returns]{{\market},
                        ``forecastability'' of far term rate of
                        revenue returns. Although the error function
                        is the most accurate, for the far term,
                        $\frac{1}{\sqrt{t}}$ may be used as a reliable
                        metric of ``forecastability'' of the rate of
                        revenue returns.}
                    \label{\SETLABEL:FF}
                \end{minipage}
            \end{center}
        \end{figure}

        \idx{operations research}
        As an interesting interpretation of the data presented in
        Figure~\ref{\SETLABEL:FN}, there may be, perhaps, some
        applicability to such operational agendas as inventory
        control. Maintaining too little inventory, obviously, will
        create a situation where the organization can not exploit
        market expansion, and maintaining too much inventory,
        likewise, would over extend the company, creating unnecessary
        losses when the market contracts. The company should maintain
        inventory levels that do not exceed, from
        Equation~\ref{\SETLABEL:MA}, ${\thurstlow}^{n} = 0.5$
        {\timescale}s of operations. Since the optimal amount of
        inventory and, from Equation~\ref{\SETLABEL:V}, the variance
        of change in the rate of revenue returns in the future can be
        calculated, there may, perhaps, be some applicability to a
        forecasting methodology that can be incorporated into other
        areas of operations research, for example the linear algebras
        using simplex methodologies for optimization of manufacturing
        processes. Traditionally, these forecasts are made by the
        sales department, and are subject to various subjective
        biases.

% Local Variables:
% TeX-parse-self: t
% TeX-auto-save: t
% TeX-master: "fractal.tex"
% End:


        \subsubsection{Observations on the Fixed Increment Approximation for Operational Strategy}

            As an interesting interpretation of
            Figure~\ref{\SETLABEL:FF}, and evaluating the
            approximation $\frac{1}{\sqrt{t}}$ at 60 months gives a
            probability that the market will still have the same
            agenda of about $0.12909945$, or about 1 in 8. This is
            commensurate with numbers from the venture
            community\footnote{For example, see ``IEEE Engineering
            Management Review,'' Volume 23 Number 3, Fall 1995,
            pp. 83}. Of course new venture backed companies fail for
            many reasons, but market appropriateness to product
            portfolio 60 months in the future may be a major
            contributor. Additionally, the success rate of development
            projects of 8 month duration, which have a market success
            rate of about 1 in 3, seems consistent with
            $\frac{1}{\sqrt{3}} = 0.353553391$. Naturally, projects
            fail in the market for many reasons, but market
            appropriateness, in a dynamic market environment may be a
            major contributor to failure.

            As mentioned in Section~\ref{\SETLABEL:H},
            Equation~\ref{\SETLABEL:MA}, and the preceeding section,
            approximately 3 times the value where ${\thurstlow}^{n} =
            0.5$ could be interpreted as an approximation to the
            ``average'' product life cycle. This seems consistent with
            the 6 to 12 month life cycles quoted by many industry
            analyst. In addition, maintaining inventory levels that do
            not exceed the anticipated requirements of
            $\frac{\ln{0.5}}{\ln{\thurstlow}}$ many {\timescale}s
            seems consistent with the author's experience in the
            industry.

        %
% -----------------------------------------------------------------------------
%
% A license is hereby granted to reproduce this software source code and
% to create executable versions from this source code for personal,
% non-commercial use.  The copyright notice included with the software
% must be maintained in all copies produced.
%
% THIS PROGRAM IS PROVIDED "AS IS". THE AUTHOR PROVIDES NO WARRANTIES
% WHATSOEVER, EXPRESSED OR IMPLIED, INCLUDING WARRANTIES OF
% MERCHANTABILITY, TITLE, OR FITNESS FOR ANY PARTICULAR PURPOSE.  THE
% AUTHOR DOES NOT WARRANT THAT USE OF THIS PROGRAM DOES NOT INFRINGE THE
% INTELLECTUAL PROPERTY RIGHTS OF ANY THIRD PARTY IN ANY COUNTRY.
%
% Copyright (c) 1994-2006, John Conover, All Rights Reserved.
%
% Comments and/or bug reports should be addressed to:
%
%     john@email.johncon.com (John Conover)
%
% -----------------------------------------------------------------------------
%
% Revision: \RCSRevision \\
% Revision Time: \RCSTime UMT \\
% Revision Date: \RCSDate \\
% Revision Id: \RCSId \\
% Revision File: \RCSLog \\
\RCS $Revision: 0.0 $
\RCS $Date: 2006/01/20 04:38:13 $
\RCS $Id: simulation.tex,v 0.0 2006/01/20 04:38:13 john Exp $
% $Log: simulation.tex,v $
% Revision 0.0  2006/01/20 04:38:13  john
% Initial version
%
%
    \subsection{Simulation of Fixed Increment Approximation for Fiscal Strategy}
        \label{\SETLABEL:TSUNFAIRBROWNIAN}

        \subidx{\market}{market simulation}
        The data in this section is presented in tabular form in
        Section~\ref{\SETLABELREF:SIM}.
        Figure~\ref{\SETLABEL:TSUNFAIRBROWNIAN0} represents a
        constructional simulation of the time series data presented in
        Figure~\ref{\SETLABEL:TS}. The program {\it
        tsunfairbrownian}\/, which is briefly described in
        appendix~\ref{programs}, was used in the reconstruction. The
        reconstructed data is superimposed on the original time series
        data.  The program, {\it tsunfairbrownian}\/, essentially,
        constructs the new time series as a Brownian fractal with
        fixed increments---the value of the fixed increment is derived
        from the root mean square average of the normalized increments
        presented in Figure~\ref{\SETLABEL:TF}. The ``quality'' of
        such a reconstruction should be subject to adequate scepticism
        and scrutiny since, in all probability, the normalized
        increments presented in Figure~\ref{\SETLABEL:TF} represent a
        relatively complex process, that may not be ``modeled'' with
        such a simple methodology.

        As a further comparison of the the constructional simulation
        with the original time series data,
        Figure~\ref{\SETLABEL:TSUNFAIRBROWNIAN1} presents a normalized
        histogram of the normalized increments of the reconstructed
        time series, superimposed on the normalized histogram
        presented in Figure~\ref{\SETLABEL:NH}.

        \subidx{\market}{fiscal strategy, simulation}
        \subidx{markets}{simulation}
        \subidx{simulation}{markets}
        \subidx{strategy}{fiscal, simulation}
        \subidx{fiscal}{strategy, simulation}
        \subidx{programs}{tsunfairbrownian}
        \subidx{tsunfairbrownian}{program}
        \begin{figure}[ht]
            \begin{center}
                \begin{minipage}[t]{0.45\textwidth}
                    \epsfxsize=1.0\linewidth
                    \epsffile{\directory/tsunfairbrownian-f.eps}
                    \caption[{\market}, Time series data, empirical and
                        simulated]{{\market}, Time series data, empirical
                        and simulated, using the program {\it tsunfairbrownian}\/
                        with f = {\datafractionrms}. This data is
                        superimposed on the data presented in
                        Figure~\ref{\SETLABEL:TS}.}
                    \label{\SETLABEL:TSUNFAIRBROWNIAN0}
                \end{minipage}
                \hfill
                \begin{minipage}[t]{0.45\textwidth}
                    \epsfxsize=1.0\linewidth
                    \epsffile{\directory/tsunfairbrownian-f.tsfraction.tsnormal-s30.eps}
                    \caption[{\market}, normalized histogram,
                        empirical and simulated]{{\market}, normalized
                        histogram of the normalized increments of the
                        time series data shown in
                        Figure~\ref{\SETLABEL:TSUNFAIRBROWNIAN0},
                        empirical and simulated.  The empirical data
                        has a mean of {\datafractionmean}, with a
                        standard deviation of {\datafractionstddev}.
                        By comparison, the simulated data has a mean
                        of {\tsunfairbrownianfractionmean} with a
                        standard deviation of
                        {\tsunfairbrownianfractionstddev}. This data
                        is superimposed on the data presented in
                        Figure~\ref{\SETLABEL:NH}. The area under the
                        four curves is identical.}
                    \label{\SETLABEL:TSUNFAIRBROWNIAN1}
                \end{minipage}
            \end{center}
        \end{figure}

% Local Variables:
% TeX-parse-self: t
% TeX-auto-save: t
% TeX-master: "fractal.tex"
% End:


        %
% -----------------------------------------------------------------------------
%
% A license is hereby granted to reproduce this software source code and
% to create executable versions from this source code for personal,
% non-commercial use.  The copyright notice included with the software
% must be maintained in all copies produced.
%
% THIS PROGRAM IS PROVIDED "AS IS". THE AUTHOR PROVIDES NO WARRANTIES
% WHATSOEVER, EXPRESSED OR IMPLIED, INCLUDING WARRANTIES OF
% MERCHANTABILITY, TITLE, OR FITNESS FOR ANY PARTICULAR PURPOSE.  THE
% AUTHOR DOES NOT WARRANT THAT USE OF THIS PROGRAM DOES NOT INFRINGE THE
% INTELLECTUAL PROPERTY RIGHTS OF ANY THIRD PARTY IN ANY COUNTRY.
%
% Copyright (c) 1994-2006, John Conover, All Rights Reserved.
%
% Comments and/or bug reports should be addressed to:
%
%     john@email.johncon.com (John Conover)
%
% -----------------------------------------------------------------------------
%
% Revision: \RCSRevision \\
% Revision Time: \RCSTime UMT \\
% Revision Date: \RCSDate \\
% Revision Id: \RCSId \\
% Revision File: \RCSLog \\
\RCS $Revision: 0.0 $
\RCS $Date: 2006/01/20 04:38:13 $
\RCS $Id: maximum.tex,v 0.0 2006/01/20 04:38:13 john Exp $
% $Log: maximum.tex,v $
% Revision 0.0  2006/01/20 04:38:13  john
% Initial version
%
%
    \subsection{Simulation of Fixed Increment Approximation for Optimally Maximal Fiscal Strategy}
        \label{\SETLABEL:MAXSHANNON}
        \subidx{\market}{fiscal strategy, simulation}
        \subidx{\market}{maximum Shannon probability}
        \subidx{markets}{simulation}
        \subidx{simulation}{markets}
        \subidx{strategy}{optimum fiscal, simulation}
        \subidx{fiscal}{optimum strategy, simulation}
        \subidx{programs}{tsunfairbrownian}
        \subidx{tsunfairbrownian}{program}
        \subidx{Shannon}{probability}
        \subidx{probability}{Shannon}

        \subidx{strategy}{exploitable}
        \subidx{exploitable}{strategy}
        \subidx{programs}{tsshannonmax}
        \subidx{tsshannonmax}{program}
        \subidx{programs}{tsunfairbrownian}
        \subidx{tsunfairbrownian}{program}
        \subidx{strategy}{fiscal}
        \subidx{fiscal}{strategy}
        The data in this section is presented in tabular form in
        Section~\ref{\SETLABELREF:MAXSHANNON}. One of the issues of
        analysis, as mentioned in Section~\ref{\SETLABEL:OPS}, is to
        determine the maximum Shannon probability for the time series
        presented in Figure~\ref{\SETLABEL:TS}. Potentially, this
        could be exploited with an aggressive fiscal
        strategy. Figure~\ref{\SETLABEL:SHANNONMAX0} is a graph of the
        output of the {\it tsshannonmax}\/ program, which is described
        briefly in appendix~\ref{programs}. The maximum of this
        function is the maximum Shannon probability for the time
        series data presented in Figure~\ref{\SETLABEL:TS}.
        Figure~\ref{\SETLABEL:SHANNONMAX1} was constructed using {\it
        tsunfairbrownian}\/ program, which is also described in
        appendix~\ref{programs}, with the maximum Shannon probability,
        and the time series data presented in
        Figure~\ref{\SETLABEL:TS}. This represents a ``what if'' the
        investment strategy was changed from a Shannon probability of
        {\shannonlogreturns}, as derived in Section~\ref{\SETLABEL:SP}
        to {\shannonmax}. This process, essentially, extracts the
        random statistical data from the time series presented in
        Figure~\ref{\SETLABEL:TS}, and constructs a new time series,
        using the random statistical data, with a different investment
        strategy.  The program, {\it tsunfairbrownian}\/, essentially,
        constructs the new time series as a Brownian fractal with
        fixed increments.  The ``quality'' of such a reconstruction
        should be subject to adequate scepticism and scrutiny since,
        in all probability, the increments in the original data
        represent a relatively complex process, that may not be
        ``modeled'' with such a simple methodology.

        \begin{figure}[ht]
            \begin{center}
                \begin{minipage}[t]{0.45\textwidth}
                    \epsfxsize=1.0\linewidth
                    \epsffile{\directory/data.tsshannonmax.eps}
                    \caption[{\market}, maximum rate of revenue
                        returns] {{\market}, maximum rate of revenue
                        returns, per {\timescale}, vs. Shannon
                        probability. The maximum rate of revenue
                        returns, per {\timescale}, occurs at a Shannon
                        probability of {\shannonmax}.}
                    \label{\SETLABEL:SHANNONMAX0}
                \end{minipage}
                \hfill
                \begin{minipage}[t]{0.45\textwidth}
                    \epsfxsize=1.0\linewidth
                    \epsffile{\directory/data.tsshannonmax-p.tsunfairbrownian-p.eps}
                    \caption[{\market}, maximum rate of revenue
                        returns] {{\market}, maximum rate of revenue
                        returns, per {\timescale}, at a Shannon
                        probability, of {\shannonmax}, corresponding
                        to a ``wager'' fraction of {\twoponemax}.}
                    \label{\SETLABEL:SHANNONMAX1}
                \end{minipage}
            \end{center}
        \end{figure}

        \subidx{fractional}{Brownian motion}
        \subidx{Brownian motion}{fractional}
        \subidx{Shannon}{probability}
        \subidx{probability}{Shannon}
        \subidx{programs}{tsshannonmax}
        \subidx{tsshannonmax}{program}
        If it is assumed that the time series data set, presented in
        Figure~\ref{\SETLABEL:TS}, constitutes classical Brownian
        motion, then the Shannon probability can be calculated by
        counting the total number of {\timescale}s that the {\market}
        movement was positive, and dividing by the total number of
        {timescale}s represented in the time series. This quotient is
        {\pmax}, as compared with the predicted value from the program
        {\it tsshannonmax}\/ of {\shannonmax}.

% Local Variables:
% TeX-parse-self: t
% TeX-auto-save: t
% TeX-master: "fractal.tex"
% End:


        \subsubsection{Observations on the Simulation of Fixed Increment Approximation for Optimally Maximal Fiscal Strategy}

            Note that these simulations are base on a very, perhaps
            overly, simplified model. For example, from
            Section~\ref{\SETLABEL:TSA}, Figure~\ref{\SETLABEL:NH}, it
            would appear that the {\market}'s normalized increments
            are characterized by fractional Brownian motion---but the
            simulations used classical Brownian motion as the
            model. One consequence of this is that a re-investment
            strategy that is to ``wager'' a fraction of {\twoponemax}
            of the rate of returns every {\timescale} is overly
            aggressive, since in the classical Brownian scenario, the
            maximum loss, in any {\timescale}, was no more that what
            was ``wagered.'' However, in the fractional Brownian
            scenario, much more can be lost. From
            Equation~\ref{fopt2},

            \begin{equation}
                \frac{avg}{rms^2} = \frac{f_{opt}}{rms} = K
            \end{equation}

            \noindent where, under the optimum classical Brownian
            scenario, $K$ is unity, or $avg = rms^2$. Notice that,
            since $f = rms$, whether the scenario is optimal or not,
            that the operational ``wager'' fraction, from
            Figure~\ref{\SETLABEL:TF} of {\datafractionrms}, vs.\ an
            ``theoretical optimal'' value of {\twoponemax} seems
            overly conservative. Additionally, notice that, at least
            in principle, the chance of failure in the fractional
            Brownian scenario, which is more accurate, would
            correspond to 1 standard deviation, or about 15.865\% per
            {\timescale}, which is unacceptably high. However, it is
            not clear why the {\market} is running at a value of
            {\datafractionrms}, which seems very
            conservative. However, a re-investment strategy of
            {\datafractionrms} per {\timescale} does not seem
            inconsistent with a failure rate, on the Fortune 500 list,
            which it is inferred that the {\market} is similar to, of
            about 50\% in ten years, which corresponds to $(1 -
            p_f)^{120} \approx 0.5$, or $p_f$, the probability of
            failure, is $0.005759576$, which is, approximately, 2.5
            standard deviations, meaning that to be consistent with
            the large companies in the Fortune 500, the re-investment
            rate should be, approximately, $\frac{\twoponemax}{2.5}$,
            compared with an operational value, from
            Figure~\ref{\SETLABEL:NH} of {\datafractionrms}.

            An interesting, and intriguing, interpretation and
            discussion of the maximum Shannon probability, is an
            explanation as to why the companies in the {\market} are
            not running an optimal re-investment strategy. This seems
            enigmatic, since those companies that run, on a long term
            average, below the optimally maximal value would seem to
            be eclipsed by those that didn't. And those that run above
            the optimally maximal value would be over extended, and
            become financially destitute during market down turns,
            which is inevitable in a fractal time series as presented
            in Figure~\ref{\SETLABEL:TS}.  It would seem that the
            natural selection process of the competitive environment
            would allow only those companies that run near the
            optimally maximal value to survive, in the long run. One
            possible explanation, foremost, is that the analytical
            methodology presented herein is inappropriate.  Another
            explanation is that the gross margins are less than the
            fraction {\shannonmax} of the rate of revenue returns, and
            thus could not accommodate such an aggressive
            re-investment strategy. If this is the case, then it
            presents an intriguing issue. If, in a capitalistic
            market, the natural outcome of the competitive situation,
            according to game-theoretic analysis, is that there will
            be many competitors, each making minimal gross margins,
            then how do the companies grow their markets?  Naturally,
            those that run the most efficient will have lower costs,
            making larger percentage of rate of revenue returns
            re-investment possible. Yet another interpretation is that
            the number of competitors would grow at an exponential
            rate, but all of them would make minimal returns. However,
            an operational Shannon probability of {\shannonlogreturns}
            is not just marginally lower than the maximum Shannon
            probability of {\shannonmax}. There is a significant
            disparity which is difficult to explain. It would seem
            that the game-theoretic eventual outcome of a competitive
            market place would be a solution that hinders growth,
            wealth and jobs creation, etc., which does not seem
            consistent with capitalistic theory. On the other hand, is
            there an optimum number of competitors in a market place,
            where the gross margins can be higher, permitting wealth
            and job creation, and also a competitive situation? If
            this analysis is correct, and that should be subject to
            scrutiny, then it would appear that this is the case. But
            this brings up another issue---that of taxation, and other
            contributions to the social welfare function. If there is
            an optimum number of competitors in the market place, that
            maximizes wealth and job creation, then, perhaps by lemma,
            there is also an optimal value of taxation rate, and other
            contributions to the social welfare function, that will
            permit maximal industrial growth, and thus maximal growth
            in the tax base. But this would seem to be inconsistent
            with the work of Kenneth Arrow and the so called
            Impossibility Theorem, which states that such
            optimizations can not be determined because the ordering
            of priorities are intransitive.  All very perplexing,
            since the simulation of the maximum Shannon probability in
            the next section seems to indicate that such an aggressive
            re-investment strategy is, indeed, feasible.

            Yet another possibility for the industry not running at
            maximum Shannon probability is the high cost of expansion
            of operations. Some of these industries require very
            sophisticated manufacturing processes, which have high
            barrier costs.

            Additionally, as mentioned in both~\cite[pp. 29]{Brock},
            and~\cite[pp. 8]{Arthur:CTIRALIBHE}, optimal efficiency
            may not be attainable in increasing-return economic
            scenarios.

        %
% -----------------------------------------------------------------------------
%
% A license is hereby granted to reproduce this software source code and
% to create executable versions from this source code for personal,
% non-commercial use.  The copyright notice included with the software
% must be maintained in all copies produced.
%
% THIS PROGRAM IS PROVIDED "AS IS". THE AUTHOR PROVIDES NO WARRANTIES
% WHATSOEVER, EXPRESSED OR IMPLIED, INCLUDING WARRANTIES OF
% MERCHANTABILITY, TITLE, OR FITNESS FOR ANY PARTICULAR PURPOSE.  THE
% AUTHOR DOES NOT WARRANT THAT USE OF THIS PROGRAM DOES NOT INFRINGE THE
% INTELLECTUAL PROPERTY RIGHTS OF ANY THIRD PARTY IN ANY COUNTRY.
%
% Copyright (c) 1994-2006, John Conover, All Rights Reserved.
%
% Comments and/or bug reports should be addressed to:
%
%     john@email.johncon.com (John Conover)
%
% -----------------------------------------------------------------------------
%
% Revision: \RCSRevision \\
% Revision Time: \RCSTime UMT \\
% Revision Date: \RCSDate \\
% Revision Id: \RCSId \\
% Revision File: \RCSLog \\
\RCS $Revision: 0.0 $
\RCS $Date: 2006/01/20 04:38:13 $
\RCS $Id: verification.tex,v 0.0 2006/01/20 04:38:13 john Exp $
% $Log: verification.tex,v $
% Revision 0.0  2006/01/20 04:38:13  john
% Initial version
%
%
    \subsection{Qualitative Verification of Fixed Increment Approximation Analysis}
        \label{\SETLABEL:QVA}

        \subidx{\market}{verification of analysis}
        \subidx{verification}{analysis}
        \subidx{analysis}{verification}
        \subidx{quality}{of analysis}
        \subidx{verification}{of methodology}
        \subidx{methodology}{verification of}
        \subidx{Shannon}{probability}
        \subidx{probability}{Shannon}

        This section evaluates various values based on the ``average''
        of the normalized increments presented in
        Figure~\ref{\SETLABEL:TFA}. These values are an approximation
        to a, probably, complex process with a distribution shown in
        Figure~\ref{\SETLABEL:TF}. These values will be used in a
        fixed increment Brownian fractal analysis of the {\market},
        and may, or may not, provide adequate accuracy for
        projections.

        The data in this section is presented in tabular form in
        sections~\ref{\SETLABELREF:VI1} and~\ref{\SETLABELREF:VI2}.
        As a subjective evaluation of the ``quality'' of the analysis
        of the {\market}, from Chapter~\ref{methodology},
        Equation~\ref{metricvalues1}, and using the mean and root mean
        square values of the normalized increments of the time series
        data presented in Figure~\ref{\SETLABEL:TS} from
        Figure~\ref{\SETLABEL:TF}, and the Shannon probability as
        calculated by counting the total number of {\timescale}s that
        the {\market} movement was positive, as presented in
        Section~\ref{\SETLABEL:MAXSHANNON}:

        \begin{eqnarray}
                  P & \approx & \frac{\frac{avg}{rms} + 1}{2}\\
            {\pmax} & \approx & \frac{\frac{\datafractionmean}{\datafractionrms} + 1}{2}\\
            {\pmax} & \approx & {\avgrms}
            \label{\SETLABEL:AVGS}
        \end{eqnarray}

        \subidx{Shannon}{probability}
        \subidx{probability}{Shannon}
        \noindent and comparing these values to the Shannon
        probability, as found by the {\it tsshannonmax}\/ program, which
        iterates for a maximum:

        \begin{eqnarray}
            {\pmax} \approx {\avgrms} \approx {\shannonmax}
        \end{eqnarray}

        \subidx{logarithmic}{returns}
        \subidx{returns}{logarithmic}
        In addition, the different methods of calculating the
        logarithmic returns, presented in Section~\ref{\SETLABEL:FS},
        should be compared. The four methods used were the mean of
        Figure~\ref{\SETLABEL:TF}, the constant in the least squares
        approximation to Figure~\ref{\SETLABEL:TF}, the least squares
        exponential approximation to Figure~\ref{\SETLABEL:TS}, and
        the logarithmic returns of Figure~\ref{\SETLABEL:TS}, derived
        as the mean of the logarithms of the quotients of the
        increments. The values for each of the methods are,
        respectively:

        \begin{equation}
            \datafractionmeanbits \approx \datafractionconstantbits \approx \datatslsqepbits \approx \logreturns
        \end{equation}

        It is implied in Section~\ref{\SETLABEL:FS},
        Subsection~\ref{\SETLABEL:SP} and in
        Section~\ref{\SETLABEL:TSUNFAIRBROWNIAN} that, a Brownian
        motion with fixed increments fractal may ``model'' the
        {\market}. Using Equation~\ref{stddev9} from
        Chapter~\ref{general}, Section~\ref{abmfi}:

        \begin{eqnarray}
                                    rms \left(2P - 1\right) & \approx & \frac{\sigma \left(2P - 1\right)}{2 \sqrt{P\left(1 - P\right)}}\\
            \datafractionrms \left(2 \cdot \pmax - 1\right) & \approx & \frac{\datafractionstddev \left(2 \cdot \pmax - 1\right)}{2\sqrt{\pmax \left(1 - \pmax\right)}}\\
                       \datafractionrms \cdot \twopminusone & \approx & \datafractionstddev \cdot \twopx\\
                                                      \rmsp & \approx & \sigmap
        \end{eqnarray}

        \noindent and, equating to the mean:

        \begin{equation}
            \datafractionmean \approx \rmsp \approx \sigmap
        \end{equation}

        \subidx{Shannon}{probability}
        \subidx{probability}{Shannon}
        \noindent where, as in Equation~\ref{\SETLABEL:AVGS} using the
        mean, root mean square, and standard deviation values of the
        normalized increments of the time series data presented in
        Figure~\ref{\SETLABEL:TS} from Figure~\ref{\SETLABEL:TF}, and
        the Shannon probability as calculated by counting the total
        number of {\timescale}s that the {\market} movement was
        positive, as presented in Section~\ref{\SETLABEL:MAXSHANNON}.

        As a final qualitative comparison, the absolute value of the
        normalized increments should be the same as the root mean
        square value\footnote{The absolute value of the normalized
        increments, when averaged, is related to the root mean square
        of the increments by a constant. If the normalized increments
        are a fixed increment, the constant is unity. If the
        normalized increments have a Gaussian distribution, the
        constant is $\approx 0.8$ depending on the accuracy of of
        ``fit'' to a Gaussian distribution.}, where the absolute value
        is presented in Figure~\ref{\SETLABEL:TFA}, and the root mean
        square value is presented in Figure~\ref{\SETLABEL:TF}:

        \begin{equation}
            \datafractionabsmean \approx \datafractionrms
        \end{equation}

        Note, that if the {\market} could be ``modeled'' as a Brownian
        motion with fixed increments fractal, then the standard
        deviation of the absolute value of the normalized increments
        of the time series data presented in Figure~\ref{\SETLABEL:TS}
        from Figure~\ref{\SETLABEL:TF} should be zero. It is
        $\datafractionabsstddev$.

% Local Variables:
% TeX-parse-self: t
% TeX-auto-save: t
% TeX-master: "fractal.tex"
% End:


    \renewcommand{\market}{United States Electronics Market}
    \renewcommand{\directory}{../markets/electronics}
    \renewcommand{\datafractionmean}{0.008052}
\renewcommand{\datafractionmeanbits}{0.011570}
\renewcommand{\datafractionmeanq}{0.002684}
\renewcommand{\datafractionmeanbitsq}{0.003867}
\renewcommand{\datafractionstddev}{0.038579}
\renewcommand{\datafractionrms}{0.039311}
\renewcommand{\avgrms}{0.602414}
\renewcommand{\ncompanies}{5.210454}
\renewcommand{\pncompanies}{0.544866}
\renewcommand{\datafractionabsmean}{0.029745}
\renewcommand{\datafractionabsstddev}{0.025769}
\renewcommand{\datafractionconstant}{0.010041}
\renewcommand{\datafractionconstantbits}{0.014414}
\renewcommand{\datafractionconstantq}{0.003347}
\renewcommand{\datafractionconstantbitsq}{0.004821}
\renewcommand{\datafractionslope}{-0.000021}
\renewcommand{\datafractionabsconstant}{0.035145}
\renewcommand{\datafractionabsslope}{-0.000057}
\renewcommand{\hurstall}{0.659558}
\renewcommand{\hurstlow}{0.707509}
\renewcommand{\hurstlowtwo}{1.415018}
\renewcommand{\hurstlowhundred}{70.750900}
\renewcommand{\hcalcall}{0.184942}
\renewcommand{\hcalclow}{0.102042}
\renewcommand{\shannonmax}{0.604167}
\renewcommand{\twoponemax}{0.208334}
\renewcommand{\logreturns}{0.010456}
\renewcommand{\twologreturns}{1.007274}
\renewcommand{\twologreturnshundred}{0.727387}
\renewcommand{\oneoverlogreturns}{95.638868}
\renewcommand{\pmax}{0.602094}
\renewcommand{\twopminusone}{0.204188}
\renewcommand{\rmsp}{0.008027}
\renewcommand{\twopx}{0.208583}
\renewcommand{\sigmap}{0.008047}
\renewcommand{\tsunfairbrownianfractionmean}{0.007862}
\renewcommand{\tsunfairbrownianfractionstddev}{0.038619}
\renewcommand{\shannonlogreturns}{0.560125}
\renewcommand{\shannonlogreturnshundred}{56.012500}
\renewcommand{\twopone}{0.120250}
\renewcommand{\twoponehundred}{12.025000}
\renewcommand{\hundredtwoponehundred}{87.975000}
\renewcommand{\hundredshannonlogreturnshundred}{43.987500}
\renewcommand{\datatslsqepbits}{0.007623}
\renewcommand{\thurstall}{0.633980}
\renewcommand{\thurstlow}{0.710108}
\renewcommand{\thurstlowtwo}{1.420216}
\renewcommand{\thurstlowhundred}{71.010800}
\renewcommand{\thcalcall}{0.247886}
\renewcommand{\thcalclow}{0.171737}
\renewcommand{\chisquared}{2.862000}
\renewcommand{\critical}{42.557000}

    \renewcommand{\timescale}{month}
    \subidx{market}{\market}
    \idx{\market}

    \section{\market}

        \renewcommand{\SETLABEL}{\LABPRE:NAEM}
        \renewcommand{\SETLABELQ}{\LABPRE:NAEMQ}
        \label{\SETLABEL}
        \renewcommand{\SETLABELREF}{\LABPREREF:NAEM}

        \idx{United States Department of Commerce}
        For the analysis, the data was in the directory
        {\directory}\footnote{Data from the United States Department
        of Commerce, 1980---1994, by {\timescale}s, in millions of
        dollars, US.}.

        The data in this section is presented in tabular form in
        Section~\ref{\SETLABELREF}.

        %
% -----------------------------------------------------------------------------
%
% A license is hereby granted to reproduce this software source code and
% to create executable versions from this source code for personal,
% non-commercial use.  The copyright notice included with the software
% must be maintained in all copies produced.
%
% THIS PROGRAM IS PROVIDED "AS IS". THE AUTHOR PROVIDES NO WARRANTIES
% WHATSOEVER, EXPRESSED OR IMPLIED, INCLUDING WARRANTIES OF
% MERCHANTABILITY, TITLE, OR FITNESS FOR ANY PARTICULAR PURPOSE.  THE
% AUTHOR DOES NOT WARRANT THAT USE OF THIS PROGRAM DOES NOT INFRINGE THE
% INTELLECTUAL PROPERTY RIGHTS OF ANY THIRD PARTY IN ANY COUNTRY.
%
% Copyright (c) 1994-2006, John Conover, All Rights Reserved.
%
% Comments and/or bug reports should be addressed to:
%
%     john@email.johncon.com (John Conover)
%
% -----------------------------------------------------------------------------
%
% Revision: \RCSRevision \\
% Revision Time: \RCSTime UMT \\
% Revision Date: \RCSDate \\
% Revision Id: \RCSId \\
% Revision File: \RCSLog \\
\RCS $Revision: 0.0 $
\RCS $Date: 2006/01/20 04:38:13 $
\RCS $Id: fraction.tex,v 0.0 2006/01/20 04:38:13 john Exp $
% $Log: fraction.tex,v $
% Revision 0.0  2006/01/20 04:38:13  john
% Initial version
%
%
    \subsection{Time Series Increments Analysis}
        \label{\SETLABEL:TSA}

        \subidx{\market}{Time series analysis}
        \subidx{time series}{increments}
        \subidx{time series}{analysis}
        \subidx{cumulative sum}{analysis}
        \subidx{analysis}{cumulative sum}
        \subidx{analysis}{random process}
        \subidx{random process}{analysis}
        \subidx{Gaussian}{increments}
        \subidx{increments}{Gaussian}
        \subidx{Brownian}{motion, fractional}
        \subidx{fractional}{Brownian motion}
        \subidx{fractal}{Brownian motion}
        The data in this section is presented in tabular form in
        Section~\ref{\SETLABELREF:TSA}.  Figure~\ref{\SETLABEL:TS} is
        a graph of the time series data for the {\market}.

        \subidx{increments}{normalized}
        \subidx{normalized}{increments}
        \subidx{programs}{tsfraction}
        \subidx{tsfraction}{program}
        Figure~\ref{\SETLABEL:TF} is a graph of the normalized
        increments of the time series data presented in
        Figure~\ref{\SETLABEL:TS}. The data presented was made by
        running the program {\it tsfraction}\/ on the time series
        data. The program {\it tsfraction}\/ is described briefly in
        Appendix~\ref{programs}, and subtracts the previous value from
        the next value, dividing this difference by the previous
        value, for each element in the time series data. The new time
        series contains the instantaneous change in the rate of
        revenue returns, divided by the magnitude of the instantaneous
        rate of revenue returns.

        \subidx{mean}{standard deviation}
        \subidx{standard deviation}{mean}
        \idx{root mean square}
        \idx{least squares approximation}
        \begin{figure}[ht]
            \begin{center}
                \begin{minipage}[t]{0.45\textwidth}
                    \epsfxsize=1.0\linewidth
                    \epsffile{\directory/data.eps}
                    \caption{{\market}, time series data.}
                    \label{\SETLABEL:TS}
                    \label{\SETLABELQ:TS}
                \end{minipage}
                \hfill
                \begin{minipage}[t]{0.45\textwidth}
                    \epsfxsize=1.0\linewidth
                    \epsffile{\directory/data.tsfraction.eps}
                    \caption[{\market}, normalized
                        increments]{{\market}, normalized increments
                        of the time series data presented in
                        Figure~\ref{\SETLABEL:TS}. The mean is
                        {\datafractionmean} with a standard deviation
                        of {\datafractionstddev}. The formula for the
                        least squares approximation is
                        ${\datafractionconstant} +
                        {\datafractionslope}t$, and the root mean
                        squared value is {\datafractionrms}. The
                        graph, labeled ``data\-.tsfraction\-.tsrms,''
                        is the running root mean square, and
                        ``data\-.tsfraction\-.tsavg'' is the running
                        average of the normalized increments.  This
                        graph is the fraction of change in the time
                        series, as a function of time. Note that the
                        slope of the mean, {\datafractionslope}, is
                        the coefficient of the nonlinearity term in
                        the normalized increments. See
                        Chapter~\ref{general}, Section~\ref{nlextend}
                        for a possible application of the logistic
                        function to this data set.}
                    \label{\SETLABEL:TF}
                    \label{\SETLABELQ:TF}
                \end{minipage}
            \end{center}
        \end{figure}

        \subidx{absolute value}{increments}
        \subidx{increments}{absolute value}

        Figure~\ref{\SETLABEL:TFA} is a graph of the absolute value of
        the normalized increments of the time series data presented in
        Figure~\ref{\SETLABEL:TF}. The data presented was made by
        running the Unix utility sed(1) on the normalized increments
        time series data to remove the negative signs. This is an
        absolute value procedure.  The resulting time series contains
        the absolute value of the instantaneous change in the rate of
        revenue returns, divided by the magnitude of the instantaneous
        rate of revenue returns\footnote{The absolute value of the
        normalized increments, when averaged, is related to the root
        mean square of the increments by a constant. If the normalized
        increments are a fixed increment, the constant is unity. If
        the normalized increments have a Gaussian distribution, the
        constant is $\approx 0.8$ depending on the accuracy of of
        ``fit'' to a Gaussian distribution.}.

        \subidx{histogram}{normalized}
        \subidx{normalized}{histogram}
        \subidx{programs}{tsnormal}
        \subidx{tsnormal}{program}
        \subidx{mean}{standard deviation}
        \subidx{standard deviation}{mean}
        \idx{root mean square}
        \idx{least squares approximation}
        \subidx{\market}{analysis of increments}
        Figure~\ref{\SETLABEL:NH} is the normalized histogram of the
        normalized increments of the time series data shown in
        Figure~\ref{\SETLABEL:TF}. The abscissa is 3 $\sigma$ limits,
        and the area under the two curves is identical. The data for
        this figure was produced by the program {\it tsnormal}\/,
        which is described briefly in Appendix~\ref{programs}.

        \begin{figure}[ht]
            \begin{center}
                \begin{minipage}[t]{0.45\textwidth}
                    \epsfxsize=1.0\linewidth
                    \epsffile{\directory/data.tsfraction.abs.eps}
                    \caption[{\market}, absolute value of the
                        normalized increments]{{\market}, absolute
                        value of the normalized increments of the time
                        series data presented in
                        Figure~\ref{\SETLABEL:TF}.  The mean is
                        {\datafractionabsmean} with a standard
                        deviation of {\datafractionabsstddev}. The
                        formula for the least squares approximation is
                        ${\datafractionabsconstant} +
                        {\datafractionabsslope}t$, and the root mean
                        square value, from Figure~\ref{\SETLABEL:TF},
                        is {\datafractionrms}.  The graph, labeled
                        ``data\-.tsfraction\-.tsrms,'' is the running
                        root mean square, and
                        ``data\-.tsfraction\-.tsavg'' is the running
                        average of the normalized increments presented
                        in Figure~\ref{\SETLABEL:TF}, superimposed
                        here for convenience. This graph is the
                        absolute value of the fraction of change in
                        the time series, as a function of time.}
                    \label{\SETLABEL:TFA}
                    \label{\SETLABELQ:TFA}
                \end{minipage}
                \hfill
                \begin{minipage}[t]{0.45\textwidth}
                    \epsfxsize=1.0\linewidth
                    \epsffile{\directory/data.tsfraction.tsnormal-s30.eps}
                    \caption[{\market}, normalized histogram of the
                        normalized increments]{{\market}, normalized
                        histogram of the normalized increments of the
                        time series data shown in
                        Figure~\ref{\SETLABEL:TF}.  The data has a
                        mean of {\datafractionmean}, with a standard
                        deviation of {\datafractionstddev}.  The area
                        under the two curves is identical. The
                        $\chi^2$ value of the observed and expected
                        values of the two curves is {\chisquared},
                        with a critical value of {\critical}.}
                    \label{\SETLABEL:NH}
                \end{minipage}
            \end{center}
        \end{figure}

        \subidx{programs}{tsXsquared}
        \subidx{tsXsquared}{program}
        \subidx{\market}{chi-squared values of increments}
        The program {\it tsXsquared}\/, which is briefly described in
        appendix~\ref{programs}, was used to derive the $\chi^2$
        statistics for the data presented in
        Figure~\ref{\SETLABEL:NH}.

        \subidx{programs}{tsstatest}
        \subidx{tsstatest}{program}
        \subidx{\market}{statistical estimates}

        Figure~\ref{\SETLABEL:SE} is the statistical estimate for the
        data presented in Figure~\ref{\SETLABEL:TF}, as derived by the
        program {\it tsstatest}\/, which is briefly described in
        appendix~\ref{programs}.

        \begin{figure}[ht]
            \begin{center}
                \begin{minipage}[t]{\textwidth}
                    \center{\fbox{\parbox{0.9\textwidth}{\XXX{\directory/data.tsstatest-f0.1-c0.9-i.tex}}}}
                    \caption[{\market}, statistical estimates of the
                        normalized increments]{{\market}, statistical
                        estimates of the normalized increments of the
                        time series shown in Figure~\ref{\SETLABEL:TF}.
                        The table was produced with the {\it
                        tsstatest}\/ program, and illustrates the
                        size of the data set required for a confidence
                        level of 90\%, with an error estimate of $\pm$
                        10\%, or alternately, the error estimate on
                        the time series shown in Figure~\ref{\SETLABEL:TF}.}
                    \label{\SETLABEL:SE}
                \end{minipage}
            \end{center}
        \end{figure}

        Note that the data set size estimations, as produced by the
        {\it tsstatest}\/ program, are probably very conservative,
        depending on the magnitude of the Shannon probability, $P =
        \shannonlogreturns$, as derived in
        Section~\ref{\SETLABEL:SP}. See Chapter~\ref{general},
        Section~\ref{serdss} for possible alternative methodologies
        for addressing the analysis of fractal time series with
        limited data set sizes. Depending on the magnitude of the
        Shannon probability, $P$, these estimates can be several
        orders of magnitude too high.

        \subidx{derivative of increments}{normalized}
        \subidx{normalized}{derivative of increments}
        \subidx{programs}{tsderivative}
        \subidx{tsderivative}{program}
        Figure~\ref{\SETLABEL:TF1} is the normalized histogram of the
        first derivative of the normalized increments of the time
        series data shown in Figure~\ref{\SETLABEL:TF}. In principle,
        if the distribution of the normalized increments presented in
        Figure~\ref{\SETLABEL:NH} is Gaussian in nature, this
        distribution would be similar to ``white noise,'' as presented
        in appendix~\ref{programs}, Figure~\ref{whiteexample}. The
        data was generated by the {\it tsderivative}\/ program, which
        is briefly described in
        appendix~\ref{programs}. Figure~\ref{\SETLABEL:TF2} is the
        normalized histogram of the second derivative of the
        normalized increments of the time series data shown in
        Figure~\ref{\SETLABEL:TF}. In principle, if the distribution
        of the normalized increments presented in
        Figure~\ref{\SETLABEL:NH} is an integrated Gaussian
        distribution in nature, this distribution would be similar to
        ``white noise,'' as presented in appendix~\ref{programs},
        Figure~\ref{whiteexample}.

        \begin{figure}[ht]
            \begin{center}
                \begin{minipage}[t]{0.45\textwidth}
                    \epsfxsize=1.0\linewidth
                    \epsffile{\directory/data.tsfraction.tsderivative.tsnormal-s30.eps}
                    \caption[{\market}, histogram of the first
                        derivative of the increments]{{\market},
                        normalized histogram of the first derivative
                        of the normalized increments of the time
                        series data shown in
                        Figure~\ref{\SETLABEL:TF}.}
                    \label{\SETLABEL:TF1}
                \end{minipage}
                \hfill
                \begin{minipage}[t]{0.45\textwidth}
                    \epsfxsize=1.0\linewidth
                    \epsffile{\directory/data.tsfraction.2tsderivative.tsnormal-s30.eps}
                    \caption[{\market}, histogram of the second
                        derivative of the increments]{{\market},
                        normalized histogram of second derivative of
                        the the normalized increments of the time
                        series data shown in
                        Figure~\ref{\SETLABEL:TF}.}
                    \label{\SETLABEL:TF2}
                \end{minipage}
            \end{center}
        \end{figure}

        \subidx{fractal}{range}
        \subidx{fractal}{R/S analysis}
        \subidx{\market}{rate of revenue returns, range}
        \subidx{\market}{deterministic mechanism}
        \subidx{deterministic}{mechanism}
        \subidx{mechanism}{deterministic}
        Figure~\ref{\SETLABEL:TR} is the range of values of the time
        series shown in Figure~\ref{\SETLABEL:TS}. The horizontal axis
        is time into the future. In principle, if the time series was
        characterized as fractional Brownian motion the graph in
        Figure~\ref{\SETLABEL:TR} would be a square root
        function\footnote{Note that the ``roughness,'' or ``sawtooth''
        characteristics of the graph in Figure~\ref{\SETLABEL:TR} are
        a computational artifact---caused by not using the -m option
        to the program {\it tshurst}\/, which is computationally
        inefficient.}. Figure~\ref{\SETLABEL:TD} is the deterministic
        map of the normalized increments of the time series data shown
        in Figure~\ref{\SETLABEL:TF}. The deterministic map is useful
        for determining if a time series was created by a
        deterministic mechanism. This, essentially, maps each element
        in the time series with the previous element in the time
        series.  See,~\cite[pp. 745]{Peitgen}.

        \begin{figure}[ht]
            \begin{center}
                \begin{minipage}[t]{0.45\textwidth}
                    \epsfxsize=1.0\linewidth
                    \epsffile{\directory/data.tshurst-f.eps}
                    \caption[{\market}, range]{{\market}, range of the
                        time series data shown in
                        Figure~\ref{\SETLABEL:TS}.}
                    \label{\SETLABEL:TR}
                \end{minipage}
                \hfill
                \begin{minipage}[t]{0.45\textwidth}
                    \epsfxsize=1.0\linewidth
                    \epsffile{\directory/data.tsfraction.tsdeterministic.eps}
                    \caption[{\market}, deterministic map]{{\market},
                        deterministic map of the normalized increments
                        of the time series data shown in
                        Figure~\ref{\SETLABEL:TF}.}
                    \label{\SETLABEL:TD}
                \end{minipage}
            \end{center}
        \end{figure}

% Local Variables:
% TeX-parse-self: t
% TeX-auto-save: t
% TeX-master: "fractal.tex"
% End:


        \subsubsection{Observations on the Time Series Increments Analysis}

            Figure~\ref{\SETLABEL:NH} would seem to indicate that the
            time series data for the {\market} represents a cumulative
            sum/integration of a random process that has a Gaussian
            distribution, (ie., satisfies the Gaussian increments
            property of fractional Brownian
            motion~\cite[pp. 250]{Crownover},) tending to justify the
            assumption that the time series data represents fractional
            Brownian motion.

        %
% -----------------------------------------------------------------------------
%
% A license is hereby granted to reproduce this software source code and
% to create executable versions from this source code for personal,
% non-commercial use.  The copyright notice included with the software
% must be maintained in all copies produced.
%
% THIS PROGRAM IS PROVIDED "AS IS". THE AUTHOR PROVIDES NO WARRANTIES
% WHATSOEVER, EXPRESSED OR IMPLIED, INCLUDING WARRANTIES OF
% MERCHANTABILITY, TITLE, OR FITNESS FOR ANY PARTICULAR PURPOSE.  THE
% AUTHOR DOES NOT WARRANT THAT USE OF THIS PROGRAM DOES NOT INFRINGE THE
% INTELLECTUAL PROPERTY RIGHTS OF ANY THIRD PARTY IN ANY COUNTRY.
%
% Copyright (c) 1994-2006, John Conover, All Rights Reserved.
%
% Comments and/or bug reports should be addressed to:
%
%     john@email.johncon.com (John Conover)
%
% -----------------------------------------------------------------------------
%
% Revision: \RCSRevision \\
% Revision Time: \RCSTime UMT \\
% Revision Date: \RCSDate \\
% Revision Id: \RCSId \\
% Revision File: \RCSLog \\
\RCS $Revision: 0.0 $
\RCS $Date: 2006/01/20 04:38:13 $
\RCS $Id: instant.tex,v 0.0 2006/01/20 04:38:13 john Exp $
% $Log: instant.tex,v $
% Revision 0.0  2006/01/20 04:38:13  john
% Initial version
%
%
    \subsection{Instantaneous Analysis of Normalized Increments}
        \label{\SETLABEL:IA}

        \subidx{\market}{instantaneous analysis of normalized increments}
        \idx{average of normalized increments}
        \idx{root mean square of normalized increments}
        \subidx{Shannon probability}{instantaneous computation of}
        \subidx{average of normalized increments}{instantaneous computation of}
        \subidx{root mean square of normalized increments}{instantaneous computation of}
        \subidx{instantaneous computation}{Shannon probability}
        \subidx{instantaneous computation}{average of normalized increments}
        \subidx{instantaneous computation}{root mean square of normalized increments}
        \idx{time series}
        \subidx{time series}{instantaneous analysis}
        \subidx{instantaneous analysis}{time series}
        \subidx{time series}{increments}
        \subidx{time series}{analysis}
        \subidx{Shannon}{probability}
        \subidx{probability}{Shannon}
        \subidx{normalized}{increments}
        \subidx{increments}{normalized}

        The program {\it tsinstant}\/, which is briefly described in
        Appendix~\ref{programs}, is for finding the instantaneous
        fraction of change in a time series. The value of a sample in
        the time series is subtracted from the previous sample in the
        time series, and divided by the value of the previous sample.
        As explained in Chapter~\ref{general},
        Sections~\ref{derivation},~\ref{GA},~\ref{abmfi},~\ref{aftsma}
        and,~\ref{ompl} for Brownian motion, random walk fractals, the
        absolute value of the instantaneous fraction of change is also
        the root mean square of the instantaneous fraction of
        change\footnote{The absolute value of the normalized
        increments, when averaged, is related to the root mean square
        of the increments by a constant. If the normalized increments
        are a fixed increment, the constant is unity. If the
        normalized increments have a Gaussian distribution, the
        constant is $\approx 0.8$ depending on the accuracy of of
        ``fit'' to a Gaussian distribution.}. Squaring this value is
        the average of the instantaneous fraction of change, and
        adding unity to the absolute value of the instantaneous
        fraction of change, and dividing by two, is the Shannon
        probability of the instantaneous fraction of change.

        Figure~\ref{\SETLABEL:IA1} is the instantaneous value of the
        root mean square of the normalized increments for the
        {\market}, and Figure~\ref{\SETLABEL:IA2} is the instantaneous
        Shannon probability for the normalized increments.

        \begin{figure}[ht]
            \begin{center}
                \begin{minipage}[t]{0.45\textwidth}
                    \epsfxsize=1.0\linewidth
                    \epsffile{\directory/data.tsinstant-r.eps}
                    \caption[{\market}, instantaneous value of
                        rms.]{{\market}, instantaneous value of the
                        root mean square of the normalized increments,
                        provided by running the program {\it
                        tsinstant}\/ with the -r option on the data
                        presented in Figure~\ref{\SETLABEL:TS}.}
                    \label{\SETLABEL:IA1}
                    \label{\SETLABELQ:IA1}
                \end{minipage}
                \hfill
                \begin{minipage}[t]{0.45\textwidth}
                    \epsfxsize=1.0\linewidth
                    \epsffile{\directory/data.tsinstant-s.eps}
                    \caption[{\market}, instantaneous value of
                        Shannon probability.]{{\market}, instantaneous
                        value of the Shannon probability of the
                        normalized increments, provided by running the
                        program {\it tsinstant}\/ with the -s option
                        on the data presented in
                        Figure~\ref{\SETLABEL:TS}.}
                    \label{\SETLABEL:IA2}
                    \label{\SETLABELQ:IA2}
                \end{minipage}
            \end{center}
        \end{figure}

% Local Variables:
% TeX-parse-self: t
% TeX-auto-save: t
% TeX-master: "fractal.tex"
% End:


        %
% -----------------------------------------------------------------------------
%
% A license is hereby granted to reproduce this software source code and
% to create executable versions from this source code for personal,
% non-commercial use.  The copyright notice included with the software
% must be maintained in all copies produced.
%
% THIS PROGRAM IS PROVIDED "AS IS". THE AUTHOR PROVIDES NO WARRANTIES
% WHATSOEVER, EXPRESSED OR IMPLIED, INCLUDING WARRANTIES OF
% MERCHANTABILITY, TITLE, OR FITNESS FOR ANY PARTICULAR PURPOSE.  THE
% AUTHOR DOES NOT WARRANT THAT USE OF THIS PROGRAM DOES NOT INFRINGE THE
% INTELLECTUAL PROPERTY RIGHTS OF ANY THIRD PARTY IN ANY COUNTRY.
%
% Copyright (c) 1994-2006, John Conover, All Rights Reserved.
%
% Comments and/or bug reports should be addressed to:
%
%     john@email.johncon.com (John Conover)
%
% -----------------------------------------------------------------------------
%
% Revision: \RCSRevision \\
% Revision Time: \RCSTime UMT \\
% Revision Date: \RCSDate \\
% Revision Id: \RCSId \\
% Revision File: \RCSLog \\
\RCS $Revision: 0.0 $
\RCS $Date: 2006/01/20 04:38:13 $
\RCS $Id: logistic.tex,v 0.0 2006/01/20 04:38:13 john Exp $
% $Log: logistic.tex,v $
% Revision 0.0  2006/01/20 04:38:13  john
% Initial version
%
%
    \subsection{Logistic Analysis}
        \label{\SETLABEL:LA}

        \subidx{\market}{Logistic function analysis}
        \subidx{time series}{logistic function}
        \subidx{logistic function}{time series}
        \subidx{time series}{increments}
        \subidx{time series}{analysis}
        \subidx{cumulative sum}{analysis}
        \subidx{analysis}{cumulative sum}
        \subidx{analysis}{random process}
        \subidx{random process}{analysis}
        The data in this section is presented in tabular form in
        Section~\ref{\SETLABELREF:LAA}.  Figure~\ref{\SETLABEL:LA1} is
        a graph of the logistic function estimates of the time series
        data for the {\market}. The reader is cautioned that these
        graphs are constructed using the method suggested in
        Chapter~\ref{general}, Section~\ref{nlextend} and enormous
        precision is required for adequate prediction of the logistic
        function,~\cite{Modis}. Particularly, the non-linear term will
        usually require intervention to produce a practical fit to the
        data. In addition, there are numerical stability issues with
        logistic function methodologies\footnote{For example, in
        Figures~\ref{\SETLABEL:LA1} and~\ref{\SETLABEL:LA2}, if the
        non-linear term, $b$, was greater than zero, it was set to
        zero to produce the graphs. See Section~\ref{\SETLABELREF:LAA}
        for the actual derived values. In other cases, the magnitude
        of $b$ was too large, resulting in a graph that was decreasing
        as a function of time}.  The methodology should be regarded as
        ``fragile.'' It is included for completeness.

        \idx{least squares approximation}
        Figure~\ref{\SETLABEL:LA1} is a graph of the logistic function
        for the time series data presented in
        Figure~\ref{\SETLABEL:TS}. The data presented was made by
        running the program {\it tsdlogistic}\/, which is described
        briefly in Appendix~\ref{programs}, on the parameters
        extracted from the time series data as suggested in
        Figure~\ref{\SETLABEL:TF}. The program {\it tslsq}\/ was used
        to derive the constant and the slope of the normalized
        increments of the data presented in Figure~\ref{\SETLABEL:TF}.
        Figure~\ref{\SETLABEL:LA2} is the same graph, but with the
        time scale expanded by a factor of two.

        \begin{figure}[ht]
            \begin{center}
                \begin{minipage}[t]{0.45\textwidth}
                    \epsfxsize=1.0\linewidth
                    \epsffile{\directory/data.tsfraction.tslsq-p.tsdlogistic.eps}
                    \caption[{\market}, logistic function
                        estimates.]{{\market}, logistic function
                        estimates, provided by running the {\it
                        tslsq}\/ program on the normalized increments
                        presented in Figure~\ref{\SETLABEL:TF} with
                        the -p option. These parameters were used as
                        arguments to the {\it tsdlogistic}\/ program.}
                    \label{\SETLABEL:LA1}
                    \label{\SETLABELQ:LA1}
                \end{minipage}
                \hfill
                \begin{minipage}[t]{0.45\textwidth}
                    \epsfxsize=1.0\linewidth
                    \epsffile{\directory/data.tsfraction.tslsq-p.tsdlogistic2.eps}
                    \caption[{\market}, logistic function
                        estimates.]{{\market}, logistic function
                        estimates of Figure~\ref{\SETLABEL:LA1} with
                        the time scale expanded by a factor of two.}
                    \label{\SETLABEL:LA2}
                    \label{\SETLABELQ:LA2}
                \end{minipage}
            \end{center}
        \end{figure}

% Local Variables:
% TeX-parse-self: t
% TeX-auto-save: t
% TeX-master: "fractal.tex"
% End:


        %
% -----------------------------------------------------------------------------
%
% A license is hereby granted to reproduce this software source code and
% to create executable versions from this source code for personal,
% non-commercial use.  The copyright notice included with the software
% must be maintained in all copies produced.
%
% THIS PROGRAM IS PROVIDED "AS IS". THE AUTHOR PROVIDES NO WARRANTIES
% WHATSOEVER, EXPRESSED OR IMPLIED, INCLUDING WARRANTIES OF
% MERCHANTABILITY, TITLE, OR FITNESS FOR ANY PARTICULAR PURPOSE.  THE
% AUTHOR DOES NOT WARRANT THAT USE OF THIS PROGRAM DOES NOT INFRINGE THE
% INTELLECTUAL PROPERTY RIGHTS OF ANY THIRD PARTY IN ANY COUNTRY.
%
% Copyright (c) 1994-2006, John Conover, All Rights Reserved.
%
% Comments and/or bug reports should be addressed to:
%
%     john@email.johncon.com (John Conover)
%
% -----------------------------------------------------------------------------
%
% Revision: \RCSRevision \\
% Revision Time: \RCSTime UMT \\
% Revision Date: \RCSDate \\
% Revision Id: \RCSId \\
% Revision File: \RCSLog \\
\RCS $Revision: 0.0 $
\RCS $Date: 2006/01/20 04:38:13 $
\RCS $Id: hurst.tex,v 0.0 2006/01/20 04:38:13 john Exp $
% $Log: hurst.tex,v $
% Revision 0.0  2006/01/20 04:38:13  john
% Initial version
%
%
    \subsection{Hurst Coefficient Analysis}
        \label{\SETLABEL:H}

        \subidx{\market}{Hurst coefficient analysis}
        \subidx{Hurst coefficient}{analysis}
        \subidx{increments}{normalized}
        \subidx{normalized}{increments}
        \subidx{programs}{tshurst}
        \subidx{tshurst}{program}
        The data in this section is presented in tabular form in
        Section~\ref{\SETLABELREF:HCHP}. Figure~\ref{\SETLABEL:HC} is
        a graph of the Hurst coefficient data time series data shown
        in Figure~\ref{\SETLABEL:TS}. The slope of the graph is the
        Hurst coefficient.  The data for this figure was produced by
        the program {\it tshurst}\/, which is described briefly in
        Appendix~\ref{programs}.

        \subidx{\market}{H parameter analysis}
        \subidx{H parameter}{analysis}
        \subidx{programs}{tshcalc}
        \subidx{tshcalc}{program}
        Figure~\ref{\SETLABEL:HP} is a graph of the H parameter data
        for the normalized increments of the time series data shown in
        Figure~\ref{\SETLABEL:TF}. The data for this figure was
        produced by the program {\it tshcalc}\/, which is described
        briefly in Appendix~\ref{programs}.

        \begin{figure}[ht]
            \begin{center}
                \begin{minipage}[t]{0.45\textwidth}
                    \epsfxsize=1.0\linewidth
                    \epsffile{\directory/data.tshurst.eps}
                    \caption[{\market}, Hurst coefficient data]{{\market},
                        Hurst coefficient data for the normalized
                        increments of the time series data shown in
                        Figure~\ref{\SETLABEL:TF}.  The slope of the graph
                        is the Hurst coefficient.}
                    \label{\SETLABEL:HC}
                \end{minipage}
                \hfill
                \begin{minipage}[t]{0.45\textwidth}
                    \epsfxsize=1.0\linewidth
                    \epsffile{\directory/data.tshcalc.eps}
                    \caption[{\market}, H parameter data]{{\market}, H
                        parameter data for the normalized increments of
                        the time series data shown in
                        Figure~\ref{\SETLABEL:TF} The slope of the graph
                        is the H parameter.}
                    \label{\SETLABEL:HP}
                \end{minipage}
            \end{center}
        \end{figure}

        \subidx{revenue}{See, rate of revenue returns}
        \subidx{returns}{See, rate of revenue returns}
        \subidx{\market}{revenues}
        \subidx{Hurst coefficient}{analysis}
        \subidx{\market}{Hurst coefficient analysis}
        \subidx{\market}{rate of change}
        \subidx{\market}{windows of opportunity}
        \subidx{rate of revenue returns}{forecast}
        \subidx{forecast}{rate of revenue returns}
        \idx{windows of opportunity}
        \subidx{programs}{tslsq}
        \subidx{tslsq}{program}

        The approximately linear slope of the graph in
        Figure~\ref{\SETLABEL:HC} implies that the variance of the
        rate of revenue returns, (per {\timescale},) in the {\market},
        $V(t_2 - t_1)$, over a period of time is proportional to the
        period of time raised to twice the Hurst
        coefficient~\cite[pp. 180]{Feder},~\cite[pp. 246]{Crownover}.
        This seems to be a quantitative statement concerning how fast,
        and to what degree, the rate of revenue returns' state of
        affairs can change over a period of time.  An additional
        implication, for Hurst coefficients sufficiently close to 0.5,
        is that the probability of the state of affairs repeating
        sometime in the future goes down with increasing
        time\footnote{It can be shown that the number of expected
        market ``high'' and ``low'' transitions, $N$, scales with the
        square root of time, or $N \propto \sqrt {t}$, meaning that
        the cumulative distribution of the probability, $P$, of the
        duration of a market's ``high'' or ``low'' exceeding a given
        time interval, $t$, is proportional to the reciprocal of the
        square root of the time interval, $P \propto 1 / \sqrt {t}$,
        (or, conversely, that the probability of the duration of a
        market's ``high'' or ``low'' exceeding a given time interval
        is proportional to the reciprocal of the time interval raised
        to the power $3 / 2$, ie., $P \propto 1 / t^{3 /
        2}$,~\cite[pp. 153]{Schroeder}. What this means is that a
        histogram of the ``zero free'' run-lengths of a market being
        ``high'' or ``low,'' over a long time, would have a $1 / t^{3
        / 2}$ characteristic.)}, $t$, $p(t) = erf (1/\sqrt{2t})$ which
        is approximately $1/\sqrt{t}$ for $t \gg
        1$~\cite[pp. 160]{Schroeder}. Figures~\ref{\SETLABEL:FN},
        and,~\ref{\SETLABEL:FF} compare methods of approximation of
        the ``forecastability'' of the rate of revenue returns in the
        {\market} for the near term and far term,
        respectively~\cite[pp. 83-84]{Peters:CAOITCM}\footnote{The
        author is not comfortable with Peters' interpretation. For
        example, if the algorithm explained
        in~\cite[pp. 82]{Peters:CAOITCM} is used on ``white noise''
        which, by definition, never has any correlations, the short
        term Hurst coefficient, and thus the ``forecastability,'' is
        still near unity---a bit of an enigma. This can be verified
        with the {\it tswhite}\/ and {\it tshurst}\/ programs, which
        are briefly described in Appendix~\ref{programs}.}.  This
        seems to be a quantitative statement concerning ``windows of
        opportunity'' in the rate of revenue returns, (per
        {\timescale}.)  The program {\it tslsq}\/ was used on the
        Hurst coefficient data, presented in
        Figure~\ref{\SETLABEL:HC}, to provide a least squares
        approximation to the Hurst coefficient. The superimposed least
        squares approximation with on original Hurst coefficient data
        is presented.  The time series data has a Hurst coefficient of
        {\thurstlow}, so that:

        \subidx{\market}{Hurst coefficient analysis}
        \begin{eqnarray}
            V\left(t_2 - t_1\right) & \propto & \left(t_2 - t_1\right)^{2 \cdot H}\\
            V\left(t_2 - t_1\right) & \propto & \left(t_2 - t_1\right)^{2 \cdot {\thurstlow}}\\
                                    & \propto & \left(t_2 - t_1\right)^{\thurstlowtwo}
            \label{\SETLABEL:V}
        \end{eqnarray}

        \subidx{fractional}{Brownian motion}
        \subidx{Brownian motion}{fractional}
        \idx{fractal}
        \noindent where $V(t_2 - t_1)$ is the variance of the
        increments of the rate of revenue returns, (per {\timescale},)
        over the time interval $t_2 -
        t_1$,~\cite[pp. 177]{Feder},~\cite[pp. 494]{Peitgen}. If $H >
        \frac{1}{2}$, then the time series is termed as being
        characterized by ``fractional Brownian
        motion~\cite[pp. 170]{Feder}.''

        \subidx{rate of revenue returns}{predictability}
        \subidx{rate of revenue returns}{forecastability}
        \subidx{rate of revenue returns}{consistency}
        \subidx{predictability}{rate of revenue returns}
        \subidx{forecastability}{rate of revenue returns}
        \subidx{consistency}{rate of revenue returns}
        \subidx{\market}{rate of revenue returns, predictability}
        \subidx{\market}{rate of revenue returns, forecastability}
        \subidx{\market}{rate of revenue returns, consistency}
        \subidx{Hurst coefficient}{analysis}
        \subidx{\market}{Hurst coefficient analysis}
        \subidx{\market}{rate of change}

        In some sense, the Hurst coefficient is a quantitative
        expression of the ``forecastability'' of the future based on
        the past\footnote{Actually, in general, when summing fractal
        entities, the method used should be a root mean square
        process, dependent on the Hurst Coefficient, $H$, where
        $P_{total}^H = P_1^H + P_2^H + \cdots$, where $P_n$ is the
        fractal entities. For a Brownian motion, or random walk type
        of fractal the Hurst Coefficient is a function of time into
        the future. For the ``near term,'' the Hurst coefficient is
        very near unity, meaning the summation process is linear. For
        the ``long term,'' $H \approx 0.5$, or a standard root mean
        square summation process should be used. If $H$ is $0.5$ then
        the market is termed a Brownian motion, or random walk
        process. If it is larger than 0.5, it is termed fractional
        Brownian motion process. For a random walk process, ``near
        term'' and ``far term'' are quantitatively differentiated on
        the Hurst Coefficient graph where $1 - \ln (t) = 0.5 \cdot \ln
        (t)$, or when $\ln (t) = 2$, or $t = 7.389\ldots$ See
        Section~\ref{\SETLABEL:FS} for the particulars on using Hurst
        Coefficient to sum fractal process' for the {\market}. See
        also~\cite[pp. 67, 83-84]{Peters:CAOITCM} and~\cite[pp. 129,
        159]{Schroeder} for particulars on the implications of the
        Hurst Coefficient and root mean square summation issues.}.  A
        Hurst coefficient of {\thurstlow}, (for the near future, and
        {\thurstall} for the distant future.) implies that the
        likelihood of the rate of revenue returns, (per {\timescale},)
        for any two consecutive {\timescale}s being the same is
        {\thurstlowhundred}\%~\cite[pp. 66]{Peters:CAOITCM} for the
        near future, and {\thurstall} for the distant
        future. Likewise, there is a {\thurstlowhundred}\% chance of
        the rate of revenue returns, (per {\timescale},) movements
        being the same in consecutive time periods---ie., if, in a
        given {\timescale}, the rate of revenue returns, (per
        {\timescale},) is increasing, there is a {\thurstlowhundred}\%
        that the rate of revenue returns, (per {\timescale},) will
        increase in the following period, also. In some sense, this is
        a quantitative statement on how ``predictable,'' or
        ``forecastable'' the rate of revenue returns, (per
        {\timescale},) for the {\market} are over time, since the
        probability of having $n$ many consecutive {\timescale}s of
        the same agenda is $H^n$ where $H$ is the Hurst coefficient,
        or, letting the short term probability of having $n$ many
        {\timescale}s of the same market agenda, $p_a$, is:

        \begin{eqnarray}
            p_a\left(n\right) & = & H^{n}\\
                              & = & {\thurstlow}^{n}
            \label{\SETLABEL:MA}
        \end{eqnarray}

        \subidx{rate of revenue returns}{predictability}
        \subidx{rate of revenue returns}{forecastability}
        \subidx{rate of revenue returns}{consistency}
        \subidx{predictability}{rate of revenue returns}
        \subidx{forecastability}{rate of revenue returns}
        \subidx{consistency}{rate of revenue returns}
        As an interesting interpretation of the normalized increments
        of the time series data presented in
        Figure~\ref{\SETLABEL:TF}, if the vertical axis is multiplied
        by 100, to convert to percent, then the graph represents the
        error, in percent, that would be made by forecasting, month by
        month, that the next {\timescale}'s rate of revenue returns
        would be the same as the current {\timescale}'s revenue
        rate. Interestingly, it is $\datafractionmean \cdot 100$
        percent, on the average, with a standard deviation of
        $\datafractionstddev \cdot 100$ percent, and a root mean
        square error value of $\datafractionrms \cdot 100$
        percent---small values for such a simple forecasting
        mechanism.

        \subidx{\market}{rate of revenue returns, range}
        \subidx{Hurst coefficient}{analysis}
        \subidx{\market}{Hurst coefficient analysis}
        \subidx{\market}{rate of change}

        This is, essentially, a statement of the range of values, in
        the increments of the rate of revenue returns, (per
        {\timescale},) that is to be expected over the time interval,
        $t_2 - t_1$,
        $R_v$,~\cite[pp. 178]{Feder},~\cite[pp. 172]{Cambel}:

        \begin{eqnarray}
            R_v\left(t_2 - t_1\right) & \propto & \left(t_2 - t_1\right)^{H}\\
                                      & \propto & \left(t_2 - t_1\right)^{\thurstlow}
            \label{\SETLABEL:R}
        \end{eqnarray}

        \subidx{\market}{rate of revenue returns, range}
        \subidx{Hurst coefficient}{analysis}
        \subidx{\market}{Hurst coefficient analysis}
        \subidx{\market}{rate of change}
        \subidx{Markov}{statistics}
        \subidx{statistics}{Markov}
        \noindent where $R$ is the range of values in the increments
        of the rate of revenue returns, (per {\timescale}.) A Hurst
        coefficient, $H$, that is much larger than $\frac{1}{2}$, (but
        less than 1,) implies a strongly non-Gaussian distribution in
        the increments of the rate of revenue returns, (per
        {\timescale},)~\cite[pp. 152, 194]{Feder}, and a Hurst
        coefficient near $\frac{1}{2}$ implies that the increments of
        the rate of revenue returns, (per {\timescale}) is
        characteristic of an independent
        process~\cite[pp. 195]{Feder}. Extreme caution should be
        exercised in using Markov statistics in any analysis where the
        Hurst coefficient is not
        $\frac{1}{2}$,~\cite[pp. 124]{Crownover},~\cite[pp. 106]{Peters:CAOITCM}.


        As a useful approximation, if $H$, is approximately
        $\frac{1}{2}$, Equation~\ref{\SETLABEL:R} reduces
        to,~\cite[pp. 129]{Schroeder}:

        \begin{eqnarray}
            R\left(t_2 - t_1\right) & \propto & (t_2 - t_1)^{\frac{1}{2}}\\
                                    & \propto & \sqrt{\left(t_2 - t_1\right)}
        \end{eqnarray}

        \subidx{\market}{rate of revenue returns, range}
        \subidx{\market}{rate of revenue returns, increase and decrease}
        \subidx{Hurst coefficient}{analysis}
        \subidx{\market}{Hurst coefficient analysis}
        \subidx{\market}{rate of change}
        \subidx{Markov}{statistics}
        \subidx{statistics}{Markov}

        In the case where the Hurst coefficient, $H$, is
        $\frac{1}{2}$, the range of values in the increments of the
        rate of revenue returns, (per {\timescale},) divided by the
        standard deviation of these values, $S$, can be anticipated to
        increase over time according to the following
        relation,~\cite[pp. 154]{Feder},~\cite[pp. 129]{Schroeder}:

        \begin{equation}
            \frac{R\left(t_2 - t_1\right)}{S} \propto \left(t_2 - t_1\right)^{\frac{1}{2}}
        \end{equation}

        \subidx{\market}{rate of revenue returns, range}
        \subidx{\market}{rate of revenue returns, increase and decrease}
        \subidx{Hurst coefficient}{analysis}
        \subidx{\market}{Hurst coefficient analysis}
        \subidx{\market}{rate of change}
        \noindent which is a useful conceptual approximation, since it
        involves only the square root function---if the range and the
        standard deviation of the increments of the rate of revenue
        returns, (per {\timescale},) are known, (and $H \approx
        \frac{1}{2}$,) then the expected change in $\frac{R}{S}$, will
        increase with the square root of time\footnote{To be precise,
        it is actually asymptotically proportional to
        $\tau^{\frac{1}{2}}$}.

        Another useful approximation when rescaling processes that are
        characterize by Brownian motion, (ie., when $H \approx
        \frac{1}{2}$,) is that:

        \begin{eqnarray}
            X\left(t\right) & \propto & \frac{X\left(rt\right)}{r^{H}}\\
                            & \propto & \frac{X\left(rt\right)}{r^{\thurstlow}}
        \end{eqnarray}

        \idx{Brownian motion}
        \idx{fractal}
        Where $X(t)$ is the process characterized by Brownian motion,
        and $r$ is a scaling factor,~\cite[pp. 494]{Peitgen}.

        \subidx{programs}{tslsq}
        \subidx{tslsq}{program}
        The program {\it tslsq}\/ was used on the H parameter data,
        presented in Figure~\ref{\SETLABEL:HP}, to provide a least
        squares approximation to the H parameter for the
        {\market}. The superimposed least squares approximation on the
        original H parameter data is presented.  By contrast, the H
        parameter, as derived by the methodology outlined
        in~\cite[pp. 249]{Crownover}, is {\thcalclow} for the near
        future, and {\thcalcall} for the distant future.

        \subidx{\market}{Hurst coefficient analysis}
        \subidx{Hurst coefficient}{analysis}
        \subidx{increments}{normalized}
        \subidx{normalized}{increments}
        \subidx{programs}{tshurst}
        \subidx{tshurst}{program}
        \subidx{\market}{H parameter analysis}
        \subidx{H parameter}{analysis}
        \subidx{programs}{tshcalc}
        \subidx{tshcalc}{program}
        Figures~\ref{\SETLABEL:HC} and~\ref{\SETLABEL:HP} represent
        Hurst coefficient and H parameter data that are derived from
        the normalized increments, shown in
        Figure~\ref{\SETLABEL:TF}. In this case, the data is
        considered a normalized derivative of the time series data
        presented in Figure~\ref{\SETLABEL:TF}, instead of a
        cumulative sum.  The program, {\it tshurst}\/, is described
        briefly in appendix~\ref{programs}, and the data for
        figures~\ref{\SETLABEL:THC} and~\ref{\SETLABEL:THP} was made
        using the -d option.

        \begin{figure}[ht]
            \begin{center}
                \begin{minipage}[t]{0.45\textwidth}
                    \epsfxsize=1.0\linewidth
                    \epsffile{\directory/data.tsfraction.tshurst-d.eps}
                    \caption[{\market}, traditional Hurst coefficient
                        data]{{\market}, traditional Hurst coefficient
                        data for the time series data shown in
                        Figure~\ref{\SETLABEL:TS}.  The slope of the
                        graph is the Hurst coefficient, and is
                        {\hurstlow} for the near term, and
                        {\hurstall} for the far term.}
                    \label{\SETLABEL:THC}
                \end{minipage}
                \hfill
                \begin{minipage}[t]{0.45\textwidth}
                    \epsfxsize=1.0\linewidth
                    \epsffile{\directory/data.tsfraction.tshcalc-d.eps}
                    \caption[{\market}, traditional H parameter
                        data]{{\market}, traditional H parameter data
                        for the time series data shown in
                        Figure~\ref{\SETLABEL:TS} The slope of the
                        graph is the H parameter, and is {\hcalclow}
                        for the near term, and {\hcalcall} for the
                        far term.}
                    \label{\SETLABEL:THP}
                \end{minipage}
            \end{center}
        \end{figure}

% Local Variables:
% TeX-parse-self: t
% TeX-auto-save: t
% TeX-master: "fractal.tex"
% End:


        \subsubsection{Observations on the Hurst Coefficient Analysis}

            Note that the H parameter data is not linear, and both the
            short term and long term predictability are better than
            the mid term predictability. This is also indicated by a
            Hurst coefficient of {\thurstall}, which is less than 0.5,
            and would tend to indicate that there is a predisposition
            to antipersistence, or ergodic, market behavior. What this
            means is that the system is mean reverting, and that a
            {\timescale} where rate of revenue returns increased, will
            have a predisposition to be followed by a {\timescale}
            where the rate of revenue returns decrease, and vice
            versa. See~\cite[pp. 64]{Peters:CAOITCM},~\cite[pp. 170]{Feder},~\cite[pp. 496]{Peitgen}~\cite[pp. 130]{Schroeder},~\cite[pp. 172]{Cambel}.

        %
% -----------------------------------------------------------------------------
%
% A license is hereby granted to reproduce this software source code and
% to create executable versions from this source code for personal,
% non-commercial use.  The copyright notice included with the software
% must be maintained in all copies produced.
%
% THIS PROGRAM IS PROVIDED "AS IS". THE AUTHOR PROVIDES NO WARRANTIES
% WHATSOEVER, EXPRESSED OR IMPLIED, INCLUDING WARRANTIES OF
% MERCHANTABILITY, TITLE, OR FITNESS FOR ANY PARTICULAR PURPOSE.  THE
% AUTHOR DOES NOT WARRANT THAT USE OF THIS PROGRAM DOES NOT INFRINGE THE
% INTELLECTUAL PROPERTY RIGHTS OF ANY THIRD PARTY IN ANY COUNTRY.
%
% Copyright (c) 1994-2006, John Conover, All Rights Reserved.
%
% Comments and/or bug reports should be addressed to:
%
%     john@email.johncon.com (John Conover)
%
% -----------------------------------------------------------------------------
%
% Revision: \RCSRevision \\
% Revision Time: \RCSTime UMT \\
% Revision Date: \RCSDate \\
% Revision Id: \RCSId \\
% Revision File: \RCSLog \\
\RCS $Revision: 0.0 $
\RCS $Date: 2006/01/20 04:38:13 $
\RCS $Id: fiscal.tex,v 0.0 2006/01/20 04:38:13 john Exp $
% $Log: fiscal.tex,v $
% Revision 0.0  2006/01/20 04:38:13  john
% Initial version
%
%
    \subsection{Fixed Increment Approximation for Fiscal Strategy}
        \label{\SETLABEL:FS}

        \subidx{\market}{fiscal strategy}
        \subidx{markets}{analysis}
        \subidx{analysis}{markets}
        \subidx{strategy}{fiscal}
        \subidx{fiscal}{strategy}
        The data in this section is presented in tabular form in
        Section~\ref{\SETLABELREF:LR}. This section derives various
        values based on the ``average'' of the normalized increments
        presented in Figure~\ref{\SETLABEL:TFA}. These values are an
        approximation to a, probably, complex process with a
        distribution shown in Figure~\ref{\SETLABEL:TF}. These values
        will be used in a fixed increment Brownian fractal analysis
        and simulation of the {\market}, and may, or may not, provide
        adequate accuracy for projections.

        For an organization operating in the {\market}, the fiscal
        strategy, commensurate with the aggregate environment, can be
        derived as follows~\cite[pp. 128, pp
        151]{Schroeder},~\cite[pp. 450]{Reza},~\cite[pp. 270]{Pierce}:
        \vspace{0.15in}

        \subsubsection{Logarithmic Returns}
            \label{\SETLABEL:LR}

            \subidx{logarithmic}{returns}
            \subidx{returns}{logarithmic}
            \subidx{\market}{logarithmic returns}
            The logarithmic returns can be calculated by various
            means. Four will be presented here, for comparison.

            \subidx{programs}{tsnormal}
            \subidx{tsnormal}{program}
            \subidx{logarithmic}{returns}
            \subidx{returns}{logarithmic}
            The logarithmic returns, in bits, $bits$, as computed from
            the mean, by the program {\it tsnormal}\/, which is
            described in Chapter~\ref{programs}, and is presented in
            Figure~\ref{\SETLABEL:TF}, and Equation~\ref{abits} from
            Section~\ref{ereturns} in Chapter~\ref{general}:

            \begin{equation}
                bits = \frac{\ln \left({\datafractionmean} + 1\right)}{\ln \left(2\right)} = \datafractionmeanbits
            \end{equation}

            \subidx{programs}{tslsq}
            \subidx{tslsq}{program}
            \subidx{logarithmic}{returns}
            \subidx{returns}{logarithmic}
            \noindent By comparison, the logarithmic returns, in bits,
            $bits$, as computed from the constant in the least squares
            approximation, using the program {\it tslsq}\/, which is briefly
            described in Chapter~\ref{programs}, as presented in
            Figure~\ref{\SETLABEL:TF}, and Equation~\ref{abits} from
            Section~\ref{ereturns} in Chapter~\ref{general}:

            \begin{equation}
                bits = \frac{\ln \left({\datafractionconstant} + 1\right)}{\ln \left(2\right)} = \datafractionconstantbits
            \end{equation}

            Note that if the mean is not constant in
            Figure~\ref{\SETLABEL:TF}, this method will not provide
            accurate results.

            \subidx{programs}{tslsq}
            \subidx{tslsq}{program}
            \subidx{logarithmic}{returns}
            \subidx{returns}{logarithmic}
            \noindent And by yet another comparison, using the program
            {\it tslsq}\/, which is briefly described in
            Chapter~\ref{programs}, with the -e -p options, to provide
            a formula for the least squares exponential fit to the
            time series data set presented in
            Figure~\ref{\SETLABEL:TS}:

            \begin{equation}
                bits = {\datatslsqepbits}
            \end{equation}

            \subidx{programs}{tslogreturns}
            \subidx{tslogreturns}{program}
            \subidx{logarithmic}{returns}
            \subidx{returns}{logarithmic}
            \noindent And finally, by comparison, from the
            {\it tslogreturns}\/ program, which is briefly described
            in Chapter~\ref{programs}, with the -p option, to provide
            a formula for the logarithmic returns of the time series
            data set presented in Figure~\ref{\SETLABEL:TS}:

            \begin{equation}
                bits = {\logreturns}
            \end{equation}

        \subsubsection{Calculation of Shannon Probability}
            \label{\SETLABEL:SP}

            \subidx{\market}{Shannon probability}
            Ideally, all of the values presented in
            Section~\ref{\SETLABEL:LR} would be equal. Using the
            logarithmic returns provided by the {\it tslogreturns}\/
            program, to be consistent
            with~\cite[pp. 81]{Peters:CAOITCM}

            \subidx{programs}{tslogreturns}
            \subidx{tslogreturns}{program}
            \begin{equation}
                2^{{\logreturns}t}
            \end{equation}

            \noindent therefore:
            \begin{equation}
                C\left(p\right) = {\logreturns}
            \end{equation}
            \subidx{programs}{tsshannon}
            \subidx{tsshannon}{program}
            \subidx{Shannon}{probability}
            \subidx{probability}{Shannon}
            \noindent and, {\it tsshannon}\/ {\logreturns} gives:
            \begin{equation}
                \label{\SETLABEL:F0}
                C\left({\shannonlogreturns}\right) = {\logreturns}
            \end{equation}
            \noindent therefore:
            \begin{eqnarray}
                2^{C\left({\shannonlogreturns}\right)} & = & 2^{\logreturns}\\
                                                       & = & {\twologreturns}\\
                                                       & = & {\twologreturnshundred}\%
            \end{eqnarray}
            \noindent and:
            \begin{eqnarray}
                2p - 1 & = & \left(2 \cdot {\shannonlogreturns}\right) - 1\\
                       & = & {\twopone}\\
                       \label{\SETLABEL:F1}
                       & = & {\twoponehundred}\%
            \end{eqnarray}

            \subidx{\market}{fiscal strategy}
            \subidx{markets}{analysis}
            \subidx{analysis}{markets}
            \subidx{strategy}{fiscal}
            \subidx{fiscal}{strategy}
            \subidx{\market}{fiscal strategy}
            \subidx{\market}{growth rate}
            Presuming the simplified assumptions outlined in
            Section~\ref{assumptions}, the ``typical'' organization
            operating in the {\market} executes a long term fiscal
            strategy, commensurate with the aggregate environment,
            that is to invest, every {\timescale}, in sufficient
            additional resources and infrastructure, to increase the
            manufacturing of goods and services by {\twoponehundred}\%
            of its rate of revenue returns, (per {\timescale}.) As a
            conceptual model, the remaining {\hundredtwoponehundred}\%
            will be held in ``reserve'' with a
            {\shannonlogreturnshundred}\% chance of making twice the
            {\twoponehundred}\% back, (and a
            {\hundredshannonlogreturnshundred}\% chance of making
            0.0,) in one {\timescale}, on the average, for an average
            growth in its rate of revenue returns, (per {\timescale},)
            of {\twologreturnshundred}\%, or a doubling of its rate of
            revenue returns, (per {\timescale},) in
            {\oneoverlogreturns} {\timescale}s.

        \subsubsection{Example Fixed Increment Approximation Fiscal Strategies}

            \subidx{\market}{fiscal strategy}
            \subidx{markets}{analysis}
            \subidx{analysis}{markets}
            \subidx{strategy}{fiscal}
            \subidx{fiscal}{strategy}
            \subidx{\market}{fiscal strategy}
            \subidx{\market}{growth rate}
            \subidx{\market}{management metric}
            \idx{management metric}
            A possible metric on the effectiveness of long term fiscal
            management could possibly be that if an investment of
            {\twoponehundred}\% per {\timescale} of the rate of
            revenue returns, (per {\timescale},) is made in resources
            and infrastructure, then the rate of revenue returns would
            be expected to increase by {\twologreturnshundred}\%, per
            {\timescale}, on average.

            Note that the metrics presented in this section are
            representative of the {\market} as an aggregate whole, and
            may or may not be accurate representations for any
            particular participant in the environment. Of interest to
            the participants in the environment would be a similar
            analysis of each product or service rendered in the
            marketplace.

            \subidx{\market}{fiscal strategy}
            \subidx{markets}{analysis}
            \subidx{analysis}{markets}
            \subidx{strategy}{fiscal}
            \subidx{fiscal}{strategy}
            \subidx{\market}{fiscal strategy}
            As a simple illustrative example, a company operating in
            this environment might obtain a credit line from a bank
            that is equal to {\twoponehundred}\% of its rate of
            revenue returns, (per {\timescale},) to finance additional
            operations. In this simple scenario, the company would use
            its revenue base as collateral for the loan. Some
            {\timescale}s, depending on the {\market}'s environment,
            the company's rate of revenue returns exceeds what was
            borrowed from the bank, and the loan is repaid in
            full. Other {\timescale}s, the company must default, and
            the bank seizes a portion of the company's revenue base to
            pay the delinquent loan. However, on the average, the
            company will expand its rate of revenue returns at
            {\twologreturnshundred}\% per {\timescale}.

            \subidx{\market}{fiscal strategy}
            \subidx{markets}{analysis}
            \subidx{analysis}{markets}
            \subidx{strategy}{fiscal}
            \subidx{fiscal}{strategy}
            \subidx{\market}{fiscal strategy}
            As another simple example, a company re-invests
            {\twoponehundred}\% of its rate of revenue returns, (per
            {\timescale},) in development, marketing, sales, and
            distribution of new products.  Although some products will
            be successful and the return on the investment will exceed
            the {\twoponehundred}\% per {\timescale} investment,
            others will not. However, on the average, the company will
            expand it gross rate of revenue returns at
            {\twologreturnshundred}\% per {\timescale}.

            \subidx{\market}{fiscal strategy}
            \subidx{markets}{analysis}
            \subidx{analysis}{markets}
            \subidx{strategy}{fiscal}
            \subidx{fiscal}{strategy}
            \subidx{\market}{fiscal strategy}
            \subidx{\market}{product portfolio}
            \subidx{\market}{product diversity}
            \subidx{\market}{product mix}
            \subidx{\market}{optimum number of products}
            \idx{product portfolio}
            \idx{product diversity}
            \idx{optimum number of products}
            \idx{product mix}

            As an example of ``product portfolio'' management, suppose
            a company re-invests {\twoponehundred}\% of its rate of
            revenue returns, (per {\timescale},) in development,
            marketing, sales, and distribution of new products.
            Further suppose that the company has two products, and a
            fractal analysis of the individual product rate of revenue
            return time series indicates that one product has a
            Shannon probability of 0.65, and the other has a Shannon
            probability of 0.55. Then the percentage of re-investment
            in the first product would be $(2 \cdot 0.65 - 1) \cdot
            {\twoponehundred}$, percent of the rate of revenue
            returns, and $(2 \cdot 0.55 - 1) \cdot {\twoponehundred}$
            percent for the second product, implying that the company
            should diversify its product line\footnote{The astute
            reader would note that the linear addition was used to add
            the contribution to development of each product. This is a
            ``near term'' interpretation. Actually, in general, the
            method used should be a root mean square process,
            dependent on the Hurst Coefficient, $H$, where
            $P_{total}^H = P_1^H + P_2^H + \cdots$, where $P_n$ is the
            contribution to each individual product. For a Brownian
            motion, or random walk type of fractal the Hurst
            Coefficient is a function of time into the future. For the
            ``near term,'' the Hurst coefficient is very near unity,
            meaning the summation process is linear. For the ``long
            term,'' $H \approx 0.5$, or a standard root mean square
            summation process should be used. If $H$ is $0.5$ then the
            market is termed a Brownian motion, or random walk
            process. If it is larger than 0.5, it is termed fractional
            Brownian motion process. For a random walk process, ``near
            term'' and ``far term'' are quantitatively differentiated
            on the Hurst Coefficient graph where $1 - \ln (t) = 0.5
            \cdot \ln (t)$, or when $\ln (t) = 2$, or $t =
            7.389\ldots$ See~\cite[pp. 67, 83-84]{Peters:CAOITCM}
            and~\cite[pp. 129, 159]{Schroeder} for particulars on the
            implications of the Hurst Coefficient and root mean square
            summation issues.}.  Note that this is a ``bet hedging''
            metric methodology, and assumes that the products have
            uncorrelated revenue return rates. If this re-investment
            methodology is not feasible, perhaps for strategic
            financial reasons, then the re-investment in both products
            should total the ${\twoponehundred}$\%, and the investment
            in each product should be made at a ratio of $\frac{(2
            \cdot 0.65 - 1)}{(2 \cdot 0.55 - 1)} = 3 : 1$,
            respectively. Note that this ``bet hedging'' can be used
            to define the optimal number of products that can be
            supported on the rate of revenue returns. If it assumed
            that all products are ``typical'' for the {\market}, as a
            standard bench mark, then the optimal number will be
            $\frac{1}{{\twopone}}$. Note that this is a
            ``theoretical'' value, since not all products are
            ``typical,'' and there may be strategic reasons, for
            example product leveraging, that may increase the number
            of products above the optimum. However, most of the
            revenue should come from the optimal number of products,
            since having more products will decrease the amount of the
            potential investment in each product, and having less than
            the optimum number of products will increase the risk that
            many of the products could suffer a ``down market''
            concurrently, impacting the rate of revenue returns.  As
            another interesting interpretation of the optimal
            ``hedging of bets,'' in product portfolio strategy, and
            considering the graph of the normalized increments
            presented in Figure~\ref{\SETLABEL:TF}, if the
            organization is running optimally, then these products
            will generate, at least in principle, one standard
            deviation, approximately $0.8413 = 84.13$\% of the future
            growth in rate of revenue returns. Naturally, these are
            approximations, and the values are an approximation to a,
            probably, complex process, and appropriate scrutiny should
            be exercised before making specific projections.  As yet
            another example of ``product portfolio'' management,
            consider the issue of product mix. In this interpretation,
            {\twoponehundred}\% of the product manufactured should be
            ``proprietary,'' while the rest is ``industry standard.''
            As yet another possibility, {\twoponehundred}\% of the
            product manufactured should be predatory into new markets,
            and the remainder in markets that are ``traditional'' for
            the company.

% Local Variables:
% TeX-parse-self: t
% TeX-auto-save: t
% TeX-master: "fractal.tex"
% End:


        \subsubsection{Observations on the Fixed Increment Approximation for Fiscal Strategy}

            A re-investment of {\twoponehundred} of the rate of
            revenue returns per {\timescale} does not seem
            inconsistent with the industry averages, since it includes
            investments in research and development, additional
            manufacturing infrastructure, advertising,
            etc. Additionally, a product mix of {\twoponehundred}\%
            ``proprietary'' and the remainder ``industry standard''
            products seems consistent with the industry analyst
            ``20/80'' rule. The value of one standard deviation,
            $84.13$\%, of the revenue return rate being generated by
            $\frac{1}{{\twopone}}$ products seems consistent with the
            industry, also.

        %
% -----------------------------------------------------------------------------
%
% A license is hereby granted to reproduce this software source code and
% to create executable versions from this source code for personal,
% non-commercial use.  The copyright notice included with the software
% must be maintained in all copies produced.
%
% THIS PROGRAM IS PROVIDED "AS IS". THE AUTHOR PROVIDES NO WARRANTIES
% WHATSOEVER, EXPRESSED OR IMPLIED, INCLUDING WARRANTIES OF
% MERCHANTABILITY, TITLE, OR FITNESS FOR ANY PARTICULAR PURPOSE.  THE
% AUTHOR DOES NOT WARRANT THAT USE OF THIS PROGRAM DOES NOT INFRINGE THE
% INTELLECTUAL PROPERTY RIGHTS OF ANY THIRD PARTY IN ANY COUNTRY.
%
% Copyright (c) 1994-2006, John Conover, All Rights Reserved.
%
% Comments and/or bug reports should be addressed to:
%
%     john@email.johncon.com (John Conover)
%
% -----------------------------------------------------------------------------
%
% Revision: \RCSRevision \\
% Revision Time: \RCSTime UMT \\
% Revision Date: \RCSDate \\
% Revision Id: \RCSId \\
% Revision File: \RCSLog \\
\RCS $Revision: 0.0 $
\RCS $Date: 2006/01/20 04:38:13 $
\RCS $Id: companies.tex,v 0.0 2006/01/20 04:38:13 john Exp $
% $Log: companies.tex,v $
% Revision 0.0  2006/01/20 04:38:13  john
% Initial version
%
%
    \subsection{Number of Companies}
        \label{\SETLABEL:QNC}

        \subidx{\market}{number of companies}
        \subidx{number of companies}{analysis}
        \subidx{analysis}{number of companies}
        \subidx{Shannon}{probability}
        \subidx{probability}{Shannon}
        This section evaluates the approximate, or ``average,'' number
        of companies in the {\market}, and uses the method outlined in
        Chapter~\ref{general}, Section~\ref{aftsma}. Since the
        average, $avg_{ind}$, and the root mean square, $rms_{ind}$,
        of the normalized increments of the {\market} time series is
        \datafractionmean, and \datafractionrms respectively, the
        number of companies participating in the market can be
        calculated by Equation~\ref{ncompanies} to be {\ncompanies}.

        If this value seems consistent number of companies in the
        {\market}, within the assumptions outlined in
        Chapter~\ref{general}, Section~\ref{aftsma}, then it would
        seem that there is some circumstantial or indirect evidence
        that the companies participating in the {\market} are
        operating optimally, and the ``average'' Shannon probability,
        $P$ for each participating company would be, using
        Equation~\ref{pncompanies}, {\pncompanies}, which would be the
        value which should be used in Section~\ref{\SETLABEL:FS} for
        each participating company if market expansion was to be
        consistent with the rest of the industry. However, if the
        Shannon probability derived in Section~\ref{\SETLABEL:FS} is
        greater than the average Shannon probability for the companies
        participating in the {\market}, as derived in this section,
        then the market would, possibly, be exploitable with the
        fiscal strategy outlined in Section~\ref{\SETLABEL:FS}. The
        maximum exploitability for the {\market} is derived in
        Section~\ref{\SETLABEL:MAXSHANNON}, but it is probably of
        doubtful practicality.

        Note that these optimizations would maximize a company's
        market growth. Since there are probably many companies
        competing in the market place, this would not necessarily
        maximize a company's P\&L, as described in
        Chapter~\ref{general}, Section~\ref{ompl}. The Shannon
        probability that maximizes market share in the {\market} is
        \pncompanies, with several alternative solutions listed in the
        previous paragraph. However, these should be contrasted to the
        Shannon probability that maximizes a company's P\&L which is
        \avgrms~in the {\market}. In all cases, the fraction of the
        P\&L that should be ``wagered'' on the future, $f$, should be:

        \begin{equation}
            f = 2P - 1
        \end{equation}

        \noindent where $P$ is the particular Shannon probability
        chosen optimize a particular fiscal strategy. Interestingly,
        the measured Shannon probability of the {\market} would tend
        to indicate that the companies participating in the market
        have chosen a fiscal strategy that optimizes market growth, as
        opposed to capital growth.

        \subidx{\market}{increasing returns}
        \subidx{economic increasing returns}{\market}
        As interesting interpretation of these exploitive issues,
        since all three fiscal strategies will result in exponential
        market growth for every company participating in the market,
        is that they may represent, perhaps, an example of
        ``increasing returns.''

% Local Variables:
% TeX-parse-self: t
% TeX-auto-save: t
% TeX-master: "fractal.tex"
% End:


        %
% -----------------------------------------------------------------------------
%
% A license is hereby granted to reproduce this software source code and
% to create executable versions from this source code for personal,
% non-commercial use.  The copyright notice included with the software
% must be maintained in all copies produced.
%
% THIS PROGRAM IS PROVIDED "AS IS". THE AUTHOR PROVIDES NO WARRANTIES
% WHATSOEVER, EXPRESSED OR IMPLIED, INCLUDING WARRANTIES OF
% MERCHANTABILITY, TITLE, OR FITNESS FOR ANY PARTICULAR PURPOSE.  THE
% AUTHOR DOES NOT WARRANT THAT USE OF THIS PROGRAM DOES NOT INFRINGE THE
% INTELLECTUAL PROPERTY RIGHTS OF ANY THIRD PARTY IN ANY COUNTRY.
%
% Copyright (c) 1994-2006, John Conover, All Rights Reserved.
%
% Comments and/or bug reports should be addressed to:
%
%     john@email.johncon.com (John Conover)
%
% -----------------------------------------------------------------------------
%
% Revision: \RCSRevision \\
% Revision Time: \RCSTime UMT \\
% Revision Date: \RCSDate \\
% Revision Id: \RCSId \\
% Revision File: \RCSLog \\
\RCS $Revision: 0.0 $
\RCS $Date: 2006/01/20 04:38:13 $
\RCS $Id: operations.tex,v 0.0 2006/01/20 04:38:13 john Exp $
% $Log: operations.tex,v $
% Revision 0.0  2006/01/20 04:38:13  john
% Initial version
%
%
    \subsection{Fixed Increment Approximation for Operational Strategy}
        \label{\SETLABEL:OPS}.

        This section derives various values based on the ``average''
        of the normalized increments presented in
        Figure~\ref{\SETLABEL:TFA}. These values are an approximation
        to a, probably, complex process with a distribution shown in
        Figure~\ref{\SETLABEL:TF}. These values will be used in a
        fixed increment Brownian fractal analysis and simulation of
        the {\market}, and may, or may not, provide adequate accuracy
        for projections.

        \subidx{\market}{fiscal strategy}
        \subidx{\market}{Shannon probability}
        \subidx{strategy}{fiscal}
        \subidx{fiscal}{strategy}
        \subidx{Shannon}{probability}
        \subidx{probability}{Shannon}
        It should be noted that the analysis of fiscal strategy,
        presented in Section~\ref{\SETLABEL:FS}, is derived from the
        {\market} metrics and may, or may not, be maximally
        optimal. For the optimal fiscal strategy, which may be
        exploitable, see Section~\ref{\SETLABEL:MAXSHANNON}.

        \subidx{strategy}{exploitable}
        \subidx{exploitable}{strategy}
        \subidx{\market}{windows of opportunity}
        \idx{windows of opportunity}
        \subidx{decision}{obsolete}
        \subidx{obsolete}{decision}
        \subidx{decision}{timeliness}
        \subidx{timeliness}{decision}
        \subidx{rate of revenue returns}{forecast}
        \subidx{forecast}{rate of revenue returns}
        An additional exploitable strategy may be time itself.
        Equations~\ref{\SETLABEL:V},~\ref{\SETLABEL:R},
        and,~\ref{\SETLABEL:MA}, are, essentially, metrics on how fast
        a decision, which is based on information concerning the
        current status of the {\market}, becomes obsolete. Obviously,
        how long a decision is expected to remain relevant should be
        addressed as an operational necessity in strategic planning
        and project management. Figures~\ref{\SETLABEL:FN},
        and,~\ref{\SETLABEL:FF} compare methods of approximation of
        the ``forecastability'' of rate of revenue returns in the
        {\market} for the near term and far
        term~\cite[pp. 83-84]{Peters:CAOITCM}, respectively. As a
        general rule, caution must be exercised when making decisions
        that will span a time interval larger than the time interval
        where the ``forecastability'' of rate of revenue returns drops
        below 50\%. Beyond this time interval, the chances increase
        that the competitive and market forces will alter the market
        environment in a possibly detrimental unanticipated
        fashion. Obviously, there is significant advantage in
        ``timeliness'' of development, manufacturing, and distribution
        of products and services that are consistent with this
        temporal agenda. Automation of these processes, if executed
        consistently with this agenda, should be considered a
        competitive advantage.

        \subidx{strategy}{exploitable}
        \subidx{exploitable}{strategy}
        \subidx{rate of revenue returns}{forecast}
        \subidx{forecast}{rate of revenue returns}
        \idx{product life cycle}
        \idx{life cycle, product}
        In some sense, this temporal agenda defines the ``average''
        product or service life cycle in the {\market}. When the
        ``forecastability'' of rate of revenue returns drops below
        50\%, there is an even chance that the rate of revenue returns
        for the product or service will change in a detrimental
        fashion. If it is assumed that a product or service life cycle
        consists of a ramp up, a maintenence interval, and a ramp
        down, then, if all three life cycle intervals are equal, the
        product life cycle will be, approximately, three times the
        time interval where the ``forecastability'' of rate of revenue
        returns drops below 50\%. Although probably not an accurate
        prediction of product or service life cycle, the technique may
        be used as a conceptual approximation to the dynamics of
        ``market windows.\footnote{For example, consider the market
        for table salt. Since it has inelastic supply and demand
        curves, and is a necessary requirement for life, it would be
        expected that the Hurst coefficient would be very near
        unity---ignoring competitive pressures in the market. The
        predictability of the table salt market would, therefore, be
        expected to be relatively good, over time.}''  The conceptual
        approximation will probably predict a ``conservative'' or
        ``pessimistic'' value in relation to actual markets.

        \begin{figure}[ht]
            \begin{center}
                \begin{minipage}[t]{0.45\textwidth}
                    \epsfxsize=1.0\linewidth
                    \epsffile{\directory/datahurstlownear.eps}
                    \caption[{\market}, ``forecastability'' of near
                        term rate of revenue returns]{{\market},
                        ``forecastability'' of near term rate of
                        revenue returns. Although the error function
                        is the most accurate, for the near term,
                        $H^{t} = \thurstlow^{t}$ may be used as a
                        reliable metric of ``forecastability'' of the
                        rate of revenue returns.}
                    \label{\SETLABEL:FN}
                \end{minipage}
                \hfill
                \begin{minipage}[t]{0.45\textwidth}
                    \epsfxsize=1.0\linewidth
                    \epsffile{\directory/datahurstlowfar.eps}
                    \caption[{\market}, ``forecastability'' of far
                        term rate of revenue returns]{{\market},
                        ``forecastability'' of far term rate of
                        revenue returns. Although the error function
                        is the most accurate, for the far term,
                        $\frac{1}{\sqrt{t}}$ may be used as a reliable
                        metric of ``forecastability'' of the rate of
                        revenue returns.}
                    \label{\SETLABEL:FF}
                \end{minipage}
            \end{center}
        \end{figure}

        \idx{operations research}
        As an interesting interpretation of the data presented in
        Figure~\ref{\SETLABEL:FN}, there may be, perhaps, some
        applicability to such operational agendas as inventory
        control. Maintaining too little inventory, obviously, will
        create a situation where the organization can not exploit
        market expansion, and maintaining too much inventory,
        likewise, would over extend the company, creating unnecessary
        losses when the market contracts. The company should maintain
        inventory levels that do not exceed, from
        Equation~\ref{\SETLABEL:MA}, ${\thurstlow}^{n} = 0.5$
        {\timescale}s of operations. Since the optimal amount of
        inventory and, from Equation~\ref{\SETLABEL:V}, the variance
        of change in the rate of revenue returns in the future can be
        calculated, there may, perhaps, be some applicability to a
        forecasting methodology that can be incorporated into other
        areas of operations research, for example the linear algebras
        using simplex methodologies for optimization of manufacturing
        processes. Traditionally, these forecasts are made by the
        sales department, and are subject to various subjective
        biases.

% Local Variables:
% TeX-parse-self: t
% TeX-auto-save: t
% TeX-master: "fractal.tex"
% End:


        \subsubsection{Observations on the Fixed Increment Approximation for Operational Strategy}

            As an interesting interpretation of
            Figure~\ref{\SETLABEL:FF}, and evaluating the
            approximation $\frac{1}{\sqrt{t}}$ at 60 months gives a
            probability that the market will still have the same
            agenda of about $0.12909945$, or about 1 in 8. This is
            commensurate with numbers from the venture
            community\footnote{For example, see ``IEEE Engineering
            Management Review,'' Volume 23 Number 3, Fall 1995,
            pp. 83}. Of course new venture backed companies fail for
            many reasons, but market appropriateness to product
            portfolio 60 months in the future may be a major
            contributor. Additionally, the success rate of development
            projects of 8 month duration, which have a market success
            rate of about 1 in 3, seems consistent with
            $\frac{1}{\sqrt{3}} = 0.353553391$. Naturally, projects
            fail in the market for many reasons, but market
            appropriateness, in a dynamic market environment may be a
            major contributor to failure.

            As mentioned in Section~\ref{\SETLABEL:H},
            Equation~\ref{\SETLABEL:MA}, and the preceeding section,
            approximately 3 times the value where ${\thurstlow}^{n} =
            0.5$ could be interpreted as an approximation to the
            ``average'' product life cycle. This seems consistent with
            the 6 to 12 month life cycles quoted by many industry
            analyst. In addition, maintaining inventory levels that do
            not exceed the anticipated requirements of
            $\frac{\ln{0.5}}{\ln{\thurstlow}}$ many {\timescale}s
            seems consistent with the author's experience in the
            industry.

        %
% -----------------------------------------------------------------------------
%
% A license is hereby granted to reproduce this software source code and
% to create executable versions from this source code for personal,
% non-commercial use.  The copyright notice included with the software
% must be maintained in all copies produced.
%
% THIS PROGRAM IS PROVIDED "AS IS". THE AUTHOR PROVIDES NO WARRANTIES
% WHATSOEVER, EXPRESSED OR IMPLIED, INCLUDING WARRANTIES OF
% MERCHANTABILITY, TITLE, OR FITNESS FOR ANY PARTICULAR PURPOSE.  THE
% AUTHOR DOES NOT WARRANT THAT USE OF THIS PROGRAM DOES NOT INFRINGE THE
% INTELLECTUAL PROPERTY RIGHTS OF ANY THIRD PARTY IN ANY COUNTRY.
%
% Copyright (c) 1994-2006, John Conover, All Rights Reserved.
%
% Comments and/or bug reports should be addressed to:
%
%     john@email.johncon.com (John Conover)
%
% -----------------------------------------------------------------------------
%
% Revision: \RCSRevision \\
% Revision Time: \RCSTime UMT \\
% Revision Date: \RCSDate \\
% Revision Id: \RCSId \\
% Revision File: \RCSLog \\
\RCS $Revision: 0.0 $
\RCS $Date: 2006/01/20 04:38:13 $
\RCS $Id: simulation.tex,v 0.0 2006/01/20 04:38:13 john Exp $
% $Log: simulation.tex,v $
% Revision 0.0  2006/01/20 04:38:13  john
% Initial version
%
%
    \subsection{Simulation of Fixed Increment Approximation for Fiscal Strategy}
        \label{\SETLABEL:TSUNFAIRBROWNIAN}

        \subidx{\market}{market simulation}
        The data in this section is presented in tabular form in
        Section~\ref{\SETLABELREF:SIM}.
        Figure~\ref{\SETLABEL:TSUNFAIRBROWNIAN0} represents a
        constructional simulation of the time series data presented in
        Figure~\ref{\SETLABEL:TS}. The program {\it
        tsunfairbrownian}\/, which is briefly described in
        appendix~\ref{programs}, was used in the reconstruction. The
        reconstructed data is superimposed on the original time series
        data.  The program, {\it tsunfairbrownian}\/, essentially,
        constructs the new time series as a Brownian fractal with
        fixed increments---the value of the fixed increment is derived
        from the root mean square average of the normalized increments
        presented in Figure~\ref{\SETLABEL:TF}. The ``quality'' of
        such a reconstruction should be subject to adequate scepticism
        and scrutiny since, in all probability, the normalized
        increments presented in Figure~\ref{\SETLABEL:TF} represent a
        relatively complex process, that may not be ``modeled'' with
        such a simple methodology.

        As a further comparison of the the constructional simulation
        with the original time series data,
        Figure~\ref{\SETLABEL:TSUNFAIRBROWNIAN1} presents a normalized
        histogram of the normalized increments of the reconstructed
        time series, superimposed on the normalized histogram
        presented in Figure~\ref{\SETLABEL:NH}.

        \subidx{\market}{fiscal strategy, simulation}
        \subidx{markets}{simulation}
        \subidx{simulation}{markets}
        \subidx{strategy}{fiscal, simulation}
        \subidx{fiscal}{strategy, simulation}
        \subidx{programs}{tsunfairbrownian}
        \subidx{tsunfairbrownian}{program}
        \begin{figure}[ht]
            \begin{center}
                \begin{minipage}[t]{0.45\textwidth}
                    \epsfxsize=1.0\linewidth
                    \epsffile{\directory/tsunfairbrownian-f.eps}
                    \caption[{\market}, Time series data, empirical and
                        simulated]{{\market}, Time series data, empirical
                        and simulated, using the program {\it tsunfairbrownian}\/
                        with f = {\datafractionrms}. This data is
                        superimposed on the data presented in
                        Figure~\ref{\SETLABEL:TS}.}
                    \label{\SETLABEL:TSUNFAIRBROWNIAN0}
                \end{minipage}
                \hfill
                \begin{minipage}[t]{0.45\textwidth}
                    \epsfxsize=1.0\linewidth
                    \epsffile{\directory/tsunfairbrownian-f.tsfraction.tsnormal-s30.eps}
                    \caption[{\market}, normalized histogram,
                        empirical and simulated]{{\market}, normalized
                        histogram of the normalized increments of the
                        time series data shown in
                        Figure~\ref{\SETLABEL:TSUNFAIRBROWNIAN0},
                        empirical and simulated.  The empirical data
                        has a mean of {\datafractionmean}, with a
                        standard deviation of {\datafractionstddev}.
                        By comparison, the simulated data has a mean
                        of {\tsunfairbrownianfractionmean} with a
                        standard deviation of
                        {\tsunfairbrownianfractionstddev}. This data
                        is superimposed on the data presented in
                        Figure~\ref{\SETLABEL:NH}. The area under the
                        four curves is identical.}
                    \label{\SETLABEL:TSUNFAIRBROWNIAN1}
                \end{minipage}
            \end{center}
        \end{figure}

% Local Variables:
% TeX-parse-self: t
% TeX-auto-save: t
% TeX-master: "fractal.tex"
% End:


        %
% -----------------------------------------------------------------------------
%
% A license is hereby granted to reproduce this software source code and
% to create executable versions from this source code for personal,
% non-commercial use.  The copyright notice included with the software
% must be maintained in all copies produced.
%
% THIS PROGRAM IS PROVIDED "AS IS". THE AUTHOR PROVIDES NO WARRANTIES
% WHATSOEVER, EXPRESSED OR IMPLIED, INCLUDING WARRANTIES OF
% MERCHANTABILITY, TITLE, OR FITNESS FOR ANY PARTICULAR PURPOSE.  THE
% AUTHOR DOES NOT WARRANT THAT USE OF THIS PROGRAM DOES NOT INFRINGE THE
% INTELLECTUAL PROPERTY RIGHTS OF ANY THIRD PARTY IN ANY COUNTRY.
%
% Copyright (c) 1994-2006, John Conover, All Rights Reserved.
%
% Comments and/or bug reports should be addressed to:
%
%     john@email.johncon.com (John Conover)
%
% -----------------------------------------------------------------------------
%
% Revision: \RCSRevision \\
% Revision Time: \RCSTime UMT \\
% Revision Date: \RCSDate \\
% Revision Id: \RCSId \\
% Revision File: \RCSLog \\
\RCS $Revision: 0.0 $
\RCS $Date: 2006/01/20 04:38:13 $
\RCS $Id: maximum.tex,v 0.0 2006/01/20 04:38:13 john Exp $
% $Log: maximum.tex,v $
% Revision 0.0  2006/01/20 04:38:13  john
% Initial version
%
%
    \subsection{Simulation of Fixed Increment Approximation for Optimally Maximal Fiscal Strategy}
        \label{\SETLABEL:MAXSHANNON}
        \subidx{\market}{fiscal strategy, simulation}
        \subidx{\market}{maximum Shannon probability}
        \subidx{markets}{simulation}
        \subidx{simulation}{markets}
        \subidx{strategy}{optimum fiscal, simulation}
        \subidx{fiscal}{optimum strategy, simulation}
        \subidx{programs}{tsunfairbrownian}
        \subidx{tsunfairbrownian}{program}
        \subidx{Shannon}{probability}
        \subidx{probability}{Shannon}

        \subidx{strategy}{exploitable}
        \subidx{exploitable}{strategy}
        \subidx{programs}{tsshannonmax}
        \subidx{tsshannonmax}{program}
        \subidx{programs}{tsunfairbrownian}
        \subidx{tsunfairbrownian}{program}
        \subidx{strategy}{fiscal}
        \subidx{fiscal}{strategy}
        The data in this section is presented in tabular form in
        Section~\ref{\SETLABELREF:MAXSHANNON}. One of the issues of
        analysis, as mentioned in Section~\ref{\SETLABEL:OPS}, is to
        determine the maximum Shannon probability for the time series
        presented in Figure~\ref{\SETLABEL:TS}. Potentially, this
        could be exploited with an aggressive fiscal
        strategy. Figure~\ref{\SETLABEL:SHANNONMAX0} is a graph of the
        output of the {\it tsshannonmax}\/ program, which is described
        briefly in appendix~\ref{programs}. The maximum of this
        function is the maximum Shannon probability for the time
        series data presented in Figure~\ref{\SETLABEL:TS}.
        Figure~\ref{\SETLABEL:SHANNONMAX1} was constructed using {\it
        tsunfairbrownian}\/ program, which is also described in
        appendix~\ref{programs}, with the maximum Shannon probability,
        and the time series data presented in
        Figure~\ref{\SETLABEL:TS}. This represents a ``what if'' the
        investment strategy was changed from a Shannon probability of
        {\shannonlogreturns}, as derived in Section~\ref{\SETLABEL:SP}
        to {\shannonmax}. This process, essentially, extracts the
        random statistical data from the time series presented in
        Figure~\ref{\SETLABEL:TS}, and constructs a new time series,
        using the random statistical data, with a different investment
        strategy.  The program, {\it tsunfairbrownian}\/, essentially,
        constructs the new time series as a Brownian fractal with
        fixed increments.  The ``quality'' of such a reconstruction
        should be subject to adequate scepticism and scrutiny since,
        in all probability, the increments in the original data
        represent a relatively complex process, that may not be
        ``modeled'' with such a simple methodology.

        \begin{figure}[ht]
            \begin{center}
                \begin{minipage}[t]{0.45\textwidth}
                    \epsfxsize=1.0\linewidth
                    \epsffile{\directory/data.tsshannonmax.eps}
                    \caption[{\market}, maximum rate of revenue
                        returns] {{\market}, maximum rate of revenue
                        returns, per {\timescale}, vs. Shannon
                        probability. The maximum rate of revenue
                        returns, per {\timescale}, occurs at a Shannon
                        probability of {\shannonmax}.}
                    \label{\SETLABEL:SHANNONMAX0}
                \end{minipage}
                \hfill
                \begin{minipage}[t]{0.45\textwidth}
                    \epsfxsize=1.0\linewidth
                    \epsffile{\directory/data.tsshannonmax-p.tsunfairbrownian-p.eps}
                    \caption[{\market}, maximum rate of revenue
                        returns] {{\market}, maximum rate of revenue
                        returns, per {\timescale}, at a Shannon
                        probability, of {\shannonmax}, corresponding
                        to a ``wager'' fraction of {\twoponemax}.}
                    \label{\SETLABEL:SHANNONMAX1}
                \end{minipage}
            \end{center}
        \end{figure}

        \subidx{fractional}{Brownian motion}
        \subidx{Brownian motion}{fractional}
        \subidx{Shannon}{probability}
        \subidx{probability}{Shannon}
        \subidx{programs}{tsshannonmax}
        \subidx{tsshannonmax}{program}
        If it is assumed that the time series data set, presented in
        Figure~\ref{\SETLABEL:TS}, constitutes classical Brownian
        motion, then the Shannon probability can be calculated by
        counting the total number of {\timescale}s that the {\market}
        movement was positive, and dividing by the total number of
        {timescale}s represented in the time series. This quotient is
        {\pmax}, as compared with the predicted value from the program
        {\it tsshannonmax}\/ of {\shannonmax}.

% Local Variables:
% TeX-parse-self: t
% TeX-auto-save: t
% TeX-master: "fractal.tex"
% End:


        \subsubsection{Observations on the Simulation of Fixed Increment Approximation for Optimally Maximal Fiscal Strategy}

            Note that these simulations are base on a very, perhaps
            overly, simplified model. For example, from
            Section~\ref{\SETLABEL:TSA}, Figure~\ref{\SETLABEL:NH}, it
            would appear that the {\market}'s normalized increments
            are characterized by fractional Brownian motion---but the
            simulations used classical Brownian motion as the
            model. One consequence of this is that a re-investment
            strategy that is to ``wager'' a fraction of {\twoponemax}
            of the rate of returns every {\timescale} is overly
            aggressive, since in the classical Brownian scenario, the
            maximum loss, in any {\timescale}, was no more that what
            was ``wagered.'' However, in the fractional Brownian
            scenario, much more can be lost. From
            Equation~\ref{fopt2},

            \begin{equation}
                \frac{avg}{rms^2} = \frac{f_{opt}}{rms} = K
            \end{equation}

            \noindent where, under the optimum classical Brownian
            scenario, $K$ is unity, or $avg = rms^2$. Notice that,
            since $f = rms$, whether the scenario is optimal or not,
            that the operational ``wager'' fraction, from
            Figure~\ref{\SETLABEL:TF} of {\datafractionrms}, vs.\ an
            ``theoretical optimal'' value of {\twoponemax} seems
            overly conservative. Additionally, notice that, at least
            in principle, the chance of failure in the fractional
            Brownian scenario, which is more accurate, would
            correspond to 1 standard deviation, or about 15.865\% per
            {\timescale}, which is unacceptably high. However, it is
            not clear why the {\market} is running at a value of
            {\datafractionrms}, which seems very
            conservative. However, a re-investment strategy of
            {\datafractionrms} per {\timescale} does not seem
            inconsistent with a failure rate, on the Fortune 500 list,
            which it is inferred that the {\market} is similar to, of
            about 50\% in ten years, which corresponds to $(1 -
            p_f)^{120} \approx 0.5$, or $p_f$, the probability of
            failure, is $0.005759576$, which is, approximately, 2.5
            standard deviations, meaning that to be consistent with
            the large companies in the Fortune 500, the re-investment
            rate should be, approximately, $\frac{\twoponemax}{2.5}$,
            compared with an operational value, from
            Figure~\ref{\SETLABEL:NH} of {\datafractionrms}.

            An interesting, and intriguing, interpretation and
            discussion of the maximum Shannon probability, is an
            explanation as to why the companies in the {\market} are
            not running an optimal re-investment strategy. This seems
            enigmatic, since those companies that run, on a long term
            average, below the optimally maximal value would seem to
            be eclipsed by those that didn't. And those that run above
            the optimally maximal value would be over extended, and
            become financially destitute during market down turns,
            which is inevitable in a fractal time series as presented
            in Figure~\ref{\SETLABEL:TS}.  It would seem that the
            natural selection process of the competitive environment
            would allow only those companies that run near the
            optimally maximal value to survive, in the long run. One
            possible explanation, foremost, is that the analytical
            methodology presented herein is inappropriate.  Another
            explanation is that the gross margins are less than the
            fraction {\shannonmax} of the rate of revenue returns, and
            thus could not accommodate such an aggressive
            re-investment strategy. If this is the case, then it
            presents an intriguing issue. If, in a capitalistic
            market, the natural outcome of the competitive situation,
            according to game-theoretic analysis, is that there will
            be many competitors, each making minimal gross margins,
            then how do the companies grow their markets?  Naturally,
            those that run the most efficient will have lower costs,
            making larger percentage of rate of revenue returns
            re-investment possible. Yet another interpretation is that
            the number of competitors would grow at an exponential
            rate, but all of them would make minimal returns. However,
            an operational Shannon probability of {\shannonlogreturns}
            is not just marginally lower than the maximum Shannon
            probability of {\shannonmax}. There is a significant
            disparity which is difficult to explain. It would seem
            that the game-theoretic eventual outcome of a competitive
            market place would be a solution that hinders growth,
            wealth and jobs creation, etc., which does not seem
            consistent with capitalistic theory. On the other hand, is
            there an optimum number of competitors in a market place,
            where the gross margins can be higher, permitting wealth
            and job creation, and also a competitive situation? If
            this analysis is correct, and that should be subject to
            scrutiny, then it would appear that this is the case. But
            this brings up another issue---that of taxation, and other
            contributions to the social welfare function. If there is
            an optimum number of competitors in the market place, that
            maximizes wealth and job creation, then, perhaps by lemma,
            there is also an optimal value of taxation rate, and other
            contributions to the social welfare function, that will
            permit maximal industrial growth, and thus maximal growth
            in the tax base. But this would seem to be inconsistent
            with the work of Kenneth Arrow and the so called
            Impossibility Theorem, which states that such
            optimizations can not be determined because the ordering
            of priorities are intransitive.  All very perplexing,
            since the simulation of the maximum Shannon probability in
            the next section seems to indicate that such an aggressive
            re-investment strategy is, indeed, feasible.

            Yet another possibility for the industry not running at
            maximum Shannon probability is the high cost of expansion
            of operations. Some of these industries require very
            sophisticated manufacturing processes, which have high
            barrier costs.

            Additionally, as mentioned in both~\cite[pp. 29]{Brock},
            and~\cite[pp. 8]{Arthur:CTIRALIBHE}, optimal efficiency
            may not be attainable in increasing-return economic
            scenarios.

        %
% -----------------------------------------------------------------------------
%
% A license is hereby granted to reproduce this software source code and
% to create executable versions from this source code for personal,
% non-commercial use.  The copyright notice included with the software
% must be maintained in all copies produced.
%
% THIS PROGRAM IS PROVIDED "AS IS". THE AUTHOR PROVIDES NO WARRANTIES
% WHATSOEVER, EXPRESSED OR IMPLIED, INCLUDING WARRANTIES OF
% MERCHANTABILITY, TITLE, OR FITNESS FOR ANY PARTICULAR PURPOSE.  THE
% AUTHOR DOES NOT WARRANT THAT USE OF THIS PROGRAM DOES NOT INFRINGE THE
% INTELLECTUAL PROPERTY RIGHTS OF ANY THIRD PARTY IN ANY COUNTRY.
%
% Copyright (c) 1994-2006, John Conover, All Rights Reserved.
%
% Comments and/or bug reports should be addressed to:
%
%     john@email.johncon.com (John Conover)
%
% -----------------------------------------------------------------------------
%
% Revision: \RCSRevision \\
% Revision Time: \RCSTime UMT \\
% Revision Date: \RCSDate \\
% Revision Id: \RCSId \\
% Revision File: \RCSLog \\
\RCS $Revision: 0.0 $
\RCS $Date: 2006/01/20 04:38:13 $
\RCS $Id: verification.tex,v 0.0 2006/01/20 04:38:13 john Exp $
% $Log: verification.tex,v $
% Revision 0.0  2006/01/20 04:38:13  john
% Initial version
%
%
    \subsection{Qualitative Verification of Fixed Increment Approximation Analysis}
        \label{\SETLABEL:QVA}

        \subidx{\market}{verification of analysis}
        \subidx{verification}{analysis}
        \subidx{analysis}{verification}
        \subidx{quality}{of analysis}
        \subidx{verification}{of methodology}
        \subidx{methodology}{verification of}
        \subidx{Shannon}{probability}
        \subidx{probability}{Shannon}

        This section evaluates various values based on the ``average''
        of the normalized increments presented in
        Figure~\ref{\SETLABEL:TFA}. These values are an approximation
        to a, probably, complex process with a distribution shown in
        Figure~\ref{\SETLABEL:TF}. These values will be used in a
        fixed increment Brownian fractal analysis of the {\market},
        and may, or may not, provide adequate accuracy for
        projections.

        The data in this section is presented in tabular form in
        sections~\ref{\SETLABELREF:VI1} and~\ref{\SETLABELREF:VI2}.
        As a subjective evaluation of the ``quality'' of the analysis
        of the {\market}, from Chapter~\ref{methodology},
        Equation~\ref{metricvalues1}, and using the mean and root mean
        square values of the normalized increments of the time series
        data presented in Figure~\ref{\SETLABEL:TS} from
        Figure~\ref{\SETLABEL:TF}, and the Shannon probability as
        calculated by counting the total number of {\timescale}s that
        the {\market} movement was positive, as presented in
        Section~\ref{\SETLABEL:MAXSHANNON}:

        \begin{eqnarray}
                  P & \approx & \frac{\frac{avg}{rms} + 1}{2}\\
            {\pmax} & \approx & \frac{\frac{\datafractionmean}{\datafractionrms} + 1}{2}\\
            {\pmax} & \approx & {\avgrms}
            \label{\SETLABEL:AVGS}
        \end{eqnarray}

        \subidx{Shannon}{probability}
        \subidx{probability}{Shannon}
        \noindent and comparing these values to the Shannon
        probability, as found by the {\it tsshannonmax}\/ program, which
        iterates for a maximum:

        \begin{eqnarray}
            {\pmax} \approx {\avgrms} \approx {\shannonmax}
        \end{eqnarray}

        \subidx{logarithmic}{returns}
        \subidx{returns}{logarithmic}
        In addition, the different methods of calculating the
        logarithmic returns, presented in Section~\ref{\SETLABEL:FS},
        should be compared. The four methods used were the mean of
        Figure~\ref{\SETLABEL:TF}, the constant in the least squares
        approximation to Figure~\ref{\SETLABEL:TF}, the least squares
        exponential approximation to Figure~\ref{\SETLABEL:TS}, and
        the logarithmic returns of Figure~\ref{\SETLABEL:TS}, derived
        as the mean of the logarithms of the quotients of the
        increments. The values for each of the methods are,
        respectively:

        \begin{equation}
            \datafractionmeanbits \approx \datafractionconstantbits \approx \datatslsqepbits \approx \logreturns
        \end{equation}

        It is implied in Section~\ref{\SETLABEL:FS},
        Subsection~\ref{\SETLABEL:SP} and in
        Section~\ref{\SETLABEL:TSUNFAIRBROWNIAN} that, a Brownian
        motion with fixed increments fractal may ``model'' the
        {\market}. Using Equation~\ref{stddev9} from
        Chapter~\ref{general}, Section~\ref{abmfi}:

        \begin{eqnarray}
                                    rms \left(2P - 1\right) & \approx & \frac{\sigma \left(2P - 1\right)}{2 \sqrt{P\left(1 - P\right)}}\\
            \datafractionrms \left(2 \cdot \pmax - 1\right) & \approx & \frac{\datafractionstddev \left(2 \cdot \pmax - 1\right)}{2\sqrt{\pmax \left(1 - \pmax\right)}}\\
                       \datafractionrms \cdot \twopminusone & \approx & \datafractionstddev \cdot \twopx\\
                                                      \rmsp & \approx & \sigmap
        \end{eqnarray}

        \noindent and, equating to the mean:

        \begin{equation}
            \datafractionmean \approx \rmsp \approx \sigmap
        \end{equation}

        \subidx{Shannon}{probability}
        \subidx{probability}{Shannon}
        \noindent where, as in Equation~\ref{\SETLABEL:AVGS} using the
        mean, root mean square, and standard deviation values of the
        normalized increments of the time series data presented in
        Figure~\ref{\SETLABEL:TS} from Figure~\ref{\SETLABEL:TF}, and
        the Shannon probability as calculated by counting the total
        number of {\timescale}s that the {\market} movement was
        positive, as presented in Section~\ref{\SETLABEL:MAXSHANNON}.

        As a final qualitative comparison, the absolute value of the
        normalized increments should be the same as the root mean
        square value\footnote{The absolute value of the normalized
        increments, when averaged, is related to the root mean square
        of the increments by a constant. If the normalized increments
        are a fixed increment, the constant is unity. If the
        normalized increments have a Gaussian distribution, the
        constant is $\approx 0.8$ depending on the accuracy of of
        ``fit'' to a Gaussian distribution.}, where the absolute value
        is presented in Figure~\ref{\SETLABEL:TFA}, and the root mean
        square value is presented in Figure~\ref{\SETLABEL:TF}:

        \begin{equation}
            \datafractionabsmean \approx \datafractionrms
        \end{equation}

        Note, that if the {\market} could be ``modeled'' as a Brownian
        motion with fixed increments fractal, then the standard
        deviation of the absolute value of the normalized increments
        of the time series data presented in Figure~\ref{\SETLABEL:TS}
        from Figure~\ref{\SETLABEL:TF} should be zero. It is
        $\datafractionabsstddev$.

% Local Variables:
% TeX-parse-self: t
% TeX-auto-save: t
% TeX-master: "fractal.tex"
% End:


        \subsubsection{Observations on the Qualitative Verification of Fixed Increment Approximation Analysis}

            In the equation:

            \begin{equation}
                \datafractionmean \approx \rmsp \approx \sigmap
            \end{equation}

            Note that the mean is unusually large, in relation to
            values for $rms (2P - 1)$ and $\frac{\sigma (2P - 1)}{2
            \sqrt{P(1 - P)}}$, respectively-in principle, they should
            all be equal. Also note that the standard deviation of the
            increments, {\datafractionstddev}, is much larger than the
            mean. These issues, coupled with a Hurst coefficient of
            {\thurstall}, probably prohibit ``modeling'' the
            {~\market} with the methodologies presented in this
            manuscript. Note, however, the poor accuracy performance
            of the methodology was estimated by the equation, above,
            so was anticipated.  As an aside, it is not clear what
            market mechanisms create the numbers in the equation
            above, or a Hurst coefficient that is indicative of an
            antipersistent rate of revenue returns.  Investigation of
            the metric methodologies for this market place, perhaps,
            may be interesting and provide some insight.

    \renewcommand{\market}{United States Office Computer Market}
    \renewcommand{\directory}{../markets/computer.office}
    \renewcommand{\datafractionmean}{0.008052}
\renewcommand{\datafractionmeanbits}{0.011570}
\renewcommand{\datafractionmeanq}{0.002684}
\renewcommand{\datafractionmeanbitsq}{0.003867}
\renewcommand{\datafractionstddev}{0.038579}
\renewcommand{\datafractionrms}{0.039311}
\renewcommand{\avgrms}{0.602414}
\renewcommand{\ncompanies}{5.210454}
\renewcommand{\pncompanies}{0.544866}
\renewcommand{\datafractionabsmean}{0.029745}
\renewcommand{\datafractionabsstddev}{0.025769}
\renewcommand{\datafractionconstant}{0.010041}
\renewcommand{\datafractionconstantbits}{0.014414}
\renewcommand{\datafractionconstantq}{0.003347}
\renewcommand{\datafractionconstantbitsq}{0.004821}
\renewcommand{\datafractionslope}{-0.000021}
\renewcommand{\datafractionabsconstant}{0.035145}
\renewcommand{\datafractionabsslope}{-0.000057}
\renewcommand{\hurstall}{0.659558}
\renewcommand{\hurstlow}{0.707509}
\renewcommand{\hurstlowtwo}{1.415018}
\renewcommand{\hurstlowhundred}{70.750900}
\renewcommand{\hcalcall}{0.184942}
\renewcommand{\hcalclow}{0.102042}
\renewcommand{\shannonmax}{0.604167}
\renewcommand{\twoponemax}{0.208334}
\renewcommand{\logreturns}{0.010456}
\renewcommand{\twologreturns}{1.007274}
\renewcommand{\twologreturnshundred}{0.727387}
\renewcommand{\oneoverlogreturns}{95.638868}
\renewcommand{\pmax}{0.602094}
\renewcommand{\twopminusone}{0.204188}
\renewcommand{\rmsp}{0.008027}
\renewcommand{\twopx}{0.208583}
\renewcommand{\sigmap}{0.008047}
\renewcommand{\tsunfairbrownianfractionmean}{0.007862}
\renewcommand{\tsunfairbrownianfractionstddev}{0.038619}
\renewcommand{\shannonlogreturns}{0.560125}
\renewcommand{\shannonlogreturnshundred}{56.012500}
\renewcommand{\twopone}{0.120250}
\renewcommand{\twoponehundred}{12.025000}
\renewcommand{\hundredtwoponehundred}{87.975000}
\renewcommand{\hundredshannonlogreturnshundred}{43.987500}
\renewcommand{\datatslsqepbits}{0.007623}
\renewcommand{\thurstall}{0.633980}
\renewcommand{\thurstlow}{0.710108}
\renewcommand{\thurstlowtwo}{1.420216}
\renewcommand{\thurstlowhundred}{71.010800}
\renewcommand{\thcalcall}{0.247886}
\renewcommand{\thcalclow}{0.171737}
\renewcommand{\chisquared}{2.862000}
\renewcommand{\critical}{42.557000}

    \renewcommand{\timescale}{month}
    \subidx{market}{\market}
    \idx{\market}

    \section{\market}

        \renewcommand{\SETLABEL}{\LABPRE:NAOCM}
        \renewcommand{\SETLABELQ}{\LABPRE:NAOCMQ}
        \label{\SETLABEL}
        \renewcommand{\SETLABELREF}{\LABPREREF:NAOCM}

        \idx{United States Department of Commerce}
        For the analysis, the data was in the directory
        {\directory}\footnote{Data from the United States Department
        of Commerce, 1982---1994, by {\timescale}s, as an index, 1987
        = 100.}.

        The data in this section is presented in tabular form in
        Section~\ref{\SETLABELREF}.

        %
% -----------------------------------------------------------------------------
%
% A license is hereby granted to reproduce this software source code and
% to create executable versions from this source code for personal,
% non-commercial use.  The copyright notice included with the software
% must be maintained in all copies produced.
%
% THIS PROGRAM IS PROVIDED "AS IS". THE AUTHOR PROVIDES NO WARRANTIES
% WHATSOEVER, EXPRESSED OR IMPLIED, INCLUDING WARRANTIES OF
% MERCHANTABILITY, TITLE, OR FITNESS FOR ANY PARTICULAR PURPOSE.  THE
% AUTHOR DOES NOT WARRANT THAT USE OF THIS PROGRAM DOES NOT INFRINGE THE
% INTELLECTUAL PROPERTY RIGHTS OF ANY THIRD PARTY IN ANY COUNTRY.
%
% Copyright (c) 1994-2006, John Conover, All Rights Reserved.
%
% Comments and/or bug reports should be addressed to:
%
%     john@email.johncon.com (John Conover)
%
% -----------------------------------------------------------------------------
%
% Revision: \RCSRevision \\
% Revision Time: \RCSTime UMT \\
% Revision Date: \RCSDate \\
% Revision Id: \RCSId \\
% Revision File: \RCSLog \\
\RCS $Revision: 0.0 $
\RCS $Date: 2006/01/20 04:38:13 $
\RCS $Id: fraction.tex,v 0.0 2006/01/20 04:38:13 john Exp $
% $Log: fraction.tex,v $
% Revision 0.0  2006/01/20 04:38:13  john
% Initial version
%
%
    \subsection{Time Series Increments Analysis}
        \label{\SETLABEL:TSA}

        \subidx{\market}{Time series analysis}
        \subidx{time series}{increments}
        \subidx{time series}{analysis}
        \subidx{cumulative sum}{analysis}
        \subidx{analysis}{cumulative sum}
        \subidx{analysis}{random process}
        \subidx{random process}{analysis}
        \subidx{Gaussian}{increments}
        \subidx{increments}{Gaussian}
        \subidx{Brownian}{motion, fractional}
        \subidx{fractional}{Brownian motion}
        \subidx{fractal}{Brownian motion}
        The data in this section is presented in tabular form in
        Section~\ref{\SETLABELREF:TSA}.  Figure~\ref{\SETLABEL:TS} is
        a graph of the time series data for the {\market}.

        \subidx{increments}{normalized}
        \subidx{normalized}{increments}
        \subidx{programs}{tsfraction}
        \subidx{tsfraction}{program}
        Figure~\ref{\SETLABEL:TF} is a graph of the normalized
        increments of the time series data presented in
        Figure~\ref{\SETLABEL:TS}. The data presented was made by
        running the program {\it tsfraction}\/ on the time series
        data. The program {\it tsfraction}\/ is described briefly in
        Appendix~\ref{programs}, and subtracts the previous value from
        the next value, dividing this difference by the previous
        value, for each element in the time series data. The new time
        series contains the instantaneous change in the rate of
        revenue returns, divided by the magnitude of the instantaneous
        rate of revenue returns.

        \subidx{mean}{standard deviation}
        \subidx{standard deviation}{mean}
        \idx{root mean square}
        \idx{least squares approximation}
        \begin{figure}[ht]
            \begin{center}
                \begin{minipage}[t]{0.45\textwidth}
                    \epsfxsize=1.0\linewidth
                    \epsffile{\directory/data.eps}
                    \caption{{\market}, time series data.}
                    \label{\SETLABEL:TS}
                    \label{\SETLABELQ:TS}
                \end{minipage}
                \hfill
                \begin{minipage}[t]{0.45\textwidth}
                    \epsfxsize=1.0\linewidth
                    \epsffile{\directory/data.tsfraction.eps}
                    \caption[{\market}, normalized
                        increments]{{\market}, normalized increments
                        of the time series data presented in
                        Figure~\ref{\SETLABEL:TS}. The mean is
                        {\datafractionmean} with a standard deviation
                        of {\datafractionstddev}. The formula for the
                        least squares approximation is
                        ${\datafractionconstant} +
                        {\datafractionslope}t$, and the root mean
                        squared value is {\datafractionrms}. The
                        graph, labeled ``data\-.tsfraction\-.tsrms,''
                        is the running root mean square, and
                        ``data\-.tsfraction\-.tsavg'' is the running
                        average of the normalized increments.  This
                        graph is the fraction of change in the time
                        series, as a function of time. Note that the
                        slope of the mean, {\datafractionslope}, is
                        the coefficient of the nonlinearity term in
                        the normalized increments. See
                        Chapter~\ref{general}, Section~\ref{nlextend}
                        for a possible application of the logistic
                        function to this data set.}
                    \label{\SETLABEL:TF}
                    \label{\SETLABELQ:TF}
                \end{minipage}
            \end{center}
        \end{figure}

        \subidx{absolute value}{increments}
        \subidx{increments}{absolute value}

        Figure~\ref{\SETLABEL:TFA} is a graph of the absolute value of
        the normalized increments of the time series data presented in
        Figure~\ref{\SETLABEL:TF}. The data presented was made by
        running the Unix utility sed(1) on the normalized increments
        time series data to remove the negative signs. This is an
        absolute value procedure.  The resulting time series contains
        the absolute value of the instantaneous change in the rate of
        revenue returns, divided by the magnitude of the instantaneous
        rate of revenue returns\footnote{The absolute value of the
        normalized increments, when averaged, is related to the root
        mean square of the increments by a constant. If the normalized
        increments are a fixed increment, the constant is unity. If
        the normalized increments have a Gaussian distribution, the
        constant is $\approx 0.8$ depending on the accuracy of of
        ``fit'' to a Gaussian distribution.}.

        \subidx{histogram}{normalized}
        \subidx{normalized}{histogram}
        \subidx{programs}{tsnormal}
        \subidx{tsnormal}{program}
        \subidx{mean}{standard deviation}
        \subidx{standard deviation}{mean}
        \idx{root mean square}
        \idx{least squares approximation}
        \subidx{\market}{analysis of increments}
        Figure~\ref{\SETLABEL:NH} is the normalized histogram of the
        normalized increments of the time series data shown in
        Figure~\ref{\SETLABEL:TF}. The abscissa is 3 $\sigma$ limits,
        and the area under the two curves is identical. The data for
        this figure was produced by the program {\it tsnormal}\/,
        which is described briefly in Appendix~\ref{programs}.

        \begin{figure}[ht]
            \begin{center}
                \begin{minipage}[t]{0.45\textwidth}
                    \epsfxsize=1.0\linewidth
                    \epsffile{\directory/data.tsfraction.abs.eps}
                    \caption[{\market}, absolute value of the
                        normalized increments]{{\market}, absolute
                        value of the normalized increments of the time
                        series data presented in
                        Figure~\ref{\SETLABEL:TF}.  The mean is
                        {\datafractionabsmean} with a standard
                        deviation of {\datafractionabsstddev}. The
                        formula for the least squares approximation is
                        ${\datafractionabsconstant} +
                        {\datafractionabsslope}t$, and the root mean
                        square value, from Figure~\ref{\SETLABEL:TF},
                        is {\datafractionrms}.  The graph, labeled
                        ``data\-.tsfraction\-.tsrms,'' is the running
                        root mean square, and
                        ``data\-.tsfraction\-.tsavg'' is the running
                        average of the normalized increments presented
                        in Figure~\ref{\SETLABEL:TF}, superimposed
                        here for convenience. This graph is the
                        absolute value of the fraction of change in
                        the time series, as a function of time.}
                    \label{\SETLABEL:TFA}
                    \label{\SETLABELQ:TFA}
                \end{minipage}
                \hfill
                \begin{minipage}[t]{0.45\textwidth}
                    \epsfxsize=1.0\linewidth
                    \epsffile{\directory/data.tsfraction.tsnormal-s30.eps}
                    \caption[{\market}, normalized histogram of the
                        normalized increments]{{\market}, normalized
                        histogram of the normalized increments of the
                        time series data shown in
                        Figure~\ref{\SETLABEL:TF}.  The data has a
                        mean of {\datafractionmean}, with a standard
                        deviation of {\datafractionstddev}.  The area
                        under the two curves is identical. The
                        $\chi^2$ value of the observed and expected
                        values of the two curves is {\chisquared},
                        with a critical value of {\critical}.}
                    \label{\SETLABEL:NH}
                \end{minipage}
            \end{center}
        \end{figure}

        \subidx{programs}{tsXsquared}
        \subidx{tsXsquared}{program}
        \subidx{\market}{chi-squared values of increments}
        The program {\it tsXsquared}\/, which is briefly described in
        appendix~\ref{programs}, was used to derive the $\chi^2$
        statistics for the data presented in
        Figure~\ref{\SETLABEL:NH}.

        \subidx{programs}{tsstatest}
        \subidx{tsstatest}{program}
        \subidx{\market}{statistical estimates}

        Figure~\ref{\SETLABEL:SE} is the statistical estimate for the
        data presented in Figure~\ref{\SETLABEL:TF}, as derived by the
        program {\it tsstatest}\/, which is briefly described in
        appendix~\ref{programs}.

        \begin{figure}[ht]
            \begin{center}
                \begin{minipage}[t]{\textwidth}
                    \center{\fbox{\parbox{0.9\textwidth}{\XXX{\directory/data.tsstatest-f0.1-c0.9-i.tex}}}}
                    \caption[{\market}, statistical estimates of the
                        normalized increments]{{\market}, statistical
                        estimates of the normalized increments of the
                        time series shown in Figure~\ref{\SETLABEL:TF}.
                        The table was produced with the {\it
                        tsstatest}\/ program, and illustrates the
                        size of the data set required for a confidence
                        level of 90\%, with an error estimate of $\pm$
                        10\%, or alternately, the error estimate on
                        the time series shown in Figure~\ref{\SETLABEL:TF}.}
                    \label{\SETLABEL:SE}
                \end{minipage}
            \end{center}
        \end{figure}

        Note that the data set size estimations, as produced by the
        {\it tsstatest}\/ program, are probably very conservative,
        depending on the magnitude of the Shannon probability, $P =
        \shannonlogreturns$, as derived in
        Section~\ref{\SETLABEL:SP}. See Chapter~\ref{general},
        Section~\ref{serdss} for possible alternative methodologies
        for addressing the analysis of fractal time series with
        limited data set sizes. Depending on the magnitude of the
        Shannon probability, $P$, these estimates can be several
        orders of magnitude too high.

        \subidx{derivative of increments}{normalized}
        \subidx{normalized}{derivative of increments}
        \subidx{programs}{tsderivative}
        \subidx{tsderivative}{program}
        Figure~\ref{\SETLABEL:TF1} is the normalized histogram of the
        first derivative of the normalized increments of the time
        series data shown in Figure~\ref{\SETLABEL:TF}. In principle,
        if the distribution of the normalized increments presented in
        Figure~\ref{\SETLABEL:NH} is Gaussian in nature, this
        distribution would be similar to ``white noise,'' as presented
        in appendix~\ref{programs}, Figure~\ref{whiteexample}. The
        data was generated by the {\it tsderivative}\/ program, which
        is briefly described in
        appendix~\ref{programs}. Figure~\ref{\SETLABEL:TF2} is the
        normalized histogram of the second derivative of the
        normalized increments of the time series data shown in
        Figure~\ref{\SETLABEL:TF}. In principle, if the distribution
        of the normalized increments presented in
        Figure~\ref{\SETLABEL:NH} is an integrated Gaussian
        distribution in nature, this distribution would be similar to
        ``white noise,'' as presented in appendix~\ref{programs},
        Figure~\ref{whiteexample}.

        \begin{figure}[ht]
            \begin{center}
                \begin{minipage}[t]{0.45\textwidth}
                    \epsfxsize=1.0\linewidth
                    \epsffile{\directory/data.tsfraction.tsderivative.tsnormal-s30.eps}
                    \caption[{\market}, histogram of the first
                        derivative of the increments]{{\market},
                        normalized histogram of the first derivative
                        of the normalized increments of the time
                        series data shown in
                        Figure~\ref{\SETLABEL:TF}.}
                    \label{\SETLABEL:TF1}
                \end{minipage}
                \hfill
                \begin{minipage}[t]{0.45\textwidth}
                    \epsfxsize=1.0\linewidth
                    \epsffile{\directory/data.tsfraction.2tsderivative.tsnormal-s30.eps}
                    \caption[{\market}, histogram of the second
                        derivative of the increments]{{\market},
                        normalized histogram of second derivative of
                        the the normalized increments of the time
                        series data shown in
                        Figure~\ref{\SETLABEL:TF}.}
                    \label{\SETLABEL:TF2}
                \end{minipage}
            \end{center}
        \end{figure}

        \subidx{fractal}{range}
        \subidx{fractal}{R/S analysis}
        \subidx{\market}{rate of revenue returns, range}
        \subidx{\market}{deterministic mechanism}
        \subidx{deterministic}{mechanism}
        \subidx{mechanism}{deterministic}
        Figure~\ref{\SETLABEL:TR} is the range of values of the time
        series shown in Figure~\ref{\SETLABEL:TS}. The horizontal axis
        is time into the future. In principle, if the time series was
        characterized as fractional Brownian motion the graph in
        Figure~\ref{\SETLABEL:TR} would be a square root
        function\footnote{Note that the ``roughness,'' or ``sawtooth''
        characteristics of the graph in Figure~\ref{\SETLABEL:TR} are
        a computational artifact---caused by not using the -m option
        to the program {\it tshurst}\/, which is computationally
        inefficient.}. Figure~\ref{\SETLABEL:TD} is the deterministic
        map of the normalized increments of the time series data shown
        in Figure~\ref{\SETLABEL:TF}. The deterministic map is useful
        for determining if a time series was created by a
        deterministic mechanism. This, essentially, maps each element
        in the time series with the previous element in the time
        series.  See,~\cite[pp. 745]{Peitgen}.

        \begin{figure}[ht]
            \begin{center}
                \begin{minipage}[t]{0.45\textwidth}
                    \epsfxsize=1.0\linewidth
                    \epsffile{\directory/data.tshurst-f.eps}
                    \caption[{\market}, range]{{\market}, range of the
                        time series data shown in
                        Figure~\ref{\SETLABEL:TS}.}
                    \label{\SETLABEL:TR}
                \end{minipage}
                \hfill
                \begin{minipage}[t]{0.45\textwidth}
                    \epsfxsize=1.0\linewidth
                    \epsffile{\directory/data.tsfraction.tsdeterministic.eps}
                    \caption[{\market}, deterministic map]{{\market},
                        deterministic map of the normalized increments
                        of the time series data shown in
                        Figure~\ref{\SETLABEL:TF}.}
                    \label{\SETLABEL:TD}
                \end{minipage}
            \end{center}
        \end{figure}

% Local Variables:
% TeX-parse-self: t
% TeX-auto-save: t
% TeX-master: "fractal.tex"
% End:


        \subsubsection{Observations on the Time Series Increments Analysis}

            Figure~\ref{\SETLABEL:NH} would seem to indicate that the
            time series data for the {\market} represents a cumulative
            sum/integration of a random process that has a Gaussian
            distribution, (ie., satisfies the Gaussian increments
            property of fractional Brownian
            motion~\cite[pp. 250]{Crownover},) tending to justify the
            assumption that the time series data represents fractional
            Brownian motion.

        %
% -----------------------------------------------------------------------------
%
% A license is hereby granted to reproduce this software source code and
% to create executable versions from this source code for personal,
% non-commercial use.  The copyright notice included with the software
% must be maintained in all copies produced.
%
% THIS PROGRAM IS PROVIDED "AS IS". THE AUTHOR PROVIDES NO WARRANTIES
% WHATSOEVER, EXPRESSED OR IMPLIED, INCLUDING WARRANTIES OF
% MERCHANTABILITY, TITLE, OR FITNESS FOR ANY PARTICULAR PURPOSE.  THE
% AUTHOR DOES NOT WARRANT THAT USE OF THIS PROGRAM DOES NOT INFRINGE THE
% INTELLECTUAL PROPERTY RIGHTS OF ANY THIRD PARTY IN ANY COUNTRY.
%
% Copyright (c) 1994-2006, John Conover, All Rights Reserved.
%
% Comments and/or bug reports should be addressed to:
%
%     john@email.johncon.com (John Conover)
%
% -----------------------------------------------------------------------------
%
% Revision: \RCSRevision \\
% Revision Time: \RCSTime UMT \\
% Revision Date: \RCSDate \\
% Revision Id: \RCSId \\
% Revision File: \RCSLog \\
\RCS $Revision: 0.0 $
\RCS $Date: 2006/01/20 04:38:13 $
\RCS $Id: instant.tex,v 0.0 2006/01/20 04:38:13 john Exp $
% $Log: instant.tex,v $
% Revision 0.0  2006/01/20 04:38:13  john
% Initial version
%
%
    \subsection{Instantaneous Analysis of Normalized Increments}
        \label{\SETLABEL:IA}

        \subidx{\market}{instantaneous analysis of normalized increments}
        \idx{average of normalized increments}
        \idx{root mean square of normalized increments}
        \subidx{Shannon probability}{instantaneous computation of}
        \subidx{average of normalized increments}{instantaneous computation of}
        \subidx{root mean square of normalized increments}{instantaneous computation of}
        \subidx{instantaneous computation}{Shannon probability}
        \subidx{instantaneous computation}{average of normalized increments}
        \subidx{instantaneous computation}{root mean square of normalized increments}
        \idx{time series}
        \subidx{time series}{instantaneous analysis}
        \subidx{instantaneous analysis}{time series}
        \subidx{time series}{increments}
        \subidx{time series}{analysis}
        \subidx{Shannon}{probability}
        \subidx{probability}{Shannon}
        \subidx{normalized}{increments}
        \subidx{increments}{normalized}

        The program {\it tsinstant}\/, which is briefly described in
        Appendix~\ref{programs}, is for finding the instantaneous
        fraction of change in a time series. The value of a sample in
        the time series is subtracted from the previous sample in the
        time series, and divided by the value of the previous sample.
        As explained in Chapter~\ref{general},
        Sections~\ref{derivation},~\ref{GA},~\ref{abmfi},~\ref{aftsma}
        and,~\ref{ompl} for Brownian motion, random walk fractals, the
        absolute value of the instantaneous fraction of change is also
        the root mean square of the instantaneous fraction of
        change\footnote{The absolute value of the normalized
        increments, when averaged, is related to the root mean square
        of the increments by a constant. If the normalized increments
        are a fixed increment, the constant is unity. If the
        normalized increments have a Gaussian distribution, the
        constant is $\approx 0.8$ depending on the accuracy of of
        ``fit'' to a Gaussian distribution.}. Squaring this value is
        the average of the instantaneous fraction of change, and
        adding unity to the absolute value of the instantaneous
        fraction of change, and dividing by two, is the Shannon
        probability of the instantaneous fraction of change.

        Figure~\ref{\SETLABEL:IA1} is the instantaneous value of the
        root mean square of the normalized increments for the
        {\market}, and Figure~\ref{\SETLABEL:IA2} is the instantaneous
        Shannon probability for the normalized increments.

        \begin{figure}[ht]
            \begin{center}
                \begin{minipage}[t]{0.45\textwidth}
                    \epsfxsize=1.0\linewidth
                    \epsffile{\directory/data.tsinstant-r.eps}
                    \caption[{\market}, instantaneous value of
                        rms.]{{\market}, instantaneous value of the
                        root mean square of the normalized increments,
                        provided by running the program {\it
                        tsinstant}\/ with the -r option on the data
                        presented in Figure~\ref{\SETLABEL:TS}.}
                    \label{\SETLABEL:IA1}
                    \label{\SETLABELQ:IA1}
                \end{minipage}
                \hfill
                \begin{minipage}[t]{0.45\textwidth}
                    \epsfxsize=1.0\linewidth
                    \epsffile{\directory/data.tsinstant-s.eps}
                    \caption[{\market}, instantaneous value of
                        Shannon probability.]{{\market}, instantaneous
                        value of the Shannon probability of the
                        normalized increments, provided by running the
                        program {\it tsinstant}\/ with the -s option
                        on the data presented in
                        Figure~\ref{\SETLABEL:TS}.}
                    \label{\SETLABEL:IA2}
                    \label{\SETLABELQ:IA2}
                \end{minipage}
            \end{center}
        \end{figure}

% Local Variables:
% TeX-parse-self: t
% TeX-auto-save: t
% TeX-master: "fractal.tex"
% End:


        %
% -----------------------------------------------------------------------------
%
% A license is hereby granted to reproduce this software source code and
% to create executable versions from this source code for personal,
% non-commercial use.  The copyright notice included with the software
% must be maintained in all copies produced.
%
% THIS PROGRAM IS PROVIDED "AS IS". THE AUTHOR PROVIDES NO WARRANTIES
% WHATSOEVER, EXPRESSED OR IMPLIED, INCLUDING WARRANTIES OF
% MERCHANTABILITY, TITLE, OR FITNESS FOR ANY PARTICULAR PURPOSE.  THE
% AUTHOR DOES NOT WARRANT THAT USE OF THIS PROGRAM DOES NOT INFRINGE THE
% INTELLECTUAL PROPERTY RIGHTS OF ANY THIRD PARTY IN ANY COUNTRY.
%
% Copyright (c) 1994-2006, John Conover, All Rights Reserved.
%
% Comments and/or bug reports should be addressed to:
%
%     john@email.johncon.com (John Conover)
%
% -----------------------------------------------------------------------------
%
% Revision: \RCSRevision \\
% Revision Time: \RCSTime UMT \\
% Revision Date: \RCSDate \\
% Revision Id: \RCSId \\
% Revision File: \RCSLog \\
\RCS $Revision: 0.0 $
\RCS $Date: 2006/01/20 04:38:13 $
\RCS $Id: logistic.tex,v 0.0 2006/01/20 04:38:13 john Exp $
% $Log: logistic.tex,v $
% Revision 0.0  2006/01/20 04:38:13  john
% Initial version
%
%
    \subsection{Logistic Analysis}
        \label{\SETLABEL:LA}

        \subidx{\market}{Logistic function analysis}
        \subidx{time series}{logistic function}
        \subidx{logistic function}{time series}
        \subidx{time series}{increments}
        \subidx{time series}{analysis}
        \subidx{cumulative sum}{analysis}
        \subidx{analysis}{cumulative sum}
        \subidx{analysis}{random process}
        \subidx{random process}{analysis}
        The data in this section is presented in tabular form in
        Section~\ref{\SETLABELREF:LAA}.  Figure~\ref{\SETLABEL:LA1} is
        a graph of the logistic function estimates of the time series
        data for the {\market}. The reader is cautioned that these
        graphs are constructed using the method suggested in
        Chapter~\ref{general}, Section~\ref{nlextend} and enormous
        precision is required for adequate prediction of the logistic
        function,~\cite{Modis}. Particularly, the non-linear term will
        usually require intervention to produce a practical fit to the
        data. In addition, there are numerical stability issues with
        logistic function methodologies\footnote{For example, in
        Figures~\ref{\SETLABEL:LA1} and~\ref{\SETLABEL:LA2}, if the
        non-linear term, $b$, was greater than zero, it was set to
        zero to produce the graphs. See Section~\ref{\SETLABELREF:LAA}
        for the actual derived values. In other cases, the magnitude
        of $b$ was too large, resulting in a graph that was decreasing
        as a function of time}.  The methodology should be regarded as
        ``fragile.'' It is included for completeness.

        \idx{least squares approximation}
        Figure~\ref{\SETLABEL:LA1} is a graph of the logistic function
        for the time series data presented in
        Figure~\ref{\SETLABEL:TS}. The data presented was made by
        running the program {\it tsdlogistic}\/, which is described
        briefly in Appendix~\ref{programs}, on the parameters
        extracted from the time series data as suggested in
        Figure~\ref{\SETLABEL:TF}. The program {\it tslsq}\/ was used
        to derive the constant and the slope of the normalized
        increments of the data presented in Figure~\ref{\SETLABEL:TF}.
        Figure~\ref{\SETLABEL:LA2} is the same graph, but with the
        time scale expanded by a factor of two.

        \begin{figure}[ht]
            \begin{center}
                \begin{minipage}[t]{0.45\textwidth}
                    \epsfxsize=1.0\linewidth
                    \epsffile{\directory/data.tsfraction.tslsq-p.tsdlogistic.eps}
                    \caption[{\market}, logistic function
                        estimates.]{{\market}, logistic function
                        estimates, provided by running the {\it
                        tslsq}\/ program on the normalized increments
                        presented in Figure~\ref{\SETLABEL:TF} with
                        the -p option. These parameters were used as
                        arguments to the {\it tsdlogistic}\/ program.}
                    \label{\SETLABEL:LA1}
                    \label{\SETLABELQ:LA1}
                \end{minipage}
                \hfill
                \begin{minipage}[t]{0.45\textwidth}
                    \epsfxsize=1.0\linewidth
                    \epsffile{\directory/data.tsfraction.tslsq-p.tsdlogistic2.eps}
                    \caption[{\market}, logistic function
                        estimates.]{{\market}, logistic function
                        estimates of Figure~\ref{\SETLABEL:LA1} with
                        the time scale expanded by a factor of two.}
                    \label{\SETLABEL:LA2}
                    \label{\SETLABELQ:LA2}
                \end{minipage}
            \end{center}
        \end{figure}

% Local Variables:
% TeX-parse-self: t
% TeX-auto-save: t
% TeX-master: "fractal.tex"
% End:


        %
% -----------------------------------------------------------------------------
%
% A license is hereby granted to reproduce this software source code and
% to create executable versions from this source code for personal,
% non-commercial use.  The copyright notice included with the software
% must be maintained in all copies produced.
%
% THIS PROGRAM IS PROVIDED "AS IS". THE AUTHOR PROVIDES NO WARRANTIES
% WHATSOEVER, EXPRESSED OR IMPLIED, INCLUDING WARRANTIES OF
% MERCHANTABILITY, TITLE, OR FITNESS FOR ANY PARTICULAR PURPOSE.  THE
% AUTHOR DOES NOT WARRANT THAT USE OF THIS PROGRAM DOES NOT INFRINGE THE
% INTELLECTUAL PROPERTY RIGHTS OF ANY THIRD PARTY IN ANY COUNTRY.
%
% Copyright (c) 1994-2006, John Conover, All Rights Reserved.
%
% Comments and/or bug reports should be addressed to:
%
%     john@email.johncon.com (John Conover)
%
% -----------------------------------------------------------------------------
%
% Revision: \RCSRevision \\
% Revision Time: \RCSTime UMT \\
% Revision Date: \RCSDate \\
% Revision Id: \RCSId \\
% Revision File: \RCSLog \\
\RCS $Revision: 0.0 $
\RCS $Date: 2006/01/20 04:38:13 $
\RCS $Id: hurst.tex,v 0.0 2006/01/20 04:38:13 john Exp $
% $Log: hurst.tex,v $
% Revision 0.0  2006/01/20 04:38:13  john
% Initial version
%
%
    \subsection{Hurst Coefficient Analysis}
        \label{\SETLABEL:H}

        \subidx{\market}{Hurst coefficient analysis}
        \subidx{Hurst coefficient}{analysis}
        \subidx{increments}{normalized}
        \subidx{normalized}{increments}
        \subidx{programs}{tshurst}
        \subidx{tshurst}{program}
        The data in this section is presented in tabular form in
        Section~\ref{\SETLABELREF:HCHP}. Figure~\ref{\SETLABEL:HC} is
        a graph of the Hurst coefficient data time series data shown
        in Figure~\ref{\SETLABEL:TS}. The slope of the graph is the
        Hurst coefficient.  The data for this figure was produced by
        the program {\it tshurst}\/, which is described briefly in
        Appendix~\ref{programs}.

        \subidx{\market}{H parameter analysis}
        \subidx{H parameter}{analysis}
        \subidx{programs}{tshcalc}
        \subidx{tshcalc}{program}
        Figure~\ref{\SETLABEL:HP} is a graph of the H parameter data
        for the normalized increments of the time series data shown in
        Figure~\ref{\SETLABEL:TF}. The data for this figure was
        produced by the program {\it tshcalc}\/, which is described
        briefly in Appendix~\ref{programs}.

        \begin{figure}[ht]
            \begin{center}
                \begin{minipage}[t]{0.45\textwidth}
                    \epsfxsize=1.0\linewidth
                    \epsffile{\directory/data.tshurst.eps}
                    \caption[{\market}, Hurst coefficient data]{{\market},
                        Hurst coefficient data for the normalized
                        increments of the time series data shown in
                        Figure~\ref{\SETLABEL:TF}.  The slope of the graph
                        is the Hurst coefficient.}
                    \label{\SETLABEL:HC}
                \end{minipage}
                \hfill
                \begin{minipage}[t]{0.45\textwidth}
                    \epsfxsize=1.0\linewidth
                    \epsffile{\directory/data.tshcalc.eps}
                    \caption[{\market}, H parameter data]{{\market}, H
                        parameter data for the normalized increments of
                        the time series data shown in
                        Figure~\ref{\SETLABEL:TF} The slope of the graph
                        is the H parameter.}
                    \label{\SETLABEL:HP}
                \end{minipage}
            \end{center}
        \end{figure}

        \subidx{revenue}{See, rate of revenue returns}
        \subidx{returns}{See, rate of revenue returns}
        \subidx{\market}{revenues}
        \subidx{Hurst coefficient}{analysis}
        \subidx{\market}{Hurst coefficient analysis}
        \subidx{\market}{rate of change}
        \subidx{\market}{windows of opportunity}
        \subidx{rate of revenue returns}{forecast}
        \subidx{forecast}{rate of revenue returns}
        \idx{windows of opportunity}
        \subidx{programs}{tslsq}
        \subidx{tslsq}{program}

        The approximately linear slope of the graph in
        Figure~\ref{\SETLABEL:HC} implies that the variance of the
        rate of revenue returns, (per {\timescale},) in the {\market},
        $V(t_2 - t_1)$, over a period of time is proportional to the
        period of time raised to twice the Hurst
        coefficient~\cite[pp. 180]{Feder},~\cite[pp. 246]{Crownover}.
        This seems to be a quantitative statement concerning how fast,
        and to what degree, the rate of revenue returns' state of
        affairs can change over a period of time.  An additional
        implication, for Hurst coefficients sufficiently close to 0.5,
        is that the probability of the state of affairs repeating
        sometime in the future goes down with increasing
        time\footnote{It can be shown that the number of expected
        market ``high'' and ``low'' transitions, $N$, scales with the
        square root of time, or $N \propto \sqrt {t}$, meaning that
        the cumulative distribution of the probability, $P$, of the
        duration of a market's ``high'' or ``low'' exceeding a given
        time interval, $t$, is proportional to the reciprocal of the
        square root of the time interval, $P \propto 1 / \sqrt {t}$,
        (or, conversely, that the probability of the duration of a
        market's ``high'' or ``low'' exceeding a given time interval
        is proportional to the reciprocal of the time interval raised
        to the power $3 / 2$, ie., $P \propto 1 / t^{3 /
        2}$,~\cite[pp. 153]{Schroeder}. What this means is that a
        histogram of the ``zero free'' run-lengths of a market being
        ``high'' or ``low,'' over a long time, would have a $1 / t^{3
        / 2}$ characteristic.)}, $t$, $p(t) = erf (1/\sqrt{2t})$ which
        is approximately $1/\sqrt{t}$ for $t \gg
        1$~\cite[pp. 160]{Schroeder}. Figures~\ref{\SETLABEL:FN},
        and,~\ref{\SETLABEL:FF} compare methods of approximation of
        the ``forecastability'' of the rate of revenue returns in the
        {\market} for the near term and far term,
        respectively~\cite[pp. 83-84]{Peters:CAOITCM}\footnote{The
        author is not comfortable with Peters' interpretation. For
        example, if the algorithm explained
        in~\cite[pp. 82]{Peters:CAOITCM} is used on ``white noise''
        which, by definition, never has any correlations, the short
        term Hurst coefficient, and thus the ``forecastability,'' is
        still near unity---a bit of an enigma. This can be verified
        with the {\it tswhite}\/ and {\it tshurst}\/ programs, which
        are briefly described in Appendix~\ref{programs}.}.  This
        seems to be a quantitative statement concerning ``windows of
        opportunity'' in the rate of revenue returns, (per
        {\timescale}.)  The program {\it tslsq}\/ was used on the
        Hurst coefficient data, presented in
        Figure~\ref{\SETLABEL:HC}, to provide a least squares
        approximation to the Hurst coefficient. The superimposed least
        squares approximation with on original Hurst coefficient data
        is presented.  The time series data has a Hurst coefficient of
        {\thurstlow}, so that:

        \subidx{\market}{Hurst coefficient analysis}
        \begin{eqnarray}
            V\left(t_2 - t_1\right) & \propto & \left(t_2 - t_1\right)^{2 \cdot H}\\
            V\left(t_2 - t_1\right) & \propto & \left(t_2 - t_1\right)^{2 \cdot {\thurstlow}}\\
                                    & \propto & \left(t_2 - t_1\right)^{\thurstlowtwo}
            \label{\SETLABEL:V}
        \end{eqnarray}

        \subidx{fractional}{Brownian motion}
        \subidx{Brownian motion}{fractional}
        \idx{fractal}
        \noindent where $V(t_2 - t_1)$ is the variance of the
        increments of the rate of revenue returns, (per {\timescale},)
        over the time interval $t_2 -
        t_1$,~\cite[pp. 177]{Feder},~\cite[pp. 494]{Peitgen}. If $H >
        \frac{1}{2}$, then the time series is termed as being
        characterized by ``fractional Brownian
        motion~\cite[pp. 170]{Feder}.''

        \subidx{rate of revenue returns}{predictability}
        \subidx{rate of revenue returns}{forecastability}
        \subidx{rate of revenue returns}{consistency}
        \subidx{predictability}{rate of revenue returns}
        \subidx{forecastability}{rate of revenue returns}
        \subidx{consistency}{rate of revenue returns}
        \subidx{\market}{rate of revenue returns, predictability}
        \subidx{\market}{rate of revenue returns, forecastability}
        \subidx{\market}{rate of revenue returns, consistency}
        \subidx{Hurst coefficient}{analysis}
        \subidx{\market}{Hurst coefficient analysis}
        \subidx{\market}{rate of change}

        In some sense, the Hurst coefficient is a quantitative
        expression of the ``forecastability'' of the future based on
        the past\footnote{Actually, in general, when summing fractal
        entities, the method used should be a root mean square
        process, dependent on the Hurst Coefficient, $H$, where
        $P_{total}^H = P_1^H + P_2^H + \cdots$, where $P_n$ is the
        fractal entities. For a Brownian motion, or random walk type
        of fractal the Hurst Coefficient is a function of time into
        the future. For the ``near term,'' the Hurst coefficient is
        very near unity, meaning the summation process is linear. For
        the ``long term,'' $H \approx 0.5$, or a standard root mean
        square summation process should be used. If $H$ is $0.5$ then
        the market is termed a Brownian motion, or random walk
        process. If it is larger than 0.5, it is termed fractional
        Brownian motion process. For a random walk process, ``near
        term'' and ``far term'' are quantitatively differentiated on
        the Hurst Coefficient graph where $1 - \ln (t) = 0.5 \cdot \ln
        (t)$, or when $\ln (t) = 2$, or $t = 7.389\ldots$ See
        Section~\ref{\SETLABEL:FS} for the particulars on using Hurst
        Coefficient to sum fractal process' for the {\market}. See
        also~\cite[pp. 67, 83-84]{Peters:CAOITCM} and~\cite[pp. 129,
        159]{Schroeder} for particulars on the implications of the
        Hurst Coefficient and root mean square summation issues.}.  A
        Hurst coefficient of {\thurstlow}, (for the near future, and
        {\thurstall} for the distant future.) implies that the
        likelihood of the rate of revenue returns, (per {\timescale},)
        for any two consecutive {\timescale}s being the same is
        {\thurstlowhundred}\%~\cite[pp. 66]{Peters:CAOITCM} for the
        near future, and {\thurstall} for the distant
        future. Likewise, there is a {\thurstlowhundred}\% chance of
        the rate of revenue returns, (per {\timescale},) movements
        being the same in consecutive time periods---ie., if, in a
        given {\timescale}, the rate of revenue returns, (per
        {\timescale},) is increasing, there is a {\thurstlowhundred}\%
        that the rate of revenue returns, (per {\timescale},) will
        increase in the following period, also. In some sense, this is
        a quantitative statement on how ``predictable,'' or
        ``forecastable'' the rate of revenue returns, (per
        {\timescale},) for the {\market} are over time, since the
        probability of having $n$ many consecutive {\timescale}s of
        the same agenda is $H^n$ where $H$ is the Hurst coefficient,
        or, letting the short term probability of having $n$ many
        {\timescale}s of the same market agenda, $p_a$, is:

        \begin{eqnarray}
            p_a\left(n\right) & = & H^{n}\\
                              & = & {\thurstlow}^{n}
            \label{\SETLABEL:MA}
        \end{eqnarray}

        \subidx{rate of revenue returns}{predictability}
        \subidx{rate of revenue returns}{forecastability}
        \subidx{rate of revenue returns}{consistency}
        \subidx{predictability}{rate of revenue returns}
        \subidx{forecastability}{rate of revenue returns}
        \subidx{consistency}{rate of revenue returns}
        As an interesting interpretation of the normalized increments
        of the time series data presented in
        Figure~\ref{\SETLABEL:TF}, if the vertical axis is multiplied
        by 100, to convert to percent, then the graph represents the
        error, in percent, that would be made by forecasting, month by
        month, that the next {\timescale}'s rate of revenue returns
        would be the same as the current {\timescale}'s revenue
        rate. Interestingly, it is $\datafractionmean \cdot 100$
        percent, on the average, with a standard deviation of
        $\datafractionstddev \cdot 100$ percent, and a root mean
        square error value of $\datafractionrms \cdot 100$
        percent---small values for such a simple forecasting
        mechanism.

        \subidx{\market}{rate of revenue returns, range}
        \subidx{Hurst coefficient}{analysis}
        \subidx{\market}{Hurst coefficient analysis}
        \subidx{\market}{rate of change}

        This is, essentially, a statement of the range of values, in
        the increments of the rate of revenue returns, (per
        {\timescale},) that is to be expected over the time interval,
        $t_2 - t_1$,
        $R_v$,~\cite[pp. 178]{Feder},~\cite[pp. 172]{Cambel}:

        \begin{eqnarray}
            R_v\left(t_2 - t_1\right) & \propto & \left(t_2 - t_1\right)^{H}\\
                                      & \propto & \left(t_2 - t_1\right)^{\thurstlow}
            \label{\SETLABEL:R}
        \end{eqnarray}

        \subidx{\market}{rate of revenue returns, range}
        \subidx{Hurst coefficient}{analysis}
        \subidx{\market}{Hurst coefficient analysis}
        \subidx{\market}{rate of change}
        \subidx{Markov}{statistics}
        \subidx{statistics}{Markov}
        \noindent where $R$ is the range of values in the increments
        of the rate of revenue returns, (per {\timescale}.) A Hurst
        coefficient, $H$, that is much larger than $\frac{1}{2}$, (but
        less than 1,) implies a strongly non-Gaussian distribution in
        the increments of the rate of revenue returns, (per
        {\timescale},)~\cite[pp. 152, 194]{Feder}, and a Hurst
        coefficient near $\frac{1}{2}$ implies that the increments of
        the rate of revenue returns, (per {\timescale}) is
        characteristic of an independent
        process~\cite[pp. 195]{Feder}. Extreme caution should be
        exercised in using Markov statistics in any analysis where the
        Hurst coefficient is not
        $\frac{1}{2}$,~\cite[pp. 124]{Crownover},~\cite[pp. 106]{Peters:CAOITCM}.


        As a useful approximation, if $H$, is approximately
        $\frac{1}{2}$, Equation~\ref{\SETLABEL:R} reduces
        to,~\cite[pp. 129]{Schroeder}:

        \begin{eqnarray}
            R\left(t_2 - t_1\right) & \propto & (t_2 - t_1)^{\frac{1}{2}}\\
                                    & \propto & \sqrt{\left(t_2 - t_1\right)}
        \end{eqnarray}

        \subidx{\market}{rate of revenue returns, range}
        \subidx{\market}{rate of revenue returns, increase and decrease}
        \subidx{Hurst coefficient}{analysis}
        \subidx{\market}{Hurst coefficient analysis}
        \subidx{\market}{rate of change}
        \subidx{Markov}{statistics}
        \subidx{statistics}{Markov}

        In the case where the Hurst coefficient, $H$, is
        $\frac{1}{2}$, the range of values in the increments of the
        rate of revenue returns, (per {\timescale},) divided by the
        standard deviation of these values, $S$, can be anticipated to
        increase over time according to the following
        relation,~\cite[pp. 154]{Feder},~\cite[pp. 129]{Schroeder}:

        \begin{equation}
            \frac{R\left(t_2 - t_1\right)}{S} \propto \left(t_2 - t_1\right)^{\frac{1}{2}}
        \end{equation}

        \subidx{\market}{rate of revenue returns, range}
        \subidx{\market}{rate of revenue returns, increase and decrease}
        \subidx{Hurst coefficient}{analysis}
        \subidx{\market}{Hurst coefficient analysis}
        \subidx{\market}{rate of change}
        \noindent which is a useful conceptual approximation, since it
        involves only the square root function---if the range and the
        standard deviation of the increments of the rate of revenue
        returns, (per {\timescale},) are known, (and $H \approx
        \frac{1}{2}$,) then the expected change in $\frac{R}{S}$, will
        increase with the square root of time\footnote{To be precise,
        it is actually asymptotically proportional to
        $\tau^{\frac{1}{2}}$}.

        Another useful approximation when rescaling processes that are
        characterize by Brownian motion, (ie., when $H \approx
        \frac{1}{2}$,) is that:

        \begin{eqnarray}
            X\left(t\right) & \propto & \frac{X\left(rt\right)}{r^{H}}\\
                            & \propto & \frac{X\left(rt\right)}{r^{\thurstlow}}
        \end{eqnarray}

        \idx{Brownian motion}
        \idx{fractal}
        Where $X(t)$ is the process characterized by Brownian motion,
        and $r$ is a scaling factor,~\cite[pp. 494]{Peitgen}.

        \subidx{programs}{tslsq}
        \subidx{tslsq}{program}
        The program {\it tslsq}\/ was used on the H parameter data,
        presented in Figure~\ref{\SETLABEL:HP}, to provide a least
        squares approximation to the H parameter for the
        {\market}. The superimposed least squares approximation on the
        original H parameter data is presented.  By contrast, the H
        parameter, as derived by the methodology outlined
        in~\cite[pp. 249]{Crownover}, is {\thcalclow} for the near
        future, and {\thcalcall} for the distant future.

        \subidx{\market}{Hurst coefficient analysis}
        \subidx{Hurst coefficient}{analysis}
        \subidx{increments}{normalized}
        \subidx{normalized}{increments}
        \subidx{programs}{tshurst}
        \subidx{tshurst}{program}
        \subidx{\market}{H parameter analysis}
        \subidx{H parameter}{analysis}
        \subidx{programs}{tshcalc}
        \subidx{tshcalc}{program}
        Figures~\ref{\SETLABEL:HC} and~\ref{\SETLABEL:HP} represent
        Hurst coefficient and H parameter data that are derived from
        the normalized increments, shown in
        Figure~\ref{\SETLABEL:TF}. In this case, the data is
        considered a normalized derivative of the time series data
        presented in Figure~\ref{\SETLABEL:TF}, instead of a
        cumulative sum.  The program, {\it tshurst}\/, is described
        briefly in appendix~\ref{programs}, and the data for
        figures~\ref{\SETLABEL:THC} and~\ref{\SETLABEL:THP} was made
        using the -d option.

        \begin{figure}[ht]
            \begin{center}
                \begin{minipage}[t]{0.45\textwidth}
                    \epsfxsize=1.0\linewidth
                    \epsffile{\directory/data.tsfraction.tshurst-d.eps}
                    \caption[{\market}, traditional Hurst coefficient
                        data]{{\market}, traditional Hurst coefficient
                        data for the time series data shown in
                        Figure~\ref{\SETLABEL:TS}.  The slope of the
                        graph is the Hurst coefficient, and is
                        {\hurstlow} for the near term, and
                        {\hurstall} for the far term.}
                    \label{\SETLABEL:THC}
                \end{minipage}
                \hfill
                \begin{minipage}[t]{0.45\textwidth}
                    \epsfxsize=1.0\linewidth
                    \epsffile{\directory/data.tsfraction.tshcalc-d.eps}
                    \caption[{\market}, traditional H parameter
                        data]{{\market}, traditional H parameter data
                        for the time series data shown in
                        Figure~\ref{\SETLABEL:TS} The slope of the
                        graph is the H parameter, and is {\hcalclow}
                        for the near term, and {\hcalcall} for the
                        far term.}
                    \label{\SETLABEL:THP}
                \end{minipage}
            \end{center}
        \end{figure}

% Local Variables:
% TeX-parse-self: t
% TeX-auto-save: t
% TeX-master: "fractal.tex"
% End:


        %
% -----------------------------------------------------------------------------
%
% A license is hereby granted to reproduce this software source code and
% to create executable versions from this source code for personal,
% non-commercial use.  The copyright notice included with the software
% must be maintained in all copies produced.
%
% THIS PROGRAM IS PROVIDED "AS IS". THE AUTHOR PROVIDES NO WARRANTIES
% WHATSOEVER, EXPRESSED OR IMPLIED, INCLUDING WARRANTIES OF
% MERCHANTABILITY, TITLE, OR FITNESS FOR ANY PARTICULAR PURPOSE.  THE
% AUTHOR DOES NOT WARRANT THAT USE OF THIS PROGRAM DOES NOT INFRINGE THE
% INTELLECTUAL PROPERTY RIGHTS OF ANY THIRD PARTY IN ANY COUNTRY.
%
% Copyright (c) 1994-2006, John Conover, All Rights Reserved.
%
% Comments and/or bug reports should be addressed to:
%
%     john@email.johncon.com (John Conover)
%
% -----------------------------------------------------------------------------
%
% Revision: \RCSRevision \\
% Revision Time: \RCSTime UMT \\
% Revision Date: \RCSDate \\
% Revision Id: \RCSId \\
% Revision File: \RCSLog \\
\RCS $Revision: 0.0 $
\RCS $Date: 2006/01/20 04:38:13 $
\RCS $Id: fiscal.tex,v 0.0 2006/01/20 04:38:13 john Exp $
% $Log: fiscal.tex,v $
% Revision 0.0  2006/01/20 04:38:13  john
% Initial version
%
%
    \subsection{Fixed Increment Approximation for Fiscal Strategy}
        \label{\SETLABEL:FS}

        \subidx{\market}{fiscal strategy}
        \subidx{markets}{analysis}
        \subidx{analysis}{markets}
        \subidx{strategy}{fiscal}
        \subidx{fiscal}{strategy}
        The data in this section is presented in tabular form in
        Section~\ref{\SETLABELREF:LR}. This section derives various
        values based on the ``average'' of the normalized increments
        presented in Figure~\ref{\SETLABEL:TFA}. These values are an
        approximation to a, probably, complex process with a
        distribution shown in Figure~\ref{\SETLABEL:TF}. These values
        will be used in a fixed increment Brownian fractal analysis
        and simulation of the {\market}, and may, or may not, provide
        adequate accuracy for projections.

        For an organization operating in the {\market}, the fiscal
        strategy, commensurate with the aggregate environment, can be
        derived as follows~\cite[pp. 128, pp
        151]{Schroeder},~\cite[pp. 450]{Reza},~\cite[pp. 270]{Pierce}:
        \vspace{0.15in}

        \subsubsection{Logarithmic Returns}
            \label{\SETLABEL:LR}

            \subidx{logarithmic}{returns}
            \subidx{returns}{logarithmic}
            \subidx{\market}{logarithmic returns}
            The logarithmic returns can be calculated by various
            means. Four will be presented here, for comparison.

            \subidx{programs}{tsnormal}
            \subidx{tsnormal}{program}
            \subidx{logarithmic}{returns}
            \subidx{returns}{logarithmic}
            The logarithmic returns, in bits, $bits$, as computed from
            the mean, by the program {\it tsnormal}\/, which is
            described in Chapter~\ref{programs}, and is presented in
            Figure~\ref{\SETLABEL:TF}, and Equation~\ref{abits} from
            Section~\ref{ereturns} in Chapter~\ref{general}:

            \begin{equation}
                bits = \frac{\ln \left({\datafractionmean} + 1\right)}{\ln \left(2\right)} = \datafractionmeanbits
            \end{equation}

            \subidx{programs}{tslsq}
            \subidx{tslsq}{program}
            \subidx{logarithmic}{returns}
            \subidx{returns}{logarithmic}
            \noindent By comparison, the logarithmic returns, in bits,
            $bits$, as computed from the constant in the least squares
            approximation, using the program {\it tslsq}\/, which is briefly
            described in Chapter~\ref{programs}, as presented in
            Figure~\ref{\SETLABEL:TF}, and Equation~\ref{abits} from
            Section~\ref{ereturns} in Chapter~\ref{general}:

            \begin{equation}
                bits = \frac{\ln \left({\datafractionconstant} + 1\right)}{\ln \left(2\right)} = \datafractionconstantbits
            \end{equation}

            Note that if the mean is not constant in
            Figure~\ref{\SETLABEL:TF}, this method will not provide
            accurate results.

            \subidx{programs}{tslsq}
            \subidx{tslsq}{program}
            \subidx{logarithmic}{returns}
            \subidx{returns}{logarithmic}
            \noindent And by yet another comparison, using the program
            {\it tslsq}\/, which is briefly described in
            Chapter~\ref{programs}, with the -e -p options, to provide
            a formula for the least squares exponential fit to the
            time series data set presented in
            Figure~\ref{\SETLABEL:TS}:

            \begin{equation}
                bits = {\datatslsqepbits}
            \end{equation}

            \subidx{programs}{tslogreturns}
            \subidx{tslogreturns}{program}
            \subidx{logarithmic}{returns}
            \subidx{returns}{logarithmic}
            \noindent And finally, by comparison, from the
            {\it tslogreturns}\/ program, which is briefly described
            in Chapter~\ref{programs}, with the -p option, to provide
            a formula for the logarithmic returns of the time series
            data set presented in Figure~\ref{\SETLABEL:TS}:

            \begin{equation}
                bits = {\logreturns}
            \end{equation}

        \subsubsection{Calculation of Shannon Probability}
            \label{\SETLABEL:SP}

            \subidx{\market}{Shannon probability}
            Ideally, all of the values presented in
            Section~\ref{\SETLABEL:LR} would be equal. Using the
            logarithmic returns provided by the {\it tslogreturns}\/
            program, to be consistent
            with~\cite[pp. 81]{Peters:CAOITCM}

            \subidx{programs}{tslogreturns}
            \subidx{tslogreturns}{program}
            \begin{equation}
                2^{{\logreturns}t}
            \end{equation}

            \noindent therefore:
            \begin{equation}
                C\left(p\right) = {\logreturns}
            \end{equation}
            \subidx{programs}{tsshannon}
            \subidx{tsshannon}{program}
            \subidx{Shannon}{probability}
            \subidx{probability}{Shannon}
            \noindent and, {\it tsshannon}\/ {\logreturns} gives:
            \begin{equation}
                \label{\SETLABEL:F0}
                C\left({\shannonlogreturns}\right) = {\logreturns}
            \end{equation}
            \noindent therefore:
            \begin{eqnarray}
                2^{C\left({\shannonlogreturns}\right)} & = & 2^{\logreturns}\\
                                                       & = & {\twologreturns}\\
                                                       & = & {\twologreturnshundred}\%
            \end{eqnarray}
            \noindent and:
            \begin{eqnarray}
                2p - 1 & = & \left(2 \cdot {\shannonlogreturns}\right) - 1\\
                       & = & {\twopone}\\
                       \label{\SETLABEL:F1}
                       & = & {\twoponehundred}\%
            \end{eqnarray}

            \subidx{\market}{fiscal strategy}
            \subidx{markets}{analysis}
            \subidx{analysis}{markets}
            \subidx{strategy}{fiscal}
            \subidx{fiscal}{strategy}
            \subidx{\market}{fiscal strategy}
            \subidx{\market}{growth rate}
            Presuming the simplified assumptions outlined in
            Section~\ref{assumptions}, the ``typical'' organization
            operating in the {\market} executes a long term fiscal
            strategy, commensurate with the aggregate environment,
            that is to invest, every {\timescale}, in sufficient
            additional resources and infrastructure, to increase the
            manufacturing of goods and services by {\twoponehundred}\%
            of its rate of revenue returns, (per {\timescale}.) As a
            conceptual model, the remaining {\hundredtwoponehundred}\%
            will be held in ``reserve'' with a
            {\shannonlogreturnshundred}\% chance of making twice the
            {\twoponehundred}\% back, (and a
            {\hundredshannonlogreturnshundred}\% chance of making
            0.0,) in one {\timescale}, on the average, for an average
            growth in its rate of revenue returns, (per {\timescale},)
            of {\twologreturnshundred}\%, or a doubling of its rate of
            revenue returns, (per {\timescale},) in
            {\oneoverlogreturns} {\timescale}s.

        \subsubsection{Example Fixed Increment Approximation Fiscal Strategies}

            \subidx{\market}{fiscal strategy}
            \subidx{markets}{analysis}
            \subidx{analysis}{markets}
            \subidx{strategy}{fiscal}
            \subidx{fiscal}{strategy}
            \subidx{\market}{fiscal strategy}
            \subidx{\market}{growth rate}
            \subidx{\market}{management metric}
            \idx{management metric}
            A possible metric on the effectiveness of long term fiscal
            management could possibly be that if an investment of
            {\twoponehundred}\% per {\timescale} of the rate of
            revenue returns, (per {\timescale},) is made in resources
            and infrastructure, then the rate of revenue returns would
            be expected to increase by {\twologreturnshundred}\%, per
            {\timescale}, on average.

            Note that the metrics presented in this section are
            representative of the {\market} as an aggregate whole, and
            may or may not be accurate representations for any
            particular participant in the environment. Of interest to
            the participants in the environment would be a similar
            analysis of each product or service rendered in the
            marketplace.

            \subidx{\market}{fiscal strategy}
            \subidx{markets}{analysis}
            \subidx{analysis}{markets}
            \subidx{strategy}{fiscal}
            \subidx{fiscal}{strategy}
            \subidx{\market}{fiscal strategy}
            As a simple illustrative example, a company operating in
            this environment might obtain a credit line from a bank
            that is equal to {\twoponehundred}\% of its rate of
            revenue returns, (per {\timescale},) to finance additional
            operations. In this simple scenario, the company would use
            its revenue base as collateral for the loan. Some
            {\timescale}s, depending on the {\market}'s environment,
            the company's rate of revenue returns exceeds what was
            borrowed from the bank, and the loan is repaid in
            full. Other {\timescale}s, the company must default, and
            the bank seizes a portion of the company's revenue base to
            pay the delinquent loan. However, on the average, the
            company will expand its rate of revenue returns at
            {\twologreturnshundred}\% per {\timescale}.

            \subidx{\market}{fiscal strategy}
            \subidx{markets}{analysis}
            \subidx{analysis}{markets}
            \subidx{strategy}{fiscal}
            \subidx{fiscal}{strategy}
            \subidx{\market}{fiscal strategy}
            As another simple example, a company re-invests
            {\twoponehundred}\% of its rate of revenue returns, (per
            {\timescale},) in development, marketing, sales, and
            distribution of new products.  Although some products will
            be successful and the return on the investment will exceed
            the {\twoponehundred}\% per {\timescale} investment,
            others will not. However, on the average, the company will
            expand it gross rate of revenue returns at
            {\twologreturnshundred}\% per {\timescale}.

            \subidx{\market}{fiscal strategy}
            \subidx{markets}{analysis}
            \subidx{analysis}{markets}
            \subidx{strategy}{fiscal}
            \subidx{fiscal}{strategy}
            \subidx{\market}{fiscal strategy}
            \subidx{\market}{product portfolio}
            \subidx{\market}{product diversity}
            \subidx{\market}{product mix}
            \subidx{\market}{optimum number of products}
            \idx{product portfolio}
            \idx{product diversity}
            \idx{optimum number of products}
            \idx{product mix}

            As an example of ``product portfolio'' management, suppose
            a company re-invests {\twoponehundred}\% of its rate of
            revenue returns, (per {\timescale},) in development,
            marketing, sales, and distribution of new products.
            Further suppose that the company has two products, and a
            fractal analysis of the individual product rate of revenue
            return time series indicates that one product has a
            Shannon probability of 0.65, and the other has a Shannon
            probability of 0.55. Then the percentage of re-investment
            in the first product would be $(2 \cdot 0.65 - 1) \cdot
            {\twoponehundred}$, percent of the rate of revenue
            returns, and $(2 \cdot 0.55 - 1) \cdot {\twoponehundred}$
            percent for the second product, implying that the company
            should diversify its product line\footnote{The astute
            reader would note that the linear addition was used to add
            the contribution to development of each product. This is a
            ``near term'' interpretation. Actually, in general, the
            method used should be a root mean square process,
            dependent on the Hurst Coefficient, $H$, where
            $P_{total}^H = P_1^H + P_2^H + \cdots$, where $P_n$ is the
            contribution to each individual product. For a Brownian
            motion, or random walk type of fractal the Hurst
            Coefficient is a function of time into the future. For the
            ``near term,'' the Hurst coefficient is very near unity,
            meaning the summation process is linear. For the ``long
            term,'' $H \approx 0.5$, or a standard root mean square
            summation process should be used. If $H$ is $0.5$ then the
            market is termed a Brownian motion, or random walk
            process. If it is larger than 0.5, it is termed fractional
            Brownian motion process. For a random walk process, ``near
            term'' and ``far term'' are quantitatively differentiated
            on the Hurst Coefficient graph where $1 - \ln (t) = 0.5
            \cdot \ln (t)$, or when $\ln (t) = 2$, or $t =
            7.389\ldots$ See~\cite[pp. 67, 83-84]{Peters:CAOITCM}
            and~\cite[pp. 129, 159]{Schroeder} for particulars on the
            implications of the Hurst Coefficient and root mean square
            summation issues.}.  Note that this is a ``bet hedging''
            metric methodology, and assumes that the products have
            uncorrelated revenue return rates. If this re-investment
            methodology is not feasible, perhaps for strategic
            financial reasons, then the re-investment in both products
            should total the ${\twoponehundred}$\%, and the investment
            in each product should be made at a ratio of $\frac{(2
            \cdot 0.65 - 1)}{(2 \cdot 0.55 - 1)} = 3 : 1$,
            respectively. Note that this ``bet hedging'' can be used
            to define the optimal number of products that can be
            supported on the rate of revenue returns. If it assumed
            that all products are ``typical'' for the {\market}, as a
            standard bench mark, then the optimal number will be
            $\frac{1}{{\twopone}}$. Note that this is a
            ``theoretical'' value, since not all products are
            ``typical,'' and there may be strategic reasons, for
            example product leveraging, that may increase the number
            of products above the optimum. However, most of the
            revenue should come from the optimal number of products,
            since having more products will decrease the amount of the
            potential investment in each product, and having less than
            the optimum number of products will increase the risk that
            many of the products could suffer a ``down market''
            concurrently, impacting the rate of revenue returns.  As
            another interesting interpretation of the optimal
            ``hedging of bets,'' in product portfolio strategy, and
            considering the graph of the normalized increments
            presented in Figure~\ref{\SETLABEL:TF}, if the
            organization is running optimally, then these products
            will generate, at least in principle, one standard
            deviation, approximately $0.8413 = 84.13$\% of the future
            growth in rate of revenue returns. Naturally, these are
            approximations, and the values are an approximation to a,
            probably, complex process, and appropriate scrutiny should
            be exercised before making specific projections.  As yet
            another example of ``product portfolio'' management,
            consider the issue of product mix. In this interpretation,
            {\twoponehundred}\% of the product manufactured should be
            ``proprietary,'' while the rest is ``industry standard.''
            As yet another possibility, {\twoponehundred}\% of the
            product manufactured should be predatory into new markets,
            and the remainder in markets that are ``traditional'' for
            the company.

% Local Variables:
% TeX-parse-self: t
% TeX-auto-save: t
% TeX-master: "fractal.tex"
% End:


        \subsubsection{Observations on the Fixed Increment Approximation for Fiscal Strategy}

            A re-investment of {\twoponehundred} of the rate of
            revenue returns per {\timescale} does not seem
            inconsistent with the industry averages, since it includes
            investments in research and development, additional
            manufacturing infrastructure, advertising,
            etc. Additionally, a product mix of {\twoponehundred}\%
            ``proprietary'' and the remainder ``industry standard''
            products seems consistent with the industry analyst
            ``20/80'' rule. The value of one standard deviation,
            $84.13$\%, of the revenue return rate being generated by
            $\frac{1}{{\twopone}}$ products seems consistent with the
            industry, also.

        %
% -----------------------------------------------------------------------------
%
% A license is hereby granted to reproduce this software source code and
% to create executable versions from this source code for personal,
% non-commercial use.  The copyright notice included with the software
% must be maintained in all copies produced.
%
% THIS PROGRAM IS PROVIDED "AS IS". THE AUTHOR PROVIDES NO WARRANTIES
% WHATSOEVER, EXPRESSED OR IMPLIED, INCLUDING WARRANTIES OF
% MERCHANTABILITY, TITLE, OR FITNESS FOR ANY PARTICULAR PURPOSE.  THE
% AUTHOR DOES NOT WARRANT THAT USE OF THIS PROGRAM DOES NOT INFRINGE THE
% INTELLECTUAL PROPERTY RIGHTS OF ANY THIRD PARTY IN ANY COUNTRY.
%
% Copyright (c) 1994-2006, John Conover, All Rights Reserved.
%
% Comments and/or bug reports should be addressed to:
%
%     john@email.johncon.com (John Conover)
%
% -----------------------------------------------------------------------------
%
% Revision: \RCSRevision \\
% Revision Time: \RCSTime UMT \\
% Revision Date: \RCSDate \\
% Revision Id: \RCSId \\
% Revision File: \RCSLog \\
\RCS $Revision: 0.0 $
\RCS $Date: 2006/01/20 04:38:13 $
\RCS $Id: companies.tex,v 0.0 2006/01/20 04:38:13 john Exp $
% $Log: companies.tex,v $
% Revision 0.0  2006/01/20 04:38:13  john
% Initial version
%
%
    \subsection{Number of Companies}
        \label{\SETLABEL:QNC}

        \subidx{\market}{number of companies}
        \subidx{number of companies}{analysis}
        \subidx{analysis}{number of companies}
        \subidx{Shannon}{probability}
        \subidx{probability}{Shannon}
        This section evaluates the approximate, or ``average,'' number
        of companies in the {\market}, and uses the method outlined in
        Chapter~\ref{general}, Section~\ref{aftsma}. Since the
        average, $avg_{ind}$, and the root mean square, $rms_{ind}$,
        of the normalized increments of the {\market} time series is
        \datafractionmean, and \datafractionrms respectively, the
        number of companies participating in the market can be
        calculated by Equation~\ref{ncompanies} to be {\ncompanies}.

        If this value seems consistent number of companies in the
        {\market}, within the assumptions outlined in
        Chapter~\ref{general}, Section~\ref{aftsma}, then it would
        seem that there is some circumstantial or indirect evidence
        that the companies participating in the {\market} are
        operating optimally, and the ``average'' Shannon probability,
        $P$ for each participating company would be, using
        Equation~\ref{pncompanies}, {\pncompanies}, which would be the
        value which should be used in Section~\ref{\SETLABEL:FS} for
        each participating company if market expansion was to be
        consistent with the rest of the industry. However, if the
        Shannon probability derived in Section~\ref{\SETLABEL:FS} is
        greater than the average Shannon probability for the companies
        participating in the {\market}, as derived in this section,
        then the market would, possibly, be exploitable with the
        fiscal strategy outlined in Section~\ref{\SETLABEL:FS}. The
        maximum exploitability for the {\market} is derived in
        Section~\ref{\SETLABEL:MAXSHANNON}, but it is probably of
        doubtful practicality.

        Note that these optimizations would maximize a company's
        market growth. Since there are probably many companies
        competing in the market place, this would not necessarily
        maximize a company's P\&L, as described in
        Chapter~\ref{general}, Section~\ref{ompl}. The Shannon
        probability that maximizes market share in the {\market} is
        \pncompanies, with several alternative solutions listed in the
        previous paragraph. However, these should be contrasted to the
        Shannon probability that maximizes a company's P\&L which is
        \avgrms~in the {\market}. In all cases, the fraction of the
        P\&L that should be ``wagered'' on the future, $f$, should be:

        \begin{equation}
            f = 2P - 1
        \end{equation}

        \noindent where $P$ is the particular Shannon probability
        chosen optimize a particular fiscal strategy. Interestingly,
        the measured Shannon probability of the {\market} would tend
        to indicate that the companies participating in the market
        have chosen a fiscal strategy that optimizes market growth, as
        opposed to capital growth.

        \subidx{\market}{increasing returns}
        \subidx{economic increasing returns}{\market}
        As interesting interpretation of these exploitive issues,
        since all three fiscal strategies will result in exponential
        market growth for every company participating in the market,
        is that they may represent, perhaps, an example of
        ``increasing returns.''

% Local Variables:
% TeX-parse-self: t
% TeX-auto-save: t
% TeX-master: "fractal.tex"
% End:


        %
% -----------------------------------------------------------------------------
%
% A license is hereby granted to reproduce this software source code and
% to create executable versions from this source code for personal,
% non-commercial use.  The copyright notice included with the software
% must be maintained in all copies produced.
%
% THIS PROGRAM IS PROVIDED "AS IS". THE AUTHOR PROVIDES NO WARRANTIES
% WHATSOEVER, EXPRESSED OR IMPLIED, INCLUDING WARRANTIES OF
% MERCHANTABILITY, TITLE, OR FITNESS FOR ANY PARTICULAR PURPOSE.  THE
% AUTHOR DOES NOT WARRANT THAT USE OF THIS PROGRAM DOES NOT INFRINGE THE
% INTELLECTUAL PROPERTY RIGHTS OF ANY THIRD PARTY IN ANY COUNTRY.
%
% Copyright (c) 1994-2006, John Conover, All Rights Reserved.
%
% Comments and/or bug reports should be addressed to:
%
%     john@email.johncon.com (John Conover)
%
% -----------------------------------------------------------------------------
%
% Revision: \RCSRevision \\
% Revision Time: \RCSTime UMT \\
% Revision Date: \RCSDate \\
% Revision Id: \RCSId \\
% Revision File: \RCSLog \\
\RCS $Revision: 0.0 $
\RCS $Date: 2006/01/20 04:38:13 $
\RCS $Id: operations.tex,v 0.0 2006/01/20 04:38:13 john Exp $
% $Log: operations.tex,v $
% Revision 0.0  2006/01/20 04:38:13  john
% Initial version
%
%
    \subsection{Fixed Increment Approximation for Operational Strategy}
        \label{\SETLABEL:OPS}.

        This section derives various values based on the ``average''
        of the normalized increments presented in
        Figure~\ref{\SETLABEL:TFA}. These values are an approximation
        to a, probably, complex process with a distribution shown in
        Figure~\ref{\SETLABEL:TF}. These values will be used in a
        fixed increment Brownian fractal analysis and simulation of
        the {\market}, and may, or may not, provide adequate accuracy
        for projections.

        \subidx{\market}{fiscal strategy}
        \subidx{\market}{Shannon probability}
        \subidx{strategy}{fiscal}
        \subidx{fiscal}{strategy}
        \subidx{Shannon}{probability}
        \subidx{probability}{Shannon}
        It should be noted that the analysis of fiscal strategy,
        presented in Section~\ref{\SETLABEL:FS}, is derived from the
        {\market} metrics and may, or may not, be maximally
        optimal. For the optimal fiscal strategy, which may be
        exploitable, see Section~\ref{\SETLABEL:MAXSHANNON}.

        \subidx{strategy}{exploitable}
        \subidx{exploitable}{strategy}
        \subidx{\market}{windows of opportunity}
        \idx{windows of opportunity}
        \subidx{decision}{obsolete}
        \subidx{obsolete}{decision}
        \subidx{decision}{timeliness}
        \subidx{timeliness}{decision}
        \subidx{rate of revenue returns}{forecast}
        \subidx{forecast}{rate of revenue returns}
        An additional exploitable strategy may be time itself.
        Equations~\ref{\SETLABEL:V},~\ref{\SETLABEL:R},
        and,~\ref{\SETLABEL:MA}, are, essentially, metrics on how fast
        a decision, which is based on information concerning the
        current status of the {\market}, becomes obsolete. Obviously,
        how long a decision is expected to remain relevant should be
        addressed as an operational necessity in strategic planning
        and project management. Figures~\ref{\SETLABEL:FN},
        and,~\ref{\SETLABEL:FF} compare methods of approximation of
        the ``forecastability'' of rate of revenue returns in the
        {\market} for the near term and far
        term~\cite[pp. 83-84]{Peters:CAOITCM}, respectively. As a
        general rule, caution must be exercised when making decisions
        that will span a time interval larger than the time interval
        where the ``forecastability'' of rate of revenue returns drops
        below 50\%. Beyond this time interval, the chances increase
        that the competitive and market forces will alter the market
        environment in a possibly detrimental unanticipated
        fashion. Obviously, there is significant advantage in
        ``timeliness'' of development, manufacturing, and distribution
        of products and services that are consistent with this
        temporal agenda. Automation of these processes, if executed
        consistently with this agenda, should be considered a
        competitive advantage.

        \subidx{strategy}{exploitable}
        \subidx{exploitable}{strategy}
        \subidx{rate of revenue returns}{forecast}
        \subidx{forecast}{rate of revenue returns}
        \idx{product life cycle}
        \idx{life cycle, product}
        In some sense, this temporal agenda defines the ``average''
        product or service life cycle in the {\market}. When the
        ``forecastability'' of rate of revenue returns drops below
        50\%, there is an even chance that the rate of revenue returns
        for the product or service will change in a detrimental
        fashion. If it is assumed that a product or service life cycle
        consists of a ramp up, a maintenence interval, and a ramp
        down, then, if all three life cycle intervals are equal, the
        product life cycle will be, approximately, three times the
        time interval where the ``forecastability'' of rate of revenue
        returns drops below 50\%. Although probably not an accurate
        prediction of product or service life cycle, the technique may
        be used as a conceptual approximation to the dynamics of
        ``market windows.\footnote{For example, consider the market
        for table salt. Since it has inelastic supply and demand
        curves, and is a necessary requirement for life, it would be
        expected that the Hurst coefficient would be very near
        unity---ignoring competitive pressures in the market. The
        predictability of the table salt market would, therefore, be
        expected to be relatively good, over time.}''  The conceptual
        approximation will probably predict a ``conservative'' or
        ``pessimistic'' value in relation to actual markets.

        \begin{figure}[ht]
            \begin{center}
                \begin{minipage}[t]{0.45\textwidth}
                    \epsfxsize=1.0\linewidth
                    \epsffile{\directory/datahurstlownear.eps}
                    \caption[{\market}, ``forecastability'' of near
                        term rate of revenue returns]{{\market},
                        ``forecastability'' of near term rate of
                        revenue returns. Although the error function
                        is the most accurate, for the near term,
                        $H^{t} = \thurstlow^{t}$ may be used as a
                        reliable metric of ``forecastability'' of the
                        rate of revenue returns.}
                    \label{\SETLABEL:FN}
                \end{minipage}
                \hfill
                \begin{minipage}[t]{0.45\textwidth}
                    \epsfxsize=1.0\linewidth
                    \epsffile{\directory/datahurstlowfar.eps}
                    \caption[{\market}, ``forecastability'' of far
                        term rate of revenue returns]{{\market},
                        ``forecastability'' of far term rate of
                        revenue returns. Although the error function
                        is the most accurate, for the far term,
                        $\frac{1}{\sqrt{t}}$ may be used as a reliable
                        metric of ``forecastability'' of the rate of
                        revenue returns.}
                    \label{\SETLABEL:FF}
                \end{minipage}
            \end{center}
        \end{figure}

        \idx{operations research}
        As an interesting interpretation of the data presented in
        Figure~\ref{\SETLABEL:FN}, there may be, perhaps, some
        applicability to such operational agendas as inventory
        control. Maintaining too little inventory, obviously, will
        create a situation where the organization can not exploit
        market expansion, and maintaining too much inventory,
        likewise, would over extend the company, creating unnecessary
        losses when the market contracts. The company should maintain
        inventory levels that do not exceed, from
        Equation~\ref{\SETLABEL:MA}, ${\thurstlow}^{n} = 0.5$
        {\timescale}s of operations. Since the optimal amount of
        inventory and, from Equation~\ref{\SETLABEL:V}, the variance
        of change in the rate of revenue returns in the future can be
        calculated, there may, perhaps, be some applicability to a
        forecasting methodology that can be incorporated into other
        areas of operations research, for example the linear algebras
        using simplex methodologies for optimization of manufacturing
        processes. Traditionally, these forecasts are made by the
        sales department, and are subject to various subjective
        biases.

% Local Variables:
% TeX-parse-self: t
% TeX-auto-save: t
% TeX-master: "fractal.tex"
% End:


        \subsubsection{Observations on the Fixed Increment Approximation for Operational Strategy}

            As an interesting interpretation of
            Figure~\ref{\SETLABEL:FF}, and evaluating the
            approximation $\frac{1}{\sqrt{t}}$ at 60 months gives a
            probability that the market will still have the same
            agenda of about $0.12909945$, or about 1 in 8. This is
            commensurate with numbers from the venture
            community\footnote{For example, see ``IEEE Engineering
            Management Review,'' Volume 23 Number 3, Fall 1995,
            pp. 83}. Of course new venture backed companies fail for
            many reasons, but market appropriateness to product
            portfolio 60 months in the future may be a major
            contributor. Additionally, the success rate of development
            projects of 8 month duration, which have a market success
            rate of about 1 in 3, seems consistent with
            $\frac{1}{\sqrt{3}} = 0.353553391$. Naturally, projects
            fail in the market for many reasons, but market
            appropriateness, in a dynamic market environment may be a
            major contributor to failure.

            As mentioned in Section~\ref{\SETLABEL:H},
            Equation~\ref{\SETLABEL:MA}, and the preceeding section,
            approximately 3 times the value where ${\thurstlow}^{n} =
            0.5$ could be interpreted as an approximation to the
            ``average'' product life cycle. This seems consistent with
            the 6 to 12 month life cycles quoted by many industry
            analyst. In addition, maintaining inventory levels that do
            not exceed the anticipated requirements of
            $\frac{\ln{0.5}}{\ln{\thurstlow}}$ many {\timescale}s
            seems consistent with the author's experience in the
            industry.

        %
% -----------------------------------------------------------------------------
%
% A license is hereby granted to reproduce this software source code and
% to create executable versions from this source code for personal,
% non-commercial use.  The copyright notice included with the software
% must be maintained in all copies produced.
%
% THIS PROGRAM IS PROVIDED "AS IS". THE AUTHOR PROVIDES NO WARRANTIES
% WHATSOEVER, EXPRESSED OR IMPLIED, INCLUDING WARRANTIES OF
% MERCHANTABILITY, TITLE, OR FITNESS FOR ANY PARTICULAR PURPOSE.  THE
% AUTHOR DOES NOT WARRANT THAT USE OF THIS PROGRAM DOES NOT INFRINGE THE
% INTELLECTUAL PROPERTY RIGHTS OF ANY THIRD PARTY IN ANY COUNTRY.
%
% Copyright (c) 1994-2006, John Conover, All Rights Reserved.
%
% Comments and/or bug reports should be addressed to:
%
%     john@email.johncon.com (John Conover)
%
% -----------------------------------------------------------------------------
%
% Revision: \RCSRevision \\
% Revision Time: \RCSTime UMT \\
% Revision Date: \RCSDate \\
% Revision Id: \RCSId \\
% Revision File: \RCSLog \\
\RCS $Revision: 0.0 $
\RCS $Date: 2006/01/20 04:38:13 $
\RCS $Id: simulation.tex,v 0.0 2006/01/20 04:38:13 john Exp $
% $Log: simulation.tex,v $
% Revision 0.0  2006/01/20 04:38:13  john
% Initial version
%
%
    \subsection{Simulation of Fixed Increment Approximation for Fiscal Strategy}
        \label{\SETLABEL:TSUNFAIRBROWNIAN}

        \subidx{\market}{market simulation}
        The data in this section is presented in tabular form in
        Section~\ref{\SETLABELREF:SIM}.
        Figure~\ref{\SETLABEL:TSUNFAIRBROWNIAN0} represents a
        constructional simulation of the time series data presented in
        Figure~\ref{\SETLABEL:TS}. The program {\it
        tsunfairbrownian}\/, which is briefly described in
        appendix~\ref{programs}, was used in the reconstruction. The
        reconstructed data is superimposed on the original time series
        data.  The program, {\it tsunfairbrownian}\/, essentially,
        constructs the new time series as a Brownian fractal with
        fixed increments---the value of the fixed increment is derived
        from the root mean square average of the normalized increments
        presented in Figure~\ref{\SETLABEL:TF}. The ``quality'' of
        such a reconstruction should be subject to adequate scepticism
        and scrutiny since, in all probability, the normalized
        increments presented in Figure~\ref{\SETLABEL:TF} represent a
        relatively complex process, that may not be ``modeled'' with
        such a simple methodology.

        As a further comparison of the the constructional simulation
        with the original time series data,
        Figure~\ref{\SETLABEL:TSUNFAIRBROWNIAN1} presents a normalized
        histogram of the normalized increments of the reconstructed
        time series, superimposed on the normalized histogram
        presented in Figure~\ref{\SETLABEL:NH}.

        \subidx{\market}{fiscal strategy, simulation}
        \subidx{markets}{simulation}
        \subidx{simulation}{markets}
        \subidx{strategy}{fiscal, simulation}
        \subidx{fiscal}{strategy, simulation}
        \subidx{programs}{tsunfairbrownian}
        \subidx{tsunfairbrownian}{program}
        \begin{figure}[ht]
            \begin{center}
                \begin{minipage}[t]{0.45\textwidth}
                    \epsfxsize=1.0\linewidth
                    \epsffile{\directory/tsunfairbrownian-f.eps}
                    \caption[{\market}, Time series data, empirical and
                        simulated]{{\market}, Time series data, empirical
                        and simulated, using the program {\it tsunfairbrownian}\/
                        with f = {\datafractionrms}. This data is
                        superimposed on the data presented in
                        Figure~\ref{\SETLABEL:TS}.}
                    \label{\SETLABEL:TSUNFAIRBROWNIAN0}
                \end{minipage}
                \hfill
                \begin{minipage}[t]{0.45\textwidth}
                    \epsfxsize=1.0\linewidth
                    \epsffile{\directory/tsunfairbrownian-f.tsfraction.tsnormal-s30.eps}
                    \caption[{\market}, normalized histogram,
                        empirical and simulated]{{\market}, normalized
                        histogram of the normalized increments of the
                        time series data shown in
                        Figure~\ref{\SETLABEL:TSUNFAIRBROWNIAN0},
                        empirical and simulated.  The empirical data
                        has a mean of {\datafractionmean}, with a
                        standard deviation of {\datafractionstddev}.
                        By comparison, the simulated data has a mean
                        of {\tsunfairbrownianfractionmean} with a
                        standard deviation of
                        {\tsunfairbrownianfractionstddev}. This data
                        is superimposed on the data presented in
                        Figure~\ref{\SETLABEL:NH}. The area under the
                        four curves is identical.}
                    \label{\SETLABEL:TSUNFAIRBROWNIAN1}
                \end{minipage}
            \end{center}
        \end{figure}

% Local Variables:
% TeX-parse-self: t
% TeX-auto-save: t
% TeX-master: "fractal.tex"
% End:


        %
% -----------------------------------------------------------------------------
%
% A license is hereby granted to reproduce this software source code and
% to create executable versions from this source code for personal,
% non-commercial use.  The copyright notice included with the software
% must be maintained in all copies produced.
%
% THIS PROGRAM IS PROVIDED "AS IS". THE AUTHOR PROVIDES NO WARRANTIES
% WHATSOEVER, EXPRESSED OR IMPLIED, INCLUDING WARRANTIES OF
% MERCHANTABILITY, TITLE, OR FITNESS FOR ANY PARTICULAR PURPOSE.  THE
% AUTHOR DOES NOT WARRANT THAT USE OF THIS PROGRAM DOES NOT INFRINGE THE
% INTELLECTUAL PROPERTY RIGHTS OF ANY THIRD PARTY IN ANY COUNTRY.
%
% Copyright (c) 1994-2006, John Conover, All Rights Reserved.
%
% Comments and/or bug reports should be addressed to:
%
%     john@email.johncon.com (John Conover)
%
% -----------------------------------------------------------------------------
%
% Revision: \RCSRevision \\
% Revision Time: \RCSTime UMT \\
% Revision Date: \RCSDate \\
% Revision Id: \RCSId \\
% Revision File: \RCSLog \\
\RCS $Revision: 0.0 $
\RCS $Date: 2006/01/20 04:38:13 $
\RCS $Id: maximum.tex,v 0.0 2006/01/20 04:38:13 john Exp $
% $Log: maximum.tex,v $
% Revision 0.0  2006/01/20 04:38:13  john
% Initial version
%
%
    \subsection{Simulation of Fixed Increment Approximation for Optimally Maximal Fiscal Strategy}
        \label{\SETLABEL:MAXSHANNON}
        \subidx{\market}{fiscal strategy, simulation}
        \subidx{\market}{maximum Shannon probability}
        \subidx{markets}{simulation}
        \subidx{simulation}{markets}
        \subidx{strategy}{optimum fiscal, simulation}
        \subidx{fiscal}{optimum strategy, simulation}
        \subidx{programs}{tsunfairbrownian}
        \subidx{tsunfairbrownian}{program}
        \subidx{Shannon}{probability}
        \subidx{probability}{Shannon}

        \subidx{strategy}{exploitable}
        \subidx{exploitable}{strategy}
        \subidx{programs}{tsshannonmax}
        \subidx{tsshannonmax}{program}
        \subidx{programs}{tsunfairbrownian}
        \subidx{tsunfairbrownian}{program}
        \subidx{strategy}{fiscal}
        \subidx{fiscal}{strategy}
        The data in this section is presented in tabular form in
        Section~\ref{\SETLABELREF:MAXSHANNON}. One of the issues of
        analysis, as mentioned in Section~\ref{\SETLABEL:OPS}, is to
        determine the maximum Shannon probability for the time series
        presented in Figure~\ref{\SETLABEL:TS}. Potentially, this
        could be exploited with an aggressive fiscal
        strategy. Figure~\ref{\SETLABEL:SHANNONMAX0} is a graph of the
        output of the {\it tsshannonmax}\/ program, which is described
        briefly in appendix~\ref{programs}. The maximum of this
        function is the maximum Shannon probability for the time
        series data presented in Figure~\ref{\SETLABEL:TS}.
        Figure~\ref{\SETLABEL:SHANNONMAX1} was constructed using {\it
        tsunfairbrownian}\/ program, which is also described in
        appendix~\ref{programs}, with the maximum Shannon probability,
        and the time series data presented in
        Figure~\ref{\SETLABEL:TS}. This represents a ``what if'' the
        investment strategy was changed from a Shannon probability of
        {\shannonlogreturns}, as derived in Section~\ref{\SETLABEL:SP}
        to {\shannonmax}. This process, essentially, extracts the
        random statistical data from the time series presented in
        Figure~\ref{\SETLABEL:TS}, and constructs a new time series,
        using the random statistical data, with a different investment
        strategy.  The program, {\it tsunfairbrownian}\/, essentially,
        constructs the new time series as a Brownian fractal with
        fixed increments.  The ``quality'' of such a reconstruction
        should be subject to adequate scepticism and scrutiny since,
        in all probability, the increments in the original data
        represent a relatively complex process, that may not be
        ``modeled'' with such a simple methodology.

        \begin{figure}[ht]
            \begin{center}
                \begin{minipage}[t]{0.45\textwidth}
                    \epsfxsize=1.0\linewidth
                    \epsffile{\directory/data.tsshannonmax.eps}
                    \caption[{\market}, maximum rate of revenue
                        returns] {{\market}, maximum rate of revenue
                        returns, per {\timescale}, vs. Shannon
                        probability. The maximum rate of revenue
                        returns, per {\timescale}, occurs at a Shannon
                        probability of {\shannonmax}.}
                    \label{\SETLABEL:SHANNONMAX0}
                \end{minipage}
                \hfill
                \begin{minipage}[t]{0.45\textwidth}
                    \epsfxsize=1.0\linewidth
                    \epsffile{\directory/data.tsshannonmax-p.tsunfairbrownian-p.eps}
                    \caption[{\market}, maximum rate of revenue
                        returns] {{\market}, maximum rate of revenue
                        returns, per {\timescale}, at a Shannon
                        probability, of {\shannonmax}, corresponding
                        to a ``wager'' fraction of {\twoponemax}.}
                    \label{\SETLABEL:SHANNONMAX1}
                \end{minipage}
            \end{center}
        \end{figure}

        \subidx{fractional}{Brownian motion}
        \subidx{Brownian motion}{fractional}
        \subidx{Shannon}{probability}
        \subidx{probability}{Shannon}
        \subidx{programs}{tsshannonmax}
        \subidx{tsshannonmax}{program}
        If it is assumed that the time series data set, presented in
        Figure~\ref{\SETLABEL:TS}, constitutes classical Brownian
        motion, then the Shannon probability can be calculated by
        counting the total number of {\timescale}s that the {\market}
        movement was positive, and dividing by the total number of
        {timescale}s represented in the time series. This quotient is
        {\pmax}, as compared with the predicted value from the program
        {\it tsshannonmax}\/ of {\shannonmax}.

% Local Variables:
% TeX-parse-self: t
% TeX-auto-save: t
% TeX-master: "fractal.tex"
% End:


        \subsubsection{Observations on the Simulation of Fixed Increment Approximation for Optimally Maximal Fiscal Strategy}

            Note that these simulations are base on a very, perhaps
            overly, simplified model. For example, from
            Section~\ref{\SETLABEL:TSA}, Figure~\ref{\SETLABEL:NH}, it
            would appear that the {\market}'s normalized increments
            are characterized by fractional Brownian motion---but the
            simulations used classical Brownian motion as the
            model. One consequence of this is that a re-investment
            strategy that is to ``wager'' a fraction of {\twoponemax}
            of the rate of returns every {\timescale} is overly
            aggressive, since in the classical Brownian scenario, the
            maximum loss, in any {\timescale}, was no more that what
            was ``wagered.'' However, in the fractional Brownian
            scenario, much more can be lost. From
            Equation~\ref{fopt2},

            \begin{equation}
                \frac{avg}{rms^2} = \frac{f_{opt}}{rms} = K
            \end{equation}

            \noindent where, under the optimum classical Brownian
            scenario, $K$ is unity, or $avg = rms^2$. Notice that,
            since $f = rms$, whether the scenario is optimal or not,
            that the operational ``wager'' fraction, from
            Figure~\ref{\SETLABEL:TF} of {\datafractionrms}, vs.\ an
            ``theoretical optimal'' value of {\twoponemax} seems
            overly conservative. Additionally, notice that, at least
            in principle, the chance of failure in the fractional
            Brownian scenario, which is more accurate, would
            correspond to 1 standard deviation, or about 15.865\% per
            {\timescale}, which is unacceptably high. However, it is
            not clear why the {\market} is running at a value of
            {\datafractionrms}, which seems very
            conservative. However, a re-investment strategy of
            {\datafractionrms} per {\timescale} does not seem
            inconsistent with a failure rate, on the Fortune 500 list,
            which it is inferred that the {\market} is similar to, of
            about 50\% in ten years, which corresponds to $(1 -
            p_f)^{120} \approx 0.5$, or $p_f$, the probability of
            failure, is $0.005759576$, which is, approximately, 2.5
            standard deviations, meaning that to be consistent with
            the large companies in the Fortune 500, the re-investment
            rate should be, approximately, $\frac{\twoponemax}{2.5}$,
            compared with an operational value, from
            Figure~\ref{\SETLABEL:NH} of {\datafractionrms}.

            An interesting, and intriguing, interpretation and
            discussion of the maximum Shannon probability, is an
            explanation as to why the companies in the {\market} are
            not running an optimal re-investment strategy. This seems
            enigmatic, since those companies that run, on a long term
            average, below the optimally maximal value would seem to
            be eclipsed by those that didn't. And those that run above
            the optimally maximal value would be over extended, and
            become financially destitute during market down turns,
            which is inevitable in a fractal time series as presented
            in Figure~\ref{\SETLABEL:TS}.  It would seem that the
            natural selection process of the competitive environment
            would allow only those companies that run near the
            optimally maximal value to survive, in the long run. One
            possible explanation, foremost, is that the analytical
            methodology presented herein is inappropriate.  Another
            explanation is that the gross margins are less than the
            fraction {\shannonmax} of the rate of revenue returns, and
            thus could not accommodate such an aggressive
            re-investment strategy. If this is the case, then it
            presents an intriguing issue. If, in a capitalistic
            market, the natural outcome of the competitive situation,
            according to game-theoretic analysis, is that there will
            be many competitors, each making minimal gross margins,
            then how do the companies grow their markets?  Naturally,
            those that run the most efficient will have lower costs,
            making larger percentage of rate of revenue returns
            re-investment possible. Yet another interpretation is that
            the number of competitors would grow at an exponential
            rate, but all of them would make minimal returns. However,
            an operational Shannon probability of {\shannonlogreturns}
            is not just marginally lower than the maximum Shannon
            probability of {\shannonmax}. There is a significant
            disparity which is difficult to explain. It would seem
            that the game-theoretic eventual outcome of a competitive
            market place would be a solution that hinders growth,
            wealth and jobs creation, etc., which does not seem
            consistent with capitalistic theory. On the other hand, is
            there an optimum number of competitors in a market place,
            where the gross margins can be higher, permitting wealth
            and job creation, and also a competitive situation? If
            this analysis is correct, and that should be subject to
            scrutiny, then it would appear that this is the case. But
            this brings up another issue---that of taxation, and other
            contributions to the social welfare function. If there is
            an optimum number of competitors in the market place, that
            maximizes wealth and job creation, then, perhaps by lemma,
            there is also an optimal value of taxation rate, and other
            contributions to the social welfare function, that will
            permit maximal industrial growth, and thus maximal growth
            in the tax base. But this would seem to be inconsistent
            with the work of Kenneth Arrow and the so called
            Impossibility Theorem, which states that such
            optimizations can not be determined because the ordering
            of priorities are intransitive.  All very perplexing,
            since the simulation of the maximum Shannon probability in
            the next section seems to indicate that such an aggressive
            re-investment strategy is, indeed, feasible.

            Yet another possibility for the industry not running at
            maximum Shannon probability is the high cost of expansion
            of operations. Some of these industries require very
            sophisticated manufacturing processes, which have high
            barrier costs.

            Additionally, as mentioned in both~\cite[pp. 29]{Brock},
            and~\cite[pp. 8]{Arthur:CTIRALIBHE}, optimal efficiency
            may not be attainable in increasing-return economic
            scenarios.

        %
% -----------------------------------------------------------------------------
%
% A license is hereby granted to reproduce this software source code and
% to create executable versions from this source code for personal,
% non-commercial use.  The copyright notice included with the software
% must be maintained in all copies produced.
%
% THIS PROGRAM IS PROVIDED "AS IS". THE AUTHOR PROVIDES NO WARRANTIES
% WHATSOEVER, EXPRESSED OR IMPLIED, INCLUDING WARRANTIES OF
% MERCHANTABILITY, TITLE, OR FITNESS FOR ANY PARTICULAR PURPOSE.  THE
% AUTHOR DOES NOT WARRANT THAT USE OF THIS PROGRAM DOES NOT INFRINGE THE
% INTELLECTUAL PROPERTY RIGHTS OF ANY THIRD PARTY IN ANY COUNTRY.
%
% Copyright (c) 1994-2006, John Conover, All Rights Reserved.
%
% Comments and/or bug reports should be addressed to:
%
%     john@email.johncon.com (John Conover)
%
% -----------------------------------------------------------------------------
%
% Revision: \RCSRevision \\
% Revision Time: \RCSTime UMT \\
% Revision Date: \RCSDate \\
% Revision Id: \RCSId \\
% Revision File: \RCSLog \\
\RCS $Revision: 0.0 $
\RCS $Date: 2006/01/20 04:38:13 $
\RCS $Id: verification.tex,v 0.0 2006/01/20 04:38:13 john Exp $
% $Log: verification.tex,v $
% Revision 0.0  2006/01/20 04:38:13  john
% Initial version
%
%
    \subsection{Qualitative Verification of Fixed Increment Approximation Analysis}
        \label{\SETLABEL:QVA}

        \subidx{\market}{verification of analysis}
        \subidx{verification}{analysis}
        \subidx{analysis}{verification}
        \subidx{quality}{of analysis}
        \subidx{verification}{of methodology}
        \subidx{methodology}{verification of}
        \subidx{Shannon}{probability}
        \subidx{probability}{Shannon}

        This section evaluates various values based on the ``average''
        of the normalized increments presented in
        Figure~\ref{\SETLABEL:TFA}. These values are an approximation
        to a, probably, complex process with a distribution shown in
        Figure~\ref{\SETLABEL:TF}. These values will be used in a
        fixed increment Brownian fractal analysis of the {\market},
        and may, or may not, provide adequate accuracy for
        projections.

        The data in this section is presented in tabular form in
        sections~\ref{\SETLABELREF:VI1} and~\ref{\SETLABELREF:VI2}.
        As a subjective evaluation of the ``quality'' of the analysis
        of the {\market}, from Chapter~\ref{methodology},
        Equation~\ref{metricvalues1}, and using the mean and root mean
        square values of the normalized increments of the time series
        data presented in Figure~\ref{\SETLABEL:TS} from
        Figure~\ref{\SETLABEL:TF}, and the Shannon probability as
        calculated by counting the total number of {\timescale}s that
        the {\market} movement was positive, as presented in
        Section~\ref{\SETLABEL:MAXSHANNON}:

        \begin{eqnarray}
                  P & \approx & \frac{\frac{avg}{rms} + 1}{2}\\
            {\pmax} & \approx & \frac{\frac{\datafractionmean}{\datafractionrms} + 1}{2}\\
            {\pmax} & \approx & {\avgrms}
            \label{\SETLABEL:AVGS}
        \end{eqnarray}

        \subidx{Shannon}{probability}
        \subidx{probability}{Shannon}
        \noindent and comparing these values to the Shannon
        probability, as found by the {\it tsshannonmax}\/ program, which
        iterates for a maximum:

        \begin{eqnarray}
            {\pmax} \approx {\avgrms} \approx {\shannonmax}
        \end{eqnarray}

        \subidx{logarithmic}{returns}
        \subidx{returns}{logarithmic}
        In addition, the different methods of calculating the
        logarithmic returns, presented in Section~\ref{\SETLABEL:FS},
        should be compared. The four methods used were the mean of
        Figure~\ref{\SETLABEL:TF}, the constant in the least squares
        approximation to Figure~\ref{\SETLABEL:TF}, the least squares
        exponential approximation to Figure~\ref{\SETLABEL:TS}, and
        the logarithmic returns of Figure~\ref{\SETLABEL:TS}, derived
        as the mean of the logarithms of the quotients of the
        increments. The values for each of the methods are,
        respectively:

        \begin{equation}
            \datafractionmeanbits \approx \datafractionconstantbits \approx \datatslsqepbits \approx \logreturns
        \end{equation}

        It is implied in Section~\ref{\SETLABEL:FS},
        Subsection~\ref{\SETLABEL:SP} and in
        Section~\ref{\SETLABEL:TSUNFAIRBROWNIAN} that, a Brownian
        motion with fixed increments fractal may ``model'' the
        {\market}. Using Equation~\ref{stddev9} from
        Chapter~\ref{general}, Section~\ref{abmfi}:

        \begin{eqnarray}
                                    rms \left(2P - 1\right) & \approx & \frac{\sigma \left(2P - 1\right)}{2 \sqrt{P\left(1 - P\right)}}\\
            \datafractionrms \left(2 \cdot \pmax - 1\right) & \approx & \frac{\datafractionstddev \left(2 \cdot \pmax - 1\right)}{2\sqrt{\pmax \left(1 - \pmax\right)}}\\
                       \datafractionrms \cdot \twopminusone & \approx & \datafractionstddev \cdot \twopx\\
                                                      \rmsp & \approx & \sigmap
        \end{eqnarray}

        \noindent and, equating to the mean:

        \begin{equation}
            \datafractionmean \approx \rmsp \approx \sigmap
        \end{equation}

        \subidx{Shannon}{probability}
        \subidx{probability}{Shannon}
        \noindent where, as in Equation~\ref{\SETLABEL:AVGS} using the
        mean, root mean square, and standard deviation values of the
        normalized increments of the time series data presented in
        Figure~\ref{\SETLABEL:TS} from Figure~\ref{\SETLABEL:TF}, and
        the Shannon probability as calculated by counting the total
        number of {\timescale}s that the {\market} movement was
        positive, as presented in Section~\ref{\SETLABEL:MAXSHANNON}.

        As a final qualitative comparison, the absolute value of the
        normalized increments should be the same as the root mean
        square value\footnote{The absolute value of the normalized
        increments, when averaged, is related to the root mean square
        of the increments by a constant. If the normalized increments
        are a fixed increment, the constant is unity. If the
        normalized increments have a Gaussian distribution, the
        constant is $\approx 0.8$ depending on the accuracy of of
        ``fit'' to a Gaussian distribution.}, where the absolute value
        is presented in Figure~\ref{\SETLABEL:TFA}, and the root mean
        square value is presented in Figure~\ref{\SETLABEL:TF}:

        \begin{equation}
            \datafractionabsmean \approx \datafractionrms
        \end{equation}

        Note, that if the {\market} could be ``modeled'' as a Brownian
        motion with fixed increments fractal, then the standard
        deviation of the absolute value of the normalized increments
        of the time series data presented in Figure~\ref{\SETLABEL:TS}
        from Figure~\ref{\SETLABEL:TF} should be zero. It is
        $\datafractionabsstddev$.

% Local Variables:
% TeX-parse-self: t
% TeX-auto-save: t
% TeX-master: "fractal.tex"
% End:


    \renewcommand{\market}{United States Information Systems Market}
    \renewcommand{\directory}{../markets/information.systems}
    \renewcommand{\datafractionmean}{0.008052}
\renewcommand{\datafractionmeanbits}{0.011570}
\renewcommand{\datafractionmeanq}{0.002684}
\renewcommand{\datafractionmeanbitsq}{0.003867}
\renewcommand{\datafractionstddev}{0.038579}
\renewcommand{\datafractionrms}{0.039311}
\renewcommand{\avgrms}{0.602414}
\renewcommand{\ncompanies}{5.210454}
\renewcommand{\pncompanies}{0.544866}
\renewcommand{\datafractionabsmean}{0.029745}
\renewcommand{\datafractionabsstddev}{0.025769}
\renewcommand{\datafractionconstant}{0.010041}
\renewcommand{\datafractionconstantbits}{0.014414}
\renewcommand{\datafractionconstantq}{0.003347}
\renewcommand{\datafractionconstantbitsq}{0.004821}
\renewcommand{\datafractionslope}{-0.000021}
\renewcommand{\datafractionabsconstant}{0.035145}
\renewcommand{\datafractionabsslope}{-0.000057}
\renewcommand{\hurstall}{0.659558}
\renewcommand{\hurstlow}{0.707509}
\renewcommand{\hurstlowtwo}{1.415018}
\renewcommand{\hurstlowhundred}{70.750900}
\renewcommand{\hcalcall}{0.184942}
\renewcommand{\hcalclow}{0.102042}
\renewcommand{\shannonmax}{0.604167}
\renewcommand{\twoponemax}{0.208334}
\renewcommand{\logreturns}{0.010456}
\renewcommand{\twologreturns}{1.007274}
\renewcommand{\twologreturnshundred}{0.727387}
\renewcommand{\oneoverlogreturns}{95.638868}
\renewcommand{\pmax}{0.602094}
\renewcommand{\twopminusone}{0.204188}
\renewcommand{\rmsp}{0.008027}
\renewcommand{\twopx}{0.208583}
\renewcommand{\sigmap}{0.008047}
\renewcommand{\tsunfairbrownianfractionmean}{0.007862}
\renewcommand{\tsunfairbrownianfractionstddev}{0.038619}
\renewcommand{\shannonlogreturns}{0.560125}
\renewcommand{\shannonlogreturnshundred}{56.012500}
\renewcommand{\twopone}{0.120250}
\renewcommand{\twoponehundred}{12.025000}
\renewcommand{\hundredtwoponehundred}{87.975000}
\renewcommand{\hundredshannonlogreturnshundred}{43.987500}
\renewcommand{\datatslsqepbits}{0.007623}
\renewcommand{\thurstall}{0.633980}
\renewcommand{\thurstlow}{0.710108}
\renewcommand{\thurstlowtwo}{1.420216}
\renewcommand{\thurstlowhundred}{71.010800}
\renewcommand{\thcalcall}{0.247886}
\renewcommand{\thcalclow}{0.171737}
\renewcommand{\chisquared}{2.862000}
\renewcommand{\critical}{42.557000}

    \renewcommand{\timescale}{month}
    \subidx{market}{\market}
    \idx{\market}

    \section{\market}

        \renewcommand{\SETLABEL}{\LABPRE:NAISM}
        \renewcommand{\SETLABELQ}{\LABPRE:NAISMQ}
        \label{\SETLABEL}
        \renewcommand{\SETLABELREF}{\LABPREREF:NAISM}

        \idx{United States Department of Commerce}
        For the analysis, the data was in the directory
        {\directory}\footnote{Data from the United States Department
        of Commerce, 1979---1994, by {\timescale}s, in millions of
        dollars, US.}.

        The data in this section is presented in tabular form in
        Section~\ref{\SETLABELREF}.

        %
% -----------------------------------------------------------------------------
%
% A license is hereby granted to reproduce this software source code and
% to create executable versions from this source code for personal,
% non-commercial use.  The copyright notice included with the software
% must be maintained in all copies produced.
%
% THIS PROGRAM IS PROVIDED "AS IS". THE AUTHOR PROVIDES NO WARRANTIES
% WHATSOEVER, EXPRESSED OR IMPLIED, INCLUDING WARRANTIES OF
% MERCHANTABILITY, TITLE, OR FITNESS FOR ANY PARTICULAR PURPOSE.  THE
% AUTHOR DOES NOT WARRANT THAT USE OF THIS PROGRAM DOES NOT INFRINGE THE
% INTELLECTUAL PROPERTY RIGHTS OF ANY THIRD PARTY IN ANY COUNTRY.
%
% Copyright (c) 1994-2006, John Conover, All Rights Reserved.
%
% Comments and/or bug reports should be addressed to:
%
%     john@email.johncon.com (John Conover)
%
% -----------------------------------------------------------------------------
%
% Revision: \RCSRevision \\
% Revision Time: \RCSTime UMT \\
% Revision Date: \RCSDate \\
% Revision Id: \RCSId \\
% Revision File: \RCSLog \\
\RCS $Revision: 0.0 $
\RCS $Date: 2006/01/20 04:38:13 $
\RCS $Id: fraction.tex,v 0.0 2006/01/20 04:38:13 john Exp $
% $Log: fraction.tex,v $
% Revision 0.0  2006/01/20 04:38:13  john
% Initial version
%
%
    \subsection{Time Series Increments Analysis}
        \label{\SETLABEL:TSA}

        \subidx{\market}{Time series analysis}
        \subidx{time series}{increments}
        \subidx{time series}{analysis}
        \subidx{cumulative sum}{analysis}
        \subidx{analysis}{cumulative sum}
        \subidx{analysis}{random process}
        \subidx{random process}{analysis}
        \subidx{Gaussian}{increments}
        \subidx{increments}{Gaussian}
        \subidx{Brownian}{motion, fractional}
        \subidx{fractional}{Brownian motion}
        \subidx{fractal}{Brownian motion}
        The data in this section is presented in tabular form in
        Section~\ref{\SETLABELREF:TSA}.  Figure~\ref{\SETLABEL:TS} is
        a graph of the time series data for the {\market}.

        \subidx{increments}{normalized}
        \subidx{normalized}{increments}
        \subidx{programs}{tsfraction}
        \subidx{tsfraction}{program}
        Figure~\ref{\SETLABEL:TF} is a graph of the normalized
        increments of the time series data presented in
        Figure~\ref{\SETLABEL:TS}. The data presented was made by
        running the program {\it tsfraction}\/ on the time series
        data. The program {\it tsfraction}\/ is described briefly in
        Appendix~\ref{programs}, and subtracts the previous value from
        the next value, dividing this difference by the previous
        value, for each element in the time series data. The new time
        series contains the instantaneous change in the rate of
        revenue returns, divided by the magnitude of the instantaneous
        rate of revenue returns.

        \subidx{mean}{standard deviation}
        \subidx{standard deviation}{mean}
        \idx{root mean square}
        \idx{least squares approximation}
        \begin{figure}[ht]
            \begin{center}
                \begin{minipage}[t]{0.45\textwidth}
                    \epsfxsize=1.0\linewidth
                    \epsffile{\directory/data.eps}
                    \caption{{\market}, time series data.}
                    \label{\SETLABEL:TS}
                    \label{\SETLABELQ:TS}
                \end{minipage}
                \hfill
                \begin{minipage}[t]{0.45\textwidth}
                    \epsfxsize=1.0\linewidth
                    \epsffile{\directory/data.tsfraction.eps}
                    \caption[{\market}, normalized
                        increments]{{\market}, normalized increments
                        of the time series data presented in
                        Figure~\ref{\SETLABEL:TS}. The mean is
                        {\datafractionmean} with a standard deviation
                        of {\datafractionstddev}. The formula for the
                        least squares approximation is
                        ${\datafractionconstant} +
                        {\datafractionslope}t$, and the root mean
                        squared value is {\datafractionrms}. The
                        graph, labeled ``data\-.tsfraction\-.tsrms,''
                        is the running root mean square, and
                        ``data\-.tsfraction\-.tsavg'' is the running
                        average of the normalized increments.  This
                        graph is the fraction of change in the time
                        series, as a function of time. Note that the
                        slope of the mean, {\datafractionslope}, is
                        the coefficient of the nonlinearity term in
                        the normalized increments. See
                        Chapter~\ref{general}, Section~\ref{nlextend}
                        for a possible application of the logistic
                        function to this data set.}
                    \label{\SETLABEL:TF}
                    \label{\SETLABELQ:TF}
                \end{minipage}
            \end{center}
        \end{figure}

        \subidx{absolute value}{increments}
        \subidx{increments}{absolute value}

        Figure~\ref{\SETLABEL:TFA} is a graph of the absolute value of
        the normalized increments of the time series data presented in
        Figure~\ref{\SETLABEL:TF}. The data presented was made by
        running the Unix utility sed(1) on the normalized increments
        time series data to remove the negative signs. This is an
        absolute value procedure.  The resulting time series contains
        the absolute value of the instantaneous change in the rate of
        revenue returns, divided by the magnitude of the instantaneous
        rate of revenue returns\footnote{The absolute value of the
        normalized increments, when averaged, is related to the root
        mean square of the increments by a constant. If the normalized
        increments are a fixed increment, the constant is unity. If
        the normalized increments have a Gaussian distribution, the
        constant is $\approx 0.8$ depending on the accuracy of of
        ``fit'' to a Gaussian distribution.}.

        \subidx{histogram}{normalized}
        \subidx{normalized}{histogram}
        \subidx{programs}{tsnormal}
        \subidx{tsnormal}{program}
        \subidx{mean}{standard deviation}
        \subidx{standard deviation}{mean}
        \idx{root mean square}
        \idx{least squares approximation}
        \subidx{\market}{analysis of increments}
        Figure~\ref{\SETLABEL:NH} is the normalized histogram of the
        normalized increments of the time series data shown in
        Figure~\ref{\SETLABEL:TF}. The abscissa is 3 $\sigma$ limits,
        and the area under the two curves is identical. The data for
        this figure was produced by the program {\it tsnormal}\/,
        which is described briefly in Appendix~\ref{programs}.

        \begin{figure}[ht]
            \begin{center}
                \begin{minipage}[t]{0.45\textwidth}
                    \epsfxsize=1.0\linewidth
                    \epsffile{\directory/data.tsfraction.abs.eps}
                    \caption[{\market}, absolute value of the
                        normalized increments]{{\market}, absolute
                        value of the normalized increments of the time
                        series data presented in
                        Figure~\ref{\SETLABEL:TF}.  The mean is
                        {\datafractionabsmean} with a standard
                        deviation of {\datafractionabsstddev}. The
                        formula for the least squares approximation is
                        ${\datafractionabsconstant} +
                        {\datafractionabsslope}t$, and the root mean
                        square value, from Figure~\ref{\SETLABEL:TF},
                        is {\datafractionrms}.  The graph, labeled
                        ``data\-.tsfraction\-.tsrms,'' is the running
                        root mean square, and
                        ``data\-.tsfraction\-.tsavg'' is the running
                        average of the normalized increments presented
                        in Figure~\ref{\SETLABEL:TF}, superimposed
                        here for convenience. This graph is the
                        absolute value of the fraction of change in
                        the time series, as a function of time.}
                    \label{\SETLABEL:TFA}
                    \label{\SETLABELQ:TFA}
                \end{minipage}
                \hfill
                \begin{minipage}[t]{0.45\textwidth}
                    \epsfxsize=1.0\linewidth
                    \epsffile{\directory/data.tsfraction.tsnormal-s30.eps}
                    \caption[{\market}, normalized histogram of the
                        normalized increments]{{\market}, normalized
                        histogram of the normalized increments of the
                        time series data shown in
                        Figure~\ref{\SETLABEL:TF}.  The data has a
                        mean of {\datafractionmean}, with a standard
                        deviation of {\datafractionstddev}.  The area
                        under the two curves is identical. The
                        $\chi^2$ value of the observed and expected
                        values of the two curves is {\chisquared},
                        with a critical value of {\critical}.}
                    \label{\SETLABEL:NH}
                \end{minipage}
            \end{center}
        \end{figure}

        \subidx{programs}{tsXsquared}
        \subidx{tsXsquared}{program}
        \subidx{\market}{chi-squared values of increments}
        The program {\it tsXsquared}\/, which is briefly described in
        appendix~\ref{programs}, was used to derive the $\chi^2$
        statistics for the data presented in
        Figure~\ref{\SETLABEL:NH}.

        \subidx{programs}{tsstatest}
        \subidx{tsstatest}{program}
        \subidx{\market}{statistical estimates}

        Figure~\ref{\SETLABEL:SE} is the statistical estimate for the
        data presented in Figure~\ref{\SETLABEL:TF}, as derived by the
        program {\it tsstatest}\/, which is briefly described in
        appendix~\ref{programs}.

        \begin{figure}[ht]
            \begin{center}
                \begin{minipage}[t]{\textwidth}
                    \center{\fbox{\parbox{0.9\textwidth}{\XXX{\directory/data.tsstatest-f0.1-c0.9-i.tex}}}}
                    \caption[{\market}, statistical estimates of the
                        normalized increments]{{\market}, statistical
                        estimates of the normalized increments of the
                        time series shown in Figure~\ref{\SETLABEL:TF}.
                        The table was produced with the {\it
                        tsstatest}\/ program, and illustrates the
                        size of the data set required for a confidence
                        level of 90\%, with an error estimate of $\pm$
                        10\%, or alternately, the error estimate on
                        the time series shown in Figure~\ref{\SETLABEL:TF}.}
                    \label{\SETLABEL:SE}
                \end{minipage}
            \end{center}
        \end{figure}

        Note that the data set size estimations, as produced by the
        {\it tsstatest}\/ program, are probably very conservative,
        depending on the magnitude of the Shannon probability, $P =
        \shannonlogreturns$, as derived in
        Section~\ref{\SETLABEL:SP}. See Chapter~\ref{general},
        Section~\ref{serdss} for possible alternative methodologies
        for addressing the analysis of fractal time series with
        limited data set sizes. Depending on the magnitude of the
        Shannon probability, $P$, these estimates can be several
        orders of magnitude too high.

        \subidx{derivative of increments}{normalized}
        \subidx{normalized}{derivative of increments}
        \subidx{programs}{tsderivative}
        \subidx{tsderivative}{program}
        Figure~\ref{\SETLABEL:TF1} is the normalized histogram of the
        first derivative of the normalized increments of the time
        series data shown in Figure~\ref{\SETLABEL:TF}. In principle,
        if the distribution of the normalized increments presented in
        Figure~\ref{\SETLABEL:NH} is Gaussian in nature, this
        distribution would be similar to ``white noise,'' as presented
        in appendix~\ref{programs}, Figure~\ref{whiteexample}. The
        data was generated by the {\it tsderivative}\/ program, which
        is briefly described in
        appendix~\ref{programs}. Figure~\ref{\SETLABEL:TF2} is the
        normalized histogram of the second derivative of the
        normalized increments of the time series data shown in
        Figure~\ref{\SETLABEL:TF}. In principle, if the distribution
        of the normalized increments presented in
        Figure~\ref{\SETLABEL:NH} is an integrated Gaussian
        distribution in nature, this distribution would be similar to
        ``white noise,'' as presented in appendix~\ref{programs},
        Figure~\ref{whiteexample}.

        \begin{figure}[ht]
            \begin{center}
                \begin{minipage}[t]{0.45\textwidth}
                    \epsfxsize=1.0\linewidth
                    \epsffile{\directory/data.tsfraction.tsderivative.tsnormal-s30.eps}
                    \caption[{\market}, histogram of the first
                        derivative of the increments]{{\market},
                        normalized histogram of the first derivative
                        of the normalized increments of the time
                        series data shown in
                        Figure~\ref{\SETLABEL:TF}.}
                    \label{\SETLABEL:TF1}
                \end{minipage}
                \hfill
                \begin{minipage}[t]{0.45\textwidth}
                    \epsfxsize=1.0\linewidth
                    \epsffile{\directory/data.tsfraction.2tsderivative.tsnormal-s30.eps}
                    \caption[{\market}, histogram of the second
                        derivative of the increments]{{\market},
                        normalized histogram of second derivative of
                        the the normalized increments of the time
                        series data shown in
                        Figure~\ref{\SETLABEL:TF}.}
                    \label{\SETLABEL:TF2}
                \end{minipage}
            \end{center}
        \end{figure}

        \subidx{fractal}{range}
        \subidx{fractal}{R/S analysis}
        \subidx{\market}{rate of revenue returns, range}
        \subidx{\market}{deterministic mechanism}
        \subidx{deterministic}{mechanism}
        \subidx{mechanism}{deterministic}
        Figure~\ref{\SETLABEL:TR} is the range of values of the time
        series shown in Figure~\ref{\SETLABEL:TS}. The horizontal axis
        is time into the future. In principle, if the time series was
        characterized as fractional Brownian motion the graph in
        Figure~\ref{\SETLABEL:TR} would be a square root
        function\footnote{Note that the ``roughness,'' or ``sawtooth''
        characteristics of the graph in Figure~\ref{\SETLABEL:TR} are
        a computational artifact---caused by not using the -m option
        to the program {\it tshurst}\/, which is computationally
        inefficient.}. Figure~\ref{\SETLABEL:TD} is the deterministic
        map of the normalized increments of the time series data shown
        in Figure~\ref{\SETLABEL:TF}. The deterministic map is useful
        for determining if a time series was created by a
        deterministic mechanism. This, essentially, maps each element
        in the time series with the previous element in the time
        series.  See,~\cite[pp. 745]{Peitgen}.

        \begin{figure}[ht]
            \begin{center}
                \begin{minipage}[t]{0.45\textwidth}
                    \epsfxsize=1.0\linewidth
                    \epsffile{\directory/data.tshurst-f.eps}
                    \caption[{\market}, range]{{\market}, range of the
                        time series data shown in
                        Figure~\ref{\SETLABEL:TS}.}
                    \label{\SETLABEL:TR}
                \end{minipage}
                \hfill
                \begin{minipage}[t]{0.45\textwidth}
                    \epsfxsize=1.0\linewidth
                    \epsffile{\directory/data.tsfraction.tsdeterministic.eps}
                    \caption[{\market}, deterministic map]{{\market},
                        deterministic map of the normalized increments
                        of the time series data shown in
                        Figure~\ref{\SETLABEL:TF}.}
                    \label{\SETLABEL:TD}
                \end{minipage}
            \end{center}
        \end{figure}

% Local Variables:
% TeX-parse-self: t
% TeX-auto-save: t
% TeX-master: "fractal.tex"
% End:


        \subsubsection{Observations on the Time Series Increments Analysis}

            Figure~\ref{\SETLABEL:NH} would seem to indicate that the
            time series data for the {\market} represents a cumulative
            sum/integration of a random process that has a Gaussian
            distribution, (ie., satisfies the Gaussian increments
            property of fractional Brownian
            motion~\cite[pp. 250]{Crownover},) tending to justify the
            assumption that the time series data represents fractional
            Brownian motion.

        %
% -----------------------------------------------------------------------------
%
% A license is hereby granted to reproduce this software source code and
% to create executable versions from this source code for personal,
% non-commercial use.  The copyright notice included with the software
% must be maintained in all copies produced.
%
% THIS PROGRAM IS PROVIDED "AS IS". THE AUTHOR PROVIDES NO WARRANTIES
% WHATSOEVER, EXPRESSED OR IMPLIED, INCLUDING WARRANTIES OF
% MERCHANTABILITY, TITLE, OR FITNESS FOR ANY PARTICULAR PURPOSE.  THE
% AUTHOR DOES NOT WARRANT THAT USE OF THIS PROGRAM DOES NOT INFRINGE THE
% INTELLECTUAL PROPERTY RIGHTS OF ANY THIRD PARTY IN ANY COUNTRY.
%
% Copyright (c) 1994-2006, John Conover, All Rights Reserved.
%
% Comments and/or bug reports should be addressed to:
%
%     john@email.johncon.com (John Conover)
%
% -----------------------------------------------------------------------------
%
% Revision: \RCSRevision \\
% Revision Time: \RCSTime UMT \\
% Revision Date: \RCSDate \\
% Revision Id: \RCSId \\
% Revision File: \RCSLog \\
\RCS $Revision: 0.0 $
\RCS $Date: 2006/01/20 04:38:13 $
\RCS $Id: instant.tex,v 0.0 2006/01/20 04:38:13 john Exp $
% $Log: instant.tex,v $
% Revision 0.0  2006/01/20 04:38:13  john
% Initial version
%
%
    \subsection{Instantaneous Analysis of Normalized Increments}
        \label{\SETLABEL:IA}

        \subidx{\market}{instantaneous analysis of normalized increments}
        \idx{average of normalized increments}
        \idx{root mean square of normalized increments}
        \subidx{Shannon probability}{instantaneous computation of}
        \subidx{average of normalized increments}{instantaneous computation of}
        \subidx{root mean square of normalized increments}{instantaneous computation of}
        \subidx{instantaneous computation}{Shannon probability}
        \subidx{instantaneous computation}{average of normalized increments}
        \subidx{instantaneous computation}{root mean square of normalized increments}
        \idx{time series}
        \subidx{time series}{instantaneous analysis}
        \subidx{instantaneous analysis}{time series}
        \subidx{time series}{increments}
        \subidx{time series}{analysis}
        \subidx{Shannon}{probability}
        \subidx{probability}{Shannon}
        \subidx{normalized}{increments}
        \subidx{increments}{normalized}

        The program {\it tsinstant}\/, which is briefly described in
        Appendix~\ref{programs}, is for finding the instantaneous
        fraction of change in a time series. The value of a sample in
        the time series is subtracted from the previous sample in the
        time series, and divided by the value of the previous sample.
        As explained in Chapter~\ref{general},
        Sections~\ref{derivation},~\ref{GA},~\ref{abmfi},~\ref{aftsma}
        and,~\ref{ompl} for Brownian motion, random walk fractals, the
        absolute value of the instantaneous fraction of change is also
        the root mean square of the instantaneous fraction of
        change\footnote{The absolute value of the normalized
        increments, when averaged, is related to the root mean square
        of the increments by a constant. If the normalized increments
        are a fixed increment, the constant is unity. If the
        normalized increments have a Gaussian distribution, the
        constant is $\approx 0.8$ depending on the accuracy of of
        ``fit'' to a Gaussian distribution.}. Squaring this value is
        the average of the instantaneous fraction of change, and
        adding unity to the absolute value of the instantaneous
        fraction of change, and dividing by two, is the Shannon
        probability of the instantaneous fraction of change.

        Figure~\ref{\SETLABEL:IA1} is the instantaneous value of the
        root mean square of the normalized increments for the
        {\market}, and Figure~\ref{\SETLABEL:IA2} is the instantaneous
        Shannon probability for the normalized increments.

        \begin{figure}[ht]
            \begin{center}
                \begin{minipage}[t]{0.45\textwidth}
                    \epsfxsize=1.0\linewidth
                    \epsffile{\directory/data.tsinstant-r.eps}
                    \caption[{\market}, instantaneous value of
                        rms.]{{\market}, instantaneous value of the
                        root mean square of the normalized increments,
                        provided by running the program {\it
                        tsinstant}\/ with the -r option on the data
                        presented in Figure~\ref{\SETLABEL:TS}.}
                    \label{\SETLABEL:IA1}
                    \label{\SETLABELQ:IA1}
                \end{minipage}
                \hfill
                \begin{minipage}[t]{0.45\textwidth}
                    \epsfxsize=1.0\linewidth
                    \epsffile{\directory/data.tsinstant-s.eps}
                    \caption[{\market}, instantaneous value of
                        Shannon probability.]{{\market}, instantaneous
                        value of the Shannon probability of the
                        normalized increments, provided by running the
                        program {\it tsinstant}\/ with the -s option
                        on the data presented in
                        Figure~\ref{\SETLABEL:TS}.}
                    \label{\SETLABEL:IA2}
                    \label{\SETLABELQ:IA2}
                \end{minipage}
            \end{center}
        \end{figure}

% Local Variables:
% TeX-parse-self: t
% TeX-auto-save: t
% TeX-master: "fractal.tex"
% End:


        %
% -----------------------------------------------------------------------------
%
% A license is hereby granted to reproduce this software source code and
% to create executable versions from this source code for personal,
% non-commercial use.  The copyright notice included with the software
% must be maintained in all copies produced.
%
% THIS PROGRAM IS PROVIDED "AS IS". THE AUTHOR PROVIDES NO WARRANTIES
% WHATSOEVER, EXPRESSED OR IMPLIED, INCLUDING WARRANTIES OF
% MERCHANTABILITY, TITLE, OR FITNESS FOR ANY PARTICULAR PURPOSE.  THE
% AUTHOR DOES NOT WARRANT THAT USE OF THIS PROGRAM DOES NOT INFRINGE THE
% INTELLECTUAL PROPERTY RIGHTS OF ANY THIRD PARTY IN ANY COUNTRY.
%
% Copyright (c) 1994-2006, John Conover, All Rights Reserved.
%
% Comments and/or bug reports should be addressed to:
%
%     john@email.johncon.com (John Conover)
%
% -----------------------------------------------------------------------------
%
% Revision: \RCSRevision \\
% Revision Time: \RCSTime UMT \\
% Revision Date: \RCSDate \\
% Revision Id: \RCSId \\
% Revision File: \RCSLog \\
\RCS $Revision: 0.0 $
\RCS $Date: 2006/01/20 04:38:13 $
\RCS $Id: logistic.tex,v 0.0 2006/01/20 04:38:13 john Exp $
% $Log: logistic.tex,v $
% Revision 0.0  2006/01/20 04:38:13  john
% Initial version
%
%
    \subsection{Logistic Analysis}
        \label{\SETLABEL:LA}

        \subidx{\market}{Logistic function analysis}
        \subidx{time series}{logistic function}
        \subidx{logistic function}{time series}
        \subidx{time series}{increments}
        \subidx{time series}{analysis}
        \subidx{cumulative sum}{analysis}
        \subidx{analysis}{cumulative sum}
        \subidx{analysis}{random process}
        \subidx{random process}{analysis}
        The data in this section is presented in tabular form in
        Section~\ref{\SETLABELREF:LAA}.  Figure~\ref{\SETLABEL:LA1} is
        a graph of the logistic function estimates of the time series
        data for the {\market}. The reader is cautioned that these
        graphs are constructed using the method suggested in
        Chapter~\ref{general}, Section~\ref{nlextend} and enormous
        precision is required for adequate prediction of the logistic
        function,~\cite{Modis}. Particularly, the non-linear term will
        usually require intervention to produce a practical fit to the
        data. In addition, there are numerical stability issues with
        logistic function methodologies\footnote{For example, in
        Figures~\ref{\SETLABEL:LA1} and~\ref{\SETLABEL:LA2}, if the
        non-linear term, $b$, was greater than zero, it was set to
        zero to produce the graphs. See Section~\ref{\SETLABELREF:LAA}
        for the actual derived values. In other cases, the magnitude
        of $b$ was too large, resulting in a graph that was decreasing
        as a function of time}.  The methodology should be regarded as
        ``fragile.'' It is included for completeness.

        \idx{least squares approximation}
        Figure~\ref{\SETLABEL:LA1} is a graph of the logistic function
        for the time series data presented in
        Figure~\ref{\SETLABEL:TS}. The data presented was made by
        running the program {\it tsdlogistic}\/, which is described
        briefly in Appendix~\ref{programs}, on the parameters
        extracted from the time series data as suggested in
        Figure~\ref{\SETLABEL:TF}. The program {\it tslsq}\/ was used
        to derive the constant and the slope of the normalized
        increments of the data presented in Figure~\ref{\SETLABEL:TF}.
        Figure~\ref{\SETLABEL:LA2} is the same graph, but with the
        time scale expanded by a factor of two.

        \begin{figure}[ht]
            \begin{center}
                \begin{minipage}[t]{0.45\textwidth}
                    \epsfxsize=1.0\linewidth
                    \epsffile{\directory/data.tsfraction.tslsq-p.tsdlogistic.eps}
                    \caption[{\market}, logistic function
                        estimates.]{{\market}, logistic function
                        estimates, provided by running the {\it
                        tslsq}\/ program on the normalized increments
                        presented in Figure~\ref{\SETLABEL:TF} with
                        the -p option. These parameters were used as
                        arguments to the {\it tsdlogistic}\/ program.}
                    \label{\SETLABEL:LA1}
                    \label{\SETLABELQ:LA1}
                \end{minipage}
                \hfill
                \begin{minipage}[t]{0.45\textwidth}
                    \epsfxsize=1.0\linewidth
                    \epsffile{\directory/data.tsfraction.tslsq-p.tsdlogistic2.eps}
                    \caption[{\market}, logistic function
                        estimates.]{{\market}, logistic function
                        estimates of Figure~\ref{\SETLABEL:LA1} with
                        the time scale expanded by a factor of two.}
                    \label{\SETLABEL:LA2}
                    \label{\SETLABELQ:LA2}
                \end{minipage}
            \end{center}
        \end{figure}

% Local Variables:
% TeX-parse-self: t
% TeX-auto-save: t
% TeX-master: "fractal.tex"
% End:


        %
% -----------------------------------------------------------------------------
%
% A license is hereby granted to reproduce this software source code and
% to create executable versions from this source code for personal,
% non-commercial use.  The copyright notice included with the software
% must be maintained in all copies produced.
%
% THIS PROGRAM IS PROVIDED "AS IS". THE AUTHOR PROVIDES NO WARRANTIES
% WHATSOEVER, EXPRESSED OR IMPLIED, INCLUDING WARRANTIES OF
% MERCHANTABILITY, TITLE, OR FITNESS FOR ANY PARTICULAR PURPOSE.  THE
% AUTHOR DOES NOT WARRANT THAT USE OF THIS PROGRAM DOES NOT INFRINGE THE
% INTELLECTUAL PROPERTY RIGHTS OF ANY THIRD PARTY IN ANY COUNTRY.
%
% Copyright (c) 1994-2006, John Conover, All Rights Reserved.
%
% Comments and/or bug reports should be addressed to:
%
%     john@email.johncon.com (John Conover)
%
% -----------------------------------------------------------------------------
%
% Revision: \RCSRevision \\
% Revision Time: \RCSTime UMT \\
% Revision Date: \RCSDate \\
% Revision Id: \RCSId \\
% Revision File: \RCSLog \\
\RCS $Revision: 0.0 $
\RCS $Date: 2006/01/20 04:38:13 $
\RCS $Id: hurst.tex,v 0.0 2006/01/20 04:38:13 john Exp $
% $Log: hurst.tex,v $
% Revision 0.0  2006/01/20 04:38:13  john
% Initial version
%
%
    \subsection{Hurst Coefficient Analysis}
        \label{\SETLABEL:H}

        \subidx{\market}{Hurst coefficient analysis}
        \subidx{Hurst coefficient}{analysis}
        \subidx{increments}{normalized}
        \subidx{normalized}{increments}
        \subidx{programs}{tshurst}
        \subidx{tshurst}{program}
        The data in this section is presented in tabular form in
        Section~\ref{\SETLABELREF:HCHP}. Figure~\ref{\SETLABEL:HC} is
        a graph of the Hurst coefficient data time series data shown
        in Figure~\ref{\SETLABEL:TS}. The slope of the graph is the
        Hurst coefficient.  The data for this figure was produced by
        the program {\it tshurst}\/, which is described briefly in
        Appendix~\ref{programs}.

        \subidx{\market}{H parameter analysis}
        \subidx{H parameter}{analysis}
        \subidx{programs}{tshcalc}
        \subidx{tshcalc}{program}
        Figure~\ref{\SETLABEL:HP} is a graph of the H parameter data
        for the normalized increments of the time series data shown in
        Figure~\ref{\SETLABEL:TF}. The data for this figure was
        produced by the program {\it tshcalc}\/, which is described
        briefly in Appendix~\ref{programs}.

        \begin{figure}[ht]
            \begin{center}
                \begin{minipage}[t]{0.45\textwidth}
                    \epsfxsize=1.0\linewidth
                    \epsffile{\directory/data.tshurst.eps}
                    \caption[{\market}, Hurst coefficient data]{{\market},
                        Hurst coefficient data for the normalized
                        increments of the time series data shown in
                        Figure~\ref{\SETLABEL:TF}.  The slope of the graph
                        is the Hurst coefficient.}
                    \label{\SETLABEL:HC}
                \end{minipage}
                \hfill
                \begin{minipage}[t]{0.45\textwidth}
                    \epsfxsize=1.0\linewidth
                    \epsffile{\directory/data.tshcalc.eps}
                    \caption[{\market}, H parameter data]{{\market}, H
                        parameter data for the normalized increments of
                        the time series data shown in
                        Figure~\ref{\SETLABEL:TF} The slope of the graph
                        is the H parameter.}
                    \label{\SETLABEL:HP}
                \end{minipage}
            \end{center}
        \end{figure}

        \subidx{revenue}{See, rate of revenue returns}
        \subidx{returns}{See, rate of revenue returns}
        \subidx{\market}{revenues}
        \subidx{Hurst coefficient}{analysis}
        \subidx{\market}{Hurst coefficient analysis}
        \subidx{\market}{rate of change}
        \subidx{\market}{windows of opportunity}
        \subidx{rate of revenue returns}{forecast}
        \subidx{forecast}{rate of revenue returns}
        \idx{windows of opportunity}
        \subidx{programs}{tslsq}
        \subidx{tslsq}{program}

        The approximately linear slope of the graph in
        Figure~\ref{\SETLABEL:HC} implies that the variance of the
        rate of revenue returns, (per {\timescale},) in the {\market},
        $V(t_2 - t_1)$, over a period of time is proportional to the
        period of time raised to twice the Hurst
        coefficient~\cite[pp. 180]{Feder},~\cite[pp. 246]{Crownover}.
        This seems to be a quantitative statement concerning how fast,
        and to what degree, the rate of revenue returns' state of
        affairs can change over a period of time.  An additional
        implication, for Hurst coefficients sufficiently close to 0.5,
        is that the probability of the state of affairs repeating
        sometime in the future goes down with increasing
        time\footnote{It can be shown that the number of expected
        market ``high'' and ``low'' transitions, $N$, scales with the
        square root of time, or $N \propto \sqrt {t}$, meaning that
        the cumulative distribution of the probability, $P$, of the
        duration of a market's ``high'' or ``low'' exceeding a given
        time interval, $t$, is proportional to the reciprocal of the
        square root of the time interval, $P \propto 1 / \sqrt {t}$,
        (or, conversely, that the probability of the duration of a
        market's ``high'' or ``low'' exceeding a given time interval
        is proportional to the reciprocal of the time interval raised
        to the power $3 / 2$, ie., $P \propto 1 / t^{3 /
        2}$,~\cite[pp. 153]{Schroeder}. What this means is that a
        histogram of the ``zero free'' run-lengths of a market being
        ``high'' or ``low,'' over a long time, would have a $1 / t^{3
        / 2}$ characteristic.)}, $t$, $p(t) = erf (1/\sqrt{2t})$ which
        is approximately $1/\sqrt{t}$ for $t \gg
        1$~\cite[pp. 160]{Schroeder}. Figures~\ref{\SETLABEL:FN},
        and,~\ref{\SETLABEL:FF} compare methods of approximation of
        the ``forecastability'' of the rate of revenue returns in the
        {\market} for the near term and far term,
        respectively~\cite[pp. 83-84]{Peters:CAOITCM}\footnote{The
        author is not comfortable with Peters' interpretation. For
        example, if the algorithm explained
        in~\cite[pp. 82]{Peters:CAOITCM} is used on ``white noise''
        which, by definition, never has any correlations, the short
        term Hurst coefficient, and thus the ``forecastability,'' is
        still near unity---a bit of an enigma. This can be verified
        with the {\it tswhite}\/ and {\it tshurst}\/ programs, which
        are briefly described in Appendix~\ref{programs}.}.  This
        seems to be a quantitative statement concerning ``windows of
        opportunity'' in the rate of revenue returns, (per
        {\timescale}.)  The program {\it tslsq}\/ was used on the
        Hurst coefficient data, presented in
        Figure~\ref{\SETLABEL:HC}, to provide a least squares
        approximation to the Hurst coefficient. The superimposed least
        squares approximation with on original Hurst coefficient data
        is presented.  The time series data has a Hurst coefficient of
        {\thurstlow}, so that:

        \subidx{\market}{Hurst coefficient analysis}
        \begin{eqnarray}
            V\left(t_2 - t_1\right) & \propto & \left(t_2 - t_1\right)^{2 \cdot H}\\
            V\left(t_2 - t_1\right) & \propto & \left(t_2 - t_1\right)^{2 \cdot {\thurstlow}}\\
                                    & \propto & \left(t_2 - t_1\right)^{\thurstlowtwo}
            \label{\SETLABEL:V}
        \end{eqnarray}

        \subidx{fractional}{Brownian motion}
        \subidx{Brownian motion}{fractional}
        \idx{fractal}
        \noindent where $V(t_2 - t_1)$ is the variance of the
        increments of the rate of revenue returns, (per {\timescale},)
        over the time interval $t_2 -
        t_1$,~\cite[pp. 177]{Feder},~\cite[pp. 494]{Peitgen}. If $H >
        \frac{1}{2}$, then the time series is termed as being
        characterized by ``fractional Brownian
        motion~\cite[pp. 170]{Feder}.''

        \subidx{rate of revenue returns}{predictability}
        \subidx{rate of revenue returns}{forecastability}
        \subidx{rate of revenue returns}{consistency}
        \subidx{predictability}{rate of revenue returns}
        \subidx{forecastability}{rate of revenue returns}
        \subidx{consistency}{rate of revenue returns}
        \subidx{\market}{rate of revenue returns, predictability}
        \subidx{\market}{rate of revenue returns, forecastability}
        \subidx{\market}{rate of revenue returns, consistency}
        \subidx{Hurst coefficient}{analysis}
        \subidx{\market}{Hurst coefficient analysis}
        \subidx{\market}{rate of change}

        In some sense, the Hurst coefficient is a quantitative
        expression of the ``forecastability'' of the future based on
        the past\footnote{Actually, in general, when summing fractal
        entities, the method used should be a root mean square
        process, dependent on the Hurst Coefficient, $H$, where
        $P_{total}^H = P_1^H + P_2^H + \cdots$, where $P_n$ is the
        fractal entities. For a Brownian motion, or random walk type
        of fractal the Hurst Coefficient is a function of time into
        the future. For the ``near term,'' the Hurst coefficient is
        very near unity, meaning the summation process is linear. For
        the ``long term,'' $H \approx 0.5$, or a standard root mean
        square summation process should be used. If $H$ is $0.5$ then
        the market is termed a Brownian motion, or random walk
        process. If it is larger than 0.5, it is termed fractional
        Brownian motion process. For a random walk process, ``near
        term'' and ``far term'' are quantitatively differentiated on
        the Hurst Coefficient graph where $1 - \ln (t) = 0.5 \cdot \ln
        (t)$, or when $\ln (t) = 2$, or $t = 7.389\ldots$ See
        Section~\ref{\SETLABEL:FS} for the particulars on using Hurst
        Coefficient to sum fractal process' for the {\market}. See
        also~\cite[pp. 67, 83-84]{Peters:CAOITCM} and~\cite[pp. 129,
        159]{Schroeder} for particulars on the implications of the
        Hurst Coefficient and root mean square summation issues.}.  A
        Hurst coefficient of {\thurstlow}, (for the near future, and
        {\thurstall} for the distant future.) implies that the
        likelihood of the rate of revenue returns, (per {\timescale},)
        for any two consecutive {\timescale}s being the same is
        {\thurstlowhundred}\%~\cite[pp. 66]{Peters:CAOITCM} for the
        near future, and {\thurstall} for the distant
        future. Likewise, there is a {\thurstlowhundred}\% chance of
        the rate of revenue returns, (per {\timescale},) movements
        being the same in consecutive time periods---ie., if, in a
        given {\timescale}, the rate of revenue returns, (per
        {\timescale},) is increasing, there is a {\thurstlowhundred}\%
        that the rate of revenue returns, (per {\timescale},) will
        increase in the following period, also. In some sense, this is
        a quantitative statement on how ``predictable,'' or
        ``forecastable'' the rate of revenue returns, (per
        {\timescale},) for the {\market} are over time, since the
        probability of having $n$ many consecutive {\timescale}s of
        the same agenda is $H^n$ where $H$ is the Hurst coefficient,
        or, letting the short term probability of having $n$ many
        {\timescale}s of the same market agenda, $p_a$, is:

        \begin{eqnarray}
            p_a\left(n\right) & = & H^{n}\\
                              & = & {\thurstlow}^{n}
            \label{\SETLABEL:MA}
        \end{eqnarray}

        \subidx{rate of revenue returns}{predictability}
        \subidx{rate of revenue returns}{forecastability}
        \subidx{rate of revenue returns}{consistency}
        \subidx{predictability}{rate of revenue returns}
        \subidx{forecastability}{rate of revenue returns}
        \subidx{consistency}{rate of revenue returns}
        As an interesting interpretation of the normalized increments
        of the time series data presented in
        Figure~\ref{\SETLABEL:TF}, if the vertical axis is multiplied
        by 100, to convert to percent, then the graph represents the
        error, in percent, that would be made by forecasting, month by
        month, that the next {\timescale}'s rate of revenue returns
        would be the same as the current {\timescale}'s revenue
        rate. Interestingly, it is $\datafractionmean \cdot 100$
        percent, on the average, with a standard deviation of
        $\datafractionstddev \cdot 100$ percent, and a root mean
        square error value of $\datafractionrms \cdot 100$
        percent---small values for such a simple forecasting
        mechanism.

        \subidx{\market}{rate of revenue returns, range}
        \subidx{Hurst coefficient}{analysis}
        \subidx{\market}{Hurst coefficient analysis}
        \subidx{\market}{rate of change}

        This is, essentially, a statement of the range of values, in
        the increments of the rate of revenue returns, (per
        {\timescale},) that is to be expected over the time interval,
        $t_2 - t_1$,
        $R_v$,~\cite[pp. 178]{Feder},~\cite[pp. 172]{Cambel}:

        \begin{eqnarray}
            R_v\left(t_2 - t_1\right) & \propto & \left(t_2 - t_1\right)^{H}\\
                                      & \propto & \left(t_2 - t_1\right)^{\thurstlow}
            \label{\SETLABEL:R}
        \end{eqnarray}

        \subidx{\market}{rate of revenue returns, range}
        \subidx{Hurst coefficient}{analysis}
        \subidx{\market}{Hurst coefficient analysis}
        \subidx{\market}{rate of change}
        \subidx{Markov}{statistics}
        \subidx{statistics}{Markov}
        \noindent where $R$ is the range of values in the increments
        of the rate of revenue returns, (per {\timescale}.) A Hurst
        coefficient, $H$, that is much larger than $\frac{1}{2}$, (but
        less than 1,) implies a strongly non-Gaussian distribution in
        the increments of the rate of revenue returns, (per
        {\timescale},)~\cite[pp. 152, 194]{Feder}, and a Hurst
        coefficient near $\frac{1}{2}$ implies that the increments of
        the rate of revenue returns, (per {\timescale}) is
        characteristic of an independent
        process~\cite[pp. 195]{Feder}. Extreme caution should be
        exercised in using Markov statistics in any analysis where the
        Hurst coefficient is not
        $\frac{1}{2}$,~\cite[pp. 124]{Crownover},~\cite[pp. 106]{Peters:CAOITCM}.


        As a useful approximation, if $H$, is approximately
        $\frac{1}{2}$, Equation~\ref{\SETLABEL:R} reduces
        to,~\cite[pp. 129]{Schroeder}:

        \begin{eqnarray}
            R\left(t_2 - t_1\right) & \propto & (t_2 - t_1)^{\frac{1}{2}}\\
                                    & \propto & \sqrt{\left(t_2 - t_1\right)}
        \end{eqnarray}

        \subidx{\market}{rate of revenue returns, range}
        \subidx{\market}{rate of revenue returns, increase and decrease}
        \subidx{Hurst coefficient}{analysis}
        \subidx{\market}{Hurst coefficient analysis}
        \subidx{\market}{rate of change}
        \subidx{Markov}{statistics}
        \subidx{statistics}{Markov}

        In the case where the Hurst coefficient, $H$, is
        $\frac{1}{2}$, the range of values in the increments of the
        rate of revenue returns, (per {\timescale},) divided by the
        standard deviation of these values, $S$, can be anticipated to
        increase over time according to the following
        relation,~\cite[pp. 154]{Feder},~\cite[pp. 129]{Schroeder}:

        \begin{equation}
            \frac{R\left(t_2 - t_1\right)}{S} \propto \left(t_2 - t_1\right)^{\frac{1}{2}}
        \end{equation}

        \subidx{\market}{rate of revenue returns, range}
        \subidx{\market}{rate of revenue returns, increase and decrease}
        \subidx{Hurst coefficient}{analysis}
        \subidx{\market}{Hurst coefficient analysis}
        \subidx{\market}{rate of change}
        \noindent which is a useful conceptual approximation, since it
        involves only the square root function---if the range and the
        standard deviation of the increments of the rate of revenue
        returns, (per {\timescale},) are known, (and $H \approx
        \frac{1}{2}$,) then the expected change in $\frac{R}{S}$, will
        increase with the square root of time\footnote{To be precise,
        it is actually asymptotically proportional to
        $\tau^{\frac{1}{2}}$}.

        Another useful approximation when rescaling processes that are
        characterize by Brownian motion, (ie., when $H \approx
        \frac{1}{2}$,) is that:

        \begin{eqnarray}
            X\left(t\right) & \propto & \frac{X\left(rt\right)}{r^{H}}\\
                            & \propto & \frac{X\left(rt\right)}{r^{\thurstlow}}
        \end{eqnarray}

        \idx{Brownian motion}
        \idx{fractal}
        Where $X(t)$ is the process characterized by Brownian motion,
        and $r$ is a scaling factor,~\cite[pp. 494]{Peitgen}.

        \subidx{programs}{tslsq}
        \subidx{tslsq}{program}
        The program {\it tslsq}\/ was used on the H parameter data,
        presented in Figure~\ref{\SETLABEL:HP}, to provide a least
        squares approximation to the H parameter for the
        {\market}. The superimposed least squares approximation on the
        original H parameter data is presented.  By contrast, the H
        parameter, as derived by the methodology outlined
        in~\cite[pp. 249]{Crownover}, is {\thcalclow} for the near
        future, and {\thcalcall} for the distant future.

        \subidx{\market}{Hurst coefficient analysis}
        \subidx{Hurst coefficient}{analysis}
        \subidx{increments}{normalized}
        \subidx{normalized}{increments}
        \subidx{programs}{tshurst}
        \subidx{tshurst}{program}
        \subidx{\market}{H parameter analysis}
        \subidx{H parameter}{analysis}
        \subidx{programs}{tshcalc}
        \subidx{tshcalc}{program}
        Figures~\ref{\SETLABEL:HC} and~\ref{\SETLABEL:HP} represent
        Hurst coefficient and H parameter data that are derived from
        the normalized increments, shown in
        Figure~\ref{\SETLABEL:TF}. In this case, the data is
        considered a normalized derivative of the time series data
        presented in Figure~\ref{\SETLABEL:TF}, instead of a
        cumulative sum.  The program, {\it tshurst}\/, is described
        briefly in appendix~\ref{programs}, and the data for
        figures~\ref{\SETLABEL:THC} and~\ref{\SETLABEL:THP} was made
        using the -d option.

        \begin{figure}[ht]
            \begin{center}
                \begin{minipage}[t]{0.45\textwidth}
                    \epsfxsize=1.0\linewidth
                    \epsffile{\directory/data.tsfraction.tshurst-d.eps}
                    \caption[{\market}, traditional Hurst coefficient
                        data]{{\market}, traditional Hurst coefficient
                        data for the time series data shown in
                        Figure~\ref{\SETLABEL:TS}.  The slope of the
                        graph is the Hurst coefficient, and is
                        {\hurstlow} for the near term, and
                        {\hurstall} for the far term.}
                    \label{\SETLABEL:THC}
                \end{minipage}
                \hfill
                \begin{minipage}[t]{0.45\textwidth}
                    \epsfxsize=1.0\linewidth
                    \epsffile{\directory/data.tsfraction.tshcalc-d.eps}
                    \caption[{\market}, traditional H parameter
                        data]{{\market}, traditional H parameter data
                        for the time series data shown in
                        Figure~\ref{\SETLABEL:TS} The slope of the
                        graph is the H parameter, and is {\hcalclow}
                        for the near term, and {\hcalcall} for the
                        far term.}
                    \label{\SETLABEL:THP}
                \end{minipage}
            \end{center}
        \end{figure}

% Local Variables:
% TeX-parse-self: t
% TeX-auto-save: t
% TeX-master: "fractal.tex"
% End:


        \subsubsection{Observations on the Hurst Coefficient Analysis}

            Note that the H parameter data is not linear, and the long
            term predictability is better than the short term
            predictability, indicating that the least squares
            approximation is low.

        %
% -----------------------------------------------------------------------------
%
% A license is hereby granted to reproduce this software source code and
% to create executable versions from this source code for personal,
% non-commercial use.  The copyright notice included with the software
% must be maintained in all copies produced.
%
% THIS PROGRAM IS PROVIDED "AS IS". THE AUTHOR PROVIDES NO WARRANTIES
% WHATSOEVER, EXPRESSED OR IMPLIED, INCLUDING WARRANTIES OF
% MERCHANTABILITY, TITLE, OR FITNESS FOR ANY PARTICULAR PURPOSE.  THE
% AUTHOR DOES NOT WARRANT THAT USE OF THIS PROGRAM DOES NOT INFRINGE THE
% INTELLECTUAL PROPERTY RIGHTS OF ANY THIRD PARTY IN ANY COUNTRY.
%
% Copyright (c) 1994-2006, John Conover, All Rights Reserved.
%
% Comments and/or bug reports should be addressed to:
%
%     john@email.johncon.com (John Conover)
%
% -----------------------------------------------------------------------------
%
% Revision: \RCSRevision \\
% Revision Time: \RCSTime UMT \\
% Revision Date: \RCSDate \\
% Revision Id: \RCSId \\
% Revision File: \RCSLog \\
\RCS $Revision: 0.0 $
\RCS $Date: 2006/01/20 04:38:13 $
\RCS $Id: fiscal.tex,v 0.0 2006/01/20 04:38:13 john Exp $
% $Log: fiscal.tex,v $
% Revision 0.0  2006/01/20 04:38:13  john
% Initial version
%
%
    \subsection{Fixed Increment Approximation for Fiscal Strategy}
        \label{\SETLABEL:FS}

        \subidx{\market}{fiscal strategy}
        \subidx{markets}{analysis}
        \subidx{analysis}{markets}
        \subidx{strategy}{fiscal}
        \subidx{fiscal}{strategy}
        The data in this section is presented in tabular form in
        Section~\ref{\SETLABELREF:LR}. This section derives various
        values based on the ``average'' of the normalized increments
        presented in Figure~\ref{\SETLABEL:TFA}. These values are an
        approximation to a, probably, complex process with a
        distribution shown in Figure~\ref{\SETLABEL:TF}. These values
        will be used in a fixed increment Brownian fractal analysis
        and simulation of the {\market}, and may, or may not, provide
        adequate accuracy for projections.

        For an organization operating in the {\market}, the fiscal
        strategy, commensurate with the aggregate environment, can be
        derived as follows~\cite[pp. 128, pp
        151]{Schroeder},~\cite[pp. 450]{Reza},~\cite[pp. 270]{Pierce}:
        \vspace{0.15in}

        \subsubsection{Logarithmic Returns}
            \label{\SETLABEL:LR}

            \subidx{logarithmic}{returns}
            \subidx{returns}{logarithmic}
            \subidx{\market}{logarithmic returns}
            The logarithmic returns can be calculated by various
            means. Four will be presented here, for comparison.

            \subidx{programs}{tsnormal}
            \subidx{tsnormal}{program}
            \subidx{logarithmic}{returns}
            \subidx{returns}{logarithmic}
            The logarithmic returns, in bits, $bits$, as computed from
            the mean, by the program {\it tsnormal}\/, which is
            described in Chapter~\ref{programs}, and is presented in
            Figure~\ref{\SETLABEL:TF}, and Equation~\ref{abits} from
            Section~\ref{ereturns} in Chapter~\ref{general}:

            \begin{equation}
                bits = \frac{\ln \left({\datafractionmean} + 1\right)}{\ln \left(2\right)} = \datafractionmeanbits
            \end{equation}

            \subidx{programs}{tslsq}
            \subidx{tslsq}{program}
            \subidx{logarithmic}{returns}
            \subidx{returns}{logarithmic}
            \noindent By comparison, the logarithmic returns, in bits,
            $bits$, as computed from the constant in the least squares
            approximation, using the program {\it tslsq}\/, which is briefly
            described in Chapter~\ref{programs}, as presented in
            Figure~\ref{\SETLABEL:TF}, and Equation~\ref{abits} from
            Section~\ref{ereturns} in Chapter~\ref{general}:

            \begin{equation}
                bits = \frac{\ln \left({\datafractionconstant} + 1\right)}{\ln \left(2\right)} = \datafractionconstantbits
            \end{equation}

            Note that if the mean is not constant in
            Figure~\ref{\SETLABEL:TF}, this method will not provide
            accurate results.

            \subidx{programs}{tslsq}
            \subidx{tslsq}{program}
            \subidx{logarithmic}{returns}
            \subidx{returns}{logarithmic}
            \noindent And by yet another comparison, using the program
            {\it tslsq}\/, which is briefly described in
            Chapter~\ref{programs}, with the -e -p options, to provide
            a formula for the least squares exponential fit to the
            time series data set presented in
            Figure~\ref{\SETLABEL:TS}:

            \begin{equation}
                bits = {\datatslsqepbits}
            \end{equation}

            \subidx{programs}{tslogreturns}
            \subidx{tslogreturns}{program}
            \subidx{logarithmic}{returns}
            \subidx{returns}{logarithmic}
            \noindent And finally, by comparison, from the
            {\it tslogreturns}\/ program, which is briefly described
            in Chapter~\ref{programs}, with the -p option, to provide
            a formula for the logarithmic returns of the time series
            data set presented in Figure~\ref{\SETLABEL:TS}:

            \begin{equation}
                bits = {\logreturns}
            \end{equation}

        \subsubsection{Calculation of Shannon Probability}
            \label{\SETLABEL:SP}

            \subidx{\market}{Shannon probability}
            Ideally, all of the values presented in
            Section~\ref{\SETLABEL:LR} would be equal. Using the
            logarithmic returns provided by the {\it tslogreturns}\/
            program, to be consistent
            with~\cite[pp. 81]{Peters:CAOITCM}

            \subidx{programs}{tslogreturns}
            \subidx{tslogreturns}{program}
            \begin{equation}
                2^{{\logreturns}t}
            \end{equation}

            \noindent therefore:
            \begin{equation}
                C\left(p\right) = {\logreturns}
            \end{equation}
            \subidx{programs}{tsshannon}
            \subidx{tsshannon}{program}
            \subidx{Shannon}{probability}
            \subidx{probability}{Shannon}
            \noindent and, {\it tsshannon}\/ {\logreturns} gives:
            \begin{equation}
                \label{\SETLABEL:F0}
                C\left({\shannonlogreturns}\right) = {\logreturns}
            \end{equation}
            \noindent therefore:
            \begin{eqnarray}
                2^{C\left({\shannonlogreturns}\right)} & = & 2^{\logreturns}\\
                                                       & = & {\twologreturns}\\
                                                       & = & {\twologreturnshundred}\%
            \end{eqnarray}
            \noindent and:
            \begin{eqnarray}
                2p - 1 & = & \left(2 \cdot {\shannonlogreturns}\right) - 1\\
                       & = & {\twopone}\\
                       \label{\SETLABEL:F1}
                       & = & {\twoponehundred}\%
            \end{eqnarray}

            \subidx{\market}{fiscal strategy}
            \subidx{markets}{analysis}
            \subidx{analysis}{markets}
            \subidx{strategy}{fiscal}
            \subidx{fiscal}{strategy}
            \subidx{\market}{fiscal strategy}
            \subidx{\market}{growth rate}
            Presuming the simplified assumptions outlined in
            Section~\ref{assumptions}, the ``typical'' organization
            operating in the {\market} executes a long term fiscal
            strategy, commensurate with the aggregate environment,
            that is to invest, every {\timescale}, in sufficient
            additional resources and infrastructure, to increase the
            manufacturing of goods and services by {\twoponehundred}\%
            of its rate of revenue returns, (per {\timescale}.) As a
            conceptual model, the remaining {\hundredtwoponehundred}\%
            will be held in ``reserve'' with a
            {\shannonlogreturnshundred}\% chance of making twice the
            {\twoponehundred}\% back, (and a
            {\hundredshannonlogreturnshundred}\% chance of making
            0.0,) in one {\timescale}, on the average, for an average
            growth in its rate of revenue returns, (per {\timescale},)
            of {\twologreturnshundred}\%, or a doubling of its rate of
            revenue returns, (per {\timescale},) in
            {\oneoverlogreturns} {\timescale}s.

        \subsubsection{Example Fixed Increment Approximation Fiscal Strategies}

            \subidx{\market}{fiscal strategy}
            \subidx{markets}{analysis}
            \subidx{analysis}{markets}
            \subidx{strategy}{fiscal}
            \subidx{fiscal}{strategy}
            \subidx{\market}{fiscal strategy}
            \subidx{\market}{growth rate}
            \subidx{\market}{management metric}
            \idx{management metric}
            A possible metric on the effectiveness of long term fiscal
            management could possibly be that if an investment of
            {\twoponehundred}\% per {\timescale} of the rate of
            revenue returns, (per {\timescale},) is made in resources
            and infrastructure, then the rate of revenue returns would
            be expected to increase by {\twologreturnshundred}\%, per
            {\timescale}, on average.

            Note that the metrics presented in this section are
            representative of the {\market} as an aggregate whole, and
            may or may not be accurate representations for any
            particular participant in the environment. Of interest to
            the participants in the environment would be a similar
            analysis of each product or service rendered in the
            marketplace.

            \subidx{\market}{fiscal strategy}
            \subidx{markets}{analysis}
            \subidx{analysis}{markets}
            \subidx{strategy}{fiscal}
            \subidx{fiscal}{strategy}
            \subidx{\market}{fiscal strategy}
            As a simple illustrative example, a company operating in
            this environment might obtain a credit line from a bank
            that is equal to {\twoponehundred}\% of its rate of
            revenue returns, (per {\timescale},) to finance additional
            operations. In this simple scenario, the company would use
            its revenue base as collateral for the loan. Some
            {\timescale}s, depending on the {\market}'s environment,
            the company's rate of revenue returns exceeds what was
            borrowed from the bank, and the loan is repaid in
            full. Other {\timescale}s, the company must default, and
            the bank seizes a portion of the company's revenue base to
            pay the delinquent loan. However, on the average, the
            company will expand its rate of revenue returns at
            {\twologreturnshundred}\% per {\timescale}.

            \subidx{\market}{fiscal strategy}
            \subidx{markets}{analysis}
            \subidx{analysis}{markets}
            \subidx{strategy}{fiscal}
            \subidx{fiscal}{strategy}
            \subidx{\market}{fiscal strategy}
            As another simple example, a company re-invests
            {\twoponehundred}\% of its rate of revenue returns, (per
            {\timescale},) in development, marketing, sales, and
            distribution of new products.  Although some products will
            be successful and the return on the investment will exceed
            the {\twoponehundred}\% per {\timescale} investment,
            others will not. However, on the average, the company will
            expand it gross rate of revenue returns at
            {\twologreturnshundred}\% per {\timescale}.

            \subidx{\market}{fiscal strategy}
            \subidx{markets}{analysis}
            \subidx{analysis}{markets}
            \subidx{strategy}{fiscal}
            \subidx{fiscal}{strategy}
            \subidx{\market}{fiscal strategy}
            \subidx{\market}{product portfolio}
            \subidx{\market}{product diversity}
            \subidx{\market}{product mix}
            \subidx{\market}{optimum number of products}
            \idx{product portfolio}
            \idx{product diversity}
            \idx{optimum number of products}
            \idx{product mix}

            As an example of ``product portfolio'' management, suppose
            a company re-invests {\twoponehundred}\% of its rate of
            revenue returns, (per {\timescale},) in development,
            marketing, sales, and distribution of new products.
            Further suppose that the company has two products, and a
            fractal analysis of the individual product rate of revenue
            return time series indicates that one product has a
            Shannon probability of 0.65, and the other has a Shannon
            probability of 0.55. Then the percentage of re-investment
            in the first product would be $(2 \cdot 0.65 - 1) \cdot
            {\twoponehundred}$, percent of the rate of revenue
            returns, and $(2 \cdot 0.55 - 1) \cdot {\twoponehundred}$
            percent for the second product, implying that the company
            should diversify its product line\footnote{The astute
            reader would note that the linear addition was used to add
            the contribution to development of each product. This is a
            ``near term'' interpretation. Actually, in general, the
            method used should be a root mean square process,
            dependent on the Hurst Coefficient, $H$, where
            $P_{total}^H = P_1^H + P_2^H + \cdots$, where $P_n$ is the
            contribution to each individual product. For a Brownian
            motion, or random walk type of fractal the Hurst
            Coefficient is a function of time into the future. For the
            ``near term,'' the Hurst coefficient is very near unity,
            meaning the summation process is linear. For the ``long
            term,'' $H \approx 0.5$, or a standard root mean square
            summation process should be used. If $H$ is $0.5$ then the
            market is termed a Brownian motion, or random walk
            process. If it is larger than 0.5, it is termed fractional
            Brownian motion process. For a random walk process, ``near
            term'' and ``far term'' are quantitatively differentiated
            on the Hurst Coefficient graph where $1 - \ln (t) = 0.5
            \cdot \ln (t)$, or when $\ln (t) = 2$, or $t =
            7.389\ldots$ See~\cite[pp. 67, 83-84]{Peters:CAOITCM}
            and~\cite[pp. 129, 159]{Schroeder} for particulars on the
            implications of the Hurst Coefficient and root mean square
            summation issues.}.  Note that this is a ``bet hedging''
            metric methodology, and assumes that the products have
            uncorrelated revenue return rates. If this re-investment
            methodology is not feasible, perhaps for strategic
            financial reasons, then the re-investment in both products
            should total the ${\twoponehundred}$\%, and the investment
            in each product should be made at a ratio of $\frac{(2
            \cdot 0.65 - 1)}{(2 \cdot 0.55 - 1)} = 3 : 1$,
            respectively. Note that this ``bet hedging'' can be used
            to define the optimal number of products that can be
            supported on the rate of revenue returns. If it assumed
            that all products are ``typical'' for the {\market}, as a
            standard bench mark, then the optimal number will be
            $\frac{1}{{\twopone}}$. Note that this is a
            ``theoretical'' value, since not all products are
            ``typical,'' and there may be strategic reasons, for
            example product leveraging, that may increase the number
            of products above the optimum. However, most of the
            revenue should come from the optimal number of products,
            since having more products will decrease the amount of the
            potential investment in each product, and having less than
            the optimum number of products will increase the risk that
            many of the products could suffer a ``down market''
            concurrently, impacting the rate of revenue returns.  As
            another interesting interpretation of the optimal
            ``hedging of bets,'' in product portfolio strategy, and
            considering the graph of the normalized increments
            presented in Figure~\ref{\SETLABEL:TF}, if the
            organization is running optimally, then these products
            will generate, at least in principle, one standard
            deviation, approximately $0.8413 = 84.13$\% of the future
            growth in rate of revenue returns. Naturally, these are
            approximations, and the values are an approximation to a,
            probably, complex process, and appropriate scrutiny should
            be exercised before making specific projections.  As yet
            another example of ``product portfolio'' management,
            consider the issue of product mix. In this interpretation,
            {\twoponehundred}\% of the product manufactured should be
            ``proprietary,'' while the rest is ``industry standard.''
            As yet another possibility, {\twoponehundred}\% of the
            product manufactured should be predatory into new markets,
            and the remainder in markets that are ``traditional'' for
            the company.

% Local Variables:
% TeX-parse-self: t
% TeX-auto-save: t
% TeX-master: "fractal.tex"
% End:


        \subsubsection{Observations on the Fixed Increment Approximation for Fiscal Strategy}

            A re-investment of {\twoponehundred} of the rate of
            revenue returns per {\timescale} does not seem
            inconsistent with the industry averages, since it includes
            investments in research and development, additional
            manufacturing infrastructure, advertising,
            etc. Additionally, a product mix of {\twoponehundred}\%
            ``proprietary'' and the remainder ``industry standard''
            products seems consistent with the industry analyst
            ``20/80'' rule. The value of one standard deviation,
            $84.13$\%, of the revenue return rate being generated by
            $\frac{1}{{\twopone}}$ products seems consistent with the
            industry, also.

        %
% -----------------------------------------------------------------------------
%
% A license is hereby granted to reproduce this software source code and
% to create executable versions from this source code for personal,
% non-commercial use.  The copyright notice included with the software
% must be maintained in all copies produced.
%
% THIS PROGRAM IS PROVIDED "AS IS". THE AUTHOR PROVIDES NO WARRANTIES
% WHATSOEVER, EXPRESSED OR IMPLIED, INCLUDING WARRANTIES OF
% MERCHANTABILITY, TITLE, OR FITNESS FOR ANY PARTICULAR PURPOSE.  THE
% AUTHOR DOES NOT WARRANT THAT USE OF THIS PROGRAM DOES NOT INFRINGE THE
% INTELLECTUAL PROPERTY RIGHTS OF ANY THIRD PARTY IN ANY COUNTRY.
%
% Copyright (c) 1994-2006, John Conover, All Rights Reserved.
%
% Comments and/or bug reports should be addressed to:
%
%     john@email.johncon.com (John Conover)
%
% -----------------------------------------------------------------------------
%
% Revision: \RCSRevision \\
% Revision Time: \RCSTime UMT \\
% Revision Date: \RCSDate \\
% Revision Id: \RCSId \\
% Revision File: \RCSLog \\
\RCS $Revision: 0.0 $
\RCS $Date: 2006/01/20 04:38:13 $
\RCS $Id: companies.tex,v 0.0 2006/01/20 04:38:13 john Exp $
% $Log: companies.tex,v $
% Revision 0.0  2006/01/20 04:38:13  john
% Initial version
%
%
    \subsection{Number of Companies}
        \label{\SETLABEL:QNC}

        \subidx{\market}{number of companies}
        \subidx{number of companies}{analysis}
        \subidx{analysis}{number of companies}
        \subidx{Shannon}{probability}
        \subidx{probability}{Shannon}
        This section evaluates the approximate, or ``average,'' number
        of companies in the {\market}, and uses the method outlined in
        Chapter~\ref{general}, Section~\ref{aftsma}. Since the
        average, $avg_{ind}$, and the root mean square, $rms_{ind}$,
        of the normalized increments of the {\market} time series is
        \datafractionmean, and \datafractionrms respectively, the
        number of companies participating in the market can be
        calculated by Equation~\ref{ncompanies} to be {\ncompanies}.

        If this value seems consistent number of companies in the
        {\market}, within the assumptions outlined in
        Chapter~\ref{general}, Section~\ref{aftsma}, then it would
        seem that there is some circumstantial or indirect evidence
        that the companies participating in the {\market} are
        operating optimally, and the ``average'' Shannon probability,
        $P$ for each participating company would be, using
        Equation~\ref{pncompanies}, {\pncompanies}, which would be the
        value which should be used in Section~\ref{\SETLABEL:FS} for
        each participating company if market expansion was to be
        consistent with the rest of the industry. However, if the
        Shannon probability derived in Section~\ref{\SETLABEL:FS} is
        greater than the average Shannon probability for the companies
        participating in the {\market}, as derived in this section,
        then the market would, possibly, be exploitable with the
        fiscal strategy outlined in Section~\ref{\SETLABEL:FS}. The
        maximum exploitability for the {\market} is derived in
        Section~\ref{\SETLABEL:MAXSHANNON}, but it is probably of
        doubtful practicality.

        Note that these optimizations would maximize a company's
        market growth. Since there are probably many companies
        competing in the market place, this would not necessarily
        maximize a company's P\&L, as described in
        Chapter~\ref{general}, Section~\ref{ompl}. The Shannon
        probability that maximizes market share in the {\market} is
        \pncompanies, with several alternative solutions listed in the
        previous paragraph. However, these should be contrasted to the
        Shannon probability that maximizes a company's P\&L which is
        \avgrms~in the {\market}. In all cases, the fraction of the
        P\&L that should be ``wagered'' on the future, $f$, should be:

        \begin{equation}
            f = 2P - 1
        \end{equation}

        \noindent where $P$ is the particular Shannon probability
        chosen optimize a particular fiscal strategy. Interestingly,
        the measured Shannon probability of the {\market} would tend
        to indicate that the companies participating in the market
        have chosen a fiscal strategy that optimizes market growth, as
        opposed to capital growth.

        \subidx{\market}{increasing returns}
        \subidx{economic increasing returns}{\market}
        As interesting interpretation of these exploitive issues,
        since all three fiscal strategies will result in exponential
        market growth for every company participating in the market,
        is that they may represent, perhaps, an example of
        ``increasing returns.''

% Local Variables:
% TeX-parse-self: t
% TeX-auto-save: t
% TeX-master: "fractal.tex"
% End:


        %
% -----------------------------------------------------------------------------
%
% A license is hereby granted to reproduce this software source code and
% to create executable versions from this source code for personal,
% non-commercial use.  The copyright notice included with the software
% must be maintained in all copies produced.
%
% THIS PROGRAM IS PROVIDED "AS IS". THE AUTHOR PROVIDES NO WARRANTIES
% WHATSOEVER, EXPRESSED OR IMPLIED, INCLUDING WARRANTIES OF
% MERCHANTABILITY, TITLE, OR FITNESS FOR ANY PARTICULAR PURPOSE.  THE
% AUTHOR DOES NOT WARRANT THAT USE OF THIS PROGRAM DOES NOT INFRINGE THE
% INTELLECTUAL PROPERTY RIGHTS OF ANY THIRD PARTY IN ANY COUNTRY.
%
% Copyright (c) 1994-2006, John Conover, All Rights Reserved.
%
% Comments and/or bug reports should be addressed to:
%
%     john@email.johncon.com (John Conover)
%
% -----------------------------------------------------------------------------
%
% Revision: \RCSRevision \\
% Revision Time: \RCSTime UMT \\
% Revision Date: \RCSDate \\
% Revision Id: \RCSId \\
% Revision File: \RCSLog \\
\RCS $Revision: 0.0 $
\RCS $Date: 2006/01/20 04:38:13 $
\RCS $Id: operations.tex,v 0.0 2006/01/20 04:38:13 john Exp $
% $Log: operations.tex,v $
% Revision 0.0  2006/01/20 04:38:13  john
% Initial version
%
%
    \subsection{Fixed Increment Approximation for Operational Strategy}
        \label{\SETLABEL:OPS}.

        This section derives various values based on the ``average''
        of the normalized increments presented in
        Figure~\ref{\SETLABEL:TFA}. These values are an approximation
        to a, probably, complex process with a distribution shown in
        Figure~\ref{\SETLABEL:TF}. These values will be used in a
        fixed increment Brownian fractal analysis and simulation of
        the {\market}, and may, or may not, provide adequate accuracy
        for projections.

        \subidx{\market}{fiscal strategy}
        \subidx{\market}{Shannon probability}
        \subidx{strategy}{fiscal}
        \subidx{fiscal}{strategy}
        \subidx{Shannon}{probability}
        \subidx{probability}{Shannon}
        It should be noted that the analysis of fiscal strategy,
        presented in Section~\ref{\SETLABEL:FS}, is derived from the
        {\market} metrics and may, or may not, be maximally
        optimal. For the optimal fiscal strategy, which may be
        exploitable, see Section~\ref{\SETLABEL:MAXSHANNON}.

        \subidx{strategy}{exploitable}
        \subidx{exploitable}{strategy}
        \subidx{\market}{windows of opportunity}
        \idx{windows of opportunity}
        \subidx{decision}{obsolete}
        \subidx{obsolete}{decision}
        \subidx{decision}{timeliness}
        \subidx{timeliness}{decision}
        \subidx{rate of revenue returns}{forecast}
        \subidx{forecast}{rate of revenue returns}
        An additional exploitable strategy may be time itself.
        Equations~\ref{\SETLABEL:V},~\ref{\SETLABEL:R},
        and,~\ref{\SETLABEL:MA}, are, essentially, metrics on how fast
        a decision, which is based on information concerning the
        current status of the {\market}, becomes obsolete. Obviously,
        how long a decision is expected to remain relevant should be
        addressed as an operational necessity in strategic planning
        and project management. Figures~\ref{\SETLABEL:FN},
        and,~\ref{\SETLABEL:FF} compare methods of approximation of
        the ``forecastability'' of rate of revenue returns in the
        {\market} for the near term and far
        term~\cite[pp. 83-84]{Peters:CAOITCM}, respectively. As a
        general rule, caution must be exercised when making decisions
        that will span a time interval larger than the time interval
        where the ``forecastability'' of rate of revenue returns drops
        below 50\%. Beyond this time interval, the chances increase
        that the competitive and market forces will alter the market
        environment in a possibly detrimental unanticipated
        fashion. Obviously, there is significant advantage in
        ``timeliness'' of development, manufacturing, and distribution
        of products and services that are consistent with this
        temporal agenda. Automation of these processes, if executed
        consistently with this agenda, should be considered a
        competitive advantage.

        \subidx{strategy}{exploitable}
        \subidx{exploitable}{strategy}
        \subidx{rate of revenue returns}{forecast}
        \subidx{forecast}{rate of revenue returns}
        \idx{product life cycle}
        \idx{life cycle, product}
        In some sense, this temporal agenda defines the ``average''
        product or service life cycle in the {\market}. When the
        ``forecastability'' of rate of revenue returns drops below
        50\%, there is an even chance that the rate of revenue returns
        for the product or service will change in a detrimental
        fashion. If it is assumed that a product or service life cycle
        consists of a ramp up, a maintenence interval, and a ramp
        down, then, if all three life cycle intervals are equal, the
        product life cycle will be, approximately, three times the
        time interval where the ``forecastability'' of rate of revenue
        returns drops below 50\%. Although probably not an accurate
        prediction of product or service life cycle, the technique may
        be used as a conceptual approximation to the dynamics of
        ``market windows.\footnote{For example, consider the market
        for table salt. Since it has inelastic supply and demand
        curves, and is a necessary requirement for life, it would be
        expected that the Hurst coefficient would be very near
        unity---ignoring competitive pressures in the market. The
        predictability of the table salt market would, therefore, be
        expected to be relatively good, over time.}''  The conceptual
        approximation will probably predict a ``conservative'' or
        ``pessimistic'' value in relation to actual markets.

        \begin{figure}[ht]
            \begin{center}
                \begin{minipage}[t]{0.45\textwidth}
                    \epsfxsize=1.0\linewidth
                    \epsffile{\directory/datahurstlownear.eps}
                    \caption[{\market}, ``forecastability'' of near
                        term rate of revenue returns]{{\market},
                        ``forecastability'' of near term rate of
                        revenue returns. Although the error function
                        is the most accurate, for the near term,
                        $H^{t} = \thurstlow^{t}$ may be used as a
                        reliable metric of ``forecastability'' of the
                        rate of revenue returns.}
                    \label{\SETLABEL:FN}
                \end{minipage}
                \hfill
                \begin{minipage}[t]{0.45\textwidth}
                    \epsfxsize=1.0\linewidth
                    \epsffile{\directory/datahurstlowfar.eps}
                    \caption[{\market}, ``forecastability'' of far
                        term rate of revenue returns]{{\market},
                        ``forecastability'' of far term rate of
                        revenue returns. Although the error function
                        is the most accurate, for the far term,
                        $\frac{1}{\sqrt{t}}$ may be used as a reliable
                        metric of ``forecastability'' of the rate of
                        revenue returns.}
                    \label{\SETLABEL:FF}
                \end{minipage}
            \end{center}
        \end{figure}

        \idx{operations research}
        As an interesting interpretation of the data presented in
        Figure~\ref{\SETLABEL:FN}, there may be, perhaps, some
        applicability to such operational agendas as inventory
        control. Maintaining too little inventory, obviously, will
        create a situation where the organization can not exploit
        market expansion, and maintaining too much inventory,
        likewise, would over extend the company, creating unnecessary
        losses when the market contracts. The company should maintain
        inventory levels that do not exceed, from
        Equation~\ref{\SETLABEL:MA}, ${\thurstlow}^{n} = 0.5$
        {\timescale}s of operations. Since the optimal amount of
        inventory and, from Equation~\ref{\SETLABEL:V}, the variance
        of change in the rate of revenue returns in the future can be
        calculated, there may, perhaps, be some applicability to a
        forecasting methodology that can be incorporated into other
        areas of operations research, for example the linear algebras
        using simplex methodologies for optimization of manufacturing
        processes. Traditionally, these forecasts are made by the
        sales department, and are subject to various subjective
        biases.

% Local Variables:
% TeX-parse-self: t
% TeX-auto-save: t
% TeX-master: "fractal.tex"
% End:


        \subsubsection{Observations on the Fixed Increment Approximation for Operational Strategy}

            As an interesting interpretation of
            Figure~\ref{\SETLABEL:FF}, and evaluating the
            approximation $\frac{1}{\sqrt{t}}$ at 60 months gives a
            probability that the market will still have the same
            agenda of about $0.12909945$, or about 1 in 8. This is
            commensurate with numbers from the venture
            community\footnote{For example, see ``IEEE Engineering
            Management Review,'' Volume 23 Number 3, Fall 1995,
            pp. 83}. Of course new venture backed companies fail for
            many reasons, but market appropriateness to product
            portfolio 60 months in the future may be a major
            contributor. Additionally, the success rate of development
            projects of 8 month duration, which have a market success
            rate of about 1 in 3, seems consistent with
            $\frac{1}{\sqrt{3}} = 0.353553391$. Naturally, projects
            fail in the market for many reasons, but market
            appropriateness, in a dynamic market environment may be a
            major contributor to failure.

            As mentioned in Section~\ref{\SETLABEL:H},
            Equation~\ref{\SETLABEL:MA}, and the preceeding section,
            approximately 3 times the value where ${\thurstlow}^{n} =
            0.5$ could be interpreted as an approximation to the
            ``average'' product life cycle. This seems consistent with
            the 6 to 12 month life cycles quoted by many industry
            analyst. In addition, maintaining inventory levels that do
            not exceed the anticipated requirements of
            $\frac{\ln{0.5}}{\ln{\thurstlow}}$ many {\timescale}s
            seems consistent with the author's experience in the
            industry.

        %
% -----------------------------------------------------------------------------
%
% A license is hereby granted to reproduce this software source code and
% to create executable versions from this source code for personal,
% non-commercial use.  The copyright notice included with the software
% must be maintained in all copies produced.
%
% THIS PROGRAM IS PROVIDED "AS IS". THE AUTHOR PROVIDES NO WARRANTIES
% WHATSOEVER, EXPRESSED OR IMPLIED, INCLUDING WARRANTIES OF
% MERCHANTABILITY, TITLE, OR FITNESS FOR ANY PARTICULAR PURPOSE.  THE
% AUTHOR DOES NOT WARRANT THAT USE OF THIS PROGRAM DOES NOT INFRINGE THE
% INTELLECTUAL PROPERTY RIGHTS OF ANY THIRD PARTY IN ANY COUNTRY.
%
% Copyright (c) 1994-2006, John Conover, All Rights Reserved.
%
% Comments and/or bug reports should be addressed to:
%
%     john@email.johncon.com (John Conover)
%
% -----------------------------------------------------------------------------
%
% Revision: \RCSRevision \\
% Revision Time: \RCSTime UMT \\
% Revision Date: \RCSDate \\
% Revision Id: \RCSId \\
% Revision File: \RCSLog \\
\RCS $Revision: 0.0 $
\RCS $Date: 2006/01/20 04:38:13 $
\RCS $Id: simulation.tex,v 0.0 2006/01/20 04:38:13 john Exp $
% $Log: simulation.tex,v $
% Revision 0.0  2006/01/20 04:38:13  john
% Initial version
%
%
    \subsection{Simulation of Fixed Increment Approximation for Fiscal Strategy}
        \label{\SETLABEL:TSUNFAIRBROWNIAN}

        \subidx{\market}{market simulation}
        The data in this section is presented in tabular form in
        Section~\ref{\SETLABELREF:SIM}.
        Figure~\ref{\SETLABEL:TSUNFAIRBROWNIAN0} represents a
        constructional simulation of the time series data presented in
        Figure~\ref{\SETLABEL:TS}. The program {\it
        tsunfairbrownian}\/, which is briefly described in
        appendix~\ref{programs}, was used in the reconstruction. The
        reconstructed data is superimposed on the original time series
        data.  The program, {\it tsunfairbrownian}\/, essentially,
        constructs the new time series as a Brownian fractal with
        fixed increments---the value of the fixed increment is derived
        from the root mean square average of the normalized increments
        presented in Figure~\ref{\SETLABEL:TF}. The ``quality'' of
        such a reconstruction should be subject to adequate scepticism
        and scrutiny since, in all probability, the normalized
        increments presented in Figure~\ref{\SETLABEL:TF} represent a
        relatively complex process, that may not be ``modeled'' with
        such a simple methodology.

        As a further comparison of the the constructional simulation
        with the original time series data,
        Figure~\ref{\SETLABEL:TSUNFAIRBROWNIAN1} presents a normalized
        histogram of the normalized increments of the reconstructed
        time series, superimposed on the normalized histogram
        presented in Figure~\ref{\SETLABEL:NH}.

        \subidx{\market}{fiscal strategy, simulation}
        \subidx{markets}{simulation}
        \subidx{simulation}{markets}
        \subidx{strategy}{fiscal, simulation}
        \subidx{fiscal}{strategy, simulation}
        \subidx{programs}{tsunfairbrownian}
        \subidx{tsunfairbrownian}{program}
        \begin{figure}[ht]
            \begin{center}
                \begin{minipage}[t]{0.45\textwidth}
                    \epsfxsize=1.0\linewidth
                    \epsffile{\directory/tsunfairbrownian-f.eps}
                    \caption[{\market}, Time series data, empirical and
                        simulated]{{\market}, Time series data, empirical
                        and simulated, using the program {\it tsunfairbrownian}\/
                        with f = {\datafractionrms}. This data is
                        superimposed on the data presented in
                        Figure~\ref{\SETLABEL:TS}.}
                    \label{\SETLABEL:TSUNFAIRBROWNIAN0}
                \end{minipage}
                \hfill
                \begin{minipage}[t]{0.45\textwidth}
                    \epsfxsize=1.0\linewidth
                    \epsffile{\directory/tsunfairbrownian-f.tsfraction.tsnormal-s30.eps}
                    \caption[{\market}, normalized histogram,
                        empirical and simulated]{{\market}, normalized
                        histogram of the normalized increments of the
                        time series data shown in
                        Figure~\ref{\SETLABEL:TSUNFAIRBROWNIAN0},
                        empirical and simulated.  The empirical data
                        has a mean of {\datafractionmean}, with a
                        standard deviation of {\datafractionstddev}.
                        By comparison, the simulated data has a mean
                        of {\tsunfairbrownianfractionmean} with a
                        standard deviation of
                        {\tsunfairbrownianfractionstddev}. This data
                        is superimposed on the data presented in
                        Figure~\ref{\SETLABEL:NH}. The area under the
                        four curves is identical.}
                    \label{\SETLABEL:TSUNFAIRBROWNIAN1}
                \end{minipage}
            \end{center}
        \end{figure}

% Local Variables:
% TeX-parse-self: t
% TeX-auto-save: t
% TeX-master: "fractal.tex"
% End:


        %
% -----------------------------------------------------------------------------
%
% A license is hereby granted to reproduce this software source code and
% to create executable versions from this source code for personal,
% non-commercial use.  The copyright notice included with the software
% must be maintained in all copies produced.
%
% THIS PROGRAM IS PROVIDED "AS IS". THE AUTHOR PROVIDES NO WARRANTIES
% WHATSOEVER, EXPRESSED OR IMPLIED, INCLUDING WARRANTIES OF
% MERCHANTABILITY, TITLE, OR FITNESS FOR ANY PARTICULAR PURPOSE.  THE
% AUTHOR DOES NOT WARRANT THAT USE OF THIS PROGRAM DOES NOT INFRINGE THE
% INTELLECTUAL PROPERTY RIGHTS OF ANY THIRD PARTY IN ANY COUNTRY.
%
% Copyright (c) 1994-2006, John Conover, All Rights Reserved.
%
% Comments and/or bug reports should be addressed to:
%
%     john@email.johncon.com (John Conover)
%
% -----------------------------------------------------------------------------
%
% Revision: \RCSRevision \\
% Revision Time: \RCSTime UMT \\
% Revision Date: \RCSDate \\
% Revision Id: \RCSId \\
% Revision File: \RCSLog \\
\RCS $Revision: 0.0 $
\RCS $Date: 2006/01/20 04:38:13 $
\RCS $Id: maximum.tex,v 0.0 2006/01/20 04:38:13 john Exp $
% $Log: maximum.tex,v $
% Revision 0.0  2006/01/20 04:38:13  john
% Initial version
%
%
    \subsection{Simulation of Fixed Increment Approximation for Optimally Maximal Fiscal Strategy}
        \label{\SETLABEL:MAXSHANNON}
        \subidx{\market}{fiscal strategy, simulation}
        \subidx{\market}{maximum Shannon probability}
        \subidx{markets}{simulation}
        \subidx{simulation}{markets}
        \subidx{strategy}{optimum fiscal, simulation}
        \subidx{fiscal}{optimum strategy, simulation}
        \subidx{programs}{tsunfairbrownian}
        \subidx{tsunfairbrownian}{program}
        \subidx{Shannon}{probability}
        \subidx{probability}{Shannon}

        \subidx{strategy}{exploitable}
        \subidx{exploitable}{strategy}
        \subidx{programs}{tsshannonmax}
        \subidx{tsshannonmax}{program}
        \subidx{programs}{tsunfairbrownian}
        \subidx{tsunfairbrownian}{program}
        \subidx{strategy}{fiscal}
        \subidx{fiscal}{strategy}
        The data in this section is presented in tabular form in
        Section~\ref{\SETLABELREF:MAXSHANNON}. One of the issues of
        analysis, as mentioned in Section~\ref{\SETLABEL:OPS}, is to
        determine the maximum Shannon probability for the time series
        presented in Figure~\ref{\SETLABEL:TS}. Potentially, this
        could be exploited with an aggressive fiscal
        strategy. Figure~\ref{\SETLABEL:SHANNONMAX0} is a graph of the
        output of the {\it tsshannonmax}\/ program, which is described
        briefly in appendix~\ref{programs}. The maximum of this
        function is the maximum Shannon probability for the time
        series data presented in Figure~\ref{\SETLABEL:TS}.
        Figure~\ref{\SETLABEL:SHANNONMAX1} was constructed using {\it
        tsunfairbrownian}\/ program, which is also described in
        appendix~\ref{programs}, with the maximum Shannon probability,
        and the time series data presented in
        Figure~\ref{\SETLABEL:TS}. This represents a ``what if'' the
        investment strategy was changed from a Shannon probability of
        {\shannonlogreturns}, as derived in Section~\ref{\SETLABEL:SP}
        to {\shannonmax}. This process, essentially, extracts the
        random statistical data from the time series presented in
        Figure~\ref{\SETLABEL:TS}, and constructs a new time series,
        using the random statistical data, with a different investment
        strategy.  The program, {\it tsunfairbrownian}\/, essentially,
        constructs the new time series as a Brownian fractal with
        fixed increments.  The ``quality'' of such a reconstruction
        should be subject to adequate scepticism and scrutiny since,
        in all probability, the increments in the original data
        represent a relatively complex process, that may not be
        ``modeled'' with such a simple methodology.

        \begin{figure}[ht]
            \begin{center}
                \begin{minipage}[t]{0.45\textwidth}
                    \epsfxsize=1.0\linewidth
                    \epsffile{\directory/data.tsshannonmax.eps}
                    \caption[{\market}, maximum rate of revenue
                        returns] {{\market}, maximum rate of revenue
                        returns, per {\timescale}, vs. Shannon
                        probability. The maximum rate of revenue
                        returns, per {\timescale}, occurs at a Shannon
                        probability of {\shannonmax}.}
                    \label{\SETLABEL:SHANNONMAX0}
                \end{minipage}
                \hfill
                \begin{minipage}[t]{0.45\textwidth}
                    \epsfxsize=1.0\linewidth
                    \epsffile{\directory/data.tsshannonmax-p.tsunfairbrownian-p.eps}
                    \caption[{\market}, maximum rate of revenue
                        returns] {{\market}, maximum rate of revenue
                        returns, per {\timescale}, at a Shannon
                        probability, of {\shannonmax}, corresponding
                        to a ``wager'' fraction of {\twoponemax}.}
                    \label{\SETLABEL:SHANNONMAX1}
                \end{minipage}
            \end{center}
        \end{figure}

        \subidx{fractional}{Brownian motion}
        \subidx{Brownian motion}{fractional}
        \subidx{Shannon}{probability}
        \subidx{probability}{Shannon}
        \subidx{programs}{tsshannonmax}
        \subidx{tsshannonmax}{program}
        If it is assumed that the time series data set, presented in
        Figure~\ref{\SETLABEL:TS}, constitutes classical Brownian
        motion, then the Shannon probability can be calculated by
        counting the total number of {\timescale}s that the {\market}
        movement was positive, and dividing by the total number of
        {timescale}s represented in the time series. This quotient is
        {\pmax}, as compared with the predicted value from the program
        {\it tsshannonmax}\/ of {\shannonmax}.

% Local Variables:
% TeX-parse-self: t
% TeX-auto-save: t
% TeX-master: "fractal.tex"
% End:


        \subsubsection{Observations on the Simulation of Fixed Increment Approximation for Optimally Maximal Fiscal Strategy}

            Note that these simulations are base on a very, perhaps
            overly, simplified model. For example, from
            Section~\ref{\SETLABEL:TSA}, Figure~\ref{\SETLABEL:NH}, it
            would appear that the {\market}'s normalized increments
            are characterized by fractional Brownian motion---but the
            simulations used classical Brownian motion as the
            model. One consequence of this is that a re-investment
            strategy that is to ``wager'' a fraction of {\twoponemax}
            of the rate of returns every {\timescale} is overly
            aggressive, since in the classical Brownian scenario, the
            maximum loss, in any {\timescale}, was no more that what
            was ``wagered.'' However, in the fractional Brownian
            scenario, much more can be lost. From
            Equation~\ref{fopt2},

            \begin{equation}
                \frac{avg}{rms^2} = \frac{f_{opt}}{rms} = K
            \end{equation}

            \noindent where, under the optimum classical Brownian
            scenario, $K$ is unity, or $avg = rms^2$. Notice that,
            since $f = rms$, whether the scenario is optimal or not,
            that the operational ``wager'' fraction, from
            Figure~\ref{\SETLABEL:TF} of {\datafractionrms}, vs.\ an
            ``theoretical optimal'' value of {\twoponemax} seems
            overly conservative. Additionally, notice that, at least
            in principle, the chance of failure in the fractional
            Brownian scenario, which is more accurate, would
            correspond to 1 standard deviation, or about 15.865\% per
            {\timescale}, which is unacceptably high. However, it is
            not clear why the {\market} is running at a value of
            {\datafractionrms}, which seems very
            conservative. However, a re-investment strategy of
            {\datafractionrms} per {\timescale} does not seem
            inconsistent with a failure rate, on the Fortune 500 list,
            which it is inferred that the {\market} is similar to, of
            about 50\% in ten years, which corresponds to $(1 -
            p_f)^{120} \approx 0.5$, or $p_f$, the probability of
            failure, is $0.005759576$, which is, approximately, 2.5
            standard deviations, meaning that to be consistent with
            the large companies in the Fortune 500, the re-investment
            rate should be, approximately, $\frac{\twoponemax}{2.5}$,
            compared with an operational value, from
            Figure~\ref{\SETLABEL:NH} of {\datafractionrms}.

            An interesting, and intriguing, interpretation and
            discussion of the maximum Shannon probability, is an
            explanation as to why the companies in the {\market} are
            not running an optimal re-investment strategy. This seems
            enigmatic, since those companies that run, on a long term
            average, below the optimally maximal value would seem to
            be eclipsed by those that didn't. And those that run above
            the optimally maximal value would be over extended, and
            become financially destitute during market down turns,
            which is inevitable in a fractal time series as presented
            in Figure~\ref{\SETLABEL:TS}.  It would seem that the
            natural selection process of the competitive environment
            would allow only those companies that run near the
            optimally maximal value to survive, in the long run. One
            possible explanation, foremost, is that the analytical
            methodology presented herein is inappropriate.  Another
            explanation is that the gross margins are less than the
            fraction {\shannonmax} of the rate of revenue returns, and
            thus could not accommodate such an aggressive
            re-investment strategy. If this is the case, then it
            presents an intriguing issue. If, in a capitalistic
            market, the natural outcome of the competitive situation,
            according to game-theoretic analysis, is that there will
            be many competitors, each making minimal gross margins,
            then how do the companies grow their markets?  Naturally,
            those that run the most efficient will have lower costs,
            making larger percentage of rate of revenue returns
            re-investment possible. Yet another interpretation is that
            the number of competitors would grow at an exponential
            rate, but all of them would make minimal returns. However,
            an operational Shannon probability of {\shannonlogreturns}
            is not just marginally lower than the maximum Shannon
            probability of {\shannonmax}. There is a significant
            disparity which is difficult to explain. It would seem
            that the game-theoretic eventual outcome of a competitive
            market place would be a solution that hinders growth,
            wealth and jobs creation, etc., which does not seem
            consistent with capitalistic theory. On the other hand, is
            there an optimum number of competitors in a market place,
            where the gross margins can be higher, permitting wealth
            and job creation, and also a competitive situation? If
            this analysis is correct, and that should be subject to
            scrutiny, then it would appear that this is the case. But
            this brings up another issue---that of taxation, and other
            contributions to the social welfare function. If there is
            an optimum number of competitors in the market place, that
            maximizes wealth and job creation, then, perhaps by lemma,
            there is also an optimal value of taxation rate, and other
            contributions to the social welfare function, that will
            permit maximal industrial growth, and thus maximal growth
            in the tax base. But this would seem to be inconsistent
            with the work of Kenneth Arrow and the so called
            Impossibility Theorem, which states that such
            optimizations can not be determined because the ordering
            of priorities are intransitive.  All very perplexing,
            since the simulation of the maximum Shannon probability in
            the next section seems to indicate that such an aggressive
            re-investment strategy is, indeed, feasible.

            Yet another possibility for the industry not running at
            maximum Shannon probability is the high cost of expansion
            of operations. Some of these industries require very
            sophisticated manufacturing processes, which have high
            barrier costs.

            Additionally, as mentioned in both~\cite[pp. 29]{Brock},
            and~\cite[pp. 8]{Arthur:CTIRALIBHE}, optimal efficiency
            may not be attainable in increasing-return economic
            scenarios.

        %
% -----------------------------------------------------------------------------
%
% A license is hereby granted to reproduce this software source code and
% to create executable versions from this source code for personal,
% non-commercial use.  The copyright notice included with the software
% must be maintained in all copies produced.
%
% THIS PROGRAM IS PROVIDED "AS IS". THE AUTHOR PROVIDES NO WARRANTIES
% WHATSOEVER, EXPRESSED OR IMPLIED, INCLUDING WARRANTIES OF
% MERCHANTABILITY, TITLE, OR FITNESS FOR ANY PARTICULAR PURPOSE.  THE
% AUTHOR DOES NOT WARRANT THAT USE OF THIS PROGRAM DOES NOT INFRINGE THE
% INTELLECTUAL PROPERTY RIGHTS OF ANY THIRD PARTY IN ANY COUNTRY.
%
% Copyright (c) 1994-2006, John Conover, All Rights Reserved.
%
% Comments and/or bug reports should be addressed to:
%
%     john@email.johncon.com (John Conover)
%
% -----------------------------------------------------------------------------
%
% Revision: \RCSRevision \\
% Revision Time: \RCSTime UMT \\
% Revision Date: \RCSDate \\
% Revision Id: \RCSId \\
% Revision File: \RCSLog \\
\RCS $Revision: 0.0 $
\RCS $Date: 2006/01/20 04:38:13 $
\RCS $Id: verification.tex,v 0.0 2006/01/20 04:38:13 john Exp $
% $Log: verification.tex,v $
% Revision 0.0  2006/01/20 04:38:13  john
% Initial version
%
%
    \subsection{Qualitative Verification of Fixed Increment Approximation Analysis}
        \label{\SETLABEL:QVA}

        \subidx{\market}{verification of analysis}
        \subidx{verification}{analysis}
        \subidx{analysis}{verification}
        \subidx{quality}{of analysis}
        \subidx{verification}{of methodology}
        \subidx{methodology}{verification of}
        \subidx{Shannon}{probability}
        \subidx{probability}{Shannon}

        This section evaluates various values based on the ``average''
        of the normalized increments presented in
        Figure~\ref{\SETLABEL:TFA}. These values are an approximation
        to a, probably, complex process with a distribution shown in
        Figure~\ref{\SETLABEL:TF}. These values will be used in a
        fixed increment Brownian fractal analysis of the {\market},
        and may, or may not, provide adequate accuracy for
        projections.

        The data in this section is presented in tabular form in
        sections~\ref{\SETLABELREF:VI1} and~\ref{\SETLABELREF:VI2}.
        As a subjective evaluation of the ``quality'' of the analysis
        of the {\market}, from Chapter~\ref{methodology},
        Equation~\ref{metricvalues1}, and using the mean and root mean
        square values of the normalized increments of the time series
        data presented in Figure~\ref{\SETLABEL:TS} from
        Figure~\ref{\SETLABEL:TF}, and the Shannon probability as
        calculated by counting the total number of {\timescale}s that
        the {\market} movement was positive, as presented in
        Section~\ref{\SETLABEL:MAXSHANNON}:

        \begin{eqnarray}
                  P & \approx & \frac{\frac{avg}{rms} + 1}{2}\\
            {\pmax} & \approx & \frac{\frac{\datafractionmean}{\datafractionrms} + 1}{2}\\
            {\pmax} & \approx & {\avgrms}
            \label{\SETLABEL:AVGS}
        \end{eqnarray}

        \subidx{Shannon}{probability}
        \subidx{probability}{Shannon}
        \noindent and comparing these values to the Shannon
        probability, as found by the {\it tsshannonmax}\/ program, which
        iterates for a maximum:

        \begin{eqnarray}
            {\pmax} \approx {\avgrms} \approx {\shannonmax}
        \end{eqnarray}

        \subidx{logarithmic}{returns}
        \subidx{returns}{logarithmic}
        In addition, the different methods of calculating the
        logarithmic returns, presented in Section~\ref{\SETLABEL:FS},
        should be compared. The four methods used were the mean of
        Figure~\ref{\SETLABEL:TF}, the constant in the least squares
        approximation to Figure~\ref{\SETLABEL:TF}, the least squares
        exponential approximation to Figure~\ref{\SETLABEL:TS}, and
        the logarithmic returns of Figure~\ref{\SETLABEL:TS}, derived
        as the mean of the logarithms of the quotients of the
        increments. The values for each of the methods are,
        respectively:

        \begin{equation}
            \datafractionmeanbits \approx \datafractionconstantbits \approx \datatslsqepbits \approx \logreturns
        \end{equation}

        It is implied in Section~\ref{\SETLABEL:FS},
        Subsection~\ref{\SETLABEL:SP} and in
        Section~\ref{\SETLABEL:TSUNFAIRBROWNIAN} that, a Brownian
        motion with fixed increments fractal may ``model'' the
        {\market}. Using Equation~\ref{stddev9} from
        Chapter~\ref{general}, Section~\ref{abmfi}:

        \begin{eqnarray}
                                    rms \left(2P - 1\right) & \approx & \frac{\sigma \left(2P - 1\right)}{2 \sqrt{P\left(1 - P\right)}}\\
            \datafractionrms \left(2 \cdot \pmax - 1\right) & \approx & \frac{\datafractionstddev \left(2 \cdot \pmax - 1\right)}{2\sqrt{\pmax \left(1 - \pmax\right)}}\\
                       \datafractionrms \cdot \twopminusone & \approx & \datafractionstddev \cdot \twopx\\
                                                      \rmsp & \approx & \sigmap
        \end{eqnarray}

        \noindent and, equating to the mean:

        \begin{equation}
            \datafractionmean \approx \rmsp \approx \sigmap
        \end{equation}

        \subidx{Shannon}{probability}
        \subidx{probability}{Shannon}
        \noindent where, as in Equation~\ref{\SETLABEL:AVGS} using the
        mean, root mean square, and standard deviation values of the
        normalized increments of the time series data presented in
        Figure~\ref{\SETLABEL:TS} from Figure~\ref{\SETLABEL:TF}, and
        the Shannon probability as calculated by counting the total
        number of {\timescale}s that the {\market} movement was
        positive, as presented in Section~\ref{\SETLABEL:MAXSHANNON}.

        As a final qualitative comparison, the absolute value of the
        normalized increments should be the same as the root mean
        square value\footnote{The absolute value of the normalized
        increments, when averaged, is related to the root mean square
        of the increments by a constant. If the normalized increments
        are a fixed increment, the constant is unity. If the
        normalized increments have a Gaussian distribution, the
        constant is $\approx 0.8$ depending on the accuracy of of
        ``fit'' to a Gaussian distribution.}, where the absolute value
        is presented in Figure~\ref{\SETLABEL:TFA}, and the root mean
        square value is presented in Figure~\ref{\SETLABEL:TF}:

        \begin{equation}
            \datafractionabsmean \approx \datafractionrms
        \end{equation}

        Note, that if the {\market} could be ``modeled'' as a Brownian
        motion with fixed increments fractal, then the standard
        deviation of the absolute value of the normalized increments
        of the time series data presented in Figure~\ref{\SETLABEL:TS}
        from Figure~\ref{\SETLABEL:TF} should be zero. It is
        $\datafractionabsstddev$.

% Local Variables:
% TeX-parse-self: t
% TeX-auto-save: t
% TeX-master: "fractal.tex"
% End:


    \renewcommand{\market}{Dow Jones Average}
    \renewcommand{\directory}{../markets/dj}
    \renewcommand{\datafractionmean}{0.008052}
\renewcommand{\datafractionmeanbits}{0.011570}
\renewcommand{\datafractionmeanq}{0.002684}
\renewcommand{\datafractionmeanbitsq}{0.003867}
\renewcommand{\datafractionstddev}{0.038579}
\renewcommand{\datafractionrms}{0.039311}
\renewcommand{\avgrms}{0.602414}
\renewcommand{\ncompanies}{5.210454}
\renewcommand{\pncompanies}{0.544866}
\renewcommand{\datafractionabsmean}{0.029745}
\renewcommand{\datafractionabsstddev}{0.025769}
\renewcommand{\datafractionconstant}{0.010041}
\renewcommand{\datafractionconstantbits}{0.014414}
\renewcommand{\datafractionconstantq}{0.003347}
\renewcommand{\datafractionconstantbitsq}{0.004821}
\renewcommand{\datafractionslope}{-0.000021}
\renewcommand{\datafractionabsconstant}{0.035145}
\renewcommand{\datafractionabsslope}{-0.000057}
\renewcommand{\hurstall}{0.659558}
\renewcommand{\hurstlow}{0.707509}
\renewcommand{\hurstlowtwo}{1.415018}
\renewcommand{\hurstlowhundred}{70.750900}
\renewcommand{\hcalcall}{0.184942}
\renewcommand{\hcalclow}{0.102042}
\renewcommand{\shannonmax}{0.604167}
\renewcommand{\twoponemax}{0.208334}
\renewcommand{\logreturns}{0.010456}
\renewcommand{\twologreturns}{1.007274}
\renewcommand{\twologreturnshundred}{0.727387}
\renewcommand{\oneoverlogreturns}{95.638868}
\renewcommand{\pmax}{0.602094}
\renewcommand{\twopminusone}{0.204188}
\renewcommand{\rmsp}{0.008027}
\renewcommand{\twopx}{0.208583}
\renewcommand{\sigmap}{0.008047}
\renewcommand{\tsunfairbrownianfractionmean}{0.007862}
\renewcommand{\tsunfairbrownianfractionstddev}{0.038619}
\renewcommand{\shannonlogreturns}{0.560125}
\renewcommand{\shannonlogreturnshundred}{56.012500}
\renewcommand{\twopone}{0.120250}
\renewcommand{\twoponehundred}{12.025000}
\renewcommand{\hundredtwoponehundred}{87.975000}
\renewcommand{\hundredshannonlogreturnshundred}{43.987500}
\renewcommand{\datatslsqepbits}{0.007623}
\renewcommand{\thurstall}{0.633980}
\renewcommand{\thurstlow}{0.710108}
\renewcommand{\thurstlowtwo}{1.420216}
\renewcommand{\thurstlowhundred}{71.010800}
\renewcommand{\thcalcall}{0.247886}
\renewcommand{\thcalclow}{0.171737}
\renewcommand{\chisquared}{2.862000}
\renewcommand{\critical}{42.557000}

    \renewcommand{\timescale}{month}
    \subidx{market}{\market}
    \idx{\market}

    \section{\market}

        \renewcommand{\SETLABEL}{\LABPRE:DJA}
        \renewcommand{\SETLABELQ}{\LABPRE:DJAQ}
        \label{\SETLABEL}
        \renewcommand{\SETLABELREF}{\LABPREREF:DJA}

        \idx{Dow Jones News Information Retrieval Service}
        For the analysis, the data was in the directory
        {\directory}\footnote{Data from Dow Jones News Information
        Retrieval Service, 1981---1994, by {\timescale}s, as an
        index.}.

        The data in this section is presented in tabular form in
        Section~\ref{\SETLABELREF}. Note that in this analysis, the
        rate of revenue returns means the increase or decrease in the
        value, or price, of the stocks in the {\market}, and not stock
        yield, or dividends. This is included for comparative
        purposes.

        %
% -----------------------------------------------------------------------------
%
% A license is hereby granted to reproduce this software source code and
% to create executable versions from this source code for personal,
% non-commercial use.  The copyright notice included with the software
% must be maintained in all copies produced.
%
% THIS PROGRAM IS PROVIDED "AS IS". THE AUTHOR PROVIDES NO WARRANTIES
% WHATSOEVER, EXPRESSED OR IMPLIED, INCLUDING WARRANTIES OF
% MERCHANTABILITY, TITLE, OR FITNESS FOR ANY PARTICULAR PURPOSE.  THE
% AUTHOR DOES NOT WARRANT THAT USE OF THIS PROGRAM DOES NOT INFRINGE THE
% INTELLECTUAL PROPERTY RIGHTS OF ANY THIRD PARTY IN ANY COUNTRY.
%
% Copyright (c) 1994-2006, John Conover, All Rights Reserved.
%
% Comments and/or bug reports should be addressed to:
%
%     john@email.johncon.com (John Conover)
%
% -----------------------------------------------------------------------------
%
% Revision: \RCSRevision \\
% Revision Time: \RCSTime UMT \\
% Revision Date: \RCSDate \\
% Revision Id: \RCSId \\
% Revision File: \RCSLog \\
\RCS $Revision: 0.0 $
\RCS $Date: 2006/01/20 04:38:13 $
\RCS $Id: fraction.tex,v 0.0 2006/01/20 04:38:13 john Exp $
% $Log: fraction.tex,v $
% Revision 0.0  2006/01/20 04:38:13  john
% Initial version
%
%
    \subsection{Time Series Increments Analysis}
        \label{\SETLABEL:TSA}

        \subidx{\market}{Time series analysis}
        \subidx{time series}{increments}
        \subidx{time series}{analysis}
        \subidx{cumulative sum}{analysis}
        \subidx{analysis}{cumulative sum}
        \subidx{analysis}{random process}
        \subidx{random process}{analysis}
        \subidx{Gaussian}{increments}
        \subidx{increments}{Gaussian}
        \subidx{Brownian}{motion, fractional}
        \subidx{fractional}{Brownian motion}
        \subidx{fractal}{Brownian motion}
        The data in this section is presented in tabular form in
        Section~\ref{\SETLABELREF:TSA}.  Figure~\ref{\SETLABEL:TS} is
        a graph of the time series data for the {\market}.

        \subidx{increments}{normalized}
        \subidx{normalized}{increments}
        \subidx{programs}{tsfraction}
        \subidx{tsfraction}{program}
        Figure~\ref{\SETLABEL:TF} is a graph of the normalized
        increments of the time series data presented in
        Figure~\ref{\SETLABEL:TS}. The data presented was made by
        running the program {\it tsfraction}\/ on the time series
        data. The program {\it tsfraction}\/ is described briefly in
        Appendix~\ref{programs}, and subtracts the previous value from
        the next value, dividing this difference by the previous
        value, for each element in the time series data. The new time
        series contains the instantaneous change in the rate of
        revenue returns, divided by the magnitude of the instantaneous
        rate of revenue returns.

        \subidx{mean}{standard deviation}
        \subidx{standard deviation}{mean}
        \idx{root mean square}
        \idx{least squares approximation}
        \begin{figure}[ht]
            \begin{center}
                \begin{minipage}[t]{0.45\textwidth}
                    \epsfxsize=1.0\linewidth
                    \epsffile{\directory/data.eps}
                    \caption{{\market}, time series data.}
                    \label{\SETLABEL:TS}
                    \label{\SETLABELQ:TS}
                \end{minipage}
                \hfill
                \begin{minipage}[t]{0.45\textwidth}
                    \epsfxsize=1.0\linewidth
                    \epsffile{\directory/data.tsfraction.eps}
                    \caption[{\market}, normalized
                        increments]{{\market}, normalized increments
                        of the time series data presented in
                        Figure~\ref{\SETLABEL:TS}. The mean is
                        {\datafractionmean} with a standard deviation
                        of {\datafractionstddev}. The formula for the
                        least squares approximation is
                        ${\datafractionconstant} +
                        {\datafractionslope}t$, and the root mean
                        squared value is {\datafractionrms}. The
                        graph, labeled ``data\-.tsfraction\-.tsrms,''
                        is the running root mean square, and
                        ``data\-.tsfraction\-.tsavg'' is the running
                        average of the normalized increments.  This
                        graph is the fraction of change in the time
                        series, as a function of time. Note that the
                        slope of the mean, {\datafractionslope}, is
                        the coefficient of the nonlinearity term in
                        the normalized increments. See
                        Chapter~\ref{general}, Section~\ref{nlextend}
                        for a possible application of the logistic
                        function to this data set.}
                    \label{\SETLABEL:TF}
                    \label{\SETLABELQ:TF}
                \end{minipage}
            \end{center}
        \end{figure}

        \subidx{absolute value}{increments}
        \subidx{increments}{absolute value}

        Figure~\ref{\SETLABEL:TFA} is a graph of the absolute value of
        the normalized increments of the time series data presented in
        Figure~\ref{\SETLABEL:TF}. The data presented was made by
        running the Unix utility sed(1) on the normalized increments
        time series data to remove the negative signs. This is an
        absolute value procedure.  The resulting time series contains
        the absolute value of the instantaneous change in the rate of
        revenue returns, divided by the magnitude of the instantaneous
        rate of revenue returns\footnote{The absolute value of the
        normalized increments, when averaged, is related to the root
        mean square of the increments by a constant. If the normalized
        increments are a fixed increment, the constant is unity. If
        the normalized increments have a Gaussian distribution, the
        constant is $\approx 0.8$ depending on the accuracy of of
        ``fit'' to a Gaussian distribution.}.

        \subidx{histogram}{normalized}
        \subidx{normalized}{histogram}
        \subidx{programs}{tsnormal}
        \subidx{tsnormal}{program}
        \subidx{mean}{standard deviation}
        \subidx{standard deviation}{mean}
        \idx{root mean square}
        \idx{least squares approximation}
        \subidx{\market}{analysis of increments}
        Figure~\ref{\SETLABEL:NH} is the normalized histogram of the
        normalized increments of the time series data shown in
        Figure~\ref{\SETLABEL:TF}. The abscissa is 3 $\sigma$ limits,
        and the area under the two curves is identical. The data for
        this figure was produced by the program {\it tsnormal}\/,
        which is described briefly in Appendix~\ref{programs}.

        \begin{figure}[ht]
            \begin{center}
                \begin{minipage}[t]{0.45\textwidth}
                    \epsfxsize=1.0\linewidth
                    \epsffile{\directory/data.tsfraction.abs.eps}
                    \caption[{\market}, absolute value of the
                        normalized increments]{{\market}, absolute
                        value of the normalized increments of the time
                        series data presented in
                        Figure~\ref{\SETLABEL:TF}.  The mean is
                        {\datafractionabsmean} with a standard
                        deviation of {\datafractionabsstddev}. The
                        formula for the least squares approximation is
                        ${\datafractionabsconstant} +
                        {\datafractionabsslope}t$, and the root mean
                        square value, from Figure~\ref{\SETLABEL:TF},
                        is {\datafractionrms}.  The graph, labeled
                        ``data\-.tsfraction\-.tsrms,'' is the running
                        root mean square, and
                        ``data\-.tsfraction\-.tsavg'' is the running
                        average of the normalized increments presented
                        in Figure~\ref{\SETLABEL:TF}, superimposed
                        here for convenience. This graph is the
                        absolute value of the fraction of change in
                        the time series, as a function of time.}
                    \label{\SETLABEL:TFA}
                    \label{\SETLABELQ:TFA}
                \end{minipage}
                \hfill
                \begin{minipage}[t]{0.45\textwidth}
                    \epsfxsize=1.0\linewidth
                    \epsffile{\directory/data.tsfraction.tsnormal-s30.eps}
                    \caption[{\market}, normalized histogram of the
                        normalized increments]{{\market}, normalized
                        histogram of the normalized increments of the
                        time series data shown in
                        Figure~\ref{\SETLABEL:TF}.  The data has a
                        mean of {\datafractionmean}, with a standard
                        deviation of {\datafractionstddev}.  The area
                        under the two curves is identical. The
                        $\chi^2$ value of the observed and expected
                        values of the two curves is {\chisquared},
                        with a critical value of {\critical}.}
                    \label{\SETLABEL:NH}
                \end{minipage}
            \end{center}
        \end{figure}

        \subidx{programs}{tsXsquared}
        \subidx{tsXsquared}{program}
        \subidx{\market}{chi-squared values of increments}
        The program {\it tsXsquared}\/, which is briefly described in
        appendix~\ref{programs}, was used to derive the $\chi^2$
        statistics for the data presented in
        Figure~\ref{\SETLABEL:NH}.

        \subidx{programs}{tsstatest}
        \subidx{tsstatest}{program}
        \subidx{\market}{statistical estimates}

        Figure~\ref{\SETLABEL:SE} is the statistical estimate for the
        data presented in Figure~\ref{\SETLABEL:TF}, as derived by the
        program {\it tsstatest}\/, which is briefly described in
        appendix~\ref{programs}.

        \begin{figure}[ht]
            \begin{center}
                \begin{minipage}[t]{\textwidth}
                    \center{\fbox{\parbox{0.9\textwidth}{\XXX{\directory/data.tsstatest-f0.1-c0.9-i.tex}}}}
                    \caption[{\market}, statistical estimates of the
                        normalized increments]{{\market}, statistical
                        estimates of the normalized increments of the
                        time series shown in Figure~\ref{\SETLABEL:TF}.
                        The table was produced with the {\it
                        tsstatest}\/ program, and illustrates the
                        size of the data set required for a confidence
                        level of 90\%, with an error estimate of $\pm$
                        10\%, or alternately, the error estimate on
                        the time series shown in Figure~\ref{\SETLABEL:TF}.}
                    \label{\SETLABEL:SE}
                \end{minipage}
            \end{center}
        \end{figure}

        Note that the data set size estimations, as produced by the
        {\it tsstatest}\/ program, are probably very conservative,
        depending on the magnitude of the Shannon probability, $P =
        \shannonlogreturns$, as derived in
        Section~\ref{\SETLABEL:SP}. See Chapter~\ref{general},
        Section~\ref{serdss} for possible alternative methodologies
        for addressing the analysis of fractal time series with
        limited data set sizes. Depending on the magnitude of the
        Shannon probability, $P$, these estimates can be several
        orders of magnitude too high.

        \subidx{derivative of increments}{normalized}
        \subidx{normalized}{derivative of increments}
        \subidx{programs}{tsderivative}
        \subidx{tsderivative}{program}
        Figure~\ref{\SETLABEL:TF1} is the normalized histogram of the
        first derivative of the normalized increments of the time
        series data shown in Figure~\ref{\SETLABEL:TF}. In principle,
        if the distribution of the normalized increments presented in
        Figure~\ref{\SETLABEL:NH} is Gaussian in nature, this
        distribution would be similar to ``white noise,'' as presented
        in appendix~\ref{programs}, Figure~\ref{whiteexample}. The
        data was generated by the {\it tsderivative}\/ program, which
        is briefly described in
        appendix~\ref{programs}. Figure~\ref{\SETLABEL:TF2} is the
        normalized histogram of the second derivative of the
        normalized increments of the time series data shown in
        Figure~\ref{\SETLABEL:TF}. In principle, if the distribution
        of the normalized increments presented in
        Figure~\ref{\SETLABEL:NH} is an integrated Gaussian
        distribution in nature, this distribution would be similar to
        ``white noise,'' as presented in appendix~\ref{programs},
        Figure~\ref{whiteexample}.

        \begin{figure}[ht]
            \begin{center}
                \begin{minipage}[t]{0.45\textwidth}
                    \epsfxsize=1.0\linewidth
                    \epsffile{\directory/data.tsfraction.tsderivative.tsnormal-s30.eps}
                    \caption[{\market}, histogram of the first
                        derivative of the increments]{{\market},
                        normalized histogram of the first derivative
                        of the normalized increments of the time
                        series data shown in
                        Figure~\ref{\SETLABEL:TF}.}
                    \label{\SETLABEL:TF1}
                \end{minipage}
                \hfill
                \begin{minipage}[t]{0.45\textwidth}
                    \epsfxsize=1.0\linewidth
                    \epsffile{\directory/data.tsfraction.2tsderivative.tsnormal-s30.eps}
                    \caption[{\market}, histogram of the second
                        derivative of the increments]{{\market},
                        normalized histogram of second derivative of
                        the the normalized increments of the time
                        series data shown in
                        Figure~\ref{\SETLABEL:TF}.}
                    \label{\SETLABEL:TF2}
                \end{minipage}
            \end{center}
        \end{figure}

        \subidx{fractal}{range}
        \subidx{fractal}{R/S analysis}
        \subidx{\market}{rate of revenue returns, range}
        \subidx{\market}{deterministic mechanism}
        \subidx{deterministic}{mechanism}
        \subidx{mechanism}{deterministic}
        Figure~\ref{\SETLABEL:TR} is the range of values of the time
        series shown in Figure~\ref{\SETLABEL:TS}. The horizontal axis
        is time into the future. In principle, if the time series was
        characterized as fractional Brownian motion the graph in
        Figure~\ref{\SETLABEL:TR} would be a square root
        function\footnote{Note that the ``roughness,'' or ``sawtooth''
        characteristics of the graph in Figure~\ref{\SETLABEL:TR} are
        a computational artifact---caused by not using the -m option
        to the program {\it tshurst}\/, which is computationally
        inefficient.}. Figure~\ref{\SETLABEL:TD} is the deterministic
        map of the normalized increments of the time series data shown
        in Figure~\ref{\SETLABEL:TF}. The deterministic map is useful
        for determining if a time series was created by a
        deterministic mechanism. This, essentially, maps each element
        in the time series with the previous element in the time
        series.  See,~\cite[pp. 745]{Peitgen}.

        \begin{figure}[ht]
            \begin{center}
                \begin{minipage}[t]{0.45\textwidth}
                    \epsfxsize=1.0\linewidth
                    \epsffile{\directory/data.tshurst-f.eps}
                    \caption[{\market}, range]{{\market}, range of the
                        time series data shown in
                        Figure~\ref{\SETLABEL:TS}.}
                    \label{\SETLABEL:TR}
                \end{minipage}
                \hfill
                \begin{minipage}[t]{0.45\textwidth}
                    \epsfxsize=1.0\linewidth
                    \epsffile{\directory/data.tsfraction.tsdeterministic.eps}
                    \caption[{\market}, deterministic map]{{\market},
                        deterministic map of the normalized increments
                        of the time series data shown in
                        Figure~\ref{\SETLABEL:TF}.}
                    \label{\SETLABEL:TD}
                \end{minipage}
            \end{center}
        \end{figure}

% Local Variables:
% TeX-parse-self: t
% TeX-auto-save: t
% TeX-master: "fractal.tex"
% End:


        \subsubsection{Observations on the Time Series Increments Analysis}

            Figure~\ref{\SETLABEL:NH} would seem to indicate that the
            time series data for the {\market} represents a cumulative
            sum/integration of a random process that has a Gaussian
            distribution, (ie., satisfies the Gaussian increments
            property of fractional Brownian
            motion~\cite[pp. 250]{Crownover},) tending to justify the
            assumption that the time series data represents fractional
            Brownian motion.

        %
% -----------------------------------------------------------------------------
%
% A license is hereby granted to reproduce this software source code and
% to create executable versions from this source code for personal,
% non-commercial use.  The copyright notice included with the software
% must be maintained in all copies produced.
%
% THIS PROGRAM IS PROVIDED "AS IS". THE AUTHOR PROVIDES NO WARRANTIES
% WHATSOEVER, EXPRESSED OR IMPLIED, INCLUDING WARRANTIES OF
% MERCHANTABILITY, TITLE, OR FITNESS FOR ANY PARTICULAR PURPOSE.  THE
% AUTHOR DOES NOT WARRANT THAT USE OF THIS PROGRAM DOES NOT INFRINGE THE
% INTELLECTUAL PROPERTY RIGHTS OF ANY THIRD PARTY IN ANY COUNTRY.
%
% Copyright (c) 1994-2006, John Conover, All Rights Reserved.
%
% Comments and/or bug reports should be addressed to:
%
%     john@email.johncon.com (John Conover)
%
% -----------------------------------------------------------------------------
%
% Revision: \RCSRevision \\
% Revision Time: \RCSTime UMT \\
% Revision Date: \RCSDate \\
% Revision Id: \RCSId \\
% Revision File: \RCSLog \\
\RCS $Revision: 0.0 $
\RCS $Date: 2006/01/20 04:38:13 $
\RCS $Id: instant.tex,v 0.0 2006/01/20 04:38:13 john Exp $
% $Log: instant.tex,v $
% Revision 0.0  2006/01/20 04:38:13  john
% Initial version
%
%
    \subsection{Instantaneous Analysis of Normalized Increments}
        \label{\SETLABEL:IA}

        \subidx{\market}{instantaneous analysis of normalized increments}
        \idx{average of normalized increments}
        \idx{root mean square of normalized increments}
        \subidx{Shannon probability}{instantaneous computation of}
        \subidx{average of normalized increments}{instantaneous computation of}
        \subidx{root mean square of normalized increments}{instantaneous computation of}
        \subidx{instantaneous computation}{Shannon probability}
        \subidx{instantaneous computation}{average of normalized increments}
        \subidx{instantaneous computation}{root mean square of normalized increments}
        \idx{time series}
        \subidx{time series}{instantaneous analysis}
        \subidx{instantaneous analysis}{time series}
        \subidx{time series}{increments}
        \subidx{time series}{analysis}
        \subidx{Shannon}{probability}
        \subidx{probability}{Shannon}
        \subidx{normalized}{increments}
        \subidx{increments}{normalized}

        The program {\it tsinstant}\/, which is briefly described in
        Appendix~\ref{programs}, is for finding the instantaneous
        fraction of change in a time series. The value of a sample in
        the time series is subtracted from the previous sample in the
        time series, and divided by the value of the previous sample.
        As explained in Chapter~\ref{general},
        Sections~\ref{derivation},~\ref{GA},~\ref{abmfi},~\ref{aftsma}
        and,~\ref{ompl} for Brownian motion, random walk fractals, the
        absolute value of the instantaneous fraction of change is also
        the root mean square of the instantaneous fraction of
        change\footnote{The absolute value of the normalized
        increments, when averaged, is related to the root mean square
        of the increments by a constant. If the normalized increments
        are a fixed increment, the constant is unity. If the
        normalized increments have a Gaussian distribution, the
        constant is $\approx 0.8$ depending on the accuracy of of
        ``fit'' to a Gaussian distribution.}. Squaring this value is
        the average of the instantaneous fraction of change, and
        adding unity to the absolute value of the instantaneous
        fraction of change, and dividing by two, is the Shannon
        probability of the instantaneous fraction of change.

        Figure~\ref{\SETLABEL:IA1} is the instantaneous value of the
        root mean square of the normalized increments for the
        {\market}, and Figure~\ref{\SETLABEL:IA2} is the instantaneous
        Shannon probability for the normalized increments.

        \begin{figure}[ht]
            \begin{center}
                \begin{minipage}[t]{0.45\textwidth}
                    \epsfxsize=1.0\linewidth
                    \epsffile{\directory/data.tsinstant-r.eps}
                    \caption[{\market}, instantaneous value of
                        rms.]{{\market}, instantaneous value of the
                        root mean square of the normalized increments,
                        provided by running the program {\it
                        tsinstant}\/ with the -r option on the data
                        presented in Figure~\ref{\SETLABEL:TS}.}
                    \label{\SETLABEL:IA1}
                    \label{\SETLABELQ:IA1}
                \end{minipage}
                \hfill
                \begin{minipage}[t]{0.45\textwidth}
                    \epsfxsize=1.0\linewidth
                    \epsffile{\directory/data.tsinstant-s.eps}
                    \caption[{\market}, instantaneous value of
                        Shannon probability.]{{\market}, instantaneous
                        value of the Shannon probability of the
                        normalized increments, provided by running the
                        program {\it tsinstant}\/ with the -s option
                        on the data presented in
                        Figure~\ref{\SETLABEL:TS}.}
                    \label{\SETLABEL:IA2}
                    \label{\SETLABELQ:IA2}
                \end{minipage}
            \end{center}
        \end{figure}

% Local Variables:
% TeX-parse-self: t
% TeX-auto-save: t
% TeX-master: "fractal.tex"
% End:


        %
% -----------------------------------------------------------------------------
%
% A license is hereby granted to reproduce this software source code and
% to create executable versions from this source code for personal,
% non-commercial use.  The copyright notice included with the software
% must be maintained in all copies produced.
%
% THIS PROGRAM IS PROVIDED "AS IS". THE AUTHOR PROVIDES NO WARRANTIES
% WHATSOEVER, EXPRESSED OR IMPLIED, INCLUDING WARRANTIES OF
% MERCHANTABILITY, TITLE, OR FITNESS FOR ANY PARTICULAR PURPOSE.  THE
% AUTHOR DOES NOT WARRANT THAT USE OF THIS PROGRAM DOES NOT INFRINGE THE
% INTELLECTUAL PROPERTY RIGHTS OF ANY THIRD PARTY IN ANY COUNTRY.
%
% Copyright (c) 1994-2006, John Conover, All Rights Reserved.
%
% Comments and/or bug reports should be addressed to:
%
%     john@email.johncon.com (John Conover)
%
% -----------------------------------------------------------------------------
%
% Revision: \RCSRevision \\
% Revision Time: \RCSTime UMT \\
% Revision Date: \RCSDate \\
% Revision Id: \RCSId \\
% Revision File: \RCSLog \\
\RCS $Revision: 0.0 $
\RCS $Date: 2006/01/20 04:38:13 $
\RCS $Id: logistic.tex,v 0.0 2006/01/20 04:38:13 john Exp $
% $Log: logistic.tex,v $
% Revision 0.0  2006/01/20 04:38:13  john
% Initial version
%
%
    \subsection{Logistic Analysis}
        \label{\SETLABEL:LA}

        \subidx{\market}{Logistic function analysis}
        \subidx{time series}{logistic function}
        \subidx{logistic function}{time series}
        \subidx{time series}{increments}
        \subidx{time series}{analysis}
        \subidx{cumulative sum}{analysis}
        \subidx{analysis}{cumulative sum}
        \subidx{analysis}{random process}
        \subidx{random process}{analysis}
        The data in this section is presented in tabular form in
        Section~\ref{\SETLABELREF:LAA}.  Figure~\ref{\SETLABEL:LA1} is
        a graph of the logistic function estimates of the time series
        data for the {\market}. The reader is cautioned that these
        graphs are constructed using the method suggested in
        Chapter~\ref{general}, Section~\ref{nlextend} and enormous
        precision is required for adequate prediction of the logistic
        function,~\cite{Modis}. Particularly, the non-linear term will
        usually require intervention to produce a practical fit to the
        data. In addition, there are numerical stability issues with
        logistic function methodologies\footnote{For example, in
        Figures~\ref{\SETLABEL:LA1} and~\ref{\SETLABEL:LA2}, if the
        non-linear term, $b$, was greater than zero, it was set to
        zero to produce the graphs. See Section~\ref{\SETLABELREF:LAA}
        for the actual derived values. In other cases, the magnitude
        of $b$ was too large, resulting in a graph that was decreasing
        as a function of time}.  The methodology should be regarded as
        ``fragile.'' It is included for completeness.

        \idx{least squares approximation}
        Figure~\ref{\SETLABEL:LA1} is a graph of the logistic function
        for the time series data presented in
        Figure~\ref{\SETLABEL:TS}. The data presented was made by
        running the program {\it tsdlogistic}\/, which is described
        briefly in Appendix~\ref{programs}, on the parameters
        extracted from the time series data as suggested in
        Figure~\ref{\SETLABEL:TF}. The program {\it tslsq}\/ was used
        to derive the constant and the slope of the normalized
        increments of the data presented in Figure~\ref{\SETLABEL:TF}.
        Figure~\ref{\SETLABEL:LA2} is the same graph, but with the
        time scale expanded by a factor of two.

        \begin{figure}[ht]
            \begin{center}
                \begin{minipage}[t]{0.45\textwidth}
                    \epsfxsize=1.0\linewidth
                    \epsffile{\directory/data.tsfraction.tslsq-p.tsdlogistic.eps}
                    \caption[{\market}, logistic function
                        estimates.]{{\market}, logistic function
                        estimates, provided by running the {\it
                        tslsq}\/ program on the normalized increments
                        presented in Figure~\ref{\SETLABEL:TF} with
                        the -p option. These parameters were used as
                        arguments to the {\it tsdlogistic}\/ program.}
                    \label{\SETLABEL:LA1}
                    \label{\SETLABELQ:LA1}
                \end{minipage}
                \hfill
                \begin{minipage}[t]{0.45\textwidth}
                    \epsfxsize=1.0\linewidth
                    \epsffile{\directory/data.tsfraction.tslsq-p.tsdlogistic2.eps}
                    \caption[{\market}, logistic function
                        estimates.]{{\market}, logistic function
                        estimates of Figure~\ref{\SETLABEL:LA1} with
                        the time scale expanded by a factor of two.}
                    \label{\SETLABEL:LA2}
                    \label{\SETLABELQ:LA2}
                \end{minipage}
            \end{center}
        \end{figure}

% Local Variables:
% TeX-parse-self: t
% TeX-auto-save: t
% TeX-master: "fractal.tex"
% End:


        %
% -----------------------------------------------------------------------------
%
% A license is hereby granted to reproduce this software source code and
% to create executable versions from this source code for personal,
% non-commercial use.  The copyright notice included with the software
% must be maintained in all copies produced.
%
% THIS PROGRAM IS PROVIDED "AS IS". THE AUTHOR PROVIDES NO WARRANTIES
% WHATSOEVER, EXPRESSED OR IMPLIED, INCLUDING WARRANTIES OF
% MERCHANTABILITY, TITLE, OR FITNESS FOR ANY PARTICULAR PURPOSE.  THE
% AUTHOR DOES NOT WARRANT THAT USE OF THIS PROGRAM DOES NOT INFRINGE THE
% INTELLECTUAL PROPERTY RIGHTS OF ANY THIRD PARTY IN ANY COUNTRY.
%
% Copyright (c) 1994-2006, John Conover, All Rights Reserved.
%
% Comments and/or bug reports should be addressed to:
%
%     john@email.johncon.com (John Conover)
%
% -----------------------------------------------------------------------------
%
% Revision: \RCSRevision \\
% Revision Time: \RCSTime UMT \\
% Revision Date: \RCSDate \\
% Revision Id: \RCSId \\
% Revision File: \RCSLog \\
\RCS $Revision: 0.0 $
\RCS $Date: 2006/01/20 04:38:13 $
\RCS $Id: hurst.tex,v 0.0 2006/01/20 04:38:13 john Exp $
% $Log: hurst.tex,v $
% Revision 0.0  2006/01/20 04:38:13  john
% Initial version
%
%
    \subsection{Hurst Coefficient Analysis}
        \label{\SETLABEL:H}

        \subidx{\market}{Hurst coefficient analysis}
        \subidx{Hurst coefficient}{analysis}
        \subidx{increments}{normalized}
        \subidx{normalized}{increments}
        \subidx{programs}{tshurst}
        \subidx{tshurst}{program}
        The data in this section is presented in tabular form in
        Section~\ref{\SETLABELREF:HCHP}. Figure~\ref{\SETLABEL:HC} is
        a graph of the Hurst coefficient data time series data shown
        in Figure~\ref{\SETLABEL:TS}. The slope of the graph is the
        Hurst coefficient.  The data for this figure was produced by
        the program {\it tshurst}\/, which is described briefly in
        Appendix~\ref{programs}.

        \subidx{\market}{H parameter analysis}
        \subidx{H parameter}{analysis}
        \subidx{programs}{tshcalc}
        \subidx{tshcalc}{program}
        Figure~\ref{\SETLABEL:HP} is a graph of the H parameter data
        for the normalized increments of the time series data shown in
        Figure~\ref{\SETLABEL:TF}. The data for this figure was
        produced by the program {\it tshcalc}\/, which is described
        briefly in Appendix~\ref{programs}.

        \begin{figure}[ht]
            \begin{center}
                \begin{minipage}[t]{0.45\textwidth}
                    \epsfxsize=1.0\linewidth
                    \epsffile{\directory/data.tshurst.eps}
                    \caption[{\market}, Hurst coefficient data]{{\market},
                        Hurst coefficient data for the normalized
                        increments of the time series data shown in
                        Figure~\ref{\SETLABEL:TF}.  The slope of the graph
                        is the Hurst coefficient.}
                    \label{\SETLABEL:HC}
                \end{minipage}
                \hfill
                \begin{minipage}[t]{0.45\textwidth}
                    \epsfxsize=1.0\linewidth
                    \epsffile{\directory/data.tshcalc.eps}
                    \caption[{\market}, H parameter data]{{\market}, H
                        parameter data for the normalized increments of
                        the time series data shown in
                        Figure~\ref{\SETLABEL:TF} The slope of the graph
                        is the H parameter.}
                    \label{\SETLABEL:HP}
                \end{minipage}
            \end{center}
        \end{figure}

        \subidx{revenue}{See, rate of revenue returns}
        \subidx{returns}{See, rate of revenue returns}
        \subidx{\market}{revenues}
        \subidx{Hurst coefficient}{analysis}
        \subidx{\market}{Hurst coefficient analysis}
        \subidx{\market}{rate of change}
        \subidx{\market}{windows of opportunity}
        \subidx{rate of revenue returns}{forecast}
        \subidx{forecast}{rate of revenue returns}
        \idx{windows of opportunity}
        \subidx{programs}{tslsq}
        \subidx{tslsq}{program}

        The approximately linear slope of the graph in
        Figure~\ref{\SETLABEL:HC} implies that the variance of the
        rate of revenue returns, (per {\timescale},) in the {\market},
        $V(t_2 - t_1)$, over a period of time is proportional to the
        period of time raised to twice the Hurst
        coefficient~\cite[pp. 180]{Feder},~\cite[pp. 246]{Crownover}.
        This seems to be a quantitative statement concerning how fast,
        and to what degree, the rate of revenue returns' state of
        affairs can change over a period of time.  An additional
        implication, for Hurst coefficients sufficiently close to 0.5,
        is that the probability of the state of affairs repeating
        sometime in the future goes down with increasing
        time\footnote{It can be shown that the number of expected
        market ``high'' and ``low'' transitions, $N$, scales with the
        square root of time, or $N \propto \sqrt {t}$, meaning that
        the cumulative distribution of the probability, $P$, of the
        duration of a market's ``high'' or ``low'' exceeding a given
        time interval, $t$, is proportional to the reciprocal of the
        square root of the time interval, $P \propto 1 / \sqrt {t}$,
        (or, conversely, that the probability of the duration of a
        market's ``high'' or ``low'' exceeding a given time interval
        is proportional to the reciprocal of the time interval raised
        to the power $3 / 2$, ie., $P \propto 1 / t^{3 /
        2}$,~\cite[pp. 153]{Schroeder}. What this means is that a
        histogram of the ``zero free'' run-lengths of a market being
        ``high'' or ``low,'' over a long time, would have a $1 / t^{3
        / 2}$ characteristic.)}, $t$, $p(t) = erf (1/\sqrt{2t})$ which
        is approximately $1/\sqrt{t}$ for $t \gg
        1$~\cite[pp. 160]{Schroeder}. Figures~\ref{\SETLABEL:FN},
        and,~\ref{\SETLABEL:FF} compare methods of approximation of
        the ``forecastability'' of the rate of revenue returns in the
        {\market} for the near term and far term,
        respectively~\cite[pp. 83-84]{Peters:CAOITCM}\footnote{The
        author is not comfortable with Peters' interpretation. For
        example, if the algorithm explained
        in~\cite[pp. 82]{Peters:CAOITCM} is used on ``white noise''
        which, by definition, never has any correlations, the short
        term Hurst coefficient, and thus the ``forecastability,'' is
        still near unity---a bit of an enigma. This can be verified
        with the {\it tswhite}\/ and {\it tshurst}\/ programs, which
        are briefly described in Appendix~\ref{programs}.}.  This
        seems to be a quantitative statement concerning ``windows of
        opportunity'' in the rate of revenue returns, (per
        {\timescale}.)  The program {\it tslsq}\/ was used on the
        Hurst coefficient data, presented in
        Figure~\ref{\SETLABEL:HC}, to provide a least squares
        approximation to the Hurst coefficient. The superimposed least
        squares approximation with on original Hurst coefficient data
        is presented.  The time series data has a Hurst coefficient of
        {\thurstlow}, so that:

        \subidx{\market}{Hurst coefficient analysis}
        \begin{eqnarray}
            V\left(t_2 - t_1\right) & \propto & \left(t_2 - t_1\right)^{2 \cdot H}\\
            V\left(t_2 - t_1\right) & \propto & \left(t_2 - t_1\right)^{2 \cdot {\thurstlow}}\\
                                    & \propto & \left(t_2 - t_1\right)^{\thurstlowtwo}
            \label{\SETLABEL:V}
        \end{eqnarray}

        \subidx{fractional}{Brownian motion}
        \subidx{Brownian motion}{fractional}
        \idx{fractal}
        \noindent where $V(t_2 - t_1)$ is the variance of the
        increments of the rate of revenue returns, (per {\timescale},)
        over the time interval $t_2 -
        t_1$,~\cite[pp. 177]{Feder},~\cite[pp. 494]{Peitgen}. If $H >
        \frac{1}{2}$, then the time series is termed as being
        characterized by ``fractional Brownian
        motion~\cite[pp. 170]{Feder}.''

        \subidx{rate of revenue returns}{predictability}
        \subidx{rate of revenue returns}{forecastability}
        \subidx{rate of revenue returns}{consistency}
        \subidx{predictability}{rate of revenue returns}
        \subidx{forecastability}{rate of revenue returns}
        \subidx{consistency}{rate of revenue returns}
        \subidx{\market}{rate of revenue returns, predictability}
        \subidx{\market}{rate of revenue returns, forecastability}
        \subidx{\market}{rate of revenue returns, consistency}
        \subidx{Hurst coefficient}{analysis}
        \subidx{\market}{Hurst coefficient analysis}
        \subidx{\market}{rate of change}

        In some sense, the Hurst coefficient is a quantitative
        expression of the ``forecastability'' of the future based on
        the past\footnote{Actually, in general, when summing fractal
        entities, the method used should be a root mean square
        process, dependent on the Hurst Coefficient, $H$, where
        $P_{total}^H = P_1^H + P_2^H + \cdots$, where $P_n$ is the
        fractal entities. For a Brownian motion, or random walk type
        of fractal the Hurst Coefficient is a function of time into
        the future. For the ``near term,'' the Hurst coefficient is
        very near unity, meaning the summation process is linear. For
        the ``long term,'' $H \approx 0.5$, or a standard root mean
        square summation process should be used. If $H$ is $0.5$ then
        the market is termed a Brownian motion, or random walk
        process. If it is larger than 0.5, it is termed fractional
        Brownian motion process. For a random walk process, ``near
        term'' and ``far term'' are quantitatively differentiated on
        the Hurst Coefficient graph where $1 - \ln (t) = 0.5 \cdot \ln
        (t)$, or when $\ln (t) = 2$, or $t = 7.389\ldots$ See
        Section~\ref{\SETLABEL:FS} for the particulars on using Hurst
        Coefficient to sum fractal process' for the {\market}. See
        also~\cite[pp. 67, 83-84]{Peters:CAOITCM} and~\cite[pp. 129,
        159]{Schroeder} for particulars on the implications of the
        Hurst Coefficient and root mean square summation issues.}.  A
        Hurst coefficient of {\thurstlow}, (for the near future, and
        {\thurstall} for the distant future.) implies that the
        likelihood of the rate of revenue returns, (per {\timescale},)
        for any two consecutive {\timescale}s being the same is
        {\thurstlowhundred}\%~\cite[pp. 66]{Peters:CAOITCM} for the
        near future, and {\thurstall} for the distant
        future. Likewise, there is a {\thurstlowhundred}\% chance of
        the rate of revenue returns, (per {\timescale},) movements
        being the same in consecutive time periods---ie., if, in a
        given {\timescale}, the rate of revenue returns, (per
        {\timescale},) is increasing, there is a {\thurstlowhundred}\%
        that the rate of revenue returns, (per {\timescale},) will
        increase in the following period, also. In some sense, this is
        a quantitative statement on how ``predictable,'' or
        ``forecastable'' the rate of revenue returns, (per
        {\timescale},) for the {\market} are over time, since the
        probability of having $n$ many consecutive {\timescale}s of
        the same agenda is $H^n$ where $H$ is the Hurst coefficient,
        or, letting the short term probability of having $n$ many
        {\timescale}s of the same market agenda, $p_a$, is:

        \begin{eqnarray}
            p_a\left(n\right) & = & H^{n}\\
                              & = & {\thurstlow}^{n}
            \label{\SETLABEL:MA}
        \end{eqnarray}

        \subidx{rate of revenue returns}{predictability}
        \subidx{rate of revenue returns}{forecastability}
        \subidx{rate of revenue returns}{consistency}
        \subidx{predictability}{rate of revenue returns}
        \subidx{forecastability}{rate of revenue returns}
        \subidx{consistency}{rate of revenue returns}
        As an interesting interpretation of the normalized increments
        of the time series data presented in
        Figure~\ref{\SETLABEL:TF}, if the vertical axis is multiplied
        by 100, to convert to percent, then the graph represents the
        error, in percent, that would be made by forecasting, month by
        month, that the next {\timescale}'s rate of revenue returns
        would be the same as the current {\timescale}'s revenue
        rate. Interestingly, it is $\datafractionmean \cdot 100$
        percent, on the average, with a standard deviation of
        $\datafractionstddev \cdot 100$ percent, and a root mean
        square error value of $\datafractionrms \cdot 100$
        percent---small values for such a simple forecasting
        mechanism.

        \subidx{\market}{rate of revenue returns, range}
        \subidx{Hurst coefficient}{analysis}
        \subidx{\market}{Hurst coefficient analysis}
        \subidx{\market}{rate of change}

        This is, essentially, a statement of the range of values, in
        the increments of the rate of revenue returns, (per
        {\timescale},) that is to be expected over the time interval,
        $t_2 - t_1$,
        $R_v$,~\cite[pp. 178]{Feder},~\cite[pp. 172]{Cambel}:

        \begin{eqnarray}
            R_v\left(t_2 - t_1\right) & \propto & \left(t_2 - t_1\right)^{H}\\
                                      & \propto & \left(t_2 - t_1\right)^{\thurstlow}
            \label{\SETLABEL:R}
        \end{eqnarray}

        \subidx{\market}{rate of revenue returns, range}
        \subidx{Hurst coefficient}{analysis}
        \subidx{\market}{Hurst coefficient analysis}
        \subidx{\market}{rate of change}
        \subidx{Markov}{statistics}
        \subidx{statistics}{Markov}
        \noindent where $R$ is the range of values in the increments
        of the rate of revenue returns, (per {\timescale}.) A Hurst
        coefficient, $H$, that is much larger than $\frac{1}{2}$, (but
        less than 1,) implies a strongly non-Gaussian distribution in
        the increments of the rate of revenue returns, (per
        {\timescale},)~\cite[pp. 152, 194]{Feder}, and a Hurst
        coefficient near $\frac{1}{2}$ implies that the increments of
        the rate of revenue returns, (per {\timescale}) is
        characteristic of an independent
        process~\cite[pp. 195]{Feder}. Extreme caution should be
        exercised in using Markov statistics in any analysis where the
        Hurst coefficient is not
        $\frac{1}{2}$,~\cite[pp. 124]{Crownover},~\cite[pp. 106]{Peters:CAOITCM}.


        As a useful approximation, if $H$, is approximately
        $\frac{1}{2}$, Equation~\ref{\SETLABEL:R} reduces
        to,~\cite[pp. 129]{Schroeder}:

        \begin{eqnarray}
            R\left(t_2 - t_1\right) & \propto & (t_2 - t_1)^{\frac{1}{2}}\\
                                    & \propto & \sqrt{\left(t_2 - t_1\right)}
        \end{eqnarray}

        \subidx{\market}{rate of revenue returns, range}
        \subidx{\market}{rate of revenue returns, increase and decrease}
        \subidx{Hurst coefficient}{analysis}
        \subidx{\market}{Hurst coefficient analysis}
        \subidx{\market}{rate of change}
        \subidx{Markov}{statistics}
        \subidx{statistics}{Markov}

        In the case where the Hurst coefficient, $H$, is
        $\frac{1}{2}$, the range of values in the increments of the
        rate of revenue returns, (per {\timescale},) divided by the
        standard deviation of these values, $S$, can be anticipated to
        increase over time according to the following
        relation,~\cite[pp. 154]{Feder},~\cite[pp. 129]{Schroeder}:

        \begin{equation}
            \frac{R\left(t_2 - t_1\right)}{S} \propto \left(t_2 - t_1\right)^{\frac{1}{2}}
        \end{equation}

        \subidx{\market}{rate of revenue returns, range}
        \subidx{\market}{rate of revenue returns, increase and decrease}
        \subidx{Hurst coefficient}{analysis}
        \subidx{\market}{Hurst coefficient analysis}
        \subidx{\market}{rate of change}
        \noindent which is a useful conceptual approximation, since it
        involves only the square root function---if the range and the
        standard deviation of the increments of the rate of revenue
        returns, (per {\timescale},) are known, (and $H \approx
        \frac{1}{2}$,) then the expected change in $\frac{R}{S}$, will
        increase with the square root of time\footnote{To be precise,
        it is actually asymptotically proportional to
        $\tau^{\frac{1}{2}}$}.

        Another useful approximation when rescaling processes that are
        characterize by Brownian motion, (ie., when $H \approx
        \frac{1}{2}$,) is that:

        \begin{eqnarray}
            X\left(t\right) & \propto & \frac{X\left(rt\right)}{r^{H}}\\
                            & \propto & \frac{X\left(rt\right)}{r^{\thurstlow}}
        \end{eqnarray}

        \idx{Brownian motion}
        \idx{fractal}
        Where $X(t)$ is the process characterized by Brownian motion,
        and $r$ is a scaling factor,~\cite[pp. 494]{Peitgen}.

        \subidx{programs}{tslsq}
        \subidx{tslsq}{program}
        The program {\it tslsq}\/ was used on the H parameter data,
        presented in Figure~\ref{\SETLABEL:HP}, to provide a least
        squares approximation to the H parameter for the
        {\market}. The superimposed least squares approximation on the
        original H parameter data is presented.  By contrast, the H
        parameter, as derived by the methodology outlined
        in~\cite[pp. 249]{Crownover}, is {\thcalclow} for the near
        future, and {\thcalcall} for the distant future.

        \subidx{\market}{Hurst coefficient analysis}
        \subidx{Hurst coefficient}{analysis}
        \subidx{increments}{normalized}
        \subidx{normalized}{increments}
        \subidx{programs}{tshurst}
        \subidx{tshurst}{program}
        \subidx{\market}{H parameter analysis}
        \subidx{H parameter}{analysis}
        \subidx{programs}{tshcalc}
        \subidx{tshcalc}{program}
        Figures~\ref{\SETLABEL:HC} and~\ref{\SETLABEL:HP} represent
        Hurst coefficient and H parameter data that are derived from
        the normalized increments, shown in
        Figure~\ref{\SETLABEL:TF}. In this case, the data is
        considered a normalized derivative of the time series data
        presented in Figure~\ref{\SETLABEL:TF}, instead of a
        cumulative sum.  The program, {\it tshurst}\/, is described
        briefly in appendix~\ref{programs}, and the data for
        figures~\ref{\SETLABEL:THC} and~\ref{\SETLABEL:THP} was made
        using the -d option.

        \begin{figure}[ht]
            \begin{center}
                \begin{minipage}[t]{0.45\textwidth}
                    \epsfxsize=1.0\linewidth
                    \epsffile{\directory/data.tsfraction.tshurst-d.eps}
                    \caption[{\market}, traditional Hurst coefficient
                        data]{{\market}, traditional Hurst coefficient
                        data for the time series data shown in
                        Figure~\ref{\SETLABEL:TS}.  The slope of the
                        graph is the Hurst coefficient, and is
                        {\hurstlow} for the near term, and
                        {\hurstall} for the far term.}
                    \label{\SETLABEL:THC}
                \end{minipage}
                \hfill
                \begin{minipage}[t]{0.45\textwidth}
                    \epsfxsize=1.0\linewidth
                    \epsffile{\directory/data.tsfraction.tshcalc-d.eps}
                    \caption[{\market}, traditional H parameter
                        data]{{\market}, traditional H parameter data
                        for the time series data shown in
                        Figure~\ref{\SETLABEL:TS} The slope of the
                        graph is the H parameter, and is {\hcalclow}
                        for the near term, and {\hcalcall} for the
                        far term.}
                    \label{\SETLABEL:THP}
                \end{minipage}
            \end{center}
        \end{figure}

% Local Variables:
% TeX-parse-self: t
% TeX-auto-save: t
% TeX-master: "fractal.tex"
% End:


        %
% -----------------------------------------------------------------------------
%
% A license is hereby granted to reproduce this software source code and
% to create executable versions from this source code for personal,
% non-commercial use.  The copyright notice included with the software
% must be maintained in all copies produced.
%
% THIS PROGRAM IS PROVIDED "AS IS". THE AUTHOR PROVIDES NO WARRANTIES
% WHATSOEVER, EXPRESSED OR IMPLIED, INCLUDING WARRANTIES OF
% MERCHANTABILITY, TITLE, OR FITNESS FOR ANY PARTICULAR PURPOSE.  THE
% AUTHOR DOES NOT WARRANT THAT USE OF THIS PROGRAM DOES NOT INFRINGE THE
% INTELLECTUAL PROPERTY RIGHTS OF ANY THIRD PARTY IN ANY COUNTRY.
%
% Copyright (c) 1994-2006, John Conover, All Rights Reserved.
%
% Comments and/or bug reports should be addressed to:
%
%     john@email.johncon.com (John Conover)
%
% -----------------------------------------------------------------------------
%
% Revision: \RCSRevision \\
% Revision Time: \RCSTime UMT \\
% Revision Date: \RCSDate \\
% Revision Id: \RCSId \\
% Revision File: \RCSLog \\
\RCS $Revision: 0.0 $
\RCS $Date: 2006/01/20 04:38:13 $
\RCS $Id: fiscal.tex,v 0.0 2006/01/20 04:38:13 john Exp $
% $Log: fiscal.tex,v $
% Revision 0.0  2006/01/20 04:38:13  john
% Initial version
%
%
    \subsection{Fixed Increment Approximation for Fiscal Strategy}
        \label{\SETLABEL:FS}

        \subidx{\market}{fiscal strategy}
        \subidx{markets}{analysis}
        \subidx{analysis}{markets}
        \subidx{strategy}{fiscal}
        \subidx{fiscal}{strategy}
        The data in this section is presented in tabular form in
        Section~\ref{\SETLABELREF:LR}. This section derives various
        values based on the ``average'' of the normalized increments
        presented in Figure~\ref{\SETLABEL:TFA}. These values are an
        approximation to a, probably, complex process with a
        distribution shown in Figure~\ref{\SETLABEL:TF}. These values
        will be used in a fixed increment Brownian fractal analysis
        and simulation of the {\market}, and may, or may not, provide
        adequate accuracy for projections.

        For an organization operating in the {\market}, the fiscal
        strategy, commensurate with the aggregate environment, can be
        derived as follows~\cite[pp. 128, pp
        151]{Schroeder},~\cite[pp. 450]{Reza},~\cite[pp. 270]{Pierce}:
        \vspace{0.15in}

        \subsubsection{Logarithmic Returns}
            \label{\SETLABEL:LR}

            \subidx{logarithmic}{returns}
            \subidx{returns}{logarithmic}
            \subidx{\market}{logarithmic returns}
            The logarithmic returns can be calculated by various
            means. Four will be presented here, for comparison.

            \subidx{programs}{tsnormal}
            \subidx{tsnormal}{program}
            \subidx{logarithmic}{returns}
            \subidx{returns}{logarithmic}
            The logarithmic returns, in bits, $bits$, as computed from
            the mean, by the program {\it tsnormal}\/, which is
            described in Chapter~\ref{programs}, and is presented in
            Figure~\ref{\SETLABEL:TF}, and Equation~\ref{abits} from
            Section~\ref{ereturns} in Chapter~\ref{general}:

            \begin{equation}
                bits = \frac{\ln \left({\datafractionmean} + 1\right)}{\ln \left(2\right)} = \datafractionmeanbits
            \end{equation}

            \subidx{programs}{tslsq}
            \subidx{tslsq}{program}
            \subidx{logarithmic}{returns}
            \subidx{returns}{logarithmic}
            \noindent By comparison, the logarithmic returns, in bits,
            $bits$, as computed from the constant in the least squares
            approximation, using the program {\it tslsq}\/, which is briefly
            described in Chapter~\ref{programs}, as presented in
            Figure~\ref{\SETLABEL:TF}, and Equation~\ref{abits} from
            Section~\ref{ereturns} in Chapter~\ref{general}:

            \begin{equation}
                bits = \frac{\ln \left({\datafractionconstant} + 1\right)}{\ln \left(2\right)} = \datafractionconstantbits
            \end{equation}

            Note that if the mean is not constant in
            Figure~\ref{\SETLABEL:TF}, this method will not provide
            accurate results.

            \subidx{programs}{tslsq}
            \subidx{tslsq}{program}
            \subidx{logarithmic}{returns}
            \subidx{returns}{logarithmic}
            \noindent And by yet another comparison, using the program
            {\it tslsq}\/, which is briefly described in
            Chapter~\ref{programs}, with the -e -p options, to provide
            a formula for the least squares exponential fit to the
            time series data set presented in
            Figure~\ref{\SETLABEL:TS}:

            \begin{equation}
                bits = {\datatslsqepbits}
            \end{equation}

            \subidx{programs}{tslogreturns}
            \subidx{tslogreturns}{program}
            \subidx{logarithmic}{returns}
            \subidx{returns}{logarithmic}
            \noindent And finally, by comparison, from the
            {\it tslogreturns}\/ program, which is briefly described
            in Chapter~\ref{programs}, with the -p option, to provide
            a formula for the logarithmic returns of the time series
            data set presented in Figure~\ref{\SETLABEL:TS}:

            \begin{equation}
                bits = {\logreturns}
            \end{equation}

        \subsubsection{Calculation of Shannon Probability}
            \label{\SETLABEL:SP}

            \subidx{\market}{Shannon probability}
            Ideally, all of the values presented in
            Section~\ref{\SETLABEL:LR} would be equal. Using the
            logarithmic returns provided by the {\it tslogreturns}\/
            program, to be consistent
            with~\cite[pp. 81]{Peters:CAOITCM}

            \subidx{programs}{tslogreturns}
            \subidx{tslogreturns}{program}
            \begin{equation}
                2^{{\logreturns}t}
            \end{equation}

            \noindent therefore:
            \begin{equation}
                C\left(p\right) = {\logreturns}
            \end{equation}
            \subidx{programs}{tsshannon}
            \subidx{tsshannon}{program}
            \subidx{Shannon}{probability}
            \subidx{probability}{Shannon}
            \noindent and, {\it tsshannon}\/ {\logreturns} gives:
            \begin{equation}
                \label{\SETLABEL:F0}
                C\left({\shannonlogreturns}\right) = {\logreturns}
            \end{equation}
            \noindent therefore:
            \begin{eqnarray}
                2^{C\left({\shannonlogreturns}\right)} & = & 2^{\logreturns}\\
                                                       & = & {\twologreturns}\\
                                                       & = & {\twologreturnshundred}\%
            \end{eqnarray}
            \noindent and:
            \begin{eqnarray}
                2p - 1 & = & \left(2 \cdot {\shannonlogreturns}\right) - 1\\
                       & = & {\twopone}\\
                       \label{\SETLABEL:F1}
                       & = & {\twoponehundred}\%
            \end{eqnarray}

            \subidx{\market}{fiscal strategy}
            \subidx{markets}{analysis}
            \subidx{analysis}{markets}
            \subidx{strategy}{fiscal}
            \subidx{fiscal}{strategy}
            \subidx{\market}{fiscal strategy}
            \subidx{\market}{growth rate}
            Presuming the simplified assumptions outlined in
            Section~\ref{assumptions}, the ``typical'' organization
            operating in the {\market} executes a long term fiscal
            strategy, commensurate with the aggregate environment,
            that is to invest, every {\timescale}, in sufficient
            additional resources and infrastructure, to increase the
            manufacturing of goods and services by {\twoponehundred}\%
            of its rate of revenue returns, (per {\timescale}.) As a
            conceptual model, the remaining {\hundredtwoponehundred}\%
            will be held in ``reserve'' with a
            {\shannonlogreturnshundred}\% chance of making twice the
            {\twoponehundred}\% back, (and a
            {\hundredshannonlogreturnshundred}\% chance of making
            0.0,) in one {\timescale}, on the average, for an average
            growth in its rate of revenue returns, (per {\timescale},)
            of {\twologreturnshundred}\%, or a doubling of its rate of
            revenue returns, (per {\timescale},) in
            {\oneoverlogreturns} {\timescale}s.

        \subsubsection{Example Fixed Increment Approximation Fiscal Strategies}

            \subidx{\market}{fiscal strategy}
            \subidx{markets}{analysis}
            \subidx{analysis}{markets}
            \subidx{strategy}{fiscal}
            \subidx{fiscal}{strategy}
            \subidx{\market}{fiscal strategy}
            \subidx{\market}{growth rate}
            \subidx{\market}{management metric}
            \idx{management metric}
            A possible metric on the effectiveness of long term fiscal
            management could possibly be that if an investment of
            {\twoponehundred}\% per {\timescale} of the rate of
            revenue returns, (per {\timescale},) is made in resources
            and infrastructure, then the rate of revenue returns would
            be expected to increase by {\twologreturnshundred}\%, per
            {\timescale}, on average.

            Note that the metrics presented in this section are
            representative of the {\market} as an aggregate whole, and
            may or may not be accurate representations for any
            particular participant in the environment. Of interest to
            the participants in the environment would be a similar
            analysis of each product or service rendered in the
            marketplace.

            \subidx{\market}{fiscal strategy}
            \subidx{markets}{analysis}
            \subidx{analysis}{markets}
            \subidx{strategy}{fiscal}
            \subidx{fiscal}{strategy}
            \subidx{\market}{fiscal strategy}
            As a simple illustrative example, a company operating in
            this environment might obtain a credit line from a bank
            that is equal to {\twoponehundred}\% of its rate of
            revenue returns, (per {\timescale},) to finance additional
            operations. In this simple scenario, the company would use
            its revenue base as collateral for the loan. Some
            {\timescale}s, depending on the {\market}'s environment,
            the company's rate of revenue returns exceeds what was
            borrowed from the bank, and the loan is repaid in
            full. Other {\timescale}s, the company must default, and
            the bank seizes a portion of the company's revenue base to
            pay the delinquent loan. However, on the average, the
            company will expand its rate of revenue returns at
            {\twologreturnshundred}\% per {\timescale}.

            \subidx{\market}{fiscal strategy}
            \subidx{markets}{analysis}
            \subidx{analysis}{markets}
            \subidx{strategy}{fiscal}
            \subidx{fiscal}{strategy}
            \subidx{\market}{fiscal strategy}
            As another simple example, a company re-invests
            {\twoponehundred}\% of its rate of revenue returns, (per
            {\timescale},) in development, marketing, sales, and
            distribution of new products.  Although some products will
            be successful and the return on the investment will exceed
            the {\twoponehundred}\% per {\timescale} investment,
            others will not. However, on the average, the company will
            expand it gross rate of revenue returns at
            {\twologreturnshundred}\% per {\timescale}.

            \subidx{\market}{fiscal strategy}
            \subidx{markets}{analysis}
            \subidx{analysis}{markets}
            \subidx{strategy}{fiscal}
            \subidx{fiscal}{strategy}
            \subidx{\market}{fiscal strategy}
            \subidx{\market}{product portfolio}
            \subidx{\market}{product diversity}
            \subidx{\market}{product mix}
            \subidx{\market}{optimum number of products}
            \idx{product portfolio}
            \idx{product diversity}
            \idx{optimum number of products}
            \idx{product mix}

            As an example of ``product portfolio'' management, suppose
            a company re-invests {\twoponehundred}\% of its rate of
            revenue returns, (per {\timescale},) in development,
            marketing, sales, and distribution of new products.
            Further suppose that the company has two products, and a
            fractal analysis of the individual product rate of revenue
            return time series indicates that one product has a
            Shannon probability of 0.65, and the other has a Shannon
            probability of 0.55. Then the percentage of re-investment
            in the first product would be $(2 \cdot 0.65 - 1) \cdot
            {\twoponehundred}$, percent of the rate of revenue
            returns, and $(2 \cdot 0.55 - 1) \cdot {\twoponehundred}$
            percent for the second product, implying that the company
            should diversify its product line\footnote{The astute
            reader would note that the linear addition was used to add
            the contribution to development of each product. This is a
            ``near term'' interpretation. Actually, in general, the
            method used should be a root mean square process,
            dependent on the Hurst Coefficient, $H$, where
            $P_{total}^H = P_1^H + P_2^H + \cdots$, where $P_n$ is the
            contribution to each individual product. For a Brownian
            motion, or random walk type of fractal the Hurst
            Coefficient is a function of time into the future. For the
            ``near term,'' the Hurst coefficient is very near unity,
            meaning the summation process is linear. For the ``long
            term,'' $H \approx 0.5$, or a standard root mean square
            summation process should be used. If $H$ is $0.5$ then the
            market is termed a Brownian motion, or random walk
            process. If it is larger than 0.5, it is termed fractional
            Brownian motion process. For a random walk process, ``near
            term'' and ``far term'' are quantitatively differentiated
            on the Hurst Coefficient graph where $1 - \ln (t) = 0.5
            \cdot \ln (t)$, or when $\ln (t) = 2$, or $t =
            7.389\ldots$ See~\cite[pp. 67, 83-84]{Peters:CAOITCM}
            and~\cite[pp. 129, 159]{Schroeder} for particulars on the
            implications of the Hurst Coefficient and root mean square
            summation issues.}.  Note that this is a ``bet hedging''
            metric methodology, and assumes that the products have
            uncorrelated revenue return rates. If this re-investment
            methodology is not feasible, perhaps for strategic
            financial reasons, then the re-investment in both products
            should total the ${\twoponehundred}$\%, and the investment
            in each product should be made at a ratio of $\frac{(2
            \cdot 0.65 - 1)}{(2 \cdot 0.55 - 1)} = 3 : 1$,
            respectively. Note that this ``bet hedging'' can be used
            to define the optimal number of products that can be
            supported on the rate of revenue returns. If it assumed
            that all products are ``typical'' for the {\market}, as a
            standard bench mark, then the optimal number will be
            $\frac{1}{{\twopone}}$. Note that this is a
            ``theoretical'' value, since not all products are
            ``typical,'' and there may be strategic reasons, for
            example product leveraging, that may increase the number
            of products above the optimum. However, most of the
            revenue should come from the optimal number of products,
            since having more products will decrease the amount of the
            potential investment in each product, and having less than
            the optimum number of products will increase the risk that
            many of the products could suffer a ``down market''
            concurrently, impacting the rate of revenue returns.  As
            another interesting interpretation of the optimal
            ``hedging of bets,'' in product portfolio strategy, and
            considering the graph of the normalized increments
            presented in Figure~\ref{\SETLABEL:TF}, if the
            organization is running optimally, then these products
            will generate, at least in principle, one standard
            deviation, approximately $0.8413 = 84.13$\% of the future
            growth in rate of revenue returns. Naturally, these are
            approximations, and the values are an approximation to a,
            probably, complex process, and appropriate scrutiny should
            be exercised before making specific projections.  As yet
            another example of ``product portfolio'' management,
            consider the issue of product mix. In this interpretation,
            {\twoponehundred}\% of the product manufactured should be
            ``proprietary,'' while the rest is ``industry standard.''
            As yet another possibility, {\twoponehundred}\% of the
            product manufactured should be predatory into new markets,
            and the remainder in markets that are ``traditional'' for
            the company.

% Local Variables:
% TeX-parse-self: t
% TeX-auto-save: t
% TeX-master: "fractal.tex"
% End:


        %
% -----------------------------------------------------------------------------
%
% A license is hereby granted to reproduce this software source code and
% to create executable versions from this source code for personal,
% non-commercial use.  The copyright notice included with the software
% must be maintained in all copies produced.
%
% THIS PROGRAM IS PROVIDED "AS IS". THE AUTHOR PROVIDES NO WARRANTIES
% WHATSOEVER, EXPRESSED OR IMPLIED, INCLUDING WARRANTIES OF
% MERCHANTABILITY, TITLE, OR FITNESS FOR ANY PARTICULAR PURPOSE.  THE
% AUTHOR DOES NOT WARRANT THAT USE OF THIS PROGRAM DOES NOT INFRINGE THE
% INTELLECTUAL PROPERTY RIGHTS OF ANY THIRD PARTY IN ANY COUNTRY.
%
% Copyright (c) 1994-2006, John Conover, All Rights Reserved.
%
% Comments and/or bug reports should be addressed to:
%
%     john@email.johncon.com (John Conover)
%
% -----------------------------------------------------------------------------
%
% Revision: \RCSRevision \\
% Revision Time: \RCSTime UMT \\
% Revision Date: \RCSDate \\
% Revision Id: \RCSId \\
% Revision File: \RCSLog \\
\RCS $Revision: 0.0 $
\RCS $Date: 2006/01/20 04:38:13 $
\RCS $Id: companies.tex,v 0.0 2006/01/20 04:38:13 john Exp $
% $Log: companies.tex,v $
% Revision 0.0  2006/01/20 04:38:13  john
% Initial version
%
%
    \subsection{Number of Companies}
        \label{\SETLABEL:QNC}

        \subidx{\market}{number of companies}
        \subidx{number of companies}{analysis}
        \subidx{analysis}{number of companies}
        \subidx{Shannon}{probability}
        \subidx{probability}{Shannon}
        This section evaluates the approximate, or ``average,'' number
        of companies in the {\market}, and uses the method outlined in
        Chapter~\ref{general}, Section~\ref{aftsma}. Since the
        average, $avg_{ind}$, and the root mean square, $rms_{ind}$,
        of the normalized increments of the {\market} time series is
        \datafractionmean, and \datafractionrms respectively, the
        number of companies participating in the market can be
        calculated by Equation~\ref{ncompanies} to be {\ncompanies}.

        If this value seems consistent number of companies in the
        {\market}, within the assumptions outlined in
        Chapter~\ref{general}, Section~\ref{aftsma}, then it would
        seem that there is some circumstantial or indirect evidence
        that the companies participating in the {\market} are
        operating optimally, and the ``average'' Shannon probability,
        $P$ for each participating company would be, using
        Equation~\ref{pncompanies}, {\pncompanies}, which would be the
        value which should be used in Section~\ref{\SETLABEL:FS} for
        each participating company if market expansion was to be
        consistent with the rest of the industry. However, if the
        Shannon probability derived in Section~\ref{\SETLABEL:FS} is
        greater than the average Shannon probability for the companies
        participating in the {\market}, as derived in this section,
        then the market would, possibly, be exploitable with the
        fiscal strategy outlined in Section~\ref{\SETLABEL:FS}. The
        maximum exploitability for the {\market} is derived in
        Section~\ref{\SETLABEL:MAXSHANNON}, but it is probably of
        doubtful practicality.

        Note that these optimizations would maximize a company's
        market growth. Since there are probably many companies
        competing in the market place, this would not necessarily
        maximize a company's P\&L, as described in
        Chapter~\ref{general}, Section~\ref{ompl}. The Shannon
        probability that maximizes market share in the {\market} is
        \pncompanies, with several alternative solutions listed in the
        previous paragraph. However, these should be contrasted to the
        Shannon probability that maximizes a company's P\&L which is
        \avgrms~in the {\market}. In all cases, the fraction of the
        P\&L that should be ``wagered'' on the future, $f$, should be:

        \begin{equation}
            f = 2P - 1
        \end{equation}

        \noindent where $P$ is the particular Shannon probability
        chosen optimize a particular fiscal strategy. Interestingly,
        the measured Shannon probability of the {\market} would tend
        to indicate that the companies participating in the market
        have chosen a fiscal strategy that optimizes market growth, as
        opposed to capital growth.

        \subidx{\market}{increasing returns}
        \subidx{economic increasing returns}{\market}
        As interesting interpretation of these exploitive issues,
        since all three fiscal strategies will result in exponential
        market growth for every company participating in the market,
        is that they may represent, perhaps, an example of
        ``increasing returns.''

% Local Variables:
% TeX-parse-self: t
% TeX-auto-save: t
% TeX-master: "fractal.tex"
% End:


        %
% -----------------------------------------------------------------------------
%
% A license is hereby granted to reproduce this software source code and
% to create executable versions from this source code for personal,
% non-commercial use.  The copyright notice included with the software
% must be maintained in all copies produced.
%
% THIS PROGRAM IS PROVIDED "AS IS". THE AUTHOR PROVIDES NO WARRANTIES
% WHATSOEVER, EXPRESSED OR IMPLIED, INCLUDING WARRANTIES OF
% MERCHANTABILITY, TITLE, OR FITNESS FOR ANY PARTICULAR PURPOSE.  THE
% AUTHOR DOES NOT WARRANT THAT USE OF THIS PROGRAM DOES NOT INFRINGE THE
% INTELLECTUAL PROPERTY RIGHTS OF ANY THIRD PARTY IN ANY COUNTRY.
%
% Copyright (c) 1994-2006, John Conover, All Rights Reserved.
%
% Comments and/or bug reports should be addressed to:
%
%     john@email.johncon.com (John Conover)
%
% -----------------------------------------------------------------------------
%
% Revision: \RCSRevision \\
% Revision Time: \RCSTime UMT \\
% Revision Date: \RCSDate \\
% Revision Id: \RCSId \\
% Revision File: \RCSLog \\
\RCS $Revision: 0.0 $
\RCS $Date: 2006/01/20 04:38:13 $
\RCS $Id: operations.tex,v 0.0 2006/01/20 04:38:13 john Exp $
% $Log: operations.tex,v $
% Revision 0.0  2006/01/20 04:38:13  john
% Initial version
%
%
    \subsection{Fixed Increment Approximation for Operational Strategy}
        \label{\SETLABEL:OPS}.

        This section derives various values based on the ``average''
        of the normalized increments presented in
        Figure~\ref{\SETLABEL:TFA}. These values are an approximation
        to a, probably, complex process with a distribution shown in
        Figure~\ref{\SETLABEL:TF}. These values will be used in a
        fixed increment Brownian fractal analysis and simulation of
        the {\market}, and may, or may not, provide adequate accuracy
        for projections.

        \subidx{\market}{fiscal strategy}
        \subidx{\market}{Shannon probability}
        \subidx{strategy}{fiscal}
        \subidx{fiscal}{strategy}
        \subidx{Shannon}{probability}
        \subidx{probability}{Shannon}
        It should be noted that the analysis of fiscal strategy,
        presented in Section~\ref{\SETLABEL:FS}, is derived from the
        {\market} metrics and may, or may not, be maximally
        optimal. For the optimal fiscal strategy, which may be
        exploitable, see Section~\ref{\SETLABEL:MAXSHANNON}.

        \subidx{strategy}{exploitable}
        \subidx{exploitable}{strategy}
        \subidx{\market}{windows of opportunity}
        \idx{windows of opportunity}
        \subidx{decision}{obsolete}
        \subidx{obsolete}{decision}
        \subidx{decision}{timeliness}
        \subidx{timeliness}{decision}
        \subidx{rate of revenue returns}{forecast}
        \subidx{forecast}{rate of revenue returns}
        An additional exploitable strategy may be time itself.
        Equations~\ref{\SETLABEL:V},~\ref{\SETLABEL:R},
        and,~\ref{\SETLABEL:MA}, are, essentially, metrics on how fast
        a decision, which is based on information concerning the
        current status of the {\market}, becomes obsolete. Obviously,
        how long a decision is expected to remain relevant should be
        addressed as an operational necessity in strategic planning
        and project management. Figures~\ref{\SETLABEL:FN},
        and,~\ref{\SETLABEL:FF} compare methods of approximation of
        the ``forecastability'' of rate of revenue returns in the
        {\market} for the near term and far
        term~\cite[pp. 83-84]{Peters:CAOITCM}, respectively. As a
        general rule, caution must be exercised when making decisions
        that will span a time interval larger than the time interval
        where the ``forecastability'' of rate of revenue returns drops
        below 50\%. Beyond this time interval, the chances increase
        that the competitive and market forces will alter the market
        environment in a possibly detrimental unanticipated
        fashion. Obviously, there is significant advantage in
        ``timeliness'' of development, manufacturing, and distribution
        of products and services that are consistent with this
        temporal agenda. Automation of these processes, if executed
        consistently with this agenda, should be considered a
        competitive advantage.

        \subidx{strategy}{exploitable}
        \subidx{exploitable}{strategy}
        \subidx{rate of revenue returns}{forecast}
        \subidx{forecast}{rate of revenue returns}
        \idx{product life cycle}
        \idx{life cycle, product}
        In some sense, this temporal agenda defines the ``average''
        product or service life cycle in the {\market}. When the
        ``forecastability'' of rate of revenue returns drops below
        50\%, there is an even chance that the rate of revenue returns
        for the product or service will change in a detrimental
        fashion. If it is assumed that a product or service life cycle
        consists of a ramp up, a maintenence interval, and a ramp
        down, then, if all three life cycle intervals are equal, the
        product life cycle will be, approximately, three times the
        time interval where the ``forecastability'' of rate of revenue
        returns drops below 50\%. Although probably not an accurate
        prediction of product or service life cycle, the technique may
        be used as a conceptual approximation to the dynamics of
        ``market windows.\footnote{For example, consider the market
        for table salt. Since it has inelastic supply and demand
        curves, and is a necessary requirement for life, it would be
        expected that the Hurst coefficient would be very near
        unity---ignoring competitive pressures in the market. The
        predictability of the table salt market would, therefore, be
        expected to be relatively good, over time.}''  The conceptual
        approximation will probably predict a ``conservative'' or
        ``pessimistic'' value in relation to actual markets.

        \begin{figure}[ht]
            \begin{center}
                \begin{minipage}[t]{0.45\textwidth}
                    \epsfxsize=1.0\linewidth
                    \epsffile{\directory/datahurstlownear.eps}
                    \caption[{\market}, ``forecastability'' of near
                        term rate of revenue returns]{{\market},
                        ``forecastability'' of near term rate of
                        revenue returns. Although the error function
                        is the most accurate, for the near term,
                        $H^{t} = \thurstlow^{t}$ may be used as a
                        reliable metric of ``forecastability'' of the
                        rate of revenue returns.}
                    \label{\SETLABEL:FN}
                \end{minipage}
                \hfill
                \begin{minipage}[t]{0.45\textwidth}
                    \epsfxsize=1.0\linewidth
                    \epsffile{\directory/datahurstlowfar.eps}
                    \caption[{\market}, ``forecastability'' of far
                        term rate of revenue returns]{{\market},
                        ``forecastability'' of far term rate of
                        revenue returns. Although the error function
                        is the most accurate, for the far term,
                        $\frac{1}{\sqrt{t}}$ may be used as a reliable
                        metric of ``forecastability'' of the rate of
                        revenue returns.}
                    \label{\SETLABEL:FF}
                \end{minipage}
            \end{center}
        \end{figure}

        \idx{operations research}
        As an interesting interpretation of the data presented in
        Figure~\ref{\SETLABEL:FN}, there may be, perhaps, some
        applicability to such operational agendas as inventory
        control. Maintaining too little inventory, obviously, will
        create a situation where the organization can not exploit
        market expansion, and maintaining too much inventory,
        likewise, would over extend the company, creating unnecessary
        losses when the market contracts. The company should maintain
        inventory levels that do not exceed, from
        Equation~\ref{\SETLABEL:MA}, ${\thurstlow}^{n} = 0.5$
        {\timescale}s of operations. Since the optimal amount of
        inventory and, from Equation~\ref{\SETLABEL:V}, the variance
        of change in the rate of revenue returns in the future can be
        calculated, there may, perhaps, be some applicability to a
        forecasting methodology that can be incorporated into other
        areas of operations research, for example the linear algebras
        using simplex methodologies for optimization of manufacturing
        processes. Traditionally, these forecasts are made by the
        sales department, and are subject to various subjective
        biases.

% Local Variables:
% TeX-parse-self: t
% TeX-auto-save: t
% TeX-master: "fractal.tex"
% End:


        %
% -----------------------------------------------------------------------------
%
% A license is hereby granted to reproduce this software source code and
% to create executable versions from this source code for personal,
% non-commercial use.  The copyright notice included with the software
% must be maintained in all copies produced.
%
% THIS PROGRAM IS PROVIDED "AS IS". THE AUTHOR PROVIDES NO WARRANTIES
% WHATSOEVER, EXPRESSED OR IMPLIED, INCLUDING WARRANTIES OF
% MERCHANTABILITY, TITLE, OR FITNESS FOR ANY PARTICULAR PURPOSE.  THE
% AUTHOR DOES NOT WARRANT THAT USE OF THIS PROGRAM DOES NOT INFRINGE THE
% INTELLECTUAL PROPERTY RIGHTS OF ANY THIRD PARTY IN ANY COUNTRY.
%
% Copyright (c) 1994-2006, John Conover, All Rights Reserved.
%
% Comments and/or bug reports should be addressed to:
%
%     john@email.johncon.com (John Conover)
%
% -----------------------------------------------------------------------------
%
% Revision: \RCSRevision \\
% Revision Time: \RCSTime UMT \\
% Revision Date: \RCSDate \\
% Revision Id: \RCSId \\
% Revision File: \RCSLog \\
\RCS $Revision: 0.0 $
\RCS $Date: 2006/01/20 04:38:13 $
\RCS $Id: simulation.tex,v 0.0 2006/01/20 04:38:13 john Exp $
% $Log: simulation.tex,v $
% Revision 0.0  2006/01/20 04:38:13  john
% Initial version
%
%
    \subsection{Simulation of Fixed Increment Approximation for Fiscal Strategy}
        \label{\SETLABEL:TSUNFAIRBROWNIAN}

        \subidx{\market}{market simulation}
        The data in this section is presented in tabular form in
        Section~\ref{\SETLABELREF:SIM}.
        Figure~\ref{\SETLABEL:TSUNFAIRBROWNIAN0} represents a
        constructional simulation of the time series data presented in
        Figure~\ref{\SETLABEL:TS}. The program {\it
        tsunfairbrownian}\/, which is briefly described in
        appendix~\ref{programs}, was used in the reconstruction. The
        reconstructed data is superimposed on the original time series
        data.  The program, {\it tsunfairbrownian}\/, essentially,
        constructs the new time series as a Brownian fractal with
        fixed increments---the value of the fixed increment is derived
        from the root mean square average of the normalized increments
        presented in Figure~\ref{\SETLABEL:TF}. The ``quality'' of
        such a reconstruction should be subject to adequate scepticism
        and scrutiny since, in all probability, the normalized
        increments presented in Figure~\ref{\SETLABEL:TF} represent a
        relatively complex process, that may not be ``modeled'' with
        such a simple methodology.

        As a further comparison of the the constructional simulation
        with the original time series data,
        Figure~\ref{\SETLABEL:TSUNFAIRBROWNIAN1} presents a normalized
        histogram of the normalized increments of the reconstructed
        time series, superimposed on the normalized histogram
        presented in Figure~\ref{\SETLABEL:NH}.

        \subidx{\market}{fiscal strategy, simulation}
        \subidx{markets}{simulation}
        \subidx{simulation}{markets}
        \subidx{strategy}{fiscal, simulation}
        \subidx{fiscal}{strategy, simulation}
        \subidx{programs}{tsunfairbrownian}
        \subidx{tsunfairbrownian}{program}
        \begin{figure}[ht]
            \begin{center}
                \begin{minipage}[t]{0.45\textwidth}
                    \epsfxsize=1.0\linewidth
                    \epsffile{\directory/tsunfairbrownian-f.eps}
                    \caption[{\market}, Time series data, empirical and
                        simulated]{{\market}, Time series data, empirical
                        and simulated, using the program {\it tsunfairbrownian}\/
                        with f = {\datafractionrms}. This data is
                        superimposed on the data presented in
                        Figure~\ref{\SETLABEL:TS}.}
                    \label{\SETLABEL:TSUNFAIRBROWNIAN0}
                \end{minipage}
                \hfill
                \begin{minipage}[t]{0.45\textwidth}
                    \epsfxsize=1.0\linewidth
                    \epsffile{\directory/tsunfairbrownian-f.tsfraction.tsnormal-s30.eps}
                    \caption[{\market}, normalized histogram,
                        empirical and simulated]{{\market}, normalized
                        histogram of the normalized increments of the
                        time series data shown in
                        Figure~\ref{\SETLABEL:TSUNFAIRBROWNIAN0},
                        empirical and simulated.  The empirical data
                        has a mean of {\datafractionmean}, with a
                        standard deviation of {\datafractionstddev}.
                        By comparison, the simulated data has a mean
                        of {\tsunfairbrownianfractionmean} with a
                        standard deviation of
                        {\tsunfairbrownianfractionstddev}. This data
                        is superimposed on the data presented in
                        Figure~\ref{\SETLABEL:NH}. The area under the
                        four curves is identical.}
                    \label{\SETLABEL:TSUNFAIRBROWNIAN1}
                \end{minipage}
            \end{center}
        \end{figure}

% Local Variables:
% TeX-parse-self: t
% TeX-auto-save: t
% TeX-master: "fractal.tex"
% End:


        %
% -----------------------------------------------------------------------------
%
% A license is hereby granted to reproduce this software source code and
% to create executable versions from this source code for personal,
% non-commercial use.  The copyright notice included with the software
% must be maintained in all copies produced.
%
% THIS PROGRAM IS PROVIDED "AS IS". THE AUTHOR PROVIDES NO WARRANTIES
% WHATSOEVER, EXPRESSED OR IMPLIED, INCLUDING WARRANTIES OF
% MERCHANTABILITY, TITLE, OR FITNESS FOR ANY PARTICULAR PURPOSE.  THE
% AUTHOR DOES NOT WARRANT THAT USE OF THIS PROGRAM DOES NOT INFRINGE THE
% INTELLECTUAL PROPERTY RIGHTS OF ANY THIRD PARTY IN ANY COUNTRY.
%
% Copyright (c) 1994-2006, John Conover, All Rights Reserved.
%
% Comments and/or bug reports should be addressed to:
%
%     john@email.johncon.com (John Conover)
%
% -----------------------------------------------------------------------------
%
% Revision: \RCSRevision \\
% Revision Time: \RCSTime UMT \\
% Revision Date: \RCSDate \\
% Revision Id: \RCSId \\
% Revision File: \RCSLog \\
\RCS $Revision: 0.0 $
\RCS $Date: 2006/01/20 04:38:13 $
\RCS $Id: maximum.tex,v 0.0 2006/01/20 04:38:13 john Exp $
% $Log: maximum.tex,v $
% Revision 0.0  2006/01/20 04:38:13  john
% Initial version
%
%
    \subsection{Simulation of Fixed Increment Approximation for Optimally Maximal Fiscal Strategy}
        \label{\SETLABEL:MAXSHANNON}
        \subidx{\market}{fiscal strategy, simulation}
        \subidx{\market}{maximum Shannon probability}
        \subidx{markets}{simulation}
        \subidx{simulation}{markets}
        \subidx{strategy}{optimum fiscal, simulation}
        \subidx{fiscal}{optimum strategy, simulation}
        \subidx{programs}{tsunfairbrownian}
        \subidx{tsunfairbrownian}{program}
        \subidx{Shannon}{probability}
        \subidx{probability}{Shannon}

        \subidx{strategy}{exploitable}
        \subidx{exploitable}{strategy}
        \subidx{programs}{tsshannonmax}
        \subidx{tsshannonmax}{program}
        \subidx{programs}{tsunfairbrownian}
        \subidx{tsunfairbrownian}{program}
        \subidx{strategy}{fiscal}
        \subidx{fiscal}{strategy}
        The data in this section is presented in tabular form in
        Section~\ref{\SETLABELREF:MAXSHANNON}. One of the issues of
        analysis, as mentioned in Section~\ref{\SETLABEL:OPS}, is to
        determine the maximum Shannon probability for the time series
        presented in Figure~\ref{\SETLABEL:TS}. Potentially, this
        could be exploited with an aggressive fiscal
        strategy. Figure~\ref{\SETLABEL:SHANNONMAX0} is a graph of the
        output of the {\it tsshannonmax}\/ program, which is described
        briefly in appendix~\ref{programs}. The maximum of this
        function is the maximum Shannon probability for the time
        series data presented in Figure~\ref{\SETLABEL:TS}.
        Figure~\ref{\SETLABEL:SHANNONMAX1} was constructed using {\it
        tsunfairbrownian}\/ program, which is also described in
        appendix~\ref{programs}, with the maximum Shannon probability,
        and the time series data presented in
        Figure~\ref{\SETLABEL:TS}. This represents a ``what if'' the
        investment strategy was changed from a Shannon probability of
        {\shannonlogreturns}, as derived in Section~\ref{\SETLABEL:SP}
        to {\shannonmax}. This process, essentially, extracts the
        random statistical data from the time series presented in
        Figure~\ref{\SETLABEL:TS}, and constructs a new time series,
        using the random statistical data, with a different investment
        strategy.  The program, {\it tsunfairbrownian}\/, essentially,
        constructs the new time series as a Brownian fractal with
        fixed increments.  The ``quality'' of such a reconstruction
        should be subject to adequate scepticism and scrutiny since,
        in all probability, the increments in the original data
        represent a relatively complex process, that may not be
        ``modeled'' with such a simple methodology.

        \begin{figure}[ht]
            \begin{center}
                \begin{minipage}[t]{0.45\textwidth}
                    \epsfxsize=1.0\linewidth
                    \epsffile{\directory/data.tsshannonmax.eps}
                    \caption[{\market}, maximum rate of revenue
                        returns] {{\market}, maximum rate of revenue
                        returns, per {\timescale}, vs. Shannon
                        probability. The maximum rate of revenue
                        returns, per {\timescale}, occurs at a Shannon
                        probability of {\shannonmax}.}
                    \label{\SETLABEL:SHANNONMAX0}
                \end{minipage}
                \hfill
                \begin{minipage}[t]{0.45\textwidth}
                    \epsfxsize=1.0\linewidth
                    \epsffile{\directory/data.tsshannonmax-p.tsunfairbrownian-p.eps}
                    \caption[{\market}, maximum rate of revenue
                        returns] {{\market}, maximum rate of revenue
                        returns, per {\timescale}, at a Shannon
                        probability, of {\shannonmax}, corresponding
                        to a ``wager'' fraction of {\twoponemax}.}
                    \label{\SETLABEL:SHANNONMAX1}
                \end{minipage}
            \end{center}
        \end{figure}

        \subidx{fractional}{Brownian motion}
        \subidx{Brownian motion}{fractional}
        \subidx{Shannon}{probability}
        \subidx{probability}{Shannon}
        \subidx{programs}{tsshannonmax}
        \subidx{tsshannonmax}{program}
        If it is assumed that the time series data set, presented in
        Figure~\ref{\SETLABEL:TS}, constitutes classical Brownian
        motion, then the Shannon probability can be calculated by
        counting the total number of {\timescale}s that the {\market}
        movement was positive, and dividing by the total number of
        {timescale}s represented in the time series. This quotient is
        {\pmax}, as compared with the predicted value from the program
        {\it tsshannonmax}\/ of {\shannonmax}.

% Local Variables:
% TeX-parse-self: t
% TeX-auto-save: t
% TeX-master: "fractal.tex"
% End:


        %
% -----------------------------------------------------------------------------
%
% A license is hereby granted to reproduce this software source code and
% to create executable versions from this source code for personal,
% non-commercial use.  The copyright notice included with the software
% must be maintained in all copies produced.
%
% THIS PROGRAM IS PROVIDED "AS IS". THE AUTHOR PROVIDES NO WARRANTIES
% WHATSOEVER, EXPRESSED OR IMPLIED, INCLUDING WARRANTIES OF
% MERCHANTABILITY, TITLE, OR FITNESS FOR ANY PARTICULAR PURPOSE.  THE
% AUTHOR DOES NOT WARRANT THAT USE OF THIS PROGRAM DOES NOT INFRINGE THE
% INTELLECTUAL PROPERTY RIGHTS OF ANY THIRD PARTY IN ANY COUNTRY.
%
% Copyright (c) 1994-2006, John Conover, All Rights Reserved.
%
% Comments and/or bug reports should be addressed to:
%
%     john@email.johncon.com (John Conover)
%
% -----------------------------------------------------------------------------
%
% Revision: \RCSRevision \\
% Revision Time: \RCSTime UMT \\
% Revision Date: \RCSDate \\
% Revision Id: \RCSId \\
% Revision File: \RCSLog \\
\RCS $Revision: 0.0 $
\RCS $Date: 2006/01/20 04:38:13 $
\RCS $Id: verification.tex,v 0.0 2006/01/20 04:38:13 john Exp $
% $Log: verification.tex,v $
% Revision 0.0  2006/01/20 04:38:13  john
% Initial version
%
%
    \subsection{Qualitative Verification of Fixed Increment Approximation Analysis}
        \label{\SETLABEL:QVA}

        \subidx{\market}{verification of analysis}
        \subidx{verification}{analysis}
        \subidx{analysis}{verification}
        \subidx{quality}{of analysis}
        \subidx{verification}{of methodology}
        \subidx{methodology}{verification of}
        \subidx{Shannon}{probability}
        \subidx{probability}{Shannon}

        This section evaluates various values based on the ``average''
        of the normalized increments presented in
        Figure~\ref{\SETLABEL:TFA}. These values are an approximation
        to a, probably, complex process with a distribution shown in
        Figure~\ref{\SETLABEL:TF}. These values will be used in a
        fixed increment Brownian fractal analysis of the {\market},
        and may, or may not, provide adequate accuracy for
        projections.

        The data in this section is presented in tabular form in
        sections~\ref{\SETLABELREF:VI1} and~\ref{\SETLABELREF:VI2}.
        As a subjective evaluation of the ``quality'' of the analysis
        of the {\market}, from Chapter~\ref{methodology},
        Equation~\ref{metricvalues1}, and using the mean and root mean
        square values of the normalized increments of the time series
        data presented in Figure~\ref{\SETLABEL:TS} from
        Figure~\ref{\SETLABEL:TF}, and the Shannon probability as
        calculated by counting the total number of {\timescale}s that
        the {\market} movement was positive, as presented in
        Section~\ref{\SETLABEL:MAXSHANNON}:

        \begin{eqnarray}
                  P & \approx & \frac{\frac{avg}{rms} + 1}{2}\\
            {\pmax} & \approx & \frac{\frac{\datafractionmean}{\datafractionrms} + 1}{2}\\
            {\pmax} & \approx & {\avgrms}
            \label{\SETLABEL:AVGS}
        \end{eqnarray}

        \subidx{Shannon}{probability}
        \subidx{probability}{Shannon}
        \noindent and comparing these values to the Shannon
        probability, as found by the {\it tsshannonmax}\/ program, which
        iterates for a maximum:

        \begin{eqnarray}
            {\pmax} \approx {\avgrms} \approx {\shannonmax}
        \end{eqnarray}

        \subidx{logarithmic}{returns}
        \subidx{returns}{logarithmic}
        In addition, the different methods of calculating the
        logarithmic returns, presented in Section~\ref{\SETLABEL:FS},
        should be compared. The four methods used were the mean of
        Figure~\ref{\SETLABEL:TF}, the constant in the least squares
        approximation to Figure~\ref{\SETLABEL:TF}, the least squares
        exponential approximation to Figure~\ref{\SETLABEL:TS}, and
        the logarithmic returns of Figure~\ref{\SETLABEL:TS}, derived
        as the mean of the logarithms of the quotients of the
        increments. The values for each of the methods are,
        respectively:

        \begin{equation}
            \datafractionmeanbits \approx \datafractionconstantbits \approx \datatslsqepbits \approx \logreturns
        \end{equation}

        It is implied in Section~\ref{\SETLABEL:FS},
        Subsection~\ref{\SETLABEL:SP} and in
        Section~\ref{\SETLABEL:TSUNFAIRBROWNIAN} that, a Brownian
        motion with fixed increments fractal may ``model'' the
        {\market}. Using Equation~\ref{stddev9} from
        Chapter~\ref{general}, Section~\ref{abmfi}:

        \begin{eqnarray}
                                    rms \left(2P - 1\right) & \approx & \frac{\sigma \left(2P - 1\right)}{2 \sqrt{P\left(1 - P\right)}}\\
            \datafractionrms \left(2 \cdot \pmax - 1\right) & \approx & \frac{\datafractionstddev \left(2 \cdot \pmax - 1\right)}{2\sqrt{\pmax \left(1 - \pmax\right)}}\\
                       \datafractionrms \cdot \twopminusone & \approx & \datafractionstddev \cdot \twopx\\
                                                      \rmsp & \approx & \sigmap
        \end{eqnarray}

        \noindent and, equating to the mean:

        \begin{equation}
            \datafractionmean \approx \rmsp \approx \sigmap
        \end{equation}

        \subidx{Shannon}{probability}
        \subidx{probability}{Shannon}
        \noindent where, as in Equation~\ref{\SETLABEL:AVGS} using the
        mean, root mean square, and standard deviation values of the
        normalized increments of the time series data presented in
        Figure~\ref{\SETLABEL:TS} from Figure~\ref{\SETLABEL:TF}, and
        the Shannon probability as calculated by counting the total
        number of {\timescale}s that the {\market} movement was
        positive, as presented in Section~\ref{\SETLABEL:MAXSHANNON}.

        As a final qualitative comparison, the absolute value of the
        normalized increments should be the same as the root mean
        square value\footnote{The absolute value of the normalized
        increments, when averaged, is related to the root mean square
        of the increments by a constant. If the normalized increments
        are a fixed increment, the constant is unity. If the
        normalized increments have a Gaussian distribution, the
        constant is $\approx 0.8$ depending on the accuracy of of
        ``fit'' to a Gaussian distribution.}, where the absolute value
        is presented in Figure~\ref{\SETLABEL:TFA}, and the root mean
        square value is presented in Figure~\ref{\SETLABEL:TF}:

        \begin{equation}
            \datafractionabsmean \approx \datafractionrms
        \end{equation}

        Note, that if the {\market} could be ``modeled'' as a Brownian
        motion with fixed increments fractal, then the standard
        deviation of the absolute value of the normalized increments
        of the time series data presented in Figure~\ref{\SETLABEL:TS}
        from Figure~\ref{\SETLABEL:TF} should be zero. It is
        $\datafractionabsstddev$.

% Local Variables:
% TeX-parse-self: t
% TeX-auto-save: t
% TeX-master: "fractal.tex"
% End:


    \renewcommand{\market}{Cirrus Logic Stock}
    \renewcommand{\directory}{../markets/crus}
    \renewcommand{\datafractionmean}{0.008052}
\renewcommand{\datafractionmeanbits}{0.011570}
\renewcommand{\datafractionmeanq}{0.002684}
\renewcommand{\datafractionmeanbitsq}{0.003867}
\renewcommand{\datafractionstddev}{0.038579}
\renewcommand{\datafractionrms}{0.039311}
\renewcommand{\avgrms}{0.602414}
\renewcommand{\ncompanies}{5.210454}
\renewcommand{\pncompanies}{0.544866}
\renewcommand{\datafractionabsmean}{0.029745}
\renewcommand{\datafractionabsstddev}{0.025769}
\renewcommand{\datafractionconstant}{0.010041}
\renewcommand{\datafractionconstantbits}{0.014414}
\renewcommand{\datafractionconstantq}{0.003347}
\renewcommand{\datafractionconstantbitsq}{0.004821}
\renewcommand{\datafractionslope}{-0.000021}
\renewcommand{\datafractionabsconstant}{0.035145}
\renewcommand{\datafractionabsslope}{-0.000057}
\renewcommand{\hurstall}{0.659558}
\renewcommand{\hurstlow}{0.707509}
\renewcommand{\hurstlowtwo}{1.415018}
\renewcommand{\hurstlowhundred}{70.750900}
\renewcommand{\hcalcall}{0.184942}
\renewcommand{\hcalclow}{0.102042}
\renewcommand{\shannonmax}{0.604167}
\renewcommand{\twoponemax}{0.208334}
\renewcommand{\logreturns}{0.010456}
\renewcommand{\twologreturns}{1.007274}
\renewcommand{\twologreturnshundred}{0.727387}
\renewcommand{\oneoverlogreturns}{95.638868}
\renewcommand{\pmax}{0.602094}
\renewcommand{\twopminusone}{0.204188}
\renewcommand{\rmsp}{0.008027}
\renewcommand{\twopx}{0.208583}
\renewcommand{\sigmap}{0.008047}
\renewcommand{\tsunfairbrownianfractionmean}{0.007862}
\renewcommand{\tsunfairbrownianfractionstddev}{0.038619}
\renewcommand{\shannonlogreturns}{0.560125}
\renewcommand{\shannonlogreturnshundred}{56.012500}
\renewcommand{\twopone}{0.120250}
\renewcommand{\twoponehundred}{12.025000}
\renewcommand{\hundredtwoponehundred}{87.975000}
\renewcommand{\hundredshannonlogreturnshundred}{43.987500}
\renewcommand{\datatslsqepbits}{0.007623}
\renewcommand{\thurstall}{0.633980}
\renewcommand{\thurstlow}{0.710108}
\renewcommand{\thurstlowtwo}{1.420216}
\renewcommand{\thurstlowhundred}{71.010800}
\renewcommand{\thcalcall}{0.247886}
\renewcommand{\thcalclow}{0.171737}
\renewcommand{\chisquared}{2.862000}
\renewcommand{\critical}{42.557000}

    \renewcommand{\timescale}{day}
    \subidx{market}{\market}
    \idx{\market}

    \section{\market}

        \renewcommand{\SETLABEL}{\LABPRE:CRUS}
        \renewcommand{\SETLABELQ}{\LABPRE:CRUSQ}
        \label{\SETLABEL}
        \renewcommand{\SETLABELREF}{\LABPREREF:CRUS}

        \subidx{crus}{program}
        \subidx{programs}{crus}
        For the analysis, the data was in the directory
        {\directory}\footnote{Cirrus Logic stock price, November 15,
        1994, through April 8, 1996, inclusive.  The data is by
        {\timescale}s.}.

        The data in this section is presented in tabular form in
        Section~\ref{\SETLABELREF}. Note that in this analysis, the
        rate of revenue returns means the increase or decrease in the
        cumulative sum of the {\market}. This is included for
        ``theoretical'' comparative purposes, and has no meaning,
        unless it is considered as a ``future.''

        %
% -----------------------------------------------------------------------------
%
% A license is hereby granted to reproduce this software source code and
% to create executable versions from this source code for personal,
% non-commercial use.  The copyright notice included with the software
% must be maintained in all copies produced.
%
% THIS PROGRAM IS PROVIDED "AS IS". THE AUTHOR PROVIDES NO WARRANTIES
% WHATSOEVER, EXPRESSED OR IMPLIED, INCLUDING WARRANTIES OF
% MERCHANTABILITY, TITLE, OR FITNESS FOR ANY PARTICULAR PURPOSE.  THE
% AUTHOR DOES NOT WARRANT THAT USE OF THIS PROGRAM DOES NOT INFRINGE THE
% INTELLECTUAL PROPERTY RIGHTS OF ANY THIRD PARTY IN ANY COUNTRY.
%
% Copyright (c) 1994-2006, John Conover, All Rights Reserved.
%
% Comments and/or bug reports should be addressed to:
%
%     john@email.johncon.com (John Conover)
%
% -----------------------------------------------------------------------------
%
% Revision: \RCSRevision \\
% Revision Time: \RCSTime UMT \\
% Revision Date: \RCSDate \\
% Revision Id: \RCSId \\
% Revision File: \RCSLog \\
\RCS $Revision: 0.0 $
\RCS $Date: 2006/01/20 04:38:13 $
\RCS $Id: fraction.tex,v 0.0 2006/01/20 04:38:13 john Exp $
% $Log: fraction.tex,v $
% Revision 0.0  2006/01/20 04:38:13  john
% Initial version
%
%
    \subsection{Time Series Increments Analysis}
        \label{\SETLABEL:TSA}

        \subidx{\market}{Time series analysis}
        \subidx{time series}{increments}
        \subidx{time series}{analysis}
        \subidx{cumulative sum}{analysis}
        \subidx{analysis}{cumulative sum}
        \subidx{analysis}{random process}
        \subidx{random process}{analysis}
        \subidx{Gaussian}{increments}
        \subidx{increments}{Gaussian}
        \subidx{Brownian}{motion, fractional}
        \subidx{fractional}{Brownian motion}
        \subidx{fractal}{Brownian motion}
        The data in this section is presented in tabular form in
        Section~\ref{\SETLABELREF:TSA}.  Figure~\ref{\SETLABEL:TS} is
        a graph of the time series data for the {\market}.

        \subidx{increments}{normalized}
        \subidx{normalized}{increments}
        \subidx{programs}{tsfraction}
        \subidx{tsfraction}{program}
        Figure~\ref{\SETLABEL:TF} is a graph of the normalized
        increments of the time series data presented in
        Figure~\ref{\SETLABEL:TS}. The data presented was made by
        running the program {\it tsfraction}\/ on the time series
        data. The program {\it tsfraction}\/ is described briefly in
        Appendix~\ref{programs}, and subtracts the previous value from
        the next value, dividing this difference by the previous
        value, for each element in the time series data. The new time
        series contains the instantaneous change in the rate of
        revenue returns, divided by the magnitude of the instantaneous
        rate of revenue returns.

        \subidx{mean}{standard deviation}
        \subidx{standard deviation}{mean}
        \idx{root mean square}
        \idx{least squares approximation}
        \begin{figure}[ht]
            \begin{center}
                \begin{minipage}[t]{0.45\textwidth}
                    \epsfxsize=1.0\linewidth
                    \epsffile{\directory/data.eps}
                    \caption{{\market}, time series data.}
                    \label{\SETLABEL:TS}
                    \label{\SETLABELQ:TS}
                \end{minipage}
                \hfill
                \begin{minipage}[t]{0.45\textwidth}
                    \epsfxsize=1.0\linewidth
                    \epsffile{\directory/data.tsfraction.eps}
                    \caption[{\market}, normalized
                        increments]{{\market}, normalized increments
                        of the time series data presented in
                        Figure~\ref{\SETLABEL:TS}. The mean is
                        {\datafractionmean} with a standard deviation
                        of {\datafractionstddev}. The formula for the
                        least squares approximation is
                        ${\datafractionconstant} +
                        {\datafractionslope}t$, and the root mean
                        squared value is {\datafractionrms}. The
                        graph, labeled ``data\-.tsfraction\-.tsrms,''
                        is the running root mean square, and
                        ``data\-.tsfraction\-.tsavg'' is the running
                        average of the normalized increments.  This
                        graph is the fraction of change in the time
                        series, as a function of time. Note that the
                        slope of the mean, {\datafractionslope}, is
                        the coefficient of the nonlinearity term in
                        the normalized increments. See
                        Chapter~\ref{general}, Section~\ref{nlextend}
                        for a possible application of the logistic
                        function to this data set.}
                    \label{\SETLABEL:TF}
                    \label{\SETLABELQ:TF}
                \end{minipage}
            \end{center}
        \end{figure}

        \subidx{absolute value}{increments}
        \subidx{increments}{absolute value}

        Figure~\ref{\SETLABEL:TFA} is a graph of the absolute value of
        the normalized increments of the time series data presented in
        Figure~\ref{\SETLABEL:TF}. The data presented was made by
        running the Unix utility sed(1) on the normalized increments
        time series data to remove the negative signs. This is an
        absolute value procedure.  The resulting time series contains
        the absolute value of the instantaneous change in the rate of
        revenue returns, divided by the magnitude of the instantaneous
        rate of revenue returns\footnote{The absolute value of the
        normalized increments, when averaged, is related to the root
        mean square of the increments by a constant. If the normalized
        increments are a fixed increment, the constant is unity. If
        the normalized increments have a Gaussian distribution, the
        constant is $\approx 0.8$ depending on the accuracy of of
        ``fit'' to a Gaussian distribution.}.

        \subidx{histogram}{normalized}
        \subidx{normalized}{histogram}
        \subidx{programs}{tsnormal}
        \subidx{tsnormal}{program}
        \subidx{mean}{standard deviation}
        \subidx{standard deviation}{mean}
        \idx{root mean square}
        \idx{least squares approximation}
        \subidx{\market}{analysis of increments}
        Figure~\ref{\SETLABEL:NH} is the normalized histogram of the
        normalized increments of the time series data shown in
        Figure~\ref{\SETLABEL:TF}. The abscissa is 3 $\sigma$ limits,
        and the area under the two curves is identical. The data for
        this figure was produced by the program {\it tsnormal}\/,
        which is described briefly in Appendix~\ref{programs}.

        \begin{figure}[ht]
            \begin{center}
                \begin{minipage}[t]{0.45\textwidth}
                    \epsfxsize=1.0\linewidth
                    \epsffile{\directory/data.tsfraction.abs.eps}
                    \caption[{\market}, absolute value of the
                        normalized increments]{{\market}, absolute
                        value of the normalized increments of the time
                        series data presented in
                        Figure~\ref{\SETLABEL:TF}.  The mean is
                        {\datafractionabsmean} with a standard
                        deviation of {\datafractionabsstddev}. The
                        formula for the least squares approximation is
                        ${\datafractionabsconstant} +
                        {\datafractionabsslope}t$, and the root mean
                        square value, from Figure~\ref{\SETLABEL:TF},
                        is {\datafractionrms}.  The graph, labeled
                        ``data\-.tsfraction\-.tsrms,'' is the running
                        root mean square, and
                        ``data\-.tsfraction\-.tsavg'' is the running
                        average of the normalized increments presented
                        in Figure~\ref{\SETLABEL:TF}, superimposed
                        here for convenience. This graph is the
                        absolute value of the fraction of change in
                        the time series, as a function of time.}
                    \label{\SETLABEL:TFA}
                    \label{\SETLABELQ:TFA}
                \end{minipage}
                \hfill
                \begin{minipage}[t]{0.45\textwidth}
                    \epsfxsize=1.0\linewidth
                    \epsffile{\directory/data.tsfraction.tsnormal-s30.eps}
                    \caption[{\market}, normalized histogram of the
                        normalized increments]{{\market}, normalized
                        histogram of the normalized increments of the
                        time series data shown in
                        Figure~\ref{\SETLABEL:TF}.  The data has a
                        mean of {\datafractionmean}, with a standard
                        deviation of {\datafractionstddev}.  The area
                        under the two curves is identical. The
                        $\chi^2$ value of the observed and expected
                        values of the two curves is {\chisquared},
                        with a critical value of {\critical}.}
                    \label{\SETLABEL:NH}
                \end{minipage}
            \end{center}
        \end{figure}

        \subidx{programs}{tsXsquared}
        \subidx{tsXsquared}{program}
        \subidx{\market}{chi-squared values of increments}
        The program {\it tsXsquared}\/, which is briefly described in
        appendix~\ref{programs}, was used to derive the $\chi^2$
        statistics for the data presented in
        Figure~\ref{\SETLABEL:NH}.

        \subidx{programs}{tsstatest}
        \subidx{tsstatest}{program}
        \subidx{\market}{statistical estimates}

        Figure~\ref{\SETLABEL:SE} is the statistical estimate for the
        data presented in Figure~\ref{\SETLABEL:TF}, as derived by the
        program {\it tsstatest}\/, which is briefly described in
        appendix~\ref{programs}.

        \begin{figure}[ht]
            \begin{center}
                \begin{minipage}[t]{\textwidth}
                    \center{\fbox{\parbox{0.9\textwidth}{\XXX{\directory/data.tsstatest-f0.1-c0.9-i.tex}}}}
                    \caption[{\market}, statistical estimates of the
                        normalized increments]{{\market}, statistical
                        estimates of the normalized increments of the
                        time series shown in Figure~\ref{\SETLABEL:TF}.
                        The table was produced with the {\it
                        tsstatest}\/ program, and illustrates the
                        size of the data set required for a confidence
                        level of 90\%, with an error estimate of $\pm$
                        10\%, or alternately, the error estimate on
                        the time series shown in Figure~\ref{\SETLABEL:TF}.}
                    \label{\SETLABEL:SE}
                \end{minipage}
            \end{center}
        \end{figure}

        Note that the data set size estimations, as produced by the
        {\it tsstatest}\/ program, are probably very conservative,
        depending on the magnitude of the Shannon probability, $P =
        \shannonlogreturns$, as derived in
        Section~\ref{\SETLABEL:SP}. See Chapter~\ref{general},
        Section~\ref{serdss} for possible alternative methodologies
        for addressing the analysis of fractal time series with
        limited data set sizes. Depending on the magnitude of the
        Shannon probability, $P$, these estimates can be several
        orders of magnitude too high.

        \subidx{derivative of increments}{normalized}
        \subidx{normalized}{derivative of increments}
        \subidx{programs}{tsderivative}
        \subidx{tsderivative}{program}
        Figure~\ref{\SETLABEL:TF1} is the normalized histogram of the
        first derivative of the normalized increments of the time
        series data shown in Figure~\ref{\SETLABEL:TF}. In principle,
        if the distribution of the normalized increments presented in
        Figure~\ref{\SETLABEL:NH} is Gaussian in nature, this
        distribution would be similar to ``white noise,'' as presented
        in appendix~\ref{programs}, Figure~\ref{whiteexample}. The
        data was generated by the {\it tsderivative}\/ program, which
        is briefly described in
        appendix~\ref{programs}. Figure~\ref{\SETLABEL:TF2} is the
        normalized histogram of the second derivative of the
        normalized increments of the time series data shown in
        Figure~\ref{\SETLABEL:TF}. In principle, if the distribution
        of the normalized increments presented in
        Figure~\ref{\SETLABEL:NH} is an integrated Gaussian
        distribution in nature, this distribution would be similar to
        ``white noise,'' as presented in appendix~\ref{programs},
        Figure~\ref{whiteexample}.

        \begin{figure}[ht]
            \begin{center}
                \begin{minipage}[t]{0.45\textwidth}
                    \epsfxsize=1.0\linewidth
                    \epsffile{\directory/data.tsfraction.tsderivative.tsnormal-s30.eps}
                    \caption[{\market}, histogram of the first
                        derivative of the increments]{{\market},
                        normalized histogram of the first derivative
                        of the normalized increments of the time
                        series data shown in
                        Figure~\ref{\SETLABEL:TF}.}
                    \label{\SETLABEL:TF1}
                \end{minipage}
                \hfill
                \begin{minipage}[t]{0.45\textwidth}
                    \epsfxsize=1.0\linewidth
                    \epsffile{\directory/data.tsfraction.2tsderivative.tsnormal-s30.eps}
                    \caption[{\market}, histogram of the second
                        derivative of the increments]{{\market},
                        normalized histogram of second derivative of
                        the the normalized increments of the time
                        series data shown in
                        Figure~\ref{\SETLABEL:TF}.}
                    \label{\SETLABEL:TF2}
                \end{minipage}
            \end{center}
        \end{figure}

        \subidx{fractal}{range}
        \subidx{fractal}{R/S analysis}
        \subidx{\market}{rate of revenue returns, range}
        \subidx{\market}{deterministic mechanism}
        \subidx{deterministic}{mechanism}
        \subidx{mechanism}{deterministic}
        Figure~\ref{\SETLABEL:TR} is the range of values of the time
        series shown in Figure~\ref{\SETLABEL:TS}. The horizontal axis
        is time into the future. In principle, if the time series was
        characterized as fractional Brownian motion the graph in
        Figure~\ref{\SETLABEL:TR} would be a square root
        function\footnote{Note that the ``roughness,'' or ``sawtooth''
        characteristics of the graph in Figure~\ref{\SETLABEL:TR} are
        a computational artifact---caused by not using the -m option
        to the program {\it tshurst}\/, which is computationally
        inefficient.}. Figure~\ref{\SETLABEL:TD} is the deterministic
        map of the normalized increments of the time series data shown
        in Figure~\ref{\SETLABEL:TF}. The deterministic map is useful
        for determining if a time series was created by a
        deterministic mechanism. This, essentially, maps each element
        in the time series with the previous element in the time
        series.  See,~\cite[pp. 745]{Peitgen}.

        \begin{figure}[ht]
            \begin{center}
                \begin{minipage}[t]{0.45\textwidth}
                    \epsfxsize=1.0\linewidth
                    \epsffile{\directory/data.tshurst-f.eps}
                    \caption[{\market}, range]{{\market}, range of the
                        time series data shown in
                        Figure~\ref{\SETLABEL:TS}.}
                    \label{\SETLABEL:TR}
                \end{minipage}
                \hfill
                \begin{minipage}[t]{0.45\textwidth}
                    \epsfxsize=1.0\linewidth
                    \epsffile{\directory/data.tsfraction.tsdeterministic.eps}
                    \caption[{\market}, deterministic map]{{\market},
                        deterministic map of the normalized increments
                        of the time series data shown in
                        Figure~\ref{\SETLABEL:TF}.}
                    \label{\SETLABEL:TD}
                \end{minipage}
            \end{center}
        \end{figure}

% Local Variables:
% TeX-parse-self: t
% TeX-auto-save: t
% TeX-master: "fractal.tex"
% End:


            Figure~\ref{\SETLABEL:NH} would seem to indicate that the
            time series data for the {\market} represents a cumulative
            sum/integration of a random process that has a Gaussian
            distribution, (ie., satisfies the Gaussian increments
            property of fractional Brownian
            motion~\cite[pp. 250]{Crownover},) tending to justify the
            assumption that the time series data represents fractional
            Brownian motion.

        %
% -----------------------------------------------------------------------------
%
% A license is hereby granted to reproduce this software source code and
% to create executable versions from this source code for personal,
% non-commercial use.  The copyright notice included with the software
% must be maintained in all copies produced.
%
% THIS PROGRAM IS PROVIDED "AS IS". THE AUTHOR PROVIDES NO WARRANTIES
% WHATSOEVER, EXPRESSED OR IMPLIED, INCLUDING WARRANTIES OF
% MERCHANTABILITY, TITLE, OR FITNESS FOR ANY PARTICULAR PURPOSE.  THE
% AUTHOR DOES NOT WARRANT THAT USE OF THIS PROGRAM DOES NOT INFRINGE THE
% INTELLECTUAL PROPERTY RIGHTS OF ANY THIRD PARTY IN ANY COUNTRY.
%
% Copyright (c) 1994-2006, John Conover, All Rights Reserved.
%
% Comments and/or bug reports should be addressed to:
%
%     john@email.johncon.com (John Conover)
%
% -----------------------------------------------------------------------------
%
% Revision: \RCSRevision \\
% Revision Time: \RCSTime UMT \\
% Revision Date: \RCSDate \\
% Revision Id: \RCSId \\
% Revision File: \RCSLog \\
\RCS $Revision: 0.0 $
\RCS $Date: 2006/01/20 04:38:13 $
\RCS $Id: instant.tex,v 0.0 2006/01/20 04:38:13 john Exp $
% $Log: instant.tex,v $
% Revision 0.0  2006/01/20 04:38:13  john
% Initial version
%
%
    \subsection{Instantaneous Analysis of Normalized Increments}
        \label{\SETLABEL:IA}

        \subidx{\market}{instantaneous analysis of normalized increments}
        \idx{average of normalized increments}
        \idx{root mean square of normalized increments}
        \subidx{Shannon probability}{instantaneous computation of}
        \subidx{average of normalized increments}{instantaneous computation of}
        \subidx{root mean square of normalized increments}{instantaneous computation of}
        \subidx{instantaneous computation}{Shannon probability}
        \subidx{instantaneous computation}{average of normalized increments}
        \subidx{instantaneous computation}{root mean square of normalized increments}
        \idx{time series}
        \subidx{time series}{instantaneous analysis}
        \subidx{instantaneous analysis}{time series}
        \subidx{time series}{increments}
        \subidx{time series}{analysis}
        \subidx{Shannon}{probability}
        \subidx{probability}{Shannon}
        \subidx{normalized}{increments}
        \subidx{increments}{normalized}

        The program {\it tsinstant}\/, which is briefly described in
        Appendix~\ref{programs}, is for finding the instantaneous
        fraction of change in a time series. The value of a sample in
        the time series is subtracted from the previous sample in the
        time series, and divided by the value of the previous sample.
        As explained in Chapter~\ref{general},
        Sections~\ref{derivation},~\ref{GA},~\ref{abmfi},~\ref{aftsma}
        and,~\ref{ompl} for Brownian motion, random walk fractals, the
        absolute value of the instantaneous fraction of change is also
        the root mean square of the instantaneous fraction of
        change\footnote{The absolute value of the normalized
        increments, when averaged, is related to the root mean square
        of the increments by a constant. If the normalized increments
        are a fixed increment, the constant is unity. If the
        normalized increments have a Gaussian distribution, the
        constant is $\approx 0.8$ depending on the accuracy of of
        ``fit'' to a Gaussian distribution.}. Squaring this value is
        the average of the instantaneous fraction of change, and
        adding unity to the absolute value of the instantaneous
        fraction of change, and dividing by two, is the Shannon
        probability of the instantaneous fraction of change.

        Figure~\ref{\SETLABEL:IA1} is the instantaneous value of the
        root mean square of the normalized increments for the
        {\market}, and Figure~\ref{\SETLABEL:IA2} is the instantaneous
        Shannon probability for the normalized increments.

        \begin{figure}[ht]
            \begin{center}
                \begin{minipage}[t]{0.45\textwidth}
                    \epsfxsize=1.0\linewidth
                    \epsffile{\directory/data.tsinstant-r.eps}
                    \caption[{\market}, instantaneous value of
                        rms.]{{\market}, instantaneous value of the
                        root mean square of the normalized increments,
                        provided by running the program {\it
                        tsinstant}\/ with the -r option on the data
                        presented in Figure~\ref{\SETLABEL:TS}.}
                    \label{\SETLABEL:IA1}
                    \label{\SETLABELQ:IA1}
                \end{minipage}
                \hfill
                \begin{minipage}[t]{0.45\textwidth}
                    \epsfxsize=1.0\linewidth
                    \epsffile{\directory/data.tsinstant-s.eps}
                    \caption[{\market}, instantaneous value of
                        Shannon probability.]{{\market}, instantaneous
                        value of the Shannon probability of the
                        normalized increments, provided by running the
                        program {\it tsinstant}\/ with the -s option
                        on the data presented in
                        Figure~\ref{\SETLABEL:TS}.}
                    \label{\SETLABEL:IA2}
                    \label{\SETLABELQ:IA2}
                \end{minipage}
            \end{center}
        \end{figure}

% Local Variables:
% TeX-parse-self: t
% TeX-auto-save: t
% TeX-master: "fractal.tex"
% End:


        %
% -----------------------------------------------------------------------------
%
% A license is hereby granted to reproduce this software source code and
% to create executable versions from this source code for personal,
% non-commercial use.  The copyright notice included with the software
% must be maintained in all copies produced.
%
% THIS PROGRAM IS PROVIDED "AS IS". THE AUTHOR PROVIDES NO WARRANTIES
% WHATSOEVER, EXPRESSED OR IMPLIED, INCLUDING WARRANTIES OF
% MERCHANTABILITY, TITLE, OR FITNESS FOR ANY PARTICULAR PURPOSE.  THE
% AUTHOR DOES NOT WARRANT THAT USE OF THIS PROGRAM DOES NOT INFRINGE THE
% INTELLECTUAL PROPERTY RIGHTS OF ANY THIRD PARTY IN ANY COUNTRY.
%
% Copyright (c) 1994-2006, John Conover, All Rights Reserved.
%
% Comments and/or bug reports should be addressed to:
%
%     john@email.johncon.com (John Conover)
%
% -----------------------------------------------------------------------------
%
% Revision: \RCSRevision \\
% Revision Time: \RCSTime UMT \\
% Revision Date: \RCSDate \\
% Revision Id: \RCSId \\
% Revision File: \RCSLog \\
\RCS $Revision: 0.0 $
\RCS $Date: 2006/01/20 04:38:13 $
\RCS $Id: logistic.tex,v 0.0 2006/01/20 04:38:13 john Exp $
% $Log: logistic.tex,v $
% Revision 0.0  2006/01/20 04:38:13  john
% Initial version
%
%
    \subsection{Logistic Analysis}
        \label{\SETLABEL:LA}

        \subidx{\market}{Logistic function analysis}
        \subidx{time series}{logistic function}
        \subidx{logistic function}{time series}
        \subidx{time series}{increments}
        \subidx{time series}{analysis}
        \subidx{cumulative sum}{analysis}
        \subidx{analysis}{cumulative sum}
        \subidx{analysis}{random process}
        \subidx{random process}{analysis}
        The data in this section is presented in tabular form in
        Section~\ref{\SETLABELREF:LAA}.  Figure~\ref{\SETLABEL:LA1} is
        a graph of the logistic function estimates of the time series
        data for the {\market}. The reader is cautioned that these
        graphs are constructed using the method suggested in
        Chapter~\ref{general}, Section~\ref{nlextend} and enormous
        precision is required for adequate prediction of the logistic
        function,~\cite{Modis}. Particularly, the non-linear term will
        usually require intervention to produce a practical fit to the
        data. In addition, there are numerical stability issues with
        logistic function methodologies\footnote{For example, in
        Figures~\ref{\SETLABEL:LA1} and~\ref{\SETLABEL:LA2}, if the
        non-linear term, $b$, was greater than zero, it was set to
        zero to produce the graphs. See Section~\ref{\SETLABELREF:LAA}
        for the actual derived values. In other cases, the magnitude
        of $b$ was too large, resulting in a graph that was decreasing
        as a function of time}.  The methodology should be regarded as
        ``fragile.'' It is included for completeness.

        \idx{least squares approximation}
        Figure~\ref{\SETLABEL:LA1} is a graph of the logistic function
        for the time series data presented in
        Figure~\ref{\SETLABEL:TS}. The data presented was made by
        running the program {\it tsdlogistic}\/, which is described
        briefly in Appendix~\ref{programs}, on the parameters
        extracted from the time series data as suggested in
        Figure~\ref{\SETLABEL:TF}. The program {\it tslsq}\/ was used
        to derive the constant and the slope of the normalized
        increments of the data presented in Figure~\ref{\SETLABEL:TF}.
        Figure~\ref{\SETLABEL:LA2} is the same graph, but with the
        time scale expanded by a factor of two.

        \begin{figure}[ht]
            \begin{center}
                \begin{minipage}[t]{0.45\textwidth}
                    \epsfxsize=1.0\linewidth
                    \epsffile{\directory/data.tsfraction.tslsq-p.tsdlogistic.eps}
                    \caption[{\market}, logistic function
                        estimates.]{{\market}, logistic function
                        estimates, provided by running the {\it
                        tslsq}\/ program on the normalized increments
                        presented in Figure~\ref{\SETLABEL:TF} with
                        the -p option. These parameters were used as
                        arguments to the {\it tsdlogistic}\/ program.}
                    \label{\SETLABEL:LA1}
                    \label{\SETLABELQ:LA1}
                \end{minipage}
                \hfill
                \begin{minipage}[t]{0.45\textwidth}
                    \epsfxsize=1.0\linewidth
                    \epsffile{\directory/data.tsfraction.tslsq-p.tsdlogistic2.eps}
                    \caption[{\market}, logistic function
                        estimates.]{{\market}, logistic function
                        estimates of Figure~\ref{\SETLABEL:LA1} with
                        the time scale expanded by a factor of two.}
                    \label{\SETLABEL:LA2}
                    \label{\SETLABELQ:LA2}
                \end{minipage}
            \end{center}
        \end{figure}

% Local Variables:
% TeX-parse-self: t
% TeX-auto-save: t
% TeX-master: "fractal.tex"
% End:


        %
% -----------------------------------------------------------------------------
%
% A license is hereby granted to reproduce this software source code and
% to create executable versions from this source code for personal,
% non-commercial use.  The copyright notice included with the software
% must be maintained in all copies produced.
%
% THIS PROGRAM IS PROVIDED "AS IS". THE AUTHOR PROVIDES NO WARRANTIES
% WHATSOEVER, EXPRESSED OR IMPLIED, INCLUDING WARRANTIES OF
% MERCHANTABILITY, TITLE, OR FITNESS FOR ANY PARTICULAR PURPOSE.  THE
% AUTHOR DOES NOT WARRANT THAT USE OF THIS PROGRAM DOES NOT INFRINGE THE
% INTELLECTUAL PROPERTY RIGHTS OF ANY THIRD PARTY IN ANY COUNTRY.
%
% Copyright (c) 1994-2006, John Conover, All Rights Reserved.
%
% Comments and/or bug reports should be addressed to:
%
%     john@email.johncon.com (John Conover)
%
% -----------------------------------------------------------------------------
%
% Revision: \RCSRevision \\
% Revision Time: \RCSTime UMT \\
% Revision Date: \RCSDate \\
% Revision Id: \RCSId \\
% Revision File: \RCSLog \\
\RCS $Revision: 0.0 $
\RCS $Date: 2006/01/20 04:38:13 $
\RCS $Id: hurst.tex,v 0.0 2006/01/20 04:38:13 john Exp $
% $Log: hurst.tex,v $
% Revision 0.0  2006/01/20 04:38:13  john
% Initial version
%
%
    \subsection{Hurst Coefficient Analysis}
        \label{\SETLABEL:H}

        \subidx{\market}{Hurst coefficient analysis}
        \subidx{Hurst coefficient}{analysis}
        \subidx{increments}{normalized}
        \subidx{normalized}{increments}
        \subidx{programs}{tshurst}
        \subidx{tshurst}{program}
        The data in this section is presented in tabular form in
        Section~\ref{\SETLABELREF:HCHP}. Figure~\ref{\SETLABEL:HC} is
        a graph of the Hurst coefficient data time series data shown
        in Figure~\ref{\SETLABEL:TS}. The slope of the graph is the
        Hurst coefficient.  The data for this figure was produced by
        the program {\it tshurst}\/, which is described briefly in
        Appendix~\ref{programs}.

        \subidx{\market}{H parameter analysis}
        \subidx{H parameter}{analysis}
        \subidx{programs}{tshcalc}
        \subidx{tshcalc}{program}
        Figure~\ref{\SETLABEL:HP} is a graph of the H parameter data
        for the normalized increments of the time series data shown in
        Figure~\ref{\SETLABEL:TF}. The data for this figure was
        produced by the program {\it tshcalc}\/, which is described
        briefly in Appendix~\ref{programs}.

        \begin{figure}[ht]
            \begin{center}
                \begin{minipage}[t]{0.45\textwidth}
                    \epsfxsize=1.0\linewidth
                    \epsffile{\directory/data.tshurst.eps}
                    \caption[{\market}, Hurst coefficient data]{{\market},
                        Hurst coefficient data for the normalized
                        increments of the time series data shown in
                        Figure~\ref{\SETLABEL:TF}.  The slope of the graph
                        is the Hurst coefficient.}
                    \label{\SETLABEL:HC}
                \end{minipage}
                \hfill
                \begin{minipage}[t]{0.45\textwidth}
                    \epsfxsize=1.0\linewidth
                    \epsffile{\directory/data.tshcalc.eps}
                    \caption[{\market}, H parameter data]{{\market}, H
                        parameter data for the normalized increments of
                        the time series data shown in
                        Figure~\ref{\SETLABEL:TF} The slope of the graph
                        is the H parameter.}
                    \label{\SETLABEL:HP}
                \end{minipage}
            \end{center}
        \end{figure}

        \subidx{revenue}{See, rate of revenue returns}
        \subidx{returns}{See, rate of revenue returns}
        \subidx{\market}{revenues}
        \subidx{Hurst coefficient}{analysis}
        \subidx{\market}{Hurst coefficient analysis}
        \subidx{\market}{rate of change}
        \subidx{\market}{windows of opportunity}
        \subidx{rate of revenue returns}{forecast}
        \subidx{forecast}{rate of revenue returns}
        \idx{windows of opportunity}
        \subidx{programs}{tslsq}
        \subidx{tslsq}{program}

        The approximately linear slope of the graph in
        Figure~\ref{\SETLABEL:HC} implies that the variance of the
        rate of revenue returns, (per {\timescale},) in the {\market},
        $V(t_2 - t_1)$, over a period of time is proportional to the
        period of time raised to twice the Hurst
        coefficient~\cite[pp. 180]{Feder},~\cite[pp. 246]{Crownover}.
        This seems to be a quantitative statement concerning how fast,
        and to what degree, the rate of revenue returns' state of
        affairs can change over a period of time.  An additional
        implication, for Hurst coefficients sufficiently close to 0.5,
        is that the probability of the state of affairs repeating
        sometime in the future goes down with increasing
        time\footnote{It can be shown that the number of expected
        market ``high'' and ``low'' transitions, $N$, scales with the
        square root of time, or $N \propto \sqrt {t}$, meaning that
        the cumulative distribution of the probability, $P$, of the
        duration of a market's ``high'' or ``low'' exceeding a given
        time interval, $t$, is proportional to the reciprocal of the
        square root of the time interval, $P \propto 1 / \sqrt {t}$,
        (or, conversely, that the probability of the duration of a
        market's ``high'' or ``low'' exceeding a given time interval
        is proportional to the reciprocal of the time interval raised
        to the power $3 / 2$, ie., $P \propto 1 / t^{3 /
        2}$,~\cite[pp. 153]{Schroeder}. What this means is that a
        histogram of the ``zero free'' run-lengths of a market being
        ``high'' or ``low,'' over a long time, would have a $1 / t^{3
        / 2}$ characteristic.)}, $t$, $p(t) = erf (1/\sqrt{2t})$ which
        is approximately $1/\sqrt{t}$ for $t \gg
        1$~\cite[pp. 160]{Schroeder}. Figures~\ref{\SETLABEL:FN},
        and,~\ref{\SETLABEL:FF} compare methods of approximation of
        the ``forecastability'' of the rate of revenue returns in the
        {\market} for the near term and far term,
        respectively~\cite[pp. 83-84]{Peters:CAOITCM}\footnote{The
        author is not comfortable with Peters' interpretation. For
        example, if the algorithm explained
        in~\cite[pp. 82]{Peters:CAOITCM} is used on ``white noise''
        which, by definition, never has any correlations, the short
        term Hurst coefficient, and thus the ``forecastability,'' is
        still near unity---a bit of an enigma. This can be verified
        with the {\it tswhite}\/ and {\it tshurst}\/ programs, which
        are briefly described in Appendix~\ref{programs}.}.  This
        seems to be a quantitative statement concerning ``windows of
        opportunity'' in the rate of revenue returns, (per
        {\timescale}.)  The program {\it tslsq}\/ was used on the
        Hurst coefficient data, presented in
        Figure~\ref{\SETLABEL:HC}, to provide a least squares
        approximation to the Hurst coefficient. The superimposed least
        squares approximation with on original Hurst coefficient data
        is presented.  The time series data has a Hurst coefficient of
        {\thurstlow}, so that:

        \subidx{\market}{Hurst coefficient analysis}
        \begin{eqnarray}
            V\left(t_2 - t_1\right) & \propto & \left(t_2 - t_1\right)^{2 \cdot H}\\
            V\left(t_2 - t_1\right) & \propto & \left(t_2 - t_1\right)^{2 \cdot {\thurstlow}}\\
                                    & \propto & \left(t_2 - t_1\right)^{\thurstlowtwo}
            \label{\SETLABEL:V}
        \end{eqnarray}

        \subidx{fractional}{Brownian motion}
        \subidx{Brownian motion}{fractional}
        \idx{fractal}
        \noindent where $V(t_2 - t_1)$ is the variance of the
        increments of the rate of revenue returns, (per {\timescale},)
        over the time interval $t_2 -
        t_1$,~\cite[pp. 177]{Feder},~\cite[pp. 494]{Peitgen}. If $H >
        \frac{1}{2}$, then the time series is termed as being
        characterized by ``fractional Brownian
        motion~\cite[pp. 170]{Feder}.''

        \subidx{rate of revenue returns}{predictability}
        \subidx{rate of revenue returns}{forecastability}
        \subidx{rate of revenue returns}{consistency}
        \subidx{predictability}{rate of revenue returns}
        \subidx{forecastability}{rate of revenue returns}
        \subidx{consistency}{rate of revenue returns}
        \subidx{\market}{rate of revenue returns, predictability}
        \subidx{\market}{rate of revenue returns, forecastability}
        \subidx{\market}{rate of revenue returns, consistency}
        \subidx{Hurst coefficient}{analysis}
        \subidx{\market}{Hurst coefficient analysis}
        \subidx{\market}{rate of change}

        In some sense, the Hurst coefficient is a quantitative
        expression of the ``forecastability'' of the future based on
        the past\footnote{Actually, in general, when summing fractal
        entities, the method used should be a root mean square
        process, dependent on the Hurst Coefficient, $H$, where
        $P_{total}^H = P_1^H + P_2^H + \cdots$, where $P_n$ is the
        fractal entities. For a Brownian motion, or random walk type
        of fractal the Hurst Coefficient is a function of time into
        the future. For the ``near term,'' the Hurst coefficient is
        very near unity, meaning the summation process is linear. For
        the ``long term,'' $H \approx 0.5$, or a standard root mean
        square summation process should be used. If $H$ is $0.5$ then
        the market is termed a Brownian motion, or random walk
        process. If it is larger than 0.5, it is termed fractional
        Brownian motion process. For a random walk process, ``near
        term'' and ``far term'' are quantitatively differentiated on
        the Hurst Coefficient graph where $1 - \ln (t) = 0.5 \cdot \ln
        (t)$, or when $\ln (t) = 2$, or $t = 7.389\ldots$ See
        Section~\ref{\SETLABEL:FS} for the particulars on using Hurst
        Coefficient to sum fractal process' for the {\market}. See
        also~\cite[pp. 67, 83-84]{Peters:CAOITCM} and~\cite[pp. 129,
        159]{Schroeder} for particulars on the implications of the
        Hurst Coefficient and root mean square summation issues.}.  A
        Hurst coefficient of {\thurstlow}, (for the near future, and
        {\thurstall} for the distant future.) implies that the
        likelihood of the rate of revenue returns, (per {\timescale},)
        for any two consecutive {\timescale}s being the same is
        {\thurstlowhundred}\%~\cite[pp. 66]{Peters:CAOITCM} for the
        near future, and {\thurstall} for the distant
        future. Likewise, there is a {\thurstlowhundred}\% chance of
        the rate of revenue returns, (per {\timescale},) movements
        being the same in consecutive time periods---ie., if, in a
        given {\timescale}, the rate of revenue returns, (per
        {\timescale},) is increasing, there is a {\thurstlowhundred}\%
        that the rate of revenue returns, (per {\timescale},) will
        increase in the following period, also. In some sense, this is
        a quantitative statement on how ``predictable,'' or
        ``forecastable'' the rate of revenue returns, (per
        {\timescale},) for the {\market} are over time, since the
        probability of having $n$ many consecutive {\timescale}s of
        the same agenda is $H^n$ where $H$ is the Hurst coefficient,
        or, letting the short term probability of having $n$ many
        {\timescale}s of the same market agenda, $p_a$, is:

        \begin{eqnarray}
            p_a\left(n\right) & = & H^{n}\\
                              & = & {\thurstlow}^{n}
            \label{\SETLABEL:MA}
        \end{eqnarray}

        \subidx{rate of revenue returns}{predictability}
        \subidx{rate of revenue returns}{forecastability}
        \subidx{rate of revenue returns}{consistency}
        \subidx{predictability}{rate of revenue returns}
        \subidx{forecastability}{rate of revenue returns}
        \subidx{consistency}{rate of revenue returns}
        As an interesting interpretation of the normalized increments
        of the time series data presented in
        Figure~\ref{\SETLABEL:TF}, if the vertical axis is multiplied
        by 100, to convert to percent, then the graph represents the
        error, in percent, that would be made by forecasting, month by
        month, that the next {\timescale}'s rate of revenue returns
        would be the same as the current {\timescale}'s revenue
        rate. Interestingly, it is $\datafractionmean \cdot 100$
        percent, on the average, with a standard deviation of
        $\datafractionstddev \cdot 100$ percent, and a root mean
        square error value of $\datafractionrms \cdot 100$
        percent---small values for such a simple forecasting
        mechanism.

        \subidx{\market}{rate of revenue returns, range}
        \subidx{Hurst coefficient}{analysis}
        \subidx{\market}{Hurst coefficient analysis}
        \subidx{\market}{rate of change}

        This is, essentially, a statement of the range of values, in
        the increments of the rate of revenue returns, (per
        {\timescale},) that is to be expected over the time interval,
        $t_2 - t_1$,
        $R_v$,~\cite[pp. 178]{Feder},~\cite[pp. 172]{Cambel}:

        \begin{eqnarray}
            R_v\left(t_2 - t_1\right) & \propto & \left(t_2 - t_1\right)^{H}\\
                                      & \propto & \left(t_2 - t_1\right)^{\thurstlow}
            \label{\SETLABEL:R}
        \end{eqnarray}

        \subidx{\market}{rate of revenue returns, range}
        \subidx{Hurst coefficient}{analysis}
        \subidx{\market}{Hurst coefficient analysis}
        \subidx{\market}{rate of change}
        \subidx{Markov}{statistics}
        \subidx{statistics}{Markov}
        \noindent where $R$ is the range of values in the increments
        of the rate of revenue returns, (per {\timescale}.) A Hurst
        coefficient, $H$, that is much larger than $\frac{1}{2}$, (but
        less than 1,) implies a strongly non-Gaussian distribution in
        the increments of the rate of revenue returns, (per
        {\timescale},)~\cite[pp. 152, 194]{Feder}, and a Hurst
        coefficient near $\frac{1}{2}$ implies that the increments of
        the rate of revenue returns, (per {\timescale}) is
        characteristic of an independent
        process~\cite[pp. 195]{Feder}. Extreme caution should be
        exercised in using Markov statistics in any analysis where the
        Hurst coefficient is not
        $\frac{1}{2}$,~\cite[pp. 124]{Crownover},~\cite[pp. 106]{Peters:CAOITCM}.


        As a useful approximation, if $H$, is approximately
        $\frac{1}{2}$, Equation~\ref{\SETLABEL:R} reduces
        to,~\cite[pp. 129]{Schroeder}:

        \begin{eqnarray}
            R\left(t_2 - t_1\right) & \propto & (t_2 - t_1)^{\frac{1}{2}}\\
                                    & \propto & \sqrt{\left(t_2 - t_1\right)}
        \end{eqnarray}

        \subidx{\market}{rate of revenue returns, range}
        \subidx{\market}{rate of revenue returns, increase and decrease}
        \subidx{Hurst coefficient}{analysis}
        \subidx{\market}{Hurst coefficient analysis}
        \subidx{\market}{rate of change}
        \subidx{Markov}{statistics}
        \subidx{statistics}{Markov}

        In the case where the Hurst coefficient, $H$, is
        $\frac{1}{2}$, the range of values in the increments of the
        rate of revenue returns, (per {\timescale},) divided by the
        standard deviation of these values, $S$, can be anticipated to
        increase over time according to the following
        relation,~\cite[pp. 154]{Feder},~\cite[pp. 129]{Schroeder}:

        \begin{equation}
            \frac{R\left(t_2 - t_1\right)}{S} \propto \left(t_2 - t_1\right)^{\frac{1}{2}}
        \end{equation}

        \subidx{\market}{rate of revenue returns, range}
        \subidx{\market}{rate of revenue returns, increase and decrease}
        \subidx{Hurst coefficient}{analysis}
        \subidx{\market}{Hurst coefficient analysis}
        \subidx{\market}{rate of change}
        \noindent which is a useful conceptual approximation, since it
        involves only the square root function---if the range and the
        standard deviation of the increments of the rate of revenue
        returns, (per {\timescale},) are known, (and $H \approx
        \frac{1}{2}$,) then the expected change in $\frac{R}{S}$, will
        increase with the square root of time\footnote{To be precise,
        it is actually asymptotically proportional to
        $\tau^{\frac{1}{2}}$}.

        Another useful approximation when rescaling processes that are
        characterize by Brownian motion, (ie., when $H \approx
        \frac{1}{2}$,) is that:

        \begin{eqnarray}
            X\left(t\right) & \propto & \frac{X\left(rt\right)}{r^{H}}\\
                            & \propto & \frac{X\left(rt\right)}{r^{\thurstlow}}
        \end{eqnarray}

        \idx{Brownian motion}
        \idx{fractal}
        Where $X(t)$ is the process characterized by Brownian motion,
        and $r$ is a scaling factor,~\cite[pp. 494]{Peitgen}.

        \subidx{programs}{tslsq}
        \subidx{tslsq}{program}
        The program {\it tslsq}\/ was used on the H parameter data,
        presented in Figure~\ref{\SETLABEL:HP}, to provide a least
        squares approximation to the H parameter for the
        {\market}. The superimposed least squares approximation on the
        original H parameter data is presented.  By contrast, the H
        parameter, as derived by the methodology outlined
        in~\cite[pp. 249]{Crownover}, is {\thcalclow} for the near
        future, and {\thcalcall} for the distant future.

        \subidx{\market}{Hurst coefficient analysis}
        \subidx{Hurst coefficient}{analysis}
        \subidx{increments}{normalized}
        \subidx{normalized}{increments}
        \subidx{programs}{tshurst}
        \subidx{tshurst}{program}
        \subidx{\market}{H parameter analysis}
        \subidx{H parameter}{analysis}
        \subidx{programs}{tshcalc}
        \subidx{tshcalc}{program}
        Figures~\ref{\SETLABEL:HC} and~\ref{\SETLABEL:HP} represent
        Hurst coefficient and H parameter data that are derived from
        the normalized increments, shown in
        Figure~\ref{\SETLABEL:TF}. In this case, the data is
        considered a normalized derivative of the time series data
        presented in Figure~\ref{\SETLABEL:TF}, instead of a
        cumulative sum.  The program, {\it tshurst}\/, is described
        briefly in appendix~\ref{programs}, and the data for
        figures~\ref{\SETLABEL:THC} and~\ref{\SETLABEL:THP} was made
        using the -d option.

        \begin{figure}[ht]
            \begin{center}
                \begin{minipage}[t]{0.45\textwidth}
                    \epsfxsize=1.0\linewidth
                    \epsffile{\directory/data.tsfraction.tshurst-d.eps}
                    \caption[{\market}, traditional Hurst coefficient
                        data]{{\market}, traditional Hurst coefficient
                        data for the time series data shown in
                        Figure~\ref{\SETLABEL:TS}.  The slope of the
                        graph is the Hurst coefficient, and is
                        {\hurstlow} for the near term, and
                        {\hurstall} for the far term.}
                    \label{\SETLABEL:THC}
                \end{minipage}
                \hfill
                \begin{minipage}[t]{0.45\textwidth}
                    \epsfxsize=1.0\linewidth
                    \epsffile{\directory/data.tsfraction.tshcalc-d.eps}
                    \caption[{\market}, traditional H parameter
                        data]{{\market}, traditional H parameter data
                        for the time series data shown in
                        Figure~\ref{\SETLABEL:TS} The slope of the
                        graph is the H parameter, and is {\hcalclow}
                        for the near term, and {\hcalcall} for the
                        far term.}
                    \label{\SETLABEL:THP}
                \end{minipage}
            \end{center}
        \end{figure}

% Local Variables:
% TeX-parse-self: t
% TeX-auto-save: t
% TeX-master: "fractal.tex"
% End:


        %
% -----------------------------------------------------------------------------
%
% A license is hereby granted to reproduce this software source code and
% to create executable versions from this source code for personal,
% non-commercial use.  The copyright notice included with the software
% must be maintained in all copies produced.
%
% THIS PROGRAM IS PROVIDED "AS IS". THE AUTHOR PROVIDES NO WARRANTIES
% WHATSOEVER, EXPRESSED OR IMPLIED, INCLUDING WARRANTIES OF
% MERCHANTABILITY, TITLE, OR FITNESS FOR ANY PARTICULAR PURPOSE.  THE
% AUTHOR DOES NOT WARRANT THAT USE OF THIS PROGRAM DOES NOT INFRINGE THE
% INTELLECTUAL PROPERTY RIGHTS OF ANY THIRD PARTY IN ANY COUNTRY.
%
% Copyright (c) 1994-2006, John Conover, All Rights Reserved.
%
% Comments and/or bug reports should be addressed to:
%
%     john@email.johncon.com (John Conover)
%
% -----------------------------------------------------------------------------
%
% Revision: \RCSRevision \\
% Revision Time: \RCSTime UMT \\
% Revision Date: \RCSDate \\
% Revision Id: \RCSId \\
% Revision File: \RCSLog \\
\RCS $Revision: 0.0 $
\RCS $Date: 2006/01/20 04:38:13 $
\RCS $Id: fiscal.tex,v 0.0 2006/01/20 04:38:13 john Exp $
% $Log: fiscal.tex,v $
% Revision 0.0  2006/01/20 04:38:13  john
% Initial version
%
%
    \subsection{Fixed Increment Approximation for Fiscal Strategy}
        \label{\SETLABEL:FS}

        \subidx{\market}{fiscal strategy}
        \subidx{markets}{analysis}
        \subidx{analysis}{markets}
        \subidx{strategy}{fiscal}
        \subidx{fiscal}{strategy}
        The data in this section is presented in tabular form in
        Section~\ref{\SETLABELREF:LR}. This section derives various
        values based on the ``average'' of the normalized increments
        presented in Figure~\ref{\SETLABEL:TFA}. These values are an
        approximation to a, probably, complex process with a
        distribution shown in Figure~\ref{\SETLABEL:TF}. These values
        will be used in a fixed increment Brownian fractal analysis
        and simulation of the {\market}, and may, or may not, provide
        adequate accuracy for projections.

        For an organization operating in the {\market}, the fiscal
        strategy, commensurate with the aggregate environment, can be
        derived as follows~\cite[pp. 128, pp
        151]{Schroeder},~\cite[pp. 450]{Reza},~\cite[pp. 270]{Pierce}:
        \vspace{0.15in}

        \subsubsection{Logarithmic Returns}
            \label{\SETLABEL:LR}

            \subidx{logarithmic}{returns}
            \subidx{returns}{logarithmic}
            \subidx{\market}{logarithmic returns}
            The logarithmic returns can be calculated by various
            means. Four will be presented here, for comparison.

            \subidx{programs}{tsnormal}
            \subidx{tsnormal}{program}
            \subidx{logarithmic}{returns}
            \subidx{returns}{logarithmic}
            The logarithmic returns, in bits, $bits$, as computed from
            the mean, by the program {\it tsnormal}\/, which is
            described in Chapter~\ref{programs}, and is presented in
            Figure~\ref{\SETLABEL:TF}, and Equation~\ref{abits} from
            Section~\ref{ereturns} in Chapter~\ref{general}:

            \begin{equation}
                bits = \frac{\ln \left({\datafractionmean} + 1\right)}{\ln \left(2\right)} = \datafractionmeanbits
            \end{equation}

            \subidx{programs}{tslsq}
            \subidx{tslsq}{program}
            \subidx{logarithmic}{returns}
            \subidx{returns}{logarithmic}
            \noindent By comparison, the logarithmic returns, in bits,
            $bits$, as computed from the constant in the least squares
            approximation, using the program {\it tslsq}\/, which is briefly
            described in Chapter~\ref{programs}, as presented in
            Figure~\ref{\SETLABEL:TF}, and Equation~\ref{abits} from
            Section~\ref{ereturns} in Chapter~\ref{general}:

            \begin{equation}
                bits = \frac{\ln \left({\datafractionconstant} + 1\right)}{\ln \left(2\right)} = \datafractionconstantbits
            \end{equation}

            Note that if the mean is not constant in
            Figure~\ref{\SETLABEL:TF}, this method will not provide
            accurate results.

            \subidx{programs}{tslsq}
            \subidx{tslsq}{program}
            \subidx{logarithmic}{returns}
            \subidx{returns}{logarithmic}
            \noindent And by yet another comparison, using the program
            {\it tslsq}\/, which is briefly described in
            Chapter~\ref{programs}, with the -e -p options, to provide
            a formula for the least squares exponential fit to the
            time series data set presented in
            Figure~\ref{\SETLABEL:TS}:

            \begin{equation}
                bits = {\datatslsqepbits}
            \end{equation}

            \subidx{programs}{tslogreturns}
            \subidx{tslogreturns}{program}
            \subidx{logarithmic}{returns}
            \subidx{returns}{logarithmic}
            \noindent And finally, by comparison, from the
            {\it tslogreturns}\/ program, which is briefly described
            in Chapter~\ref{programs}, with the -p option, to provide
            a formula for the logarithmic returns of the time series
            data set presented in Figure~\ref{\SETLABEL:TS}:

            \begin{equation}
                bits = {\logreturns}
            \end{equation}

        \subsubsection{Calculation of Shannon Probability}
            \label{\SETLABEL:SP}

            \subidx{\market}{Shannon probability}
            Ideally, all of the values presented in
            Section~\ref{\SETLABEL:LR} would be equal. Using the
            logarithmic returns provided by the {\it tslogreturns}\/
            program, to be consistent
            with~\cite[pp. 81]{Peters:CAOITCM}

            \subidx{programs}{tslogreturns}
            \subidx{tslogreturns}{program}
            \begin{equation}
                2^{{\logreturns}t}
            \end{equation}

            \noindent therefore:
            \begin{equation}
                C\left(p\right) = {\logreturns}
            \end{equation}
            \subidx{programs}{tsshannon}
            \subidx{tsshannon}{program}
            \subidx{Shannon}{probability}
            \subidx{probability}{Shannon}
            \noindent and, {\it tsshannon}\/ {\logreturns} gives:
            \begin{equation}
                \label{\SETLABEL:F0}
                C\left({\shannonlogreturns}\right) = {\logreturns}
            \end{equation}
            \noindent therefore:
            \begin{eqnarray}
                2^{C\left({\shannonlogreturns}\right)} & = & 2^{\logreturns}\\
                                                       & = & {\twologreturns}\\
                                                       & = & {\twologreturnshundred}\%
            \end{eqnarray}
            \noindent and:
            \begin{eqnarray}
                2p - 1 & = & \left(2 \cdot {\shannonlogreturns}\right) - 1\\
                       & = & {\twopone}\\
                       \label{\SETLABEL:F1}
                       & = & {\twoponehundred}\%
            \end{eqnarray}

            \subidx{\market}{fiscal strategy}
            \subidx{markets}{analysis}
            \subidx{analysis}{markets}
            \subidx{strategy}{fiscal}
            \subidx{fiscal}{strategy}
            \subidx{\market}{fiscal strategy}
            \subidx{\market}{growth rate}
            Presuming the simplified assumptions outlined in
            Section~\ref{assumptions}, the ``typical'' organization
            operating in the {\market} executes a long term fiscal
            strategy, commensurate with the aggregate environment,
            that is to invest, every {\timescale}, in sufficient
            additional resources and infrastructure, to increase the
            manufacturing of goods and services by {\twoponehundred}\%
            of its rate of revenue returns, (per {\timescale}.) As a
            conceptual model, the remaining {\hundredtwoponehundred}\%
            will be held in ``reserve'' with a
            {\shannonlogreturnshundred}\% chance of making twice the
            {\twoponehundred}\% back, (and a
            {\hundredshannonlogreturnshundred}\% chance of making
            0.0,) in one {\timescale}, on the average, for an average
            growth in its rate of revenue returns, (per {\timescale},)
            of {\twologreturnshundred}\%, or a doubling of its rate of
            revenue returns, (per {\timescale},) in
            {\oneoverlogreturns} {\timescale}s.

        \subsubsection{Example Fixed Increment Approximation Fiscal Strategies}

            \subidx{\market}{fiscal strategy}
            \subidx{markets}{analysis}
            \subidx{analysis}{markets}
            \subidx{strategy}{fiscal}
            \subidx{fiscal}{strategy}
            \subidx{\market}{fiscal strategy}
            \subidx{\market}{growth rate}
            \subidx{\market}{management metric}
            \idx{management metric}
            A possible metric on the effectiveness of long term fiscal
            management could possibly be that if an investment of
            {\twoponehundred}\% per {\timescale} of the rate of
            revenue returns, (per {\timescale},) is made in resources
            and infrastructure, then the rate of revenue returns would
            be expected to increase by {\twologreturnshundred}\%, per
            {\timescale}, on average.

            Note that the metrics presented in this section are
            representative of the {\market} as an aggregate whole, and
            may or may not be accurate representations for any
            particular participant in the environment. Of interest to
            the participants in the environment would be a similar
            analysis of each product or service rendered in the
            marketplace.

            \subidx{\market}{fiscal strategy}
            \subidx{markets}{analysis}
            \subidx{analysis}{markets}
            \subidx{strategy}{fiscal}
            \subidx{fiscal}{strategy}
            \subidx{\market}{fiscal strategy}
            As a simple illustrative example, a company operating in
            this environment might obtain a credit line from a bank
            that is equal to {\twoponehundred}\% of its rate of
            revenue returns, (per {\timescale},) to finance additional
            operations. In this simple scenario, the company would use
            its revenue base as collateral for the loan. Some
            {\timescale}s, depending on the {\market}'s environment,
            the company's rate of revenue returns exceeds what was
            borrowed from the bank, and the loan is repaid in
            full. Other {\timescale}s, the company must default, and
            the bank seizes a portion of the company's revenue base to
            pay the delinquent loan. However, on the average, the
            company will expand its rate of revenue returns at
            {\twologreturnshundred}\% per {\timescale}.

            \subidx{\market}{fiscal strategy}
            \subidx{markets}{analysis}
            \subidx{analysis}{markets}
            \subidx{strategy}{fiscal}
            \subidx{fiscal}{strategy}
            \subidx{\market}{fiscal strategy}
            As another simple example, a company re-invests
            {\twoponehundred}\% of its rate of revenue returns, (per
            {\timescale},) in development, marketing, sales, and
            distribution of new products.  Although some products will
            be successful and the return on the investment will exceed
            the {\twoponehundred}\% per {\timescale} investment,
            others will not. However, on the average, the company will
            expand it gross rate of revenue returns at
            {\twologreturnshundred}\% per {\timescale}.

            \subidx{\market}{fiscal strategy}
            \subidx{markets}{analysis}
            \subidx{analysis}{markets}
            \subidx{strategy}{fiscal}
            \subidx{fiscal}{strategy}
            \subidx{\market}{fiscal strategy}
            \subidx{\market}{product portfolio}
            \subidx{\market}{product diversity}
            \subidx{\market}{product mix}
            \subidx{\market}{optimum number of products}
            \idx{product portfolio}
            \idx{product diversity}
            \idx{optimum number of products}
            \idx{product mix}

            As an example of ``product portfolio'' management, suppose
            a company re-invests {\twoponehundred}\% of its rate of
            revenue returns, (per {\timescale},) in development,
            marketing, sales, and distribution of new products.
            Further suppose that the company has two products, and a
            fractal analysis of the individual product rate of revenue
            return time series indicates that one product has a
            Shannon probability of 0.65, and the other has a Shannon
            probability of 0.55. Then the percentage of re-investment
            in the first product would be $(2 \cdot 0.65 - 1) \cdot
            {\twoponehundred}$, percent of the rate of revenue
            returns, and $(2 \cdot 0.55 - 1) \cdot {\twoponehundred}$
            percent for the second product, implying that the company
            should diversify its product line\footnote{The astute
            reader would note that the linear addition was used to add
            the contribution to development of each product. This is a
            ``near term'' interpretation. Actually, in general, the
            method used should be a root mean square process,
            dependent on the Hurst Coefficient, $H$, where
            $P_{total}^H = P_1^H + P_2^H + \cdots$, where $P_n$ is the
            contribution to each individual product. For a Brownian
            motion, or random walk type of fractal the Hurst
            Coefficient is a function of time into the future. For the
            ``near term,'' the Hurst coefficient is very near unity,
            meaning the summation process is linear. For the ``long
            term,'' $H \approx 0.5$, or a standard root mean square
            summation process should be used. If $H$ is $0.5$ then the
            market is termed a Brownian motion, or random walk
            process. If it is larger than 0.5, it is termed fractional
            Brownian motion process. For a random walk process, ``near
            term'' and ``far term'' are quantitatively differentiated
            on the Hurst Coefficient graph where $1 - \ln (t) = 0.5
            \cdot \ln (t)$, or when $\ln (t) = 2$, or $t =
            7.389\ldots$ See~\cite[pp. 67, 83-84]{Peters:CAOITCM}
            and~\cite[pp. 129, 159]{Schroeder} for particulars on the
            implications of the Hurst Coefficient and root mean square
            summation issues.}.  Note that this is a ``bet hedging''
            metric methodology, and assumes that the products have
            uncorrelated revenue return rates. If this re-investment
            methodology is not feasible, perhaps for strategic
            financial reasons, then the re-investment in both products
            should total the ${\twoponehundred}$\%, and the investment
            in each product should be made at a ratio of $\frac{(2
            \cdot 0.65 - 1)}{(2 \cdot 0.55 - 1)} = 3 : 1$,
            respectively. Note that this ``bet hedging'' can be used
            to define the optimal number of products that can be
            supported on the rate of revenue returns. If it assumed
            that all products are ``typical'' for the {\market}, as a
            standard bench mark, then the optimal number will be
            $\frac{1}{{\twopone}}$. Note that this is a
            ``theoretical'' value, since not all products are
            ``typical,'' and there may be strategic reasons, for
            example product leveraging, that may increase the number
            of products above the optimum. However, most of the
            revenue should come from the optimal number of products,
            since having more products will decrease the amount of the
            potential investment in each product, and having less than
            the optimum number of products will increase the risk that
            many of the products could suffer a ``down market''
            concurrently, impacting the rate of revenue returns.  As
            another interesting interpretation of the optimal
            ``hedging of bets,'' in product portfolio strategy, and
            considering the graph of the normalized increments
            presented in Figure~\ref{\SETLABEL:TF}, if the
            organization is running optimally, then these products
            will generate, at least in principle, one standard
            deviation, approximately $0.8413 = 84.13$\% of the future
            growth in rate of revenue returns. Naturally, these are
            approximations, and the values are an approximation to a,
            probably, complex process, and appropriate scrutiny should
            be exercised before making specific projections.  As yet
            another example of ``product portfolio'' management,
            consider the issue of product mix. In this interpretation,
            {\twoponehundred}\% of the product manufactured should be
            ``proprietary,'' while the rest is ``industry standard.''
            As yet another possibility, {\twoponehundred}\% of the
            product manufactured should be predatory into new markets,
            and the remainder in markets that are ``traditional'' for
            the company.

% Local Variables:
% TeX-parse-self: t
% TeX-auto-save: t
% TeX-master: "fractal.tex"
% End:


        %
% -----------------------------------------------------------------------------
%
% A license is hereby granted to reproduce this software source code and
% to create executable versions from this source code for personal,
% non-commercial use.  The copyright notice included with the software
% must be maintained in all copies produced.
%
% THIS PROGRAM IS PROVIDED "AS IS". THE AUTHOR PROVIDES NO WARRANTIES
% WHATSOEVER, EXPRESSED OR IMPLIED, INCLUDING WARRANTIES OF
% MERCHANTABILITY, TITLE, OR FITNESS FOR ANY PARTICULAR PURPOSE.  THE
% AUTHOR DOES NOT WARRANT THAT USE OF THIS PROGRAM DOES NOT INFRINGE THE
% INTELLECTUAL PROPERTY RIGHTS OF ANY THIRD PARTY IN ANY COUNTRY.
%
% Copyright (c) 1994-2006, John Conover, All Rights Reserved.
%
% Comments and/or bug reports should be addressed to:
%
%     john@email.johncon.com (John Conover)
%
% -----------------------------------------------------------------------------
%
% Revision: \RCSRevision \\
% Revision Time: \RCSTime UMT \\
% Revision Date: \RCSDate \\
% Revision Id: \RCSId \\
% Revision File: \RCSLog \\
\RCS $Revision: 0.0 $
\RCS $Date: 2006/01/20 04:38:13 $
\RCS $Id: companies.tex,v 0.0 2006/01/20 04:38:13 john Exp $
% $Log: companies.tex,v $
% Revision 0.0  2006/01/20 04:38:13  john
% Initial version
%
%
    \subsection{Number of Companies}
        \label{\SETLABEL:QNC}

        \subidx{\market}{number of companies}
        \subidx{number of companies}{analysis}
        \subidx{analysis}{number of companies}
        \subidx{Shannon}{probability}
        \subidx{probability}{Shannon}
        This section evaluates the approximate, or ``average,'' number
        of companies in the {\market}, and uses the method outlined in
        Chapter~\ref{general}, Section~\ref{aftsma}. Since the
        average, $avg_{ind}$, and the root mean square, $rms_{ind}$,
        of the normalized increments of the {\market} time series is
        \datafractionmean, and \datafractionrms respectively, the
        number of companies participating in the market can be
        calculated by Equation~\ref{ncompanies} to be {\ncompanies}.

        If this value seems consistent number of companies in the
        {\market}, within the assumptions outlined in
        Chapter~\ref{general}, Section~\ref{aftsma}, then it would
        seem that there is some circumstantial or indirect evidence
        that the companies participating in the {\market} are
        operating optimally, and the ``average'' Shannon probability,
        $P$ for each participating company would be, using
        Equation~\ref{pncompanies}, {\pncompanies}, which would be the
        value which should be used in Section~\ref{\SETLABEL:FS} for
        each participating company if market expansion was to be
        consistent with the rest of the industry. However, if the
        Shannon probability derived in Section~\ref{\SETLABEL:FS} is
        greater than the average Shannon probability for the companies
        participating in the {\market}, as derived in this section,
        then the market would, possibly, be exploitable with the
        fiscal strategy outlined in Section~\ref{\SETLABEL:FS}. The
        maximum exploitability for the {\market} is derived in
        Section~\ref{\SETLABEL:MAXSHANNON}, but it is probably of
        doubtful practicality.

        Note that these optimizations would maximize a company's
        market growth. Since there are probably many companies
        competing in the market place, this would not necessarily
        maximize a company's P\&L, as described in
        Chapter~\ref{general}, Section~\ref{ompl}. The Shannon
        probability that maximizes market share in the {\market} is
        \pncompanies, with several alternative solutions listed in the
        previous paragraph. However, these should be contrasted to the
        Shannon probability that maximizes a company's P\&L which is
        \avgrms~in the {\market}. In all cases, the fraction of the
        P\&L that should be ``wagered'' on the future, $f$, should be:

        \begin{equation}
            f = 2P - 1
        \end{equation}

        \noindent where $P$ is the particular Shannon probability
        chosen optimize a particular fiscal strategy. Interestingly,
        the measured Shannon probability of the {\market} would tend
        to indicate that the companies participating in the market
        have chosen a fiscal strategy that optimizes market growth, as
        opposed to capital growth.

        \subidx{\market}{increasing returns}
        \subidx{economic increasing returns}{\market}
        As interesting interpretation of these exploitive issues,
        since all three fiscal strategies will result in exponential
        market growth for every company participating in the market,
        is that they may represent, perhaps, an example of
        ``increasing returns.''

% Local Variables:
% TeX-parse-self: t
% TeX-auto-save: t
% TeX-master: "fractal.tex"
% End:


        %
% -----------------------------------------------------------------------------
%
% A license is hereby granted to reproduce this software source code and
% to create executable versions from this source code for personal,
% non-commercial use.  The copyright notice included with the software
% must be maintained in all copies produced.
%
% THIS PROGRAM IS PROVIDED "AS IS". THE AUTHOR PROVIDES NO WARRANTIES
% WHATSOEVER, EXPRESSED OR IMPLIED, INCLUDING WARRANTIES OF
% MERCHANTABILITY, TITLE, OR FITNESS FOR ANY PARTICULAR PURPOSE.  THE
% AUTHOR DOES NOT WARRANT THAT USE OF THIS PROGRAM DOES NOT INFRINGE THE
% INTELLECTUAL PROPERTY RIGHTS OF ANY THIRD PARTY IN ANY COUNTRY.
%
% Copyright (c) 1994-2006, John Conover, All Rights Reserved.
%
% Comments and/or bug reports should be addressed to:
%
%     john@email.johncon.com (John Conover)
%
% -----------------------------------------------------------------------------
%
% Revision: \RCSRevision \\
% Revision Time: \RCSTime UMT \\
% Revision Date: \RCSDate \\
% Revision Id: \RCSId \\
% Revision File: \RCSLog \\
\RCS $Revision: 0.0 $
\RCS $Date: 2006/01/20 04:38:13 $
\RCS $Id: operations.tex,v 0.0 2006/01/20 04:38:13 john Exp $
% $Log: operations.tex,v $
% Revision 0.0  2006/01/20 04:38:13  john
% Initial version
%
%
    \subsection{Fixed Increment Approximation for Operational Strategy}
        \label{\SETLABEL:OPS}.

        This section derives various values based on the ``average''
        of the normalized increments presented in
        Figure~\ref{\SETLABEL:TFA}. These values are an approximation
        to a, probably, complex process with a distribution shown in
        Figure~\ref{\SETLABEL:TF}. These values will be used in a
        fixed increment Brownian fractal analysis and simulation of
        the {\market}, and may, or may not, provide adequate accuracy
        for projections.

        \subidx{\market}{fiscal strategy}
        \subidx{\market}{Shannon probability}
        \subidx{strategy}{fiscal}
        \subidx{fiscal}{strategy}
        \subidx{Shannon}{probability}
        \subidx{probability}{Shannon}
        It should be noted that the analysis of fiscal strategy,
        presented in Section~\ref{\SETLABEL:FS}, is derived from the
        {\market} metrics and may, or may not, be maximally
        optimal. For the optimal fiscal strategy, which may be
        exploitable, see Section~\ref{\SETLABEL:MAXSHANNON}.

        \subidx{strategy}{exploitable}
        \subidx{exploitable}{strategy}
        \subidx{\market}{windows of opportunity}
        \idx{windows of opportunity}
        \subidx{decision}{obsolete}
        \subidx{obsolete}{decision}
        \subidx{decision}{timeliness}
        \subidx{timeliness}{decision}
        \subidx{rate of revenue returns}{forecast}
        \subidx{forecast}{rate of revenue returns}
        An additional exploitable strategy may be time itself.
        Equations~\ref{\SETLABEL:V},~\ref{\SETLABEL:R},
        and,~\ref{\SETLABEL:MA}, are, essentially, metrics on how fast
        a decision, which is based on information concerning the
        current status of the {\market}, becomes obsolete. Obviously,
        how long a decision is expected to remain relevant should be
        addressed as an operational necessity in strategic planning
        and project management. Figures~\ref{\SETLABEL:FN},
        and,~\ref{\SETLABEL:FF} compare methods of approximation of
        the ``forecastability'' of rate of revenue returns in the
        {\market} for the near term and far
        term~\cite[pp. 83-84]{Peters:CAOITCM}, respectively. As a
        general rule, caution must be exercised when making decisions
        that will span a time interval larger than the time interval
        where the ``forecastability'' of rate of revenue returns drops
        below 50\%. Beyond this time interval, the chances increase
        that the competitive and market forces will alter the market
        environment in a possibly detrimental unanticipated
        fashion. Obviously, there is significant advantage in
        ``timeliness'' of development, manufacturing, and distribution
        of products and services that are consistent with this
        temporal agenda. Automation of these processes, if executed
        consistently with this agenda, should be considered a
        competitive advantage.

        \subidx{strategy}{exploitable}
        \subidx{exploitable}{strategy}
        \subidx{rate of revenue returns}{forecast}
        \subidx{forecast}{rate of revenue returns}
        \idx{product life cycle}
        \idx{life cycle, product}
        In some sense, this temporal agenda defines the ``average''
        product or service life cycle in the {\market}. When the
        ``forecastability'' of rate of revenue returns drops below
        50\%, there is an even chance that the rate of revenue returns
        for the product or service will change in a detrimental
        fashion. If it is assumed that a product or service life cycle
        consists of a ramp up, a maintenence interval, and a ramp
        down, then, if all three life cycle intervals are equal, the
        product life cycle will be, approximately, three times the
        time interval where the ``forecastability'' of rate of revenue
        returns drops below 50\%. Although probably not an accurate
        prediction of product or service life cycle, the technique may
        be used as a conceptual approximation to the dynamics of
        ``market windows.\footnote{For example, consider the market
        for table salt. Since it has inelastic supply and demand
        curves, and is a necessary requirement for life, it would be
        expected that the Hurst coefficient would be very near
        unity---ignoring competitive pressures in the market. The
        predictability of the table salt market would, therefore, be
        expected to be relatively good, over time.}''  The conceptual
        approximation will probably predict a ``conservative'' or
        ``pessimistic'' value in relation to actual markets.

        \begin{figure}[ht]
            \begin{center}
                \begin{minipage}[t]{0.45\textwidth}
                    \epsfxsize=1.0\linewidth
                    \epsffile{\directory/datahurstlownear.eps}
                    \caption[{\market}, ``forecastability'' of near
                        term rate of revenue returns]{{\market},
                        ``forecastability'' of near term rate of
                        revenue returns. Although the error function
                        is the most accurate, for the near term,
                        $H^{t} = \thurstlow^{t}$ may be used as a
                        reliable metric of ``forecastability'' of the
                        rate of revenue returns.}
                    \label{\SETLABEL:FN}
                \end{minipage}
                \hfill
                \begin{minipage}[t]{0.45\textwidth}
                    \epsfxsize=1.0\linewidth
                    \epsffile{\directory/datahurstlowfar.eps}
                    \caption[{\market}, ``forecastability'' of far
                        term rate of revenue returns]{{\market},
                        ``forecastability'' of far term rate of
                        revenue returns. Although the error function
                        is the most accurate, for the far term,
                        $\frac{1}{\sqrt{t}}$ may be used as a reliable
                        metric of ``forecastability'' of the rate of
                        revenue returns.}
                    \label{\SETLABEL:FF}
                \end{minipage}
            \end{center}
        \end{figure}

        \idx{operations research}
        As an interesting interpretation of the data presented in
        Figure~\ref{\SETLABEL:FN}, there may be, perhaps, some
        applicability to such operational agendas as inventory
        control. Maintaining too little inventory, obviously, will
        create a situation where the organization can not exploit
        market expansion, and maintaining too much inventory,
        likewise, would over extend the company, creating unnecessary
        losses when the market contracts. The company should maintain
        inventory levels that do not exceed, from
        Equation~\ref{\SETLABEL:MA}, ${\thurstlow}^{n} = 0.5$
        {\timescale}s of operations. Since the optimal amount of
        inventory and, from Equation~\ref{\SETLABEL:V}, the variance
        of change in the rate of revenue returns in the future can be
        calculated, there may, perhaps, be some applicability to a
        forecasting methodology that can be incorporated into other
        areas of operations research, for example the linear algebras
        using simplex methodologies for optimization of manufacturing
        processes. Traditionally, these forecasts are made by the
        sales department, and are subject to various subjective
        biases.

% Local Variables:
% TeX-parse-self: t
% TeX-auto-save: t
% TeX-master: "fractal.tex"
% End:


        %
% -----------------------------------------------------------------------------
%
% A license is hereby granted to reproduce this software source code and
% to create executable versions from this source code for personal,
% non-commercial use.  The copyright notice included with the software
% must be maintained in all copies produced.
%
% THIS PROGRAM IS PROVIDED "AS IS". THE AUTHOR PROVIDES NO WARRANTIES
% WHATSOEVER, EXPRESSED OR IMPLIED, INCLUDING WARRANTIES OF
% MERCHANTABILITY, TITLE, OR FITNESS FOR ANY PARTICULAR PURPOSE.  THE
% AUTHOR DOES NOT WARRANT THAT USE OF THIS PROGRAM DOES NOT INFRINGE THE
% INTELLECTUAL PROPERTY RIGHTS OF ANY THIRD PARTY IN ANY COUNTRY.
%
% Copyright (c) 1994-2006, John Conover, All Rights Reserved.
%
% Comments and/or bug reports should be addressed to:
%
%     john@email.johncon.com (John Conover)
%
% -----------------------------------------------------------------------------
%
% Revision: \RCSRevision \\
% Revision Time: \RCSTime UMT \\
% Revision Date: \RCSDate \\
% Revision Id: \RCSId \\
% Revision File: \RCSLog \\
\RCS $Revision: 0.0 $
\RCS $Date: 2006/01/20 04:38:13 $
\RCS $Id: simulation.tex,v 0.0 2006/01/20 04:38:13 john Exp $
% $Log: simulation.tex,v $
% Revision 0.0  2006/01/20 04:38:13  john
% Initial version
%
%
    \subsection{Simulation of Fixed Increment Approximation for Fiscal Strategy}
        \label{\SETLABEL:TSUNFAIRBROWNIAN}

        \subidx{\market}{market simulation}
        The data in this section is presented in tabular form in
        Section~\ref{\SETLABELREF:SIM}.
        Figure~\ref{\SETLABEL:TSUNFAIRBROWNIAN0} represents a
        constructional simulation of the time series data presented in
        Figure~\ref{\SETLABEL:TS}. The program {\it
        tsunfairbrownian}\/, which is briefly described in
        appendix~\ref{programs}, was used in the reconstruction. The
        reconstructed data is superimposed on the original time series
        data.  The program, {\it tsunfairbrownian}\/, essentially,
        constructs the new time series as a Brownian fractal with
        fixed increments---the value of the fixed increment is derived
        from the root mean square average of the normalized increments
        presented in Figure~\ref{\SETLABEL:TF}. The ``quality'' of
        such a reconstruction should be subject to adequate scepticism
        and scrutiny since, in all probability, the normalized
        increments presented in Figure~\ref{\SETLABEL:TF} represent a
        relatively complex process, that may not be ``modeled'' with
        such a simple methodology.

        As a further comparison of the the constructional simulation
        with the original time series data,
        Figure~\ref{\SETLABEL:TSUNFAIRBROWNIAN1} presents a normalized
        histogram of the normalized increments of the reconstructed
        time series, superimposed on the normalized histogram
        presented in Figure~\ref{\SETLABEL:NH}.

        \subidx{\market}{fiscal strategy, simulation}
        \subidx{markets}{simulation}
        \subidx{simulation}{markets}
        \subidx{strategy}{fiscal, simulation}
        \subidx{fiscal}{strategy, simulation}
        \subidx{programs}{tsunfairbrownian}
        \subidx{tsunfairbrownian}{program}
        \begin{figure}[ht]
            \begin{center}
                \begin{minipage}[t]{0.45\textwidth}
                    \epsfxsize=1.0\linewidth
                    \epsffile{\directory/tsunfairbrownian-f.eps}
                    \caption[{\market}, Time series data, empirical and
                        simulated]{{\market}, Time series data, empirical
                        and simulated, using the program {\it tsunfairbrownian}\/
                        with f = {\datafractionrms}. This data is
                        superimposed on the data presented in
                        Figure~\ref{\SETLABEL:TS}.}
                    \label{\SETLABEL:TSUNFAIRBROWNIAN0}
                \end{minipage}
                \hfill
                \begin{minipage}[t]{0.45\textwidth}
                    \epsfxsize=1.0\linewidth
                    \epsffile{\directory/tsunfairbrownian-f.tsfraction.tsnormal-s30.eps}
                    \caption[{\market}, normalized histogram,
                        empirical and simulated]{{\market}, normalized
                        histogram of the normalized increments of the
                        time series data shown in
                        Figure~\ref{\SETLABEL:TSUNFAIRBROWNIAN0},
                        empirical and simulated.  The empirical data
                        has a mean of {\datafractionmean}, with a
                        standard deviation of {\datafractionstddev}.
                        By comparison, the simulated data has a mean
                        of {\tsunfairbrownianfractionmean} with a
                        standard deviation of
                        {\tsunfairbrownianfractionstddev}. This data
                        is superimposed on the data presented in
                        Figure~\ref{\SETLABEL:NH}. The area under the
                        four curves is identical.}
                    \label{\SETLABEL:TSUNFAIRBROWNIAN1}
                \end{minipage}
            \end{center}
        \end{figure}

% Local Variables:
% TeX-parse-self: t
% TeX-auto-save: t
% TeX-master: "fractal.tex"
% End:


        %
% -----------------------------------------------------------------------------
%
% A license is hereby granted to reproduce this software source code and
% to create executable versions from this source code for personal,
% non-commercial use.  The copyright notice included with the software
% must be maintained in all copies produced.
%
% THIS PROGRAM IS PROVIDED "AS IS". THE AUTHOR PROVIDES NO WARRANTIES
% WHATSOEVER, EXPRESSED OR IMPLIED, INCLUDING WARRANTIES OF
% MERCHANTABILITY, TITLE, OR FITNESS FOR ANY PARTICULAR PURPOSE.  THE
% AUTHOR DOES NOT WARRANT THAT USE OF THIS PROGRAM DOES NOT INFRINGE THE
% INTELLECTUAL PROPERTY RIGHTS OF ANY THIRD PARTY IN ANY COUNTRY.
%
% Copyright (c) 1994-2006, John Conover, All Rights Reserved.
%
% Comments and/or bug reports should be addressed to:
%
%     john@email.johncon.com (John Conover)
%
% -----------------------------------------------------------------------------
%
% Revision: \RCSRevision \\
% Revision Time: \RCSTime UMT \\
% Revision Date: \RCSDate \\
% Revision Id: \RCSId \\
% Revision File: \RCSLog \\
\RCS $Revision: 0.0 $
\RCS $Date: 2006/01/20 04:38:13 $
\RCS $Id: maximum.tex,v 0.0 2006/01/20 04:38:13 john Exp $
% $Log: maximum.tex,v $
% Revision 0.0  2006/01/20 04:38:13  john
% Initial version
%
%
    \subsection{Simulation of Fixed Increment Approximation for Optimally Maximal Fiscal Strategy}
        \label{\SETLABEL:MAXSHANNON}
        \subidx{\market}{fiscal strategy, simulation}
        \subidx{\market}{maximum Shannon probability}
        \subidx{markets}{simulation}
        \subidx{simulation}{markets}
        \subidx{strategy}{optimum fiscal, simulation}
        \subidx{fiscal}{optimum strategy, simulation}
        \subidx{programs}{tsunfairbrownian}
        \subidx{tsunfairbrownian}{program}
        \subidx{Shannon}{probability}
        \subidx{probability}{Shannon}

        \subidx{strategy}{exploitable}
        \subidx{exploitable}{strategy}
        \subidx{programs}{tsshannonmax}
        \subidx{tsshannonmax}{program}
        \subidx{programs}{tsunfairbrownian}
        \subidx{tsunfairbrownian}{program}
        \subidx{strategy}{fiscal}
        \subidx{fiscal}{strategy}
        The data in this section is presented in tabular form in
        Section~\ref{\SETLABELREF:MAXSHANNON}. One of the issues of
        analysis, as mentioned in Section~\ref{\SETLABEL:OPS}, is to
        determine the maximum Shannon probability for the time series
        presented in Figure~\ref{\SETLABEL:TS}. Potentially, this
        could be exploited with an aggressive fiscal
        strategy. Figure~\ref{\SETLABEL:SHANNONMAX0} is a graph of the
        output of the {\it tsshannonmax}\/ program, which is described
        briefly in appendix~\ref{programs}. The maximum of this
        function is the maximum Shannon probability for the time
        series data presented in Figure~\ref{\SETLABEL:TS}.
        Figure~\ref{\SETLABEL:SHANNONMAX1} was constructed using {\it
        tsunfairbrownian}\/ program, which is also described in
        appendix~\ref{programs}, with the maximum Shannon probability,
        and the time series data presented in
        Figure~\ref{\SETLABEL:TS}. This represents a ``what if'' the
        investment strategy was changed from a Shannon probability of
        {\shannonlogreturns}, as derived in Section~\ref{\SETLABEL:SP}
        to {\shannonmax}. This process, essentially, extracts the
        random statistical data from the time series presented in
        Figure~\ref{\SETLABEL:TS}, and constructs a new time series,
        using the random statistical data, with a different investment
        strategy.  The program, {\it tsunfairbrownian}\/, essentially,
        constructs the new time series as a Brownian fractal with
        fixed increments.  The ``quality'' of such a reconstruction
        should be subject to adequate scepticism and scrutiny since,
        in all probability, the increments in the original data
        represent a relatively complex process, that may not be
        ``modeled'' with such a simple methodology.

        \begin{figure}[ht]
            \begin{center}
                \begin{minipage}[t]{0.45\textwidth}
                    \epsfxsize=1.0\linewidth
                    \epsffile{\directory/data.tsshannonmax.eps}
                    \caption[{\market}, maximum rate of revenue
                        returns] {{\market}, maximum rate of revenue
                        returns, per {\timescale}, vs. Shannon
                        probability. The maximum rate of revenue
                        returns, per {\timescale}, occurs at a Shannon
                        probability of {\shannonmax}.}
                    \label{\SETLABEL:SHANNONMAX0}
                \end{minipage}
                \hfill
                \begin{minipage}[t]{0.45\textwidth}
                    \epsfxsize=1.0\linewidth
                    \epsffile{\directory/data.tsshannonmax-p.tsunfairbrownian-p.eps}
                    \caption[{\market}, maximum rate of revenue
                        returns] {{\market}, maximum rate of revenue
                        returns, per {\timescale}, at a Shannon
                        probability, of {\shannonmax}, corresponding
                        to a ``wager'' fraction of {\twoponemax}.}
                    \label{\SETLABEL:SHANNONMAX1}
                \end{minipage}
            \end{center}
        \end{figure}

        \subidx{fractional}{Brownian motion}
        \subidx{Brownian motion}{fractional}
        \subidx{Shannon}{probability}
        \subidx{probability}{Shannon}
        \subidx{programs}{tsshannonmax}
        \subidx{tsshannonmax}{program}
        If it is assumed that the time series data set, presented in
        Figure~\ref{\SETLABEL:TS}, constitutes classical Brownian
        motion, then the Shannon probability can be calculated by
        counting the total number of {\timescale}s that the {\market}
        movement was positive, and dividing by the total number of
        {timescale}s represented in the time series. This quotient is
        {\pmax}, as compared with the predicted value from the program
        {\it tsshannonmax}\/ of {\shannonmax}.

% Local Variables:
% TeX-parse-self: t
% TeX-auto-save: t
% TeX-master: "fractal.tex"
% End:


        %
% -----------------------------------------------------------------------------
%
% A license is hereby granted to reproduce this software source code and
% to create executable versions from this source code for personal,
% non-commercial use.  The copyright notice included with the software
% must be maintained in all copies produced.
%
% THIS PROGRAM IS PROVIDED "AS IS". THE AUTHOR PROVIDES NO WARRANTIES
% WHATSOEVER, EXPRESSED OR IMPLIED, INCLUDING WARRANTIES OF
% MERCHANTABILITY, TITLE, OR FITNESS FOR ANY PARTICULAR PURPOSE.  THE
% AUTHOR DOES NOT WARRANT THAT USE OF THIS PROGRAM DOES NOT INFRINGE THE
% INTELLECTUAL PROPERTY RIGHTS OF ANY THIRD PARTY IN ANY COUNTRY.
%
% Copyright (c) 1994-2006, John Conover, All Rights Reserved.
%
% Comments and/or bug reports should be addressed to:
%
%     john@email.johncon.com (John Conover)
%
% -----------------------------------------------------------------------------
%
% Revision: \RCSRevision \\
% Revision Time: \RCSTime UMT \\
% Revision Date: \RCSDate \\
% Revision Id: \RCSId \\
% Revision File: \RCSLog \\
\RCS $Revision: 0.0 $
\RCS $Date: 2006/01/20 04:38:13 $
\RCS $Id: verification.tex,v 0.0 2006/01/20 04:38:13 john Exp $
% $Log: verification.tex,v $
% Revision 0.0  2006/01/20 04:38:13  john
% Initial version
%
%
    \subsection{Qualitative Verification of Fixed Increment Approximation Analysis}
        \label{\SETLABEL:QVA}

        \subidx{\market}{verification of analysis}
        \subidx{verification}{analysis}
        \subidx{analysis}{verification}
        \subidx{quality}{of analysis}
        \subidx{verification}{of methodology}
        \subidx{methodology}{verification of}
        \subidx{Shannon}{probability}
        \subidx{probability}{Shannon}

        This section evaluates various values based on the ``average''
        of the normalized increments presented in
        Figure~\ref{\SETLABEL:TFA}. These values are an approximation
        to a, probably, complex process with a distribution shown in
        Figure~\ref{\SETLABEL:TF}. These values will be used in a
        fixed increment Brownian fractal analysis of the {\market},
        and may, or may not, provide adequate accuracy for
        projections.

        The data in this section is presented in tabular form in
        sections~\ref{\SETLABELREF:VI1} and~\ref{\SETLABELREF:VI2}.
        As a subjective evaluation of the ``quality'' of the analysis
        of the {\market}, from Chapter~\ref{methodology},
        Equation~\ref{metricvalues1}, and using the mean and root mean
        square values of the normalized increments of the time series
        data presented in Figure~\ref{\SETLABEL:TS} from
        Figure~\ref{\SETLABEL:TF}, and the Shannon probability as
        calculated by counting the total number of {\timescale}s that
        the {\market} movement was positive, as presented in
        Section~\ref{\SETLABEL:MAXSHANNON}:

        \begin{eqnarray}
                  P & \approx & \frac{\frac{avg}{rms} + 1}{2}\\
            {\pmax} & \approx & \frac{\frac{\datafractionmean}{\datafractionrms} + 1}{2}\\
            {\pmax} & \approx & {\avgrms}
            \label{\SETLABEL:AVGS}
        \end{eqnarray}

        \subidx{Shannon}{probability}
        \subidx{probability}{Shannon}
        \noindent and comparing these values to the Shannon
        probability, as found by the {\it tsshannonmax}\/ program, which
        iterates for a maximum:

        \begin{eqnarray}
            {\pmax} \approx {\avgrms} \approx {\shannonmax}
        \end{eqnarray}

        \subidx{logarithmic}{returns}
        \subidx{returns}{logarithmic}
        In addition, the different methods of calculating the
        logarithmic returns, presented in Section~\ref{\SETLABEL:FS},
        should be compared. The four methods used were the mean of
        Figure~\ref{\SETLABEL:TF}, the constant in the least squares
        approximation to Figure~\ref{\SETLABEL:TF}, the least squares
        exponential approximation to Figure~\ref{\SETLABEL:TS}, and
        the logarithmic returns of Figure~\ref{\SETLABEL:TS}, derived
        as the mean of the logarithms of the quotients of the
        increments. The values for each of the methods are,
        respectively:

        \begin{equation}
            \datafractionmeanbits \approx \datafractionconstantbits \approx \datatslsqepbits \approx \logreturns
        \end{equation}

        It is implied in Section~\ref{\SETLABEL:FS},
        Subsection~\ref{\SETLABEL:SP} and in
        Section~\ref{\SETLABEL:TSUNFAIRBROWNIAN} that, a Brownian
        motion with fixed increments fractal may ``model'' the
        {\market}. Using Equation~\ref{stddev9} from
        Chapter~\ref{general}, Section~\ref{abmfi}:

        \begin{eqnarray}
                                    rms \left(2P - 1\right) & \approx & \frac{\sigma \left(2P - 1\right)}{2 \sqrt{P\left(1 - P\right)}}\\
            \datafractionrms \left(2 \cdot \pmax - 1\right) & \approx & \frac{\datafractionstddev \left(2 \cdot \pmax - 1\right)}{2\sqrt{\pmax \left(1 - \pmax\right)}}\\
                       \datafractionrms \cdot \twopminusone & \approx & \datafractionstddev \cdot \twopx\\
                                                      \rmsp & \approx & \sigmap
        \end{eqnarray}

        \noindent and, equating to the mean:

        \begin{equation}
            \datafractionmean \approx \rmsp \approx \sigmap
        \end{equation}

        \subidx{Shannon}{probability}
        \subidx{probability}{Shannon}
        \noindent where, as in Equation~\ref{\SETLABEL:AVGS} using the
        mean, root mean square, and standard deviation values of the
        normalized increments of the time series data presented in
        Figure~\ref{\SETLABEL:TS} from Figure~\ref{\SETLABEL:TF}, and
        the Shannon probability as calculated by counting the total
        number of {\timescale}s that the {\market} movement was
        positive, as presented in Section~\ref{\SETLABEL:MAXSHANNON}.

        As a final qualitative comparison, the absolute value of the
        normalized increments should be the same as the root mean
        square value\footnote{The absolute value of the normalized
        increments, when averaged, is related to the root mean square
        of the increments by a constant. If the normalized increments
        are a fixed increment, the constant is unity. If the
        normalized increments have a Gaussian distribution, the
        constant is $\approx 0.8$ depending on the accuracy of of
        ``fit'' to a Gaussian distribution.}, where the absolute value
        is presented in Figure~\ref{\SETLABEL:TFA}, and the root mean
        square value is presented in Figure~\ref{\SETLABEL:TF}:

        \begin{equation}
            \datafractionabsmean \approx \datafractionrms
        \end{equation}

        Note, that if the {\market} could be ``modeled'' as a Brownian
        motion with fixed increments fractal, then the standard
        deviation of the absolute value of the normalized increments
        of the time series data presented in Figure~\ref{\SETLABEL:TS}
        from Figure~\ref{\SETLABEL:TF} should be zero. It is
        $\datafractionabsstddev$.

% Local Variables:
% TeX-parse-self: t
% TeX-auto-save: t
% TeX-master: "fractal.tex"
% End:


    \renewcommand{\market}{United States Gross Domestic Product}
    \renewcommand{\directory}{../markets/us.gdp}
    \renewcommand{\datafractionmean}{0.008052}
\renewcommand{\datafractionmeanbits}{0.011570}
\renewcommand{\datafractionmeanq}{0.002684}
\renewcommand{\datafractionmeanbitsq}{0.003867}
\renewcommand{\datafractionstddev}{0.038579}
\renewcommand{\datafractionrms}{0.039311}
\renewcommand{\avgrms}{0.602414}
\renewcommand{\ncompanies}{5.210454}
\renewcommand{\pncompanies}{0.544866}
\renewcommand{\datafractionabsmean}{0.029745}
\renewcommand{\datafractionabsstddev}{0.025769}
\renewcommand{\datafractionconstant}{0.010041}
\renewcommand{\datafractionconstantbits}{0.014414}
\renewcommand{\datafractionconstantq}{0.003347}
\renewcommand{\datafractionconstantbitsq}{0.004821}
\renewcommand{\datafractionslope}{-0.000021}
\renewcommand{\datafractionabsconstant}{0.035145}
\renewcommand{\datafractionabsslope}{-0.000057}
\renewcommand{\hurstall}{0.659558}
\renewcommand{\hurstlow}{0.707509}
\renewcommand{\hurstlowtwo}{1.415018}
\renewcommand{\hurstlowhundred}{70.750900}
\renewcommand{\hcalcall}{0.184942}
\renewcommand{\hcalclow}{0.102042}
\renewcommand{\shannonmax}{0.604167}
\renewcommand{\twoponemax}{0.208334}
\renewcommand{\logreturns}{0.010456}
\renewcommand{\twologreturns}{1.007274}
\renewcommand{\twologreturnshundred}{0.727387}
\renewcommand{\oneoverlogreturns}{95.638868}
\renewcommand{\pmax}{0.602094}
\renewcommand{\twopminusone}{0.204188}
\renewcommand{\rmsp}{0.008027}
\renewcommand{\twopx}{0.208583}
\renewcommand{\sigmap}{0.008047}
\renewcommand{\tsunfairbrownianfractionmean}{0.007862}
\renewcommand{\tsunfairbrownianfractionstddev}{0.038619}
\renewcommand{\shannonlogreturns}{0.560125}
\renewcommand{\shannonlogreturnshundred}{56.012500}
\renewcommand{\twopone}{0.120250}
\renewcommand{\twoponehundred}{12.025000}
\renewcommand{\hundredtwoponehundred}{87.975000}
\renewcommand{\hundredshannonlogreturnshundred}{43.987500}
\renewcommand{\datatslsqepbits}{0.007623}
\renewcommand{\thurstall}{0.633980}
\renewcommand{\thurstlow}{0.710108}
\renewcommand{\thurstlowtwo}{1.420216}
\renewcommand{\thurstlowhundred}{71.010800}
\renewcommand{\thcalcall}{0.247886}
\renewcommand{\thcalclow}{0.171737}
\renewcommand{\chisquared}{2.862000}
\renewcommand{\critical}{42.557000}

    \renewcommand{\timescale}{month}
    \subidx{market}{\market}
    \idx{\market}

    \section{\market}

        \renewcommand{\SETLABEL}{\LABPRE:USGDP}
        \renewcommand{\SETLABELQ}{\LABPRE:USGDPQ}
        \label{\SETLABEL}
        \renewcommand{\SETLABELREF}{\LABPREREF:USGDP}

        \idx{United States Department of Commerce}
        For the analysis, the data was in the directory
        {\directory}\footnote{Data from the United States Department
        of Commerce, 1979---1994, by {\timescale}s, in billions of
        1987 dollars, US.}.

        The data in this section is presented in tabular form in
        Section~\ref{\SETLABELREF}. Note that in this analysis, the
        rate of revenue returns means the increase or decrease in the
        {\market}. This is included for comparative purposes.

        %
% -----------------------------------------------------------------------------
%
% A license is hereby granted to reproduce this software source code and
% to create executable versions from this source code for personal,
% non-commercial use.  The copyright notice included with the software
% must be maintained in all copies produced.
%
% THIS PROGRAM IS PROVIDED "AS IS". THE AUTHOR PROVIDES NO WARRANTIES
% WHATSOEVER, EXPRESSED OR IMPLIED, INCLUDING WARRANTIES OF
% MERCHANTABILITY, TITLE, OR FITNESS FOR ANY PARTICULAR PURPOSE.  THE
% AUTHOR DOES NOT WARRANT THAT USE OF THIS PROGRAM DOES NOT INFRINGE THE
% INTELLECTUAL PROPERTY RIGHTS OF ANY THIRD PARTY IN ANY COUNTRY.
%
% Copyright (c) 1994-2006, John Conover, All Rights Reserved.
%
% Comments and/or bug reports should be addressed to:
%
%     john@email.johncon.com (John Conover)
%
% -----------------------------------------------------------------------------
%
% Revision: \RCSRevision \\
% Revision Time: \RCSTime UMT \\
% Revision Date: \RCSDate \\
% Revision Id: \RCSId \\
% Revision File: \RCSLog \\
\RCS $Revision: 0.0 $
\RCS $Date: 2006/01/20 04:38:13 $
\RCS $Id: fraction.tex,v 0.0 2006/01/20 04:38:13 john Exp $
% $Log: fraction.tex,v $
% Revision 0.0  2006/01/20 04:38:13  john
% Initial version
%
%
    \subsection{Time Series Increments Analysis}
        \label{\SETLABEL:TSA}

        \subidx{\market}{Time series analysis}
        \subidx{time series}{increments}
        \subidx{time series}{analysis}
        \subidx{cumulative sum}{analysis}
        \subidx{analysis}{cumulative sum}
        \subidx{analysis}{random process}
        \subidx{random process}{analysis}
        \subidx{Gaussian}{increments}
        \subidx{increments}{Gaussian}
        \subidx{Brownian}{motion, fractional}
        \subidx{fractional}{Brownian motion}
        \subidx{fractal}{Brownian motion}
        The data in this section is presented in tabular form in
        Section~\ref{\SETLABELREF:TSA}.  Figure~\ref{\SETLABEL:TS} is
        a graph of the time series data for the {\market}.

        \subidx{increments}{normalized}
        \subidx{normalized}{increments}
        \subidx{programs}{tsfraction}
        \subidx{tsfraction}{program}
        Figure~\ref{\SETLABEL:TF} is a graph of the normalized
        increments of the time series data presented in
        Figure~\ref{\SETLABEL:TS}. The data presented was made by
        running the program {\it tsfraction}\/ on the time series
        data. The program {\it tsfraction}\/ is described briefly in
        Appendix~\ref{programs}, and subtracts the previous value from
        the next value, dividing this difference by the previous
        value, for each element in the time series data. The new time
        series contains the instantaneous change in the rate of
        revenue returns, divided by the magnitude of the instantaneous
        rate of revenue returns.

        \subidx{mean}{standard deviation}
        \subidx{standard deviation}{mean}
        \idx{root mean square}
        \idx{least squares approximation}
        \begin{figure}[ht]
            \begin{center}
                \begin{minipage}[t]{0.45\textwidth}
                    \epsfxsize=1.0\linewidth
                    \epsffile{\directory/data.eps}
                    \caption{{\market}, time series data.}
                    \label{\SETLABEL:TS}
                    \label{\SETLABELQ:TS}
                \end{minipage}
                \hfill
                \begin{minipage}[t]{0.45\textwidth}
                    \epsfxsize=1.0\linewidth
                    \epsffile{\directory/data.tsfraction.eps}
                    \caption[{\market}, normalized
                        increments]{{\market}, normalized increments
                        of the time series data presented in
                        Figure~\ref{\SETLABEL:TS}. The mean is
                        {\datafractionmean} with a standard deviation
                        of {\datafractionstddev}. The formula for the
                        least squares approximation is
                        ${\datafractionconstant} +
                        {\datafractionslope}t$, and the root mean
                        squared value is {\datafractionrms}. The
                        graph, labeled ``data\-.tsfraction\-.tsrms,''
                        is the running root mean square, and
                        ``data\-.tsfraction\-.tsavg'' is the running
                        average of the normalized increments.  This
                        graph is the fraction of change in the time
                        series, as a function of time. Note that the
                        slope of the mean, {\datafractionslope}, is
                        the coefficient of the nonlinearity term in
                        the normalized increments. See
                        Chapter~\ref{general}, Section~\ref{nlextend}
                        for a possible application of the logistic
                        function to this data set.}
                    \label{\SETLABEL:TF}
                    \label{\SETLABELQ:TF}
                \end{minipage}
            \end{center}
        \end{figure}

        \subidx{absolute value}{increments}
        \subidx{increments}{absolute value}

        Figure~\ref{\SETLABEL:TFA} is a graph of the absolute value of
        the normalized increments of the time series data presented in
        Figure~\ref{\SETLABEL:TF}. The data presented was made by
        running the Unix utility sed(1) on the normalized increments
        time series data to remove the negative signs. This is an
        absolute value procedure.  The resulting time series contains
        the absolute value of the instantaneous change in the rate of
        revenue returns, divided by the magnitude of the instantaneous
        rate of revenue returns\footnote{The absolute value of the
        normalized increments, when averaged, is related to the root
        mean square of the increments by a constant. If the normalized
        increments are a fixed increment, the constant is unity. If
        the normalized increments have a Gaussian distribution, the
        constant is $\approx 0.8$ depending on the accuracy of of
        ``fit'' to a Gaussian distribution.}.

        \subidx{histogram}{normalized}
        \subidx{normalized}{histogram}
        \subidx{programs}{tsnormal}
        \subidx{tsnormal}{program}
        \subidx{mean}{standard deviation}
        \subidx{standard deviation}{mean}
        \idx{root mean square}
        \idx{least squares approximation}
        \subidx{\market}{analysis of increments}
        Figure~\ref{\SETLABEL:NH} is the normalized histogram of the
        normalized increments of the time series data shown in
        Figure~\ref{\SETLABEL:TF}. The abscissa is 3 $\sigma$ limits,
        and the area under the two curves is identical. The data for
        this figure was produced by the program {\it tsnormal}\/,
        which is described briefly in Appendix~\ref{programs}.

        \begin{figure}[ht]
            \begin{center}
                \begin{minipage}[t]{0.45\textwidth}
                    \epsfxsize=1.0\linewidth
                    \epsffile{\directory/data.tsfraction.abs.eps}
                    \caption[{\market}, absolute value of the
                        normalized increments]{{\market}, absolute
                        value of the normalized increments of the time
                        series data presented in
                        Figure~\ref{\SETLABEL:TF}.  The mean is
                        {\datafractionabsmean} with a standard
                        deviation of {\datafractionabsstddev}. The
                        formula for the least squares approximation is
                        ${\datafractionabsconstant} +
                        {\datafractionabsslope}t$, and the root mean
                        square value, from Figure~\ref{\SETLABEL:TF},
                        is {\datafractionrms}.  The graph, labeled
                        ``data\-.tsfraction\-.tsrms,'' is the running
                        root mean square, and
                        ``data\-.tsfraction\-.tsavg'' is the running
                        average of the normalized increments presented
                        in Figure~\ref{\SETLABEL:TF}, superimposed
                        here for convenience. This graph is the
                        absolute value of the fraction of change in
                        the time series, as a function of time.}
                    \label{\SETLABEL:TFA}
                    \label{\SETLABELQ:TFA}
                \end{minipage}
                \hfill
                \begin{minipage}[t]{0.45\textwidth}
                    \epsfxsize=1.0\linewidth
                    \epsffile{\directory/data.tsfraction.tsnormal-s30.eps}
                    \caption[{\market}, normalized histogram of the
                        normalized increments]{{\market}, normalized
                        histogram of the normalized increments of the
                        time series data shown in
                        Figure~\ref{\SETLABEL:TF}.  The data has a
                        mean of {\datafractionmean}, with a standard
                        deviation of {\datafractionstddev}.  The area
                        under the two curves is identical. The
                        $\chi^2$ value of the observed and expected
                        values of the two curves is {\chisquared},
                        with a critical value of {\critical}.}
                    \label{\SETLABEL:NH}
                \end{minipage}
            \end{center}
        \end{figure}

        \subidx{programs}{tsXsquared}
        \subidx{tsXsquared}{program}
        \subidx{\market}{chi-squared values of increments}
        The program {\it tsXsquared}\/, which is briefly described in
        appendix~\ref{programs}, was used to derive the $\chi^2$
        statistics for the data presented in
        Figure~\ref{\SETLABEL:NH}.

        \subidx{programs}{tsstatest}
        \subidx{tsstatest}{program}
        \subidx{\market}{statistical estimates}

        Figure~\ref{\SETLABEL:SE} is the statistical estimate for the
        data presented in Figure~\ref{\SETLABEL:TF}, as derived by the
        program {\it tsstatest}\/, which is briefly described in
        appendix~\ref{programs}.

        \begin{figure}[ht]
            \begin{center}
                \begin{minipage}[t]{\textwidth}
                    \center{\fbox{\parbox{0.9\textwidth}{\XXX{\directory/data.tsstatest-f0.1-c0.9-i.tex}}}}
                    \caption[{\market}, statistical estimates of the
                        normalized increments]{{\market}, statistical
                        estimates of the normalized increments of the
                        time series shown in Figure~\ref{\SETLABEL:TF}.
                        The table was produced with the {\it
                        tsstatest}\/ program, and illustrates the
                        size of the data set required for a confidence
                        level of 90\%, with an error estimate of $\pm$
                        10\%, or alternately, the error estimate on
                        the time series shown in Figure~\ref{\SETLABEL:TF}.}
                    \label{\SETLABEL:SE}
                \end{minipage}
            \end{center}
        \end{figure}

        Note that the data set size estimations, as produced by the
        {\it tsstatest}\/ program, are probably very conservative,
        depending on the magnitude of the Shannon probability, $P =
        \shannonlogreturns$, as derived in
        Section~\ref{\SETLABEL:SP}. See Chapter~\ref{general},
        Section~\ref{serdss} for possible alternative methodologies
        for addressing the analysis of fractal time series with
        limited data set sizes. Depending on the magnitude of the
        Shannon probability, $P$, these estimates can be several
        orders of magnitude too high.

        \subidx{derivative of increments}{normalized}
        \subidx{normalized}{derivative of increments}
        \subidx{programs}{tsderivative}
        \subidx{tsderivative}{program}
        Figure~\ref{\SETLABEL:TF1} is the normalized histogram of the
        first derivative of the normalized increments of the time
        series data shown in Figure~\ref{\SETLABEL:TF}. In principle,
        if the distribution of the normalized increments presented in
        Figure~\ref{\SETLABEL:NH} is Gaussian in nature, this
        distribution would be similar to ``white noise,'' as presented
        in appendix~\ref{programs}, Figure~\ref{whiteexample}. The
        data was generated by the {\it tsderivative}\/ program, which
        is briefly described in
        appendix~\ref{programs}. Figure~\ref{\SETLABEL:TF2} is the
        normalized histogram of the second derivative of the
        normalized increments of the time series data shown in
        Figure~\ref{\SETLABEL:TF}. In principle, if the distribution
        of the normalized increments presented in
        Figure~\ref{\SETLABEL:NH} is an integrated Gaussian
        distribution in nature, this distribution would be similar to
        ``white noise,'' as presented in appendix~\ref{programs},
        Figure~\ref{whiteexample}.

        \begin{figure}[ht]
            \begin{center}
                \begin{minipage}[t]{0.45\textwidth}
                    \epsfxsize=1.0\linewidth
                    \epsffile{\directory/data.tsfraction.tsderivative.tsnormal-s30.eps}
                    \caption[{\market}, histogram of the first
                        derivative of the increments]{{\market},
                        normalized histogram of the first derivative
                        of the normalized increments of the time
                        series data shown in
                        Figure~\ref{\SETLABEL:TF}.}
                    \label{\SETLABEL:TF1}
                \end{minipage}
                \hfill
                \begin{minipage}[t]{0.45\textwidth}
                    \epsfxsize=1.0\linewidth
                    \epsffile{\directory/data.tsfraction.2tsderivative.tsnormal-s30.eps}
                    \caption[{\market}, histogram of the second
                        derivative of the increments]{{\market},
                        normalized histogram of second derivative of
                        the the normalized increments of the time
                        series data shown in
                        Figure~\ref{\SETLABEL:TF}.}
                    \label{\SETLABEL:TF2}
                \end{minipage}
            \end{center}
        \end{figure}

        \subidx{fractal}{range}
        \subidx{fractal}{R/S analysis}
        \subidx{\market}{rate of revenue returns, range}
        \subidx{\market}{deterministic mechanism}
        \subidx{deterministic}{mechanism}
        \subidx{mechanism}{deterministic}
        Figure~\ref{\SETLABEL:TR} is the range of values of the time
        series shown in Figure~\ref{\SETLABEL:TS}. The horizontal axis
        is time into the future. In principle, if the time series was
        characterized as fractional Brownian motion the graph in
        Figure~\ref{\SETLABEL:TR} would be a square root
        function\footnote{Note that the ``roughness,'' or ``sawtooth''
        characteristics of the graph in Figure~\ref{\SETLABEL:TR} are
        a computational artifact---caused by not using the -m option
        to the program {\it tshurst}\/, which is computationally
        inefficient.}. Figure~\ref{\SETLABEL:TD} is the deterministic
        map of the normalized increments of the time series data shown
        in Figure~\ref{\SETLABEL:TF}. The deterministic map is useful
        for determining if a time series was created by a
        deterministic mechanism. This, essentially, maps each element
        in the time series with the previous element in the time
        series.  See,~\cite[pp. 745]{Peitgen}.

        \begin{figure}[ht]
            \begin{center}
                \begin{minipage}[t]{0.45\textwidth}
                    \epsfxsize=1.0\linewidth
                    \epsffile{\directory/data.tshurst-f.eps}
                    \caption[{\market}, range]{{\market}, range of the
                        time series data shown in
                        Figure~\ref{\SETLABEL:TS}.}
                    \label{\SETLABEL:TR}
                \end{minipage}
                \hfill
                \begin{minipage}[t]{0.45\textwidth}
                    \epsfxsize=1.0\linewidth
                    \epsffile{\directory/data.tsfraction.tsdeterministic.eps}
                    \caption[{\market}, deterministic map]{{\market},
                        deterministic map of the normalized increments
                        of the time series data shown in
                        Figure~\ref{\SETLABEL:TF}.}
                    \label{\SETLABEL:TD}
                \end{minipage}
            \end{center}
        \end{figure}

% Local Variables:
% TeX-parse-self: t
% TeX-auto-save: t
% TeX-master: "fractal.tex"
% End:


        \subsubsection{Observations on the Time Series Increments Analysis}

            Figure~\ref{\SETLABEL:NH} would seem to indicate that the
            time series data for the {\market} represents a cumulative
            sum/integration of a random process that has a Gaussian
            distribution, (ie., satisfies the Gaussian increments
            property of fractional Brownian
            motion~\cite[pp. 250]{Crownover},) tending to justify the
            assumption that the time series data represents fractional
            Brownian motion.

        %
% -----------------------------------------------------------------------------
%
% A license is hereby granted to reproduce this software source code and
% to create executable versions from this source code for personal,
% non-commercial use.  The copyright notice included with the software
% must be maintained in all copies produced.
%
% THIS PROGRAM IS PROVIDED "AS IS". THE AUTHOR PROVIDES NO WARRANTIES
% WHATSOEVER, EXPRESSED OR IMPLIED, INCLUDING WARRANTIES OF
% MERCHANTABILITY, TITLE, OR FITNESS FOR ANY PARTICULAR PURPOSE.  THE
% AUTHOR DOES NOT WARRANT THAT USE OF THIS PROGRAM DOES NOT INFRINGE THE
% INTELLECTUAL PROPERTY RIGHTS OF ANY THIRD PARTY IN ANY COUNTRY.
%
% Copyright (c) 1994-2006, John Conover, All Rights Reserved.
%
% Comments and/or bug reports should be addressed to:
%
%     john@email.johncon.com (John Conover)
%
% -----------------------------------------------------------------------------
%
% Revision: \RCSRevision \\
% Revision Time: \RCSTime UMT \\
% Revision Date: \RCSDate \\
% Revision Id: \RCSId \\
% Revision File: \RCSLog \\
\RCS $Revision: 0.0 $
\RCS $Date: 2006/01/20 04:38:13 $
\RCS $Id: instant.tex,v 0.0 2006/01/20 04:38:13 john Exp $
% $Log: instant.tex,v $
% Revision 0.0  2006/01/20 04:38:13  john
% Initial version
%
%
    \subsection{Instantaneous Analysis of Normalized Increments}
        \label{\SETLABEL:IA}

        \subidx{\market}{instantaneous analysis of normalized increments}
        \idx{average of normalized increments}
        \idx{root mean square of normalized increments}
        \subidx{Shannon probability}{instantaneous computation of}
        \subidx{average of normalized increments}{instantaneous computation of}
        \subidx{root mean square of normalized increments}{instantaneous computation of}
        \subidx{instantaneous computation}{Shannon probability}
        \subidx{instantaneous computation}{average of normalized increments}
        \subidx{instantaneous computation}{root mean square of normalized increments}
        \idx{time series}
        \subidx{time series}{instantaneous analysis}
        \subidx{instantaneous analysis}{time series}
        \subidx{time series}{increments}
        \subidx{time series}{analysis}
        \subidx{Shannon}{probability}
        \subidx{probability}{Shannon}
        \subidx{normalized}{increments}
        \subidx{increments}{normalized}

        The program {\it tsinstant}\/, which is briefly described in
        Appendix~\ref{programs}, is for finding the instantaneous
        fraction of change in a time series. The value of a sample in
        the time series is subtracted from the previous sample in the
        time series, and divided by the value of the previous sample.
        As explained in Chapter~\ref{general},
        Sections~\ref{derivation},~\ref{GA},~\ref{abmfi},~\ref{aftsma}
        and,~\ref{ompl} for Brownian motion, random walk fractals, the
        absolute value of the instantaneous fraction of change is also
        the root mean square of the instantaneous fraction of
        change\footnote{The absolute value of the normalized
        increments, when averaged, is related to the root mean square
        of the increments by a constant. If the normalized increments
        are a fixed increment, the constant is unity. If the
        normalized increments have a Gaussian distribution, the
        constant is $\approx 0.8$ depending on the accuracy of of
        ``fit'' to a Gaussian distribution.}. Squaring this value is
        the average of the instantaneous fraction of change, and
        adding unity to the absolute value of the instantaneous
        fraction of change, and dividing by two, is the Shannon
        probability of the instantaneous fraction of change.

        Figure~\ref{\SETLABEL:IA1} is the instantaneous value of the
        root mean square of the normalized increments for the
        {\market}, and Figure~\ref{\SETLABEL:IA2} is the instantaneous
        Shannon probability for the normalized increments.

        \begin{figure}[ht]
            \begin{center}
                \begin{minipage}[t]{0.45\textwidth}
                    \epsfxsize=1.0\linewidth
                    \epsffile{\directory/data.tsinstant-r.eps}
                    \caption[{\market}, instantaneous value of
                        rms.]{{\market}, instantaneous value of the
                        root mean square of the normalized increments,
                        provided by running the program {\it
                        tsinstant}\/ with the -r option on the data
                        presented in Figure~\ref{\SETLABEL:TS}.}
                    \label{\SETLABEL:IA1}
                    \label{\SETLABELQ:IA1}
                \end{minipage}
                \hfill
                \begin{minipage}[t]{0.45\textwidth}
                    \epsfxsize=1.0\linewidth
                    \epsffile{\directory/data.tsinstant-s.eps}
                    \caption[{\market}, instantaneous value of
                        Shannon probability.]{{\market}, instantaneous
                        value of the Shannon probability of the
                        normalized increments, provided by running the
                        program {\it tsinstant}\/ with the -s option
                        on the data presented in
                        Figure~\ref{\SETLABEL:TS}.}
                    \label{\SETLABEL:IA2}
                    \label{\SETLABELQ:IA2}
                \end{minipage}
            \end{center}
        \end{figure}

% Local Variables:
% TeX-parse-self: t
% TeX-auto-save: t
% TeX-master: "fractal.tex"
% End:


        %
% -----------------------------------------------------------------------------
%
% A license is hereby granted to reproduce this software source code and
% to create executable versions from this source code for personal,
% non-commercial use.  The copyright notice included with the software
% must be maintained in all copies produced.
%
% THIS PROGRAM IS PROVIDED "AS IS". THE AUTHOR PROVIDES NO WARRANTIES
% WHATSOEVER, EXPRESSED OR IMPLIED, INCLUDING WARRANTIES OF
% MERCHANTABILITY, TITLE, OR FITNESS FOR ANY PARTICULAR PURPOSE.  THE
% AUTHOR DOES NOT WARRANT THAT USE OF THIS PROGRAM DOES NOT INFRINGE THE
% INTELLECTUAL PROPERTY RIGHTS OF ANY THIRD PARTY IN ANY COUNTRY.
%
% Copyright (c) 1994-2006, John Conover, All Rights Reserved.
%
% Comments and/or bug reports should be addressed to:
%
%     john@email.johncon.com (John Conover)
%
% -----------------------------------------------------------------------------
%
% Revision: \RCSRevision \\
% Revision Time: \RCSTime UMT \\
% Revision Date: \RCSDate \\
% Revision Id: \RCSId \\
% Revision File: \RCSLog \\
\RCS $Revision: 0.0 $
\RCS $Date: 2006/01/20 04:38:13 $
\RCS $Id: logistic.tex,v 0.0 2006/01/20 04:38:13 john Exp $
% $Log: logistic.tex,v $
% Revision 0.0  2006/01/20 04:38:13  john
% Initial version
%
%
    \subsection{Logistic Analysis}
        \label{\SETLABEL:LA}

        \subidx{\market}{Logistic function analysis}
        \subidx{time series}{logistic function}
        \subidx{logistic function}{time series}
        \subidx{time series}{increments}
        \subidx{time series}{analysis}
        \subidx{cumulative sum}{analysis}
        \subidx{analysis}{cumulative sum}
        \subidx{analysis}{random process}
        \subidx{random process}{analysis}
        The data in this section is presented in tabular form in
        Section~\ref{\SETLABELREF:LAA}.  Figure~\ref{\SETLABEL:LA1} is
        a graph of the logistic function estimates of the time series
        data for the {\market}. The reader is cautioned that these
        graphs are constructed using the method suggested in
        Chapter~\ref{general}, Section~\ref{nlextend} and enormous
        precision is required for adequate prediction of the logistic
        function,~\cite{Modis}. Particularly, the non-linear term will
        usually require intervention to produce a practical fit to the
        data. In addition, there are numerical stability issues with
        logistic function methodologies\footnote{For example, in
        Figures~\ref{\SETLABEL:LA1} and~\ref{\SETLABEL:LA2}, if the
        non-linear term, $b$, was greater than zero, it was set to
        zero to produce the graphs. See Section~\ref{\SETLABELREF:LAA}
        for the actual derived values. In other cases, the magnitude
        of $b$ was too large, resulting in a graph that was decreasing
        as a function of time}.  The methodology should be regarded as
        ``fragile.'' It is included for completeness.

        \idx{least squares approximation}
        Figure~\ref{\SETLABEL:LA1} is a graph of the logistic function
        for the time series data presented in
        Figure~\ref{\SETLABEL:TS}. The data presented was made by
        running the program {\it tsdlogistic}\/, which is described
        briefly in Appendix~\ref{programs}, on the parameters
        extracted from the time series data as suggested in
        Figure~\ref{\SETLABEL:TF}. The program {\it tslsq}\/ was used
        to derive the constant and the slope of the normalized
        increments of the data presented in Figure~\ref{\SETLABEL:TF}.
        Figure~\ref{\SETLABEL:LA2} is the same graph, but with the
        time scale expanded by a factor of two.

        \begin{figure}[ht]
            \begin{center}
                \begin{minipage}[t]{0.45\textwidth}
                    \epsfxsize=1.0\linewidth
                    \epsffile{\directory/data.tsfraction.tslsq-p.tsdlogistic.eps}
                    \caption[{\market}, logistic function
                        estimates.]{{\market}, logistic function
                        estimates, provided by running the {\it
                        tslsq}\/ program on the normalized increments
                        presented in Figure~\ref{\SETLABEL:TF} with
                        the -p option. These parameters were used as
                        arguments to the {\it tsdlogistic}\/ program.}
                    \label{\SETLABEL:LA1}
                    \label{\SETLABELQ:LA1}
                \end{minipage}
                \hfill
                \begin{minipage}[t]{0.45\textwidth}
                    \epsfxsize=1.0\linewidth
                    \epsffile{\directory/data.tsfraction.tslsq-p.tsdlogistic2.eps}
                    \caption[{\market}, logistic function
                        estimates.]{{\market}, logistic function
                        estimates of Figure~\ref{\SETLABEL:LA1} with
                        the time scale expanded by a factor of two.}
                    \label{\SETLABEL:LA2}
                    \label{\SETLABELQ:LA2}
                \end{minipage}
            \end{center}
        \end{figure}

% Local Variables:
% TeX-parse-self: t
% TeX-auto-save: t
% TeX-master: "fractal.tex"
% End:


        %
% -----------------------------------------------------------------------------
%
% A license is hereby granted to reproduce this software source code and
% to create executable versions from this source code for personal,
% non-commercial use.  The copyright notice included with the software
% must be maintained in all copies produced.
%
% THIS PROGRAM IS PROVIDED "AS IS". THE AUTHOR PROVIDES NO WARRANTIES
% WHATSOEVER, EXPRESSED OR IMPLIED, INCLUDING WARRANTIES OF
% MERCHANTABILITY, TITLE, OR FITNESS FOR ANY PARTICULAR PURPOSE.  THE
% AUTHOR DOES NOT WARRANT THAT USE OF THIS PROGRAM DOES NOT INFRINGE THE
% INTELLECTUAL PROPERTY RIGHTS OF ANY THIRD PARTY IN ANY COUNTRY.
%
% Copyright (c) 1994-2006, John Conover, All Rights Reserved.
%
% Comments and/or bug reports should be addressed to:
%
%     john@email.johncon.com (John Conover)
%
% -----------------------------------------------------------------------------
%
% Revision: \RCSRevision \\
% Revision Time: \RCSTime UMT \\
% Revision Date: \RCSDate \\
% Revision Id: \RCSId \\
% Revision File: \RCSLog \\
\RCS $Revision: 0.0 $
\RCS $Date: 2006/01/20 04:38:13 $
\RCS $Id: hurst.tex,v 0.0 2006/01/20 04:38:13 john Exp $
% $Log: hurst.tex,v $
% Revision 0.0  2006/01/20 04:38:13  john
% Initial version
%
%
    \subsection{Hurst Coefficient Analysis}
        \label{\SETLABEL:H}

        \subidx{\market}{Hurst coefficient analysis}
        \subidx{Hurst coefficient}{analysis}
        \subidx{increments}{normalized}
        \subidx{normalized}{increments}
        \subidx{programs}{tshurst}
        \subidx{tshurst}{program}
        The data in this section is presented in tabular form in
        Section~\ref{\SETLABELREF:HCHP}. Figure~\ref{\SETLABEL:HC} is
        a graph of the Hurst coefficient data time series data shown
        in Figure~\ref{\SETLABEL:TS}. The slope of the graph is the
        Hurst coefficient.  The data for this figure was produced by
        the program {\it tshurst}\/, which is described briefly in
        Appendix~\ref{programs}.

        \subidx{\market}{H parameter analysis}
        \subidx{H parameter}{analysis}
        \subidx{programs}{tshcalc}
        \subidx{tshcalc}{program}
        Figure~\ref{\SETLABEL:HP} is a graph of the H parameter data
        for the normalized increments of the time series data shown in
        Figure~\ref{\SETLABEL:TF}. The data for this figure was
        produced by the program {\it tshcalc}\/, which is described
        briefly in Appendix~\ref{programs}.

        \begin{figure}[ht]
            \begin{center}
                \begin{minipage}[t]{0.45\textwidth}
                    \epsfxsize=1.0\linewidth
                    \epsffile{\directory/data.tshurst.eps}
                    \caption[{\market}, Hurst coefficient data]{{\market},
                        Hurst coefficient data for the normalized
                        increments of the time series data shown in
                        Figure~\ref{\SETLABEL:TF}.  The slope of the graph
                        is the Hurst coefficient.}
                    \label{\SETLABEL:HC}
                \end{minipage}
                \hfill
                \begin{minipage}[t]{0.45\textwidth}
                    \epsfxsize=1.0\linewidth
                    \epsffile{\directory/data.tshcalc.eps}
                    \caption[{\market}, H parameter data]{{\market}, H
                        parameter data for the normalized increments of
                        the time series data shown in
                        Figure~\ref{\SETLABEL:TF} The slope of the graph
                        is the H parameter.}
                    \label{\SETLABEL:HP}
                \end{minipage}
            \end{center}
        \end{figure}

        \subidx{revenue}{See, rate of revenue returns}
        \subidx{returns}{See, rate of revenue returns}
        \subidx{\market}{revenues}
        \subidx{Hurst coefficient}{analysis}
        \subidx{\market}{Hurst coefficient analysis}
        \subidx{\market}{rate of change}
        \subidx{\market}{windows of opportunity}
        \subidx{rate of revenue returns}{forecast}
        \subidx{forecast}{rate of revenue returns}
        \idx{windows of opportunity}
        \subidx{programs}{tslsq}
        \subidx{tslsq}{program}

        The approximately linear slope of the graph in
        Figure~\ref{\SETLABEL:HC} implies that the variance of the
        rate of revenue returns, (per {\timescale},) in the {\market},
        $V(t_2 - t_1)$, over a period of time is proportional to the
        period of time raised to twice the Hurst
        coefficient~\cite[pp. 180]{Feder},~\cite[pp. 246]{Crownover}.
        This seems to be a quantitative statement concerning how fast,
        and to what degree, the rate of revenue returns' state of
        affairs can change over a period of time.  An additional
        implication, for Hurst coefficients sufficiently close to 0.5,
        is that the probability of the state of affairs repeating
        sometime in the future goes down with increasing
        time\footnote{It can be shown that the number of expected
        market ``high'' and ``low'' transitions, $N$, scales with the
        square root of time, or $N \propto \sqrt {t}$, meaning that
        the cumulative distribution of the probability, $P$, of the
        duration of a market's ``high'' or ``low'' exceeding a given
        time interval, $t$, is proportional to the reciprocal of the
        square root of the time interval, $P \propto 1 / \sqrt {t}$,
        (or, conversely, that the probability of the duration of a
        market's ``high'' or ``low'' exceeding a given time interval
        is proportional to the reciprocal of the time interval raised
        to the power $3 / 2$, ie., $P \propto 1 / t^{3 /
        2}$,~\cite[pp. 153]{Schroeder}. What this means is that a
        histogram of the ``zero free'' run-lengths of a market being
        ``high'' or ``low,'' over a long time, would have a $1 / t^{3
        / 2}$ characteristic.)}, $t$, $p(t) = erf (1/\sqrt{2t})$ which
        is approximately $1/\sqrt{t}$ for $t \gg
        1$~\cite[pp. 160]{Schroeder}. Figures~\ref{\SETLABEL:FN},
        and,~\ref{\SETLABEL:FF} compare methods of approximation of
        the ``forecastability'' of the rate of revenue returns in the
        {\market} for the near term and far term,
        respectively~\cite[pp. 83-84]{Peters:CAOITCM}\footnote{The
        author is not comfortable with Peters' interpretation. For
        example, if the algorithm explained
        in~\cite[pp. 82]{Peters:CAOITCM} is used on ``white noise''
        which, by definition, never has any correlations, the short
        term Hurst coefficient, and thus the ``forecastability,'' is
        still near unity---a bit of an enigma. This can be verified
        with the {\it tswhite}\/ and {\it tshurst}\/ programs, which
        are briefly described in Appendix~\ref{programs}.}.  This
        seems to be a quantitative statement concerning ``windows of
        opportunity'' in the rate of revenue returns, (per
        {\timescale}.)  The program {\it tslsq}\/ was used on the
        Hurst coefficient data, presented in
        Figure~\ref{\SETLABEL:HC}, to provide a least squares
        approximation to the Hurst coefficient. The superimposed least
        squares approximation with on original Hurst coefficient data
        is presented.  The time series data has a Hurst coefficient of
        {\thurstlow}, so that:

        \subidx{\market}{Hurst coefficient analysis}
        \begin{eqnarray}
            V\left(t_2 - t_1\right) & \propto & \left(t_2 - t_1\right)^{2 \cdot H}\\
            V\left(t_2 - t_1\right) & \propto & \left(t_2 - t_1\right)^{2 \cdot {\thurstlow}}\\
                                    & \propto & \left(t_2 - t_1\right)^{\thurstlowtwo}
            \label{\SETLABEL:V}
        \end{eqnarray}

        \subidx{fractional}{Brownian motion}
        \subidx{Brownian motion}{fractional}
        \idx{fractal}
        \noindent where $V(t_2 - t_1)$ is the variance of the
        increments of the rate of revenue returns, (per {\timescale},)
        over the time interval $t_2 -
        t_1$,~\cite[pp. 177]{Feder},~\cite[pp. 494]{Peitgen}. If $H >
        \frac{1}{2}$, then the time series is termed as being
        characterized by ``fractional Brownian
        motion~\cite[pp. 170]{Feder}.''

        \subidx{rate of revenue returns}{predictability}
        \subidx{rate of revenue returns}{forecastability}
        \subidx{rate of revenue returns}{consistency}
        \subidx{predictability}{rate of revenue returns}
        \subidx{forecastability}{rate of revenue returns}
        \subidx{consistency}{rate of revenue returns}
        \subidx{\market}{rate of revenue returns, predictability}
        \subidx{\market}{rate of revenue returns, forecastability}
        \subidx{\market}{rate of revenue returns, consistency}
        \subidx{Hurst coefficient}{analysis}
        \subidx{\market}{Hurst coefficient analysis}
        \subidx{\market}{rate of change}

        In some sense, the Hurst coefficient is a quantitative
        expression of the ``forecastability'' of the future based on
        the past\footnote{Actually, in general, when summing fractal
        entities, the method used should be a root mean square
        process, dependent on the Hurst Coefficient, $H$, where
        $P_{total}^H = P_1^H + P_2^H + \cdots$, where $P_n$ is the
        fractal entities. For a Brownian motion, or random walk type
        of fractal the Hurst Coefficient is a function of time into
        the future. For the ``near term,'' the Hurst coefficient is
        very near unity, meaning the summation process is linear. For
        the ``long term,'' $H \approx 0.5$, or a standard root mean
        square summation process should be used. If $H$ is $0.5$ then
        the market is termed a Brownian motion, or random walk
        process. If it is larger than 0.5, it is termed fractional
        Brownian motion process. For a random walk process, ``near
        term'' and ``far term'' are quantitatively differentiated on
        the Hurst Coefficient graph where $1 - \ln (t) = 0.5 \cdot \ln
        (t)$, or when $\ln (t) = 2$, or $t = 7.389\ldots$ See
        Section~\ref{\SETLABEL:FS} for the particulars on using Hurst
        Coefficient to sum fractal process' for the {\market}. See
        also~\cite[pp. 67, 83-84]{Peters:CAOITCM} and~\cite[pp. 129,
        159]{Schroeder} for particulars on the implications of the
        Hurst Coefficient and root mean square summation issues.}.  A
        Hurst coefficient of {\thurstlow}, (for the near future, and
        {\thurstall} for the distant future.) implies that the
        likelihood of the rate of revenue returns, (per {\timescale},)
        for any two consecutive {\timescale}s being the same is
        {\thurstlowhundred}\%~\cite[pp. 66]{Peters:CAOITCM} for the
        near future, and {\thurstall} for the distant
        future. Likewise, there is a {\thurstlowhundred}\% chance of
        the rate of revenue returns, (per {\timescale},) movements
        being the same in consecutive time periods---ie., if, in a
        given {\timescale}, the rate of revenue returns, (per
        {\timescale},) is increasing, there is a {\thurstlowhundred}\%
        that the rate of revenue returns, (per {\timescale},) will
        increase in the following period, also. In some sense, this is
        a quantitative statement on how ``predictable,'' or
        ``forecastable'' the rate of revenue returns, (per
        {\timescale},) for the {\market} are over time, since the
        probability of having $n$ many consecutive {\timescale}s of
        the same agenda is $H^n$ where $H$ is the Hurst coefficient,
        or, letting the short term probability of having $n$ many
        {\timescale}s of the same market agenda, $p_a$, is:

        \begin{eqnarray}
            p_a\left(n\right) & = & H^{n}\\
                              & = & {\thurstlow}^{n}
            \label{\SETLABEL:MA}
        \end{eqnarray}

        \subidx{rate of revenue returns}{predictability}
        \subidx{rate of revenue returns}{forecastability}
        \subidx{rate of revenue returns}{consistency}
        \subidx{predictability}{rate of revenue returns}
        \subidx{forecastability}{rate of revenue returns}
        \subidx{consistency}{rate of revenue returns}
        As an interesting interpretation of the normalized increments
        of the time series data presented in
        Figure~\ref{\SETLABEL:TF}, if the vertical axis is multiplied
        by 100, to convert to percent, then the graph represents the
        error, in percent, that would be made by forecasting, month by
        month, that the next {\timescale}'s rate of revenue returns
        would be the same as the current {\timescale}'s revenue
        rate. Interestingly, it is $\datafractionmean \cdot 100$
        percent, on the average, with a standard deviation of
        $\datafractionstddev \cdot 100$ percent, and a root mean
        square error value of $\datafractionrms \cdot 100$
        percent---small values for such a simple forecasting
        mechanism.

        \subidx{\market}{rate of revenue returns, range}
        \subidx{Hurst coefficient}{analysis}
        \subidx{\market}{Hurst coefficient analysis}
        \subidx{\market}{rate of change}

        This is, essentially, a statement of the range of values, in
        the increments of the rate of revenue returns, (per
        {\timescale},) that is to be expected over the time interval,
        $t_2 - t_1$,
        $R_v$,~\cite[pp. 178]{Feder},~\cite[pp. 172]{Cambel}:

        \begin{eqnarray}
            R_v\left(t_2 - t_1\right) & \propto & \left(t_2 - t_1\right)^{H}\\
                                      & \propto & \left(t_2 - t_1\right)^{\thurstlow}
            \label{\SETLABEL:R}
        \end{eqnarray}

        \subidx{\market}{rate of revenue returns, range}
        \subidx{Hurst coefficient}{analysis}
        \subidx{\market}{Hurst coefficient analysis}
        \subidx{\market}{rate of change}
        \subidx{Markov}{statistics}
        \subidx{statistics}{Markov}
        \noindent where $R$ is the range of values in the increments
        of the rate of revenue returns, (per {\timescale}.) A Hurst
        coefficient, $H$, that is much larger than $\frac{1}{2}$, (but
        less than 1,) implies a strongly non-Gaussian distribution in
        the increments of the rate of revenue returns, (per
        {\timescale},)~\cite[pp. 152, 194]{Feder}, and a Hurst
        coefficient near $\frac{1}{2}$ implies that the increments of
        the rate of revenue returns, (per {\timescale}) is
        characteristic of an independent
        process~\cite[pp. 195]{Feder}. Extreme caution should be
        exercised in using Markov statistics in any analysis where the
        Hurst coefficient is not
        $\frac{1}{2}$,~\cite[pp. 124]{Crownover},~\cite[pp. 106]{Peters:CAOITCM}.


        As a useful approximation, if $H$, is approximately
        $\frac{1}{2}$, Equation~\ref{\SETLABEL:R} reduces
        to,~\cite[pp. 129]{Schroeder}:

        \begin{eqnarray}
            R\left(t_2 - t_1\right) & \propto & (t_2 - t_1)^{\frac{1}{2}}\\
                                    & \propto & \sqrt{\left(t_2 - t_1\right)}
        \end{eqnarray}

        \subidx{\market}{rate of revenue returns, range}
        \subidx{\market}{rate of revenue returns, increase and decrease}
        \subidx{Hurst coefficient}{analysis}
        \subidx{\market}{Hurst coefficient analysis}
        \subidx{\market}{rate of change}
        \subidx{Markov}{statistics}
        \subidx{statistics}{Markov}

        In the case where the Hurst coefficient, $H$, is
        $\frac{1}{2}$, the range of values in the increments of the
        rate of revenue returns, (per {\timescale},) divided by the
        standard deviation of these values, $S$, can be anticipated to
        increase over time according to the following
        relation,~\cite[pp. 154]{Feder},~\cite[pp. 129]{Schroeder}:

        \begin{equation}
            \frac{R\left(t_2 - t_1\right)}{S} \propto \left(t_2 - t_1\right)^{\frac{1}{2}}
        \end{equation}

        \subidx{\market}{rate of revenue returns, range}
        \subidx{\market}{rate of revenue returns, increase and decrease}
        \subidx{Hurst coefficient}{analysis}
        \subidx{\market}{Hurst coefficient analysis}
        \subidx{\market}{rate of change}
        \noindent which is a useful conceptual approximation, since it
        involves only the square root function---if the range and the
        standard deviation of the increments of the rate of revenue
        returns, (per {\timescale},) are known, (and $H \approx
        \frac{1}{2}$,) then the expected change in $\frac{R}{S}$, will
        increase with the square root of time\footnote{To be precise,
        it is actually asymptotically proportional to
        $\tau^{\frac{1}{2}}$}.

        Another useful approximation when rescaling processes that are
        characterize by Brownian motion, (ie., when $H \approx
        \frac{1}{2}$,) is that:

        \begin{eqnarray}
            X\left(t\right) & \propto & \frac{X\left(rt\right)}{r^{H}}\\
                            & \propto & \frac{X\left(rt\right)}{r^{\thurstlow}}
        \end{eqnarray}

        \idx{Brownian motion}
        \idx{fractal}
        Where $X(t)$ is the process characterized by Brownian motion,
        and $r$ is a scaling factor,~\cite[pp. 494]{Peitgen}.

        \subidx{programs}{tslsq}
        \subidx{tslsq}{program}
        The program {\it tslsq}\/ was used on the H parameter data,
        presented in Figure~\ref{\SETLABEL:HP}, to provide a least
        squares approximation to the H parameter for the
        {\market}. The superimposed least squares approximation on the
        original H parameter data is presented.  By contrast, the H
        parameter, as derived by the methodology outlined
        in~\cite[pp. 249]{Crownover}, is {\thcalclow} for the near
        future, and {\thcalcall} for the distant future.

        \subidx{\market}{Hurst coefficient analysis}
        \subidx{Hurst coefficient}{analysis}
        \subidx{increments}{normalized}
        \subidx{normalized}{increments}
        \subidx{programs}{tshurst}
        \subidx{tshurst}{program}
        \subidx{\market}{H parameter analysis}
        \subidx{H parameter}{analysis}
        \subidx{programs}{tshcalc}
        \subidx{tshcalc}{program}
        Figures~\ref{\SETLABEL:HC} and~\ref{\SETLABEL:HP} represent
        Hurst coefficient and H parameter data that are derived from
        the normalized increments, shown in
        Figure~\ref{\SETLABEL:TF}. In this case, the data is
        considered a normalized derivative of the time series data
        presented in Figure~\ref{\SETLABEL:TF}, instead of a
        cumulative sum.  The program, {\it tshurst}\/, is described
        briefly in appendix~\ref{programs}, and the data for
        figures~\ref{\SETLABEL:THC} and~\ref{\SETLABEL:THP} was made
        using the -d option.

        \begin{figure}[ht]
            \begin{center}
                \begin{minipage}[t]{0.45\textwidth}
                    \epsfxsize=1.0\linewidth
                    \epsffile{\directory/data.tsfraction.tshurst-d.eps}
                    \caption[{\market}, traditional Hurst coefficient
                        data]{{\market}, traditional Hurst coefficient
                        data for the time series data shown in
                        Figure~\ref{\SETLABEL:TS}.  The slope of the
                        graph is the Hurst coefficient, and is
                        {\hurstlow} for the near term, and
                        {\hurstall} for the far term.}
                    \label{\SETLABEL:THC}
                \end{minipage}
                \hfill
                \begin{minipage}[t]{0.45\textwidth}
                    \epsfxsize=1.0\linewidth
                    \epsffile{\directory/data.tsfraction.tshcalc-d.eps}
                    \caption[{\market}, traditional H parameter
                        data]{{\market}, traditional H parameter data
                        for the time series data shown in
                        Figure~\ref{\SETLABEL:TS} The slope of the
                        graph is the H parameter, and is {\hcalclow}
                        for the near term, and {\hcalcall} for the
                        far term.}
                    \label{\SETLABEL:THP}
                \end{minipage}
            \end{center}
        \end{figure}

% Local Variables:
% TeX-parse-self: t
% TeX-auto-save: t
% TeX-master: "fractal.tex"
% End:


        %
% -----------------------------------------------------------------------------
%
% A license is hereby granted to reproduce this software source code and
% to create executable versions from this source code for personal,
% non-commercial use.  The copyright notice included with the software
% must be maintained in all copies produced.
%
% THIS PROGRAM IS PROVIDED "AS IS". THE AUTHOR PROVIDES NO WARRANTIES
% WHATSOEVER, EXPRESSED OR IMPLIED, INCLUDING WARRANTIES OF
% MERCHANTABILITY, TITLE, OR FITNESS FOR ANY PARTICULAR PURPOSE.  THE
% AUTHOR DOES NOT WARRANT THAT USE OF THIS PROGRAM DOES NOT INFRINGE THE
% INTELLECTUAL PROPERTY RIGHTS OF ANY THIRD PARTY IN ANY COUNTRY.
%
% Copyright (c) 1994-2006, John Conover, All Rights Reserved.
%
% Comments and/or bug reports should be addressed to:
%
%     john@email.johncon.com (John Conover)
%
% -----------------------------------------------------------------------------
%
% Revision: \RCSRevision \\
% Revision Time: \RCSTime UMT \\
% Revision Date: \RCSDate \\
% Revision Id: \RCSId \\
% Revision File: \RCSLog \\
\RCS $Revision: 0.0 $
\RCS $Date: 2006/01/20 04:38:13 $
\RCS $Id: fiscal.tex,v 0.0 2006/01/20 04:38:13 john Exp $
% $Log: fiscal.tex,v $
% Revision 0.0  2006/01/20 04:38:13  john
% Initial version
%
%
    \subsection{Fixed Increment Approximation for Fiscal Strategy}
        \label{\SETLABEL:FS}

        \subidx{\market}{fiscal strategy}
        \subidx{markets}{analysis}
        \subidx{analysis}{markets}
        \subidx{strategy}{fiscal}
        \subidx{fiscal}{strategy}
        The data in this section is presented in tabular form in
        Section~\ref{\SETLABELREF:LR}. This section derives various
        values based on the ``average'' of the normalized increments
        presented in Figure~\ref{\SETLABEL:TFA}. These values are an
        approximation to a, probably, complex process with a
        distribution shown in Figure~\ref{\SETLABEL:TF}. These values
        will be used in a fixed increment Brownian fractal analysis
        and simulation of the {\market}, and may, or may not, provide
        adequate accuracy for projections.

        For an organization operating in the {\market}, the fiscal
        strategy, commensurate with the aggregate environment, can be
        derived as follows~\cite[pp. 128, pp
        151]{Schroeder},~\cite[pp. 450]{Reza},~\cite[pp. 270]{Pierce}:
        \vspace{0.15in}

        \subsubsection{Logarithmic Returns}
            \label{\SETLABEL:LR}

            \subidx{logarithmic}{returns}
            \subidx{returns}{logarithmic}
            \subidx{\market}{logarithmic returns}
            The logarithmic returns can be calculated by various
            means. Four will be presented here, for comparison.

            \subidx{programs}{tsnormal}
            \subidx{tsnormal}{program}
            \subidx{logarithmic}{returns}
            \subidx{returns}{logarithmic}
            The logarithmic returns, in bits, $bits$, as computed from
            the mean, by the program {\it tsnormal}\/, which is
            described in Chapter~\ref{programs}, and is presented in
            Figure~\ref{\SETLABEL:TF}, and Equation~\ref{abits} from
            Section~\ref{ereturns} in Chapter~\ref{general}:

            \begin{equation}
                bits = \frac{\ln \left({\datafractionmean} + 1\right)}{\ln \left(2\right)} = \datafractionmeanbits
            \end{equation}

            \subidx{programs}{tslsq}
            \subidx{tslsq}{program}
            \subidx{logarithmic}{returns}
            \subidx{returns}{logarithmic}
            \noindent By comparison, the logarithmic returns, in bits,
            $bits$, as computed from the constant in the least squares
            approximation, using the program {\it tslsq}\/, which is briefly
            described in Chapter~\ref{programs}, as presented in
            Figure~\ref{\SETLABEL:TF}, and Equation~\ref{abits} from
            Section~\ref{ereturns} in Chapter~\ref{general}:

            \begin{equation}
                bits = \frac{\ln \left({\datafractionconstant} + 1\right)}{\ln \left(2\right)} = \datafractionconstantbits
            \end{equation}

            Note that if the mean is not constant in
            Figure~\ref{\SETLABEL:TF}, this method will not provide
            accurate results.

            \subidx{programs}{tslsq}
            \subidx{tslsq}{program}
            \subidx{logarithmic}{returns}
            \subidx{returns}{logarithmic}
            \noindent And by yet another comparison, using the program
            {\it tslsq}\/, which is briefly described in
            Chapter~\ref{programs}, with the -e -p options, to provide
            a formula for the least squares exponential fit to the
            time series data set presented in
            Figure~\ref{\SETLABEL:TS}:

            \begin{equation}
                bits = {\datatslsqepbits}
            \end{equation}

            \subidx{programs}{tslogreturns}
            \subidx{tslogreturns}{program}
            \subidx{logarithmic}{returns}
            \subidx{returns}{logarithmic}
            \noindent And finally, by comparison, from the
            {\it tslogreturns}\/ program, which is briefly described
            in Chapter~\ref{programs}, with the -p option, to provide
            a formula for the logarithmic returns of the time series
            data set presented in Figure~\ref{\SETLABEL:TS}:

            \begin{equation}
                bits = {\logreturns}
            \end{equation}

        \subsubsection{Calculation of Shannon Probability}
            \label{\SETLABEL:SP}

            \subidx{\market}{Shannon probability}
            Ideally, all of the values presented in
            Section~\ref{\SETLABEL:LR} would be equal. Using the
            logarithmic returns provided by the {\it tslogreturns}\/
            program, to be consistent
            with~\cite[pp. 81]{Peters:CAOITCM}

            \subidx{programs}{tslogreturns}
            \subidx{tslogreturns}{program}
            \begin{equation}
                2^{{\logreturns}t}
            \end{equation}

            \noindent therefore:
            \begin{equation}
                C\left(p\right) = {\logreturns}
            \end{equation}
            \subidx{programs}{tsshannon}
            \subidx{tsshannon}{program}
            \subidx{Shannon}{probability}
            \subidx{probability}{Shannon}
            \noindent and, {\it tsshannon}\/ {\logreturns} gives:
            \begin{equation}
                \label{\SETLABEL:F0}
                C\left({\shannonlogreturns}\right) = {\logreturns}
            \end{equation}
            \noindent therefore:
            \begin{eqnarray}
                2^{C\left({\shannonlogreturns}\right)} & = & 2^{\logreturns}\\
                                                       & = & {\twologreturns}\\
                                                       & = & {\twologreturnshundred}\%
            \end{eqnarray}
            \noindent and:
            \begin{eqnarray}
                2p - 1 & = & \left(2 \cdot {\shannonlogreturns}\right) - 1\\
                       & = & {\twopone}\\
                       \label{\SETLABEL:F1}
                       & = & {\twoponehundred}\%
            \end{eqnarray}

            \subidx{\market}{fiscal strategy}
            \subidx{markets}{analysis}
            \subidx{analysis}{markets}
            \subidx{strategy}{fiscal}
            \subidx{fiscal}{strategy}
            \subidx{\market}{fiscal strategy}
            \subidx{\market}{growth rate}
            Presuming the simplified assumptions outlined in
            Section~\ref{assumptions}, the ``typical'' organization
            operating in the {\market} executes a long term fiscal
            strategy, commensurate with the aggregate environment,
            that is to invest, every {\timescale}, in sufficient
            additional resources and infrastructure, to increase the
            manufacturing of goods and services by {\twoponehundred}\%
            of its rate of revenue returns, (per {\timescale}.) As a
            conceptual model, the remaining {\hundredtwoponehundred}\%
            will be held in ``reserve'' with a
            {\shannonlogreturnshundred}\% chance of making twice the
            {\twoponehundred}\% back, (and a
            {\hundredshannonlogreturnshundred}\% chance of making
            0.0,) in one {\timescale}, on the average, for an average
            growth in its rate of revenue returns, (per {\timescale},)
            of {\twologreturnshundred}\%, or a doubling of its rate of
            revenue returns, (per {\timescale},) in
            {\oneoverlogreturns} {\timescale}s.

        \subsubsection{Example Fixed Increment Approximation Fiscal Strategies}

            \subidx{\market}{fiscal strategy}
            \subidx{markets}{analysis}
            \subidx{analysis}{markets}
            \subidx{strategy}{fiscal}
            \subidx{fiscal}{strategy}
            \subidx{\market}{fiscal strategy}
            \subidx{\market}{growth rate}
            \subidx{\market}{management metric}
            \idx{management metric}
            A possible metric on the effectiveness of long term fiscal
            management could possibly be that if an investment of
            {\twoponehundred}\% per {\timescale} of the rate of
            revenue returns, (per {\timescale},) is made in resources
            and infrastructure, then the rate of revenue returns would
            be expected to increase by {\twologreturnshundred}\%, per
            {\timescale}, on average.

            Note that the metrics presented in this section are
            representative of the {\market} as an aggregate whole, and
            may or may not be accurate representations for any
            particular participant in the environment. Of interest to
            the participants in the environment would be a similar
            analysis of each product or service rendered in the
            marketplace.

            \subidx{\market}{fiscal strategy}
            \subidx{markets}{analysis}
            \subidx{analysis}{markets}
            \subidx{strategy}{fiscal}
            \subidx{fiscal}{strategy}
            \subidx{\market}{fiscal strategy}
            As a simple illustrative example, a company operating in
            this environment might obtain a credit line from a bank
            that is equal to {\twoponehundred}\% of its rate of
            revenue returns, (per {\timescale},) to finance additional
            operations. In this simple scenario, the company would use
            its revenue base as collateral for the loan. Some
            {\timescale}s, depending on the {\market}'s environment,
            the company's rate of revenue returns exceeds what was
            borrowed from the bank, and the loan is repaid in
            full. Other {\timescale}s, the company must default, and
            the bank seizes a portion of the company's revenue base to
            pay the delinquent loan. However, on the average, the
            company will expand its rate of revenue returns at
            {\twologreturnshundred}\% per {\timescale}.

            \subidx{\market}{fiscal strategy}
            \subidx{markets}{analysis}
            \subidx{analysis}{markets}
            \subidx{strategy}{fiscal}
            \subidx{fiscal}{strategy}
            \subidx{\market}{fiscal strategy}
            As another simple example, a company re-invests
            {\twoponehundred}\% of its rate of revenue returns, (per
            {\timescale},) in development, marketing, sales, and
            distribution of new products.  Although some products will
            be successful and the return on the investment will exceed
            the {\twoponehundred}\% per {\timescale} investment,
            others will not. However, on the average, the company will
            expand it gross rate of revenue returns at
            {\twologreturnshundred}\% per {\timescale}.

            \subidx{\market}{fiscal strategy}
            \subidx{markets}{analysis}
            \subidx{analysis}{markets}
            \subidx{strategy}{fiscal}
            \subidx{fiscal}{strategy}
            \subidx{\market}{fiscal strategy}
            \subidx{\market}{product portfolio}
            \subidx{\market}{product diversity}
            \subidx{\market}{product mix}
            \subidx{\market}{optimum number of products}
            \idx{product portfolio}
            \idx{product diversity}
            \idx{optimum number of products}
            \idx{product mix}

            As an example of ``product portfolio'' management, suppose
            a company re-invests {\twoponehundred}\% of its rate of
            revenue returns, (per {\timescale},) in development,
            marketing, sales, and distribution of new products.
            Further suppose that the company has two products, and a
            fractal analysis of the individual product rate of revenue
            return time series indicates that one product has a
            Shannon probability of 0.65, and the other has a Shannon
            probability of 0.55. Then the percentage of re-investment
            in the first product would be $(2 \cdot 0.65 - 1) \cdot
            {\twoponehundred}$, percent of the rate of revenue
            returns, and $(2 \cdot 0.55 - 1) \cdot {\twoponehundred}$
            percent for the second product, implying that the company
            should diversify its product line\footnote{The astute
            reader would note that the linear addition was used to add
            the contribution to development of each product. This is a
            ``near term'' interpretation. Actually, in general, the
            method used should be a root mean square process,
            dependent on the Hurst Coefficient, $H$, where
            $P_{total}^H = P_1^H + P_2^H + \cdots$, where $P_n$ is the
            contribution to each individual product. For a Brownian
            motion, or random walk type of fractal the Hurst
            Coefficient is a function of time into the future. For the
            ``near term,'' the Hurst coefficient is very near unity,
            meaning the summation process is linear. For the ``long
            term,'' $H \approx 0.5$, or a standard root mean square
            summation process should be used. If $H$ is $0.5$ then the
            market is termed a Brownian motion, or random walk
            process. If it is larger than 0.5, it is termed fractional
            Brownian motion process. For a random walk process, ``near
            term'' and ``far term'' are quantitatively differentiated
            on the Hurst Coefficient graph where $1 - \ln (t) = 0.5
            \cdot \ln (t)$, or when $\ln (t) = 2$, or $t =
            7.389\ldots$ See~\cite[pp. 67, 83-84]{Peters:CAOITCM}
            and~\cite[pp. 129, 159]{Schroeder} for particulars on the
            implications of the Hurst Coefficient and root mean square
            summation issues.}.  Note that this is a ``bet hedging''
            metric methodology, and assumes that the products have
            uncorrelated revenue return rates. If this re-investment
            methodology is not feasible, perhaps for strategic
            financial reasons, then the re-investment in both products
            should total the ${\twoponehundred}$\%, and the investment
            in each product should be made at a ratio of $\frac{(2
            \cdot 0.65 - 1)}{(2 \cdot 0.55 - 1)} = 3 : 1$,
            respectively. Note that this ``bet hedging'' can be used
            to define the optimal number of products that can be
            supported on the rate of revenue returns. If it assumed
            that all products are ``typical'' for the {\market}, as a
            standard bench mark, then the optimal number will be
            $\frac{1}{{\twopone}}$. Note that this is a
            ``theoretical'' value, since not all products are
            ``typical,'' and there may be strategic reasons, for
            example product leveraging, that may increase the number
            of products above the optimum. However, most of the
            revenue should come from the optimal number of products,
            since having more products will decrease the amount of the
            potential investment in each product, and having less than
            the optimum number of products will increase the risk that
            many of the products could suffer a ``down market''
            concurrently, impacting the rate of revenue returns.  As
            another interesting interpretation of the optimal
            ``hedging of bets,'' in product portfolio strategy, and
            considering the graph of the normalized increments
            presented in Figure~\ref{\SETLABEL:TF}, if the
            organization is running optimally, then these products
            will generate, at least in principle, one standard
            deviation, approximately $0.8413 = 84.13$\% of the future
            growth in rate of revenue returns. Naturally, these are
            approximations, and the values are an approximation to a,
            probably, complex process, and appropriate scrutiny should
            be exercised before making specific projections.  As yet
            another example of ``product portfolio'' management,
            consider the issue of product mix. In this interpretation,
            {\twoponehundred}\% of the product manufactured should be
            ``proprietary,'' while the rest is ``industry standard.''
            As yet another possibility, {\twoponehundred}\% of the
            product manufactured should be predatory into new markets,
            and the remainder in markets that are ``traditional'' for
            the company.

% Local Variables:
% TeX-parse-self: t
% TeX-auto-save: t
% TeX-master: "fractal.tex"
% End:


        %
% -----------------------------------------------------------------------------
%
% A license is hereby granted to reproduce this software source code and
% to create executable versions from this source code for personal,
% non-commercial use.  The copyright notice included with the software
% must be maintained in all copies produced.
%
% THIS PROGRAM IS PROVIDED "AS IS". THE AUTHOR PROVIDES NO WARRANTIES
% WHATSOEVER, EXPRESSED OR IMPLIED, INCLUDING WARRANTIES OF
% MERCHANTABILITY, TITLE, OR FITNESS FOR ANY PARTICULAR PURPOSE.  THE
% AUTHOR DOES NOT WARRANT THAT USE OF THIS PROGRAM DOES NOT INFRINGE THE
% INTELLECTUAL PROPERTY RIGHTS OF ANY THIRD PARTY IN ANY COUNTRY.
%
% Copyright (c) 1994-2006, John Conover, All Rights Reserved.
%
% Comments and/or bug reports should be addressed to:
%
%     john@email.johncon.com (John Conover)
%
% -----------------------------------------------------------------------------
%
% Revision: \RCSRevision \\
% Revision Time: \RCSTime UMT \\
% Revision Date: \RCSDate \\
% Revision Id: \RCSId \\
% Revision File: \RCSLog \\
\RCS $Revision: 0.0 $
\RCS $Date: 2006/01/20 04:38:13 $
\RCS $Id: companies.tex,v 0.0 2006/01/20 04:38:13 john Exp $
% $Log: companies.tex,v $
% Revision 0.0  2006/01/20 04:38:13  john
% Initial version
%
%
    \subsection{Number of Companies}
        \label{\SETLABEL:QNC}

        \subidx{\market}{number of companies}
        \subidx{number of companies}{analysis}
        \subidx{analysis}{number of companies}
        \subidx{Shannon}{probability}
        \subidx{probability}{Shannon}
        This section evaluates the approximate, or ``average,'' number
        of companies in the {\market}, and uses the method outlined in
        Chapter~\ref{general}, Section~\ref{aftsma}. Since the
        average, $avg_{ind}$, and the root mean square, $rms_{ind}$,
        of the normalized increments of the {\market} time series is
        \datafractionmean, and \datafractionrms respectively, the
        number of companies participating in the market can be
        calculated by Equation~\ref{ncompanies} to be {\ncompanies}.

        If this value seems consistent number of companies in the
        {\market}, within the assumptions outlined in
        Chapter~\ref{general}, Section~\ref{aftsma}, then it would
        seem that there is some circumstantial or indirect evidence
        that the companies participating in the {\market} are
        operating optimally, and the ``average'' Shannon probability,
        $P$ for each participating company would be, using
        Equation~\ref{pncompanies}, {\pncompanies}, which would be the
        value which should be used in Section~\ref{\SETLABEL:FS} for
        each participating company if market expansion was to be
        consistent with the rest of the industry. However, if the
        Shannon probability derived in Section~\ref{\SETLABEL:FS} is
        greater than the average Shannon probability for the companies
        participating in the {\market}, as derived in this section,
        then the market would, possibly, be exploitable with the
        fiscal strategy outlined in Section~\ref{\SETLABEL:FS}. The
        maximum exploitability for the {\market} is derived in
        Section~\ref{\SETLABEL:MAXSHANNON}, but it is probably of
        doubtful practicality.

        Note that these optimizations would maximize a company's
        market growth. Since there are probably many companies
        competing in the market place, this would not necessarily
        maximize a company's P\&L, as described in
        Chapter~\ref{general}, Section~\ref{ompl}. The Shannon
        probability that maximizes market share in the {\market} is
        \pncompanies, with several alternative solutions listed in the
        previous paragraph. However, these should be contrasted to the
        Shannon probability that maximizes a company's P\&L which is
        \avgrms~in the {\market}. In all cases, the fraction of the
        P\&L that should be ``wagered'' on the future, $f$, should be:

        \begin{equation}
            f = 2P - 1
        \end{equation}

        \noindent where $P$ is the particular Shannon probability
        chosen optimize a particular fiscal strategy. Interestingly,
        the measured Shannon probability of the {\market} would tend
        to indicate that the companies participating in the market
        have chosen a fiscal strategy that optimizes market growth, as
        opposed to capital growth.

        \subidx{\market}{increasing returns}
        \subidx{economic increasing returns}{\market}
        As interesting interpretation of these exploitive issues,
        since all three fiscal strategies will result in exponential
        market growth for every company participating in the market,
        is that they may represent, perhaps, an example of
        ``increasing returns.''

% Local Variables:
% TeX-parse-self: t
% TeX-auto-save: t
% TeX-master: "fractal.tex"
% End:


        %
% -----------------------------------------------------------------------------
%
% A license is hereby granted to reproduce this software source code and
% to create executable versions from this source code for personal,
% non-commercial use.  The copyright notice included with the software
% must be maintained in all copies produced.
%
% THIS PROGRAM IS PROVIDED "AS IS". THE AUTHOR PROVIDES NO WARRANTIES
% WHATSOEVER, EXPRESSED OR IMPLIED, INCLUDING WARRANTIES OF
% MERCHANTABILITY, TITLE, OR FITNESS FOR ANY PARTICULAR PURPOSE.  THE
% AUTHOR DOES NOT WARRANT THAT USE OF THIS PROGRAM DOES NOT INFRINGE THE
% INTELLECTUAL PROPERTY RIGHTS OF ANY THIRD PARTY IN ANY COUNTRY.
%
% Copyright (c) 1994-2006, John Conover, All Rights Reserved.
%
% Comments and/or bug reports should be addressed to:
%
%     john@email.johncon.com (John Conover)
%
% -----------------------------------------------------------------------------
%
% Revision: \RCSRevision \\
% Revision Time: \RCSTime UMT \\
% Revision Date: \RCSDate \\
% Revision Id: \RCSId \\
% Revision File: \RCSLog \\
\RCS $Revision: 0.0 $
\RCS $Date: 2006/01/20 04:38:13 $
\RCS $Id: operations.tex,v 0.0 2006/01/20 04:38:13 john Exp $
% $Log: operations.tex,v $
% Revision 0.0  2006/01/20 04:38:13  john
% Initial version
%
%
    \subsection{Fixed Increment Approximation for Operational Strategy}
        \label{\SETLABEL:OPS}.

        This section derives various values based on the ``average''
        of the normalized increments presented in
        Figure~\ref{\SETLABEL:TFA}. These values are an approximation
        to a, probably, complex process with a distribution shown in
        Figure~\ref{\SETLABEL:TF}. These values will be used in a
        fixed increment Brownian fractal analysis and simulation of
        the {\market}, and may, or may not, provide adequate accuracy
        for projections.

        \subidx{\market}{fiscal strategy}
        \subidx{\market}{Shannon probability}
        \subidx{strategy}{fiscal}
        \subidx{fiscal}{strategy}
        \subidx{Shannon}{probability}
        \subidx{probability}{Shannon}
        It should be noted that the analysis of fiscal strategy,
        presented in Section~\ref{\SETLABEL:FS}, is derived from the
        {\market} metrics and may, or may not, be maximally
        optimal. For the optimal fiscal strategy, which may be
        exploitable, see Section~\ref{\SETLABEL:MAXSHANNON}.

        \subidx{strategy}{exploitable}
        \subidx{exploitable}{strategy}
        \subidx{\market}{windows of opportunity}
        \idx{windows of opportunity}
        \subidx{decision}{obsolete}
        \subidx{obsolete}{decision}
        \subidx{decision}{timeliness}
        \subidx{timeliness}{decision}
        \subidx{rate of revenue returns}{forecast}
        \subidx{forecast}{rate of revenue returns}
        An additional exploitable strategy may be time itself.
        Equations~\ref{\SETLABEL:V},~\ref{\SETLABEL:R},
        and,~\ref{\SETLABEL:MA}, are, essentially, metrics on how fast
        a decision, which is based on information concerning the
        current status of the {\market}, becomes obsolete. Obviously,
        how long a decision is expected to remain relevant should be
        addressed as an operational necessity in strategic planning
        and project management. Figures~\ref{\SETLABEL:FN},
        and,~\ref{\SETLABEL:FF} compare methods of approximation of
        the ``forecastability'' of rate of revenue returns in the
        {\market} for the near term and far
        term~\cite[pp. 83-84]{Peters:CAOITCM}, respectively. As a
        general rule, caution must be exercised when making decisions
        that will span a time interval larger than the time interval
        where the ``forecastability'' of rate of revenue returns drops
        below 50\%. Beyond this time interval, the chances increase
        that the competitive and market forces will alter the market
        environment in a possibly detrimental unanticipated
        fashion. Obviously, there is significant advantage in
        ``timeliness'' of development, manufacturing, and distribution
        of products and services that are consistent with this
        temporal agenda. Automation of these processes, if executed
        consistently with this agenda, should be considered a
        competitive advantage.

        \subidx{strategy}{exploitable}
        \subidx{exploitable}{strategy}
        \subidx{rate of revenue returns}{forecast}
        \subidx{forecast}{rate of revenue returns}
        \idx{product life cycle}
        \idx{life cycle, product}
        In some sense, this temporal agenda defines the ``average''
        product or service life cycle in the {\market}. When the
        ``forecastability'' of rate of revenue returns drops below
        50\%, there is an even chance that the rate of revenue returns
        for the product or service will change in a detrimental
        fashion. If it is assumed that a product or service life cycle
        consists of a ramp up, a maintenence interval, and a ramp
        down, then, if all three life cycle intervals are equal, the
        product life cycle will be, approximately, three times the
        time interval where the ``forecastability'' of rate of revenue
        returns drops below 50\%. Although probably not an accurate
        prediction of product or service life cycle, the technique may
        be used as a conceptual approximation to the dynamics of
        ``market windows.\footnote{For example, consider the market
        for table salt. Since it has inelastic supply and demand
        curves, and is a necessary requirement for life, it would be
        expected that the Hurst coefficient would be very near
        unity---ignoring competitive pressures in the market. The
        predictability of the table salt market would, therefore, be
        expected to be relatively good, over time.}''  The conceptual
        approximation will probably predict a ``conservative'' or
        ``pessimistic'' value in relation to actual markets.

        \begin{figure}[ht]
            \begin{center}
                \begin{minipage}[t]{0.45\textwidth}
                    \epsfxsize=1.0\linewidth
                    \epsffile{\directory/datahurstlownear.eps}
                    \caption[{\market}, ``forecastability'' of near
                        term rate of revenue returns]{{\market},
                        ``forecastability'' of near term rate of
                        revenue returns. Although the error function
                        is the most accurate, for the near term,
                        $H^{t} = \thurstlow^{t}$ may be used as a
                        reliable metric of ``forecastability'' of the
                        rate of revenue returns.}
                    \label{\SETLABEL:FN}
                \end{minipage}
                \hfill
                \begin{minipage}[t]{0.45\textwidth}
                    \epsfxsize=1.0\linewidth
                    \epsffile{\directory/datahurstlowfar.eps}
                    \caption[{\market}, ``forecastability'' of far
                        term rate of revenue returns]{{\market},
                        ``forecastability'' of far term rate of
                        revenue returns. Although the error function
                        is the most accurate, for the far term,
                        $\frac{1}{\sqrt{t}}$ may be used as a reliable
                        metric of ``forecastability'' of the rate of
                        revenue returns.}
                    \label{\SETLABEL:FF}
                \end{minipage}
            \end{center}
        \end{figure}

        \idx{operations research}
        As an interesting interpretation of the data presented in
        Figure~\ref{\SETLABEL:FN}, there may be, perhaps, some
        applicability to such operational agendas as inventory
        control. Maintaining too little inventory, obviously, will
        create a situation where the organization can not exploit
        market expansion, and maintaining too much inventory,
        likewise, would over extend the company, creating unnecessary
        losses when the market contracts. The company should maintain
        inventory levels that do not exceed, from
        Equation~\ref{\SETLABEL:MA}, ${\thurstlow}^{n} = 0.5$
        {\timescale}s of operations. Since the optimal amount of
        inventory and, from Equation~\ref{\SETLABEL:V}, the variance
        of change in the rate of revenue returns in the future can be
        calculated, there may, perhaps, be some applicability to a
        forecasting methodology that can be incorporated into other
        areas of operations research, for example the linear algebras
        using simplex methodologies for optimization of manufacturing
        processes. Traditionally, these forecasts are made by the
        sales department, and are subject to various subjective
        biases.

% Local Variables:
% TeX-parse-self: t
% TeX-auto-save: t
% TeX-master: "fractal.tex"
% End:


        %
% -----------------------------------------------------------------------------
%
% A license is hereby granted to reproduce this software source code and
% to create executable versions from this source code for personal,
% non-commercial use.  The copyright notice included with the software
% must be maintained in all copies produced.
%
% THIS PROGRAM IS PROVIDED "AS IS". THE AUTHOR PROVIDES NO WARRANTIES
% WHATSOEVER, EXPRESSED OR IMPLIED, INCLUDING WARRANTIES OF
% MERCHANTABILITY, TITLE, OR FITNESS FOR ANY PARTICULAR PURPOSE.  THE
% AUTHOR DOES NOT WARRANT THAT USE OF THIS PROGRAM DOES NOT INFRINGE THE
% INTELLECTUAL PROPERTY RIGHTS OF ANY THIRD PARTY IN ANY COUNTRY.
%
% Copyright (c) 1994-2006, John Conover, All Rights Reserved.
%
% Comments and/or bug reports should be addressed to:
%
%     john@email.johncon.com (John Conover)
%
% -----------------------------------------------------------------------------
%
% Revision: \RCSRevision \\
% Revision Time: \RCSTime UMT \\
% Revision Date: \RCSDate \\
% Revision Id: \RCSId \\
% Revision File: \RCSLog \\
\RCS $Revision: 0.0 $
\RCS $Date: 2006/01/20 04:38:13 $
\RCS $Id: simulation.tex,v 0.0 2006/01/20 04:38:13 john Exp $
% $Log: simulation.tex,v $
% Revision 0.0  2006/01/20 04:38:13  john
% Initial version
%
%
    \subsection{Simulation of Fixed Increment Approximation for Fiscal Strategy}
        \label{\SETLABEL:TSUNFAIRBROWNIAN}

        \subidx{\market}{market simulation}
        The data in this section is presented in tabular form in
        Section~\ref{\SETLABELREF:SIM}.
        Figure~\ref{\SETLABEL:TSUNFAIRBROWNIAN0} represents a
        constructional simulation of the time series data presented in
        Figure~\ref{\SETLABEL:TS}. The program {\it
        tsunfairbrownian}\/, which is briefly described in
        appendix~\ref{programs}, was used in the reconstruction. The
        reconstructed data is superimposed on the original time series
        data.  The program, {\it tsunfairbrownian}\/, essentially,
        constructs the new time series as a Brownian fractal with
        fixed increments---the value of the fixed increment is derived
        from the root mean square average of the normalized increments
        presented in Figure~\ref{\SETLABEL:TF}. The ``quality'' of
        such a reconstruction should be subject to adequate scepticism
        and scrutiny since, in all probability, the normalized
        increments presented in Figure~\ref{\SETLABEL:TF} represent a
        relatively complex process, that may not be ``modeled'' with
        such a simple methodology.

        As a further comparison of the the constructional simulation
        with the original time series data,
        Figure~\ref{\SETLABEL:TSUNFAIRBROWNIAN1} presents a normalized
        histogram of the normalized increments of the reconstructed
        time series, superimposed on the normalized histogram
        presented in Figure~\ref{\SETLABEL:NH}.

        \subidx{\market}{fiscal strategy, simulation}
        \subidx{markets}{simulation}
        \subidx{simulation}{markets}
        \subidx{strategy}{fiscal, simulation}
        \subidx{fiscal}{strategy, simulation}
        \subidx{programs}{tsunfairbrownian}
        \subidx{tsunfairbrownian}{program}
        \begin{figure}[ht]
            \begin{center}
                \begin{minipage}[t]{0.45\textwidth}
                    \epsfxsize=1.0\linewidth
                    \epsffile{\directory/tsunfairbrownian-f.eps}
                    \caption[{\market}, Time series data, empirical and
                        simulated]{{\market}, Time series data, empirical
                        and simulated, using the program {\it tsunfairbrownian}\/
                        with f = {\datafractionrms}. This data is
                        superimposed on the data presented in
                        Figure~\ref{\SETLABEL:TS}.}
                    \label{\SETLABEL:TSUNFAIRBROWNIAN0}
                \end{minipage}
                \hfill
                \begin{minipage}[t]{0.45\textwidth}
                    \epsfxsize=1.0\linewidth
                    \epsffile{\directory/tsunfairbrownian-f.tsfraction.tsnormal-s30.eps}
                    \caption[{\market}, normalized histogram,
                        empirical and simulated]{{\market}, normalized
                        histogram of the normalized increments of the
                        time series data shown in
                        Figure~\ref{\SETLABEL:TSUNFAIRBROWNIAN0},
                        empirical and simulated.  The empirical data
                        has a mean of {\datafractionmean}, with a
                        standard deviation of {\datafractionstddev}.
                        By comparison, the simulated data has a mean
                        of {\tsunfairbrownianfractionmean} with a
                        standard deviation of
                        {\tsunfairbrownianfractionstddev}. This data
                        is superimposed on the data presented in
                        Figure~\ref{\SETLABEL:NH}. The area under the
                        four curves is identical.}
                    \label{\SETLABEL:TSUNFAIRBROWNIAN1}
                \end{minipage}
            \end{center}
        \end{figure}

% Local Variables:
% TeX-parse-self: t
% TeX-auto-save: t
% TeX-master: "fractal.tex"
% End:


        %
% -----------------------------------------------------------------------------
%
% A license is hereby granted to reproduce this software source code and
% to create executable versions from this source code for personal,
% non-commercial use.  The copyright notice included with the software
% must be maintained in all copies produced.
%
% THIS PROGRAM IS PROVIDED "AS IS". THE AUTHOR PROVIDES NO WARRANTIES
% WHATSOEVER, EXPRESSED OR IMPLIED, INCLUDING WARRANTIES OF
% MERCHANTABILITY, TITLE, OR FITNESS FOR ANY PARTICULAR PURPOSE.  THE
% AUTHOR DOES NOT WARRANT THAT USE OF THIS PROGRAM DOES NOT INFRINGE THE
% INTELLECTUAL PROPERTY RIGHTS OF ANY THIRD PARTY IN ANY COUNTRY.
%
% Copyright (c) 1994-2006, John Conover, All Rights Reserved.
%
% Comments and/or bug reports should be addressed to:
%
%     john@email.johncon.com (John Conover)
%
% -----------------------------------------------------------------------------
%
% Revision: \RCSRevision \\
% Revision Time: \RCSTime UMT \\
% Revision Date: \RCSDate \\
% Revision Id: \RCSId \\
% Revision File: \RCSLog \\
\RCS $Revision: 0.0 $
\RCS $Date: 2006/01/20 04:38:13 $
\RCS $Id: maximum.tex,v 0.0 2006/01/20 04:38:13 john Exp $
% $Log: maximum.tex,v $
% Revision 0.0  2006/01/20 04:38:13  john
% Initial version
%
%
    \subsection{Simulation of Fixed Increment Approximation for Optimally Maximal Fiscal Strategy}
        \label{\SETLABEL:MAXSHANNON}
        \subidx{\market}{fiscal strategy, simulation}
        \subidx{\market}{maximum Shannon probability}
        \subidx{markets}{simulation}
        \subidx{simulation}{markets}
        \subidx{strategy}{optimum fiscal, simulation}
        \subidx{fiscal}{optimum strategy, simulation}
        \subidx{programs}{tsunfairbrownian}
        \subidx{tsunfairbrownian}{program}
        \subidx{Shannon}{probability}
        \subidx{probability}{Shannon}

        \subidx{strategy}{exploitable}
        \subidx{exploitable}{strategy}
        \subidx{programs}{tsshannonmax}
        \subidx{tsshannonmax}{program}
        \subidx{programs}{tsunfairbrownian}
        \subidx{tsunfairbrownian}{program}
        \subidx{strategy}{fiscal}
        \subidx{fiscal}{strategy}
        The data in this section is presented in tabular form in
        Section~\ref{\SETLABELREF:MAXSHANNON}. One of the issues of
        analysis, as mentioned in Section~\ref{\SETLABEL:OPS}, is to
        determine the maximum Shannon probability for the time series
        presented in Figure~\ref{\SETLABEL:TS}. Potentially, this
        could be exploited with an aggressive fiscal
        strategy. Figure~\ref{\SETLABEL:SHANNONMAX0} is a graph of the
        output of the {\it tsshannonmax}\/ program, which is described
        briefly in appendix~\ref{programs}. The maximum of this
        function is the maximum Shannon probability for the time
        series data presented in Figure~\ref{\SETLABEL:TS}.
        Figure~\ref{\SETLABEL:SHANNONMAX1} was constructed using {\it
        tsunfairbrownian}\/ program, which is also described in
        appendix~\ref{programs}, with the maximum Shannon probability,
        and the time series data presented in
        Figure~\ref{\SETLABEL:TS}. This represents a ``what if'' the
        investment strategy was changed from a Shannon probability of
        {\shannonlogreturns}, as derived in Section~\ref{\SETLABEL:SP}
        to {\shannonmax}. This process, essentially, extracts the
        random statistical data from the time series presented in
        Figure~\ref{\SETLABEL:TS}, and constructs a new time series,
        using the random statistical data, with a different investment
        strategy.  The program, {\it tsunfairbrownian}\/, essentially,
        constructs the new time series as a Brownian fractal with
        fixed increments.  The ``quality'' of such a reconstruction
        should be subject to adequate scepticism and scrutiny since,
        in all probability, the increments in the original data
        represent a relatively complex process, that may not be
        ``modeled'' with such a simple methodology.

        \begin{figure}[ht]
            \begin{center}
                \begin{minipage}[t]{0.45\textwidth}
                    \epsfxsize=1.0\linewidth
                    \epsffile{\directory/data.tsshannonmax.eps}
                    \caption[{\market}, maximum rate of revenue
                        returns] {{\market}, maximum rate of revenue
                        returns, per {\timescale}, vs. Shannon
                        probability. The maximum rate of revenue
                        returns, per {\timescale}, occurs at a Shannon
                        probability of {\shannonmax}.}
                    \label{\SETLABEL:SHANNONMAX0}
                \end{minipage}
                \hfill
                \begin{minipage}[t]{0.45\textwidth}
                    \epsfxsize=1.0\linewidth
                    \epsffile{\directory/data.tsshannonmax-p.tsunfairbrownian-p.eps}
                    \caption[{\market}, maximum rate of revenue
                        returns] {{\market}, maximum rate of revenue
                        returns, per {\timescale}, at a Shannon
                        probability, of {\shannonmax}, corresponding
                        to a ``wager'' fraction of {\twoponemax}.}
                    \label{\SETLABEL:SHANNONMAX1}
                \end{minipage}
            \end{center}
        \end{figure}

        \subidx{fractional}{Brownian motion}
        \subidx{Brownian motion}{fractional}
        \subidx{Shannon}{probability}
        \subidx{probability}{Shannon}
        \subidx{programs}{tsshannonmax}
        \subidx{tsshannonmax}{program}
        If it is assumed that the time series data set, presented in
        Figure~\ref{\SETLABEL:TS}, constitutes classical Brownian
        motion, then the Shannon probability can be calculated by
        counting the total number of {\timescale}s that the {\market}
        movement was positive, and dividing by the total number of
        {timescale}s represented in the time series. This quotient is
        {\pmax}, as compared with the predicted value from the program
        {\it tsshannonmax}\/ of {\shannonmax}.

% Local Variables:
% TeX-parse-self: t
% TeX-auto-save: t
% TeX-master: "fractal.tex"
% End:


        %
% -----------------------------------------------------------------------------
%
% A license is hereby granted to reproduce this software source code and
% to create executable versions from this source code for personal,
% non-commercial use.  The copyright notice included with the software
% must be maintained in all copies produced.
%
% THIS PROGRAM IS PROVIDED "AS IS". THE AUTHOR PROVIDES NO WARRANTIES
% WHATSOEVER, EXPRESSED OR IMPLIED, INCLUDING WARRANTIES OF
% MERCHANTABILITY, TITLE, OR FITNESS FOR ANY PARTICULAR PURPOSE.  THE
% AUTHOR DOES NOT WARRANT THAT USE OF THIS PROGRAM DOES NOT INFRINGE THE
% INTELLECTUAL PROPERTY RIGHTS OF ANY THIRD PARTY IN ANY COUNTRY.
%
% Copyright (c) 1994-2006, John Conover, All Rights Reserved.
%
% Comments and/or bug reports should be addressed to:
%
%     john@email.johncon.com (John Conover)
%
% -----------------------------------------------------------------------------
%
% Revision: \RCSRevision \\
% Revision Time: \RCSTime UMT \\
% Revision Date: \RCSDate \\
% Revision Id: \RCSId \\
% Revision File: \RCSLog \\
\RCS $Revision: 0.0 $
\RCS $Date: 2006/01/20 04:38:13 $
\RCS $Id: verification.tex,v 0.0 2006/01/20 04:38:13 john Exp $
% $Log: verification.tex,v $
% Revision 0.0  2006/01/20 04:38:13  john
% Initial version
%
%
    \subsection{Qualitative Verification of Fixed Increment Approximation Analysis}
        \label{\SETLABEL:QVA}

        \subidx{\market}{verification of analysis}
        \subidx{verification}{analysis}
        \subidx{analysis}{verification}
        \subidx{quality}{of analysis}
        \subidx{verification}{of methodology}
        \subidx{methodology}{verification of}
        \subidx{Shannon}{probability}
        \subidx{probability}{Shannon}

        This section evaluates various values based on the ``average''
        of the normalized increments presented in
        Figure~\ref{\SETLABEL:TFA}. These values are an approximation
        to a, probably, complex process with a distribution shown in
        Figure~\ref{\SETLABEL:TF}. These values will be used in a
        fixed increment Brownian fractal analysis of the {\market},
        and may, or may not, provide adequate accuracy for
        projections.

        The data in this section is presented in tabular form in
        sections~\ref{\SETLABELREF:VI1} and~\ref{\SETLABELREF:VI2}.
        As a subjective evaluation of the ``quality'' of the analysis
        of the {\market}, from Chapter~\ref{methodology},
        Equation~\ref{metricvalues1}, and using the mean and root mean
        square values of the normalized increments of the time series
        data presented in Figure~\ref{\SETLABEL:TS} from
        Figure~\ref{\SETLABEL:TF}, and the Shannon probability as
        calculated by counting the total number of {\timescale}s that
        the {\market} movement was positive, as presented in
        Section~\ref{\SETLABEL:MAXSHANNON}:

        \begin{eqnarray}
                  P & \approx & \frac{\frac{avg}{rms} + 1}{2}\\
            {\pmax} & \approx & \frac{\frac{\datafractionmean}{\datafractionrms} + 1}{2}\\
            {\pmax} & \approx & {\avgrms}
            \label{\SETLABEL:AVGS}
        \end{eqnarray}

        \subidx{Shannon}{probability}
        \subidx{probability}{Shannon}
        \noindent and comparing these values to the Shannon
        probability, as found by the {\it tsshannonmax}\/ program, which
        iterates for a maximum:

        \begin{eqnarray}
            {\pmax} \approx {\avgrms} \approx {\shannonmax}
        \end{eqnarray}

        \subidx{logarithmic}{returns}
        \subidx{returns}{logarithmic}
        In addition, the different methods of calculating the
        logarithmic returns, presented in Section~\ref{\SETLABEL:FS},
        should be compared. The four methods used were the mean of
        Figure~\ref{\SETLABEL:TF}, the constant in the least squares
        approximation to Figure~\ref{\SETLABEL:TF}, the least squares
        exponential approximation to Figure~\ref{\SETLABEL:TS}, and
        the logarithmic returns of Figure~\ref{\SETLABEL:TS}, derived
        as the mean of the logarithms of the quotients of the
        increments. The values for each of the methods are,
        respectively:

        \begin{equation}
            \datafractionmeanbits \approx \datafractionconstantbits \approx \datatslsqepbits \approx \logreturns
        \end{equation}

        It is implied in Section~\ref{\SETLABEL:FS},
        Subsection~\ref{\SETLABEL:SP} and in
        Section~\ref{\SETLABEL:TSUNFAIRBROWNIAN} that, a Brownian
        motion with fixed increments fractal may ``model'' the
        {\market}. Using Equation~\ref{stddev9} from
        Chapter~\ref{general}, Section~\ref{abmfi}:

        \begin{eqnarray}
                                    rms \left(2P - 1\right) & \approx & \frac{\sigma \left(2P - 1\right)}{2 \sqrt{P\left(1 - P\right)}}\\
            \datafractionrms \left(2 \cdot \pmax - 1\right) & \approx & \frac{\datafractionstddev \left(2 \cdot \pmax - 1\right)}{2\sqrt{\pmax \left(1 - \pmax\right)}}\\
                       \datafractionrms \cdot \twopminusone & \approx & \datafractionstddev \cdot \twopx\\
                                                      \rmsp & \approx & \sigmap
        \end{eqnarray}

        \noindent and, equating to the mean:

        \begin{equation}
            \datafractionmean \approx \rmsp \approx \sigmap
        \end{equation}

        \subidx{Shannon}{probability}
        \subidx{probability}{Shannon}
        \noindent where, as in Equation~\ref{\SETLABEL:AVGS} using the
        mean, root mean square, and standard deviation values of the
        normalized increments of the time series data presented in
        Figure~\ref{\SETLABEL:TS} from Figure~\ref{\SETLABEL:TF}, and
        the Shannon probability as calculated by counting the total
        number of {\timescale}s that the {\market} movement was
        positive, as presented in Section~\ref{\SETLABEL:MAXSHANNON}.

        As a final qualitative comparison, the absolute value of the
        normalized increments should be the same as the root mean
        square value\footnote{The absolute value of the normalized
        increments, when averaged, is related to the root mean square
        of the increments by a constant. If the normalized increments
        are a fixed increment, the constant is unity. If the
        normalized increments have a Gaussian distribution, the
        constant is $\approx 0.8$ depending on the accuracy of of
        ``fit'' to a Gaussian distribution.}, where the absolute value
        is presented in Figure~\ref{\SETLABEL:TFA}, and the root mean
        square value is presented in Figure~\ref{\SETLABEL:TF}:

        \begin{equation}
            \datafractionabsmean \approx \datafractionrms
        \end{equation}

        Note, that if the {\market} could be ``modeled'' as a Brownian
        motion with fixed increments fractal, then the standard
        deviation of the absolute value of the normalized increments
        of the time series data presented in Figure~\ref{\SETLABEL:TS}
        from Figure~\ref{\SETLABEL:TF} should be zero. It is
        $\datafractionabsstddev$.

% Local Variables:
% TeX-parse-self: t
% TeX-auto-save: t
% TeX-master: "fractal.tex"
% End:


    \renewcommand{\market}{United States Employment Figures}
    \renewcommand{\directory}{../markets/us.employment}
    \renewcommand{\datafractionmean}{0.008052}
\renewcommand{\datafractionmeanbits}{0.011570}
\renewcommand{\datafractionmeanq}{0.002684}
\renewcommand{\datafractionmeanbitsq}{0.003867}
\renewcommand{\datafractionstddev}{0.038579}
\renewcommand{\datafractionrms}{0.039311}
\renewcommand{\avgrms}{0.602414}
\renewcommand{\ncompanies}{5.210454}
\renewcommand{\pncompanies}{0.544866}
\renewcommand{\datafractionabsmean}{0.029745}
\renewcommand{\datafractionabsstddev}{0.025769}
\renewcommand{\datafractionconstant}{0.010041}
\renewcommand{\datafractionconstantbits}{0.014414}
\renewcommand{\datafractionconstantq}{0.003347}
\renewcommand{\datafractionconstantbitsq}{0.004821}
\renewcommand{\datafractionslope}{-0.000021}
\renewcommand{\datafractionabsconstant}{0.035145}
\renewcommand{\datafractionabsslope}{-0.000057}
\renewcommand{\hurstall}{0.659558}
\renewcommand{\hurstlow}{0.707509}
\renewcommand{\hurstlowtwo}{1.415018}
\renewcommand{\hurstlowhundred}{70.750900}
\renewcommand{\hcalcall}{0.184942}
\renewcommand{\hcalclow}{0.102042}
\renewcommand{\shannonmax}{0.604167}
\renewcommand{\twoponemax}{0.208334}
\renewcommand{\logreturns}{0.010456}
\renewcommand{\twologreturns}{1.007274}
\renewcommand{\twologreturnshundred}{0.727387}
\renewcommand{\oneoverlogreturns}{95.638868}
\renewcommand{\pmax}{0.602094}
\renewcommand{\twopminusone}{0.204188}
\renewcommand{\rmsp}{0.008027}
\renewcommand{\twopx}{0.208583}
\renewcommand{\sigmap}{0.008047}
\renewcommand{\tsunfairbrownianfractionmean}{0.007862}
\renewcommand{\tsunfairbrownianfractionstddev}{0.038619}
\renewcommand{\shannonlogreturns}{0.560125}
\renewcommand{\shannonlogreturnshundred}{56.012500}
\renewcommand{\twopone}{0.120250}
\renewcommand{\twoponehundred}{12.025000}
\renewcommand{\hundredtwoponehundred}{87.975000}
\renewcommand{\hundredshannonlogreturnshundred}{43.987500}
\renewcommand{\datatslsqepbits}{0.007623}
\renewcommand{\thurstall}{0.633980}
\renewcommand{\thurstlow}{0.710108}
\renewcommand{\thurstlowtwo}{1.420216}
\renewcommand{\thurstlowhundred}{71.010800}
\renewcommand{\thcalcall}{0.247886}
\renewcommand{\thcalclow}{0.171737}
\renewcommand{\chisquared}{2.862000}
\renewcommand{\critical}{42.557000}

    \renewcommand{\timescale}{month}
    \subidx{market}{\market}
    \idx{\market}

    \section{\market}

        \renewcommand{\SETLABEL}{\LABPRE:USEMPLOYMENT}
        \renewcommand{\SETLABELQ}{\LABPRE:USEMPLOYMENTQ}
        \label{\SETLABEL}
        \renewcommand{\SETLABELREF}{\LABPREREF:USEMPLOYMENT}

        \idx{United States Bureau of Labor and Statistics}
        For the analysis, the data was in the directory
        {\directory}\footnote{Data from the United States Bureau of
        Labor and Statistics, 1980---1994, by {\timescale}s, in
        thousands of persons.}.

        The data in this section is presented in tabular form in
        Section~\ref{\SETLABELREF}. Note that in this analysis, the
        rate of revenue returns means the increase or decrease in the
        {\market}. This is included for comparative
        purposes. Presumably, the {\market} represents something of
        value, or they could be used as a ``futures'' derivative, and
        thus, it would be considered that there is a rate of revenue
        returns.

        %
% -----------------------------------------------------------------------------
%
% A license is hereby granted to reproduce this software source code and
% to create executable versions from this source code for personal,
% non-commercial use.  The copyright notice included with the software
% must be maintained in all copies produced.
%
% THIS PROGRAM IS PROVIDED "AS IS". THE AUTHOR PROVIDES NO WARRANTIES
% WHATSOEVER, EXPRESSED OR IMPLIED, INCLUDING WARRANTIES OF
% MERCHANTABILITY, TITLE, OR FITNESS FOR ANY PARTICULAR PURPOSE.  THE
% AUTHOR DOES NOT WARRANT THAT USE OF THIS PROGRAM DOES NOT INFRINGE THE
% INTELLECTUAL PROPERTY RIGHTS OF ANY THIRD PARTY IN ANY COUNTRY.
%
% Copyright (c) 1994-2006, John Conover, All Rights Reserved.
%
% Comments and/or bug reports should be addressed to:
%
%     john@email.johncon.com (John Conover)
%
% -----------------------------------------------------------------------------
%
% Revision: \RCSRevision \\
% Revision Time: \RCSTime UMT \\
% Revision Date: \RCSDate \\
% Revision Id: \RCSId \\
% Revision File: \RCSLog \\
\RCS $Revision: 0.0 $
\RCS $Date: 2006/01/20 04:38:13 $
\RCS $Id: fraction.tex,v 0.0 2006/01/20 04:38:13 john Exp $
% $Log: fraction.tex,v $
% Revision 0.0  2006/01/20 04:38:13  john
% Initial version
%
%
    \subsection{Time Series Increments Analysis}
        \label{\SETLABEL:TSA}

        \subidx{\market}{Time series analysis}
        \subidx{time series}{increments}
        \subidx{time series}{analysis}
        \subidx{cumulative sum}{analysis}
        \subidx{analysis}{cumulative sum}
        \subidx{analysis}{random process}
        \subidx{random process}{analysis}
        \subidx{Gaussian}{increments}
        \subidx{increments}{Gaussian}
        \subidx{Brownian}{motion, fractional}
        \subidx{fractional}{Brownian motion}
        \subidx{fractal}{Brownian motion}
        The data in this section is presented in tabular form in
        Section~\ref{\SETLABELREF:TSA}.  Figure~\ref{\SETLABEL:TS} is
        a graph of the time series data for the {\market}.

        \subidx{increments}{normalized}
        \subidx{normalized}{increments}
        \subidx{programs}{tsfraction}
        \subidx{tsfraction}{program}
        Figure~\ref{\SETLABEL:TF} is a graph of the normalized
        increments of the time series data presented in
        Figure~\ref{\SETLABEL:TS}. The data presented was made by
        running the program {\it tsfraction}\/ on the time series
        data. The program {\it tsfraction}\/ is described briefly in
        Appendix~\ref{programs}, and subtracts the previous value from
        the next value, dividing this difference by the previous
        value, for each element in the time series data. The new time
        series contains the instantaneous change in the rate of
        revenue returns, divided by the magnitude of the instantaneous
        rate of revenue returns.

        \subidx{mean}{standard deviation}
        \subidx{standard deviation}{mean}
        \idx{root mean square}
        \idx{least squares approximation}
        \begin{figure}[ht]
            \begin{center}
                \begin{minipage}[t]{0.45\textwidth}
                    \epsfxsize=1.0\linewidth
                    \epsffile{\directory/data.eps}
                    \caption{{\market}, time series data.}
                    \label{\SETLABEL:TS}
                    \label{\SETLABELQ:TS}
                \end{minipage}
                \hfill
                \begin{minipage}[t]{0.45\textwidth}
                    \epsfxsize=1.0\linewidth
                    \epsffile{\directory/data.tsfraction.eps}
                    \caption[{\market}, normalized
                        increments]{{\market}, normalized increments
                        of the time series data presented in
                        Figure~\ref{\SETLABEL:TS}. The mean is
                        {\datafractionmean} with a standard deviation
                        of {\datafractionstddev}. The formula for the
                        least squares approximation is
                        ${\datafractionconstant} +
                        {\datafractionslope}t$, and the root mean
                        squared value is {\datafractionrms}. The
                        graph, labeled ``data\-.tsfraction\-.tsrms,''
                        is the running root mean square, and
                        ``data\-.tsfraction\-.tsavg'' is the running
                        average of the normalized increments.  This
                        graph is the fraction of change in the time
                        series, as a function of time. Note that the
                        slope of the mean, {\datafractionslope}, is
                        the coefficient of the nonlinearity term in
                        the normalized increments. See
                        Chapter~\ref{general}, Section~\ref{nlextend}
                        for a possible application of the logistic
                        function to this data set.}
                    \label{\SETLABEL:TF}
                    \label{\SETLABELQ:TF}
                \end{minipage}
            \end{center}
        \end{figure}

        \subidx{absolute value}{increments}
        \subidx{increments}{absolute value}

        Figure~\ref{\SETLABEL:TFA} is a graph of the absolute value of
        the normalized increments of the time series data presented in
        Figure~\ref{\SETLABEL:TF}. The data presented was made by
        running the Unix utility sed(1) on the normalized increments
        time series data to remove the negative signs. This is an
        absolute value procedure.  The resulting time series contains
        the absolute value of the instantaneous change in the rate of
        revenue returns, divided by the magnitude of the instantaneous
        rate of revenue returns\footnote{The absolute value of the
        normalized increments, when averaged, is related to the root
        mean square of the increments by a constant. If the normalized
        increments are a fixed increment, the constant is unity. If
        the normalized increments have a Gaussian distribution, the
        constant is $\approx 0.8$ depending on the accuracy of of
        ``fit'' to a Gaussian distribution.}.

        \subidx{histogram}{normalized}
        \subidx{normalized}{histogram}
        \subidx{programs}{tsnormal}
        \subidx{tsnormal}{program}
        \subidx{mean}{standard deviation}
        \subidx{standard deviation}{mean}
        \idx{root mean square}
        \idx{least squares approximation}
        \subidx{\market}{analysis of increments}
        Figure~\ref{\SETLABEL:NH} is the normalized histogram of the
        normalized increments of the time series data shown in
        Figure~\ref{\SETLABEL:TF}. The abscissa is 3 $\sigma$ limits,
        and the area under the two curves is identical. The data for
        this figure was produced by the program {\it tsnormal}\/,
        which is described briefly in Appendix~\ref{programs}.

        \begin{figure}[ht]
            \begin{center}
                \begin{minipage}[t]{0.45\textwidth}
                    \epsfxsize=1.0\linewidth
                    \epsffile{\directory/data.tsfraction.abs.eps}
                    \caption[{\market}, absolute value of the
                        normalized increments]{{\market}, absolute
                        value of the normalized increments of the time
                        series data presented in
                        Figure~\ref{\SETLABEL:TF}.  The mean is
                        {\datafractionabsmean} with a standard
                        deviation of {\datafractionabsstddev}. The
                        formula for the least squares approximation is
                        ${\datafractionabsconstant} +
                        {\datafractionabsslope}t$, and the root mean
                        square value, from Figure~\ref{\SETLABEL:TF},
                        is {\datafractionrms}.  The graph, labeled
                        ``data\-.tsfraction\-.tsrms,'' is the running
                        root mean square, and
                        ``data\-.tsfraction\-.tsavg'' is the running
                        average of the normalized increments presented
                        in Figure~\ref{\SETLABEL:TF}, superimposed
                        here for convenience. This graph is the
                        absolute value of the fraction of change in
                        the time series, as a function of time.}
                    \label{\SETLABEL:TFA}
                    \label{\SETLABELQ:TFA}
                \end{minipage}
                \hfill
                \begin{minipage}[t]{0.45\textwidth}
                    \epsfxsize=1.0\linewidth
                    \epsffile{\directory/data.tsfraction.tsnormal-s30.eps}
                    \caption[{\market}, normalized histogram of the
                        normalized increments]{{\market}, normalized
                        histogram of the normalized increments of the
                        time series data shown in
                        Figure~\ref{\SETLABEL:TF}.  The data has a
                        mean of {\datafractionmean}, with a standard
                        deviation of {\datafractionstddev}.  The area
                        under the two curves is identical. The
                        $\chi^2$ value of the observed and expected
                        values of the two curves is {\chisquared},
                        with a critical value of {\critical}.}
                    \label{\SETLABEL:NH}
                \end{minipage}
            \end{center}
        \end{figure}

        \subidx{programs}{tsXsquared}
        \subidx{tsXsquared}{program}
        \subidx{\market}{chi-squared values of increments}
        The program {\it tsXsquared}\/, which is briefly described in
        appendix~\ref{programs}, was used to derive the $\chi^2$
        statistics for the data presented in
        Figure~\ref{\SETLABEL:NH}.

        \subidx{programs}{tsstatest}
        \subidx{tsstatest}{program}
        \subidx{\market}{statistical estimates}

        Figure~\ref{\SETLABEL:SE} is the statistical estimate for the
        data presented in Figure~\ref{\SETLABEL:TF}, as derived by the
        program {\it tsstatest}\/, which is briefly described in
        appendix~\ref{programs}.

        \begin{figure}[ht]
            \begin{center}
                \begin{minipage}[t]{\textwidth}
                    \center{\fbox{\parbox{0.9\textwidth}{\XXX{\directory/data.tsstatest-f0.1-c0.9-i.tex}}}}
                    \caption[{\market}, statistical estimates of the
                        normalized increments]{{\market}, statistical
                        estimates of the normalized increments of the
                        time series shown in Figure~\ref{\SETLABEL:TF}.
                        The table was produced with the {\it
                        tsstatest}\/ program, and illustrates the
                        size of the data set required for a confidence
                        level of 90\%, with an error estimate of $\pm$
                        10\%, or alternately, the error estimate on
                        the time series shown in Figure~\ref{\SETLABEL:TF}.}
                    \label{\SETLABEL:SE}
                \end{minipage}
            \end{center}
        \end{figure}

        Note that the data set size estimations, as produced by the
        {\it tsstatest}\/ program, are probably very conservative,
        depending on the magnitude of the Shannon probability, $P =
        \shannonlogreturns$, as derived in
        Section~\ref{\SETLABEL:SP}. See Chapter~\ref{general},
        Section~\ref{serdss} for possible alternative methodologies
        for addressing the analysis of fractal time series with
        limited data set sizes. Depending on the magnitude of the
        Shannon probability, $P$, these estimates can be several
        orders of magnitude too high.

        \subidx{derivative of increments}{normalized}
        \subidx{normalized}{derivative of increments}
        \subidx{programs}{tsderivative}
        \subidx{tsderivative}{program}
        Figure~\ref{\SETLABEL:TF1} is the normalized histogram of the
        first derivative of the normalized increments of the time
        series data shown in Figure~\ref{\SETLABEL:TF}. In principle,
        if the distribution of the normalized increments presented in
        Figure~\ref{\SETLABEL:NH} is Gaussian in nature, this
        distribution would be similar to ``white noise,'' as presented
        in appendix~\ref{programs}, Figure~\ref{whiteexample}. The
        data was generated by the {\it tsderivative}\/ program, which
        is briefly described in
        appendix~\ref{programs}. Figure~\ref{\SETLABEL:TF2} is the
        normalized histogram of the second derivative of the
        normalized increments of the time series data shown in
        Figure~\ref{\SETLABEL:TF}. In principle, if the distribution
        of the normalized increments presented in
        Figure~\ref{\SETLABEL:NH} is an integrated Gaussian
        distribution in nature, this distribution would be similar to
        ``white noise,'' as presented in appendix~\ref{programs},
        Figure~\ref{whiteexample}.

        \begin{figure}[ht]
            \begin{center}
                \begin{minipage}[t]{0.45\textwidth}
                    \epsfxsize=1.0\linewidth
                    \epsffile{\directory/data.tsfraction.tsderivative.tsnormal-s30.eps}
                    \caption[{\market}, histogram of the first
                        derivative of the increments]{{\market},
                        normalized histogram of the first derivative
                        of the normalized increments of the time
                        series data shown in
                        Figure~\ref{\SETLABEL:TF}.}
                    \label{\SETLABEL:TF1}
                \end{minipage}
                \hfill
                \begin{minipage}[t]{0.45\textwidth}
                    \epsfxsize=1.0\linewidth
                    \epsffile{\directory/data.tsfraction.2tsderivative.tsnormal-s30.eps}
                    \caption[{\market}, histogram of the second
                        derivative of the increments]{{\market},
                        normalized histogram of second derivative of
                        the the normalized increments of the time
                        series data shown in
                        Figure~\ref{\SETLABEL:TF}.}
                    \label{\SETLABEL:TF2}
                \end{minipage}
            \end{center}
        \end{figure}

        \subidx{fractal}{range}
        \subidx{fractal}{R/S analysis}
        \subidx{\market}{rate of revenue returns, range}
        \subidx{\market}{deterministic mechanism}
        \subidx{deterministic}{mechanism}
        \subidx{mechanism}{deterministic}
        Figure~\ref{\SETLABEL:TR} is the range of values of the time
        series shown in Figure~\ref{\SETLABEL:TS}. The horizontal axis
        is time into the future. In principle, if the time series was
        characterized as fractional Brownian motion the graph in
        Figure~\ref{\SETLABEL:TR} would be a square root
        function\footnote{Note that the ``roughness,'' or ``sawtooth''
        characteristics of the graph in Figure~\ref{\SETLABEL:TR} are
        a computational artifact---caused by not using the -m option
        to the program {\it tshurst}\/, which is computationally
        inefficient.}. Figure~\ref{\SETLABEL:TD} is the deterministic
        map of the normalized increments of the time series data shown
        in Figure~\ref{\SETLABEL:TF}. The deterministic map is useful
        for determining if a time series was created by a
        deterministic mechanism. This, essentially, maps each element
        in the time series with the previous element in the time
        series.  See,~\cite[pp. 745]{Peitgen}.

        \begin{figure}[ht]
            \begin{center}
                \begin{minipage}[t]{0.45\textwidth}
                    \epsfxsize=1.0\linewidth
                    \epsffile{\directory/data.tshurst-f.eps}
                    \caption[{\market}, range]{{\market}, range of the
                        time series data shown in
                        Figure~\ref{\SETLABEL:TS}.}
                    \label{\SETLABEL:TR}
                \end{minipage}
                \hfill
                \begin{minipage}[t]{0.45\textwidth}
                    \epsfxsize=1.0\linewidth
                    \epsffile{\directory/data.tsfraction.tsdeterministic.eps}
                    \caption[{\market}, deterministic map]{{\market},
                        deterministic map of the normalized increments
                        of the time series data shown in
                        Figure~\ref{\SETLABEL:TF}.}
                    \label{\SETLABEL:TD}
                \end{minipage}
            \end{center}
        \end{figure}

% Local Variables:
% TeX-parse-self: t
% TeX-auto-save: t
% TeX-master: "fractal.tex"
% End:


        \subsubsection{Observations on the Time Series Increments Analysis}

            Figure~\ref{\SETLABEL:NH} would seem to indicate that the
            time series data for the {\market} represents a cumulative
            sum/integration of a random process that has a Gaussian
            distribution, (ie., satisfies the Gaussian increments
            property of fractional Brownian
            motion~\cite[pp. 250]{Crownover},) tending to justify the
            assumption that the time series data represents fractional
            Brownian motion.

        %
% -----------------------------------------------------------------------------
%
% A license is hereby granted to reproduce this software source code and
% to create executable versions from this source code for personal,
% non-commercial use.  The copyright notice included with the software
% must be maintained in all copies produced.
%
% THIS PROGRAM IS PROVIDED "AS IS". THE AUTHOR PROVIDES NO WARRANTIES
% WHATSOEVER, EXPRESSED OR IMPLIED, INCLUDING WARRANTIES OF
% MERCHANTABILITY, TITLE, OR FITNESS FOR ANY PARTICULAR PURPOSE.  THE
% AUTHOR DOES NOT WARRANT THAT USE OF THIS PROGRAM DOES NOT INFRINGE THE
% INTELLECTUAL PROPERTY RIGHTS OF ANY THIRD PARTY IN ANY COUNTRY.
%
% Copyright (c) 1994-2006, John Conover, All Rights Reserved.
%
% Comments and/or bug reports should be addressed to:
%
%     john@email.johncon.com (John Conover)
%
% -----------------------------------------------------------------------------
%
% Revision: \RCSRevision \\
% Revision Time: \RCSTime UMT \\
% Revision Date: \RCSDate \\
% Revision Id: \RCSId \\
% Revision File: \RCSLog \\
\RCS $Revision: 0.0 $
\RCS $Date: 2006/01/20 04:38:13 $
\RCS $Id: instant.tex,v 0.0 2006/01/20 04:38:13 john Exp $
% $Log: instant.tex,v $
% Revision 0.0  2006/01/20 04:38:13  john
% Initial version
%
%
    \subsection{Instantaneous Analysis of Normalized Increments}
        \label{\SETLABEL:IA}

        \subidx{\market}{instantaneous analysis of normalized increments}
        \idx{average of normalized increments}
        \idx{root mean square of normalized increments}
        \subidx{Shannon probability}{instantaneous computation of}
        \subidx{average of normalized increments}{instantaneous computation of}
        \subidx{root mean square of normalized increments}{instantaneous computation of}
        \subidx{instantaneous computation}{Shannon probability}
        \subidx{instantaneous computation}{average of normalized increments}
        \subidx{instantaneous computation}{root mean square of normalized increments}
        \idx{time series}
        \subidx{time series}{instantaneous analysis}
        \subidx{instantaneous analysis}{time series}
        \subidx{time series}{increments}
        \subidx{time series}{analysis}
        \subidx{Shannon}{probability}
        \subidx{probability}{Shannon}
        \subidx{normalized}{increments}
        \subidx{increments}{normalized}

        The program {\it tsinstant}\/, which is briefly described in
        Appendix~\ref{programs}, is for finding the instantaneous
        fraction of change in a time series. The value of a sample in
        the time series is subtracted from the previous sample in the
        time series, and divided by the value of the previous sample.
        As explained in Chapter~\ref{general},
        Sections~\ref{derivation},~\ref{GA},~\ref{abmfi},~\ref{aftsma}
        and,~\ref{ompl} for Brownian motion, random walk fractals, the
        absolute value of the instantaneous fraction of change is also
        the root mean square of the instantaneous fraction of
        change\footnote{The absolute value of the normalized
        increments, when averaged, is related to the root mean square
        of the increments by a constant. If the normalized increments
        are a fixed increment, the constant is unity. If the
        normalized increments have a Gaussian distribution, the
        constant is $\approx 0.8$ depending on the accuracy of of
        ``fit'' to a Gaussian distribution.}. Squaring this value is
        the average of the instantaneous fraction of change, and
        adding unity to the absolute value of the instantaneous
        fraction of change, and dividing by two, is the Shannon
        probability of the instantaneous fraction of change.

        Figure~\ref{\SETLABEL:IA1} is the instantaneous value of the
        root mean square of the normalized increments for the
        {\market}, and Figure~\ref{\SETLABEL:IA2} is the instantaneous
        Shannon probability for the normalized increments.

        \begin{figure}[ht]
            \begin{center}
                \begin{minipage}[t]{0.45\textwidth}
                    \epsfxsize=1.0\linewidth
                    \epsffile{\directory/data.tsinstant-r.eps}
                    \caption[{\market}, instantaneous value of
                        rms.]{{\market}, instantaneous value of the
                        root mean square of the normalized increments,
                        provided by running the program {\it
                        tsinstant}\/ with the -r option on the data
                        presented in Figure~\ref{\SETLABEL:TS}.}
                    \label{\SETLABEL:IA1}
                    \label{\SETLABELQ:IA1}
                \end{minipage}
                \hfill
                \begin{minipage}[t]{0.45\textwidth}
                    \epsfxsize=1.0\linewidth
                    \epsffile{\directory/data.tsinstant-s.eps}
                    \caption[{\market}, instantaneous value of
                        Shannon probability.]{{\market}, instantaneous
                        value of the Shannon probability of the
                        normalized increments, provided by running the
                        program {\it tsinstant}\/ with the -s option
                        on the data presented in
                        Figure~\ref{\SETLABEL:TS}.}
                    \label{\SETLABEL:IA2}
                    \label{\SETLABELQ:IA2}
                \end{minipage}
            \end{center}
        \end{figure}

% Local Variables:
% TeX-parse-self: t
% TeX-auto-save: t
% TeX-master: "fractal.tex"
% End:


        %
% -----------------------------------------------------------------------------
%
% A license is hereby granted to reproduce this software source code and
% to create executable versions from this source code for personal,
% non-commercial use.  The copyright notice included with the software
% must be maintained in all copies produced.
%
% THIS PROGRAM IS PROVIDED "AS IS". THE AUTHOR PROVIDES NO WARRANTIES
% WHATSOEVER, EXPRESSED OR IMPLIED, INCLUDING WARRANTIES OF
% MERCHANTABILITY, TITLE, OR FITNESS FOR ANY PARTICULAR PURPOSE.  THE
% AUTHOR DOES NOT WARRANT THAT USE OF THIS PROGRAM DOES NOT INFRINGE THE
% INTELLECTUAL PROPERTY RIGHTS OF ANY THIRD PARTY IN ANY COUNTRY.
%
% Copyright (c) 1994-2006, John Conover, All Rights Reserved.
%
% Comments and/or bug reports should be addressed to:
%
%     john@email.johncon.com (John Conover)
%
% -----------------------------------------------------------------------------
%
% Revision: \RCSRevision \\
% Revision Time: \RCSTime UMT \\
% Revision Date: \RCSDate \\
% Revision Id: \RCSId \\
% Revision File: \RCSLog \\
\RCS $Revision: 0.0 $
\RCS $Date: 2006/01/20 04:38:13 $
\RCS $Id: logistic.tex,v 0.0 2006/01/20 04:38:13 john Exp $
% $Log: logistic.tex,v $
% Revision 0.0  2006/01/20 04:38:13  john
% Initial version
%
%
    \subsection{Logistic Analysis}
        \label{\SETLABEL:LA}

        \subidx{\market}{Logistic function analysis}
        \subidx{time series}{logistic function}
        \subidx{logistic function}{time series}
        \subidx{time series}{increments}
        \subidx{time series}{analysis}
        \subidx{cumulative sum}{analysis}
        \subidx{analysis}{cumulative sum}
        \subidx{analysis}{random process}
        \subidx{random process}{analysis}
        The data in this section is presented in tabular form in
        Section~\ref{\SETLABELREF:LAA}.  Figure~\ref{\SETLABEL:LA1} is
        a graph of the logistic function estimates of the time series
        data for the {\market}. The reader is cautioned that these
        graphs are constructed using the method suggested in
        Chapter~\ref{general}, Section~\ref{nlextend} and enormous
        precision is required for adequate prediction of the logistic
        function,~\cite{Modis}. Particularly, the non-linear term will
        usually require intervention to produce a practical fit to the
        data. In addition, there are numerical stability issues with
        logistic function methodologies\footnote{For example, in
        Figures~\ref{\SETLABEL:LA1} and~\ref{\SETLABEL:LA2}, if the
        non-linear term, $b$, was greater than zero, it was set to
        zero to produce the graphs. See Section~\ref{\SETLABELREF:LAA}
        for the actual derived values. In other cases, the magnitude
        of $b$ was too large, resulting in a graph that was decreasing
        as a function of time}.  The methodology should be regarded as
        ``fragile.'' It is included for completeness.

        \idx{least squares approximation}
        Figure~\ref{\SETLABEL:LA1} is a graph of the logistic function
        for the time series data presented in
        Figure~\ref{\SETLABEL:TS}. The data presented was made by
        running the program {\it tsdlogistic}\/, which is described
        briefly in Appendix~\ref{programs}, on the parameters
        extracted from the time series data as suggested in
        Figure~\ref{\SETLABEL:TF}. The program {\it tslsq}\/ was used
        to derive the constant and the slope of the normalized
        increments of the data presented in Figure~\ref{\SETLABEL:TF}.
        Figure~\ref{\SETLABEL:LA2} is the same graph, but with the
        time scale expanded by a factor of two.

        \begin{figure}[ht]
            \begin{center}
                \begin{minipage}[t]{0.45\textwidth}
                    \epsfxsize=1.0\linewidth
                    \epsffile{\directory/data.tsfraction.tslsq-p.tsdlogistic.eps}
                    \caption[{\market}, logistic function
                        estimates.]{{\market}, logistic function
                        estimates, provided by running the {\it
                        tslsq}\/ program on the normalized increments
                        presented in Figure~\ref{\SETLABEL:TF} with
                        the -p option. These parameters were used as
                        arguments to the {\it tsdlogistic}\/ program.}
                    \label{\SETLABEL:LA1}
                    \label{\SETLABELQ:LA1}
                \end{minipage}
                \hfill
                \begin{minipage}[t]{0.45\textwidth}
                    \epsfxsize=1.0\linewidth
                    \epsffile{\directory/data.tsfraction.tslsq-p.tsdlogistic2.eps}
                    \caption[{\market}, logistic function
                        estimates.]{{\market}, logistic function
                        estimates of Figure~\ref{\SETLABEL:LA1} with
                        the time scale expanded by a factor of two.}
                    \label{\SETLABEL:LA2}
                    \label{\SETLABELQ:LA2}
                \end{minipage}
            \end{center}
        \end{figure}

% Local Variables:
% TeX-parse-self: t
% TeX-auto-save: t
% TeX-master: "fractal.tex"
% End:


        %
% -----------------------------------------------------------------------------
%
% A license is hereby granted to reproduce this software source code and
% to create executable versions from this source code for personal,
% non-commercial use.  The copyright notice included with the software
% must be maintained in all copies produced.
%
% THIS PROGRAM IS PROVIDED "AS IS". THE AUTHOR PROVIDES NO WARRANTIES
% WHATSOEVER, EXPRESSED OR IMPLIED, INCLUDING WARRANTIES OF
% MERCHANTABILITY, TITLE, OR FITNESS FOR ANY PARTICULAR PURPOSE.  THE
% AUTHOR DOES NOT WARRANT THAT USE OF THIS PROGRAM DOES NOT INFRINGE THE
% INTELLECTUAL PROPERTY RIGHTS OF ANY THIRD PARTY IN ANY COUNTRY.
%
% Copyright (c) 1994-2006, John Conover, All Rights Reserved.
%
% Comments and/or bug reports should be addressed to:
%
%     john@email.johncon.com (John Conover)
%
% -----------------------------------------------------------------------------
%
% Revision: \RCSRevision \\
% Revision Time: \RCSTime UMT \\
% Revision Date: \RCSDate \\
% Revision Id: \RCSId \\
% Revision File: \RCSLog \\
\RCS $Revision: 0.0 $
\RCS $Date: 2006/01/20 04:38:13 $
\RCS $Id: hurst.tex,v 0.0 2006/01/20 04:38:13 john Exp $
% $Log: hurst.tex,v $
% Revision 0.0  2006/01/20 04:38:13  john
% Initial version
%
%
    \subsection{Hurst Coefficient Analysis}
        \label{\SETLABEL:H}

        \subidx{\market}{Hurst coefficient analysis}
        \subidx{Hurst coefficient}{analysis}
        \subidx{increments}{normalized}
        \subidx{normalized}{increments}
        \subidx{programs}{tshurst}
        \subidx{tshurst}{program}
        The data in this section is presented in tabular form in
        Section~\ref{\SETLABELREF:HCHP}. Figure~\ref{\SETLABEL:HC} is
        a graph of the Hurst coefficient data time series data shown
        in Figure~\ref{\SETLABEL:TS}. The slope of the graph is the
        Hurst coefficient.  The data for this figure was produced by
        the program {\it tshurst}\/, which is described briefly in
        Appendix~\ref{programs}.

        \subidx{\market}{H parameter analysis}
        \subidx{H parameter}{analysis}
        \subidx{programs}{tshcalc}
        \subidx{tshcalc}{program}
        Figure~\ref{\SETLABEL:HP} is a graph of the H parameter data
        for the normalized increments of the time series data shown in
        Figure~\ref{\SETLABEL:TF}. The data for this figure was
        produced by the program {\it tshcalc}\/, which is described
        briefly in Appendix~\ref{programs}.

        \begin{figure}[ht]
            \begin{center}
                \begin{minipage}[t]{0.45\textwidth}
                    \epsfxsize=1.0\linewidth
                    \epsffile{\directory/data.tshurst.eps}
                    \caption[{\market}, Hurst coefficient data]{{\market},
                        Hurst coefficient data for the normalized
                        increments of the time series data shown in
                        Figure~\ref{\SETLABEL:TF}.  The slope of the graph
                        is the Hurst coefficient.}
                    \label{\SETLABEL:HC}
                \end{minipage}
                \hfill
                \begin{minipage}[t]{0.45\textwidth}
                    \epsfxsize=1.0\linewidth
                    \epsffile{\directory/data.tshcalc.eps}
                    \caption[{\market}, H parameter data]{{\market}, H
                        parameter data for the normalized increments of
                        the time series data shown in
                        Figure~\ref{\SETLABEL:TF} The slope of the graph
                        is the H parameter.}
                    \label{\SETLABEL:HP}
                \end{minipage}
            \end{center}
        \end{figure}

        \subidx{revenue}{See, rate of revenue returns}
        \subidx{returns}{See, rate of revenue returns}
        \subidx{\market}{revenues}
        \subidx{Hurst coefficient}{analysis}
        \subidx{\market}{Hurst coefficient analysis}
        \subidx{\market}{rate of change}
        \subidx{\market}{windows of opportunity}
        \subidx{rate of revenue returns}{forecast}
        \subidx{forecast}{rate of revenue returns}
        \idx{windows of opportunity}
        \subidx{programs}{tslsq}
        \subidx{tslsq}{program}

        The approximately linear slope of the graph in
        Figure~\ref{\SETLABEL:HC} implies that the variance of the
        rate of revenue returns, (per {\timescale},) in the {\market},
        $V(t_2 - t_1)$, over a period of time is proportional to the
        period of time raised to twice the Hurst
        coefficient~\cite[pp. 180]{Feder},~\cite[pp. 246]{Crownover}.
        This seems to be a quantitative statement concerning how fast,
        and to what degree, the rate of revenue returns' state of
        affairs can change over a period of time.  An additional
        implication, for Hurst coefficients sufficiently close to 0.5,
        is that the probability of the state of affairs repeating
        sometime in the future goes down with increasing
        time\footnote{It can be shown that the number of expected
        market ``high'' and ``low'' transitions, $N$, scales with the
        square root of time, or $N \propto \sqrt {t}$, meaning that
        the cumulative distribution of the probability, $P$, of the
        duration of a market's ``high'' or ``low'' exceeding a given
        time interval, $t$, is proportional to the reciprocal of the
        square root of the time interval, $P \propto 1 / \sqrt {t}$,
        (or, conversely, that the probability of the duration of a
        market's ``high'' or ``low'' exceeding a given time interval
        is proportional to the reciprocal of the time interval raised
        to the power $3 / 2$, ie., $P \propto 1 / t^{3 /
        2}$,~\cite[pp. 153]{Schroeder}. What this means is that a
        histogram of the ``zero free'' run-lengths of a market being
        ``high'' or ``low,'' over a long time, would have a $1 / t^{3
        / 2}$ characteristic.)}, $t$, $p(t) = erf (1/\sqrt{2t})$ which
        is approximately $1/\sqrt{t}$ for $t \gg
        1$~\cite[pp. 160]{Schroeder}. Figures~\ref{\SETLABEL:FN},
        and,~\ref{\SETLABEL:FF} compare methods of approximation of
        the ``forecastability'' of the rate of revenue returns in the
        {\market} for the near term and far term,
        respectively~\cite[pp. 83-84]{Peters:CAOITCM}\footnote{The
        author is not comfortable with Peters' interpretation. For
        example, if the algorithm explained
        in~\cite[pp. 82]{Peters:CAOITCM} is used on ``white noise''
        which, by definition, never has any correlations, the short
        term Hurst coefficient, and thus the ``forecastability,'' is
        still near unity---a bit of an enigma. This can be verified
        with the {\it tswhite}\/ and {\it tshurst}\/ programs, which
        are briefly described in Appendix~\ref{programs}.}.  This
        seems to be a quantitative statement concerning ``windows of
        opportunity'' in the rate of revenue returns, (per
        {\timescale}.)  The program {\it tslsq}\/ was used on the
        Hurst coefficient data, presented in
        Figure~\ref{\SETLABEL:HC}, to provide a least squares
        approximation to the Hurst coefficient. The superimposed least
        squares approximation with on original Hurst coefficient data
        is presented.  The time series data has a Hurst coefficient of
        {\thurstlow}, so that:

        \subidx{\market}{Hurst coefficient analysis}
        \begin{eqnarray}
            V\left(t_2 - t_1\right) & \propto & \left(t_2 - t_1\right)^{2 \cdot H}\\
            V\left(t_2 - t_1\right) & \propto & \left(t_2 - t_1\right)^{2 \cdot {\thurstlow}}\\
                                    & \propto & \left(t_2 - t_1\right)^{\thurstlowtwo}
            \label{\SETLABEL:V}
        \end{eqnarray}

        \subidx{fractional}{Brownian motion}
        \subidx{Brownian motion}{fractional}
        \idx{fractal}
        \noindent where $V(t_2 - t_1)$ is the variance of the
        increments of the rate of revenue returns, (per {\timescale},)
        over the time interval $t_2 -
        t_1$,~\cite[pp. 177]{Feder},~\cite[pp. 494]{Peitgen}. If $H >
        \frac{1}{2}$, then the time series is termed as being
        characterized by ``fractional Brownian
        motion~\cite[pp. 170]{Feder}.''

        \subidx{rate of revenue returns}{predictability}
        \subidx{rate of revenue returns}{forecastability}
        \subidx{rate of revenue returns}{consistency}
        \subidx{predictability}{rate of revenue returns}
        \subidx{forecastability}{rate of revenue returns}
        \subidx{consistency}{rate of revenue returns}
        \subidx{\market}{rate of revenue returns, predictability}
        \subidx{\market}{rate of revenue returns, forecastability}
        \subidx{\market}{rate of revenue returns, consistency}
        \subidx{Hurst coefficient}{analysis}
        \subidx{\market}{Hurst coefficient analysis}
        \subidx{\market}{rate of change}

        In some sense, the Hurst coefficient is a quantitative
        expression of the ``forecastability'' of the future based on
        the past\footnote{Actually, in general, when summing fractal
        entities, the method used should be a root mean square
        process, dependent on the Hurst Coefficient, $H$, where
        $P_{total}^H = P_1^H + P_2^H + \cdots$, where $P_n$ is the
        fractal entities. For a Brownian motion, or random walk type
        of fractal the Hurst Coefficient is a function of time into
        the future. For the ``near term,'' the Hurst coefficient is
        very near unity, meaning the summation process is linear. For
        the ``long term,'' $H \approx 0.5$, or a standard root mean
        square summation process should be used. If $H$ is $0.5$ then
        the market is termed a Brownian motion, or random walk
        process. If it is larger than 0.5, it is termed fractional
        Brownian motion process. For a random walk process, ``near
        term'' and ``far term'' are quantitatively differentiated on
        the Hurst Coefficient graph where $1 - \ln (t) = 0.5 \cdot \ln
        (t)$, or when $\ln (t) = 2$, or $t = 7.389\ldots$ See
        Section~\ref{\SETLABEL:FS} for the particulars on using Hurst
        Coefficient to sum fractal process' for the {\market}. See
        also~\cite[pp. 67, 83-84]{Peters:CAOITCM} and~\cite[pp. 129,
        159]{Schroeder} for particulars on the implications of the
        Hurst Coefficient and root mean square summation issues.}.  A
        Hurst coefficient of {\thurstlow}, (for the near future, and
        {\thurstall} for the distant future.) implies that the
        likelihood of the rate of revenue returns, (per {\timescale},)
        for any two consecutive {\timescale}s being the same is
        {\thurstlowhundred}\%~\cite[pp. 66]{Peters:CAOITCM} for the
        near future, and {\thurstall} for the distant
        future. Likewise, there is a {\thurstlowhundred}\% chance of
        the rate of revenue returns, (per {\timescale},) movements
        being the same in consecutive time periods---ie., if, in a
        given {\timescale}, the rate of revenue returns, (per
        {\timescale},) is increasing, there is a {\thurstlowhundred}\%
        that the rate of revenue returns, (per {\timescale},) will
        increase in the following period, also. In some sense, this is
        a quantitative statement on how ``predictable,'' or
        ``forecastable'' the rate of revenue returns, (per
        {\timescale},) for the {\market} are over time, since the
        probability of having $n$ many consecutive {\timescale}s of
        the same agenda is $H^n$ where $H$ is the Hurst coefficient,
        or, letting the short term probability of having $n$ many
        {\timescale}s of the same market agenda, $p_a$, is:

        \begin{eqnarray}
            p_a\left(n\right) & = & H^{n}\\
                              & = & {\thurstlow}^{n}
            \label{\SETLABEL:MA}
        \end{eqnarray}

        \subidx{rate of revenue returns}{predictability}
        \subidx{rate of revenue returns}{forecastability}
        \subidx{rate of revenue returns}{consistency}
        \subidx{predictability}{rate of revenue returns}
        \subidx{forecastability}{rate of revenue returns}
        \subidx{consistency}{rate of revenue returns}
        As an interesting interpretation of the normalized increments
        of the time series data presented in
        Figure~\ref{\SETLABEL:TF}, if the vertical axis is multiplied
        by 100, to convert to percent, then the graph represents the
        error, in percent, that would be made by forecasting, month by
        month, that the next {\timescale}'s rate of revenue returns
        would be the same as the current {\timescale}'s revenue
        rate. Interestingly, it is $\datafractionmean \cdot 100$
        percent, on the average, with a standard deviation of
        $\datafractionstddev \cdot 100$ percent, and a root mean
        square error value of $\datafractionrms \cdot 100$
        percent---small values for such a simple forecasting
        mechanism.

        \subidx{\market}{rate of revenue returns, range}
        \subidx{Hurst coefficient}{analysis}
        \subidx{\market}{Hurst coefficient analysis}
        \subidx{\market}{rate of change}

        This is, essentially, a statement of the range of values, in
        the increments of the rate of revenue returns, (per
        {\timescale},) that is to be expected over the time interval,
        $t_2 - t_1$,
        $R_v$,~\cite[pp. 178]{Feder},~\cite[pp. 172]{Cambel}:

        \begin{eqnarray}
            R_v\left(t_2 - t_1\right) & \propto & \left(t_2 - t_1\right)^{H}\\
                                      & \propto & \left(t_2 - t_1\right)^{\thurstlow}
            \label{\SETLABEL:R}
        \end{eqnarray}

        \subidx{\market}{rate of revenue returns, range}
        \subidx{Hurst coefficient}{analysis}
        \subidx{\market}{Hurst coefficient analysis}
        \subidx{\market}{rate of change}
        \subidx{Markov}{statistics}
        \subidx{statistics}{Markov}
        \noindent where $R$ is the range of values in the increments
        of the rate of revenue returns, (per {\timescale}.) A Hurst
        coefficient, $H$, that is much larger than $\frac{1}{2}$, (but
        less than 1,) implies a strongly non-Gaussian distribution in
        the increments of the rate of revenue returns, (per
        {\timescale},)~\cite[pp. 152, 194]{Feder}, and a Hurst
        coefficient near $\frac{1}{2}$ implies that the increments of
        the rate of revenue returns, (per {\timescale}) is
        characteristic of an independent
        process~\cite[pp. 195]{Feder}. Extreme caution should be
        exercised in using Markov statistics in any analysis where the
        Hurst coefficient is not
        $\frac{1}{2}$,~\cite[pp. 124]{Crownover},~\cite[pp. 106]{Peters:CAOITCM}.


        As a useful approximation, if $H$, is approximately
        $\frac{1}{2}$, Equation~\ref{\SETLABEL:R} reduces
        to,~\cite[pp. 129]{Schroeder}:

        \begin{eqnarray}
            R\left(t_2 - t_1\right) & \propto & (t_2 - t_1)^{\frac{1}{2}}\\
                                    & \propto & \sqrt{\left(t_2 - t_1\right)}
        \end{eqnarray}

        \subidx{\market}{rate of revenue returns, range}
        \subidx{\market}{rate of revenue returns, increase and decrease}
        \subidx{Hurst coefficient}{analysis}
        \subidx{\market}{Hurst coefficient analysis}
        \subidx{\market}{rate of change}
        \subidx{Markov}{statistics}
        \subidx{statistics}{Markov}

        In the case where the Hurst coefficient, $H$, is
        $\frac{1}{2}$, the range of values in the increments of the
        rate of revenue returns, (per {\timescale},) divided by the
        standard deviation of these values, $S$, can be anticipated to
        increase over time according to the following
        relation,~\cite[pp. 154]{Feder},~\cite[pp. 129]{Schroeder}:

        \begin{equation}
            \frac{R\left(t_2 - t_1\right)}{S} \propto \left(t_2 - t_1\right)^{\frac{1}{2}}
        \end{equation}

        \subidx{\market}{rate of revenue returns, range}
        \subidx{\market}{rate of revenue returns, increase and decrease}
        \subidx{Hurst coefficient}{analysis}
        \subidx{\market}{Hurst coefficient analysis}
        \subidx{\market}{rate of change}
        \noindent which is a useful conceptual approximation, since it
        involves only the square root function---if the range and the
        standard deviation of the increments of the rate of revenue
        returns, (per {\timescale},) are known, (and $H \approx
        \frac{1}{2}$,) then the expected change in $\frac{R}{S}$, will
        increase with the square root of time\footnote{To be precise,
        it is actually asymptotically proportional to
        $\tau^{\frac{1}{2}}$}.

        Another useful approximation when rescaling processes that are
        characterize by Brownian motion, (ie., when $H \approx
        \frac{1}{2}$,) is that:

        \begin{eqnarray}
            X\left(t\right) & \propto & \frac{X\left(rt\right)}{r^{H}}\\
                            & \propto & \frac{X\left(rt\right)}{r^{\thurstlow}}
        \end{eqnarray}

        \idx{Brownian motion}
        \idx{fractal}
        Where $X(t)$ is the process characterized by Brownian motion,
        and $r$ is a scaling factor,~\cite[pp. 494]{Peitgen}.

        \subidx{programs}{tslsq}
        \subidx{tslsq}{program}
        The program {\it tslsq}\/ was used on the H parameter data,
        presented in Figure~\ref{\SETLABEL:HP}, to provide a least
        squares approximation to the H parameter for the
        {\market}. The superimposed least squares approximation on the
        original H parameter data is presented.  By contrast, the H
        parameter, as derived by the methodology outlined
        in~\cite[pp. 249]{Crownover}, is {\thcalclow} for the near
        future, and {\thcalcall} for the distant future.

        \subidx{\market}{Hurst coefficient analysis}
        \subidx{Hurst coefficient}{analysis}
        \subidx{increments}{normalized}
        \subidx{normalized}{increments}
        \subidx{programs}{tshurst}
        \subidx{tshurst}{program}
        \subidx{\market}{H parameter analysis}
        \subidx{H parameter}{analysis}
        \subidx{programs}{tshcalc}
        \subidx{tshcalc}{program}
        Figures~\ref{\SETLABEL:HC} and~\ref{\SETLABEL:HP} represent
        Hurst coefficient and H parameter data that are derived from
        the normalized increments, shown in
        Figure~\ref{\SETLABEL:TF}. In this case, the data is
        considered a normalized derivative of the time series data
        presented in Figure~\ref{\SETLABEL:TF}, instead of a
        cumulative sum.  The program, {\it tshurst}\/, is described
        briefly in appendix~\ref{programs}, and the data for
        figures~\ref{\SETLABEL:THC} and~\ref{\SETLABEL:THP} was made
        using the -d option.

        \begin{figure}[ht]
            \begin{center}
                \begin{minipage}[t]{0.45\textwidth}
                    \epsfxsize=1.0\linewidth
                    \epsffile{\directory/data.tsfraction.tshurst-d.eps}
                    \caption[{\market}, traditional Hurst coefficient
                        data]{{\market}, traditional Hurst coefficient
                        data for the time series data shown in
                        Figure~\ref{\SETLABEL:TS}.  The slope of the
                        graph is the Hurst coefficient, and is
                        {\hurstlow} for the near term, and
                        {\hurstall} for the far term.}
                    \label{\SETLABEL:THC}
                \end{minipage}
                \hfill
                \begin{minipage}[t]{0.45\textwidth}
                    \epsfxsize=1.0\linewidth
                    \epsffile{\directory/data.tsfraction.tshcalc-d.eps}
                    \caption[{\market}, traditional H parameter
                        data]{{\market}, traditional H parameter data
                        for the time series data shown in
                        Figure~\ref{\SETLABEL:TS} The slope of the
                        graph is the H parameter, and is {\hcalclow}
                        for the near term, and {\hcalcall} for the
                        far term.}
                    \label{\SETLABEL:THP}
                \end{minipage}
            \end{center}
        \end{figure}

% Local Variables:
% TeX-parse-self: t
% TeX-auto-save: t
% TeX-master: "fractal.tex"
% End:


        %
% -----------------------------------------------------------------------------
%
% A license is hereby granted to reproduce this software source code and
% to create executable versions from this source code for personal,
% non-commercial use.  The copyright notice included with the software
% must be maintained in all copies produced.
%
% THIS PROGRAM IS PROVIDED "AS IS". THE AUTHOR PROVIDES NO WARRANTIES
% WHATSOEVER, EXPRESSED OR IMPLIED, INCLUDING WARRANTIES OF
% MERCHANTABILITY, TITLE, OR FITNESS FOR ANY PARTICULAR PURPOSE.  THE
% AUTHOR DOES NOT WARRANT THAT USE OF THIS PROGRAM DOES NOT INFRINGE THE
% INTELLECTUAL PROPERTY RIGHTS OF ANY THIRD PARTY IN ANY COUNTRY.
%
% Copyright (c) 1994-2006, John Conover, All Rights Reserved.
%
% Comments and/or bug reports should be addressed to:
%
%     john@email.johncon.com (John Conover)
%
% -----------------------------------------------------------------------------
%
% Revision: \RCSRevision \\
% Revision Time: \RCSTime UMT \\
% Revision Date: \RCSDate \\
% Revision Id: \RCSId \\
% Revision File: \RCSLog \\
\RCS $Revision: 0.0 $
\RCS $Date: 2006/01/20 04:38:13 $
\RCS $Id: fiscal.tex,v 0.0 2006/01/20 04:38:13 john Exp $
% $Log: fiscal.tex,v $
% Revision 0.0  2006/01/20 04:38:13  john
% Initial version
%
%
    \subsection{Fixed Increment Approximation for Fiscal Strategy}
        \label{\SETLABEL:FS}

        \subidx{\market}{fiscal strategy}
        \subidx{markets}{analysis}
        \subidx{analysis}{markets}
        \subidx{strategy}{fiscal}
        \subidx{fiscal}{strategy}
        The data in this section is presented in tabular form in
        Section~\ref{\SETLABELREF:LR}. This section derives various
        values based on the ``average'' of the normalized increments
        presented in Figure~\ref{\SETLABEL:TFA}. These values are an
        approximation to a, probably, complex process with a
        distribution shown in Figure~\ref{\SETLABEL:TF}. These values
        will be used in a fixed increment Brownian fractal analysis
        and simulation of the {\market}, and may, or may not, provide
        adequate accuracy for projections.

        For an organization operating in the {\market}, the fiscal
        strategy, commensurate with the aggregate environment, can be
        derived as follows~\cite[pp. 128, pp
        151]{Schroeder},~\cite[pp. 450]{Reza},~\cite[pp. 270]{Pierce}:
        \vspace{0.15in}

        \subsubsection{Logarithmic Returns}
            \label{\SETLABEL:LR}

            \subidx{logarithmic}{returns}
            \subidx{returns}{logarithmic}
            \subidx{\market}{logarithmic returns}
            The logarithmic returns can be calculated by various
            means. Four will be presented here, for comparison.

            \subidx{programs}{tsnormal}
            \subidx{tsnormal}{program}
            \subidx{logarithmic}{returns}
            \subidx{returns}{logarithmic}
            The logarithmic returns, in bits, $bits$, as computed from
            the mean, by the program {\it tsnormal}\/, which is
            described in Chapter~\ref{programs}, and is presented in
            Figure~\ref{\SETLABEL:TF}, and Equation~\ref{abits} from
            Section~\ref{ereturns} in Chapter~\ref{general}:

            \begin{equation}
                bits = \frac{\ln \left({\datafractionmean} + 1\right)}{\ln \left(2\right)} = \datafractionmeanbits
            \end{equation}

            \subidx{programs}{tslsq}
            \subidx{tslsq}{program}
            \subidx{logarithmic}{returns}
            \subidx{returns}{logarithmic}
            \noindent By comparison, the logarithmic returns, in bits,
            $bits$, as computed from the constant in the least squares
            approximation, using the program {\it tslsq}\/, which is briefly
            described in Chapter~\ref{programs}, as presented in
            Figure~\ref{\SETLABEL:TF}, and Equation~\ref{abits} from
            Section~\ref{ereturns} in Chapter~\ref{general}:

            \begin{equation}
                bits = \frac{\ln \left({\datafractionconstant} + 1\right)}{\ln \left(2\right)} = \datafractionconstantbits
            \end{equation}

            Note that if the mean is not constant in
            Figure~\ref{\SETLABEL:TF}, this method will not provide
            accurate results.

            \subidx{programs}{tslsq}
            \subidx{tslsq}{program}
            \subidx{logarithmic}{returns}
            \subidx{returns}{logarithmic}
            \noindent And by yet another comparison, using the program
            {\it tslsq}\/, which is briefly described in
            Chapter~\ref{programs}, with the -e -p options, to provide
            a formula for the least squares exponential fit to the
            time series data set presented in
            Figure~\ref{\SETLABEL:TS}:

            \begin{equation}
                bits = {\datatslsqepbits}
            \end{equation}

            \subidx{programs}{tslogreturns}
            \subidx{tslogreturns}{program}
            \subidx{logarithmic}{returns}
            \subidx{returns}{logarithmic}
            \noindent And finally, by comparison, from the
            {\it tslogreturns}\/ program, which is briefly described
            in Chapter~\ref{programs}, with the -p option, to provide
            a formula for the logarithmic returns of the time series
            data set presented in Figure~\ref{\SETLABEL:TS}:

            \begin{equation}
                bits = {\logreturns}
            \end{equation}

        \subsubsection{Calculation of Shannon Probability}
            \label{\SETLABEL:SP}

            \subidx{\market}{Shannon probability}
            Ideally, all of the values presented in
            Section~\ref{\SETLABEL:LR} would be equal. Using the
            logarithmic returns provided by the {\it tslogreturns}\/
            program, to be consistent
            with~\cite[pp. 81]{Peters:CAOITCM}

            \subidx{programs}{tslogreturns}
            \subidx{tslogreturns}{program}
            \begin{equation}
                2^{{\logreturns}t}
            \end{equation}

            \noindent therefore:
            \begin{equation}
                C\left(p\right) = {\logreturns}
            \end{equation}
            \subidx{programs}{tsshannon}
            \subidx{tsshannon}{program}
            \subidx{Shannon}{probability}
            \subidx{probability}{Shannon}
            \noindent and, {\it tsshannon}\/ {\logreturns} gives:
            \begin{equation}
                \label{\SETLABEL:F0}
                C\left({\shannonlogreturns}\right) = {\logreturns}
            \end{equation}
            \noindent therefore:
            \begin{eqnarray}
                2^{C\left({\shannonlogreturns}\right)} & = & 2^{\logreturns}\\
                                                       & = & {\twologreturns}\\
                                                       & = & {\twologreturnshundred}\%
            \end{eqnarray}
            \noindent and:
            \begin{eqnarray}
                2p - 1 & = & \left(2 \cdot {\shannonlogreturns}\right) - 1\\
                       & = & {\twopone}\\
                       \label{\SETLABEL:F1}
                       & = & {\twoponehundred}\%
            \end{eqnarray}

            \subidx{\market}{fiscal strategy}
            \subidx{markets}{analysis}
            \subidx{analysis}{markets}
            \subidx{strategy}{fiscal}
            \subidx{fiscal}{strategy}
            \subidx{\market}{fiscal strategy}
            \subidx{\market}{growth rate}
            Presuming the simplified assumptions outlined in
            Section~\ref{assumptions}, the ``typical'' organization
            operating in the {\market} executes a long term fiscal
            strategy, commensurate with the aggregate environment,
            that is to invest, every {\timescale}, in sufficient
            additional resources and infrastructure, to increase the
            manufacturing of goods and services by {\twoponehundred}\%
            of its rate of revenue returns, (per {\timescale}.) As a
            conceptual model, the remaining {\hundredtwoponehundred}\%
            will be held in ``reserve'' with a
            {\shannonlogreturnshundred}\% chance of making twice the
            {\twoponehundred}\% back, (and a
            {\hundredshannonlogreturnshundred}\% chance of making
            0.0,) in one {\timescale}, on the average, for an average
            growth in its rate of revenue returns, (per {\timescale},)
            of {\twologreturnshundred}\%, or a doubling of its rate of
            revenue returns, (per {\timescale},) in
            {\oneoverlogreturns} {\timescale}s.

        \subsubsection{Example Fixed Increment Approximation Fiscal Strategies}

            \subidx{\market}{fiscal strategy}
            \subidx{markets}{analysis}
            \subidx{analysis}{markets}
            \subidx{strategy}{fiscal}
            \subidx{fiscal}{strategy}
            \subidx{\market}{fiscal strategy}
            \subidx{\market}{growth rate}
            \subidx{\market}{management metric}
            \idx{management metric}
            A possible metric on the effectiveness of long term fiscal
            management could possibly be that if an investment of
            {\twoponehundred}\% per {\timescale} of the rate of
            revenue returns, (per {\timescale},) is made in resources
            and infrastructure, then the rate of revenue returns would
            be expected to increase by {\twologreturnshundred}\%, per
            {\timescale}, on average.

            Note that the metrics presented in this section are
            representative of the {\market} as an aggregate whole, and
            may or may not be accurate representations for any
            particular participant in the environment. Of interest to
            the participants in the environment would be a similar
            analysis of each product or service rendered in the
            marketplace.

            \subidx{\market}{fiscal strategy}
            \subidx{markets}{analysis}
            \subidx{analysis}{markets}
            \subidx{strategy}{fiscal}
            \subidx{fiscal}{strategy}
            \subidx{\market}{fiscal strategy}
            As a simple illustrative example, a company operating in
            this environment might obtain a credit line from a bank
            that is equal to {\twoponehundred}\% of its rate of
            revenue returns, (per {\timescale},) to finance additional
            operations. In this simple scenario, the company would use
            its revenue base as collateral for the loan. Some
            {\timescale}s, depending on the {\market}'s environment,
            the company's rate of revenue returns exceeds what was
            borrowed from the bank, and the loan is repaid in
            full. Other {\timescale}s, the company must default, and
            the bank seizes a portion of the company's revenue base to
            pay the delinquent loan. However, on the average, the
            company will expand its rate of revenue returns at
            {\twologreturnshundred}\% per {\timescale}.

            \subidx{\market}{fiscal strategy}
            \subidx{markets}{analysis}
            \subidx{analysis}{markets}
            \subidx{strategy}{fiscal}
            \subidx{fiscal}{strategy}
            \subidx{\market}{fiscal strategy}
            As another simple example, a company re-invests
            {\twoponehundred}\% of its rate of revenue returns, (per
            {\timescale},) in development, marketing, sales, and
            distribution of new products.  Although some products will
            be successful and the return on the investment will exceed
            the {\twoponehundred}\% per {\timescale} investment,
            others will not. However, on the average, the company will
            expand it gross rate of revenue returns at
            {\twologreturnshundred}\% per {\timescale}.

            \subidx{\market}{fiscal strategy}
            \subidx{markets}{analysis}
            \subidx{analysis}{markets}
            \subidx{strategy}{fiscal}
            \subidx{fiscal}{strategy}
            \subidx{\market}{fiscal strategy}
            \subidx{\market}{product portfolio}
            \subidx{\market}{product diversity}
            \subidx{\market}{product mix}
            \subidx{\market}{optimum number of products}
            \idx{product portfolio}
            \idx{product diversity}
            \idx{optimum number of products}
            \idx{product mix}

            As an example of ``product portfolio'' management, suppose
            a company re-invests {\twoponehundred}\% of its rate of
            revenue returns, (per {\timescale},) in development,
            marketing, sales, and distribution of new products.
            Further suppose that the company has two products, and a
            fractal analysis of the individual product rate of revenue
            return time series indicates that one product has a
            Shannon probability of 0.65, and the other has a Shannon
            probability of 0.55. Then the percentage of re-investment
            in the first product would be $(2 \cdot 0.65 - 1) \cdot
            {\twoponehundred}$, percent of the rate of revenue
            returns, and $(2 \cdot 0.55 - 1) \cdot {\twoponehundred}$
            percent for the second product, implying that the company
            should diversify its product line\footnote{The astute
            reader would note that the linear addition was used to add
            the contribution to development of each product. This is a
            ``near term'' interpretation. Actually, in general, the
            method used should be a root mean square process,
            dependent on the Hurst Coefficient, $H$, where
            $P_{total}^H = P_1^H + P_2^H + \cdots$, where $P_n$ is the
            contribution to each individual product. For a Brownian
            motion, or random walk type of fractal the Hurst
            Coefficient is a function of time into the future. For the
            ``near term,'' the Hurst coefficient is very near unity,
            meaning the summation process is linear. For the ``long
            term,'' $H \approx 0.5$, or a standard root mean square
            summation process should be used. If $H$ is $0.5$ then the
            market is termed a Brownian motion, or random walk
            process. If it is larger than 0.5, it is termed fractional
            Brownian motion process. For a random walk process, ``near
            term'' and ``far term'' are quantitatively differentiated
            on the Hurst Coefficient graph where $1 - \ln (t) = 0.5
            \cdot \ln (t)$, or when $\ln (t) = 2$, or $t =
            7.389\ldots$ See~\cite[pp. 67, 83-84]{Peters:CAOITCM}
            and~\cite[pp. 129, 159]{Schroeder} for particulars on the
            implications of the Hurst Coefficient and root mean square
            summation issues.}.  Note that this is a ``bet hedging''
            metric methodology, and assumes that the products have
            uncorrelated revenue return rates. If this re-investment
            methodology is not feasible, perhaps for strategic
            financial reasons, then the re-investment in both products
            should total the ${\twoponehundred}$\%, and the investment
            in each product should be made at a ratio of $\frac{(2
            \cdot 0.65 - 1)}{(2 \cdot 0.55 - 1)} = 3 : 1$,
            respectively. Note that this ``bet hedging'' can be used
            to define the optimal number of products that can be
            supported on the rate of revenue returns. If it assumed
            that all products are ``typical'' for the {\market}, as a
            standard bench mark, then the optimal number will be
            $\frac{1}{{\twopone}}$. Note that this is a
            ``theoretical'' value, since not all products are
            ``typical,'' and there may be strategic reasons, for
            example product leveraging, that may increase the number
            of products above the optimum. However, most of the
            revenue should come from the optimal number of products,
            since having more products will decrease the amount of the
            potential investment in each product, and having less than
            the optimum number of products will increase the risk that
            many of the products could suffer a ``down market''
            concurrently, impacting the rate of revenue returns.  As
            another interesting interpretation of the optimal
            ``hedging of bets,'' in product portfolio strategy, and
            considering the graph of the normalized increments
            presented in Figure~\ref{\SETLABEL:TF}, if the
            organization is running optimally, then these products
            will generate, at least in principle, one standard
            deviation, approximately $0.8413 = 84.13$\% of the future
            growth in rate of revenue returns. Naturally, these are
            approximations, and the values are an approximation to a,
            probably, complex process, and appropriate scrutiny should
            be exercised before making specific projections.  As yet
            another example of ``product portfolio'' management,
            consider the issue of product mix. In this interpretation,
            {\twoponehundred}\% of the product manufactured should be
            ``proprietary,'' while the rest is ``industry standard.''
            As yet another possibility, {\twoponehundred}\% of the
            product manufactured should be predatory into new markets,
            and the remainder in markets that are ``traditional'' for
            the company.

% Local Variables:
% TeX-parse-self: t
% TeX-auto-save: t
% TeX-master: "fractal.tex"
% End:


        %
% -----------------------------------------------------------------------------
%
% A license is hereby granted to reproduce this software source code and
% to create executable versions from this source code for personal,
% non-commercial use.  The copyright notice included with the software
% must be maintained in all copies produced.
%
% THIS PROGRAM IS PROVIDED "AS IS". THE AUTHOR PROVIDES NO WARRANTIES
% WHATSOEVER, EXPRESSED OR IMPLIED, INCLUDING WARRANTIES OF
% MERCHANTABILITY, TITLE, OR FITNESS FOR ANY PARTICULAR PURPOSE.  THE
% AUTHOR DOES NOT WARRANT THAT USE OF THIS PROGRAM DOES NOT INFRINGE THE
% INTELLECTUAL PROPERTY RIGHTS OF ANY THIRD PARTY IN ANY COUNTRY.
%
% Copyright (c) 1994-2006, John Conover, All Rights Reserved.
%
% Comments and/or bug reports should be addressed to:
%
%     john@email.johncon.com (John Conover)
%
% -----------------------------------------------------------------------------
%
% Revision: \RCSRevision \\
% Revision Time: \RCSTime UMT \\
% Revision Date: \RCSDate \\
% Revision Id: \RCSId \\
% Revision File: \RCSLog \\
\RCS $Revision: 0.0 $
\RCS $Date: 2006/01/20 04:38:13 $
\RCS $Id: companies.tex,v 0.0 2006/01/20 04:38:13 john Exp $
% $Log: companies.tex,v $
% Revision 0.0  2006/01/20 04:38:13  john
% Initial version
%
%
    \subsection{Number of Companies}
        \label{\SETLABEL:QNC}

        \subidx{\market}{number of companies}
        \subidx{number of companies}{analysis}
        \subidx{analysis}{number of companies}
        \subidx{Shannon}{probability}
        \subidx{probability}{Shannon}
        This section evaluates the approximate, or ``average,'' number
        of companies in the {\market}, and uses the method outlined in
        Chapter~\ref{general}, Section~\ref{aftsma}. Since the
        average, $avg_{ind}$, and the root mean square, $rms_{ind}$,
        of the normalized increments of the {\market} time series is
        \datafractionmean, and \datafractionrms respectively, the
        number of companies participating in the market can be
        calculated by Equation~\ref{ncompanies} to be {\ncompanies}.

        If this value seems consistent number of companies in the
        {\market}, within the assumptions outlined in
        Chapter~\ref{general}, Section~\ref{aftsma}, then it would
        seem that there is some circumstantial or indirect evidence
        that the companies participating in the {\market} are
        operating optimally, and the ``average'' Shannon probability,
        $P$ for each participating company would be, using
        Equation~\ref{pncompanies}, {\pncompanies}, which would be the
        value which should be used in Section~\ref{\SETLABEL:FS} for
        each participating company if market expansion was to be
        consistent with the rest of the industry. However, if the
        Shannon probability derived in Section~\ref{\SETLABEL:FS} is
        greater than the average Shannon probability for the companies
        participating in the {\market}, as derived in this section,
        then the market would, possibly, be exploitable with the
        fiscal strategy outlined in Section~\ref{\SETLABEL:FS}. The
        maximum exploitability for the {\market} is derived in
        Section~\ref{\SETLABEL:MAXSHANNON}, but it is probably of
        doubtful practicality.

        Note that these optimizations would maximize a company's
        market growth. Since there are probably many companies
        competing in the market place, this would not necessarily
        maximize a company's P\&L, as described in
        Chapter~\ref{general}, Section~\ref{ompl}. The Shannon
        probability that maximizes market share in the {\market} is
        \pncompanies, with several alternative solutions listed in the
        previous paragraph. However, these should be contrasted to the
        Shannon probability that maximizes a company's P\&L which is
        \avgrms~in the {\market}. In all cases, the fraction of the
        P\&L that should be ``wagered'' on the future, $f$, should be:

        \begin{equation}
            f = 2P - 1
        \end{equation}

        \noindent where $P$ is the particular Shannon probability
        chosen optimize a particular fiscal strategy. Interestingly,
        the measured Shannon probability of the {\market} would tend
        to indicate that the companies participating in the market
        have chosen a fiscal strategy that optimizes market growth, as
        opposed to capital growth.

        \subidx{\market}{increasing returns}
        \subidx{economic increasing returns}{\market}
        As interesting interpretation of these exploitive issues,
        since all three fiscal strategies will result in exponential
        market growth for every company participating in the market,
        is that they may represent, perhaps, an example of
        ``increasing returns.''

% Local Variables:
% TeX-parse-self: t
% TeX-auto-save: t
% TeX-master: "fractal.tex"
% End:


        %
% -----------------------------------------------------------------------------
%
% A license is hereby granted to reproduce this software source code and
% to create executable versions from this source code for personal,
% non-commercial use.  The copyright notice included with the software
% must be maintained in all copies produced.
%
% THIS PROGRAM IS PROVIDED "AS IS". THE AUTHOR PROVIDES NO WARRANTIES
% WHATSOEVER, EXPRESSED OR IMPLIED, INCLUDING WARRANTIES OF
% MERCHANTABILITY, TITLE, OR FITNESS FOR ANY PARTICULAR PURPOSE.  THE
% AUTHOR DOES NOT WARRANT THAT USE OF THIS PROGRAM DOES NOT INFRINGE THE
% INTELLECTUAL PROPERTY RIGHTS OF ANY THIRD PARTY IN ANY COUNTRY.
%
% Copyright (c) 1994-2006, John Conover, All Rights Reserved.
%
% Comments and/or bug reports should be addressed to:
%
%     john@email.johncon.com (John Conover)
%
% -----------------------------------------------------------------------------
%
% Revision: \RCSRevision \\
% Revision Time: \RCSTime UMT \\
% Revision Date: \RCSDate \\
% Revision Id: \RCSId \\
% Revision File: \RCSLog \\
\RCS $Revision: 0.0 $
\RCS $Date: 2006/01/20 04:38:13 $
\RCS $Id: operations.tex,v 0.0 2006/01/20 04:38:13 john Exp $
% $Log: operations.tex,v $
% Revision 0.0  2006/01/20 04:38:13  john
% Initial version
%
%
    \subsection{Fixed Increment Approximation for Operational Strategy}
        \label{\SETLABEL:OPS}.

        This section derives various values based on the ``average''
        of the normalized increments presented in
        Figure~\ref{\SETLABEL:TFA}. These values are an approximation
        to a, probably, complex process with a distribution shown in
        Figure~\ref{\SETLABEL:TF}. These values will be used in a
        fixed increment Brownian fractal analysis and simulation of
        the {\market}, and may, or may not, provide adequate accuracy
        for projections.

        \subidx{\market}{fiscal strategy}
        \subidx{\market}{Shannon probability}
        \subidx{strategy}{fiscal}
        \subidx{fiscal}{strategy}
        \subidx{Shannon}{probability}
        \subidx{probability}{Shannon}
        It should be noted that the analysis of fiscal strategy,
        presented in Section~\ref{\SETLABEL:FS}, is derived from the
        {\market} metrics and may, or may not, be maximally
        optimal. For the optimal fiscal strategy, which may be
        exploitable, see Section~\ref{\SETLABEL:MAXSHANNON}.

        \subidx{strategy}{exploitable}
        \subidx{exploitable}{strategy}
        \subidx{\market}{windows of opportunity}
        \idx{windows of opportunity}
        \subidx{decision}{obsolete}
        \subidx{obsolete}{decision}
        \subidx{decision}{timeliness}
        \subidx{timeliness}{decision}
        \subidx{rate of revenue returns}{forecast}
        \subidx{forecast}{rate of revenue returns}
        An additional exploitable strategy may be time itself.
        Equations~\ref{\SETLABEL:V},~\ref{\SETLABEL:R},
        and,~\ref{\SETLABEL:MA}, are, essentially, metrics on how fast
        a decision, which is based on information concerning the
        current status of the {\market}, becomes obsolete. Obviously,
        how long a decision is expected to remain relevant should be
        addressed as an operational necessity in strategic planning
        and project management. Figures~\ref{\SETLABEL:FN},
        and,~\ref{\SETLABEL:FF} compare methods of approximation of
        the ``forecastability'' of rate of revenue returns in the
        {\market} for the near term and far
        term~\cite[pp. 83-84]{Peters:CAOITCM}, respectively. As a
        general rule, caution must be exercised when making decisions
        that will span a time interval larger than the time interval
        where the ``forecastability'' of rate of revenue returns drops
        below 50\%. Beyond this time interval, the chances increase
        that the competitive and market forces will alter the market
        environment in a possibly detrimental unanticipated
        fashion. Obviously, there is significant advantage in
        ``timeliness'' of development, manufacturing, and distribution
        of products and services that are consistent with this
        temporal agenda. Automation of these processes, if executed
        consistently with this agenda, should be considered a
        competitive advantage.

        \subidx{strategy}{exploitable}
        \subidx{exploitable}{strategy}
        \subidx{rate of revenue returns}{forecast}
        \subidx{forecast}{rate of revenue returns}
        \idx{product life cycle}
        \idx{life cycle, product}
        In some sense, this temporal agenda defines the ``average''
        product or service life cycle in the {\market}. When the
        ``forecastability'' of rate of revenue returns drops below
        50\%, there is an even chance that the rate of revenue returns
        for the product or service will change in a detrimental
        fashion. If it is assumed that a product or service life cycle
        consists of a ramp up, a maintenence interval, and a ramp
        down, then, if all three life cycle intervals are equal, the
        product life cycle will be, approximately, three times the
        time interval where the ``forecastability'' of rate of revenue
        returns drops below 50\%. Although probably not an accurate
        prediction of product or service life cycle, the technique may
        be used as a conceptual approximation to the dynamics of
        ``market windows.\footnote{For example, consider the market
        for table salt. Since it has inelastic supply and demand
        curves, and is a necessary requirement for life, it would be
        expected that the Hurst coefficient would be very near
        unity---ignoring competitive pressures in the market. The
        predictability of the table salt market would, therefore, be
        expected to be relatively good, over time.}''  The conceptual
        approximation will probably predict a ``conservative'' or
        ``pessimistic'' value in relation to actual markets.

        \begin{figure}[ht]
            \begin{center}
                \begin{minipage}[t]{0.45\textwidth}
                    \epsfxsize=1.0\linewidth
                    \epsffile{\directory/datahurstlownear.eps}
                    \caption[{\market}, ``forecastability'' of near
                        term rate of revenue returns]{{\market},
                        ``forecastability'' of near term rate of
                        revenue returns. Although the error function
                        is the most accurate, for the near term,
                        $H^{t} = \thurstlow^{t}$ may be used as a
                        reliable metric of ``forecastability'' of the
                        rate of revenue returns.}
                    \label{\SETLABEL:FN}
                \end{minipage}
                \hfill
                \begin{minipage}[t]{0.45\textwidth}
                    \epsfxsize=1.0\linewidth
                    \epsffile{\directory/datahurstlowfar.eps}
                    \caption[{\market}, ``forecastability'' of far
                        term rate of revenue returns]{{\market},
                        ``forecastability'' of far term rate of
                        revenue returns. Although the error function
                        is the most accurate, for the far term,
                        $\frac{1}{\sqrt{t}}$ may be used as a reliable
                        metric of ``forecastability'' of the rate of
                        revenue returns.}
                    \label{\SETLABEL:FF}
                \end{minipage}
            \end{center}
        \end{figure}

        \idx{operations research}
        As an interesting interpretation of the data presented in
        Figure~\ref{\SETLABEL:FN}, there may be, perhaps, some
        applicability to such operational agendas as inventory
        control. Maintaining too little inventory, obviously, will
        create a situation where the organization can not exploit
        market expansion, and maintaining too much inventory,
        likewise, would over extend the company, creating unnecessary
        losses when the market contracts. The company should maintain
        inventory levels that do not exceed, from
        Equation~\ref{\SETLABEL:MA}, ${\thurstlow}^{n} = 0.5$
        {\timescale}s of operations. Since the optimal amount of
        inventory and, from Equation~\ref{\SETLABEL:V}, the variance
        of change in the rate of revenue returns in the future can be
        calculated, there may, perhaps, be some applicability to a
        forecasting methodology that can be incorporated into other
        areas of operations research, for example the linear algebras
        using simplex methodologies for optimization of manufacturing
        processes. Traditionally, these forecasts are made by the
        sales department, and are subject to various subjective
        biases.

% Local Variables:
% TeX-parse-self: t
% TeX-auto-save: t
% TeX-master: "fractal.tex"
% End:


        %
% -----------------------------------------------------------------------------
%
% A license is hereby granted to reproduce this software source code and
% to create executable versions from this source code for personal,
% non-commercial use.  The copyright notice included with the software
% must be maintained in all copies produced.
%
% THIS PROGRAM IS PROVIDED "AS IS". THE AUTHOR PROVIDES NO WARRANTIES
% WHATSOEVER, EXPRESSED OR IMPLIED, INCLUDING WARRANTIES OF
% MERCHANTABILITY, TITLE, OR FITNESS FOR ANY PARTICULAR PURPOSE.  THE
% AUTHOR DOES NOT WARRANT THAT USE OF THIS PROGRAM DOES NOT INFRINGE THE
% INTELLECTUAL PROPERTY RIGHTS OF ANY THIRD PARTY IN ANY COUNTRY.
%
% Copyright (c) 1994-2006, John Conover, All Rights Reserved.
%
% Comments and/or bug reports should be addressed to:
%
%     john@email.johncon.com (John Conover)
%
% -----------------------------------------------------------------------------
%
% Revision: \RCSRevision \\
% Revision Time: \RCSTime UMT \\
% Revision Date: \RCSDate \\
% Revision Id: \RCSId \\
% Revision File: \RCSLog \\
\RCS $Revision: 0.0 $
\RCS $Date: 2006/01/20 04:38:13 $
\RCS $Id: simulation.tex,v 0.0 2006/01/20 04:38:13 john Exp $
% $Log: simulation.tex,v $
% Revision 0.0  2006/01/20 04:38:13  john
% Initial version
%
%
    \subsection{Simulation of Fixed Increment Approximation for Fiscal Strategy}
        \label{\SETLABEL:TSUNFAIRBROWNIAN}

        \subidx{\market}{market simulation}
        The data in this section is presented in tabular form in
        Section~\ref{\SETLABELREF:SIM}.
        Figure~\ref{\SETLABEL:TSUNFAIRBROWNIAN0} represents a
        constructional simulation of the time series data presented in
        Figure~\ref{\SETLABEL:TS}. The program {\it
        tsunfairbrownian}\/, which is briefly described in
        appendix~\ref{programs}, was used in the reconstruction. The
        reconstructed data is superimposed on the original time series
        data.  The program, {\it tsunfairbrownian}\/, essentially,
        constructs the new time series as a Brownian fractal with
        fixed increments---the value of the fixed increment is derived
        from the root mean square average of the normalized increments
        presented in Figure~\ref{\SETLABEL:TF}. The ``quality'' of
        such a reconstruction should be subject to adequate scepticism
        and scrutiny since, in all probability, the normalized
        increments presented in Figure~\ref{\SETLABEL:TF} represent a
        relatively complex process, that may not be ``modeled'' with
        such a simple methodology.

        As a further comparison of the the constructional simulation
        with the original time series data,
        Figure~\ref{\SETLABEL:TSUNFAIRBROWNIAN1} presents a normalized
        histogram of the normalized increments of the reconstructed
        time series, superimposed on the normalized histogram
        presented in Figure~\ref{\SETLABEL:NH}.

        \subidx{\market}{fiscal strategy, simulation}
        \subidx{markets}{simulation}
        \subidx{simulation}{markets}
        \subidx{strategy}{fiscal, simulation}
        \subidx{fiscal}{strategy, simulation}
        \subidx{programs}{tsunfairbrownian}
        \subidx{tsunfairbrownian}{program}
        \begin{figure}[ht]
            \begin{center}
                \begin{minipage}[t]{0.45\textwidth}
                    \epsfxsize=1.0\linewidth
                    \epsffile{\directory/tsunfairbrownian-f.eps}
                    \caption[{\market}, Time series data, empirical and
                        simulated]{{\market}, Time series data, empirical
                        and simulated, using the program {\it tsunfairbrownian}\/
                        with f = {\datafractionrms}. This data is
                        superimposed on the data presented in
                        Figure~\ref{\SETLABEL:TS}.}
                    \label{\SETLABEL:TSUNFAIRBROWNIAN0}
                \end{minipage}
                \hfill
                \begin{minipage}[t]{0.45\textwidth}
                    \epsfxsize=1.0\linewidth
                    \epsffile{\directory/tsunfairbrownian-f.tsfraction.tsnormal-s30.eps}
                    \caption[{\market}, normalized histogram,
                        empirical and simulated]{{\market}, normalized
                        histogram of the normalized increments of the
                        time series data shown in
                        Figure~\ref{\SETLABEL:TSUNFAIRBROWNIAN0},
                        empirical and simulated.  The empirical data
                        has a mean of {\datafractionmean}, with a
                        standard deviation of {\datafractionstddev}.
                        By comparison, the simulated data has a mean
                        of {\tsunfairbrownianfractionmean} with a
                        standard deviation of
                        {\tsunfairbrownianfractionstddev}. This data
                        is superimposed on the data presented in
                        Figure~\ref{\SETLABEL:NH}. The area under the
                        four curves is identical.}
                    \label{\SETLABEL:TSUNFAIRBROWNIAN1}
                \end{minipage}
            \end{center}
        \end{figure}

% Local Variables:
% TeX-parse-self: t
% TeX-auto-save: t
% TeX-master: "fractal.tex"
% End:


        %
% -----------------------------------------------------------------------------
%
% A license is hereby granted to reproduce this software source code and
% to create executable versions from this source code for personal,
% non-commercial use.  The copyright notice included with the software
% must be maintained in all copies produced.
%
% THIS PROGRAM IS PROVIDED "AS IS". THE AUTHOR PROVIDES NO WARRANTIES
% WHATSOEVER, EXPRESSED OR IMPLIED, INCLUDING WARRANTIES OF
% MERCHANTABILITY, TITLE, OR FITNESS FOR ANY PARTICULAR PURPOSE.  THE
% AUTHOR DOES NOT WARRANT THAT USE OF THIS PROGRAM DOES NOT INFRINGE THE
% INTELLECTUAL PROPERTY RIGHTS OF ANY THIRD PARTY IN ANY COUNTRY.
%
% Copyright (c) 1994-2006, John Conover, All Rights Reserved.
%
% Comments and/or bug reports should be addressed to:
%
%     john@email.johncon.com (John Conover)
%
% -----------------------------------------------------------------------------
%
% Revision: \RCSRevision \\
% Revision Time: \RCSTime UMT \\
% Revision Date: \RCSDate \\
% Revision Id: \RCSId \\
% Revision File: \RCSLog \\
\RCS $Revision: 0.0 $
\RCS $Date: 2006/01/20 04:38:13 $
\RCS $Id: maximum.tex,v 0.0 2006/01/20 04:38:13 john Exp $
% $Log: maximum.tex,v $
% Revision 0.0  2006/01/20 04:38:13  john
% Initial version
%
%
    \subsection{Simulation of Fixed Increment Approximation for Optimally Maximal Fiscal Strategy}
        \label{\SETLABEL:MAXSHANNON}
        \subidx{\market}{fiscal strategy, simulation}
        \subidx{\market}{maximum Shannon probability}
        \subidx{markets}{simulation}
        \subidx{simulation}{markets}
        \subidx{strategy}{optimum fiscal, simulation}
        \subidx{fiscal}{optimum strategy, simulation}
        \subidx{programs}{tsunfairbrownian}
        \subidx{tsunfairbrownian}{program}
        \subidx{Shannon}{probability}
        \subidx{probability}{Shannon}

        \subidx{strategy}{exploitable}
        \subidx{exploitable}{strategy}
        \subidx{programs}{tsshannonmax}
        \subidx{tsshannonmax}{program}
        \subidx{programs}{tsunfairbrownian}
        \subidx{tsunfairbrownian}{program}
        \subidx{strategy}{fiscal}
        \subidx{fiscal}{strategy}
        The data in this section is presented in tabular form in
        Section~\ref{\SETLABELREF:MAXSHANNON}. One of the issues of
        analysis, as mentioned in Section~\ref{\SETLABEL:OPS}, is to
        determine the maximum Shannon probability for the time series
        presented in Figure~\ref{\SETLABEL:TS}. Potentially, this
        could be exploited with an aggressive fiscal
        strategy. Figure~\ref{\SETLABEL:SHANNONMAX0} is a graph of the
        output of the {\it tsshannonmax}\/ program, which is described
        briefly in appendix~\ref{programs}. The maximum of this
        function is the maximum Shannon probability for the time
        series data presented in Figure~\ref{\SETLABEL:TS}.
        Figure~\ref{\SETLABEL:SHANNONMAX1} was constructed using {\it
        tsunfairbrownian}\/ program, which is also described in
        appendix~\ref{programs}, with the maximum Shannon probability,
        and the time series data presented in
        Figure~\ref{\SETLABEL:TS}. This represents a ``what if'' the
        investment strategy was changed from a Shannon probability of
        {\shannonlogreturns}, as derived in Section~\ref{\SETLABEL:SP}
        to {\shannonmax}. This process, essentially, extracts the
        random statistical data from the time series presented in
        Figure~\ref{\SETLABEL:TS}, and constructs a new time series,
        using the random statistical data, with a different investment
        strategy.  The program, {\it tsunfairbrownian}\/, essentially,
        constructs the new time series as a Brownian fractal with
        fixed increments.  The ``quality'' of such a reconstruction
        should be subject to adequate scepticism and scrutiny since,
        in all probability, the increments in the original data
        represent a relatively complex process, that may not be
        ``modeled'' with such a simple methodology.

        \begin{figure}[ht]
            \begin{center}
                \begin{minipage}[t]{0.45\textwidth}
                    \epsfxsize=1.0\linewidth
                    \epsffile{\directory/data.tsshannonmax.eps}
                    \caption[{\market}, maximum rate of revenue
                        returns] {{\market}, maximum rate of revenue
                        returns, per {\timescale}, vs. Shannon
                        probability. The maximum rate of revenue
                        returns, per {\timescale}, occurs at a Shannon
                        probability of {\shannonmax}.}
                    \label{\SETLABEL:SHANNONMAX0}
                \end{minipage}
                \hfill
                \begin{minipage}[t]{0.45\textwidth}
                    \epsfxsize=1.0\linewidth
                    \epsffile{\directory/data.tsshannonmax-p.tsunfairbrownian-p.eps}
                    \caption[{\market}, maximum rate of revenue
                        returns] {{\market}, maximum rate of revenue
                        returns, per {\timescale}, at a Shannon
                        probability, of {\shannonmax}, corresponding
                        to a ``wager'' fraction of {\twoponemax}.}
                    \label{\SETLABEL:SHANNONMAX1}
                \end{minipage}
            \end{center}
        \end{figure}

        \subidx{fractional}{Brownian motion}
        \subidx{Brownian motion}{fractional}
        \subidx{Shannon}{probability}
        \subidx{probability}{Shannon}
        \subidx{programs}{tsshannonmax}
        \subidx{tsshannonmax}{program}
        If it is assumed that the time series data set, presented in
        Figure~\ref{\SETLABEL:TS}, constitutes classical Brownian
        motion, then the Shannon probability can be calculated by
        counting the total number of {\timescale}s that the {\market}
        movement was positive, and dividing by the total number of
        {timescale}s represented in the time series. This quotient is
        {\pmax}, as compared with the predicted value from the program
        {\it tsshannonmax}\/ of {\shannonmax}.

% Local Variables:
% TeX-parse-self: t
% TeX-auto-save: t
% TeX-master: "fractal.tex"
% End:


        %
% -----------------------------------------------------------------------------
%
% A license is hereby granted to reproduce this software source code and
% to create executable versions from this source code for personal,
% non-commercial use.  The copyright notice included with the software
% must be maintained in all copies produced.
%
% THIS PROGRAM IS PROVIDED "AS IS". THE AUTHOR PROVIDES NO WARRANTIES
% WHATSOEVER, EXPRESSED OR IMPLIED, INCLUDING WARRANTIES OF
% MERCHANTABILITY, TITLE, OR FITNESS FOR ANY PARTICULAR PURPOSE.  THE
% AUTHOR DOES NOT WARRANT THAT USE OF THIS PROGRAM DOES NOT INFRINGE THE
% INTELLECTUAL PROPERTY RIGHTS OF ANY THIRD PARTY IN ANY COUNTRY.
%
% Copyright (c) 1994-2006, John Conover, All Rights Reserved.
%
% Comments and/or bug reports should be addressed to:
%
%     john@email.johncon.com (John Conover)
%
% -----------------------------------------------------------------------------
%
% Revision: \RCSRevision \\
% Revision Time: \RCSTime UMT \\
% Revision Date: \RCSDate \\
% Revision Id: \RCSId \\
% Revision File: \RCSLog \\
\RCS $Revision: 0.0 $
\RCS $Date: 2006/01/20 04:38:13 $
\RCS $Id: verification.tex,v 0.0 2006/01/20 04:38:13 john Exp $
% $Log: verification.tex,v $
% Revision 0.0  2006/01/20 04:38:13  john
% Initial version
%
%
    \subsection{Qualitative Verification of Fixed Increment Approximation Analysis}
        \label{\SETLABEL:QVA}

        \subidx{\market}{verification of analysis}
        \subidx{verification}{analysis}
        \subidx{analysis}{verification}
        \subidx{quality}{of analysis}
        \subidx{verification}{of methodology}
        \subidx{methodology}{verification of}
        \subidx{Shannon}{probability}
        \subidx{probability}{Shannon}

        This section evaluates various values based on the ``average''
        of the normalized increments presented in
        Figure~\ref{\SETLABEL:TFA}. These values are an approximation
        to a, probably, complex process with a distribution shown in
        Figure~\ref{\SETLABEL:TF}. These values will be used in a
        fixed increment Brownian fractal analysis of the {\market},
        and may, or may not, provide adequate accuracy for
        projections.

        The data in this section is presented in tabular form in
        sections~\ref{\SETLABELREF:VI1} and~\ref{\SETLABELREF:VI2}.
        As a subjective evaluation of the ``quality'' of the analysis
        of the {\market}, from Chapter~\ref{methodology},
        Equation~\ref{metricvalues1}, and using the mean and root mean
        square values of the normalized increments of the time series
        data presented in Figure~\ref{\SETLABEL:TS} from
        Figure~\ref{\SETLABEL:TF}, and the Shannon probability as
        calculated by counting the total number of {\timescale}s that
        the {\market} movement was positive, as presented in
        Section~\ref{\SETLABEL:MAXSHANNON}:

        \begin{eqnarray}
                  P & \approx & \frac{\frac{avg}{rms} + 1}{2}\\
            {\pmax} & \approx & \frac{\frac{\datafractionmean}{\datafractionrms} + 1}{2}\\
            {\pmax} & \approx & {\avgrms}
            \label{\SETLABEL:AVGS}
        \end{eqnarray}

        \subidx{Shannon}{probability}
        \subidx{probability}{Shannon}
        \noindent and comparing these values to the Shannon
        probability, as found by the {\it tsshannonmax}\/ program, which
        iterates for a maximum:

        \begin{eqnarray}
            {\pmax} \approx {\avgrms} \approx {\shannonmax}
        \end{eqnarray}

        \subidx{logarithmic}{returns}
        \subidx{returns}{logarithmic}
        In addition, the different methods of calculating the
        logarithmic returns, presented in Section~\ref{\SETLABEL:FS},
        should be compared. The four methods used were the mean of
        Figure~\ref{\SETLABEL:TF}, the constant in the least squares
        approximation to Figure~\ref{\SETLABEL:TF}, the least squares
        exponential approximation to Figure~\ref{\SETLABEL:TS}, and
        the logarithmic returns of Figure~\ref{\SETLABEL:TS}, derived
        as the mean of the logarithms of the quotients of the
        increments. The values for each of the methods are,
        respectively:

        \begin{equation}
            \datafractionmeanbits \approx \datafractionconstantbits \approx \datatslsqepbits \approx \logreturns
        \end{equation}

        It is implied in Section~\ref{\SETLABEL:FS},
        Subsection~\ref{\SETLABEL:SP} and in
        Section~\ref{\SETLABEL:TSUNFAIRBROWNIAN} that, a Brownian
        motion with fixed increments fractal may ``model'' the
        {\market}. Using Equation~\ref{stddev9} from
        Chapter~\ref{general}, Section~\ref{abmfi}:

        \begin{eqnarray}
                                    rms \left(2P - 1\right) & \approx & \frac{\sigma \left(2P - 1\right)}{2 \sqrt{P\left(1 - P\right)}}\\
            \datafractionrms \left(2 \cdot \pmax - 1\right) & \approx & \frac{\datafractionstddev \left(2 \cdot \pmax - 1\right)}{2\sqrt{\pmax \left(1 - \pmax\right)}}\\
                       \datafractionrms \cdot \twopminusone & \approx & \datafractionstddev \cdot \twopx\\
                                                      \rmsp & \approx & \sigmap
        \end{eqnarray}

        \noindent and, equating to the mean:

        \begin{equation}
            \datafractionmean \approx \rmsp \approx \sigmap
        \end{equation}

        \subidx{Shannon}{probability}
        \subidx{probability}{Shannon}
        \noindent where, as in Equation~\ref{\SETLABEL:AVGS} using the
        mean, root mean square, and standard deviation values of the
        normalized increments of the time series data presented in
        Figure~\ref{\SETLABEL:TS} from Figure~\ref{\SETLABEL:TF}, and
        the Shannon probability as calculated by counting the total
        number of {\timescale}s that the {\market} movement was
        positive, as presented in Section~\ref{\SETLABEL:MAXSHANNON}.

        As a final qualitative comparison, the absolute value of the
        normalized increments should be the same as the root mean
        square value\footnote{The absolute value of the normalized
        increments, when averaged, is related to the root mean square
        of the increments by a constant. If the normalized increments
        are a fixed increment, the constant is unity. If the
        normalized increments have a Gaussian distribution, the
        constant is $\approx 0.8$ depending on the accuracy of of
        ``fit'' to a Gaussian distribution.}, where the absolute value
        is presented in Figure~\ref{\SETLABEL:TFA}, and the root mean
        square value is presented in Figure~\ref{\SETLABEL:TF}:

        \begin{equation}
            \datafractionabsmean \approx \datafractionrms
        \end{equation}

        Note, that if the {\market} could be ``modeled'' as a Brownian
        motion with fixed increments fractal, then the standard
        deviation of the absolute value of the normalized increments
        of the time series data presented in Figure~\ref{\SETLABEL:TS}
        from Figure~\ref{\SETLABEL:TF} should be zero. It is
        $\datafractionabsstddev$.

% Local Variables:
% TeX-parse-self: t
% TeX-auto-save: t
% TeX-master: "fractal.tex"
% End:


    \renewcommand{\market}{United States Leading Economic Indicators}
    \renewcommand{\directory}{../markets/us.indicators}
    \renewcommand{\datafractionmean}{0.008052}
\renewcommand{\datafractionmeanbits}{0.011570}
\renewcommand{\datafractionmeanq}{0.002684}
\renewcommand{\datafractionmeanbitsq}{0.003867}
\renewcommand{\datafractionstddev}{0.038579}
\renewcommand{\datafractionrms}{0.039311}
\renewcommand{\avgrms}{0.602414}
\renewcommand{\ncompanies}{5.210454}
\renewcommand{\pncompanies}{0.544866}
\renewcommand{\datafractionabsmean}{0.029745}
\renewcommand{\datafractionabsstddev}{0.025769}
\renewcommand{\datafractionconstant}{0.010041}
\renewcommand{\datafractionconstantbits}{0.014414}
\renewcommand{\datafractionconstantq}{0.003347}
\renewcommand{\datafractionconstantbitsq}{0.004821}
\renewcommand{\datafractionslope}{-0.000021}
\renewcommand{\datafractionabsconstant}{0.035145}
\renewcommand{\datafractionabsslope}{-0.000057}
\renewcommand{\hurstall}{0.659558}
\renewcommand{\hurstlow}{0.707509}
\renewcommand{\hurstlowtwo}{1.415018}
\renewcommand{\hurstlowhundred}{70.750900}
\renewcommand{\hcalcall}{0.184942}
\renewcommand{\hcalclow}{0.102042}
\renewcommand{\shannonmax}{0.604167}
\renewcommand{\twoponemax}{0.208334}
\renewcommand{\logreturns}{0.010456}
\renewcommand{\twologreturns}{1.007274}
\renewcommand{\twologreturnshundred}{0.727387}
\renewcommand{\oneoverlogreturns}{95.638868}
\renewcommand{\pmax}{0.602094}
\renewcommand{\twopminusone}{0.204188}
\renewcommand{\rmsp}{0.008027}
\renewcommand{\twopx}{0.208583}
\renewcommand{\sigmap}{0.008047}
\renewcommand{\tsunfairbrownianfractionmean}{0.007862}
\renewcommand{\tsunfairbrownianfractionstddev}{0.038619}
\renewcommand{\shannonlogreturns}{0.560125}
\renewcommand{\shannonlogreturnshundred}{56.012500}
\renewcommand{\twopone}{0.120250}
\renewcommand{\twoponehundred}{12.025000}
\renewcommand{\hundredtwoponehundred}{87.975000}
\renewcommand{\hundredshannonlogreturnshundred}{43.987500}
\renewcommand{\datatslsqepbits}{0.007623}
\renewcommand{\thurstall}{0.633980}
\renewcommand{\thurstlow}{0.710108}
\renewcommand{\thurstlowtwo}{1.420216}
\renewcommand{\thurstlowhundred}{71.010800}
\renewcommand{\thcalcall}{0.247886}
\renewcommand{\thcalclow}{0.171737}
\renewcommand{\chisquared}{2.862000}
\renewcommand{\critical}{42.557000}

    \renewcommand{\timescale}{month}
    \subidx{market}{\market}
    \idx{\market}

    \section{\market}

        \renewcommand{\SETLABEL}{\LABPRE:USINDICATORS}
        \renewcommand{\SETLABELQ}{\LABPRE:USINDICATORSQ}
        \label{\SETLABEL}
        \renewcommand{\SETLABELREF}{\LABPREREF:USINDICATORS}

        \idx{United States Department of Commerce}
        For the analysis, the data was in the directory
        {\directory}\footnote{Data from the United States Department
        of Commerce, 1980---1994, by {\timescale}s, as an index of
        1987 = 100.}.

        The data in this section is presented in tabular form in
        Section~\ref{\SETLABELREF}. Note that in this analysis, the
        rate of revenue returns means the increase or decrease in the
        {\market}. This is included for comparative
        purposes. Presumably, the {\market} represent something of
        value, or they could be used as a ``futures'' derivative, and
        thus, it would be considered that there is a rate of revenue
        returns.

        %
% -----------------------------------------------------------------------------
%
% A license is hereby granted to reproduce this software source code and
% to create executable versions from this source code for personal,
% non-commercial use.  The copyright notice included with the software
% must be maintained in all copies produced.
%
% THIS PROGRAM IS PROVIDED "AS IS". THE AUTHOR PROVIDES NO WARRANTIES
% WHATSOEVER, EXPRESSED OR IMPLIED, INCLUDING WARRANTIES OF
% MERCHANTABILITY, TITLE, OR FITNESS FOR ANY PARTICULAR PURPOSE.  THE
% AUTHOR DOES NOT WARRANT THAT USE OF THIS PROGRAM DOES NOT INFRINGE THE
% INTELLECTUAL PROPERTY RIGHTS OF ANY THIRD PARTY IN ANY COUNTRY.
%
% Copyright (c) 1994-2006, John Conover, All Rights Reserved.
%
% Comments and/or bug reports should be addressed to:
%
%     john@email.johncon.com (John Conover)
%
% -----------------------------------------------------------------------------
%
% Revision: \RCSRevision \\
% Revision Time: \RCSTime UMT \\
% Revision Date: \RCSDate \\
% Revision Id: \RCSId \\
% Revision File: \RCSLog \\
\RCS $Revision: 0.0 $
\RCS $Date: 2006/01/20 04:38:13 $
\RCS $Id: fraction.tex,v 0.0 2006/01/20 04:38:13 john Exp $
% $Log: fraction.tex,v $
% Revision 0.0  2006/01/20 04:38:13  john
% Initial version
%
%
    \subsection{Time Series Increments Analysis}
        \label{\SETLABEL:TSA}

        \subidx{\market}{Time series analysis}
        \subidx{time series}{increments}
        \subidx{time series}{analysis}
        \subidx{cumulative sum}{analysis}
        \subidx{analysis}{cumulative sum}
        \subidx{analysis}{random process}
        \subidx{random process}{analysis}
        \subidx{Gaussian}{increments}
        \subidx{increments}{Gaussian}
        \subidx{Brownian}{motion, fractional}
        \subidx{fractional}{Brownian motion}
        \subidx{fractal}{Brownian motion}
        The data in this section is presented in tabular form in
        Section~\ref{\SETLABELREF:TSA}.  Figure~\ref{\SETLABEL:TS} is
        a graph of the time series data for the {\market}.

        \subidx{increments}{normalized}
        \subidx{normalized}{increments}
        \subidx{programs}{tsfraction}
        \subidx{tsfraction}{program}
        Figure~\ref{\SETLABEL:TF} is a graph of the normalized
        increments of the time series data presented in
        Figure~\ref{\SETLABEL:TS}. The data presented was made by
        running the program {\it tsfraction}\/ on the time series
        data. The program {\it tsfraction}\/ is described briefly in
        Appendix~\ref{programs}, and subtracts the previous value from
        the next value, dividing this difference by the previous
        value, for each element in the time series data. The new time
        series contains the instantaneous change in the rate of
        revenue returns, divided by the magnitude of the instantaneous
        rate of revenue returns.

        \subidx{mean}{standard deviation}
        \subidx{standard deviation}{mean}
        \idx{root mean square}
        \idx{least squares approximation}
        \begin{figure}[ht]
            \begin{center}
                \begin{minipage}[t]{0.45\textwidth}
                    \epsfxsize=1.0\linewidth
                    \epsffile{\directory/data.eps}
                    \caption{{\market}, time series data.}
                    \label{\SETLABEL:TS}
                    \label{\SETLABELQ:TS}
                \end{minipage}
                \hfill
                \begin{minipage}[t]{0.45\textwidth}
                    \epsfxsize=1.0\linewidth
                    \epsffile{\directory/data.tsfraction.eps}
                    \caption[{\market}, normalized
                        increments]{{\market}, normalized increments
                        of the time series data presented in
                        Figure~\ref{\SETLABEL:TS}. The mean is
                        {\datafractionmean} with a standard deviation
                        of {\datafractionstddev}. The formula for the
                        least squares approximation is
                        ${\datafractionconstant} +
                        {\datafractionslope}t$, and the root mean
                        squared value is {\datafractionrms}. The
                        graph, labeled ``data\-.tsfraction\-.tsrms,''
                        is the running root mean square, and
                        ``data\-.tsfraction\-.tsavg'' is the running
                        average of the normalized increments.  This
                        graph is the fraction of change in the time
                        series, as a function of time. Note that the
                        slope of the mean, {\datafractionslope}, is
                        the coefficient of the nonlinearity term in
                        the normalized increments. See
                        Chapter~\ref{general}, Section~\ref{nlextend}
                        for a possible application of the logistic
                        function to this data set.}
                    \label{\SETLABEL:TF}
                    \label{\SETLABELQ:TF}
                \end{minipage}
            \end{center}
        \end{figure}

        \subidx{absolute value}{increments}
        \subidx{increments}{absolute value}

        Figure~\ref{\SETLABEL:TFA} is a graph of the absolute value of
        the normalized increments of the time series data presented in
        Figure~\ref{\SETLABEL:TF}. The data presented was made by
        running the Unix utility sed(1) on the normalized increments
        time series data to remove the negative signs. This is an
        absolute value procedure.  The resulting time series contains
        the absolute value of the instantaneous change in the rate of
        revenue returns, divided by the magnitude of the instantaneous
        rate of revenue returns\footnote{The absolute value of the
        normalized increments, when averaged, is related to the root
        mean square of the increments by a constant. If the normalized
        increments are a fixed increment, the constant is unity. If
        the normalized increments have a Gaussian distribution, the
        constant is $\approx 0.8$ depending on the accuracy of of
        ``fit'' to a Gaussian distribution.}.

        \subidx{histogram}{normalized}
        \subidx{normalized}{histogram}
        \subidx{programs}{tsnormal}
        \subidx{tsnormal}{program}
        \subidx{mean}{standard deviation}
        \subidx{standard deviation}{mean}
        \idx{root mean square}
        \idx{least squares approximation}
        \subidx{\market}{analysis of increments}
        Figure~\ref{\SETLABEL:NH} is the normalized histogram of the
        normalized increments of the time series data shown in
        Figure~\ref{\SETLABEL:TF}. The abscissa is 3 $\sigma$ limits,
        and the area under the two curves is identical. The data for
        this figure was produced by the program {\it tsnormal}\/,
        which is described briefly in Appendix~\ref{programs}.

        \begin{figure}[ht]
            \begin{center}
                \begin{minipage}[t]{0.45\textwidth}
                    \epsfxsize=1.0\linewidth
                    \epsffile{\directory/data.tsfraction.abs.eps}
                    \caption[{\market}, absolute value of the
                        normalized increments]{{\market}, absolute
                        value of the normalized increments of the time
                        series data presented in
                        Figure~\ref{\SETLABEL:TF}.  The mean is
                        {\datafractionabsmean} with a standard
                        deviation of {\datafractionabsstddev}. The
                        formula for the least squares approximation is
                        ${\datafractionabsconstant} +
                        {\datafractionabsslope}t$, and the root mean
                        square value, from Figure~\ref{\SETLABEL:TF},
                        is {\datafractionrms}.  The graph, labeled
                        ``data\-.tsfraction\-.tsrms,'' is the running
                        root mean square, and
                        ``data\-.tsfraction\-.tsavg'' is the running
                        average of the normalized increments presented
                        in Figure~\ref{\SETLABEL:TF}, superimposed
                        here for convenience. This graph is the
                        absolute value of the fraction of change in
                        the time series, as a function of time.}
                    \label{\SETLABEL:TFA}
                    \label{\SETLABELQ:TFA}
                \end{minipage}
                \hfill
                \begin{minipage}[t]{0.45\textwidth}
                    \epsfxsize=1.0\linewidth
                    \epsffile{\directory/data.tsfraction.tsnormal-s30.eps}
                    \caption[{\market}, normalized histogram of the
                        normalized increments]{{\market}, normalized
                        histogram of the normalized increments of the
                        time series data shown in
                        Figure~\ref{\SETLABEL:TF}.  The data has a
                        mean of {\datafractionmean}, with a standard
                        deviation of {\datafractionstddev}.  The area
                        under the two curves is identical. The
                        $\chi^2$ value of the observed and expected
                        values of the two curves is {\chisquared},
                        with a critical value of {\critical}.}
                    \label{\SETLABEL:NH}
                \end{minipage}
            \end{center}
        \end{figure}

        \subidx{programs}{tsXsquared}
        \subidx{tsXsquared}{program}
        \subidx{\market}{chi-squared values of increments}
        The program {\it tsXsquared}\/, which is briefly described in
        appendix~\ref{programs}, was used to derive the $\chi^2$
        statistics for the data presented in
        Figure~\ref{\SETLABEL:NH}.

        \subidx{programs}{tsstatest}
        \subidx{tsstatest}{program}
        \subidx{\market}{statistical estimates}

        Figure~\ref{\SETLABEL:SE} is the statistical estimate for the
        data presented in Figure~\ref{\SETLABEL:TF}, as derived by the
        program {\it tsstatest}\/, which is briefly described in
        appendix~\ref{programs}.

        \begin{figure}[ht]
            \begin{center}
                \begin{minipage}[t]{\textwidth}
                    \center{\fbox{\parbox{0.9\textwidth}{\XXX{\directory/data.tsstatest-f0.1-c0.9-i.tex}}}}
                    \caption[{\market}, statistical estimates of the
                        normalized increments]{{\market}, statistical
                        estimates of the normalized increments of the
                        time series shown in Figure~\ref{\SETLABEL:TF}.
                        The table was produced with the {\it
                        tsstatest}\/ program, and illustrates the
                        size of the data set required for a confidence
                        level of 90\%, with an error estimate of $\pm$
                        10\%, or alternately, the error estimate on
                        the time series shown in Figure~\ref{\SETLABEL:TF}.}
                    \label{\SETLABEL:SE}
                \end{minipage}
            \end{center}
        \end{figure}

        Note that the data set size estimations, as produced by the
        {\it tsstatest}\/ program, are probably very conservative,
        depending on the magnitude of the Shannon probability, $P =
        \shannonlogreturns$, as derived in
        Section~\ref{\SETLABEL:SP}. See Chapter~\ref{general},
        Section~\ref{serdss} for possible alternative methodologies
        for addressing the analysis of fractal time series with
        limited data set sizes. Depending on the magnitude of the
        Shannon probability, $P$, these estimates can be several
        orders of magnitude too high.

        \subidx{derivative of increments}{normalized}
        \subidx{normalized}{derivative of increments}
        \subidx{programs}{tsderivative}
        \subidx{tsderivative}{program}
        Figure~\ref{\SETLABEL:TF1} is the normalized histogram of the
        first derivative of the normalized increments of the time
        series data shown in Figure~\ref{\SETLABEL:TF}. In principle,
        if the distribution of the normalized increments presented in
        Figure~\ref{\SETLABEL:NH} is Gaussian in nature, this
        distribution would be similar to ``white noise,'' as presented
        in appendix~\ref{programs}, Figure~\ref{whiteexample}. The
        data was generated by the {\it tsderivative}\/ program, which
        is briefly described in
        appendix~\ref{programs}. Figure~\ref{\SETLABEL:TF2} is the
        normalized histogram of the second derivative of the
        normalized increments of the time series data shown in
        Figure~\ref{\SETLABEL:TF}. In principle, if the distribution
        of the normalized increments presented in
        Figure~\ref{\SETLABEL:NH} is an integrated Gaussian
        distribution in nature, this distribution would be similar to
        ``white noise,'' as presented in appendix~\ref{programs},
        Figure~\ref{whiteexample}.

        \begin{figure}[ht]
            \begin{center}
                \begin{minipage}[t]{0.45\textwidth}
                    \epsfxsize=1.0\linewidth
                    \epsffile{\directory/data.tsfraction.tsderivative.tsnormal-s30.eps}
                    \caption[{\market}, histogram of the first
                        derivative of the increments]{{\market},
                        normalized histogram of the first derivative
                        of the normalized increments of the time
                        series data shown in
                        Figure~\ref{\SETLABEL:TF}.}
                    \label{\SETLABEL:TF1}
                \end{minipage}
                \hfill
                \begin{minipage}[t]{0.45\textwidth}
                    \epsfxsize=1.0\linewidth
                    \epsffile{\directory/data.tsfraction.2tsderivative.tsnormal-s30.eps}
                    \caption[{\market}, histogram of the second
                        derivative of the increments]{{\market},
                        normalized histogram of second derivative of
                        the the normalized increments of the time
                        series data shown in
                        Figure~\ref{\SETLABEL:TF}.}
                    \label{\SETLABEL:TF2}
                \end{minipage}
            \end{center}
        \end{figure}

        \subidx{fractal}{range}
        \subidx{fractal}{R/S analysis}
        \subidx{\market}{rate of revenue returns, range}
        \subidx{\market}{deterministic mechanism}
        \subidx{deterministic}{mechanism}
        \subidx{mechanism}{deterministic}
        Figure~\ref{\SETLABEL:TR} is the range of values of the time
        series shown in Figure~\ref{\SETLABEL:TS}. The horizontal axis
        is time into the future. In principle, if the time series was
        characterized as fractional Brownian motion the graph in
        Figure~\ref{\SETLABEL:TR} would be a square root
        function\footnote{Note that the ``roughness,'' or ``sawtooth''
        characteristics of the graph in Figure~\ref{\SETLABEL:TR} are
        a computational artifact---caused by not using the -m option
        to the program {\it tshurst}\/, which is computationally
        inefficient.}. Figure~\ref{\SETLABEL:TD} is the deterministic
        map of the normalized increments of the time series data shown
        in Figure~\ref{\SETLABEL:TF}. The deterministic map is useful
        for determining if a time series was created by a
        deterministic mechanism. This, essentially, maps each element
        in the time series with the previous element in the time
        series.  See,~\cite[pp. 745]{Peitgen}.

        \begin{figure}[ht]
            \begin{center}
                \begin{minipage}[t]{0.45\textwidth}
                    \epsfxsize=1.0\linewidth
                    \epsffile{\directory/data.tshurst-f.eps}
                    \caption[{\market}, range]{{\market}, range of the
                        time series data shown in
                        Figure~\ref{\SETLABEL:TS}.}
                    \label{\SETLABEL:TR}
                \end{minipage}
                \hfill
                \begin{minipage}[t]{0.45\textwidth}
                    \epsfxsize=1.0\linewidth
                    \epsffile{\directory/data.tsfraction.tsdeterministic.eps}
                    \caption[{\market}, deterministic map]{{\market},
                        deterministic map of the normalized increments
                        of the time series data shown in
                        Figure~\ref{\SETLABEL:TF}.}
                    \label{\SETLABEL:TD}
                \end{minipage}
            \end{center}
        \end{figure}

% Local Variables:
% TeX-parse-self: t
% TeX-auto-save: t
% TeX-master: "fractal.tex"
% End:


        \subsubsection{Observations on the Time Series Increments Analysis}

            Figure~\ref{\SETLABEL:NH} would seem to indicate that the
            time series data for the {\market} represents a cumulative
            sum/integration of a random process that has a Gaussian
            distribution, (ie., satisfies the Gaussian increments
            property of fractional Brownian
            motion~\cite[pp. 250]{Crownover},) tending to justify the
            assumption that the time series data represents fractional
            Brownian motion.

        %
% -----------------------------------------------------------------------------
%
% A license is hereby granted to reproduce this software source code and
% to create executable versions from this source code for personal,
% non-commercial use.  The copyright notice included with the software
% must be maintained in all copies produced.
%
% THIS PROGRAM IS PROVIDED "AS IS". THE AUTHOR PROVIDES NO WARRANTIES
% WHATSOEVER, EXPRESSED OR IMPLIED, INCLUDING WARRANTIES OF
% MERCHANTABILITY, TITLE, OR FITNESS FOR ANY PARTICULAR PURPOSE.  THE
% AUTHOR DOES NOT WARRANT THAT USE OF THIS PROGRAM DOES NOT INFRINGE THE
% INTELLECTUAL PROPERTY RIGHTS OF ANY THIRD PARTY IN ANY COUNTRY.
%
% Copyright (c) 1994-2006, John Conover, All Rights Reserved.
%
% Comments and/or bug reports should be addressed to:
%
%     john@email.johncon.com (John Conover)
%
% -----------------------------------------------------------------------------
%
% Revision: \RCSRevision \\
% Revision Time: \RCSTime UMT \\
% Revision Date: \RCSDate \\
% Revision Id: \RCSId \\
% Revision File: \RCSLog \\
\RCS $Revision: 0.0 $
\RCS $Date: 2006/01/20 04:38:13 $
\RCS $Id: instant.tex,v 0.0 2006/01/20 04:38:13 john Exp $
% $Log: instant.tex,v $
% Revision 0.0  2006/01/20 04:38:13  john
% Initial version
%
%
    \subsection{Instantaneous Analysis of Normalized Increments}
        \label{\SETLABEL:IA}

        \subidx{\market}{instantaneous analysis of normalized increments}
        \idx{average of normalized increments}
        \idx{root mean square of normalized increments}
        \subidx{Shannon probability}{instantaneous computation of}
        \subidx{average of normalized increments}{instantaneous computation of}
        \subidx{root mean square of normalized increments}{instantaneous computation of}
        \subidx{instantaneous computation}{Shannon probability}
        \subidx{instantaneous computation}{average of normalized increments}
        \subidx{instantaneous computation}{root mean square of normalized increments}
        \idx{time series}
        \subidx{time series}{instantaneous analysis}
        \subidx{instantaneous analysis}{time series}
        \subidx{time series}{increments}
        \subidx{time series}{analysis}
        \subidx{Shannon}{probability}
        \subidx{probability}{Shannon}
        \subidx{normalized}{increments}
        \subidx{increments}{normalized}

        The program {\it tsinstant}\/, which is briefly described in
        Appendix~\ref{programs}, is for finding the instantaneous
        fraction of change in a time series. The value of a sample in
        the time series is subtracted from the previous sample in the
        time series, and divided by the value of the previous sample.
        As explained in Chapter~\ref{general},
        Sections~\ref{derivation},~\ref{GA},~\ref{abmfi},~\ref{aftsma}
        and,~\ref{ompl} for Brownian motion, random walk fractals, the
        absolute value of the instantaneous fraction of change is also
        the root mean square of the instantaneous fraction of
        change\footnote{The absolute value of the normalized
        increments, when averaged, is related to the root mean square
        of the increments by a constant. If the normalized increments
        are a fixed increment, the constant is unity. If the
        normalized increments have a Gaussian distribution, the
        constant is $\approx 0.8$ depending on the accuracy of of
        ``fit'' to a Gaussian distribution.}. Squaring this value is
        the average of the instantaneous fraction of change, and
        adding unity to the absolute value of the instantaneous
        fraction of change, and dividing by two, is the Shannon
        probability of the instantaneous fraction of change.

        Figure~\ref{\SETLABEL:IA1} is the instantaneous value of the
        root mean square of the normalized increments for the
        {\market}, and Figure~\ref{\SETLABEL:IA2} is the instantaneous
        Shannon probability for the normalized increments.

        \begin{figure}[ht]
            \begin{center}
                \begin{minipage}[t]{0.45\textwidth}
                    \epsfxsize=1.0\linewidth
                    \epsffile{\directory/data.tsinstant-r.eps}
                    \caption[{\market}, instantaneous value of
                        rms.]{{\market}, instantaneous value of the
                        root mean square of the normalized increments,
                        provided by running the program {\it
                        tsinstant}\/ with the -r option on the data
                        presented in Figure~\ref{\SETLABEL:TS}.}
                    \label{\SETLABEL:IA1}
                    \label{\SETLABELQ:IA1}
                \end{minipage}
                \hfill
                \begin{minipage}[t]{0.45\textwidth}
                    \epsfxsize=1.0\linewidth
                    \epsffile{\directory/data.tsinstant-s.eps}
                    \caption[{\market}, instantaneous value of
                        Shannon probability.]{{\market}, instantaneous
                        value of the Shannon probability of the
                        normalized increments, provided by running the
                        program {\it tsinstant}\/ with the -s option
                        on the data presented in
                        Figure~\ref{\SETLABEL:TS}.}
                    \label{\SETLABEL:IA2}
                    \label{\SETLABELQ:IA2}
                \end{minipage}
            \end{center}
        \end{figure}

% Local Variables:
% TeX-parse-self: t
% TeX-auto-save: t
% TeX-master: "fractal.tex"
% End:


        %
% -----------------------------------------------------------------------------
%
% A license is hereby granted to reproduce this software source code and
% to create executable versions from this source code for personal,
% non-commercial use.  The copyright notice included with the software
% must be maintained in all copies produced.
%
% THIS PROGRAM IS PROVIDED "AS IS". THE AUTHOR PROVIDES NO WARRANTIES
% WHATSOEVER, EXPRESSED OR IMPLIED, INCLUDING WARRANTIES OF
% MERCHANTABILITY, TITLE, OR FITNESS FOR ANY PARTICULAR PURPOSE.  THE
% AUTHOR DOES NOT WARRANT THAT USE OF THIS PROGRAM DOES NOT INFRINGE THE
% INTELLECTUAL PROPERTY RIGHTS OF ANY THIRD PARTY IN ANY COUNTRY.
%
% Copyright (c) 1994-2006, John Conover, All Rights Reserved.
%
% Comments and/or bug reports should be addressed to:
%
%     john@email.johncon.com (John Conover)
%
% -----------------------------------------------------------------------------
%
% Revision: \RCSRevision \\
% Revision Time: \RCSTime UMT \\
% Revision Date: \RCSDate \\
% Revision Id: \RCSId \\
% Revision File: \RCSLog \\
\RCS $Revision: 0.0 $
\RCS $Date: 2006/01/20 04:38:13 $
\RCS $Id: logistic.tex,v 0.0 2006/01/20 04:38:13 john Exp $
% $Log: logistic.tex,v $
% Revision 0.0  2006/01/20 04:38:13  john
% Initial version
%
%
    \subsection{Logistic Analysis}
        \label{\SETLABEL:LA}

        \subidx{\market}{Logistic function analysis}
        \subidx{time series}{logistic function}
        \subidx{logistic function}{time series}
        \subidx{time series}{increments}
        \subidx{time series}{analysis}
        \subidx{cumulative sum}{analysis}
        \subidx{analysis}{cumulative sum}
        \subidx{analysis}{random process}
        \subidx{random process}{analysis}
        The data in this section is presented in tabular form in
        Section~\ref{\SETLABELREF:LAA}.  Figure~\ref{\SETLABEL:LA1} is
        a graph of the logistic function estimates of the time series
        data for the {\market}. The reader is cautioned that these
        graphs are constructed using the method suggested in
        Chapter~\ref{general}, Section~\ref{nlextend} and enormous
        precision is required for adequate prediction of the logistic
        function,~\cite{Modis}. Particularly, the non-linear term will
        usually require intervention to produce a practical fit to the
        data. In addition, there are numerical stability issues with
        logistic function methodologies\footnote{For example, in
        Figures~\ref{\SETLABEL:LA1} and~\ref{\SETLABEL:LA2}, if the
        non-linear term, $b$, was greater than zero, it was set to
        zero to produce the graphs. See Section~\ref{\SETLABELREF:LAA}
        for the actual derived values. In other cases, the magnitude
        of $b$ was too large, resulting in a graph that was decreasing
        as a function of time}.  The methodology should be regarded as
        ``fragile.'' It is included for completeness.

        \idx{least squares approximation}
        Figure~\ref{\SETLABEL:LA1} is a graph of the logistic function
        for the time series data presented in
        Figure~\ref{\SETLABEL:TS}. The data presented was made by
        running the program {\it tsdlogistic}\/, which is described
        briefly in Appendix~\ref{programs}, on the parameters
        extracted from the time series data as suggested in
        Figure~\ref{\SETLABEL:TF}. The program {\it tslsq}\/ was used
        to derive the constant and the slope of the normalized
        increments of the data presented in Figure~\ref{\SETLABEL:TF}.
        Figure~\ref{\SETLABEL:LA2} is the same graph, but with the
        time scale expanded by a factor of two.

        \begin{figure}[ht]
            \begin{center}
                \begin{minipage}[t]{0.45\textwidth}
                    \epsfxsize=1.0\linewidth
                    \epsffile{\directory/data.tsfraction.tslsq-p.tsdlogistic.eps}
                    \caption[{\market}, logistic function
                        estimates.]{{\market}, logistic function
                        estimates, provided by running the {\it
                        tslsq}\/ program on the normalized increments
                        presented in Figure~\ref{\SETLABEL:TF} with
                        the -p option. These parameters were used as
                        arguments to the {\it tsdlogistic}\/ program.}
                    \label{\SETLABEL:LA1}
                    \label{\SETLABELQ:LA1}
                \end{minipage}
                \hfill
                \begin{minipage}[t]{0.45\textwidth}
                    \epsfxsize=1.0\linewidth
                    \epsffile{\directory/data.tsfraction.tslsq-p.tsdlogistic2.eps}
                    \caption[{\market}, logistic function
                        estimates.]{{\market}, logistic function
                        estimates of Figure~\ref{\SETLABEL:LA1} with
                        the time scale expanded by a factor of two.}
                    \label{\SETLABEL:LA2}
                    \label{\SETLABELQ:LA2}
                \end{minipage}
            \end{center}
        \end{figure}

% Local Variables:
% TeX-parse-self: t
% TeX-auto-save: t
% TeX-master: "fractal.tex"
% End:


        %
% -----------------------------------------------------------------------------
%
% A license is hereby granted to reproduce this software source code and
% to create executable versions from this source code for personal,
% non-commercial use.  The copyright notice included with the software
% must be maintained in all copies produced.
%
% THIS PROGRAM IS PROVIDED "AS IS". THE AUTHOR PROVIDES NO WARRANTIES
% WHATSOEVER, EXPRESSED OR IMPLIED, INCLUDING WARRANTIES OF
% MERCHANTABILITY, TITLE, OR FITNESS FOR ANY PARTICULAR PURPOSE.  THE
% AUTHOR DOES NOT WARRANT THAT USE OF THIS PROGRAM DOES NOT INFRINGE THE
% INTELLECTUAL PROPERTY RIGHTS OF ANY THIRD PARTY IN ANY COUNTRY.
%
% Copyright (c) 1994-2006, John Conover, All Rights Reserved.
%
% Comments and/or bug reports should be addressed to:
%
%     john@email.johncon.com (John Conover)
%
% -----------------------------------------------------------------------------
%
% Revision: \RCSRevision \\
% Revision Time: \RCSTime UMT \\
% Revision Date: \RCSDate \\
% Revision Id: \RCSId \\
% Revision File: \RCSLog \\
\RCS $Revision: 0.0 $
\RCS $Date: 2006/01/20 04:38:13 $
\RCS $Id: hurst.tex,v 0.0 2006/01/20 04:38:13 john Exp $
% $Log: hurst.tex,v $
% Revision 0.0  2006/01/20 04:38:13  john
% Initial version
%
%
    \subsection{Hurst Coefficient Analysis}
        \label{\SETLABEL:H}

        \subidx{\market}{Hurst coefficient analysis}
        \subidx{Hurst coefficient}{analysis}
        \subidx{increments}{normalized}
        \subidx{normalized}{increments}
        \subidx{programs}{tshurst}
        \subidx{tshurst}{program}
        The data in this section is presented in tabular form in
        Section~\ref{\SETLABELREF:HCHP}. Figure~\ref{\SETLABEL:HC} is
        a graph of the Hurst coefficient data time series data shown
        in Figure~\ref{\SETLABEL:TS}. The slope of the graph is the
        Hurst coefficient.  The data for this figure was produced by
        the program {\it tshurst}\/, which is described briefly in
        Appendix~\ref{programs}.

        \subidx{\market}{H parameter analysis}
        \subidx{H parameter}{analysis}
        \subidx{programs}{tshcalc}
        \subidx{tshcalc}{program}
        Figure~\ref{\SETLABEL:HP} is a graph of the H parameter data
        for the normalized increments of the time series data shown in
        Figure~\ref{\SETLABEL:TF}. The data for this figure was
        produced by the program {\it tshcalc}\/, which is described
        briefly in Appendix~\ref{programs}.

        \begin{figure}[ht]
            \begin{center}
                \begin{minipage}[t]{0.45\textwidth}
                    \epsfxsize=1.0\linewidth
                    \epsffile{\directory/data.tshurst.eps}
                    \caption[{\market}, Hurst coefficient data]{{\market},
                        Hurst coefficient data for the normalized
                        increments of the time series data shown in
                        Figure~\ref{\SETLABEL:TF}.  The slope of the graph
                        is the Hurst coefficient.}
                    \label{\SETLABEL:HC}
                \end{minipage}
                \hfill
                \begin{minipage}[t]{0.45\textwidth}
                    \epsfxsize=1.0\linewidth
                    \epsffile{\directory/data.tshcalc.eps}
                    \caption[{\market}, H parameter data]{{\market}, H
                        parameter data for the normalized increments of
                        the time series data shown in
                        Figure~\ref{\SETLABEL:TF} The slope of the graph
                        is the H parameter.}
                    \label{\SETLABEL:HP}
                \end{minipage}
            \end{center}
        \end{figure}

        \subidx{revenue}{See, rate of revenue returns}
        \subidx{returns}{See, rate of revenue returns}
        \subidx{\market}{revenues}
        \subidx{Hurst coefficient}{analysis}
        \subidx{\market}{Hurst coefficient analysis}
        \subidx{\market}{rate of change}
        \subidx{\market}{windows of opportunity}
        \subidx{rate of revenue returns}{forecast}
        \subidx{forecast}{rate of revenue returns}
        \idx{windows of opportunity}
        \subidx{programs}{tslsq}
        \subidx{tslsq}{program}

        The approximately linear slope of the graph in
        Figure~\ref{\SETLABEL:HC} implies that the variance of the
        rate of revenue returns, (per {\timescale},) in the {\market},
        $V(t_2 - t_1)$, over a period of time is proportional to the
        period of time raised to twice the Hurst
        coefficient~\cite[pp. 180]{Feder},~\cite[pp. 246]{Crownover}.
        This seems to be a quantitative statement concerning how fast,
        and to what degree, the rate of revenue returns' state of
        affairs can change over a period of time.  An additional
        implication, for Hurst coefficients sufficiently close to 0.5,
        is that the probability of the state of affairs repeating
        sometime in the future goes down with increasing
        time\footnote{It can be shown that the number of expected
        market ``high'' and ``low'' transitions, $N$, scales with the
        square root of time, or $N \propto \sqrt {t}$, meaning that
        the cumulative distribution of the probability, $P$, of the
        duration of a market's ``high'' or ``low'' exceeding a given
        time interval, $t$, is proportional to the reciprocal of the
        square root of the time interval, $P \propto 1 / \sqrt {t}$,
        (or, conversely, that the probability of the duration of a
        market's ``high'' or ``low'' exceeding a given time interval
        is proportional to the reciprocal of the time interval raised
        to the power $3 / 2$, ie., $P \propto 1 / t^{3 /
        2}$,~\cite[pp. 153]{Schroeder}. What this means is that a
        histogram of the ``zero free'' run-lengths of a market being
        ``high'' or ``low,'' over a long time, would have a $1 / t^{3
        / 2}$ characteristic.)}, $t$, $p(t) = erf (1/\sqrt{2t})$ which
        is approximately $1/\sqrt{t}$ for $t \gg
        1$~\cite[pp. 160]{Schroeder}. Figures~\ref{\SETLABEL:FN},
        and,~\ref{\SETLABEL:FF} compare methods of approximation of
        the ``forecastability'' of the rate of revenue returns in the
        {\market} for the near term and far term,
        respectively~\cite[pp. 83-84]{Peters:CAOITCM}\footnote{The
        author is not comfortable with Peters' interpretation. For
        example, if the algorithm explained
        in~\cite[pp. 82]{Peters:CAOITCM} is used on ``white noise''
        which, by definition, never has any correlations, the short
        term Hurst coefficient, and thus the ``forecastability,'' is
        still near unity---a bit of an enigma. This can be verified
        with the {\it tswhite}\/ and {\it tshurst}\/ programs, which
        are briefly described in Appendix~\ref{programs}.}.  This
        seems to be a quantitative statement concerning ``windows of
        opportunity'' in the rate of revenue returns, (per
        {\timescale}.)  The program {\it tslsq}\/ was used on the
        Hurst coefficient data, presented in
        Figure~\ref{\SETLABEL:HC}, to provide a least squares
        approximation to the Hurst coefficient. The superimposed least
        squares approximation with on original Hurst coefficient data
        is presented.  The time series data has a Hurst coefficient of
        {\thurstlow}, so that:

        \subidx{\market}{Hurst coefficient analysis}
        \begin{eqnarray}
            V\left(t_2 - t_1\right) & \propto & \left(t_2 - t_1\right)^{2 \cdot H}\\
            V\left(t_2 - t_1\right) & \propto & \left(t_2 - t_1\right)^{2 \cdot {\thurstlow}}\\
                                    & \propto & \left(t_2 - t_1\right)^{\thurstlowtwo}
            \label{\SETLABEL:V}
        \end{eqnarray}

        \subidx{fractional}{Brownian motion}
        \subidx{Brownian motion}{fractional}
        \idx{fractal}
        \noindent where $V(t_2 - t_1)$ is the variance of the
        increments of the rate of revenue returns, (per {\timescale},)
        over the time interval $t_2 -
        t_1$,~\cite[pp. 177]{Feder},~\cite[pp. 494]{Peitgen}. If $H >
        \frac{1}{2}$, then the time series is termed as being
        characterized by ``fractional Brownian
        motion~\cite[pp. 170]{Feder}.''

        \subidx{rate of revenue returns}{predictability}
        \subidx{rate of revenue returns}{forecastability}
        \subidx{rate of revenue returns}{consistency}
        \subidx{predictability}{rate of revenue returns}
        \subidx{forecastability}{rate of revenue returns}
        \subidx{consistency}{rate of revenue returns}
        \subidx{\market}{rate of revenue returns, predictability}
        \subidx{\market}{rate of revenue returns, forecastability}
        \subidx{\market}{rate of revenue returns, consistency}
        \subidx{Hurst coefficient}{analysis}
        \subidx{\market}{Hurst coefficient analysis}
        \subidx{\market}{rate of change}

        In some sense, the Hurst coefficient is a quantitative
        expression of the ``forecastability'' of the future based on
        the past\footnote{Actually, in general, when summing fractal
        entities, the method used should be a root mean square
        process, dependent on the Hurst Coefficient, $H$, where
        $P_{total}^H = P_1^H + P_2^H + \cdots$, where $P_n$ is the
        fractal entities. For a Brownian motion, or random walk type
        of fractal the Hurst Coefficient is a function of time into
        the future. For the ``near term,'' the Hurst coefficient is
        very near unity, meaning the summation process is linear. For
        the ``long term,'' $H \approx 0.5$, or a standard root mean
        square summation process should be used. If $H$ is $0.5$ then
        the market is termed a Brownian motion, or random walk
        process. If it is larger than 0.5, it is termed fractional
        Brownian motion process. For a random walk process, ``near
        term'' and ``far term'' are quantitatively differentiated on
        the Hurst Coefficient graph where $1 - \ln (t) = 0.5 \cdot \ln
        (t)$, or when $\ln (t) = 2$, or $t = 7.389\ldots$ See
        Section~\ref{\SETLABEL:FS} for the particulars on using Hurst
        Coefficient to sum fractal process' for the {\market}. See
        also~\cite[pp. 67, 83-84]{Peters:CAOITCM} and~\cite[pp. 129,
        159]{Schroeder} for particulars on the implications of the
        Hurst Coefficient and root mean square summation issues.}.  A
        Hurst coefficient of {\thurstlow}, (for the near future, and
        {\thurstall} for the distant future.) implies that the
        likelihood of the rate of revenue returns, (per {\timescale},)
        for any two consecutive {\timescale}s being the same is
        {\thurstlowhundred}\%~\cite[pp. 66]{Peters:CAOITCM} for the
        near future, and {\thurstall} for the distant
        future. Likewise, there is a {\thurstlowhundred}\% chance of
        the rate of revenue returns, (per {\timescale},) movements
        being the same in consecutive time periods---ie., if, in a
        given {\timescale}, the rate of revenue returns, (per
        {\timescale},) is increasing, there is a {\thurstlowhundred}\%
        that the rate of revenue returns, (per {\timescale},) will
        increase in the following period, also. In some sense, this is
        a quantitative statement on how ``predictable,'' or
        ``forecastable'' the rate of revenue returns, (per
        {\timescale},) for the {\market} are over time, since the
        probability of having $n$ many consecutive {\timescale}s of
        the same agenda is $H^n$ where $H$ is the Hurst coefficient,
        or, letting the short term probability of having $n$ many
        {\timescale}s of the same market agenda, $p_a$, is:

        \begin{eqnarray}
            p_a\left(n\right) & = & H^{n}\\
                              & = & {\thurstlow}^{n}
            \label{\SETLABEL:MA}
        \end{eqnarray}

        \subidx{rate of revenue returns}{predictability}
        \subidx{rate of revenue returns}{forecastability}
        \subidx{rate of revenue returns}{consistency}
        \subidx{predictability}{rate of revenue returns}
        \subidx{forecastability}{rate of revenue returns}
        \subidx{consistency}{rate of revenue returns}
        As an interesting interpretation of the normalized increments
        of the time series data presented in
        Figure~\ref{\SETLABEL:TF}, if the vertical axis is multiplied
        by 100, to convert to percent, then the graph represents the
        error, in percent, that would be made by forecasting, month by
        month, that the next {\timescale}'s rate of revenue returns
        would be the same as the current {\timescale}'s revenue
        rate. Interestingly, it is $\datafractionmean \cdot 100$
        percent, on the average, with a standard deviation of
        $\datafractionstddev \cdot 100$ percent, and a root mean
        square error value of $\datafractionrms \cdot 100$
        percent---small values for such a simple forecasting
        mechanism.

        \subidx{\market}{rate of revenue returns, range}
        \subidx{Hurst coefficient}{analysis}
        \subidx{\market}{Hurst coefficient analysis}
        \subidx{\market}{rate of change}

        This is, essentially, a statement of the range of values, in
        the increments of the rate of revenue returns, (per
        {\timescale},) that is to be expected over the time interval,
        $t_2 - t_1$,
        $R_v$,~\cite[pp. 178]{Feder},~\cite[pp. 172]{Cambel}:

        \begin{eqnarray}
            R_v\left(t_2 - t_1\right) & \propto & \left(t_2 - t_1\right)^{H}\\
                                      & \propto & \left(t_2 - t_1\right)^{\thurstlow}
            \label{\SETLABEL:R}
        \end{eqnarray}

        \subidx{\market}{rate of revenue returns, range}
        \subidx{Hurst coefficient}{analysis}
        \subidx{\market}{Hurst coefficient analysis}
        \subidx{\market}{rate of change}
        \subidx{Markov}{statistics}
        \subidx{statistics}{Markov}
        \noindent where $R$ is the range of values in the increments
        of the rate of revenue returns, (per {\timescale}.) A Hurst
        coefficient, $H$, that is much larger than $\frac{1}{2}$, (but
        less than 1,) implies a strongly non-Gaussian distribution in
        the increments of the rate of revenue returns, (per
        {\timescale},)~\cite[pp. 152, 194]{Feder}, and a Hurst
        coefficient near $\frac{1}{2}$ implies that the increments of
        the rate of revenue returns, (per {\timescale}) is
        characteristic of an independent
        process~\cite[pp. 195]{Feder}. Extreme caution should be
        exercised in using Markov statistics in any analysis where the
        Hurst coefficient is not
        $\frac{1}{2}$,~\cite[pp. 124]{Crownover},~\cite[pp. 106]{Peters:CAOITCM}.


        As a useful approximation, if $H$, is approximately
        $\frac{1}{2}$, Equation~\ref{\SETLABEL:R} reduces
        to,~\cite[pp. 129]{Schroeder}:

        \begin{eqnarray}
            R\left(t_2 - t_1\right) & \propto & (t_2 - t_1)^{\frac{1}{2}}\\
                                    & \propto & \sqrt{\left(t_2 - t_1\right)}
        \end{eqnarray}

        \subidx{\market}{rate of revenue returns, range}
        \subidx{\market}{rate of revenue returns, increase and decrease}
        \subidx{Hurst coefficient}{analysis}
        \subidx{\market}{Hurst coefficient analysis}
        \subidx{\market}{rate of change}
        \subidx{Markov}{statistics}
        \subidx{statistics}{Markov}

        In the case where the Hurst coefficient, $H$, is
        $\frac{1}{2}$, the range of values in the increments of the
        rate of revenue returns, (per {\timescale},) divided by the
        standard deviation of these values, $S$, can be anticipated to
        increase over time according to the following
        relation,~\cite[pp. 154]{Feder},~\cite[pp. 129]{Schroeder}:

        \begin{equation}
            \frac{R\left(t_2 - t_1\right)}{S} \propto \left(t_2 - t_1\right)^{\frac{1}{2}}
        \end{equation}

        \subidx{\market}{rate of revenue returns, range}
        \subidx{\market}{rate of revenue returns, increase and decrease}
        \subidx{Hurst coefficient}{analysis}
        \subidx{\market}{Hurst coefficient analysis}
        \subidx{\market}{rate of change}
        \noindent which is a useful conceptual approximation, since it
        involves only the square root function---if the range and the
        standard deviation of the increments of the rate of revenue
        returns, (per {\timescale},) are known, (and $H \approx
        \frac{1}{2}$,) then the expected change in $\frac{R}{S}$, will
        increase with the square root of time\footnote{To be precise,
        it is actually asymptotically proportional to
        $\tau^{\frac{1}{2}}$}.

        Another useful approximation when rescaling processes that are
        characterize by Brownian motion, (ie., when $H \approx
        \frac{1}{2}$,) is that:

        \begin{eqnarray}
            X\left(t\right) & \propto & \frac{X\left(rt\right)}{r^{H}}\\
                            & \propto & \frac{X\left(rt\right)}{r^{\thurstlow}}
        \end{eqnarray}

        \idx{Brownian motion}
        \idx{fractal}
        Where $X(t)$ is the process characterized by Brownian motion,
        and $r$ is a scaling factor,~\cite[pp. 494]{Peitgen}.

        \subidx{programs}{tslsq}
        \subidx{tslsq}{program}
        The program {\it tslsq}\/ was used on the H parameter data,
        presented in Figure~\ref{\SETLABEL:HP}, to provide a least
        squares approximation to the H parameter for the
        {\market}. The superimposed least squares approximation on the
        original H parameter data is presented.  By contrast, the H
        parameter, as derived by the methodology outlined
        in~\cite[pp. 249]{Crownover}, is {\thcalclow} for the near
        future, and {\thcalcall} for the distant future.

        \subidx{\market}{Hurst coefficient analysis}
        \subidx{Hurst coefficient}{analysis}
        \subidx{increments}{normalized}
        \subidx{normalized}{increments}
        \subidx{programs}{tshurst}
        \subidx{tshurst}{program}
        \subidx{\market}{H parameter analysis}
        \subidx{H parameter}{analysis}
        \subidx{programs}{tshcalc}
        \subidx{tshcalc}{program}
        Figures~\ref{\SETLABEL:HC} and~\ref{\SETLABEL:HP} represent
        Hurst coefficient and H parameter data that are derived from
        the normalized increments, shown in
        Figure~\ref{\SETLABEL:TF}. In this case, the data is
        considered a normalized derivative of the time series data
        presented in Figure~\ref{\SETLABEL:TF}, instead of a
        cumulative sum.  The program, {\it tshurst}\/, is described
        briefly in appendix~\ref{programs}, and the data for
        figures~\ref{\SETLABEL:THC} and~\ref{\SETLABEL:THP} was made
        using the -d option.

        \begin{figure}[ht]
            \begin{center}
                \begin{minipage}[t]{0.45\textwidth}
                    \epsfxsize=1.0\linewidth
                    \epsffile{\directory/data.tsfraction.tshurst-d.eps}
                    \caption[{\market}, traditional Hurst coefficient
                        data]{{\market}, traditional Hurst coefficient
                        data for the time series data shown in
                        Figure~\ref{\SETLABEL:TS}.  The slope of the
                        graph is the Hurst coefficient, and is
                        {\hurstlow} for the near term, and
                        {\hurstall} for the far term.}
                    \label{\SETLABEL:THC}
                \end{minipage}
                \hfill
                \begin{minipage}[t]{0.45\textwidth}
                    \epsfxsize=1.0\linewidth
                    \epsffile{\directory/data.tsfraction.tshcalc-d.eps}
                    \caption[{\market}, traditional H parameter
                        data]{{\market}, traditional H parameter data
                        for the time series data shown in
                        Figure~\ref{\SETLABEL:TS} The slope of the
                        graph is the H parameter, and is {\hcalclow}
                        for the near term, and {\hcalcall} for the
                        far term.}
                    \label{\SETLABEL:THP}
                \end{minipage}
            \end{center}
        \end{figure}

% Local Variables:
% TeX-parse-self: t
% TeX-auto-save: t
% TeX-master: "fractal.tex"
% End:


        %
% -----------------------------------------------------------------------------
%
% A license is hereby granted to reproduce this software source code and
% to create executable versions from this source code for personal,
% non-commercial use.  The copyright notice included with the software
% must be maintained in all copies produced.
%
% THIS PROGRAM IS PROVIDED "AS IS". THE AUTHOR PROVIDES NO WARRANTIES
% WHATSOEVER, EXPRESSED OR IMPLIED, INCLUDING WARRANTIES OF
% MERCHANTABILITY, TITLE, OR FITNESS FOR ANY PARTICULAR PURPOSE.  THE
% AUTHOR DOES NOT WARRANT THAT USE OF THIS PROGRAM DOES NOT INFRINGE THE
% INTELLECTUAL PROPERTY RIGHTS OF ANY THIRD PARTY IN ANY COUNTRY.
%
% Copyright (c) 1994-2006, John Conover, All Rights Reserved.
%
% Comments and/or bug reports should be addressed to:
%
%     john@email.johncon.com (John Conover)
%
% -----------------------------------------------------------------------------
%
% Revision: \RCSRevision \\
% Revision Time: \RCSTime UMT \\
% Revision Date: \RCSDate \\
% Revision Id: \RCSId \\
% Revision File: \RCSLog \\
\RCS $Revision: 0.0 $
\RCS $Date: 2006/01/20 04:38:13 $
\RCS $Id: fiscal.tex,v 0.0 2006/01/20 04:38:13 john Exp $
% $Log: fiscal.tex,v $
% Revision 0.0  2006/01/20 04:38:13  john
% Initial version
%
%
    \subsection{Fixed Increment Approximation for Fiscal Strategy}
        \label{\SETLABEL:FS}

        \subidx{\market}{fiscal strategy}
        \subidx{markets}{analysis}
        \subidx{analysis}{markets}
        \subidx{strategy}{fiscal}
        \subidx{fiscal}{strategy}
        The data in this section is presented in tabular form in
        Section~\ref{\SETLABELREF:LR}. This section derives various
        values based on the ``average'' of the normalized increments
        presented in Figure~\ref{\SETLABEL:TFA}. These values are an
        approximation to a, probably, complex process with a
        distribution shown in Figure~\ref{\SETLABEL:TF}. These values
        will be used in a fixed increment Brownian fractal analysis
        and simulation of the {\market}, and may, or may not, provide
        adequate accuracy for projections.

        For an organization operating in the {\market}, the fiscal
        strategy, commensurate with the aggregate environment, can be
        derived as follows~\cite[pp. 128, pp
        151]{Schroeder},~\cite[pp. 450]{Reza},~\cite[pp. 270]{Pierce}:
        \vspace{0.15in}

        \subsubsection{Logarithmic Returns}
            \label{\SETLABEL:LR}

            \subidx{logarithmic}{returns}
            \subidx{returns}{logarithmic}
            \subidx{\market}{logarithmic returns}
            The logarithmic returns can be calculated by various
            means. Four will be presented here, for comparison.

            \subidx{programs}{tsnormal}
            \subidx{tsnormal}{program}
            \subidx{logarithmic}{returns}
            \subidx{returns}{logarithmic}
            The logarithmic returns, in bits, $bits$, as computed from
            the mean, by the program {\it tsnormal}\/, which is
            described in Chapter~\ref{programs}, and is presented in
            Figure~\ref{\SETLABEL:TF}, and Equation~\ref{abits} from
            Section~\ref{ereturns} in Chapter~\ref{general}:

            \begin{equation}
                bits = \frac{\ln \left({\datafractionmean} + 1\right)}{\ln \left(2\right)} = \datafractionmeanbits
            \end{equation}

            \subidx{programs}{tslsq}
            \subidx{tslsq}{program}
            \subidx{logarithmic}{returns}
            \subidx{returns}{logarithmic}
            \noindent By comparison, the logarithmic returns, in bits,
            $bits$, as computed from the constant in the least squares
            approximation, using the program {\it tslsq}\/, which is briefly
            described in Chapter~\ref{programs}, as presented in
            Figure~\ref{\SETLABEL:TF}, and Equation~\ref{abits} from
            Section~\ref{ereturns} in Chapter~\ref{general}:

            \begin{equation}
                bits = \frac{\ln \left({\datafractionconstant} + 1\right)}{\ln \left(2\right)} = \datafractionconstantbits
            \end{equation}

            Note that if the mean is not constant in
            Figure~\ref{\SETLABEL:TF}, this method will not provide
            accurate results.

            \subidx{programs}{tslsq}
            \subidx{tslsq}{program}
            \subidx{logarithmic}{returns}
            \subidx{returns}{logarithmic}
            \noindent And by yet another comparison, using the program
            {\it tslsq}\/, which is briefly described in
            Chapter~\ref{programs}, with the -e -p options, to provide
            a formula for the least squares exponential fit to the
            time series data set presented in
            Figure~\ref{\SETLABEL:TS}:

            \begin{equation}
                bits = {\datatslsqepbits}
            \end{equation}

            \subidx{programs}{tslogreturns}
            \subidx{tslogreturns}{program}
            \subidx{logarithmic}{returns}
            \subidx{returns}{logarithmic}
            \noindent And finally, by comparison, from the
            {\it tslogreturns}\/ program, which is briefly described
            in Chapter~\ref{programs}, with the -p option, to provide
            a formula for the logarithmic returns of the time series
            data set presented in Figure~\ref{\SETLABEL:TS}:

            \begin{equation}
                bits = {\logreturns}
            \end{equation}

        \subsubsection{Calculation of Shannon Probability}
            \label{\SETLABEL:SP}

            \subidx{\market}{Shannon probability}
            Ideally, all of the values presented in
            Section~\ref{\SETLABEL:LR} would be equal. Using the
            logarithmic returns provided by the {\it tslogreturns}\/
            program, to be consistent
            with~\cite[pp. 81]{Peters:CAOITCM}

            \subidx{programs}{tslogreturns}
            \subidx{tslogreturns}{program}
            \begin{equation}
                2^{{\logreturns}t}
            \end{equation}

            \noindent therefore:
            \begin{equation}
                C\left(p\right) = {\logreturns}
            \end{equation}
            \subidx{programs}{tsshannon}
            \subidx{tsshannon}{program}
            \subidx{Shannon}{probability}
            \subidx{probability}{Shannon}
            \noindent and, {\it tsshannon}\/ {\logreturns} gives:
            \begin{equation}
                \label{\SETLABEL:F0}
                C\left({\shannonlogreturns}\right) = {\logreturns}
            \end{equation}
            \noindent therefore:
            \begin{eqnarray}
                2^{C\left({\shannonlogreturns}\right)} & = & 2^{\logreturns}\\
                                                       & = & {\twologreturns}\\
                                                       & = & {\twologreturnshundred}\%
            \end{eqnarray}
            \noindent and:
            \begin{eqnarray}
                2p - 1 & = & \left(2 \cdot {\shannonlogreturns}\right) - 1\\
                       & = & {\twopone}\\
                       \label{\SETLABEL:F1}
                       & = & {\twoponehundred}\%
            \end{eqnarray}

            \subidx{\market}{fiscal strategy}
            \subidx{markets}{analysis}
            \subidx{analysis}{markets}
            \subidx{strategy}{fiscal}
            \subidx{fiscal}{strategy}
            \subidx{\market}{fiscal strategy}
            \subidx{\market}{growth rate}
            Presuming the simplified assumptions outlined in
            Section~\ref{assumptions}, the ``typical'' organization
            operating in the {\market} executes a long term fiscal
            strategy, commensurate with the aggregate environment,
            that is to invest, every {\timescale}, in sufficient
            additional resources and infrastructure, to increase the
            manufacturing of goods and services by {\twoponehundred}\%
            of its rate of revenue returns, (per {\timescale}.) As a
            conceptual model, the remaining {\hundredtwoponehundred}\%
            will be held in ``reserve'' with a
            {\shannonlogreturnshundred}\% chance of making twice the
            {\twoponehundred}\% back, (and a
            {\hundredshannonlogreturnshundred}\% chance of making
            0.0,) in one {\timescale}, on the average, for an average
            growth in its rate of revenue returns, (per {\timescale},)
            of {\twologreturnshundred}\%, or a doubling of its rate of
            revenue returns, (per {\timescale},) in
            {\oneoverlogreturns} {\timescale}s.

        \subsubsection{Example Fixed Increment Approximation Fiscal Strategies}

            \subidx{\market}{fiscal strategy}
            \subidx{markets}{analysis}
            \subidx{analysis}{markets}
            \subidx{strategy}{fiscal}
            \subidx{fiscal}{strategy}
            \subidx{\market}{fiscal strategy}
            \subidx{\market}{growth rate}
            \subidx{\market}{management metric}
            \idx{management metric}
            A possible metric on the effectiveness of long term fiscal
            management could possibly be that if an investment of
            {\twoponehundred}\% per {\timescale} of the rate of
            revenue returns, (per {\timescale},) is made in resources
            and infrastructure, then the rate of revenue returns would
            be expected to increase by {\twologreturnshundred}\%, per
            {\timescale}, on average.

            Note that the metrics presented in this section are
            representative of the {\market} as an aggregate whole, and
            may or may not be accurate representations for any
            particular participant in the environment. Of interest to
            the participants in the environment would be a similar
            analysis of each product or service rendered in the
            marketplace.

            \subidx{\market}{fiscal strategy}
            \subidx{markets}{analysis}
            \subidx{analysis}{markets}
            \subidx{strategy}{fiscal}
            \subidx{fiscal}{strategy}
            \subidx{\market}{fiscal strategy}
            As a simple illustrative example, a company operating in
            this environment might obtain a credit line from a bank
            that is equal to {\twoponehundred}\% of its rate of
            revenue returns, (per {\timescale},) to finance additional
            operations. In this simple scenario, the company would use
            its revenue base as collateral for the loan. Some
            {\timescale}s, depending on the {\market}'s environment,
            the company's rate of revenue returns exceeds what was
            borrowed from the bank, and the loan is repaid in
            full. Other {\timescale}s, the company must default, and
            the bank seizes a portion of the company's revenue base to
            pay the delinquent loan. However, on the average, the
            company will expand its rate of revenue returns at
            {\twologreturnshundred}\% per {\timescale}.

            \subidx{\market}{fiscal strategy}
            \subidx{markets}{analysis}
            \subidx{analysis}{markets}
            \subidx{strategy}{fiscal}
            \subidx{fiscal}{strategy}
            \subidx{\market}{fiscal strategy}
            As another simple example, a company re-invests
            {\twoponehundred}\% of its rate of revenue returns, (per
            {\timescale},) in development, marketing, sales, and
            distribution of new products.  Although some products will
            be successful and the return on the investment will exceed
            the {\twoponehundred}\% per {\timescale} investment,
            others will not. However, on the average, the company will
            expand it gross rate of revenue returns at
            {\twologreturnshundred}\% per {\timescale}.

            \subidx{\market}{fiscal strategy}
            \subidx{markets}{analysis}
            \subidx{analysis}{markets}
            \subidx{strategy}{fiscal}
            \subidx{fiscal}{strategy}
            \subidx{\market}{fiscal strategy}
            \subidx{\market}{product portfolio}
            \subidx{\market}{product diversity}
            \subidx{\market}{product mix}
            \subidx{\market}{optimum number of products}
            \idx{product portfolio}
            \idx{product diversity}
            \idx{optimum number of products}
            \idx{product mix}

            As an example of ``product portfolio'' management, suppose
            a company re-invests {\twoponehundred}\% of its rate of
            revenue returns, (per {\timescale},) in development,
            marketing, sales, and distribution of new products.
            Further suppose that the company has two products, and a
            fractal analysis of the individual product rate of revenue
            return time series indicates that one product has a
            Shannon probability of 0.65, and the other has a Shannon
            probability of 0.55. Then the percentage of re-investment
            in the first product would be $(2 \cdot 0.65 - 1) \cdot
            {\twoponehundred}$, percent of the rate of revenue
            returns, and $(2 \cdot 0.55 - 1) \cdot {\twoponehundred}$
            percent for the second product, implying that the company
            should diversify its product line\footnote{The astute
            reader would note that the linear addition was used to add
            the contribution to development of each product. This is a
            ``near term'' interpretation. Actually, in general, the
            method used should be a root mean square process,
            dependent on the Hurst Coefficient, $H$, where
            $P_{total}^H = P_1^H + P_2^H + \cdots$, where $P_n$ is the
            contribution to each individual product. For a Brownian
            motion, or random walk type of fractal the Hurst
            Coefficient is a function of time into the future. For the
            ``near term,'' the Hurst coefficient is very near unity,
            meaning the summation process is linear. For the ``long
            term,'' $H \approx 0.5$, or a standard root mean square
            summation process should be used. If $H$ is $0.5$ then the
            market is termed a Brownian motion, or random walk
            process. If it is larger than 0.5, it is termed fractional
            Brownian motion process. For a random walk process, ``near
            term'' and ``far term'' are quantitatively differentiated
            on the Hurst Coefficient graph where $1 - \ln (t) = 0.5
            \cdot \ln (t)$, or when $\ln (t) = 2$, or $t =
            7.389\ldots$ See~\cite[pp. 67, 83-84]{Peters:CAOITCM}
            and~\cite[pp. 129, 159]{Schroeder} for particulars on the
            implications of the Hurst Coefficient and root mean square
            summation issues.}.  Note that this is a ``bet hedging''
            metric methodology, and assumes that the products have
            uncorrelated revenue return rates. If this re-investment
            methodology is not feasible, perhaps for strategic
            financial reasons, then the re-investment in both products
            should total the ${\twoponehundred}$\%, and the investment
            in each product should be made at a ratio of $\frac{(2
            \cdot 0.65 - 1)}{(2 \cdot 0.55 - 1)} = 3 : 1$,
            respectively. Note that this ``bet hedging'' can be used
            to define the optimal number of products that can be
            supported on the rate of revenue returns. If it assumed
            that all products are ``typical'' for the {\market}, as a
            standard bench mark, then the optimal number will be
            $\frac{1}{{\twopone}}$. Note that this is a
            ``theoretical'' value, since not all products are
            ``typical,'' and there may be strategic reasons, for
            example product leveraging, that may increase the number
            of products above the optimum. However, most of the
            revenue should come from the optimal number of products,
            since having more products will decrease the amount of the
            potential investment in each product, and having less than
            the optimum number of products will increase the risk that
            many of the products could suffer a ``down market''
            concurrently, impacting the rate of revenue returns.  As
            another interesting interpretation of the optimal
            ``hedging of bets,'' in product portfolio strategy, and
            considering the graph of the normalized increments
            presented in Figure~\ref{\SETLABEL:TF}, if the
            organization is running optimally, then these products
            will generate, at least in principle, one standard
            deviation, approximately $0.8413 = 84.13$\% of the future
            growth in rate of revenue returns. Naturally, these are
            approximations, and the values are an approximation to a,
            probably, complex process, and appropriate scrutiny should
            be exercised before making specific projections.  As yet
            another example of ``product portfolio'' management,
            consider the issue of product mix. In this interpretation,
            {\twoponehundred}\% of the product manufactured should be
            ``proprietary,'' while the rest is ``industry standard.''
            As yet another possibility, {\twoponehundred}\% of the
            product manufactured should be predatory into new markets,
            and the remainder in markets that are ``traditional'' for
            the company.

% Local Variables:
% TeX-parse-self: t
% TeX-auto-save: t
% TeX-master: "fractal.tex"
% End:


        %
% -----------------------------------------------------------------------------
%
% A license is hereby granted to reproduce this software source code and
% to create executable versions from this source code for personal,
% non-commercial use.  The copyright notice included with the software
% must be maintained in all copies produced.
%
% THIS PROGRAM IS PROVIDED "AS IS". THE AUTHOR PROVIDES NO WARRANTIES
% WHATSOEVER, EXPRESSED OR IMPLIED, INCLUDING WARRANTIES OF
% MERCHANTABILITY, TITLE, OR FITNESS FOR ANY PARTICULAR PURPOSE.  THE
% AUTHOR DOES NOT WARRANT THAT USE OF THIS PROGRAM DOES NOT INFRINGE THE
% INTELLECTUAL PROPERTY RIGHTS OF ANY THIRD PARTY IN ANY COUNTRY.
%
% Copyright (c) 1994-2006, John Conover, All Rights Reserved.
%
% Comments and/or bug reports should be addressed to:
%
%     john@email.johncon.com (John Conover)
%
% -----------------------------------------------------------------------------
%
% Revision: \RCSRevision \\
% Revision Time: \RCSTime UMT \\
% Revision Date: \RCSDate \\
% Revision Id: \RCSId \\
% Revision File: \RCSLog \\
\RCS $Revision: 0.0 $
\RCS $Date: 2006/01/20 04:38:13 $
\RCS $Id: companies.tex,v 0.0 2006/01/20 04:38:13 john Exp $
% $Log: companies.tex,v $
% Revision 0.0  2006/01/20 04:38:13  john
% Initial version
%
%
    \subsection{Number of Companies}
        \label{\SETLABEL:QNC}

        \subidx{\market}{number of companies}
        \subidx{number of companies}{analysis}
        \subidx{analysis}{number of companies}
        \subidx{Shannon}{probability}
        \subidx{probability}{Shannon}
        This section evaluates the approximate, or ``average,'' number
        of companies in the {\market}, and uses the method outlined in
        Chapter~\ref{general}, Section~\ref{aftsma}. Since the
        average, $avg_{ind}$, and the root mean square, $rms_{ind}$,
        of the normalized increments of the {\market} time series is
        \datafractionmean, and \datafractionrms respectively, the
        number of companies participating in the market can be
        calculated by Equation~\ref{ncompanies} to be {\ncompanies}.

        If this value seems consistent number of companies in the
        {\market}, within the assumptions outlined in
        Chapter~\ref{general}, Section~\ref{aftsma}, then it would
        seem that there is some circumstantial or indirect evidence
        that the companies participating in the {\market} are
        operating optimally, and the ``average'' Shannon probability,
        $P$ for each participating company would be, using
        Equation~\ref{pncompanies}, {\pncompanies}, which would be the
        value which should be used in Section~\ref{\SETLABEL:FS} for
        each participating company if market expansion was to be
        consistent with the rest of the industry. However, if the
        Shannon probability derived in Section~\ref{\SETLABEL:FS} is
        greater than the average Shannon probability for the companies
        participating in the {\market}, as derived in this section,
        then the market would, possibly, be exploitable with the
        fiscal strategy outlined in Section~\ref{\SETLABEL:FS}. The
        maximum exploitability for the {\market} is derived in
        Section~\ref{\SETLABEL:MAXSHANNON}, but it is probably of
        doubtful practicality.

        Note that these optimizations would maximize a company's
        market growth. Since there are probably many companies
        competing in the market place, this would not necessarily
        maximize a company's P\&L, as described in
        Chapter~\ref{general}, Section~\ref{ompl}. The Shannon
        probability that maximizes market share in the {\market} is
        \pncompanies, with several alternative solutions listed in the
        previous paragraph. However, these should be contrasted to the
        Shannon probability that maximizes a company's P\&L which is
        \avgrms~in the {\market}. In all cases, the fraction of the
        P\&L that should be ``wagered'' on the future, $f$, should be:

        \begin{equation}
            f = 2P - 1
        \end{equation}

        \noindent where $P$ is the particular Shannon probability
        chosen optimize a particular fiscal strategy. Interestingly,
        the measured Shannon probability of the {\market} would tend
        to indicate that the companies participating in the market
        have chosen a fiscal strategy that optimizes market growth, as
        opposed to capital growth.

        \subidx{\market}{increasing returns}
        \subidx{economic increasing returns}{\market}
        As interesting interpretation of these exploitive issues,
        since all three fiscal strategies will result in exponential
        market growth for every company participating in the market,
        is that they may represent, perhaps, an example of
        ``increasing returns.''

% Local Variables:
% TeX-parse-self: t
% TeX-auto-save: t
% TeX-master: "fractal.tex"
% End:


        %
% -----------------------------------------------------------------------------
%
% A license is hereby granted to reproduce this software source code and
% to create executable versions from this source code for personal,
% non-commercial use.  The copyright notice included with the software
% must be maintained in all copies produced.
%
% THIS PROGRAM IS PROVIDED "AS IS". THE AUTHOR PROVIDES NO WARRANTIES
% WHATSOEVER, EXPRESSED OR IMPLIED, INCLUDING WARRANTIES OF
% MERCHANTABILITY, TITLE, OR FITNESS FOR ANY PARTICULAR PURPOSE.  THE
% AUTHOR DOES NOT WARRANT THAT USE OF THIS PROGRAM DOES NOT INFRINGE THE
% INTELLECTUAL PROPERTY RIGHTS OF ANY THIRD PARTY IN ANY COUNTRY.
%
% Copyright (c) 1994-2006, John Conover, All Rights Reserved.
%
% Comments and/or bug reports should be addressed to:
%
%     john@email.johncon.com (John Conover)
%
% -----------------------------------------------------------------------------
%
% Revision: \RCSRevision \\
% Revision Time: \RCSTime UMT \\
% Revision Date: \RCSDate \\
% Revision Id: \RCSId \\
% Revision File: \RCSLog \\
\RCS $Revision: 0.0 $
\RCS $Date: 2006/01/20 04:38:13 $
\RCS $Id: operations.tex,v 0.0 2006/01/20 04:38:13 john Exp $
% $Log: operations.tex,v $
% Revision 0.0  2006/01/20 04:38:13  john
% Initial version
%
%
    \subsection{Fixed Increment Approximation for Operational Strategy}
        \label{\SETLABEL:OPS}.

        This section derives various values based on the ``average''
        of the normalized increments presented in
        Figure~\ref{\SETLABEL:TFA}. These values are an approximation
        to a, probably, complex process with a distribution shown in
        Figure~\ref{\SETLABEL:TF}. These values will be used in a
        fixed increment Brownian fractal analysis and simulation of
        the {\market}, and may, or may not, provide adequate accuracy
        for projections.

        \subidx{\market}{fiscal strategy}
        \subidx{\market}{Shannon probability}
        \subidx{strategy}{fiscal}
        \subidx{fiscal}{strategy}
        \subidx{Shannon}{probability}
        \subidx{probability}{Shannon}
        It should be noted that the analysis of fiscal strategy,
        presented in Section~\ref{\SETLABEL:FS}, is derived from the
        {\market} metrics and may, or may not, be maximally
        optimal. For the optimal fiscal strategy, which may be
        exploitable, see Section~\ref{\SETLABEL:MAXSHANNON}.

        \subidx{strategy}{exploitable}
        \subidx{exploitable}{strategy}
        \subidx{\market}{windows of opportunity}
        \idx{windows of opportunity}
        \subidx{decision}{obsolete}
        \subidx{obsolete}{decision}
        \subidx{decision}{timeliness}
        \subidx{timeliness}{decision}
        \subidx{rate of revenue returns}{forecast}
        \subidx{forecast}{rate of revenue returns}
        An additional exploitable strategy may be time itself.
        Equations~\ref{\SETLABEL:V},~\ref{\SETLABEL:R},
        and,~\ref{\SETLABEL:MA}, are, essentially, metrics on how fast
        a decision, which is based on information concerning the
        current status of the {\market}, becomes obsolete. Obviously,
        how long a decision is expected to remain relevant should be
        addressed as an operational necessity in strategic planning
        and project management. Figures~\ref{\SETLABEL:FN},
        and,~\ref{\SETLABEL:FF} compare methods of approximation of
        the ``forecastability'' of rate of revenue returns in the
        {\market} for the near term and far
        term~\cite[pp. 83-84]{Peters:CAOITCM}, respectively. As a
        general rule, caution must be exercised when making decisions
        that will span a time interval larger than the time interval
        where the ``forecastability'' of rate of revenue returns drops
        below 50\%. Beyond this time interval, the chances increase
        that the competitive and market forces will alter the market
        environment in a possibly detrimental unanticipated
        fashion. Obviously, there is significant advantage in
        ``timeliness'' of development, manufacturing, and distribution
        of products and services that are consistent with this
        temporal agenda. Automation of these processes, if executed
        consistently with this agenda, should be considered a
        competitive advantage.

        \subidx{strategy}{exploitable}
        \subidx{exploitable}{strategy}
        \subidx{rate of revenue returns}{forecast}
        \subidx{forecast}{rate of revenue returns}
        \idx{product life cycle}
        \idx{life cycle, product}
        In some sense, this temporal agenda defines the ``average''
        product or service life cycle in the {\market}. When the
        ``forecastability'' of rate of revenue returns drops below
        50\%, there is an even chance that the rate of revenue returns
        for the product or service will change in a detrimental
        fashion. If it is assumed that a product or service life cycle
        consists of a ramp up, a maintenence interval, and a ramp
        down, then, if all three life cycle intervals are equal, the
        product life cycle will be, approximately, three times the
        time interval where the ``forecastability'' of rate of revenue
        returns drops below 50\%. Although probably not an accurate
        prediction of product or service life cycle, the technique may
        be used as a conceptual approximation to the dynamics of
        ``market windows.\footnote{For example, consider the market
        for table salt. Since it has inelastic supply and demand
        curves, and is a necessary requirement for life, it would be
        expected that the Hurst coefficient would be very near
        unity---ignoring competitive pressures in the market. The
        predictability of the table salt market would, therefore, be
        expected to be relatively good, over time.}''  The conceptual
        approximation will probably predict a ``conservative'' or
        ``pessimistic'' value in relation to actual markets.

        \begin{figure}[ht]
            \begin{center}
                \begin{minipage}[t]{0.45\textwidth}
                    \epsfxsize=1.0\linewidth
                    \epsffile{\directory/datahurstlownear.eps}
                    \caption[{\market}, ``forecastability'' of near
                        term rate of revenue returns]{{\market},
                        ``forecastability'' of near term rate of
                        revenue returns. Although the error function
                        is the most accurate, for the near term,
                        $H^{t} = \thurstlow^{t}$ may be used as a
                        reliable metric of ``forecastability'' of the
                        rate of revenue returns.}
                    \label{\SETLABEL:FN}
                \end{minipage}
                \hfill
                \begin{minipage}[t]{0.45\textwidth}
                    \epsfxsize=1.0\linewidth
                    \epsffile{\directory/datahurstlowfar.eps}
                    \caption[{\market}, ``forecastability'' of far
                        term rate of revenue returns]{{\market},
                        ``forecastability'' of far term rate of
                        revenue returns. Although the error function
                        is the most accurate, for the far term,
                        $\frac{1}{\sqrt{t}}$ may be used as a reliable
                        metric of ``forecastability'' of the rate of
                        revenue returns.}
                    \label{\SETLABEL:FF}
                \end{minipage}
            \end{center}
        \end{figure}

        \idx{operations research}
        As an interesting interpretation of the data presented in
        Figure~\ref{\SETLABEL:FN}, there may be, perhaps, some
        applicability to such operational agendas as inventory
        control. Maintaining too little inventory, obviously, will
        create a situation where the organization can not exploit
        market expansion, and maintaining too much inventory,
        likewise, would over extend the company, creating unnecessary
        losses when the market contracts. The company should maintain
        inventory levels that do not exceed, from
        Equation~\ref{\SETLABEL:MA}, ${\thurstlow}^{n} = 0.5$
        {\timescale}s of operations. Since the optimal amount of
        inventory and, from Equation~\ref{\SETLABEL:V}, the variance
        of change in the rate of revenue returns in the future can be
        calculated, there may, perhaps, be some applicability to a
        forecasting methodology that can be incorporated into other
        areas of operations research, for example the linear algebras
        using simplex methodologies for optimization of manufacturing
        processes. Traditionally, these forecasts are made by the
        sales department, and are subject to various subjective
        biases.

% Local Variables:
% TeX-parse-self: t
% TeX-auto-save: t
% TeX-master: "fractal.tex"
% End:


        %
% -----------------------------------------------------------------------------
%
% A license is hereby granted to reproduce this software source code and
% to create executable versions from this source code for personal,
% non-commercial use.  The copyright notice included with the software
% must be maintained in all copies produced.
%
% THIS PROGRAM IS PROVIDED "AS IS". THE AUTHOR PROVIDES NO WARRANTIES
% WHATSOEVER, EXPRESSED OR IMPLIED, INCLUDING WARRANTIES OF
% MERCHANTABILITY, TITLE, OR FITNESS FOR ANY PARTICULAR PURPOSE.  THE
% AUTHOR DOES NOT WARRANT THAT USE OF THIS PROGRAM DOES NOT INFRINGE THE
% INTELLECTUAL PROPERTY RIGHTS OF ANY THIRD PARTY IN ANY COUNTRY.
%
% Copyright (c) 1994-2006, John Conover, All Rights Reserved.
%
% Comments and/or bug reports should be addressed to:
%
%     john@email.johncon.com (John Conover)
%
% -----------------------------------------------------------------------------
%
% Revision: \RCSRevision \\
% Revision Time: \RCSTime UMT \\
% Revision Date: \RCSDate \\
% Revision Id: \RCSId \\
% Revision File: \RCSLog \\
\RCS $Revision: 0.0 $
\RCS $Date: 2006/01/20 04:38:13 $
\RCS $Id: simulation.tex,v 0.0 2006/01/20 04:38:13 john Exp $
% $Log: simulation.tex,v $
% Revision 0.0  2006/01/20 04:38:13  john
% Initial version
%
%
    \subsection{Simulation of Fixed Increment Approximation for Fiscal Strategy}
        \label{\SETLABEL:TSUNFAIRBROWNIAN}

        \subidx{\market}{market simulation}
        The data in this section is presented in tabular form in
        Section~\ref{\SETLABELREF:SIM}.
        Figure~\ref{\SETLABEL:TSUNFAIRBROWNIAN0} represents a
        constructional simulation of the time series data presented in
        Figure~\ref{\SETLABEL:TS}. The program {\it
        tsunfairbrownian}\/, which is briefly described in
        appendix~\ref{programs}, was used in the reconstruction. The
        reconstructed data is superimposed on the original time series
        data.  The program, {\it tsunfairbrownian}\/, essentially,
        constructs the new time series as a Brownian fractal with
        fixed increments---the value of the fixed increment is derived
        from the root mean square average of the normalized increments
        presented in Figure~\ref{\SETLABEL:TF}. The ``quality'' of
        such a reconstruction should be subject to adequate scepticism
        and scrutiny since, in all probability, the normalized
        increments presented in Figure~\ref{\SETLABEL:TF} represent a
        relatively complex process, that may not be ``modeled'' with
        such a simple methodology.

        As a further comparison of the the constructional simulation
        with the original time series data,
        Figure~\ref{\SETLABEL:TSUNFAIRBROWNIAN1} presents a normalized
        histogram of the normalized increments of the reconstructed
        time series, superimposed on the normalized histogram
        presented in Figure~\ref{\SETLABEL:NH}.

        \subidx{\market}{fiscal strategy, simulation}
        \subidx{markets}{simulation}
        \subidx{simulation}{markets}
        \subidx{strategy}{fiscal, simulation}
        \subidx{fiscal}{strategy, simulation}
        \subidx{programs}{tsunfairbrownian}
        \subidx{tsunfairbrownian}{program}
        \begin{figure}[ht]
            \begin{center}
                \begin{minipage}[t]{0.45\textwidth}
                    \epsfxsize=1.0\linewidth
                    \epsffile{\directory/tsunfairbrownian-f.eps}
                    \caption[{\market}, Time series data, empirical and
                        simulated]{{\market}, Time series data, empirical
                        and simulated, using the program {\it tsunfairbrownian}\/
                        with f = {\datafractionrms}. This data is
                        superimposed on the data presented in
                        Figure~\ref{\SETLABEL:TS}.}
                    \label{\SETLABEL:TSUNFAIRBROWNIAN0}
                \end{minipage}
                \hfill
                \begin{minipage}[t]{0.45\textwidth}
                    \epsfxsize=1.0\linewidth
                    \epsffile{\directory/tsunfairbrownian-f.tsfraction.tsnormal-s30.eps}
                    \caption[{\market}, normalized histogram,
                        empirical and simulated]{{\market}, normalized
                        histogram of the normalized increments of the
                        time series data shown in
                        Figure~\ref{\SETLABEL:TSUNFAIRBROWNIAN0},
                        empirical and simulated.  The empirical data
                        has a mean of {\datafractionmean}, with a
                        standard deviation of {\datafractionstddev}.
                        By comparison, the simulated data has a mean
                        of {\tsunfairbrownianfractionmean} with a
                        standard deviation of
                        {\tsunfairbrownianfractionstddev}. This data
                        is superimposed on the data presented in
                        Figure~\ref{\SETLABEL:NH}. The area under the
                        four curves is identical.}
                    \label{\SETLABEL:TSUNFAIRBROWNIAN1}
                \end{minipage}
            \end{center}
        \end{figure}

% Local Variables:
% TeX-parse-self: t
% TeX-auto-save: t
% TeX-master: "fractal.tex"
% End:


        %
% -----------------------------------------------------------------------------
%
% A license is hereby granted to reproduce this software source code and
% to create executable versions from this source code for personal,
% non-commercial use.  The copyright notice included with the software
% must be maintained in all copies produced.
%
% THIS PROGRAM IS PROVIDED "AS IS". THE AUTHOR PROVIDES NO WARRANTIES
% WHATSOEVER, EXPRESSED OR IMPLIED, INCLUDING WARRANTIES OF
% MERCHANTABILITY, TITLE, OR FITNESS FOR ANY PARTICULAR PURPOSE.  THE
% AUTHOR DOES NOT WARRANT THAT USE OF THIS PROGRAM DOES NOT INFRINGE THE
% INTELLECTUAL PROPERTY RIGHTS OF ANY THIRD PARTY IN ANY COUNTRY.
%
% Copyright (c) 1994-2006, John Conover, All Rights Reserved.
%
% Comments and/or bug reports should be addressed to:
%
%     john@email.johncon.com (John Conover)
%
% -----------------------------------------------------------------------------
%
% Revision: \RCSRevision \\
% Revision Time: \RCSTime UMT \\
% Revision Date: \RCSDate \\
% Revision Id: \RCSId \\
% Revision File: \RCSLog \\
\RCS $Revision: 0.0 $
\RCS $Date: 2006/01/20 04:38:13 $
\RCS $Id: maximum.tex,v 0.0 2006/01/20 04:38:13 john Exp $
% $Log: maximum.tex,v $
% Revision 0.0  2006/01/20 04:38:13  john
% Initial version
%
%
    \subsection{Simulation of Fixed Increment Approximation for Optimally Maximal Fiscal Strategy}
        \label{\SETLABEL:MAXSHANNON}
        \subidx{\market}{fiscal strategy, simulation}
        \subidx{\market}{maximum Shannon probability}
        \subidx{markets}{simulation}
        \subidx{simulation}{markets}
        \subidx{strategy}{optimum fiscal, simulation}
        \subidx{fiscal}{optimum strategy, simulation}
        \subidx{programs}{tsunfairbrownian}
        \subidx{tsunfairbrownian}{program}
        \subidx{Shannon}{probability}
        \subidx{probability}{Shannon}

        \subidx{strategy}{exploitable}
        \subidx{exploitable}{strategy}
        \subidx{programs}{tsshannonmax}
        \subidx{tsshannonmax}{program}
        \subidx{programs}{tsunfairbrownian}
        \subidx{tsunfairbrownian}{program}
        \subidx{strategy}{fiscal}
        \subidx{fiscal}{strategy}
        The data in this section is presented in tabular form in
        Section~\ref{\SETLABELREF:MAXSHANNON}. One of the issues of
        analysis, as mentioned in Section~\ref{\SETLABEL:OPS}, is to
        determine the maximum Shannon probability for the time series
        presented in Figure~\ref{\SETLABEL:TS}. Potentially, this
        could be exploited with an aggressive fiscal
        strategy. Figure~\ref{\SETLABEL:SHANNONMAX0} is a graph of the
        output of the {\it tsshannonmax}\/ program, which is described
        briefly in appendix~\ref{programs}. The maximum of this
        function is the maximum Shannon probability for the time
        series data presented in Figure~\ref{\SETLABEL:TS}.
        Figure~\ref{\SETLABEL:SHANNONMAX1} was constructed using {\it
        tsunfairbrownian}\/ program, which is also described in
        appendix~\ref{programs}, with the maximum Shannon probability,
        and the time series data presented in
        Figure~\ref{\SETLABEL:TS}. This represents a ``what if'' the
        investment strategy was changed from a Shannon probability of
        {\shannonlogreturns}, as derived in Section~\ref{\SETLABEL:SP}
        to {\shannonmax}. This process, essentially, extracts the
        random statistical data from the time series presented in
        Figure~\ref{\SETLABEL:TS}, and constructs a new time series,
        using the random statistical data, with a different investment
        strategy.  The program, {\it tsunfairbrownian}\/, essentially,
        constructs the new time series as a Brownian fractal with
        fixed increments.  The ``quality'' of such a reconstruction
        should be subject to adequate scepticism and scrutiny since,
        in all probability, the increments in the original data
        represent a relatively complex process, that may not be
        ``modeled'' with such a simple methodology.

        \begin{figure}[ht]
            \begin{center}
                \begin{minipage}[t]{0.45\textwidth}
                    \epsfxsize=1.0\linewidth
                    \epsffile{\directory/data.tsshannonmax.eps}
                    \caption[{\market}, maximum rate of revenue
                        returns] {{\market}, maximum rate of revenue
                        returns, per {\timescale}, vs. Shannon
                        probability. The maximum rate of revenue
                        returns, per {\timescale}, occurs at a Shannon
                        probability of {\shannonmax}.}
                    \label{\SETLABEL:SHANNONMAX0}
                \end{minipage}
                \hfill
                \begin{minipage}[t]{0.45\textwidth}
                    \epsfxsize=1.0\linewidth
                    \epsffile{\directory/data.tsshannonmax-p.tsunfairbrownian-p.eps}
                    \caption[{\market}, maximum rate of revenue
                        returns] {{\market}, maximum rate of revenue
                        returns, per {\timescale}, at a Shannon
                        probability, of {\shannonmax}, corresponding
                        to a ``wager'' fraction of {\twoponemax}.}
                    \label{\SETLABEL:SHANNONMAX1}
                \end{minipage}
            \end{center}
        \end{figure}

        \subidx{fractional}{Brownian motion}
        \subidx{Brownian motion}{fractional}
        \subidx{Shannon}{probability}
        \subidx{probability}{Shannon}
        \subidx{programs}{tsshannonmax}
        \subidx{tsshannonmax}{program}
        If it is assumed that the time series data set, presented in
        Figure~\ref{\SETLABEL:TS}, constitutes classical Brownian
        motion, then the Shannon probability can be calculated by
        counting the total number of {\timescale}s that the {\market}
        movement was positive, and dividing by the total number of
        {timescale}s represented in the time series. This quotient is
        {\pmax}, as compared with the predicted value from the program
        {\it tsshannonmax}\/ of {\shannonmax}.

% Local Variables:
% TeX-parse-self: t
% TeX-auto-save: t
% TeX-master: "fractal.tex"
% End:


        %
% -----------------------------------------------------------------------------
%
% A license is hereby granted to reproduce this software source code and
% to create executable versions from this source code for personal,
% non-commercial use.  The copyright notice included with the software
% must be maintained in all copies produced.
%
% THIS PROGRAM IS PROVIDED "AS IS". THE AUTHOR PROVIDES NO WARRANTIES
% WHATSOEVER, EXPRESSED OR IMPLIED, INCLUDING WARRANTIES OF
% MERCHANTABILITY, TITLE, OR FITNESS FOR ANY PARTICULAR PURPOSE.  THE
% AUTHOR DOES NOT WARRANT THAT USE OF THIS PROGRAM DOES NOT INFRINGE THE
% INTELLECTUAL PROPERTY RIGHTS OF ANY THIRD PARTY IN ANY COUNTRY.
%
% Copyright (c) 1994-2006, John Conover, All Rights Reserved.
%
% Comments and/or bug reports should be addressed to:
%
%     john@email.johncon.com (John Conover)
%
% -----------------------------------------------------------------------------
%
% Revision: \RCSRevision \\
% Revision Time: \RCSTime UMT \\
% Revision Date: \RCSDate \\
% Revision Id: \RCSId \\
% Revision File: \RCSLog \\
\RCS $Revision: 0.0 $
\RCS $Date: 2006/01/20 04:38:13 $
\RCS $Id: verification.tex,v 0.0 2006/01/20 04:38:13 john Exp $
% $Log: verification.tex,v $
% Revision 0.0  2006/01/20 04:38:13  john
% Initial version
%
%
    \subsection{Qualitative Verification of Fixed Increment Approximation Analysis}
        \label{\SETLABEL:QVA}

        \subidx{\market}{verification of analysis}
        \subidx{verification}{analysis}
        \subidx{analysis}{verification}
        \subidx{quality}{of analysis}
        \subidx{verification}{of methodology}
        \subidx{methodology}{verification of}
        \subidx{Shannon}{probability}
        \subidx{probability}{Shannon}

        This section evaluates various values based on the ``average''
        of the normalized increments presented in
        Figure~\ref{\SETLABEL:TFA}. These values are an approximation
        to a, probably, complex process with a distribution shown in
        Figure~\ref{\SETLABEL:TF}. These values will be used in a
        fixed increment Brownian fractal analysis of the {\market},
        and may, or may not, provide adequate accuracy for
        projections.

        The data in this section is presented in tabular form in
        sections~\ref{\SETLABELREF:VI1} and~\ref{\SETLABELREF:VI2}.
        As a subjective evaluation of the ``quality'' of the analysis
        of the {\market}, from Chapter~\ref{methodology},
        Equation~\ref{metricvalues1}, and using the mean and root mean
        square values of the normalized increments of the time series
        data presented in Figure~\ref{\SETLABEL:TS} from
        Figure~\ref{\SETLABEL:TF}, and the Shannon probability as
        calculated by counting the total number of {\timescale}s that
        the {\market} movement was positive, as presented in
        Section~\ref{\SETLABEL:MAXSHANNON}:

        \begin{eqnarray}
                  P & \approx & \frac{\frac{avg}{rms} + 1}{2}\\
            {\pmax} & \approx & \frac{\frac{\datafractionmean}{\datafractionrms} + 1}{2}\\
            {\pmax} & \approx & {\avgrms}
            \label{\SETLABEL:AVGS}
        \end{eqnarray}

        \subidx{Shannon}{probability}
        \subidx{probability}{Shannon}
        \noindent and comparing these values to the Shannon
        probability, as found by the {\it tsshannonmax}\/ program, which
        iterates for a maximum:

        \begin{eqnarray}
            {\pmax} \approx {\avgrms} \approx {\shannonmax}
        \end{eqnarray}

        \subidx{logarithmic}{returns}
        \subidx{returns}{logarithmic}
        In addition, the different methods of calculating the
        logarithmic returns, presented in Section~\ref{\SETLABEL:FS},
        should be compared. The four methods used were the mean of
        Figure~\ref{\SETLABEL:TF}, the constant in the least squares
        approximation to Figure~\ref{\SETLABEL:TF}, the least squares
        exponential approximation to Figure~\ref{\SETLABEL:TS}, and
        the logarithmic returns of Figure~\ref{\SETLABEL:TS}, derived
        as the mean of the logarithms of the quotients of the
        increments. The values for each of the methods are,
        respectively:

        \begin{equation}
            \datafractionmeanbits \approx \datafractionconstantbits \approx \datatslsqepbits \approx \logreturns
        \end{equation}

        It is implied in Section~\ref{\SETLABEL:FS},
        Subsection~\ref{\SETLABEL:SP} and in
        Section~\ref{\SETLABEL:TSUNFAIRBROWNIAN} that, a Brownian
        motion with fixed increments fractal may ``model'' the
        {\market}. Using Equation~\ref{stddev9} from
        Chapter~\ref{general}, Section~\ref{abmfi}:

        \begin{eqnarray}
                                    rms \left(2P - 1\right) & \approx & \frac{\sigma \left(2P - 1\right)}{2 \sqrt{P\left(1 - P\right)}}\\
            \datafractionrms \left(2 \cdot \pmax - 1\right) & \approx & \frac{\datafractionstddev \left(2 \cdot \pmax - 1\right)}{2\sqrt{\pmax \left(1 - \pmax\right)}}\\
                       \datafractionrms \cdot \twopminusone & \approx & \datafractionstddev \cdot \twopx\\
                                                      \rmsp & \approx & \sigmap
        \end{eqnarray}

        \noindent and, equating to the mean:

        \begin{equation}
            \datafractionmean \approx \rmsp \approx \sigmap
        \end{equation}

        \subidx{Shannon}{probability}
        \subidx{probability}{Shannon}
        \noindent where, as in Equation~\ref{\SETLABEL:AVGS} using the
        mean, root mean square, and standard deviation values of the
        normalized increments of the time series data presented in
        Figure~\ref{\SETLABEL:TS} from Figure~\ref{\SETLABEL:TF}, and
        the Shannon probability as calculated by counting the total
        number of {\timescale}s that the {\market} movement was
        positive, as presented in Section~\ref{\SETLABEL:MAXSHANNON}.

        As a final qualitative comparison, the absolute value of the
        normalized increments should be the same as the root mean
        square value\footnote{The absolute value of the normalized
        increments, when averaged, is related to the root mean square
        of the increments by a constant. If the normalized increments
        are a fixed increment, the constant is unity. If the
        normalized increments have a Gaussian distribution, the
        constant is $\approx 0.8$ depending on the accuracy of of
        ``fit'' to a Gaussian distribution.}, where the absolute value
        is presented in Figure~\ref{\SETLABEL:TFA}, and the root mean
        square value is presented in Figure~\ref{\SETLABEL:TF}:

        \begin{equation}
            \datafractionabsmean \approx \datafractionrms
        \end{equation}

        Note, that if the {\market} could be ``modeled'' as a Brownian
        motion with fixed increments fractal, then the standard
        deviation of the absolute value of the normalized increments
        of the time series data presented in Figure~\ref{\SETLABEL:TS}
        from Figure~\ref{\SETLABEL:TF} should be zero. It is
        $\datafractionabsstddev$.

% Local Variables:
% TeX-parse-self: t
% TeX-auto-save: t
% TeX-master: "fractal.tex"
% End:


    \renewcommand{\market}{United States M2}
    \renewcommand{\directory}{../markets/us.m2}
    \renewcommand{\datafractionmean}{0.008052}
\renewcommand{\datafractionmeanbits}{0.011570}
\renewcommand{\datafractionmeanq}{0.002684}
\renewcommand{\datafractionmeanbitsq}{0.003867}
\renewcommand{\datafractionstddev}{0.038579}
\renewcommand{\datafractionrms}{0.039311}
\renewcommand{\avgrms}{0.602414}
\renewcommand{\ncompanies}{5.210454}
\renewcommand{\pncompanies}{0.544866}
\renewcommand{\datafractionabsmean}{0.029745}
\renewcommand{\datafractionabsstddev}{0.025769}
\renewcommand{\datafractionconstant}{0.010041}
\renewcommand{\datafractionconstantbits}{0.014414}
\renewcommand{\datafractionconstantq}{0.003347}
\renewcommand{\datafractionconstantbitsq}{0.004821}
\renewcommand{\datafractionslope}{-0.000021}
\renewcommand{\datafractionabsconstant}{0.035145}
\renewcommand{\datafractionabsslope}{-0.000057}
\renewcommand{\hurstall}{0.659558}
\renewcommand{\hurstlow}{0.707509}
\renewcommand{\hurstlowtwo}{1.415018}
\renewcommand{\hurstlowhundred}{70.750900}
\renewcommand{\hcalcall}{0.184942}
\renewcommand{\hcalclow}{0.102042}
\renewcommand{\shannonmax}{0.604167}
\renewcommand{\twoponemax}{0.208334}
\renewcommand{\logreturns}{0.010456}
\renewcommand{\twologreturns}{1.007274}
\renewcommand{\twologreturnshundred}{0.727387}
\renewcommand{\oneoverlogreturns}{95.638868}
\renewcommand{\pmax}{0.602094}
\renewcommand{\twopminusone}{0.204188}
\renewcommand{\rmsp}{0.008027}
\renewcommand{\twopx}{0.208583}
\renewcommand{\sigmap}{0.008047}
\renewcommand{\tsunfairbrownianfractionmean}{0.007862}
\renewcommand{\tsunfairbrownianfractionstddev}{0.038619}
\renewcommand{\shannonlogreturns}{0.560125}
\renewcommand{\shannonlogreturnshundred}{56.012500}
\renewcommand{\twopone}{0.120250}
\renewcommand{\twoponehundred}{12.025000}
\renewcommand{\hundredtwoponehundred}{87.975000}
\renewcommand{\hundredshannonlogreturnshundred}{43.987500}
\renewcommand{\datatslsqepbits}{0.007623}
\renewcommand{\thurstall}{0.633980}
\renewcommand{\thurstlow}{0.710108}
\renewcommand{\thurstlowtwo}{1.420216}
\renewcommand{\thurstlowhundred}{71.010800}
\renewcommand{\thcalcall}{0.247886}
\renewcommand{\thcalclow}{0.171737}
\renewcommand{\chisquared}{2.862000}
\renewcommand{\critical}{42.557000}

    \renewcommand{\timescale}{month}
    \subidx{market}{\market}
    \idx{\market}

    \section{\market}

        \renewcommand{\SETLABEL}{\LABPRE:USM2}
        \renewcommand{\SETLABELQ}{\LABPRE:USM2Q}
        \label{\SETLABEL}
        \renewcommand{\SETLABELREF}{\LABPREREF:USM2}

        \idx{United States Federal Reserve Board}
        For the analysis, the data was in the directory
        {\directory}\footnote{Data from the United States Federal
        Reserve Board, 1980---1994, by {\timescale}s, in billions of
        1987 dollars, US.}.

        The data in this section is presented in tabular form in
        Section~\ref{\SETLABELREF}. Note that in this analysis, the
        rate of revenue returns means the increase or decrease in the
        {\market}. This is included for comparative
        purposes. Presumably, the {\market} represents something of
        value, or it could be used as a ``futures'' derivative, and
        thus, it would be considered that there is a rate of revenue
        returns.

        %
% -----------------------------------------------------------------------------
%
% A license is hereby granted to reproduce this software source code and
% to create executable versions from this source code for personal,
% non-commercial use.  The copyright notice included with the software
% must be maintained in all copies produced.
%
% THIS PROGRAM IS PROVIDED "AS IS". THE AUTHOR PROVIDES NO WARRANTIES
% WHATSOEVER, EXPRESSED OR IMPLIED, INCLUDING WARRANTIES OF
% MERCHANTABILITY, TITLE, OR FITNESS FOR ANY PARTICULAR PURPOSE.  THE
% AUTHOR DOES NOT WARRANT THAT USE OF THIS PROGRAM DOES NOT INFRINGE THE
% INTELLECTUAL PROPERTY RIGHTS OF ANY THIRD PARTY IN ANY COUNTRY.
%
% Copyright (c) 1994-2006, John Conover, All Rights Reserved.
%
% Comments and/or bug reports should be addressed to:
%
%     john@email.johncon.com (John Conover)
%
% -----------------------------------------------------------------------------
%
% Revision: \RCSRevision \\
% Revision Time: \RCSTime UMT \\
% Revision Date: \RCSDate \\
% Revision Id: \RCSId \\
% Revision File: \RCSLog \\
\RCS $Revision: 0.0 $
\RCS $Date: 2006/01/20 04:38:13 $
\RCS $Id: fraction.tex,v 0.0 2006/01/20 04:38:13 john Exp $
% $Log: fraction.tex,v $
% Revision 0.0  2006/01/20 04:38:13  john
% Initial version
%
%
    \subsection{Time Series Increments Analysis}
        \label{\SETLABEL:TSA}

        \subidx{\market}{Time series analysis}
        \subidx{time series}{increments}
        \subidx{time series}{analysis}
        \subidx{cumulative sum}{analysis}
        \subidx{analysis}{cumulative sum}
        \subidx{analysis}{random process}
        \subidx{random process}{analysis}
        \subidx{Gaussian}{increments}
        \subidx{increments}{Gaussian}
        \subidx{Brownian}{motion, fractional}
        \subidx{fractional}{Brownian motion}
        \subidx{fractal}{Brownian motion}
        The data in this section is presented in tabular form in
        Section~\ref{\SETLABELREF:TSA}.  Figure~\ref{\SETLABEL:TS} is
        a graph of the time series data for the {\market}.

        \subidx{increments}{normalized}
        \subidx{normalized}{increments}
        \subidx{programs}{tsfraction}
        \subidx{tsfraction}{program}
        Figure~\ref{\SETLABEL:TF} is a graph of the normalized
        increments of the time series data presented in
        Figure~\ref{\SETLABEL:TS}. The data presented was made by
        running the program {\it tsfraction}\/ on the time series
        data. The program {\it tsfraction}\/ is described briefly in
        Appendix~\ref{programs}, and subtracts the previous value from
        the next value, dividing this difference by the previous
        value, for each element in the time series data. The new time
        series contains the instantaneous change in the rate of
        revenue returns, divided by the magnitude of the instantaneous
        rate of revenue returns.

        \subidx{mean}{standard deviation}
        \subidx{standard deviation}{mean}
        \idx{root mean square}
        \idx{least squares approximation}
        \begin{figure}[ht]
            \begin{center}
                \begin{minipage}[t]{0.45\textwidth}
                    \epsfxsize=1.0\linewidth
                    \epsffile{\directory/data.eps}
                    \caption{{\market}, time series data.}
                    \label{\SETLABEL:TS}
                    \label{\SETLABELQ:TS}
                \end{minipage}
                \hfill
                \begin{minipage}[t]{0.45\textwidth}
                    \epsfxsize=1.0\linewidth
                    \epsffile{\directory/data.tsfraction.eps}
                    \caption[{\market}, normalized
                        increments]{{\market}, normalized increments
                        of the time series data presented in
                        Figure~\ref{\SETLABEL:TS}. The mean is
                        {\datafractionmean} with a standard deviation
                        of {\datafractionstddev}. The formula for the
                        least squares approximation is
                        ${\datafractionconstant} +
                        {\datafractionslope}t$, and the root mean
                        squared value is {\datafractionrms}. The
                        graph, labeled ``data\-.tsfraction\-.tsrms,''
                        is the running root mean square, and
                        ``data\-.tsfraction\-.tsavg'' is the running
                        average of the normalized increments.  This
                        graph is the fraction of change in the time
                        series, as a function of time. Note that the
                        slope of the mean, {\datafractionslope}, is
                        the coefficient of the nonlinearity term in
                        the normalized increments. See
                        Chapter~\ref{general}, Section~\ref{nlextend}
                        for a possible application of the logistic
                        function to this data set.}
                    \label{\SETLABEL:TF}
                    \label{\SETLABELQ:TF}
                \end{minipage}
            \end{center}
        \end{figure}

        \subidx{absolute value}{increments}
        \subidx{increments}{absolute value}

        Figure~\ref{\SETLABEL:TFA} is a graph of the absolute value of
        the normalized increments of the time series data presented in
        Figure~\ref{\SETLABEL:TF}. The data presented was made by
        running the Unix utility sed(1) on the normalized increments
        time series data to remove the negative signs. This is an
        absolute value procedure.  The resulting time series contains
        the absolute value of the instantaneous change in the rate of
        revenue returns, divided by the magnitude of the instantaneous
        rate of revenue returns\footnote{The absolute value of the
        normalized increments, when averaged, is related to the root
        mean square of the increments by a constant. If the normalized
        increments are a fixed increment, the constant is unity. If
        the normalized increments have a Gaussian distribution, the
        constant is $\approx 0.8$ depending on the accuracy of of
        ``fit'' to a Gaussian distribution.}.

        \subidx{histogram}{normalized}
        \subidx{normalized}{histogram}
        \subidx{programs}{tsnormal}
        \subidx{tsnormal}{program}
        \subidx{mean}{standard deviation}
        \subidx{standard deviation}{mean}
        \idx{root mean square}
        \idx{least squares approximation}
        \subidx{\market}{analysis of increments}
        Figure~\ref{\SETLABEL:NH} is the normalized histogram of the
        normalized increments of the time series data shown in
        Figure~\ref{\SETLABEL:TF}. The abscissa is 3 $\sigma$ limits,
        and the area under the two curves is identical. The data for
        this figure was produced by the program {\it tsnormal}\/,
        which is described briefly in Appendix~\ref{programs}.

        \begin{figure}[ht]
            \begin{center}
                \begin{minipage}[t]{0.45\textwidth}
                    \epsfxsize=1.0\linewidth
                    \epsffile{\directory/data.tsfraction.abs.eps}
                    \caption[{\market}, absolute value of the
                        normalized increments]{{\market}, absolute
                        value of the normalized increments of the time
                        series data presented in
                        Figure~\ref{\SETLABEL:TF}.  The mean is
                        {\datafractionabsmean} with a standard
                        deviation of {\datafractionabsstddev}. The
                        formula for the least squares approximation is
                        ${\datafractionabsconstant} +
                        {\datafractionabsslope}t$, and the root mean
                        square value, from Figure~\ref{\SETLABEL:TF},
                        is {\datafractionrms}.  The graph, labeled
                        ``data\-.tsfraction\-.tsrms,'' is the running
                        root mean square, and
                        ``data\-.tsfraction\-.tsavg'' is the running
                        average of the normalized increments presented
                        in Figure~\ref{\SETLABEL:TF}, superimposed
                        here for convenience. This graph is the
                        absolute value of the fraction of change in
                        the time series, as a function of time.}
                    \label{\SETLABEL:TFA}
                    \label{\SETLABELQ:TFA}
                \end{minipage}
                \hfill
                \begin{minipage}[t]{0.45\textwidth}
                    \epsfxsize=1.0\linewidth
                    \epsffile{\directory/data.tsfraction.tsnormal-s30.eps}
                    \caption[{\market}, normalized histogram of the
                        normalized increments]{{\market}, normalized
                        histogram of the normalized increments of the
                        time series data shown in
                        Figure~\ref{\SETLABEL:TF}.  The data has a
                        mean of {\datafractionmean}, with a standard
                        deviation of {\datafractionstddev}.  The area
                        under the two curves is identical. The
                        $\chi^2$ value of the observed and expected
                        values of the two curves is {\chisquared},
                        with a critical value of {\critical}.}
                    \label{\SETLABEL:NH}
                \end{minipage}
            \end{center}
        \end{figure}

        \subidx{programs}{tsXsquared}
        \subidx{tsXsquared}{program}
        \subidx{\market}{chi-squared values of increments}
        The program {\it tsXsquared}\/, which is briefly described in
        appendix~\ref{programs}, was used to derive the $\chi^2$
        statistics for the data presented in
        Figure~\ref{\SETLABEL:NH}.

        \subidx{programs}{tsstatest}
        \subidx{tsstatest}{program}
        \subidx{\market}{statistical estimates}

        Figure~\ref{\SETLABEL:SE} is the statistical estimate for the
        data presented in Figure~\ref{\SETLABEL:TF}, as derived by the
        program {\it tsstatest}\/, which is briefly described in
        appendix~\ref{programs}.

        \begin{figure}[ht]
            \begin{center}
                \begin{minipage}[t]{\textwidth}
                    \center{\fbox{\parbox{0.9\textwidth}{\XXX{\directory/data.tsstatest-f0.1-c0.9-i.tex}}}}
                    \caption[{\market}, statistical estimates of the
                        normalized increments]{{\market}, statistical
                        estimates of the normalized increments of the
                        time series shown in Figure~\ref{\SETLABEL:TF}.
                        The table was produced with the {\it
                        tsstatest}\/ program, and illustrates the
                        size of the data set required for a confidence
                        level of 90\%, with an error estimate of $\pm$
                        10\%, or alternately, the error estimate on
                        the time series shown in Figure~\ref{\SETLABEL:TF}.}
                    \label{\SETLABEL:SE}
                \end{minipage}
            \end{center}
        \end{figure}

        Note that the data set size estimations, as produced by the
        {\it tsstatest}\/ program, are probably very conservative,
        depending on the magnitude of the Shannon probability, $P =
        \shannonlogreturns$, as derived in
        Section~\ref{\SETLABEL:SP}. See Chapter~\ref{general},
        Section~\ref{serdss} for possible alternative methodologies
        for addressing the analysis of fractal time series with
        limited data set sizes. Depending on the magnitude of the
        Shannon probability, $P$, these estimates can be several
        orders of magnitude too high.

        \subidx{derivative of increments}{normalized}
        \subidx{normalized}{derivative of increments}
        \subidx{programs}{tsderivative}
        \subidx{tsderivative}{program}
        Figure~\ref{\SETLABEL:TF1} is the normalized histogram of the
        first derivative of the normalized increments of the time
        series data shown in Figure~\ref{\SETLABEL:TF}. In principle,
        if the distribution of the normalized increments presented in
        Figure~\ref{\SETLABEL:NH} is Gaussian in nature, this
        distribution would be similar to ``white noise,'' as presented
        in appendix~\ref{programs}, Figure~\ref{whiteexample}. The
        data was generated by the {\it tsderivative}\/ program, which
        is briefly described in
        appendix~\ref{programs}. Figure~\ref{\SETLABEL:TF2} is the
        normalized histogram of the second derivative of the
        normalized increments of the time series data shown in
        Figure~\ref{\SETLABEL:TF}. In principle, if the distribution
        of the normalized increments presented in
        Figure~\ref{\SETLABEL:NH} is an integrated Gaussian
        distribution in nature, this distribution would be similar to
        ``white noise,'' as presented in appendix~\ref{programs},
        Figure~\ref{whiteexample}.

        \begin{figure}[ht]
            \begin{center}
                \begin{minipage}[t]{0.45\textwidth}
                    \epsfxsize=1.0\linewidth
                    \epsffile{\directory/data.tsfraction.tsderivative.tsnormal-s30.eps}
                    \caption[{\market}, histogram of the first
                        derivative of the increments]{{\market},
                        normalized histogram of the first derivative
                        of the normalized increments of the time
                        series data shown in
                        Figure~\ref{\SETLABEL:TF}.}
                    \label{\SETLABEL:TF1}
                \end{minipage}
                \hfill
                \begin{minipage}[t]{0.45\textwidth}
                    \epsfxsize=1.0\linewidth
                    \epsffile{\directory/data.tsfraction.2tsderivative.tsnormal-s30.eps}
                    \caption[{\market}, histogram of the second
                        derivative of the increments]{{\market},
                        normalized histogram of second derivative of
                        the the normalized increments of the time
                        series data shown in
                        Figure~\ref{\SETLABEL:TF}.}
                    \label{\SETLABEL:TF2}
                \end{minipage}
            \end{center}
        \end{figure}

        \subidx{fractal}{range}
        \subidx{fractal}{R/S analysis}
        \subidx{\market}{rate of revenue returns, range}
        \subidx{\market}{deterministic mechanism}
        \subidx{deterministic}{mechanism}
        \subidx{mechanism}{deterministic}
        Figure~\ref{\SETLABEL:TR} is the range of values of the time
        series shown in Figure~\ref{\SETLABEL:TS}. The horizontal axis
        is time into the future. In principle, if the time series was
        characterized as fractional Brownian motion the graph in
        Figure~\ref{\SETLABEL:TR} would be a square root
        function\footnote{Note that the ``roughness,'' or ``sawtooth''
        characteristics of the graph in Figure~\ref{\SETLABEL:TR} are
        a computational artifact---caused by not using the -m option
        to the program {\it tshurst}\/, which is computationally
        inefficient.}. Figure~\ref{\SETLABEL:TD} is the deterministic
        map of the normalized increments of the time series data shown
        in Figure~\ref{\SETLABEL:TF}. The deterministic map is useful
        for determining if a time series was created by a
        deterministic mechanism. This, essentially, maps each element
        in the time series with the previous element in the time
        series.  See,~\cite[pp. 745]{Peitgen}.

        \begin{figure}[ht]
            \begin{center}
                \begin{minipage}[t]{0.45\textwidth}
                    \epsfxsize=1.0\linewidth
                    \epsffile{\directory/data.tshurst-f.eps}
                    \caption[{\market}, range]{{\market}, range of the
                        time series data shown in
                        Figure~\ref{\SETLABEL:TS}.}
                    \label{\SETLABEL:TR}
                \end{minipage}
                \hfill
                \begin{minipage}[t]{0.45\textwidth}
                    \epsfxsize=1.0\linewidth
                    \epsffile{\directory/data.tsfraction.tsdeterministic.eps}
                    \caption[{\market}, deterministic map]{{\market},
                        deterministic map of the normalized increments
                        of the time series data shown in
                        Figure~\ref{\SETLABEL:TF}.}
                    \label{\SETLABEL:TD}
                \end{minipage}
            \end{center}
        \end{figure}

% Local Variables:
% TeX-parse-self: t
% TeX-auto-save: t
% TeX-master: "fractal.tex"
% End:


        \subsubsection{Observations on the Time Series Increments Analysis}

            Figure~\ref{\SETLABEL:NH} would seem to indicate that the
            time series data for the {\market} represents a cumulative
            sum/integration of a random process that has a Gaussian
            distribution, (ie., satisfies the Gaussian increments
            property of fractional Brownian
            motion~\cite[pp. 250]{Crownover},) tending to justify the
            assumption that the time series data represents fractional
            Brownian motion.

        %
% -----------------------------------------------------------------------------
%
% A license is hereby granted to reproduce this software source code and
% to create executable versions from this source code for personal,
% non-commercial use.  The copyright notice included with the software
% must be maintained in all copies produced.
%
% THIS PROGRAM IS PROVIDED "AS IS". THE AUTHOR PROVIDES NO WARRANTIES
% WHATSOEVER, EXPRESSED OR IMPLIED, INCLUDING WARRANTIES OF
% MERCHANTABILITY, TITLE, OR FITNESS FOR ANY PARTICULAR PURPOSE.  THE
% AUTHOR DOES NOT WARRANT THAT USE OF THIS PROGRAM DOES NOT INFRINGE THE
% INTELLECTUAL PROPERTY RIGHTS OF ANY THIRD PARTY IN ANY COUNTRY.
%
% Copyright (c) 1994-2006, John Conover, All Rights Reserved.
%
% Comments and/or bug reports should be addressed to:
%
%     john@email.johncon.com (John Conover)
%
% -----------------------------------------------------------------------------
%
% Revision: \RCSRevision \\
% Revision Time: \RCSTime UMT \\
% Revision Date: \RCSDate \\
% Revision Id: \RCSId \\
% Revision File: \RCSLog \\
\RCS $Revision: 0.0 $
\RCS $Date: 2006/01/20 04:38:13 $
\RCS $Id: instant.tex,v 0.0 2006/01/20 04:38:13 john Exp $
% $Log: instant.tex,v $
% Revision 0.0  2006/01/20 04:38:13  john
% Initial version
%
%
    \subsection{Instantaneous Analysis of Normalized Increments}
        \label{\SETLABEL:IA}

        \subidx{\market}{instantaneous analysis of normalized increments}
        \idx{average of normalized increments}
        \idx{root mean square of normalized increments}
        \subidx{Shannon probability}{instantaneous computation of}
        \subidx{average of normalized increments}{instantaneous computation of}
        \subidx{root mean square of normalized increments}{instantaneous computation of}
        \subidx{instantaneous computation}{Shannon probability}
        \subidx{instantaneous computation}{average of normalized increments}
        \subidx{instantaneous computation}{root mean square of normalized increments}
        \idx{time series}
        \subidx{time series}{instantaneous analysis}
        \subidx{instantaneous analysis}{time series}
        \subidx{time series}{increments}
        \subidx{time series}{analysis}
        \subidx{Shannon}{probability}
        \subidx{probability}{Shannon}
        \subidx{normalized}{increments}
        \subidx{increments}{normalized}

        The program {\it tsinstant}\/, which is briefly described in
        Appendix~\ref{programs}, is for finding the instantaneous
        fraction of change in a time series. The value of a sample in
        the time series is subtracted from the previous sample in the
        time series, and divided by the value of the previous sample.
        As explained in Chapter~\ref{general},
        Sections~\ref{derivation},~\ref{GA},~\ref{abmfi},~\ref{aftsma}
        and,~\ref{ompl} for Brownian motion, random walk fractals, the
        absolute value of the instantaneous fraction of change is also
        the root mean square of the instantaneous fraction of
        change\footnote{The absolute value of the normalized
        increments, when averaged, is related to the root mean square
        of the increments by a constant. If the normalized increments
        are a fixed increment, the constant is unity. If the
        normalized increments have a Gaussian distribution, the
        constant is $\approx 0.8$ depending on the accuracy of of
        ``fit'' to a Gaussian distribution.}. Squaring this value is
        the average of the instantaneous fraction of change, and
        adding unity to the absolute value of the instantaneous
        fraction of change, and dividing by two, is the Shannon
        probability of the instantaneous fraction of change.

        Figure~\ref{\SETLABEL:IA1} is the instantaneous value of the
        root mean square of the normalized increments for the
        {\market}, and Figure~\ref{\SETLABEL:IA2} is the instantaneous
        Shannon probability for the normalized increments.

        \begin{figure}[ht]
            \begin{center}
                \begin{minipage}[t]{0.45\textwidth}
                    \epsfxsize=1.0\linewidth
                    \epsffile{\directory/data.tsinstant-r.eps}
                    \caption[{\market}, instantaneous value of
                        rms.]{{\market}, instantaneous value of the
                        root mean square of the normalized increments,
                        provided by running the program {\it
                        tsinstant}\/ with the -r option on the data
                        presented in Figure~\ref{\SETLABEL:TS}.}
                    \label{\SETLABEL:IA1}
                    \label{\SETLABELQ:IA1}
                \end{minipage}
                \hfill
                \begin{minipage}[t]{0.45\textwidth}
                    \epsfxsize=1.0\linewidth
                    \epsffile{\directory/data.tsinstant-s.eps}
                    \caption[{\market}, instantaneous value of
                        Shannon probability.]{{\market}, instantaneous
                        value of the Shannon probability of the
                        normalized increments, provided by running the
                        program {\it tsinstant}\/ with the -s option
                        on the data presented in
                        Figure~\ref{\SETLABEL:TS}.}
                    \label{\SETLABEL:IA2}
                    \label{\SETLABELQ:IA2}
                \end{minipage}
            \end{center}
        \end{figure}

% Local Variables:
% TeX-parse-self: t
% TeX-auto-save: t
% TeX-master: "fractal.tex"
% End:


        %
% -----------------------------------------------------------------------------
%
% A license is hereby granted to reproduce this software source code and
% to create executable versions from this source code for personal,
% non-commercial use.  The copyright notice included with the software
% must be maintained in all copies produced.
%
% THIS PROGRAM IS PROVIDED "AS IS". THE AUTHOR PROVIDES NO WARRANTIES
% WHATSOEVER, EXPRESSED OR IMPLIED, INCLUDING WARRANTIES OF
% MERCHANTABILITY, TITLE, OR FITNESS FOR ANY PARTICULAR PURPOSE.  THE
% AUTHOR DOES NOT WARRANT THAT USE OF THIS PROGRAM DOES NOT INFRINGE THE
% INTELLECTUAL PROPERTY RIGHTS OF ANY THIRD PARTY IN ANY COUNTRY.
%
% Copyright (c) 1994-2006, John Conover, All Rights Reserved.
%
% Comments and/or bug reports should be addressed to:
%
%     john@email.johncon.com (John Conover)
%
% -----------------------------------------------------------------------------
%
% Revision: \RCSRevision \\
% Revision Time: \RCSTime UMT \\
% Revision Date: \RCSDate \\
% Revision Id: \RCSId \\
% Revision File: \RCSLog \\
\RCS $Revision: 0.0 $
\RCS $Date: 2006/01/20 04:38:13 $
\RCS $Id: logistic.tex,v 0.0 2006/01/20 04:38:13 john Exp $
% $Log: logistic.tex,v $
% Revision 0.0  2006/01/20 04:38:13  john
% Initial version
%
%
    \subsection{Logistic Analysis}
        \label{\SETLABEL:LA}

        \subidx{\market}{Logistic function analysis}
        \subidx{time series}{logistic function}
        \subidx{logistic function}{time series}
        \subidx{time series}{increments}
        \subidx{time series}{analysis}
        \subidx{cumulative sum}{analysis}
        \subidx{analysis}{cumulative sum}
        \subidx{analysis}{random process}
        \subidx{random process}{analysis}
        The data in this section is presented in tabular form in
        Section~\ref{\SETLABELREF:LAA}.  Figure~\ref{\SETLABEL:LA1} is
        a graph of the logistic function estimates of the time series
        data for the {\market}. The reader is cautioned that these
        graphs are constructed using the method suggested in
        Chapter~\ref{general}, Section~\ref{nlextend} and enormous
        precision is required for adequate prediction of the logistic
        function,~\cite{Modis}. Particularly, the non-linear term will
        usually require intervention to produce a practical fit to the
        data. In addition, there are numerical stability issues with
        logistic function methodologies\footnote{For example, in
        Figures~\ref{\SETLABEL:LA1} and~\ref{\SETLABEL:LA2}, if the
        non-linear term, $b$, was greater than zero, it was set to
        zero to produce the graphs. See Section~\ref{\SETLABELREF:LAA}
        for the actual derived values. In other cases, the magnitude
        of $b$ was too large, resulting in a graph that was decreasing
        as a function of time}.  The methodology should be regarded as
        ``fragile.'' It is included for completeness.

        \idx{least squares approximation}
        Figure~\ref{\SETLABEL:LA1} is a graph of the logistic function
        for the time series data presented in
        Figure~\ref{\SETLABEL:TS}. The data presented was made by
        running the program {\it tsdlogistic}\/, which is described
        briefly in Appendix~\ref{programs}, on the parameters
        extracted from the time series data as suggested in
        Figure~\ref{\SETLABEL:TF}. The program {\it tslsq}\/ was used
        to derive the constant and the slope of the normalized
        increments of the data presented in Figure~\ref{\SETLABEL:TF}.
        Figure~\ref{\SETLABEL:LA2} is the same graph, but with the
        time scale expanded by a factor of two.

        \begin{figure}[ht]
            \begin{center}
                \begin{minipage}[t]{0.45\textwidth}
                    \epsfxsize=1.0\linewidth
                    \epsffile{\directory/data.tsfraction.tslsq-p.tsdlogistic.eps}
                    \caption[{\market}, logistic function
                        estimates.]{{\market}, logistic function
                        estimates, provided by running the {\it
                        tslsq}\/ program on the normalized increments
                        presented in Figure~\ref{\SETLABEL:TF} with
                        the -p option. These parameters were used as
                        arguments to the {\it tsdlogistic}\/ program.}
                    \label{\SETLABEL:LA1}
                    \label{\SETLABELQ:LA1}
                \end{minipage}
                \hfill
                \begin{minipage}[t]{0.45\textwidth}
                    \epsfxsize=1.0\linewidth
                    \epsffile{\directory/data.tsfraction.tslsq-p.tsdlogistic2.eps}
                    \caption[{\market}, logistic function
                        estimates.]{{\market}, logistic function
                        estimates of Figure~\ref{\SETLABEL:LA1} with
                        the time scale expanded by a factor of two.}
                    \label{\SETLABEL:LA2}
                    \label{\SETLABELQ:LA2}
                \end{minipage}
            \end{center}
        \end{figure}

% Local Variables:
% TeX-parse-self: t
% TeX-auto-save: t
% TeX-master: "fractal.tex"
% End:


        %
% -----------------------------------------------------------------------------
%
% A license is hereby granted to reproduce this software source code and
% to create executable versions from this source code for personal,
% non-commercial use.  The copyright notice included with the software
% must be maintained in all copies produced.
%
% THIS PROGRAM IS PROVIDED "AS IS". THE AUTHOR PROVIDES NO WARRANTIES
% WHATSOEVER, EXPRESSED OR IMPLIED, INCLUDING WARRANTIES OF
% MERCHANTABILITY, TITLE, OR FITNESS FOR ANY PARTICULAR PURPOSE.  THE
% AUTHOR DOES NOT WARRANT THAT USE OF THIS PROGRAM DOES NOT INFRINGE THE
% INTELLECTUAL PROPERTY RIGHTS OF ANY THIRD PARTY IN ANY COUNTRY.
%
% Copyright (c) 1994-2006, John Conover, All Rights Reserved.
%
% Comments and/or bug reports should be addressed to:
%
%     john@email.johncon.com (John Conover)
%
% -----------------------------------------------------------------------------
%
% Revision: \RCSRevision \\
% Revision Time: \RCSTime UMT \\
% Revision Date: \RCSDate \\
% Revision Id: \RCSId \\
% Revision File: \RCSLog \\
\RCS $Revision: 0.0 $
\RCS $Date: 2006/01/20 04:38:13 $
\RCS $Id: hurst.tex,v 0.0 2006/01/20 04:38:13 john Exp $
% $Log: hurst.tex,v $
% Revision 0.0  2006/01/20 04:38:13  john
% Initial version
%
%
    \subsection{Hurst Coefficient Analysis}
        \label{\SETLABEL:H}

        \subidx{\market}{Hurst coefficient analysis}
        \subidx{Hurst coefficient}{analysis}
        \subidx{increments}{normalized}
        \subidx{normalized}{increments}
        \subidx{programs}{tshurst}
        \subidx{tshurst}{program}
        The data in this section is presented in tabular form in
        Section~\ref{\SETLABELREF:HCHP}. Figure~\ref{\SETLABEL:HC} is
        a graph of the Hurst coefficient data time series data shown
        in Figure~\ref{\SETLABEL:TS}. The slope of the graph is the
        Hurst coefficient.  The data for this figure was produced by
        the program {\it tshurst}\/, which is described briefly in
        Appendix~\ref{programs}.

        \subidx{\market}{H parameter analysis}
        \subidx{H parameter}{analysis}
        \subidx{programs}{tshcalc}
        \subidx{tshcalc}{program}
        Figure~\ref{\SETLABEL:HP} is a graph of the H parameter data
        for the normalized increments of the time series data shown in
        Figure~\ref{\SETLABEL:TF}. The data for this figure was
        produced by the program {\it tshcalc}\/, which is described
        briefly in Appendix~\ref{programs}.

        \begin{figure}[ht]
            \begin{center}
                \begin{minipage}[t]{0.45\textwidth}
                    \epsfxsize=1.0\linewidth
                    \epsffile{\directory/data.tshurst.eps}
                    \caption[{\market}, Hurst coefficient data]{{\market},
                        Hurst coefficient data for the normalized
                        increments of the time series data shown in
                        Figure~\ref{\SETLABEL:TF}.  The slope of the graph
                        is the Hurst coefficient.}
                    \label{\SETLABEL:HC}
                \end{minipage}
                \hfill
                \begin{minipage}[t]{0.45\textwidth}
                    \epsfxsize=1.0\linewidth
                    \epsffile{\directory/data.tshcalc.eps}
                    \caption[{\market}, H parameter data]{{\market}, H
                        parameter data for the normalized increments of
                        the time series data shown in
                        Figure~\ref{\SETLABEL:TF} The slope of the graph
                        is the H parameter.}
                    \label{\SETLABEL:HP}
                \end{minipage}
            \end{center}
        \end{figure}

        \subidx{revenue}{See, rate of revenue returns}
        \subidx{returns}{See, rate of revenue returns}
        \subidx{\market}{revenues}
        \subidx{Hurst coefficient}{analysis}
        \subidx{\market}{Hurst coefficient analysis}
        \subidx{\market}{rate of change}
        \subidx{\market}{windows of opportunity}
        \subidx{rate of revenue returns}{forecast}
        \subidx{forecast}{rate of revenue returns}
        \idx{windows of opportunity}
        \subidx{programs}{tslsq}
        \subidx{tslsq}{program}

        The approximately linear slope of the graph in
        Figure~\ref{\SETLABEL:HC} implies that the variance of the
        rate of revenue returns, (per {\timescale},) in the {\market},
        $V(t_2 - t_1)$, over a period of time is proportional to the
        period of time raised to twice the Hurst
        coefficient~\cite[pp. 180]{Feder},~\cite[pp. 246]{Crownover}.
        This seems to be a quantitative statement concerning how fast,
        and to what degree, the rate of revenue returns' state of
        affairs can change over a period of time.  An additional
        implication, for Hurst coefficients sufficiently close to 0.5,
        is that the probability of the state of affairs repeating
        sometime in the future goes down with increasing
        time\footnote{It can be shown that the number of expected
        market ``high'' and ``low'' transitions, $N$, scales with the
        square root of time, or $N \propto \sqrt {t}$, meaning that
        the cumulative distribution of the probability, $P$, of the
        duration of a market's ``high'' or ``low'' exceeding a given
        time interval, $t$, is proportional to the reciprocal of the
        square root of the time interval, $P \propto 1 / \sqrt {t}$,
        (or, conversely, that the probability of the duration of a
        market's ``high'' or ``low'' exceeding a given time interval
        is proportional to the reciprocal of the time interval raised
        to the power $3 / 2$, ie., $P \propto 1 / t^{3 /
        2}$,~\cite[pp. 153]{Schroeder}. What this means is that a
        histogram of the ``zero free'' run-lengths of a market being
        ``high'' or ``low,'' over a long time, would have a $1 / t^{3
        / 2}$ characteristic.)}, $t$, $p(t) = erf (1/\sqrt{2t})$ which
        is approximately $1/\sqrt{t}$ for $t \gg
        1$~\cite[pp. 160]{Schroeder}. Figures~\ref{\SETLABEL:FN},
        and,~\ref{\SETLABEL:FF} compare methods of approximation of
        the ``forecastability'' of the rate of revenue returns in the
        {\market} for the near term and far term,
        respectively~\cite[pp. 83-84]{Peters:CAOITCM}\footnote{The
        author is not comfortable with Peters' interpretation. For
        example, if the algorithm explained
        in~\cite[pp. 82]{Peters:CAOITCM} is used on ``white noise''
        which, by definition, never has any correlations, the short
        term Hurst coefficient, and thus the ``forecastability,'' is
        still near unity---a bit of an enigma. This can be verified
        with the {\it tswhite}\/ and {\it tshurst}\/ programs, which
        are briefly described in Appendix~\ref{programs}.}.  This
        seems to be a quantitative statement concerning ``windows of
        opportunity'' in the rate of revenue returns, (per
        {\timescale}.)  The program {\it tslsq}\/ was used on the
        Hurst coefficient data, presented in
        Figure~\ref{\SETLABEL:HC}, to provide a least squares
        approximation to the Hurst coefficient. The superimposed least
        squares approximation with on original Hurst coefficient data
        is presented.  The time series data has a Hurst coefficient of
        {\thurstlow}, so that:

        \subidx{\market}{Hurst coefficient analysis}
        \begin{eqnarray}
            V\left(t_2 - t_1\right) & \propto & \left(t_2 - t_1\right)^{2 \cdot H}\\
            V\left(t_2 - t_1\right) & \propto & \left(t_2 - t_1\right)^{2 \cdot {\thurstlow}}\\
                                    & \propto & \left(t_2 - t_1\right)^{\thurstlowtwo}
            \label{\SETLABEL:V}
        \end{eqnarray}

        \subidx{fractional}{Brownian motion}
        \subidx{Brownian motion}{fractional}
        \idx{fractal}
        \noindent where $V(t_2 - t_1)$ is the variance of the
        increments of the rate of revenue returns, (per {\timescale},)
        over the time interval $t_2 -
        t_1$,~\cite[pp. 177]{Feder},~\cite[pp. 494]{Peitgen}. If $H >
        \frac{1}{2}$, then the time series is termed as being
        characterized by ``fractional Brownian
        motion~\cite[pp. 170]{Feder}.''

        \subidx{rate of revenue returns}{predictability}
        \subidx{rate of revenue returns}{forecastability}
        \subidx{rate of revenue returns}{consistency}
        \subidx{predictability}{rate of revenue returns}
        \subidx{forecastability}{rate of revenue returns}
        \subidx{consistency}{rate of revenue returns}
        \subidx{\market}{rate of revenue returns, predictability}
        \subidx{\market}{rate of revenue returns, forecastability}
        \subidx{\market}{rate of revenue returns, consistency}
        \subidx{Hurst coefficient}{analysis}
        \subidx{\market}{Hurst coefficient analysis}
        \subidx{\market}{rate of change}

        In some sense, the Hurst coefficient is a quantitative
        expression of the ``forecastability'' of the future based on
        the past\footnote{Actually, in general, when summing fractal
        entities, the method used should be a root mean square
        process, dependent on the Hurst Coefficient, $H$, where
        $P_{total}^H = P_1^H + P_2^H + \cdots$, where $P_n$ is the
        fractal entities. For a Brownian motion, or random walk type
        of fractal the Hurst Coefficient is a function of time into
        the future. For the ``near term,'' the Hurst coefficient is
        very near unity, meaning the summation process is linear. For
        the ``long term,'' $H \approx 0.5$, or a standard root mean
        square summation process should be used. If $H$ is $0.5$ then
        the market is termed a Brownian motion, or random walk
        process. If it is larger than 0.5, it is termed fractional
        Brownian motion process. For a random walk process, ``near
        term'' and ``far term'' are quantitatively differentiated on
        the Hurst Coefficient graph where $1 - \ln (t) = 0.5 \cdot \ln
        (t)$, or when $\ln (t) = 2$, or $t = 7.389\ldots$ See
        Section~\ref{\SETLABEL:FS} for the particulars on using Hurst
        Coefficient to sum fractal process' for the {\market}. See
        also~\cite[pp. 67, 83-84]{Peters:CAOITCM} and~\cite[pp. 129,
        159]{Schroeder} for particulars on the implications of the
        Hurst Coefficient and root mean square summation issues.}.  A
        Hurst coefficient of {\thurstlow}, (for the near future, and
        {\thurstall} for the distant future.) implies that the
        likelihood of the rate of revenue returns, (per {\timescale},)
        for any two consecutive {\timescale}s being the same is
        {\thurstlowhundred}\%~\cite[pp. 66]{Peters:CAOITCM} for the
        near future, and {\thurstall} for the distant
        future. Likewise, there is a {\thurstlowhundred}\% chance of
        the rate of revenue returns, (per {\timescale},) movements
        being the same in consecutive time periods---ie., if, in a
        given {\timescale}, the rate of revenue returns, (per
        {\timescale},) is increasing, there is a {\thurstlowhundred}\%
        that the rate of revenue returns, (per {\timescale},) will
        increase in the following period, also. In some sense, this is
        a quantitative statement on how ``predictable,'' or
        ``forecastable'' the rate of revenue returns, (per
        {\timescale},) for the {\market} are over time, since the
        probability of having $n$ many consecutive {\timescale}s of
        the same agenda is $H^n$ where $H$ is the Hurst coefficient,
        or, letting the short term probability of having $n$ many
        {\timescale}s of the same market agenda, $p_a$, is:

        \begin{eqnarray}
            p_a\left(n\right) & = & H^{n}\\
                              & = & {\thurstlow}^{n}
            \label{\SETLABEL:MA}
        \end{eqnarray}

        \subidx{rate of revenue returns}{predictability}
        \subidx{rate of revenue returns}{forecastability}
        \subidx{rate of revenue returns}{consistency}
        \subidx{predictability}{rate of revenue returns}
        \subidx{forecastability}{rate of revenue returns}
        \subidx{consistency}{rate of revenue returns}
        As an interesting interpretation of the normalized increments
        of the time series data presented in
        Figure~\ref{\SETLABEL:TF}, if the vertical axis is multiplied
        by 100, to convert to percent, then the graph represents the
        error, in percent, that would be made by forecasting, month by
        month, that the next {\timescale}'s rate of revenue returns
        would be the same as the current {\timescale}'s revenue
        rate. Interestingly, it is $\datafractionmean \cdot 100$
        percent, on the average, with a standard deviation of
        $\datafractionstddev \cdot 100$ percent, and a root mean
        square error value of $\datafractionrms \cdot 100$
        percent---small values for such a simple forecasting
        mechanism.

        \subidx{\market}{rate of revenue returns, range}
        \subidx{Hurst coefficient}{analysis}
        \subidx{\market}{Hurst coefficient analysis}
        \subidx{\market}{rate of change}

        This is, essentially, a statement of the range of values, in
        the increments of the rate of revenue returns, (per
        {\timescale},) that is to be expected over the time interval,
        $t_2 - t_1$,
        $R_v$,~\cite[pp. 178]{Feder},~\cite[pp. 172]{Cambel}:

        \begin{eqnarray}
            R_v\left(t_2 - t_1\right) & \propto & \left(t_2 - t_1\right)^{H}\\
                                      & \propto & \left(t_2 - t_1\right)^{\thurstlow}
            \label{\SETLABEL:R}
        \end{eqnarray}

        \subidx{\market}{rate of revenue returns, range}
        \subidx{Hurst coefficient}{analysis}
        \subidx{\market}{Hurst coefficient analysis}
        \subidx{\market}{rate of change}
        \subidx{Markov}{statistics}
        \subidx{statistics}{Markov}
        \noindent where $R$ is the range of values in the increments
        of the rate of revenue returns, (per {\timescale}.) A Hurst
        coefficient, $H$, that is much larger than $\frac{1}{2}$, (but
        less than 1,) implies a strongly non-Gaussian distribution in
        the increments of the rate of revenue returns, (per
        {\timescale},)~\cite[pp. 152, 194]{Feder}, and a Hurst
        coefficient near $\frac{1}{2}$ implies that the increments of
        the rate of revenue returns, (per {\timescale}) is
        characteristic of an independent
        process~\cite[pp. 195]{Feder}. Extreme caution should be
        exercised in using Markov statistics in any analysis where the
        Hurst coefficient is not
        $\frac{1}{2}$,~\cite[pp. 124]{Crownover},~\cite[pp. 106]{Peters:CAOITCM}.


        As a useful approximation, if $H$, is approximately
        $\frac{1}{2}$, Equation~\ref{\SETLABEL:R} reduces
        to,~\cite[pp. 129]{Schroeder}:

        \begin{eqnarray}
            R\left(t_2 - t_1\right) & \propto & (t_2 - t_1)^{\frac{1}{2}}\\
                                    & \propto & \sqrt{\left(t_2 - t_1\right)}
        \end{eqnarray}

        \subidx{\market}{rate of revenue returns, range}
        \subidx{\market}{rate of revenue returns, increase and decrease}
        \subidx{Hurst coefficient}{analysis}
        \subidx{\market}{Hurst coefficient analysis}
        \subidx{\market}{rate of change}
        \subidx{Markov}{statistics}
        \subidx{statistics}{Markov}

        In the case where the Hurst coefficient, $H$, is
        $\frac{1}{2}$, the range of values in the increments of the
        rate of revenue returns, (per {\timescale},) divided by the
        standard deviation of these values, $S$, can be anticipated to
        increase over time according to the following
        relation,~\cite[pp. 154]{Feder},~\cite[pp. 129]{Schroeder}:

        \begin{equation}
            \frac{R\left(t_2 - t_1\right)}{S} \propto \left(t_2 - t_1\right)^{\frac{1}{2}}
        \end{equation}

        \subidx{\market}{rate of revenue returns, range}
        \subidx{\market}{rate of revenue returns, increase and decrease}
        \subidx{Hurst coefficient}{analysis}
        \subidx{\market}{Hurst coefficient analysis}
        \subidx{\market}{rate of change}
        \noindent which is a useful conceptual approximation, since it
        involves only the square root function---if the range and the
        standard deviation of the increments of the rate of revenue
        returns, (per {\timescale},) are known, (and $H \approx
        \frac{1}{2}$,) then the expected change in $\frac{R}{S}$, will
        increase with the square root of time\footnote{To be precise,
        it is actually asymptotically proportional to
        $\tau^{\frac{1}{2}}$}.

        Another useful approximation when rescaling processes that are
        characterize by Brownian motion, (ie., when $H \approx
        \frac{1}{2}$,) is that:

        \begin{eqnarray}
            X\left(t\right) & \propto & \frac{X\left(rt\right)}{r^{H}}\\
                            & \propto & \frac{X\left(rt\right)}{r^{\thurstlow}}
        \end{eqnarray}

        \idx{Brownian motion}
        \idx{fractal}
        Where $X(t)$ is the process characterized by Brownian motion,
        and $r$ is a scaling factor,~\cite[pp. 494]{Peitgen}.

        \subidx{programs}{tslsq}
        \subidx{tslsq}{program}
        The program {\it tslsq}\/ was used on the H parameter data,
        presented in Figure~\ref{\SETLABEL:HP}, to provide a least
        squares approximation to the H parameter for the
        {\market}. The superimposed least squares approximation on the
        original H parameter data is presented.  By contrast, the H
        parameter, as derived by the methodology outlined
        in~\cite[pp. 249]{Crownover}, is {\thcalclow} for the near
        future, and {\thcalcall} for the distant future.

        \subidx{\market}{Hurst coefficient analysis}
        \subidx{Hurst coefficient}{analysis}
        \subidx{increments}{normalized}
        \subidx{normalized}{increments}
        \subidx{programs}{tshurst}
        \subidx{tshurst}{program}
        \subidx{\market}{H parameter analysis}
        \subidx{H parameter}{analysis}
        \subidx{programs}{tshcalc}
        \subidx{tshcalc}{program}
        Figures~\ref{\SETLABEL:HC} and~\ref{\SETLABEL:HP} represent
        Hurst coefficient and H parameter data that are derived from
        the normalized increments, shown in
        Figure~\ref{\SETLABEL:TF}. In this case, the data is
        considered a normalized derivative of the time series data
        presented in Figure~\ref{\SETLABEL:TF}, instead of a
        cumulative sum.  The program, {\it tshurst}\/, is described
        briefly in appendix~\ref{programs}, and the data for
        figures~\ref{\SETLABEL:THC} and~\ref{\SETLABEL:THP} was made
        using the -d option.

        \begin{figure}[ht]
            \begin{center}
                \begin{minipage}[t]{0.45\textwidth}
                    \epsfxsize=1.0\linewidth
                    \epsffile{\directory/data.tsfraction.tshurst-d.eps}
                    \caption[{\market}, traditional Hurst coefficient
                        data]{{\market}, traditional Hurst coefficient
                        data for the time series data shown in
                        Figure~\ref{\SETLABEL:TS}.  The slope of the
                        graph is the Hurst coefficient, and is
                        {\hurstlow} for the near term, and
                        {\hurstall} for the far term.}
                    \label{\SETLABEL:THC}
                \end{minipage}
                \hfill
                \begin{minipage}[t]{0.45\textwidth}
                    \epsfxsize=1.0\linewidth
                    \epsffile{\directory/data.tsfraction.tshcalc-d.eps}
                    \caption[{\market}, traditional H parameter
                        data]{{\market}, traditional H parameter data
                        for the time series data shown in
                        Figure~\ref{\SETLABEL:TS} The slope of the
                        graph is the H parameter, and is {\hcalclow}
                        for the near term, and {\hcalcall} for the
                        far term.}
                    \label{\SETLABEL:THP}
                \end{minipage}
            \end{center}
        \end{figure}

% Local Variables:
% TeX-parse-self: t
% TeX-auto-save: t
% TeX-master: "fractal.tex"
% End:


        %
% -----------------------------------------------------------------------------
%
% A license is hereby granted to reproduce this software source code and
% to create executable versions from this source code for personal,
% non-commercial use.  The copyright notice included with the software
% must be maintained in all copies produced.
%
% THIS PROGRAM IS PROVIDED "AS IS". THE AUTHOR PROVIDES NO WARRANTIES
% WHATSOEVER, EXPRESSED OR IMPLIED, INCLUDING WARRANTIES OF
% MERCHANTABILITY, TITLE, OR FITNESS FOR ANY PARTICULAR PURPOSE.  THE
% AUTHOR DOES NOT WARRANT THAT USE OF THIS PROGRAM DOES NOT INFRINGE THE
% INTELLECTUAL PROPERTY RIGHTS OF ANY THIRD PARTY IN ANY COUNTRY.
%
% Copyright (c) 1994-2006, John Conover, All Rights Reserved.
%
% Comments and/or bug reports should be addressed to:
%
%     john@email.johncon.com (John Conover)
%
% -----------------------------------------------------------------------------
%
% Revision: \RCSRevision \\
% Revision Time: \RCSTime UMT \\
% Revision Date: \RCSDate \\
% Revision Id: \RCSId \\
% Revision File: \RCSLog \\
\RCS $Revision: 0.0 $
\RCS $Date: 2006/01/20 04:38:13 $
\RCS $Id: fiscal.tex,v 0.0 2006/01/20 04:38:13 john Exp $
% $Log: fiscal.tex,v $
% Revision 0.0  2006/01/20 04:38:13  john
% Initial version
%
%
    \subsection{Fixed Increment Approximation for Fiscal Strategy}
        \label{\SETLABEL:FS}

        \subidx{\market}{fiscal strategy}
        \subidx{markets}{analysis}
        \subidx{analysis}{markets}
        \subidx{strategy}{fiscal}
        \subidx{fiscal}{strategy}
        The data in this section is presented in tabular form in
        Section~\ref{\SETLABELREF:LR}. This section derives various
        values based on the ``average'' of the normalized increments
        presented in Figure~\ref{\SETLABEL:TFA}. These values are an
        approximation to a, probably, complex process with a
        distribution shown in Figure~\ref{\SETLABEL:TF}. These values
        will be used in a fixed increment Brownian fractal analysis
        and simulation of the {\market}, and may, or may not, provide
        adequate accuracy for projections.

        For an organization operating in the {\market}, the fiscal
        strategy, commensurate with the aggregate environment, can be
        derived as follows~\cite[pp. 128, pp
        151]{Schroeder},~\cite[pp. 450]{Reza},~\cite[pp. 270]{Pierce}:
        \vspace{0.15in}

        \subsubsection{Logarithmic Returns}
            \label{\SETLABEL:LR}

            \subidx{logarithmic}{returns}
            \subidx{returns}{logarithmic}
            \subidx{\market}{logarithmic returns}
            The logarithmic returns can be calculated by various
            means. Four will be presented here, for comparison.

            \subidx{programs}{tsnormal}
            \subidx{tsnormal}{program}
            \subidx{logarithmic}{returns}
            \subidx{returns}{logarithmic}
            The logarithmic returns, in bits, $bits$, as computed from
            the mean, by the program {\it tsnormal}\/, which is
            described in Chapter~\ref{programs}, and is presented in
            Figure~\ref{\SETLABEL:TF}, and Equation~\ref{abits} from
            Section~\ref{ereturns} in Chapter~\ref{general}:

            \begin{equation}
                bits = \frac{\ln \left({\datafractionmean} + 1\right)}{\ln \left(2\right)} = \datafractionmeanbits
            \end{equation}

            \subidx{programs}{tslsq}
            \subidx{tslsq}{program}
            \subidx{logarithmic}{returns}
            \subidx{returns}{logarithmic}
            \noindent By comparison, the logarithmic returns, in bits,
            $bits$, as computed from the constant in the least squares
            approximation, using the program {\it tslsq}\/, which is briefly
            described in Chapter~\ref{programs}, as presented in
            Figure~\ref{\SETLABEL:TF}, and Equation~\ref{abits} from
            Section~\ref{ereturns} in Chapter~\ref{general}:

            \begin{equation}
                bits = \frac{\ln \left({\datafractionconstant} + 1\right)}{\ln \left(2\right)} = \datafractionconstantbits
            \end{equation}

            Note that if the mean is not constant in
            Figure~\ref{\SETLABEL:TF}, this method will not provide
            accurate results.

            \subidx{programs}{tslsq}
            \subidx{tslsq}{program}
            \subidx{logarithmic}{returns}
            \subidx{returns}{logarithmic}
            \noindent And by yet another comparison, using the program
            {\it tslsq}\/, which is briefly described in
            Chapter~\ref{programs}, with the -e -p options, to provide
            a formula for the least squares exponential fit to the
            time series data set presented in
            Figure~\ref{\SETLABEL:TS}:

            \begin{equation}
                bits = {\datatslsqepbits}
            \end{equation}

            \subidx{programs}{tslogreturns}
            \subidx{tslogreturns}{program}
            \subidx{logarithmic}{returns}
            \subidx{returns}{logarithmic}
            \noindent And finally, by comparison, from the
            {\it tslogreturns}\/ program, which is briefly described
            in Chapter~\ref{programs}, with the -p option, to provide
            a formula for the logarithmic returns of the time series
            data set presented in Figure~\ref{\SETLABEL:TS}:

            \begin{equation}
                bits = {\logreturns}
            \end{equation}

        \subsubsection{Calculation of Shannon Probability}
            \label{\SETLABEL:SP}

            \subidx{\market}{Shannon probability}
            Ideally, all of the values presented in
            Section~\ref{\SETLABEL:LR} would be equal. Using the
            logarithmic returns provided by the {\it tslogreturns}\/
            program, to be consistent
            with~\cite[pp. 81]{Peters:CAOITCM}

            \subidx{programs}{tslogreturns}
            \subidx{tslogreturns}{program}
            \begin{equation}
                2^{{\logreturns}t}
            \end{equation}

            \noindent therefore:
            \begin{equation}
                C\left(p\right) = {\logreturns}
            \end{equation}
            \subidx{programs}{tsshannon}
            \subidx{tsshannon}{program}
            \subidx{Shannon}{probability}
            \subidx{probability}{Shannon}
            \noindent and, {\it tsshannon}\/ {\logreturns} gives:
            \begin{equation}
                \label{\SETLABEL:F0}
                C\left({\shannonlogreturns}\right) = {\logreturns}
            \end{equation}
            \noindent therefore:
            \begin{eqnarray}
                2^{C\left({\shannonlogreturns}\right)} & = & 2^{\logreturns}\\
                                                       & = & {\twologreturns}\\
                                                       & = & {\twologreturnshundred}\%
            \end{eqnarray}
            \noindent and:
            \begin{eqnarray}
                2p - 1 & = & \left(2 \cdot {\shannonlogreturns}\right) - 1\\
                       & = & {\twopone}\\
                       \label{\SETLABEL:F1}
                       & = & {\twoponehundred}\%
            \end{eqnarray}

            \subidx{\market}{fiscal strategy}
            \subidx{markets}{analysis}
            \subidx{analysis}{markets}
            \subidx{strategy}{fiscal}
            \subidx{fiscal}{strategy}
            \subidx{\market}{fiscal strategy}
            \subidx{\market}{growth rate}
            Presuming the simplified assumptions outlined in
            Section~\ref{assumptions}, the ``typical'' organization
            operating in the {\market} executes a long term fiscal
            strategy, commensurate with the aggregate environment,
            that is to invest, every {\timescale}, in sufficient
            additional resources and infrastructure, to increase the
            manufacturing of goods and services by {\twoponehundred}\%
            of its rate of revenue returns, (per {\timescale}.) As a
            conceptual model, the remaining {\hundredtwoponehundred}\%
            will be held in ``reserve'' with a
            {\shannonlogreturnshundred}\% chance of making twice the
            {\twoponehundred}\% back, (and a
            {\hundredshannonlogreturnshundred}\% chance of making
            0.0,) in one {\timescale}, on the average, for an average
            growth in its rate of revenue returns, (per {\timescale},)
            of {\twologreturnshundred}\%, or a doubling of its rate of
            revenue returns, (per {\timescale},) in
            {\oneoverlogreturns} {\timescale}s.

        \subsubsection{Example Fixed Increment Approximation Fiscal Strategies}

            \subidx{\market}{fiscal strategy}
            \subidx{markets}{analysis}
            \subidx{analysis}{markets}
            \subidx{strategy}{fiscal}
            \subidx{fiscal}{strategy}
            \subidx{\market}{fiscal strategy}
            \subidx{\market}{growth rate}
            \subidx{\market}{management metric}
            \idx{management metric}
            A possible metric on the effectiveness of long term fiscal
            management could possibly be that if an investment of
            {\twoponehundred}\% per {\timescale} of the rate of
            revenue returns, (per {\timescale},) is made in resources
            and infrastructure, then the rate of revenue returns would
            be expected to increase by {\twologreturnshundred}\%, per
            {\timescale}, on average.

            Note that the metrics presented in this section are
            representative of the {\market} as an aggregate whole, and
            may or may not be accurate representations for any
            particular participant in the environment. Of interest to
            the participants in the environment would be a similar
            analysis of each product or service rendered in the
            marketplace.

            \subidx{\market}{fiscal strategy}
            \subidx{markets}{analysis}
            \subidx{analysis}{markets}
            \subidx{strategy}{fiscal}
            \subidx{fiscal}{strategy}
            \subidx{\market}{fiscal strategy}
            As a simple illustrative example, a company operating in
            this environment might obtain a credit line from a bank
            that is equal to {\twoponehundred}\% of its rate of
            revenue returns, (per {\timescale},) to finance additional
            operations. In this simple scenario, the company would use
            its revenue base as collateral for the loan. Some
            {\timescale}s, depending on the {\market}'s environment,
            the company's rate of revenue returns exceeds what was
            borrowed from the bank, and the loan is repaid in
            full. Other {\timescale}s, the company must default, and
            the bank seizes a portion of the company's revenue base to
            pay the delinquent loan. However, on the average, the
            company will expand its rate of revenue returns at
            {\twologreturnshundred}\% per {\timescale}.

            \subidx{\market}{fiscal strategy}
            \subidx{markets}{analysis}
            \subidx{analysis}{markets}
            \subidx{strategy}{fiscal}
            \subidx{fiscal}{strategy}
            \subidx{\market}{fiscal strategy}
            As another simple example, a company re-invests
            {\twoponehundred}\% of its rate of revenue returns, (per
            {\timescale},) in development, marketing, sales, and
            distribution of new products.  Although some products will
            be successful and the return on the investment will exceed
            the {\twoponehundred}\% per {\timescale} investment,
            others will not. However, on the average, the company will
            expand it gross rate of revenue returns at
            {\twologreturnshundred}\% per {\timescale}.

            \subidx{\market}{fiscal strategy}
            \subidx{markets}{analysis}
            \subidx{analysis}{markets}
            \subidx{strategy}{fiscal}
            \subidx{fiscal}{strategy}
            \subidx{\market}{fiscal strategy}
            \subidx{\market}{product portfolio}
            \subidx{\market}{product diversity}
            \subidx{\market}{product mix}
            \subidx{\market}{optimum number of products}
            \idx{product portfolio}
            \idx{product diversity}
            \idx{optimum number of products}
            \idx{product mix}

            As an example of ``product portfolio'' management, suppose
            a company re-invests {\twoponehundred}\% of its rate of
            revenue returns, (per {\timescale},) in development,
            marketing, sales, and distribution of new products.
            Further suppose that the company has two products, and a
            fractal analysis of the individual product rate of revenue
            return time series indicates that one product has a
            Shannon probability of 0.65, and the other has a Shannon
            probability of 0.55. Then the percentage of re-investment
            in the first product would be $(2 \cdot 0.65 - 1) \cdot
            {\twoponehundred}$, percent of the rate of revenue
            returns, and $(2 \cdot 0.55 - 1) \cdot {\twoponehundred}$
            percent for the second product, implying that the company
            should diversify its product line\footnote{The astute
            reader would note that the linear addition was used to add
            the contribution to development of each product. This is a
            ``near term'' interpretation. Actually, in general, the
            method used should be a root mean square process,
            dependent on the Hurst Coefficient, $H$, where
            $P_{total}^H = P_1^H + P_2^H + \cdots$, where $P_n$ is the
            contribution to each individual product. For a Brownian
            motion, or random walk type of fractal the Hurst
            Coefficient is a function of time into the future. For the
            ``near term,'' the Hurst coefficient is very near unity,
            meaning the summation process is linear. For the ``long
            term,'' $H \approx 0.5$, or a standard root mean square
            summation process should be used. If $H$ is $0.5$ then the
            market is termed a Brownian motion, or random walk
            process. If it is larger than 0.5, it is termed fractional
            Brownian motion process. For a random walk process, ``near
            term'' and ``far term'' are quantitatively differentiated
            on the Hurst Coefficient graph where $1 - \ln (t) = 0.5
            \cdot \ln (t)$, or when $\ln (t) = 2$, or $t =
            7.389\ldots$ See~\cite[pp. 67, 83-84]{Peters:CAOITCM}
            and~\cite[pp. 129, 159]{Schroeder} for particulars on the
            implications of the Hurst Coefficient and root mean square
            summation issues.}.  Note that this is a ``bet hedging''
            metric methodology, and assumes that the products have
            uncorrelated revenue return rates. If this re-investment
            methodology is not feasible, perhaps for strategic
            financial reasons, then the re-investment in both products
            should total the ${\twoponehundred}$\%, and the investment
            in each product should be made at a ratio of $\frac{(2
            \cdot 0.65 - 1)}{(2 \cdot 0.55 - 1)} = 3 : 1$,
            respectively. Note that this ``bet hedging'' can be used
            to define the optimal number of products that can be
            supported on the rate of revenue returns. If it assumed
            that all products are ``typical'' for the {\market}, as a
            standard bench mark, then the optimal number will be
            $\frac{1}{{\twopone}}$. Note that this is a
            ``theoretical'' value, since not all products are
            ``typical,'' and there may be strategic reasons, for
            example product leveraging, that may increase the number
            of products above the optimum. However, most of the
            revenue should come from the optimal number of products,
            since having more products will decrease the amount of the
            potential investment in each product, and having less than
            the optimum number of products will increase the risk that
            many of the products could suffer a ``down market''
            concurrently, impacting the rate of revenue returns.  As
            another interesting interpretation of the optimal
            ``hedging of bets,'' in product portfolio strategy, and
            considering the graph of the normalized increments
            presented in Figure~\ref{\SETLABEL:TF}, if the
            organization is running optimally, then these products
            will generate, at least in principle, one standard
            deviation, approximately $0.8413 = 84.13$\% of the future
            growth in rate of revenue returns. Naturally, these are
            approximations, and the values are an approximation to a,
            probably, complex process, and appropriate scrutiny should
            be exercised before making specific projections.  As yet
            another example of ``product portfolio'' management,
            consider the issue of product mix. In this interpretation,
            {\twoponehundred}\% of the product manufactured should be
            ``proprietary,'' while the rest is ``industry standard.''
            As yet another possibility, {\twoponehundred}\% of the
            product manufactured should be predatory into new markets,
            and the remainder in markets that are ``traditional'' for
            the company.

% Local Variables:
% TeX-parse-self: t
% TeX-auto-save: t
% TeX-master: "fractal.tex"
% End:


        %
% -----------------------------------------------------------------------------
%
% A license is hereby granted to reproduce this software source code and
% to create executable versions from this source code for personal,
% non-commercial use.  The copyright notice included with the software
% must be maintained in all copies produced.
%
% THIS PROGRAM IS PROVIDED "AS IS". THE AUTHOR PROVIDES NO WARRANTIES
% WHATSOEVER, EXPRESSED OR IMPLIED, INCLUDING WARRANTIES OF
% MERCHANTABILITY, TITLE, OR FITNESS FOR ANY PARTICULAR PURPOSE.  THE
% AUTHOR DOES NOT WARRANT THAT USE OF THIS PROGRAM DOES NOT INFRINGE THE
% INTELLECTUAL PROPERTY RIGHTS OF ANY THIRD PARTY IN ANY COUNTRY.
%
% Copyright (c) 1994-2006, John Conover, All Rights Reserved.
%
% Comments and/or bug reports should be addressed to:
%
%     john@email.johncon.com (John Conover)
%
% -----------------------------------------------------------------------------
%
% Revision: \RCSRevision \\
% Revision Time: \RCSTime UMT \\
% Revision Date: \RCSDate \\
% Revision Id: \RCSId \\
% Revision File: \RCSLog \\
\RCS $Revision: 0.0 $
\RCS $Date: 2006/01/20 04:38:13 $
\RCS $Id: companies.tex,v 0.0 2006/01/20 04:38:13 john Exp $
% $Log: companies.tex,v $
% Revision 0.0  2006/01/20 04:38:13  john
% Initial version
%
%
    \subsection{Number of Companies}
        \label{\SETLABEL:QNC}

        \subidx{\market}{number of companies}
        \subidx{number of companies}{analysis}
        \subidx{analysis}{number of companies}
        \subidx{Shannon}{probability}
        \subidx{probability}{Shannon}
        This section evaluates the approximate, or ``average,'' number
        of companies in the {\market}, and uses the method outlined in
        Chapter~\ref{general}, Section~\ref{aftsma}. Since the
        average, $avg_{ind}$, and the root mean square, $rms_{ind}$,
        of the normalized increments of the {\market} time series is
        \datafractionmean, and \datafractionrms respectively, the
        number of companies participating in the market can be
        calculated by Equation~\ref{ncompanies} to be {\ncompanies}.

        If this value seems consistent number of companies in the
        {\market}, within the assumptions outlined in
        Chapter~\ref{general}, Section~\ref{aftsma}, then it would
        seem that there is some circumstantial or indirect evidence
        that the companies participating in the {\market} are
        operating optimally, and the ``average'' Shannon probability,
        $P$ for each participating company would be, using
        Equation~\ref{pncompanies}, {\pncompanies}, which would be the
        value which should be used in Section~\ref{\SETLABEL:FS} for
        each participating company if market expansion was to be
        consistent with the rest of the industry. However, if the
        Shannon probability derived in Section~\ref{\SETLABEL:FS} is
        greater than the average Shannon probability for the companies
        participating in the {\market}, as derived in this section,
        then the market would, possibly, be exploitable with the
        fiscal strategy outlined in Section~\ref{\SETLABEL:FS}. The
        maximum exploitability for the {\market} is derived in
        Section~\ref{\SETLABEL:MAXSHANNON}, but it is probably of
        doubtful practicality.

        Note that these optimizations would maximize a company's
        market growth. Since there are probably many companies
        competing in the market place, this would not necessarily
        maximize a company's P\&L, as described in
        Chapter~\ref{general}, Section~\ref{ompl}. The Shannon
        probability that maximizes market share in the {\market} is
        \pncompanies, with several alternative solutions listed in the
        previous paragraph. However, these should be contrasted to the
        Shannon probability that maximizes a company's P\&L which is
        \avgrms~in the {\market}. In all cases, the fraction of the
        P\&L that should be ``wagered'' on the future, $f$, should be:

        \begin{equation}
            f = 2P - 1
        \end{equation}

        \noindent where $P$ is the particular Shannon probability
        chosen optimize a particular fiscal strategy. Interestingly,
        the measured Shannon probability of the {\market} would tend
        to indicate that the companies participating in the market
        have chosen a fiscal strategy that optimizes market growth, as
        opposed to capital growth.

        \subidx{\market}{increasing returns}
        \subidx{economic increasing returns}{\market}
        As interesting interpretation of these exploitive issues,
        since all three fiscal strategies will result in exponential
        market growth for every company participating in the market,
        is that they may represent, perhaps, an example of
        ``increasing returns.''

% Local Variables:
% TeX-parse-self: t
% TeX-auto-save: t
% TeX-master: "fractal.tex"
% End:


        %
% -----------------------------------------------------------------------------
%
% A license is hereby granted to reproduce this software source code and
% to create executable versions from this source code for personal,
% non-commercial use.  The copyright notice included with the software
% must be maintained in all copies produced.
%
% THIS PROGRAM IS PROVIDED "AS IS". THE AUTHOR PROVIDES NO WARRANTIES
% WHATSOEVER, EXPRESSED OR IMPLIED, INCLUDING WARRANTIES OF
% MERCHANTABILITY, TITLE, OR FITNESS FOR ANY PARTICULAR PURPOSE.  THE
% AUTHOR DOES NOT WARRANT THAT USE OF THIS PROGRAM DOES NOT INFRINGE THE
% INTELLECTUAL PROPERTY RIGHTS OF ANY THIRD PARTY IN ANY COUNTRY.
%
% Copyright (c) 1994-2006, John Conover, All Rights Reserved.
%
% Comments and/or bug reports should be addressed to:
%
%     john@email.johncon.com (John Conover)
%
% -----------------------------------------------------------------------------
%
% Revision: \RCSRevision \\
% Revision Time: \RCSTime UMT \\
% Revision Date: \RCSDate \\
% Revision Id: \RCSId \\
% Revision File: \RCSLog \\
\RCS $Revision: 0.0 $
\RCS $Date: 2006/01/20 04:38:13 $
\RCS $Id: operations.tex,v 0.0 2006/01/20 04:38:13 john Exp $
% $Log: operations.tex,v $
% Revision 0.0  2006/01/20 04:38:13  john
% Initial version
%
%
    \subsection{Fixed Increment Approximation for Operational Strategy}
        \label{\SETLABEL:OPS}.

        This section derives various values based on the ``average''
        of the normalized increments presented in
        Figure~\ref{\SETLABEL:TFA}. These values are an approximation
        to a, probably, complex process with a distribution shown in
        Figure~\ref{\SETLABEL:TF}. These values will be used in a
        fixed increment Brownian fractal analysis and simulation of
        the {\market}, and may, or may not, provide adequate accuracy
        for projections.

        \subidx{\market}{fiscal strategy}
        \subidx{\market}{Shannon probability}
        \subidx{strategy}{fiscal}
        \subidx{fiscal}{strategy}
        \subidx{Shannon}{probability}
        \subidx{probability}{Shannon}
        It should be noted that the analysis of fiscal strategy,
        presented in Section~\ref{\SETLABEL:FS}, is derived from the
        {\market} metrics and may, or may not, be maximally
        optimal. For the optimal fiscal strategy, which may be
        exploitable, see Section~\ref{\SETLABEL:MAXSHANNON}.

        \subidx{strategy}{exploitable}
        \subidx{exploitable}{strategy}
        \subidx{\market}{windows of opportunity}
        \idx{windows of opportunity}
        \subidx{decision}{obsolete}
        \subidx{obsolete}{decision}
        \subidx{decision}{timeliness}
        \subidx{timeliness}{decision}
        \subidx{rate of revenue returns}{forecast}
        \subidx{forecast}{rate of revenue returns}
        An additional exploitable strategy may be time itself.
        Equations~\ref{\SETLABEL:V},~\ref{\SETLABEL:R},
        and,~\ref{\SETLABEL:MA}, are, essentially, metrics on how fast
        a decision, which is based on information concerning the
        current status of the {\market}, becomes obsolete. Obviously,
        how long a decision is expected to remain relevant should be
        addressed as an operational necessity in strategic planning
        and project management. Figures~\ref{\SETLABEL:FN},
        and,~\ref{\SETLABEL:FF} compare methods of approximation of
        the ``forecastability'' of rate of revenue returns in the
        {\market} for the near term and far
        term~\cite[pp. 83-84]{Peters:CAOITCM}, respectively. As a
        general rule, caution must be exercised when making decisions
        that will span a time interval larger than the time interval
        where the ``forecastability'' of rate of revenue returns drops
        below 50\%. Beyond this time interval, the chances increase
        that the competitive and market forces will alter the market
        environment in a possibly detrimental unanticipated
        fashion. Obviously, there is significant advantage in
        ``timeliness'' of development, manufacturing, and distribution
        of products and services that are consistent with this
        temporal agenda. Automation of these processes, if executed
        consistently with this agenda, should be considered a
        competitive advantage.

        \subidx{strategy}{exploitable}
        \subidx{exploitable}{strategy}
        \subidx{rate of revenue returns}{forecast}
        \subidx{forecast}{rate of revenue returns}
        \idx{product life cycle}
        \idx{life cycle, product}
        In some sense, this temporal agenda defines the ``average''
        product or service life cycle in the {\market}. When the
        ``forecastability'' of rate of revenue returns drops below
        50\%, there is an even chance that the rate of revenue returns
        for the product or service will change in a detrimental
        fashion. If it is assumed that a product or service life cycle
        consists of a ramp up, a maintenence interval, and a ramp
        down, then, if all three life cycle intervals are equal, the
        product life cycle will be, approximately, three times the
        time interval where the ``forecastability'' of rate of revenue
        returns drops below 50\%. Although probably not an accurate
        prediction of product or service life cycle, the technique may
        be used as a conceptual approximation to the dynamics of
        ``market windows.\footnote{For example, consider the market
        for table salt. Since it has inelastic supply and demand
        curves, and is a necessary requirement for life, it would be
        expected that the Hurst coefficient would be very near
        unity---ignoring competitive pressures in the market. The
        predictability of the table salt market would, therefore, be
        expected to be relatively good, over time.}''  The conceptual
        approximation will probably predict a ``conservative'' or
        ``pessimistic'' value in relation to actual markets.

        \begin{figure}[ht]
            \begin{center}
                \begin{minipage}[t]{0.45\textwidth}
                    \epsfxsize=1.0\linewidth
                    \epsffile{\directory/datahurstlownear.eps}
                    \caption[{\market}, ``forecastability'' of near
                        term rate of revenue returns]{{\market},
                        ``forecastability'' of near term rate of
                        revenue returns. Although the error function
                        is the most accurate, for the near term,
                        $H^{t} = \thurstlow^{t}$ may be used as a
                        reliable metric of ``forecastability'' of the
                        rate of revenue returns.}
                    \label{\SETLABEL:FN}
                \end{minipage}
                \hfill
                \begin{minipage}[t]{0.45\textwidth}
                    \epsfxsize=1.0\linewidth
                    \epsffile{\directory/datahurstlowfar.eps}
                    \caption[{\market}, ``forecastability'' of far
                        term rate of revenue returns]{{\market},
                        ``forecastability'' of far term rate of
                        revenue returns. Although the error function
                        is the most accurate, for the far term,
                        $\frac{1}{\sqrt{t}}$ may be used as a reliable
                        metric of ``forecastability'' of the rate of
                        revenue returns.}
                    \label{\SETLABEL:FF}
                \end{minipage}
            \end{center}
        \end{figure}

        \idx{operations research}
        As an interesting interpretation of the data presented in
        Figure~\ref{\SETLABEL:FN}, there may be, perhaps, some
        applicability to such operational agendas as inventory
        control. Maintaining too little inventory, obviously, will
        create a situation where the organization can not exploit
        market expansion, and maintaining too much inventory,
        likewise, would over extend the company, creating unnecessary
        losses when the market contracts. The company should maintain
        inventory levels that do not exceed, from
        Equation~\ref{\SETLABEL:MA}, ${\thurstlow}^{n} = 0.5$
        {\timescale}s of operations. Since the optimal amount of
        inventory and, from Equation~\ref{\SETLABEL:V}, the variance
        of change in the rate of revenue returns in the future can be
        calculated, there may, perhaps, be some applicability to a
        forecasting methodology that can be incorporated into other
        areas of operations research, for example the linear algebras
        using simplex methodologies for optimization of manufacturing
        processes. Traditionally, these forecasts are made by the
        sales department, and are subject to various subjective
        biases.

% Local Variables:
% TeX-parse-self: t
% TeX-auto-save: t
% TeX-master: "fractal.tex"
% End:


        %
% -----------------------------------------------------------------------------
%
% A license is hereby granted to reproduce this software source code and
% to create executable versions from this source code for personal,
% non-commercial use.  The copyright notice included with the software
% must be maintained in all copies produced.
%
% THIS PROGRAM IS PROVIDED "AS IS". THE AUTHOR PROVIDES NO WARRANTIES
% WHATSOEVER, EXPRESSED OR IMPLIED, INCLUDING WARRANTIES OF
% MERCHANTABILITY, TITLE, OR FITNESS FOR ANY PARTICULAR PURPOSE.  THE
% AUTHOR DOES NOT WARRANT THAT USE OF THIS PROGRAM DOES NOT INFRINGE THE
% INTELLECTUAL PROPERTY RIGHTS OF ANY THIRD PARTY IN ANY COUNTRY.
%
% Copyright (c) 1994-2006, John Conover, All Rights Reserved.
%
% Comments and/or bug reports should be addressed to:
%
%     john@email.johncon.com (John Conover)
%
% -----------------------------------------------------------------------------
%
% Revision: \RCSRevision \\
% Revision Time: \RCSTime UMT \\
% Revision Date: \RCSDate \\
% Revision Id: \RCSId \\
% Revision File: \RCSLog \\
\RCS $Revision: 0.0 $
\RCS $Date: 2006/01/20 04:38:13 $
\RCS $Id: simulation.tex,v 0.0 2006/01/20 04:38:13 john Exp $
% $Log: simulation.tex,v $
% Revision 0.0  2006/01/20 04:38:13  john
% Initial version
%
%
    \subsection{Simulation of Fixed Increment Approximation for Fiscal Strategy}
        \label{\SETLABEL:TSUNFAIRBROWNIAN}

        \subidx{\market}{market simulation}
        The data in this section is presented in tabular form in
        Section~\ref{\SETLABELREF:SIM}.
        Figure~\ref{\SETLABEL:TSUNFAIRBROWNIAN0} represents a
        constructional simulation of the time series data presented in
        Figure~\ref{\SETLABEL:TS}. The program {\it
        tsunfairbrownian}\/, which is briefly described in
        appendix~\ref{programs}, was used in the reconstruction. The
        reconstructed data is superimposed on the original time series
        data.  The program, {\it tsunfairbrownian}\/, essentially,
        constructs the new time series as a Brownian fractal with
        fixed increments---the value of the fixed increment is derived
        from the root mean square average of the normalized increments
        presented in Figure~\ref{\SETLABEL:TF}. The ``quality'' of
        such a reconstruction should be subject to adequate scepticism
        and scrutiny since, in all probability, the normalized
        increments presented in Figure~\ref{\SETLABEL:TF} represent a
        relatively complex process, that may not be ``modeled'' with
        such a simple methodology.

        As a further comparison of the the constructional simulation
        with the original time series data,
        Figure~\ref{\SETLABEL:TSUNFAIRBROWNIAN1} presents a normalized
        histogram of the normalized increments of the reconstructed
        time series, superimposed on the normalized histogram
        presented in Figure~\ref{\SETLABEL:NH}.

        \subidx{\market}{fiscal strategy, simulation}
        \subidx{markets}{simulation}
        \subidx{simulation}{markets}
        \subidx{strategy}{fiscal, simulation}
        \subidx{fiscal}{strategy, simulation}
        \subidx{programs}{tsunfairbrownian}
        \subidx{tsunfairbrownian}{program}
        \begin{figure}[ht]
            \begin{center}
                \begin{minipage}[t]{0.45\textwidth}
                    \epsfxsize=1.0\linewidth
                    \epsffile{\directory/tsunfairbrownian-f.eps}
                    \caption[{\market}, Time series data, empirical and
                        simulated]{{\market}, Time series data, empirical
                        and simulated, using the program {\it tsunfairbrownian}\/
                        with f = {\datafractionrms}. This data is
                        superimposed on the data presented in
                        Figure~\ref{\SETLABEL:TS}.}
                    \label{\SETLABEL:TSUNFAIRBROWNIAN0}
                \end{minipage}
                \hfill
                \begin{minipage}[t]{0.45\textwidth}
                    \epsfxsize=1.0\linewidth
                    \epsffile{\directory/tsunfairbrownian-f.tsfraction.tsnormal-s30.eps}
                    \caption[{\market}, normalized histogram,
                        empirical and simulated]{{\market}, normalized
                        histogram of the normalized increments of the
                        time series data shown in
                        Figure~\ref{\SETLABEL:TSUNFAIRBROWNIAN0},
                        empirical and simulated.  The empirical data
                        has a mean of {\datafractionmean}, with a
                        standard deviation of {\datafractionstddev}.
                        By comparison, the simulated data has a mean
                        of {\tsunfairbrownianfractionmean} with a
                        standard deviation of
                        {\tsunfairbrownianfractionstddev}. This data
                        is superimposed on the data presented in
                        Figure~\ref{\SETLABEL:NH}. The area under the
                        four curves is identical.}
                    \label{\SETLABEL:TSUNFAIRBROWNIAN1}
                \end{minipage}
            \end{center}
        \end{figure}

% Local Variables:
% TeX-parse-self: t
% TeX-auto-save: t
% TeX-master: "fractal.tex"
% End:


        %
% -----------------------------------------------------------------------------
%
% A license is hereby granted to reproduce this software source code and
% to create executable versions from this source code for personal,
% non-commercial use.  The copyright notice included with the software
% must be maintained in all copies produced.
%
% THIS PROGRAM IS PROVIDED "AS IS". THE AUTHOR PROVIDES NO WARRANTIES
% WHATSOEVER, EXPRESSED OR IMPLIED, INCLUDING WARRANTIES OF
% MERCHANTABILITY, TITLE, OR FITNESS FOR ANY PARTICULAR PURPOSE.  THE
% AUTHOR DOES NOT WARRANT THAT USE OF THIS PROGRAM DOES NOT INFRINGE THE
% INTELLECTUAL PROPERTY RIGHTS OF ANY THIRD PARTY IN ANY COUNTRY.
%
% Copyright (c) 1994-2006, John Conover, All Rights Reserved.
%
% Comments and/or bug reports should be addressed to:
%
%     john@email.johncon.com (John Conover)
%
% -----------------------------------------------------------------------------
%
% Revision: \RCSRevision \\
% Revision Time: \RCSTime UMT \\
% Revision Date: \RCSDate \\
% Revision Id: \RCSId \\
% Revision File: \RCSLog \\
\RCS $Revision: 0.0 $
\RCS $Date: 2006/01/20 04:38:13 $
\RCS $Id: maximum.tex,v 0.0 2006/01/20 04:38:13 john Exp $
% $Log: maximum.tex,v $
% Revision 0.0  2006/01/20 04:38:13  john
% Initial version
%
%
    \subsection{Simulation of Fixed Increment Approximation for Optimally Maximal Fiscal Strategy}
        \label{\SETLABEL:MAXSHANNON}
        \subidx{\market}{fiscal strategy, simulation}
        \subidx{\market}{maximum Shannon probability}
        \subidx{markets}{simulation}
        \subidx{simulation}{markets}
        \subidx{strategy}{optimum fiscal, simulation}
        \subidx{fiscal}{optimum strategy, simulation}
        \subidx{programs}{tsunfairbrownian}
        \subidx{tsunfairbrownian}{program}
        \subidx{Shannon}{probability}
        \subidx{probability}{Shannon}

        \subidx{strategy}{exploitable}
        \subidx{exploitable}{strategy}
        \subidx{programs}{tsshannonmax}
        \subidx{tsshannonmax}{program}
        \subidx{programs}{tsunfairbrownian}
        \subidx{tsunfairbrownian}{program}
        \subidx{strategy}{fiscal}
        \subidx{fiscal}{strategy}
        The data in this section is presented in tabular form in
        Section~\ref{\SETLABELREF:MAXSHANNON}. One of the issues of
        analysis, as mentioned in Section~\ref{\SETLABEL:OPS}, is to
        determine the maximum Shannon probability for the time series
        presented in Figure~\ref{\SETLABEL:TS}. Potentially, this
        could be exploited with an aggressive fiscal
        strategy. Figure~\ref{\SETLABEL:SHANNONMAX0} is a graph of the
        output of the {\it tsshannonmax}\/ program, which is described
        briefly in appendix~\ref{programs}. The maximum of this
        function is the maximum Shannon probability for the time
        series data presented in Figure~\ref{\SETLABEL:TS}.
        Figure~\ref{\SETLABEL:SHANNONMAX1} was constructed using {\it
        tsunfairbrownian}\/ program, which is also described in
        appendix~\ref{programs}, with the maximum Shannon probability,
        and the time series data presented in
        Figure~\ref{\SETLABEL:TS}. This represents a ``what if'' the
        investment strategy was changed from a Shannon probability of
        {\shannonlogreturns}, as derived in Section~\ref{\SETLABEL:SP}
        to {\shannonmax}. This process, essentially, extracts the
        random statistical data from the time series presented in
        Figure~\ref{\SETLABEL:TS}, and constructs a new time series,
        using the random statistical data, with a different investment
        strategy.  The program, {\it tsunfairbrownian}\/, essentially,
        constructs the new time series as a Brownian fractal with
        fixed increments.  The ``quality'' of such a reconstruction
        should be subject to adequate scepticism and scrutiny since,
        in all probability, the increments in the original data
        represent a relatively complex process, that may not be
        ``modeled'' with such a simple methodology.

        \begin{figure}[ht]
            \begin{center}
                \begin{minipage}[t]{0.45\textwidth}
                    \epsfxsize=1.0\linewidth
                    \epsffile{\directory/data.tsshannonmax.eps}
                    \caption[{\market}, maximum rate of revenue
                        returns] {{\market}, maximum rate of revenue
                        returns, per {\timescale}, vs. Shannon
                        probability. The maximum rate of revenue
                        returns, per {\timescale}, occurs at a Shannon
                        probability of {\shannonmax}.}
                    \label{\SETLABEL:SHANNONMAX0}
                \end{minipage}
                \hfill
                \begin{minipage}[t]{0.45\textwidth}
                    \epsfxsize=1.0\linewidth
                    \epsffile{\directory/data.tsshannonmax-p.tsunfairbrownian-p.eps}
                    \caption[{\market}, maximum rate of revenue
                        returns] {{\market}, maximum rate of revenue
                        returns, per {\timescale}, at a Shannon
                        probability, of {\shannonmax}, corresponding
                        to a ``wager'' fraction of {\twoponemax}.}
                    \label{\SETLABEL:SHANNONMAX1}
                \end{minipage}
            \end{center}
        \end{figure}

        \subidx{fractional}{Brownian motion}
        \subidx{Brownian motion}{fractional}
        \subidx{Shannon}{probability}
        \subidx{probability}{Shannon}
        \subidx{programs}{tsshannonmax}
        \subidx{tsshannonmax}{program}
        If it is assumed that the time series data set, presented in
        Figure~\ref{\SETLABEL:TS}, constitutes classical Brownian
        motion, then the Shannon probability can be calculated by
        counting the total number of {\timescale}s that the {\market}
        movement was positive, and dividing by the total number of
        {timescale}s represented in the time series. This quotient is
        {\pmax}, as compared with the predicted value from the program
        {\it tsshannonmax}\/ of {\shannonmax}.

% Local Variables:
% TeX-parse-self: t
% TeX-auto-save: t
% TeX-master: "fractal.tex"
% End:


        %
% -----------------------------------------------------------------------------
%
% A license is hereby granted to reproduce this software source code and
% to create executable versions from this source code for personal,
% non-commercial use.  The copyright notice included with the software
% must be maintained in all copies produced.
%
% THIS PROGRAM IS PROVIDED "AS IS". THE AUTHOR PROVIDES NO WARRANTIES
% WHATSOEVER, EXPRESSED OR IMPLIED, INCLUDING WARRANTIES OF
% MERCHANTABILITY, TITLE, OR FITNESS FOR ANY PARTICULAR PURPOSE.  THE
% AUTHOR DOES NOT WARRANT THAT USE OF THIS PROGRAM DOES NOT INFRINGE THE
% INTELLECTUAL PROPERTY RIGHTS OF ANY THIRD PARTY IN ANY COUNTRY.
%
% Copyright (c) 1994-2006, John Conover, All Rights Reserved.
%
% Comments and/or bug reports should be addressed to:
%
%     john@email.johncon.com (John Conover)
%
% -----------------------------------------------------------------------------
%
% Revision: \RCSRevision \\
% Revision Time: \RCSTime UMT \\
% Revision Date: \RCSDate \\
% Revision Id: \RCSId \\
% Revision File: \RCSLog \\
\RCS $Revision: 0.0 $
\RCS $Date: 2006/01/20 04:38:13 $
\RCS $Id: verification.tex,v 0.0 2006/01/20 04:38:13 john Exp $
% $Log: verification.tex,v $
% Revision 0.0  2006/01/20 04:38:13  john
% Initial version
%
%
    \subsection{Qualitative Verification of Fixed Increment Approximation Analysis}
        \label{\SETLABEL:QVA}

        \subidx{\market}{verification of analysis}
        \subidx{verification}{analysis}
        \subidx{analysis}{verification}
        \subidx{quality}{of analysis}
        \subidx{verification}{of methodology}
        \subidx{methodology}{verification of}
        \subidx{Shannon}{probability}
        \subidx{probability}{Shannon}

        This section evaluates various values based on the ``average''
        of the normalized increments presented in
        Figure~\ref{\SETLABEL:TFA}. These values are an approximation
        to a, probably, complex process with a distribution shown in
        Figure~\ref{\SETLABEL:TF}. These values will be used in a
        fixed increment Brownian fractal analysis of the {\market},
        and may, or may not, provide adequate accuracy for
        projections.

        The data in this section is presented in tabular form in
        sections~\ref{\SETLABELREF:VI1} and~\ref{\SETLABELREF:VI2}.
        As a subjective evaluation of the ``quality'' of the analysis
        of the {\market}, from Chapter~\ref{methodology},
        Equation~\ref{metricvalues1}, and using the mean and root mean
        square values of the normalized increments of the time series
        data presented in Figure~\ref{\SETLABEL:TS} from
        Figure~\ref{\SETLABEL:TF}, and the Shannon probability as
        calculated by counting the total number of {\timescale}s that
        the {\market} movement was positive, as presented in
        Section~\ref{\SETLABEL:MAXSHANNON}:

        \begin{eqnarray}
                  P & \approx & \frac{\frac{avg}{rms} + 1}{2}\\
            {\pmax} & \approx & \frac{\frac{\datafractionmean}{\datafractionrms} + 1}{2}\\
            {\pmax} & \approx & {\avgrms}
            \label{\SETLABEL:AVGS}
        \end{eqnarray}

        \subidx{Shannon}{probability}
        \subidx{probability}{Shannon}
        \noindent and comparing these values to the Shannon
        probability, as found by the {\it tsshannonmax}\/ program, which
        iterates for a maximum:

        \begin{eqnarray}
            {\pmax} \approx {\avgrms} \approx {\shannonmax}
        \end{eqnarray}

        \subidx{logarithmic}{returns}
        \subidx{returns}{logarithmic}
        In addition, the different methods of calculating the
        logarithmic returns, presented in Section~\ref{\SETLABEL:FS},
        should be compared. The four methods used were the mean of
        Figure~\ref{\SETLABEL:TF}, the constant in the least squares
        approximation to Figure~\ref{\SETLABEL:TF}, the least squares
        exponential approximation to Figure~\ref{\SETLABEL:TS}, and
        the logarithmic returns of Figure~\ref{\SETLABEL:TS}, derived
        as the mean of the logarithms of the quotients of the
        increments. The values for each of the methods are,
        respectively:

        \begin{equation}
            \datafractionmeanbits \approx \datafractionconstantbits \approx \datatslsqepbits \approx \logreturns
        \end{equation}

        It is implied in Section~\ref{\SETLABEL:FS},
        Subsection~\ref{\SETLABEL:SP} and in
        Section~\ref{\SETLABEL:TSUNFAIRBROWNIAN} that, a Brownian
        motion with fixed increments fractal may ``model'' the
        {\market}. Using Equation~\ref{stddev9} from
        Chapter~\ref{general}, Section~\ref{abmfi}:

        \begin{eqnarray}
                                    rms \left(2P - 1\right) & \approx & \frac{\sigma \left(2P - 1\right)}{2 \sqrt{P\left(1 - P\right)}}\\
            \datafractionrms \left(2 \cdot \pmax - 1\right) & \approx & \frac{\datafractionstddev \left(2 \cdot \pmax - 1\right)}{2\sqrt{\pmax \left(1 - \pmax\right)}}\\
                       \datafractionrms \cdot \twopminusone & \approx & \datafractionstddev \cdot \twopx\\
                                                      \rmsp & \approx & \sigmap
        \end{eqnarray}

        \noindent and, equating to the mean:

        \begin{equation}
            \datafractionmean \approx \rmsp \approx \sigmap
        \end{equation}

        \subidx{Shannon}{probability}
        \subidx{probability}{Shannon}
        \noindent where, as in Equation~\ref{\SETLABEL:AVGS} using the
        mean, root mean square, and standard deviation values of the
        normalized increments of the time series data presented in
        Figure~\ref{\SETLABEL:TS} from Figure~\ref{\SETLABEL:TF}, and
        the Shannon probability as calculated by counting the total
        number of {\timescale}s that the {\market} movement was
        positive, as presented in Section~\ref{\SETLABEL:MAXSHANNON}.

        As a final qualitative comparison, the absolute value of the
        normalized increments should be the same as the root mean
        square value\footnote{The absolute value of the normalized
        increments, when averaged, is related to the root mean square
        of the increments by a constant. If the normalized increments
        are a fixed increment, the constant is unity. If the
        normalized increments have a Gaussian distribution, the
        constant is $\approx 0.8$ depending on the accuracy of of
        ``fit'' to a Gaussian distribution.}, where the absolute value
        is presented in Figure~\ref{\SETLABEL:TFA}, and the root mean
        square value is presented in Figure~\ref{\SETLABEL:TF}:

        \begin{equation}
            \datafractionabsmean \approx \datafractionrms
        \end{equation}

        Note, that if the {\market} could be ``modeled'' as a Brownian
        motion with fixed increments fractal, then the standard
        deviation of the absolute value of the normalized increments
        of the time series data presented in Figure~\ref{\SETLABEL:TS}
        from Figure~\ref{\SETLABEL:TF} should be zero. It is
        $\datafractionabsstddev$.

% Local Variables:
% TeX-parse-self: t
% TeX-auto-save: t
% TeX-master: "fractal.tex"
% End:


    \renewcommand{\market}{United States Treasury Bill Returns}
    \renewcommand{\directory}{../markets/us.tbill}
    \renewcommand{\datafractionmean}{0.008052}
\renewcommand{\datafractionmeanbits}{0.011570}
\renewcommand{\datafractionmeanq}{0.002684}
\renewcommand{\datafractionmeanbitsq}{0.003867}
\renewcommand{\datafractionstddev}{0.038579}
\renewcommand{\datafractionrms}{0.039311}
\renewcommand{\avgrms}{0.602414}
\renewcommand{\ncompanies}{5.210454}
\renewcommand{\pncompanies}{0.544866}
\renewcommand{\datafractionabsmean}{0.029745}
\renewcommand{\datafractionabsstddev}{0.025769}
\renewcommand{\datafractionconstant}{0.010041}
\renewcommand{\datafractionconstantbits}{0.014414}
\renewcommand{\datafractionconstantq}{0.003347}
\renewcommand{\datafractionconstantbitsq}{0.004821}
\renewcommand{\datafractionslope}{-0.000021}
\renewcommand{\datafractionabsconstant}{0.035145}
\renewcommand{\datafractionabsslope}{-0.000057}
\renewcommand{\hurstall}{0.659558}
\renewcommand{\hurstlow}{0.707509}
\renewcommand{\hurstlowtwo}{1.415018}
\renewcommand{\hurstlowhundred}{70.750900}
\renewcommand{\hcalcall}{0.184942}
\renewcommand{\hcalclow}{0.102042}
\renewcommand{\shannonmax}{0.604167}
\renewcommand{\twoponemax}{0.208334}
\renewcommand{\logreturns}{0.010456}
\renewcommand{\twologreturns}{1.007274}
\renewcommand{\twologreturnshundred}{0.727387}
\renewcommand{\oneoverlogreturns}{95.638868}
\renewcommand{\pmax}{0.602094}
\renewcommand{\twopminusone}{0.204188}
\renewcommand{\rmsp}{0.008027}
\renewcommand{\twopx}{0.208583}
\renewcommand{\sigmap}{0.008047}
\renewcommand{\tsunfairbrownianfractionmean}{0.007862}
\renewcommand{\tsunfairbrownianfractionstddev}{0.038619}
\renewcommand{\shannonlogreturns}{0.560125}
\renewcommand{\shannonlogreturnshundred}{56.012500}
\renewcommand{\twopone}{0.120250}
\renewcommand{\twoponehundred}{12.025000}
\renewcommand{\hundredtwoponehundred}{87.975000}
\renewcommand{\hundredshannonlogreturnshundred}{43.987500}
\renewcommand{\datatslsqepbits}{0.007623}
\renewcommand{\thurstall}{0.633980}
\renewcommand{\thurstlow}{0.710108}
\renewcommand{\thurstlowtwo}{1.420216}
\renewcommand{\thurstlowhundred}{71.010800}
\renewcommand{\thcalcall}{0.247886}
\renewcommand{\thcalclow}{0.171737}
\renewcommand{\chisquared}{2.862000}
\renewcommand{\critical}{42.557000}

    \renewcommand{\timescale}{month}
    \subidx{market}{\market}
    \idx{\market}

    \section{\market}

        \renewcommand{\SETLABEL}{\LABPRE:USTBILL}
        \renewcommand{\SETLABELQ}{\LABPRE:USTBILLQ}
        \label{\SETLABEL}
        \renewcommand{\SETLABELREF}{\LABPREREF:USTBILL}

        \idx{United States Federal Reserve Board}
        For the analysis, the data was in the directory
        {\directory}\footnote{Data from the United States Federal
        Reserve Board, 1980---1994, by {\timescale}s, in percent. The
        time series, which was Treasury Bill rate of returns, in
        percent per year, was converted to cumulative growth per month
        by converting each element in the time series to a fraction,
        dividing by 12, and adding 1. The previous value of cumulative
        returns was multiplied by this number for the next value of
        cumulative returns.}.

        The data in this section is presented in tabular form in
        Section~\ref{\SETLABELREF}. Note that in this analysis, the
        rate of revenue returns means the increase or decrease in the
        {\market}. This is included for comparative purposes. The data
        file actually represents how the value of an investment in
        {\market} Returns has increased, over the years.

        %
% -----------------------------------------------------------------------------
%
% A license is hereby granted to reproduce this software source code and
% to create executable versions from this source code for personal,
% non-commercial use.  The copyright notice included with the software
% must be maintained in all copies produced.
%
% THIS PROGRAM IS PROVIDED "AS IS". THE AUTHOR PROVIDES NO WARRANTIES
% WHATSOEVER, EXPRESSED OR IMPLIED, INCLUDING WARRANTIES OF
% MERCHANTABILITY, TITLE, OR FITNESS FOR ANY PARTICULAR PURPOSE.  THE
% AUTHOR DOES NOT WARRANT THAT USE OF THIS PROGRAM DOES NOT INFRINGE THE
% INTELLECTUAL PROPERTY RIGHTS OF ANY THIRD PARTY IN ANY COUNTRY.
%
% Copyright (c) 1994-2006, John Conover, All Rights Reserved.
%
% Comments and/or bug reports should be addressed to:
%
%     john@email.johncon.com (John Conover)
%
% -----------------------------------------------------------------------------
%
% Revision: \RCSRevision \\
% Revision Time: \RCSTime UMT \\
% Revision Date: \RCSDate \\
% Revision Id: \RCSId \\
% Revision File: \RCSLog \\
\RCS $Revision: 0.0 $
\RCS $Date: 2006/01/20 04:38:13 $
\RCS $Id: fraction.tex,v 0.0 2006/01/20 04:38:13 john Exp $
% $Log: fraction.tex,v $
% Revision 0.0  2006/01/20 04:38:13  john
% Initial version
%
%
    \subsection{Time Series Increments Analysis}
        \label{\SETLABEL:TSA}

        \subidx{\market}{Time series analysis}
        \subidx{time series}{increments}
        \subidx{time series}{analysis}
        \subidx{cumulative sum}{analysis}
        \subidx{analysis}{cumulative sum}
        \subidx{analysis}{random process}
        \subidx{random process}{analysis}
        \subidx{Gaussian}{increments}
        \subidx{increments}{Gaussian}
        \subidx{Brownian}{motion, fractional}
        \subidx{fractional}{Brownian motion}
        \subidx{fractal}{Brownian motion}
        The data in this section is presented in tabular form in
        Section~\ref{\SETLABELREF:TSA}.  Figure~\ref{\SETLABEL:TS} is
        a graph of the time series data for the {\market}.

        \subidx{increments}{normalized}
        \subidx{normalized}{increments}
        \subidx{programs}{tsfraction}
        \subidx{tsfraction}{program}
        Figure~\ref{\SETLABEL:TF} is a graph of the normalized
        increments of the time series data presented in
        Figure~\ref{\SETLABEL:TS}. The data presented was made by
        running the program {\it tsfraction}\/ on the time series
        data. The program {\it tsfraction}\/ is described briefly in
        Appendix~\ref{programs}, and subtracts the previous value from
        the next value, dividing this difference by the previous
        value, for each element in the time series data. The new time
        series contains the instantaneous change in the rate of
        revenue returns, divided by the magnitude of the instantaneous
        rate of revenue returns.

        \subidx{mean}{standard deviation}
        \subidx{standard deviation}{mean}
        \idx{root mean square}
        \idx{least squares approximation}
        \begin{figure}[ht]
            \begin{center}
                \begin{minipage}[t]{0.45\textwidth}
                    \epsfxsize=1.0\linewidth
                    \epsffile{\directory/data.eps}
                    \caption{{\market}, time series data.}
                    \label{\SETLABEL:TS}
                    \label{\SETLABELQ:TS}
                \end{minipage}
                \hfill
                \begin{minipage}[t]{0.45\textwidth}
                    \epsfxsize=1.0\linewidth
                    \epsffile{\directory/data.tsfraction.eps}
                    \caption[{\market}, normalized
                        increments]{{\market}, normalized increments
                        of the time series data presented in
                        Figure~\ref{\SETLABEL:TS}. The mean is
                        {\datafractionmean} with a standard deviation
                        of {\datafractionstddev}. The formula for the
                        least squares approximation is
                        ${\datafractionconstant} +
                        {\datafractionslope}t$, and the root mean
                        squared value is {\datafractionrms}. The
                        graph, labeled ``data\-.tsfraction\-.tsrms,''
                        is the running root mean square, and
                        ``data\-.tsfraction\-.tsavg'' is the running
                        average of the normalized increments.  This
                        graph is the fraction of change in the time
                        series, as a function of time. Note that the
                        slope of the mean, {\datafractionslope}, is
                        the coefficient of the nonlinearity term in
                        the normalized increments. See
                        Chapter~\ref{general}, Section~\ref{nlextend}
                        for a possible application of the logistic
                        function to this data set.}
                    \label{\SETLABEL:TF}
                    \label{\SETLABELQ:TF}
                \end{minipage}
            \end{center}
        \end{figure}

        \subidx{absolute value}{increments}
        \subidx{increments}{absolute value}

        Figure~\ref{\SETLABEL:TFA} is a graph of the absolute value of
        the normalized increments of the time series data presented in
        Figure~\ref{\SETLABEL:TF}. The data presented was made by
        running the Unix utility sed(1) on the normalized increments
        time series data to remove the negative signs. This is an
        absolute value procedure.  The resulting time series contains
        the absolute value of the instantaneous change in the rate of
        revenue returns, divided by the magnitude of the instantaneous
        rate of revenue returns\footnote{The absolute value of the
        normalized increments, when averaged, is related to the root
        mean square of the increments by a constant. If the normalized
        increments are a fixed increment, the constant is unity. If
        the normalized increments have a Gaussian distribution, the
        constant is $\approx 0.8$ depending on the accuracy of of
        ``fit'' to a Gaussian distribution.}.

        \subidx{histogram}{normalized}
        \subidx{normalized}{histogram}
        \subidx{programs}{tsnormal}
        \subidx{tsnormal}{program}
        \subidx{mean}{standard deviation}
        \subidx{standard deviation}{mean}
        \idx{root mean square}
        \idx{least squares approximation}
        \subidx{\market}{analysis of increments}
        Figure~\ref{\SETLABEL:NH} is the normalized histogram of the
        normalized increments of the time series data shown in
        Figure~\ref{\SETLABEL:TF}. The abscissa is 3 $\sigma$ limits,
        and the area under the two curves is identical. The data for
        this figure was produced by the program {\it tsnormal}\/,
        which is described briefly in Appendix~\ref{programs}.

        \begin{figure}[ht]
            \begin{center}
                \begin{minipage}[t]{0.45\textwidth}
                    \epsfxsize=1.0\linewidth
                    \epsffile{\directory/data.tsfraction.abs.eps}
                    \caption[{\market}, absolute value of the
                        normalized increments]{{\market}, absolute
                        value of the normalized increments of the time
                        series data presented in
                        Figure~\ref{\SETLABEL:TF}.  The mean is
                        {\datafractionabsmean} with a standard
                        deviation of {\datafractionabsstddev}. The
                        formula for the least squares approximation is
                        ${\datafractionabsconstant} +
                        {\datafractionabsslope}t$, and the root mean
                        square value, from Figure~\ref{\SETLABEL:TF},
                        is {\datafractionrms}.  The graph, labeled
                        ``data\-.tsfraction\-.tsrms,'' is the running
                        root mean square, and
                        ``data\-.tsfraction\-.tsavg'' is the running
                        average of the normalized increments presented
                        in Figure~\ref{\SETLABEL:TF}, superimposed
                        here for convenience. This graph is the
                        absolute value of the fraction of change in
                        the time series, as a function of time.}
                    \label{\SETLABEL:TFA}
                    \label{\SETLABELQ:TFA}
                \end{minipage}
                \hfill
                \begin{minipage}[t]{0.45\textwidth}
                    \epsfxsize=1.0\linewidth
                    \epsffile{\directory/data.tsfraction.tsnormal-s30.eps}
                    \caption[{\market}, normalized histogram of the
                        normalized increments]{{\market}, normalized
                        histogram of the normalized increments of the
                        time series data shown in
                        Figure~\ref{\SETLABEL:TF}.  The data has a
                        mean of {\datafractionmean}, with a standard
                        deviation of {\datafractionstddev}.  The area
                        under the two curves is identical. The
                        $\chi^2$ value of the observed and expected
                        values of the two curves is {\chisquared},
                        with a critical value of {\critical}.}
                    \label{\SETLABEL:NH}
                \end{minipage}
            \end{center}
        \end{figure}

        \subidx{programs}{tsXsquared}
        \subidx{tsXsquared}{program}
        \subidx{\market}{chi-squared values of increments}
        The program {\it tsXsquared}\/, which is briefly described in
        appendix~\ref{programs}, was used to derive the $\chi^2$
        statistics for the data presented in
        Figure~\ref{\SETLABEL:NH}.

        \subidx{programs}{tsstatest}
        \subidx{tsstatest}{program}
        \subidx{\market}{statistical estimates}

        Figure~\ref{\SETLABEL:SE} is the statistical estimate for the
        data presented in Figure~\ref{\SETLABEL:TF}, as derived by the
        program {\it tsstatest}\/, which is briefly described in
        appendix~\ref{programs}.

        \begin{figure}[ht]
            \begin{center}
                \begin{minipage}[t]{\textwidth}
                    \center{\fbox{\parbox{0.9\textwidth}{\XXX{\directory/data.tsstatest-f0.1-c0.9-i.tex}}}}
                    \caption[{\market}, statistical estimates of the
                        normalized increments]{{\market}, statistical
                        estimates of the normalized increments of the
                        time series shown in Figure~\ref{\SETLABEL:TF}.
                        The table was produced with the {\it
                        tsstatest}\/ program, and illustrates the
                        size of the data set required for a confidence
                        level of 90\%, with an error estimate of $\pm$
                        10\%, or alternately, the error estimate on
                        the time series shown in Figure~\ref{\SETLABEL:TF}.}
                    \label{\SETLABEL:SE}
                \end{minipage}
            \end{center}
        \end{figure}

        Note that the data set size estimations, as produced by the
        {\it tsstatest}\/ program, are probably very conservative,
        depending on the magnitude of the Shannon probability, $P =
        \shannonlogreturns$, as derived in
        Section~\ref{\SETLABEL:SP}. See Chapter~\ref{general},
        Section~\ref{serdss} for possible alternative methodologies
        for addressing the analysis of fractal time series with
        limited data set sizes. Depending on the magnitude of the
        Shannon probability, $P$, these estimates can be several
        orders of magnitude too high.

        \subidx{derivative of increments}{normalized}
        \subidx{normalized}{derivative of increments}
        \subidx{programs}{tsderivative}
        \subidx{tsderivative}{program}
        Figure~\ref{\SETLABEL:TF1} is the normalized histogram of the
        first derivative of the normalized increments of the time
        series data shown in Figure~\ref{\SETLABEL:TF}. In principle,
        if the distribution of the normalized increments presented in
        Figure~\ref{\SETLABEL:NH} is Gaussian in nature, this
        distribution would be similar to ``white noise,'' as presented
        in appendix~\ref{programs}, Figure~\ref{whiteexample}. The
        data was generated by the {\it tsderivative}\/ program, which
        is briefly described in
        appendix~\ref{programs}. Figure~\ref{\SETLABEL:TF2} is the
        normalized histogram of the second derivative of the
        normalized increments of the time series data shown in
        Figure~\ref{\SETLABEL:TF}. In principle, if the distribution
        of the normalized increments presented in
        Figure~\ref{\SETLABEL:NH} is an integrated Gaussian
        distribution in nature, this distribution would be similar to
        ``white noise,'' as presented in appendix~\ref{programs},
        Figure~\ref{whiteexample}.

        \begin{figure}[ht]
            \begin{center}
                \begin{minipage}[t]{0.45\textwidth}
                    \epsfxsize=1.0\linewidth
                    \epsffile{\directory/data.tsfraction.tsderivative.tsnormal-s30.eps}
                    \caption[{\market}, histogram of the first
                        derivative of the increments]{{\market},
                        normalized histogram of the first derivative
                        of the normalized increments of the time
                        series data shown in
                        Figure~\ref{\SETLABEL:TF}.}
                    \label{\SETLABEL:TF1}
                \end{minipage}
                \hfill
                \begin{minipage}[t]{0.45\textwidth}
                    \epsfxsize=1.0\linewidth
                    \epsffile{\directory/data.tsfraction.2tsderivative.tsnormal-s30.eps}
                    \caption[{\market}, histogram of the second
                        derivative of the increments]{{\market},
                        normalized histogram of second derivative of
                        the the normalized increments of the time
                        series data shown in
                        Figure~\ref{\SETLABEL:TF}.}
                    \label{\SETLABEL:TF2}
                \end{minipage}
            \end{center}
        \end{figure}

        \subidx{fractal}{range}
        \subidx{fractal}{R/S analysis}
        \subidx{\market}{rate of revenue returns, range}
        \subidx{\market}{deterministic mechanism}
        \subidx{deterministic}{mechanism}
        \subidx{mechanism}{deterministic}
        Figure~\ref{\SETLABEL:TR} is the range of values of the time
        series shown in Figure~\ref{\SETLABEL:TS}. The horizontal axis
        is time into the future. In principle, if the time series was
        characterized as fractional Brownian motion the graph in
        Figure~\ref{\SETLABEL:TR} would be a square root
        function\footnote{Note that the ``roughness,'' or ``sawtooth''
        characteristics of the graph in Figure~\ref{\SETLABEL:TR} are
        a computational artifact---caused by not using the -m option
        to the program {\it tshurst}\/, which is computationally
        inefficient.}. Figure~\ref{\SETLABEL:TD} is the deterministic
        map of the normalized increments of the time series data shown
        in Figure~\ref{\SETLABEL:TF}. The deterministic map is useful
        for determining if a time series was created by a
        deterministic mechanism. This, essentially, maps each element
        in the time series with the previous element in the time
        series.  See,~\cite[pp. 745]{Peitgen}.

        \begin{figure}[ht]
            \begin{center}
                \begin{minipage}[t]{0.45\textwidth}
                    \epsfxsize=1.0\linewidth
                    \epsffile{\directory/data.tshurst-f.eps}
                    \caption[{\market}, range]{{\market}, range of the
                        time series data shown in
                        Figure~\ref{\SETLABEL:TS}.}
                    \label{\SETLABEL:TR}
                \end{minipage}
                \hfill
                \begin{minipage}[t]{0.45\textwidth}
                    \epsfxsize=1.0\linewidth
                    \epsffile{\directory/data.tsfraction.tsdeterministic.eps}
                    \caption[{\market}, deterministic map]{{\market},
                        deterministic map of the normalized increments
                        of the time series data shown in
                        Figure~\ref{\SETLABEL:TF}.}
                    \label{\SETLABEL:TD}
                \end{minipage}
            \end{center}
        \end{figure}

% Local Variables:
% TeX-parse-self: t
% TeX-auto-save: t
% TeX-master: "fractal.tex"
% End:


        \subsubsection{Observations on the Time Series Increments Analysis}

            Figure~\ref{\SETLABEL:NH} would seem to indicate that the
            time series data for the {\market} represents a cumulative
            sum/integration of a random process that has a Gaussian
            distribution, (ie., satisfies the Gaussian increments
            property of fractional Brownian
            motion~\cite[pp. 250]{Crownover},) tending to justify the
            assumption that the time series data represents fractional
            Brownian motion.

        %
% -----------------------------------------------------------------------------
%
% A license is hereby granted to reproduce this software source code and
% to create executable versions from this source code for personal,
% non-commercial use.  The copyright notice included with the software
% must be maintained in all copies produced.
%
% THIS PROGRAM IS PROVIDED "AS IS". THE AUTHOR PROVIDES NO WARRANTIES
% WHATSOEVER, EXPRESSED OR IMPLIED, INCLUDING WARRANTIES OF
% MERCHANTABILITY, TITLE, OR FITNESS FOR ANY PARTICULAR PURPOSE.  THE
% AUTHOR DOES NOT WARRANT THAT USE OF THIS PROGRAM DOES NOT INFRINGE THE
% INTELLECTUAL PROPERTY RIGHTS OF ANY THIRD PARTY IN ANY COUNTRY.
%
% Copyright (c) 1994-2006, John Conover, All Rights Reserved.
%
% Comments and/or bug reports should be addressed to:
%
%     john@email.johncon.com (John Conover)
%
% -----------------------------------------------------------------------------
%
% Revision: \RCSRevision \\
% Revision Time: \RCSTime UMT \\
% Revision Date: \RCSDate \\
% Revision Id: \RCSId \\
% Revision File: \RCSLog \\
\RCS $Revision: 0.0 $
\RCS $Date: 2006/01/20 04:38:13 $
\RCS $Id: instant.tex,v 0.0 2006/01/20 04:38:13 john Exp $
% $Log: instant.tex,v $
% Revision 0.0  2006/01/20 04:38:13  john
% Initial version
%
%
    \subsection{Instantaneous Analysis of Normalized Increments}
        \label{\SETLABEL:IA}

        \subidx{\market}{instantaneous analysis of normalized increments}
        \idx{average of normalized increments}
        \idx{root mean square of normalized increments}
        \subidx{Shannon probability}{instantaneous computation of}
        \subidx{average of normalized increments}{instantaneous computation of}
        \subidx{root mean square of normalized increments}{instantaneous computation of}
        \subidx{instantaneous computation}{Shannon probability}
        \subidx{instantaneous computation}{average of normalized increments}
        \subidx{instantaneous computation}{root mean square of normalized increments}
        \idx{time series}
        \subidx{time series}{instantaneous analysis}
        \subidx{instantaneous analysis}{time series}
        \subidx{time series}{increments}
        \subidx{time series}{analysis}
        \subidx{Shannon}{probability}
        \subidx{probability}{Shannon}
        \subidx{normalized}{increments}
        \subidx{increments}{normalized}

        The program {\it tsinstant}\/, which is briefly described in
        Appendix~\ref{programs}, is for finding the instantaneous
        fraction of change in a time series. The value of a sample in
        the time series is subtracted from the previous sample in the
        time series, and divided by the value of the previous sample.
        As explained in Chapter~\ref{general},
        Sections~\ref{derivation},~\ref{GA},~\ref{abmfi},~\ref{aftsma}
        and,~\ref{ompl} for Brownian motion, random walk fractals, the
        absolute value of the instantaneous fraction of change is also
        the root mean square of the instantaneous fraction of
        change\footnote{The absolute value of the normalized
        increments, when averaged, is related to the root mean square
        of the increments by a constant. If the normalized increments
        are a fixed increment, the constant is unity. If the
        normalized increments have a Gaussian distribution, the
        constant is $\approx 0.8$ depending on the accuracy of of
        ``fit'' to a Gaussian distribution.}. Squaring this value is
        the average of the instantaneous fraction of change, and
        adding unity to the absolute value of the instantaneous
        fraction of change, and dividing by two, is the Shannon
        probability of the instantaneous fraction of change.

        Figure~\ref{\SETLABEL:IA1} is the instantaneous value of the
        root mean square of the normalized increments for the
        {\market}, and Figure~\ref{\SETLABEL:IA2} is the instantaneous
        Shannon probability for the normalized increments.

        \begin{figure}[ht]
            \begin{center}
                \begin{minipage}[t]{0.45\textwidth}
                    \epsfxsize=1.0\linewidth
                    \epsffile{\directory/data.tsinstant-r.eps}
                    \caption[{\market}, instantaneous value of
                        rms.]{{\market}, instantaneous value of the
                        root mean square of the normalized increments,
                        provided by running the program {\it
                        tsinstant}\/ with the -r option on the data
                        presented in Figure~\ref{\SETLABEL:TS}.}
                    \label{\SETLABEL:IA1}
                    \label{\SETLABELQ:IA1}
                \end{minipage}
                \hfill
                \begin{minipage}[t]{0.45\textwidth}
                    \epsfxsize=1.0\linewidth
                    \epsffile{\directory/data.tsinstant-s.eps}
                    \caption[{\market}, instantaneous value of
                        Shannon probability.]{{\market}, instantaneous
                        value of the Shannon probability of the
                        normalized increments, provided by running the
                        program {\it tsinstant}\/ with the -s option
                        on the data presented in
                        Figure~\ref{\SETLABEL:TS}.}
                    \label{\SETLABEL:IA2}
                    \label{\SETLABELQ:IA2}
                \end{minipage}
            \end{center}
        \end{figure}

% Local Variables:
% TeX-parse-self: t
% TeX-auto-save: t
% TeX-master: "fractal.tex"
% End:


        %
% -----------------------------------------------------------------------------
%
% A license is hereby granted to reproduce this software source code and
% to create executable versions from this source code for personal,
% non-commercial use.  The copyright notice included with the software
% must be maintained in all copies produced.
%
% THIS PROGRAM IS PROVIDED "AS IS". THE AUTHOR PROVIDES NO WARRANTIES
% WHATSOEVER, EXPRESSED OR IMPLIED, INCLUDING WARRANTIES OF
% MERCHANTABILITY, TITLE, OR FITNESS FOR ANY PARTICULAR PURPOSE.  THE
% AUTHOR DOES NOT WARRANT THAT USE OF THIS PROGRAM DOES NOT INFRINGE THE
% INTELLECTUAL PROPERTY RIGHTS OF ANY THIRD PARTY IN ANY COUNTRY.
%
% Copyright (c) 1994-2006, John Conover, All Rights Reserved.
%
% Comments and/or bug reports should be addressed to:
%
%     john@email.johncon.com (John Conover)
%
% -----------------------------------------------------------------------------
%
% Revision: \RCSRevision \\
% Revision Time: \RCSTime UMT \\
% Revision Date: \RCSDate \\
% Revision Id: \RCSId \\
% Revision File: \RCSLog \\
\RCS $Revision: 0.0 $
\RCS $Date: 2006/01/20 04:38:13 $
\RCS $Id: logistic.tex,v 0.0 2006/01/20 04:38:13 john Exp $
% $Log: logistic.tex,v $
% Revision 0.0  2006/01/20 04:38:13  john
% Initial version
%
%
    \subsection{Logistic Analysis}
        \label{\SETLABEL:LA}

        \subidx{\market}{Logistic function analysis}
        \subidx{time series}{logistic function}
        \subidx{logistic function}{time series}
        \subidx{time series}{increments}
        \subidx{time series}{analysis}
        \subidx{cumulative sum}{analysis}
        \subidx{analysis}{cumulative sum}
        \subidx{analysis}{random process}
        \subidx{random process}{analysis}
        The data in this section is presented in tabular form in
        Section~\ref{\SETLABELREF:LAA}.  Figure~\ref{\SETLABEL:LA1} is
        a graph of the logistic function estimates of the time series
        data for the {\market}. The reader is cautioned that these
        graphs are constructed using the method suggested in
        Chapter~\ref{general}, Section~\ref{nlextend} and enormous
        precision is required for adequate prediction of the logistic
        function,~\cite{Modis}. Particularly, the non-linear term will
        usually require intervention to produce a practical fit to the
        data. In addition, there are numerical stability issues with
        logistic function methodologies\footnote{For example, in
        Figures~\ref{\SETLABEL:LA1} and~\ref{\SETLABEL:LA2}, if the
        non-linear term, $b$, was greater than zero, it was set to
        zero to produce the graphs. See Section~\ref{\SETLABELREF:LAA}
        for the actual derived values. In other cases, the magnitude
        of $b$ was too large, resulting in a graph that was decreasing
        as a function of time}.  The methodology should be regarded as
        ``fragile.'' It is included for completeness.

        \idx{least squares approximation}
        Figure~\ref{\SETLABEL:LA1} is a graph of the logistic function
        for the time series data presented in
        Figure~\ref{\SETLABEL:TS}. The data presented was made by
        running the program {\it tsdlogistic}\/, which is described
        briefly in Appendix~\ref{programs}, on the parameters
        extracted from the time series data as suggested in
        Figure~\ref{\SETLABEL:TF}. The program {\it tslsq}\/ was used
        to derive the constant and the slope of the normalized
        increments of the data presented in Figure~\ref{\SETLABEL:TF}.
        Figure~\ref{\SETLABEL:LA2} is the same graph, but with the
        time scale expanded by a factor of two.

        \begin{figure}[ht]
            \begin{center}
                \begin{minipage}[t]{0.45\textwidth}
                    \epsfxsize=1.0\linewidth
                    \epsffile{\directory/data.tsfraction.tslsq-p.tsdlogistic.eps}
                    \caption[{\market}, logistic function
                        estimates.]{{\market}, logistic function
                        estimates, provided by running the {\it
                        tslsq}\/ program on the normalized increments
                        presented in Figure~\ref{\SETLABEL:TF} with
                        the -p option. These parameters were used as
                        arguments to the {\it tsdlogistic}\/ program.}
                    \label{\SETLABEL:LA1}
                    \label{\SETLABELQ:LA1}
                \end{minipage}
                \hfill
                \begin{minipage}[t]{0.45\textwidth}
                    \epsfxsize=1.0\linewidth
                    \epsffile{\directory/data.tsfraction.tslsq-p.tsdlogistic2.eps}
                    \caption[{\market}, logistic function
                        estimates.]{{\market}, logistic function
                        estimates of Figure~\ref{\SETLABEL:LA1} with
                        the time scale expanded by a factor of two.}
                    \label{\SETLABEL:LA2}
                    \label{\SETLABELQ:LA2}
                \end{minipage}
            \end{center}
        \end{figure}

% Local Variables:
% TeX-parse-self: t
% TeX-auto-save: t
% TeX-master: "fractal.tex"
% End:


        %
% -----------------------------------------------------------------------------
%
% A license is hereby granted to reproduce this software source code and
% to create executable versions from this source code for personal,
% non-commercial use.  The copyright notice included with the software
% must be maintained in all copies produced.
%
% THIS PROGRAM IS PROVIDED "AS IS". THE AUTHOR PROVIDES NO WARRANTIES
% WHATSOEVER, EXPRESSED OR IMPLIED, INCLUDING WARRANTIES OF
% MERCHANTABILITY, TITLE, OR FITNESS FOR ANY PARTICULAR PURPOSE.  THE
% AUTHOR DOES NOT WARRANT THAT USE OF THIS PROGRAM DOES NOT INFRINGE THE
% INTELLECTUAL PROPERTY RIGHTS OF ANY THIRD PARTY IN ANY COUNTRY.
%
% Copyright (c) 1994-2006, John Conover, All Rights Reserved.
%
% Comments and/or bug reports should be addressed to:
%
%     john@email.johncon.com (John Conover)
%
% -----------------------------------------------------------------------------
%
% Revision: \RCSRevision \\
% Revision Time: \RCSTime UMT \\
% Revision Date: \RCSDate \\
% Revision Id: \RCSId \\
% Revision File: \RCSLog \\
\RCS $Revision: 0.0 $
\RCS $Date: 2006/01/20 04:38:13 $
\RCS $Id: hurst.tex,v 0.0 2006/01/20 04:38:13 john Exp $
% $Log: hurst.tex,v $
% Revision 0.0  2006/01/20 04:38:13  john
% Initial version
%
%
    \subsection{Hurst Coefficient Analysis}
        \label{\SETLABEL:H}

        \subidx{\market}{Hurst coefficient analysis}
        \subidx{Hurst coefficient}{analysis}
        \subidx{increments}{normalized}
        \subidx{normalized}{increments}
        \subidx{programs}{tshurst}
        \subidx{tshurst}{program}
        The data in this section is presented in tabular form in
        Section~\ref{\SETLABELREF:HCHP}. Figure~\ref{\SETLABEL:HC} is
        a graph of the Hurst coefficient data time series data shown
        in Figure~\ref{\SETLABEL:TS}. The slope of the graph is the
        Hurst coefficient.  The data for this figure was produced by
        the program {\it tshurst}\/, which is described briefly in
        Appendix~\ref{programs}.

        \subidx{\market}{H parameter analysis}
        \subidx{H parameter}{analysis}
        \subidx{programs}{tshcalc}
        \subidx{tshcalc}{program}
        Figure~\ref{\SETLABEL:HP} is a graph of the H parameter data
        for the normalized increments of the time series data shown in
        Figure~\ref{\SETLABEL:TF}. The data for this figure was
        produced by the program {\it tshcalc}\/, which is described
        briefly in Appendix~\ref{programs}.

        \begin{figure}[ht]
            \begin{center}
                \begin{minipage}[t]{0.45\textwidth}
                    \epsfxsize=1.0\linewidth
                    \epsffile{\directory/data.tshurst.eps}
                    \caption[{\market}, Hurst coefficient data]{{\market},
                        Hurst coefficient data for the normalized
                        increments of the time series data shown in
                        Figure~\ref{\SETLABEL:TF}.  The slope of the graph
                        is the Hurst coefficient.}
                    \label{\SETLABEL:HC}
                \end{minipage}
                \hfill
                \begin{minipage}[t]{0.45\textwidth}
                    \epsfxsize=1.0\linewidth
                    \epsffile{\directory/data.tshcalc.eps}
                    \caption[{\market}, H parameter data]{{\market}, H
                        parameter data for the normalized increments of
                        the time series data shown in
                        Figure~\ref{\SETLABEL:TF} The slope of the graph
                        is the H parameter.}
                    \label{\SETLABEL:HP}
                \end{minipage}
            \end{center}
        \end{figure}

        \subidx{revenue}{See, rate of revenue returns}
        \subidx{returns}{See, rate of revenue returns}
        \subidx{\market}{revenues}
        \subidx{Hurst coefficient}{analysis}
        \subidx{\market}{Hurst coefficient analysis}
        \subidx{\market}{rate of change}
        \subidx{\market}{windows of opportunity}
        \subidx{rate of revenue returns}{forecast}
        \subidx{forecast}{rate of revenue returns}
        \idx{windows of opportunity}
        \subidx{programs}{tslsq}
        \subidx{tslsq}{program}

        The approximately linear slope of the graph in
        Figure~\ref{\SETLABEL:HC} implies that the variance of the
        rate of revenue returns, (per {\timescale},) in the {\market},
        $V(t_2 - t_1)$, over a period of time is proportional to the
        period of time raised to twice the Hurst
        coefficient~\cite[pp. 180]{Feder},~\cite[pp. 246]{Crownover}.
        This seems to be a quantitative statement concerning how fast,
        and to what degree, the rate of revenue returns' state of
        affairs can change over a period of time.  An additional
        implication, for Hurst coefficients sufficiently close to 0.5,
        is that the probability of the state of affairs repeating
        sometime in the future goes down with increasing
        time\footnote{It can be shown that the number of expected
        market ``high'' and ``low'' transitions, $N$, scales with the
        square root of time, or $N \propto \sqrt {t}$, meaning that
        the cumulative distribution of the probability, $P$, of the
        duration of a market's ``high'' or ``low'' exceeding a given
        time interval, $t$, is proportional to the reciprocal of the
        square root of the time interval, $P \propto 1 / \sqrt {t}$,
        (or, conversely, that the probability of the duration of a
        market's ``high'' or ``low'' exceeding a given time interval
        is proportional to the reciprocal of the time interval raised
        to the power $3 / 2$, ie., $P \propto 1 / t^{3 /
        2}$,~\cite[pp. 153]{Schroeder}. What this means is that a
        histogram of the ``zero free'' run-lengths of a market being
        ``high'' or ``low,'' over a long time, would have a $1 / t^{3
        / 2}$ characteristic.)}, $t$, $p(t) = erf (1/\sqrt{2t})$ which
        is approximately $1/\sqrt{t}$ for $t \gg
        1$~\cite[pp. 160]{Schroeder}. Figures~\ref{\SETLABEL:FN},
        and,~\ref{\SETLABEL:FF} compare methods of approximation of
        the ``forecastability'' of the rate of revenue returns in the
        {\market} for the near term and far term,
        respectively~\cite[pp. 83-84]{Peters:CAOITCM}\footnote{The
        author is not comfortable with Peters' interpretation. For
        example, if the algorithm explained
        in~\cite[pp. 82]{Peters:CAOITCM} is used on ``white noise''
        which, by definition, never has any correlations, the short
        term Hurst coefficient, and thus the ``forecastability,'' is
        still near unity---a bit of an enigma. This can be verified
        with the {\it tswhite}\/ and {\it tshurst}\/ programs, which
        are briefly described in Appendix~\ref{programs}.}.  This
        seems to be a quantitative statement concerning ``windows of
        opportunity'' in the rate of revenue returns, (per
        {\timescale}.)  The program {\it tslsq}\/ was used on the
        Hurst coefficient data, presented in
        Figure~\ref{\SETLABEL:HC}, to provide a least squares
        approximation to the Hurst coefficient. The superimposed least
        squares approximation with on original Hurst coefficient data
        is presented.  The time series data has a Hurst coefficient of
        {\thurstlow}, so that:

        \subidx{\market}{Hurst coefficient analysis}
        \begin{eqnarray}
            V\left(t_2 - t_1\right) & \propto & \left(t_2 - t_1\right)^{2 \cdot H}\\
            V\left(t_2 - t_1\right) & \propto & \left(t_2 - t_1\right)^{2 \cdot {\thurstlow}}\\
                                    & \propto & \left(t_2 - t_1\right)^{\thurstlowtwo}
            \label{\SETLABEL:V}
        \end{eqnarray}

        \subidx{fractional}{Brownian motion}
        \subidx{Brownian motion}{fractional}
        \idx{fractal}
        \noindent where $V(t_2 - t_1)$ is the variance of the
        increments of the rate of revenue returns, (per {\timescale},)
        over the time interval $t_2 -
        t_1$,~\cite[pp. 177]{Feder},~\cite[pp. 494]{Peitgen}. If $H >
        \frac{1}{2}$, then the time series is termed as being
        characterized by ``fractional Brownian
        motion~\cite[pp. 170]{Feder}.''

        \subidx{rate of revenue returns}{predictability}
        \subidx{rate of revenue returns}{forecastability}
        \subidx{rate of revenue returns}{consistency}
        \subidx{predictability}{rate of revenue returns}
        \subidx{forecastability}{rate of revenue returns}
        \subidx{consistency}{rate of revenue returns}
        \subidx{\market}{rate of revenue returns, predictability}
        \subidx{\market}{rate of revenue returns, forecastability}
        \subidx{\market}{rate of revenue returns, consistency}
        \subidx{Hurst coefficient}{analysis}
        \subidx{\market}{Hurst coefficient analysis}
        \subidx{\market}{rate of change}

        In some sense, the Hurst coefficient is a quantitative
        expression of the ``forecastability'' of the future based on
        the past\footnote{Actually, in general, when summing fractal
        entities, the method used should be a root mean square
        process, dependent on the Hurst Coefficient, $H$, where
        $P_{total}^H = P_1^H + P_2^H + \cdots$, where $P_n$ is the
        fractal entities. For a Brownian motion, or random walk type
        of fractal the Hurst Coefficient is a function of time into
        the future. For the ``near term,'' the Hurst coefficient is
        very near unity, meaning the summation process is linear. For
        the ``long term,'' $H \approx 0.5$, or a standard root mean
        square summation process should be used. If $H$ is $0.5$ then
        the market is termed a Brownian motion, or random walk
        process. If it is larger than 0.5, it is termed fractional
        Brownian motion process. For a random walk process, ``near
        term'' and ``far term'' are quantitatively differentiated on
        the Hurst Coefficient graph where $1 - \ln (t) = 0.5 \cdot \ln
        (t)$, or when $\ln (t) = 2$, or $t = 7.389\ldots$ See
        Section~\ref{\SETLABEL:FS} for the particulars on using Hurst
        Coefficient to sum fractal process' for the {\market}. See
        also~\cite[pp. 67, 83-84]{Peters:CAOITCM} and~\cite[pp. 129,
        159]{Schroeder} for particulars on the implications of the
        Hurst Coefficient and root mean square summation issues.}.  A
        Hurst coefficient of {\thurstlow}, (for the near future, and
        {\thurstall} for the distant future.) implies that the
        likelihood of the rate of revenue returns, (per {\timescale},)
        for any two consecutive {\timescale}s being the same is
        {\thurstlowhundred}\%~\cite[pp. 66]{Peters:CAOITCM} for the
        near future, and {\thurstall} for the distant
        future. Likewise, there is a {\thurstlowhundred}\% chance of
        the rate of revenue returns, (per {\timescale},) movements
        being the same in consecutive time periods---ie., if, in a
        given {\timescale}, the rate of revenue returns, (per
        {\timescale},) is increasing, there is a {\thurstlowhundred}\%
        that the rate of revenue returns, (per {\timescale},) will
        increase in the following period, also. In some sense, this is
        a quantitative statement on how ``predictable,'' or
        ``forecastable'' the rate of revenue returns, (per
        {\timescale},) for the {\market} are over time, since the
        probability of having $n$ many consecutive {\timescale}s of
        the same agenda is $H^n$ where $H$ is the Hurst coefficient,
        or, letting the short term probability of having $n$ many
        {\timescale}s of the same market agenda, $p_a$, is:

        \begin{eqnarray}
            p_a\left(n\right) & = & H^{n}\\
                              & = & {\thurstlow}^{n}
            \label{\SETLABEL:MA}
        \end{eqnarray}

        \subidx{rate of revenue returns}{predictability}
        \subidx{rate of revenue returns}{forecastability}
        \subidx{rate of revenue returns}{consistency}
        \subidx{predictability}{rate of revenue returns}
        \subidx{forecastability}{rate of revenue returns}
        \subidx{consistency}{rate of revenue returns}
        As an interesting interpretation of the normalized increments
        of the time series data presented in
        Figure~\ref{\SETLABEL:TF}, if the vertical axis is multiplied
        by 100, to convert to percent, then the graph represents the
        error, in percent, that would be made by forecasting, month by
        month, that the next {\timescale}'s rate of revenue returns
        would be the same as the current {\timescale}'s revenue
        rate. Interestingly, it is $\datafractionmean \cdot 100$
        percent, on the average, with a standard deviation of
        $\datafractionstddev \cdot 100$ percent, and a root mean
        square error value of $\datafractionrms \cdot 100$
        percent---small values for such a simple forecasting
        mechanism.

        \subidx{\market}{rate of revenue returns, range}
        \subidx{Hurst coefficient}{analysis}
        \subidx{\market}{Hurst coefficient analysis}
        \subidx{\market}{rate of change}

        This is, essentially, a statement of the range of values, in
        the increments of the rate of revenue returns, (per
        {\timescale},) that is to be expected over the time interval,
        $t_2 - t_1$,
        $R_v$,~\cite[pp. 178]{Feder},~\cite[pp. 172]{Cambel}:

        \begin{eqnarray}
            R_v\left(t_2 - t_1\right) & \propto & \left(t_2 - t_1\right)^{H}\\
                                      & \propto & \left(t_2 - t_1\right)^{\thurstlow}
            \label{\SETLABEL:R}
        \end{eqnarray}

        \subidx{\market}{rate of revenue returns, range}
        \subidx{Hurst coefficient}{analysis}
        \subidx{\market}{Hurst coefficient analysis}
        \subidx{\market}{rate of change}
        \subidx{Markov}{statistics}
        \subidx{statistics}{Markov}
        \noindent where $R$ is the range of values in the increments
        of the rate of revenue returns, (per {\timescale}.) A Hurst
        coefficient, $H$, that is much larger than $\frac{1}{2}$, (but
        less than 1,) implies a strongly non-Gaussian distribution in
        the increments of the rate of revenue returns, (per
        {\timescale},)~\cite[pp. 152, 194]{Feder}, and a Hurst
        coefficient near $\frac{1}{2}$ implies that the increments of
        the rate of revenue returns, (per {\timescale}) is
        characteristic of an independent
        process~\cite[pp. 195]{Feder}. Extreme caution should be
        exercised in using Markov statistics in any analysis where the
        Hurst coefficient is not
        $\frac{1}{2}$,~\cite[pp. 124]{Crownover},~\cite[pp. 106]{Peters:CAOITCM}.


        As a useful approximation, if $H$, is approximately
        $\frac{1}{2}$, Equation~\ref{\SETLABEL:R} reduces
        to,~\cite[pp. 129]{Schroeder}:

        \begin{eqnarray}
            R\left(t_2 - t_1\right) & \propto & (t_2 - t_1)^{\frac{1}{2}}\\
                                    & \propto & \sqrt{\left(t_2 - t_1\right)}
        \end{eqnarray}

        \subidx{\market}{rate of revenue returns, range}
        \subidx{\market}{rate of revenue returns, increase and decrease}
        \subidx{Hurst coefficient}{analysis}
        \subidx{\market}{Hurst coefficient analysis}
        \subidx{\market}{rate of change}
        \subidx{Markov}{statistics}
        \subidx{statistics}{Markov}

        In the case where the Hurst coefficient, $H$, is
        $\frac{1}{2}$, the range of values in the increments of the
        rate of revenue returns, (per {\timescale},) divided by the
        standard deviation of these values, $S$, can be anticipated to
        increase over time according to the following
        relation,~\cite[pp. 154]{Feder},~\cite[pp. 129]{Schroeder}:

        \begin{equation}
            \frac{R\left(t_2 - t_1\right)}{S} \propto \left(t_2 - t_1\right)^{\frac{1}{2}}
        \end{equation}

        \subidx{\market}{rate of revenue returns, range}
        \subidx{\market}{rate of revenue returns, increase and decrease}
        \subidx{Hurst coefficient}{analysis}
        \subidx{\market}{Hurst coefficient analysis}
        \subidx{\market}{rate of change}
        \noindent which is a useful conceptual approximation, since it
        involves only the square root function---if the range and the
        standard deviation of the increments of the rate of revenue
        returns, (per {\timescale},) are known, (and $H \approx
        \frac{1}{2}$,) then the expected change in $\frac{R}{S}$, will
        increase with the square root of time\footnote{To be precise,
        it is actually asymptotically proportional to
        $\tau^{\frac{1}{2}}$}.

        Another useful approximation when rescaling processes that are
        characterize by Brownian motion, (ie., when $H \approx
        \frac{1}{2}$,) is that:

        \begin{eqnarray}
            X\left(t\right) & \propto & \frac{X\left(rt\right)}{r^{H}}\\
                            & \propto & \frac{X\left(rt\right)}{r^{\thurstlow}}
        \end{eqnarray}

        \idx{Brownian motion}
        \idx{fractal}
        Where $X(t)$ is the process characterized by Brownian motion,
        and $r$ is a scaling factor,~\cite[pp. 494]{Peitgen}.

        \subidx{programs}{tslsq}
        \subidx{tslsq}{program}
        The program {\it tslsq}\/ was used on the H parameter data,
        presented in Figure~\ref{\SETLABEL:HP}, to provide a least
        squares approximation to the H parameter for the
        {\market}. The superimposed least squares approximation on the
        original H parameter data is presented.  By contrast, the H
        parameter, as derived by the methodology outlined
        in~\cite[pp. 249]{Crownover}, is {\thcalclow} for the near
        future, and {\thcalcall} for the distant future.

        \subidx{\market}{Hurst coefficient analysis}
        \subidx{Hurst coefficient}{analysis}
        \subidx{increments}{normalized}
        \subidx{normalized}{increments}
        \subidx{programs}{tshurst}
        \subidx{tshurst}{program}
        \subidx{\market}{H parameter analysis}
        \subidx{H parameter}{analysis}
        \subidx{programs}{tshcalc}
        \subidx{tshcalc}{program}
        Figures~\ref{\SETLABEL:HC} and~\ref{\SETLABEL:HP} represent
        Hurst coefficient and H parameter data that are derived from
        the normalized increments, shown in
        Figure~\ref{\SETLABEL:TF}. In this case, the data is
        considered a normalized derivative of the time series data
        presented in Figure~\ref{\SETLABEL:TF}, instead of a
        cumulative sum.  The program, {\it tshurst}\/, is described
        briefly in appendix~\ref{programs}, and the data for
        figures~\ref{\SETLABEL:THC} and~\ref{\SETLABEL:THP} was made
        using the -d option.

        \begin{figure}[ht]
            \begin{center}
                \begin{minipage}[t]{0.45\textwidth}
                    \epsfxsize=1.0\linewidth
                    \epsffile{\directory/data.tsfraction.tshurst-d.eps}
                    \caption[{\market}, traditional Hurst coefficient
                        data]{{\market}, traditional Hurst coefficient
                        data for the time series data shown in
                        Figure~\ref{\SETLABEL:TS}.  The slope of the
                        graph is the Hurst coefficient, and is
                        {\hurstlow} for the near term, and
                        {\hurstall} for the far term.}
                    \label{\SETLABEL:THC}
                \end{minipage}
                \hfill
                \begin{minipage}[t]{0.45\textwidth}
                    \epsfxsize=1.0\linewidth
                    \epsffile{\directory/data.tsfraction.tshcalc-d.eps}
                    \caption[{\market}, traditional H parameter
                        data]{{\market}, traditional H parameter data
                        for the time series data shown in
                        Figure~\ref{\SETLABEL:TS} The slope of the
                        graph is the H parameter, and is {\hcalclow}
                        for the near term, and {\hcalcall} for the
                        far term.}
                    \label{\SETLABEL:THP}
                \end{minipage}
            \end{center}
        \end{figure}

% Local Variables:
% TeX-parse-self: t
% TeX-auto-save: t
% TeX-master: "fractal.tex"
% End:


        %
% -----------------------------------------------------------------------------
%
% A license is hereby granted to reproduce this software source code and
% to create executable versions from this source code for personal,
% non-commercial use.  The copyright notice included with the software
% must be maintained in all copies produced.
%
% THIS PROGRAM IS PROVIDED "AS IS". THE AUTHOR PROVIDES NO WARRANTIES
% WHATSOEVER, EXPRESSED OR IMPLIED, INCLUDING WARRANTIES OF
% MERCHANTABILITY, TITLE, OR FITNESS FOR ANY PARTICULAR PURPOSE.  THE
% AUTHOR DOES NOT WARRANT THAT USE OF THIS PROGRAM DOES NOT INFRINGE THE
% INTELLECTUAL PROPERTY RIGHTS OF ANY THIRD PARTY IN ANY COUNTRY.
%
% Copyright (c) 1994-2006, John Conover, All Rights Reserved.
%
% Comments and/or bug reports should be addressed to:
%
%     john@email.johncon.com (John Conover)
%
% -----------------------------------------------------------------------------
%
% Revision: \RCSRevision \\
% Revision Time: \RCSTime UMT \\
% Revision Date: \RCSDate \\
% Revision Id: \RCSId \\
% Revision File: \RCSLog \\
\RCS $Revision: 0.0 $
\RCS $Date: 2006/01/20 04:38:13 $
\RCS $Id: fiscal.tex,v 0.0 2006/01/20 04:38:13 john Exp $
% $Log: fiscal.tex,v $
% Revision 0.0  2006/01/20 04:38:13  john
% Initial version
%
%
    \subsection{Fixed Increment Approximation for Fiscal Strategy}
        \label{\SETLABEL:FS}

        \subidx{\market}{fiscal strategy}
        \subidx{markets}{analysis}
        \subidx{analysis}{markets}
        \subidx{strategy}{fiscal}
        \subidx{fiscal}{strategy}
        The data in this section is presented in tabular form in
        Section~\ref{\SETLABELREF:LR}. This section derives various
        values based on the ``average'' of the normalized increments
        presented in Figure~\ref{\SETLABEL:TFA}. These values are an
        approximation to a, probably, complex process with a
        distribution shown in Figure~\ref{\SETLABEL:TF}. These values
        will be used in a fixed increment Brownian fractal analysis
        and simulation of the {\market}, and may, or may not, provide
        adequate accuracy for projections.

        For an organization operating in the {\market}, the fiscal
        strategy, commensurate with the aggregate environment, can be
        derived as follows~\cite[pp. 128, pp
        151]{Schroeder},~\cite[pp. 450]{Reza},~\cite[pp. 270]{Pierce}:
        \vspace{0.15in}

        \subsubsection{Logarithmic Returns}
            \label{\SETLABEL:LR}

            \subidx{logarithmic}{returns}
            \subidx{returns}{logarithmic}
            \subidx{\market}{logarithmic returns}
            The logarithmic returns can be calculated by various
            means. Four will be presented here, for comparison.

            \subidx{programs}{tsnormal}
            \subidx{tsnormal}{program}
            \subidx{logarithmic}{returns}
            \subidx{returns}{logarithmic}
            The logarithmic returns, in bits, $bits$, as computed from
            the mean, by the program {\it tsnormal}\/, which is
            described in Chapter~\ref{programs}, and is presented in
            Figure~\ref{\SETLABEL:TF}, and Equation~\ref{abits} from
            Section~\ref{ereturns} in Chapter~\ref{general}:

            \begin{equation}
                bits = \frac{\ln \left({\datafractionmean} + 1\right)}{\ln \left(2\right)} = \datafractionmeanbits
            \end{equation}

            \subidx{programs}{tslsq}
            \subidx{tslsq}{program}
            \subidx{logarithmic}{returns}
            \subidx{returns}{logarithmic}
            \noindent By comparison, the logarithmic returns, in bits,
            $bits$, as computed from the constant in the least squares
            approximation, using the program {\it tslsq}\/, which is briefly
            described in Chapter~\ref{programs}, as presented in
            Figure~\ref{\SETLABEL:TF}, and Equation~\ref{abits} from
            Section~\ref{ereturns} in Chapter~\ref{general}:

            \begin{equation}
                bits = \frac{\ln \left({\datafractionconstant} + 1\right)}{\ln \left(2\right)} = \datafractionconstantbits
            \end{equation}

            Note that if the mean is not constant in
            Figure~\ref{\SETLABEL:TF}, this method will not provide
            accurate results.

            \subidx{programs}{tslsq}
            \subidx{tslsq}{program}
            \subidx{logarithmic}{returns}
            \subidx{returns}{logarithmic}
            \noindent And by yet another comparison, using the program
            {\it tslsq}\/, which is briefly described in
            Chapter~\ref{programs}, with the -e -p options, to provide
            a formula for the least squares exponential fit to the
            time series data set presented in
            Figure~\ref{\SETLABEL:TS}:

            \begin{equation}
                bits = {\datatslsqepbits}
            \end{equation}

            \subidx{programs}{tslogreturns}
            \subidx{tslogreturns}{program}
            \subidx{logarithmic}{returns}
            \subidx{returns}{logarithmic}
            \noindent And finally, by comparison, from the
            {\it tslogreturns}\/ program, which is briefly described
            in Chapter~\ref{programs}, with the -p option, to provide
            a formula for the logarithmic returns of the time series
            data set presented in Figure~\ref{\SETLABEL:TS}:

            \begin{equation}
                bits = {\logreturns}
            \end{equation}

        \subsubsection{Calculation of Shannon Probability}
            \label{\SETLABEL:SP}

            \subidx{\market}{Shannon probability}
            Ideally, all of the values presented in
            Section~\ref{\SETLABEL:LR} would be equal. Using the
            logarithmic returns provided by the {\it tslogreturns}\/
            program, to be consistent
            with~\cite[pp. 81]{Peters:CAOITCM}

            \subidx{programs}{tslogreturns}
            \subidx{tslogreturns}{program}
            \begin{equation}
                2^{{\logreturns}t}
            \end{equation}

            \noindent therefore:
            \begin{equation}
                C\left(p\right) = {\logreturns}
            \end{equation}
            \subidx{programs}{tsshannon}
            \subidx{tsshannon}{program}
            \subidx{Shannon}{probability}
            \subidx{probability}{Shannon}
            \noindent and, {\it tsshannon}\/ {\logreturns} gives:
            \begin{equation}
                \label{\SETLABEL:F0}
                C\left({\shannonlogreturns}\right) = {\logreturns}
            \end{equation}
            \noindent therefore:
            \begin{eqnarray}
                2^{C\left({\shannonlogreturns}\right)} & = & 2^{\logreturns}\\
                                                       & = & {\twologreturns}\\
                                                       & = & {\twologreturnshundred}\%
            \end{eqnarray}
            \noindent and:
            \begin{eqnarray}
                2p - 1 & = & \left(2 \cdot {\shannonlogreturns}\right) - 1\\
                       & = & {\twopone}\\
                       \label{\SETLABEL:F1}
                       & = & {\twoponehundred}\%
            \end{eqnarray}

            \subidx{\market}{fiscal strategy}
            \subidx{markets}{analysis}
            \subidx{analysis}{markets}
            \subidx{strategy}{fiscal}
            \subidx{fiscal}{strategy}
            \subidx{\market}{fiscal strategy}
            \subidx{\market}{growth rate}
            Presuming the simplified assumptions outlined in
            Section~\ref{assumptions}, the ``typical'' organization
            operating in the {\market} executes a long term fiscal
            strategy, commensurate with the aggregate environment,
            that is to invest, every {\timescale}, in sufficient
            additional resources and infrastructure, to increase the
            manufacturing of goods and services by {\twoponehundred}\%
            of its rate of revenue returns, (per {\timescale}.) As a
            conceptual model, the remaining {\hundredtwoponehundred}\%
            will be held in ``reserve'' with a
            {\shannonlogreturnshundred}\% chance of making twice the
            {\twoponehundred}\% back, (and a
            {\hundredshannonlogreturnshundred}\% chance of making
            0.0,) in one {\timescale}, on the average, for an average
            growth in its rate of revenue returns, (per {\timescale},)
            of {\twologreturnshundred}\%, or a doubling of its rate of
            revenue returns, (per {\timescale},) in
            {\oneoverlogreturns} {\timescale}s.

        \subsubsection{Example Fixed Increment Approximation Fiscal Strategies}

            \subidx{\market}{fiscal strategy}
            \subidx{markets}{analysis}
            \subidx{analysis}{markets}
            \subidx{strategy}{fiscal}
            \subidx{fiscal}{strategy}
            \subidx{\market}{fiscal strategy}
            \subidx{\market}{growth rate}
            \subidx{\market}{management metric}
            \idx{management metric}
            A possible metric on the effectiveness of long term fiscal
            management could possibly be that if an investment of
            {\twoponehundred}\% per {\timescale} of the rate of
            revenue returns, (per {\timescale},) is made in resources
            and infrastructure, then the rate of revenue returns would
            be expected to increase by {\twologreturnshundred}\%, per
            {\timescale}, on average.

            Note that the metrics presented in this section are
            representative of the {\market} as an aggregate whole, and
            may or may not be accurate representations for any
            particular participant in the environment. Of interest to
            the participants in the environment would be a similar
            analysis of each product or service rendered in the
            marketplace.

            \subidx{\market}{fiscal strategy}
            \subidx{markets}{analysis}
            \subidx{analysis}{markets}
            \subidx{strategy}{fiscal}
            \subidx{fiscal}{strategy}
            \subidx{\market}{fiscal strategy}
            As a simple illustrative example, a company operating in
            this environment might obtain a credit line from a bank
            that is equal to {\twoponehundred}\% of its rate of
            revenue returns, (per {\timescale},) to finance additional
            operations. In this simple scenario, the company would use
            its revenue base as collateral for the loan. Some
            {\timescale}s, depending on the {\market}'s environment,
            the company's rate of revenue returns exceeds what was
            borrowed from the bank, and the loan is repaid in
            full. Other {\timescale}s, the company must default, and
            the bank seizes a portion of the company's revenue base to
            pay the delinquent loan. However, on the average, the
            company will expand its rate of revenue returns at
            {\twologreturnshundred}\% per {\timescale}.

            \subidx{\market}{fiscal strategy}
            \subidx{markets}{analysis}
            \subidx{analysis}{markets}
            \subidx{strategy}{fiscal}
            \subidx{fiscal}{strategy}
            \subidx{\market}{fiscal strategy}
            As another simple example, a company re-invests
            {\twoponehundred}\% of its rate of revenue returns, (per
            {\timescale},) in development, marketing, sales, and
            distribution of new products.  Although some products will
            be successful and the return on the investment will exceed
            the {\twoponehundred}\% per {\timescale} investment,
            others will not. However, on the average, the company will
            expand it gross rate of revenue returns at
            {\twologreturnshundred}\% per {\timescale}.

            \subidx{\market}{fiscal strategy}
            \subidx{markets}{analysis}
            \subidx{analysis}{markets}
            \subidx{strategy}{fiscal}
            \subidx{fiscal}{strategy}
            \subidx{\market}{fiscal strategy}
            \subidx{\market}{product portfolio}
            \subidx{\market}{product diversity}
            \subidx{\market}{product mix}
            \subidx{\market}{optimum number of products}
            \idx{product portfolio}
            \idx{product diversity}
            \idx{optimum number of products}
            \idx{product mix}

            As an example of ``product portfolio'' management, suppose
            a company re-invests {\twoponehundred}\% of its rate of
            revenue returns, (per {\timescale},) in development,
            marketing, sales, and distribution of new products.
            Further suppose that the company has two products, and a
            fractal analysis of the individual product rate of revenue
            return time series indicates that one product has a
            Shannon probability of 0.65, and the other has a Shannon
            probability of 0.55. Then the percentage of re-investment
            in the first product would be $(2 \cdot 0.65 - 1) \cdot
            {\twoponehundred}$, percent of the rate of revenue
            returns, and $(2 \cdot 0.55 - 1) \cdot {\twoponehundred}$
            percent for the second product, implying that the company
            should diversify its product line\footnote{The astute
            reader would note that the linear addition was used to add
            the contribution to development of each product. This is a
            ``near term'' interpretation. Actually, in general, the
            method used should be a root mean square process,
            dependent on the Hurst Coefficient, $H$, where
            $P_{total}^H = P_1^H + P_2^H + \cdots$, where $P_n$ is the
            contribution to each individual product. For a Brownian
            motion, or random walk type of fractal the Hurst
            Coefficient is a function of time into the future. For the
            ``near term,'' the Hurst coefficient is very near unity,
            meaning the summation process is linear. For the ``long
            term,'' $H \approx 0.5$, or a standard root mean square
            summation process should be used. If $H$ is $0.5$ then the
            market is termed a Brownian motion, or random walk
            process. If it is larger than 0.5, it is termed fractional
            Brownian motion process. For a random walk process, ``near
            term'' and ``far term'' are quantitatively differentiated
            on the Hurst Coefficient graph where $1 - \ln (t) = 0.5
            \cdot \ln (t)$, or when $\ln (t) = 2$, or $t =
            7.389\ldots$ See~\cite[pp. 67, 83-84]{Peters:CAOITCM}
            and~\cite[pp. 129, 159]{Schroeder} for particulars on the
            implications of the Hurst Coefficient and root mean square
            summation issues.}.  Note that this is a ``bet hedging''
            metric methodology, and assumes that the products have
            uncorrelated revenue return rates. If this re-investment
            methodology is not feasible, perhaps for strategic
            financial reasons, then the re-investment in both products
            should total the ${\twoponehundred}$\%, and the investment
            in each product should be made at a ratio of $\frac{(2
            \cdot 0.65 - 1)}{(2 \cdot 0.55 - 1)} = 3 : 1$,
            respectively. Note that this ``bet hedging'' can be used
            to define the optimal number of products that can be
            supported on the rate of revenue returns. If it assumed
            that all products are ``typical'' for the {\market}, as a
            standard bench mark, then the optimal number will be
            $\frac{1}{{\twopone}}$. Note that this is a
            ``theoretical'' value, since not all products are
            ``typical,'' and there may be strategic reasons, for
            example product leveraging, that may increase the number
            of products above the optimum. However, most of the
            revenue should come from the optimal number of products,
            since having more products will decrease the amount of the
            potential investment in each product, and having less than
            the optimum number of products will increase the risk that
            many of the products could suffer a ``down market''
            concurrently, impacting the rate of revenue returns.  As
            another interesting interpretation of the optimal
            ``hedging of bets,'' in product portfolio strategy, and
            considering the graph of the normalized increments
            presented in Figure~\ref{\SETLABEL:TF}, if the
            organization is running optimally, then these products
            will generate, at least in principle, one standard
            deviation, approximately $0.8413 = 84.13$\% of the future
            growth in rate of revenue returns. Naturally, these are
            approximations, and the values are an approximation to a,
            probably, complex process, and appropriate scrutiny should
            be exercised before making specific projections.  As yet
            another example of ``product portfolio'' management,
            consider the issue of product mix. In this interpretation,
            {\twoponehundred}\% of the product manufactured should be
            ``proprietary,'' while the rest is ``industry standard.''
            As yet another possibility, {\twoponehundred}\% of the
            product manufactured should be predatory into new markets,
            and the remainder in markets that are ``traditional'' for
            the company.

% Local Variables:
% TeX-parse-self: t
% TeX-auto-save: t
% TeX-master: "fractal.tex"
% End:


        %
% -----------------------------------------------------------------------------
%
% A license is hereby granted to reproduce this software source code and
% to create executable versions from this source code for personal,
% non-commercial use.  The copyright notice included with the software
% must be maintained in all copies produced.
%
% THIS PROGRAM IS PROVIDED "AS IS". THE AUTHOR PROVIDES NO WARRANTIES
% WHATSOEVER, EXPRESSED OR IMPLIED, INCLUDING WARRANTIES OF
% MERCHANTABILITY, TITLE, OR FITNESS FOR ANY PARTICULAR PURPOSE.  THE
% AUTHOR DOES NOT WARRANT THAT USE OF THIS PROGRAM DOES NOT INFRINGE THE
% INTELLECTUAL PROPERTY RIGHTS OF ANY THIRD PARTY IN ANY COUNTRY.
%
% Copyright (c) 1994-2006, John Conover, All Rights Reserved.
%
% Comments and/or bug reports should be addressed to:
%
%     john@email.johncon.com (John Conover)
%
% -----------------------------------------------------------------------------
%
% Revision: \RCSRevision \\
% Revision Time: \RCSTime UMT \\
% Revision Date: \RCSDate \\
% Revision Id: \RCSId \\
% Revision File: \RCSLog \\
\RCS $Revision: 0.0 $
\RCS $Date: 2006/01/20 04:38:13 $
\RCS $Id: companies.tex,v 0.0 2006/01/20 04:38:13 john Exp $
% $Log: companies.tex,v $
% Revision 0.0  2006/01/20 04:38:13  john
% Initial version
%
%
    \subsection{Number of Companies}
        \label{\SETLABEL:QNC}

        \subidx{\market}{number of companies}
        \subidx{number of companies}{analysis}
        \subidx{analysis}{number of companies}
        \subidx{Shannon}{probability}
        \subidx{probability}{Shannon}
        This section evaluates the approximate, or ``average,'' number
        of companies in the {\market}, and uses the method outlined in
        Chapter~\ref{general}, Section~\ref{aftsma}. Since the
        average, $avg_{ind}$, and the root mean square, $rms_{ind}$,
        of the normalized increments of the {\market} time series is
        \datafractionmean, and \datafractionrms respectively, the
        number of companies participating in the market can be
        calculated by Equation~\ref{ncompanies} to be {\ncompanies}.

        If this value seems consistent number of companies in the
        {\market}, within the assumptions outlined in
        Chapter~\ref{general}, Section~\ref{aftsma}, then it would
        seem that there is some circumstantial or indirect evidence
        that the companies participating in the {\market} are
        operating optimally, and the ``average'' Shannon probability,
        $P$ for each participating company would be, using
        Equation~\ref{pncompanies}, {\pncompanies}, which would be the
        value which should be used in Section~\ref{\SETLABEL:FS} for
        each participating company if market expansion was to be
        consistent with the rest of the industry. However, if the
        Shannon probability derived in Section~\ref{\SETLABEL:FS} is
        greater than the average Shannon probability for the companies
        participating in the {\market}, as derived in this section,
        then the market would, possibly, be exploitable with the
        fiscal strategy outlined in Section~\ref{\SETLABEL:FS}. The
        maximum exploitability for the {\market} is derived in
        Section~\ref{\SETLABEL:MAXSHANNON}, but it is probably of
        doubtful practicality.

        Note that these optimizations would maximize a company's
        market growth. Since there are probably many companies
        competing in the market place, this would not necessarily
        maximize a company's P\&L, as described in
        Chapter~\ref{general}, Section~\ref{ompl}. The Shannon
        probability that maximizes market share in the {\market} is
        \pncompanies, with several alternative solutions listed in the
        previous paragraph. However, these should be contrasted to the
        Shannon probability that maximizes a company's P\&L which is
        \avgrms~in the {\market}. In all cases, the fraction of the
        P\&L that should be ``wagered'' on the future, $f$, should be:

        \begin{equation}
            f = 2P - 1
        \end{equation}

        \noindent where $P$ is the particular Shannon probability
        chosen optimize a particular fiscal strategy. Interestingly,
        the measured Shannon probability of the {\market} would tend
        to indicate that the companies participating in the market
        have chosen a fiscal strategy that optimizes market growth, as
        opposed to capital growth.

        \subidx{\market}{increasing returns}
        \subidx{economic increasing returns}{\market}
        As interesting interpretation of these exploitive issues,
        since all three fiscal strategies will result in exponential
        market growth for every company participating in the market,
        is that they may represent, perhaps, an example of
        ``increasing returns.''

% Local Variables:
% TeX-parse-self: t
% TeX-auto-save: t
% TeX-master: "fractal.tex"
% End:


        %
% -----------------------------------------------------------------------------
%
% A license is hereby granted to reproduce this software source code and
% to create executable versions from this source code for personal,
% non-commercial use.  The copyright notice included with the software
% must be maintained in all copies produced.
%
% THIS PROGRAM IS PROVIDED "AS IS". THE AUTHOR PROVIDES NO WARRANTIES
% WHATSOEVER, EXPRESSED OR IMPLIED, INCLUDING WARRANTIES OF
% MERCHANTABILITY, TITLE, OR FITNESS FOR ANY PARTICULAR PURPOSE.  THE
% AUTHOR DOES NOT WARRANT THAT USE OF THIS PROGRAM DOES NOT INFRINGE THE
% INTELLECTUAL PROPERTY RIGHTS OF ANY THIRD PARTY IN ANY COUNTRY.
%
% Copyright (c) 1994-2006, John Conover, All Rights Reserved.
%
% Comments and/or bug reports should be addressed to:
%
%     john@email.johncon.com (John Conover)
%
% -----------------------------------------------------------------------------
%
% Revision: \RCSRevision \\
% Revision Time: \RCSTime UMT \\
% Revision Date: \RCSDate \\
% Revision Id: \RCSId \\
% Revision File: \RCSLog \\
\RCS $Revision: 0.0 $
\RCS $Date: 2006/01/20 04:38:13 $
\RCS $Id: operations.tex,v 0.0 2006/01/20 04:38:13 john Exp $
% $Log: operations.tex,v $
% Revision 0.0  2006/01/20 04:38:13  john
% Initial version
%
%
    \subsection{Fixed Increment Approximation for Operational Strategy}
        \label{\SETLABEL:OPS}.

        This section derives various values based on the ``average''
        of the normalized increments presented in
        Figure~\ref{\SETLABEL:TFA}. These values are an approximation
        to a, probably, complex process with a distribution shown in
        Figure~\ref{\SETLABEL:TF}. These values will be used in a
        fixed increment Brownian fractal analysis and simulation of
        the {\market}, and may, or may not, provide adequate accuracy
        for projections.

        \subidx{\market}{fiscal strategy}
        \subidx{\market}{Shannon probability}
        \subidx{strategy}{fiscal}
        \subidx{fiscal}{strategy}
        \subidx{Shannon}{probability}
        \subidx{probability}{Shannon}
        It should be noted that the analysis of fiscal strategy,
        presented in Section~\ref{\SETLABEL:FS}, is derived from the
        {\market} metrics and may, or may not, be maximally
        optimal. For the optimal fiscal strategy, which may be
        exploitable, see Section~\ref{\SETLABEL:MAXSHANNON}.

        \subidx{strategy}{exploitable}
        \subidx{exploitable}{strategy}
        \subidx{\market}{windows of opportunity}
        \idx{windows of opportunity}
        \subidx{decision}{obsolete}
        \subidx{obsolete}{decision}
        \subidx{decision}{timeliness}
        \subidx{timeliness}{decision}
        \subidx{rate of revenue returns}{forecast}
        \subidx{forecast}{rate of revenue returns}
        An additional exploitable strategy may be time itself.
        Equations~\ref{\SETLABEL:V},~\ref{\SETLABEL:R},
        and,~\ref{\SETLABEL:MA}, are, essentially, metrics on how fast
        a decision, which is based on information concerning the
        current status of the {\market}, becomes obsolete. Obviously,
        how long a decision is expected to remain relevant should be
        addressed as an operational necessity in strategic planning
        and project management. Figures~\ref{\SETLABEL:FN},
        and,~\ref{\SETLABEL:FF} compare methods of approximation of
        the ``forecastability'' of rate of revenue returns in the
        {\market} for the near term and far
        term~\cite[pp. 83-84]{Peters:CAOITCM}, respectively. As a
        general rule, caution must be exercised when making decisions
        that will span a time interval larger than the time interval
        where the ``forecastability'' of rate of revenue returns drops
        below 50\%. Beyond this time interval, the chances increase
        that the competitive and market forces will alter the market
        environment in a possibly detrimental unanticipated
        fashion. Obviously, there is significant advantage in
        ``timeliness'' of development, manufacturing, and distribution
        of products and services that are consistent with this
        temporal agenda. Automation of these processes, if executed
        consistently with this agenda, should be considered a
        competitive advantage.

        \subidx{strategy}{exploitable}
        \subidx{exploitable}{strategy}
        \subidx{rate of revenue returns}{forecast}
        \subidx{forecast}{rate of revenue returns}
        \idx{product life cycle}
        \idx{life cycle, product}
        In some sense, this temporal agenda defines the ``average''
        product or service life cycle in the {\market}. When the
        ``forecastability'' of rate of revenue returns drops below
        50\%, there is an even chance that the rate of revenue returns
        for the product or service will change in a detrimental
        fashion. If it is assumed that a product or service life cycle
        consists of a ramp up, a maintenence interval, and a ramp
        down, then, if all three life cycle intervals are equal, the
        product life cycle will be, approximately, three times the
        time interval where the ``forecastability'' of rate of revenue
        returns drops below 50\%. Although probably not an accurate
        prediction of product or service life cycle, the technique may
        be used as a conceptual approximation to the dynamics of
        ``market windows.\footnote{For example, consider the market
        for table salt. Since it has inelastic supply and demand
        curves, and is a necessary requirement for life, it would be
        expected that the Hurst coefficient would be very near
        unity---ignoring competitive pressures in the market. The
        predictability of the table salt market would, therefore, be
        expected to be relatively good, over time.}''  The conceptual
        approximation will probably predict a ``conservative'' or
        ``pessimistic'' value in relation to actual markets.

        \begin{figure}[ht]
            \begin{center}
                \begin{minipage}[t]{0.45\textwidth}
                    \epsfxsize=1.0\linewidth
                    \epsffile{\directory/datahurstlownear.eps}
                    \caption[{\market}, ``forecastability'' of near
                        term rate of revenue returns]{{\market},
                        ``forecastability'' of near term rate of
                        revenue returns. Although the error function
                        is the most accurate, for the near term,
                        $H^{t} = \thurstlow^{t}$ may be used as a
                        reliable metric of ``forecastability'' of the
                        rate of revenue returns.}
                    \label{\SETLABEL:FN}
                \end{minipage}
                \hfill
                \begin{minipage}[t]{0.45\textwidth}
                    \epsfxsize=1.0\linewidth
                    \epsffile{\directory/datahurstlowfar.eps}
                    \caption[{\market}, ``forecastability'' of far
                        term rate of revenue returns]{{\market},
                        ``forecastability'' of far term rate of
                        revenue returns. Although the error function
                        is the most accurate, for the far term,
                        $\frac{1}{\sqrt{t}}$ may be used as a reliable
                        metric of ``forecastability'' of the rate of
                        revenue returns.}
                    \label{\SETLABEL:FF}
                \end{minipage}
            \end{center}
        \end{figure}

        \idx{operations research}
        As an interesting interpretation of the data presented in
        Figure~\ref{\SETLABEL:FN}, there may be, perhaps, some
        applicability to such operational agendas as inventory
        control. Maintaining too little inventory, obviously, will
        create a situation where the organization can not exploit
        market expansion, and maintaining too much inventory,
        likewise, would over extend the company, creating unnecessary
        losses when the market contracts. The company should maintain
        inventory levels that do not exceed, from
        Equation~\ref{\SETLABEL:MA}, ${\thurstlow}^{n} = 0.5$
        {\timescale}s of operations. Since the optimal amount of
        inventory and, from Equation~\ref{\SETLABEL:V}, the variance
        of change in the rate of revenue returns in the future can be
        calculated, there may, perhaps, be some applicability to a
        forecasting methodology that can be incorporated into other
        areas of operations research, for example the linear algebras
        using simplex methodologies for optimization of manufacturing
        processes. Traditionally, these forecasts are made by the
        sales department, and are subject to various subjective
        biases.

% Local Variables:
% TeX-parse-self: t
% TeX-auto-save: t
% TeX-master: "fractal.tex"
% End:


        %
% -----------------------------------------------------------------------------
%
% A license is hereby granted to reproduce this software source code and
% to create executable versions from this source code for personal,
% non-commercial use.  The copyright notice included with the software
% must be maintained in all copies produced.
%
% THIS PROGRAM IS PROVIDED "AS IS". THE AUTHOR PROVIDES NO WARRANTIES
% WHATSOEVER, EXPRESSED OR IMPLIED, INCLUDING WARRANTIES OF
% MERCHANTABILITY, TITLE, OR FITNESS FOR ANY PARTICULAR PURPOSE.  THE
% AUTHOR DOES NOT WARRANT THAT USE OF THIS PROGRAM DOES NOT INFRINGE THE
% INTELLECTUAL PROPERTY RIGHTS OF ANY THIRD PARTY IN ANY COUNTRY.
%
% Copyright (c) 1994-2006, John Conover, All Rights Reserved.
%
% Comments and/or bug reports should be addressed to:
%
%     john@email.johncon.com (John Conover)
%
% -----------------------------------------------------------------------------
%
% Revision: \RCSRevision \\
% Revision Time: \RCSTime UMT \\
% Revision Date: \RCSDate \\
% Revision Id: \RCSId \\
% Revision File: \RCSLog \\
\RCS $Revision: 0.0 $
\RCS $Date: 2006/01/20 04:38:13 $
\RCS $Id: simulation.tex,v 0.0 2006/01/20 04:38:13 john Exp $
% $Log: simulation.tex,v $
% Revision 0.0  2006/01/20 04:38:13  john
% Initial version
%
%
    \subsection{Simulation of Fixed Increment Approximation for Fiscal Strategy}
        \label{\SETLABEL:TSUNFAIRBROWNIAN}

        \subidx{\market}{market simulation}
        The data in this section is presented in tabular form in
        Section~\ref{\SETLABELREF:SIM}.
        Figure~\ref{\SETLABEL:TSUNFAIRBROWNIAN0} represents a
        constructional simulation of the time series data presented in
        Figure~\ref{\SETLABEL:TS}. The program {\it
        tsunfairbrownian}\/, which is briefly described in
        appendix~\ref{programs}, was used in the reconstruction. The
        reconstructed data is superimposed on the original time series
        data.  The program, {\it tsunfairbrownian}\/, essentially,
        constructs the new time series as a Brownian fractal with
        fixed increments---the value of the fixed increment is derived
        from the root mean square average of the normalized increments
        presented in Figure~\ref{\SETLABEL:TF}. The ``quality'' of
        such a reconstruction should be subject to adequate scepticism
        and scrutiny since, in all probability, the normalized
        increments presented in Figure~\ref{\SETLABEL:TF} represent a
        relatively complex process, that may not be ``modeled'' with
        such a simple methodology.

        As a further comparison of the the constructional simulation
        with the original time series data,
        Figure~\ref{\SETLABEL:TSUNFAIRBROWNIAN1} presents a normalized
        histogram of the normalized increments of the reconstructed
        time series, superimposed on the normalized histogram
        presented in Figure~\ref{\SETLABEL:NH}.

        \subidx{\market}{fiscal strategy, simulation}
        \subidx{markets}{simulation}
        \subidx{simulation}{markets}
        \subidx{strategy}{fiscal, simulation}
        \subidx{fiscal}{strategy, simulation}
        \subidx{programs}{tsunfairbrownian}
        \subidx{tsunfairbrownian}{program}
        \begin{figure}[ht]
            \begin{center}
                \begin{minipage}[t]{0.45\textwidth}
                    \epsfxsize=1.0\linewidth
                    \epsffile{\directory/tsunfairbrownian-f.eps}
                    \caption[{\market}, Time series data, empirical and
                        simulated]{{\market}, Time series data, empirical
                        and simulated, using the program {\it tsunfairbrownian}\/
                        with f = {\datafractionrms}. This data is
                        superimposed on the data presented in
                        Figure~\ref{\SETLABEL:TS}.}
                    \label{\SETLABEL:TSUNFAIRBROWNIAN0}
                \end{minipage}
                \hfill
                \begin{minipage}[t]{0.45\textwidth}
                    \epsfxsize=1.0\linewidth
                    \epsffile{\directory/tsunfairbrownian-f.tsfraction.tsnormal-s30.eps}
                    \caption[{\market}, normalized histogram,
                        empirical and simulated]{{\market}, normalized
                        histogram of the normalized increments of the
                        time series data shown in
                        Figure~\ref{\SETLABEL:TSUNFAIRBROWNIAN0},
                        empirical and simulated.  The empirical data
                        has a mean of {\datafractionmean}, with a
                        standard deviation of {\datafractionstddev}.
                        By comparison, the simulated data has a mean
                        of {\tsunfairbrownianfractionmean} with a
                        standard deviation of
                        {\tsunfairbrownianfractionstddev}. This data
                        is superimposed on the data presented in
                        Figure~\ref{\SETLABEL:NH}. The area under the
                        four curves is identical.}
                    \label{\SETLABEL:TSUNFAIRBROWNIAN1}
                \end{minipage}
            \end{center}
        \end{figure}

% Local Variables:
% TeX-parse-self: t
% TeX-auto-save: t
% TeX-master: "fractal.tex"
% End:


        %
% -----------------------------------------------------------------------------
%
% A license is hereby granted to reproduce this software source code and
% to create executable versions from this source code for personal,
% non-commercial use.  The copyright notice included with the software
% must be maintained in all copies produced.
%
% THIS PROGRAM IS PROVIDED "AS IS". THE AUTHOR PROVIDES NO WARRANTIES
% WHATSOEVER, EXPRESSED OR IMPLIED, INCLUDING WARRANTIES OF
% MERCHANTABILITY, TITLE, OR FITNESS FOR ANY PARTICULAR PURPOSE.  THE
% AUTHOR DOES NOT WARRANT THAT USE OF THIS PROGRAM DOES NOT INFRINGE THE
% INTELLECTUAL PROPERTY RIGHTS OF ANY THIRD PARTY IN ANY COUNTRY.
%
% Copyright (c) 1994-2006, John Conover, All Rights Reserved.
%
% Comments and/or bug reports should be addressed to:
%
%     john@email.johncon.com (John Conover)
%
% -----------------------------------------------------------------------------
%
% Revision: \RCSRevision \\
% Revision Time: \RCSTime UMT \\
% Revision Date: \RCSDate \\
% Revision Id: \RCSId \\
% Revision File: \RCSLog \\
\RCS $Revision: 0.0 $
\RCS $Date: 2006/01/20 04:38:13 $
\RCS $Id: maximum.tex,v 0.0 2006/01/20 04:38:13 john Exp $
% $Log: maximum.tex,v $
% Revision 0.0  2006/01/20 04:38:13  john
% Initial version
%
%
    \subsection{Simulation of Fixed Increment Approximation for Optimally Maximal Fiscal Strategy}
        \label{\SETLABEL:MAXSHANNON}
        \subidx{\market}{fiscal strategy, simulation}
        \subidx{\market}{maximum Shannon probability}
        \subidx{markets}{simulation}
        \subidx{simulation}{markets}
        \subidx{strategy}{optimum fiscal, simulation}
        \subidx{fiscal}{optimum strategy, simulation}
        \subidx{programs}{tsunfairbrownian}
        \subidx{tsunfairbrownian}{program}
        \subidx{Shannon}{probability}
        \subidx{probability}{Shannon}

        \subidx{strategy}{exploitable}
        \subidx{exploitable}{strategy}
        \subidx{programs}{tsshannonmax}
        \subidx{tsshannonmax}{program}
        \subidx{programs}{tsunfairbrownian}
        \subidx{tsunfairbrownian}{program}
        \subidx{strategy}{fiscal}
        \subidx{fiscal}{strategy}
        The data in this section is presented in tabular form in
        Section~\ref{\SETLABELREF:MAXSHANNON}. One of the issues of
        analysis, as mentioned in Section~\ref{\SETLABEL:OPS}, is to
        determine the maximum Shannon probability for the time series
        presented in Figure~\ref{\SETLABEL:TS}. Potentially, this
        could be exploited with an aggressive fiscal
        strategy. Figure~\ref{\SETLABEL:SHANNONMAX0} is a graph of the
        output of the {\it tsshannonmax}\/ program, which is described
        briefly in appendix~\ref{programs}. The maximum of this
        function is the maximum Shannon probability for the time
        series data presented in Figure~\ref{\SETLABEL:TS}.
        Figure~\ref{\SETLABEL:SHANNONMAX1} was constructed using {\it
        tsunfairbrownian}\/ program, which is also described in
        appendix~\ref{programs}, with the maximum Shannon probability,
        and the time series data presented in
        Figure~\ref{\SETLABEL:TS}. This represents a ``what if'' the
        investment strategy was changed from a Shannon probability of
        {\shannonlogreturns}, as derived in Section~\ref{\SETLABEL:SP}
        to {\shannonmax}. This process, essentially, extracts the
        random statistical data from the time series presented in
        Figure~\ref{\SETLABEL:TS}, and constructs a new time series,
        using the random statistical data, with a different investment
        strategy.  The program, {\it tsunfairbrownian}\/, essentially,
        constructs the new time series as a Brownian fractal with
        fixed increments.  The ``quality'' of such a reconstruction
        should be subject to adequate scepticism and scrutiny since,
        in all probability, the increments in the original data
        represent a relatively complex process, that may not be
        ``modeled'' with such a simple methodology.

        \begin{figure}[ht]
            \begin{center}
                \begin{minipage}[t]{0.45\textwidth}
                    \epsfxsize=1.0\linewidth
                    \epsffile{\directory/data.tsshannonmax.eps}
                    \caption[{\market}, maximum rate of revenue
                        returns] {{\market}, maximum rate of revenue
                        returns, per {\timescale}, vs. Shannon
                        probability. The maximum rate of revenue
                        returns, per {\timescale}, occurs at a Shannon
                        probability of {\shannonmax}.}
                    \label{\SETLABEL:SHANNONMAX0}
                \end{minipage}
                \hfill
                \begin{minipage}[t]{0.45\textwidth}
                    \epsfxsize=1.0\linewidth
                    \epsffile{\directory/data.tsshannonmax-p.tsunfairbrownian-p.eps}
                    \caption[{\market}, maximum rate of revenue
                        returns] {{\market}, maximum rate of revenue
                        returns, per {\timescale}, at a Shannon
                        probability, of {\shannonmax}, corresponding
                        to a ``wager'' fraction of {\twoponemax}.}
                    \label{\SETLABEL:SHANNONMAX1}
                \end{minipage}
            \end{center}
        \end{figure}

        \subidx{fractional}{Brownian motion}
        \subidx{Brownian motion}{fractional}
        \subidx{Shannon}{probability}
        \subidx{probability}{Shannon}
        \subidx{programs}{tsshannonmax}
        \subidx{tsshannonmax}{program}
        If it is assumed that the time series data set, presented in
        Figure~\ref{\SETLABEL:TS}, constitutes classical Brownian
        motion, then the Shannon probability can be calculated by
        counting the total number of {\timescale}s that the {\market}
        movement was positive, and dividing by the total number of
        {timescale}s represented in the time series. This quotient is
        {\pmax}, as compared with the predicted value from the program
        {\it tsshannonmax}\/ of {\shannonmax}.

% Local Variables:
% TeX-parse-self: t
% TeX-auto-save: t
% TeX-master: "fractal.tex"
% End:


        %
% -----------------------------------------------------------------------------
%
% A license is hereby granted to reproduce this software source code and
% to create executable versions from this source code for personal,
% non-commercial use.  The copyright notice included with the software
% must be maintained in all copies produced.
%
% THIS PROGRAM IS PROVIDED "AS IS". THE AUTHOR PROVIDES NO WARRANTIES
% WHATSOEVER, EXPRESSED OR IMPLIED, INCLUDING WARRANTIES OF
% MERCHANTABILITY, TITLE, OR FITNESS FOR ANY PARTICULAR PURPOSE.  THE
% AUTHOR DOES NOT WARRANT THAT USE OF THIS PROGRAM DOES NOT INFRINGE THE
% INTELLECTUAL PROPERTY RIGHTS OF ANY THIRD PARTY IN ANY COUNTRY.
%
% Copyright (c) 1994-2006, John Conover, All Rights Reserved.
%
% Comments and/or bug reports should be addressed to:
%
%     john@email.johncon.com (John Conover)
%
% -----------------------------------------------------------------------------
%
% Revision: \RCSRevision \\
% Revision Time: \RCSTime UMT \\
% Revision Date: \RCSDate \\
% Revision Id: \RCSId \\
% Revision File: \RCSLog \\
\RCS $Revision: 0.0 $
\RCS $Date: 2006/01/20 04:38:13 $
\RCS $Id: verification.tex,v 0.0 2006/01/20 04:38:13 john Exp $
% $Log: verification.tex,v $
% Revision 0.0  2006/01/20 04:38:13  john
% Initial version
%
%
    \subsection{Qualitative Verification of Fixed Increment Approximation Analysis}
        \label{\SETLABEL:QVA}

        \subidx{\market}{verification of analysis}
        \subidx{verification}{analysis}
        \subidx{analysis}{verification}
        \subidx{quality}{of analysis}
        \subidx{verification}{of methodology}
        \subidx{methodology}{verification of}
        \subidx{Shannon}{probability}
        \subidx{probability}{Shannon}

        This section evaluates various values based on the ``average''
        of the normalized increments presented in
        Figure~\ref{\SETLABEL:TFA}. These values are an approximation
        to a, probably, complex process with a distribution shown in
        Figure~\ref{\SETLABEL:TF}. These values will be used in a
        fixed increment Brownian fractal analysis of the {\market},
        and may, or may not, provide adequate accuracy for
        projections.

        The data in this section is presented in tabular form in
        sections~\ref{\SETLABELREF:VI1} and~\ref{\SETLABELREF:VI2}.
        As a subjective evaluation of the ``quality'' of the analysis
        of the {\market}, from Chapter~\ref{methodology},
        Equation~\ref{metricvalues1}, and using the mean and root mean
        square values of the normalized increments of the time series
        data presented in Figure~\ref{\SETLABEL:TS} from
        Figure~\ref{\SETLABEL:TF}, and the Shannon probability as
        calculated by counting the total number of {\timescale}s that
        the {\market} movement was positive, as presented in
        Section~\ref{\SETLABEL:MAXSHANNON}:

        \begin{eqnarray}
                  P & \approx & \frac{\frac{avg}{rms} + 1}{2}\\
            {\pmax} & \approx & \frac{\frac{\datafractionmean}{\datafractionrms} + 1}{2}\\
            {\pmax} & \approx & {\avgrms}
            \label{\SETLABEL:AVGS}
        \end{eqnarray}

        \subidx{Shannon}{probability}
        \subidx{probability}{Shannon}
        \noindent and comparing these values to the Shannon
        probability, as found by the {\it tsshannonmax}\/ program, which
        iterates for a maximum:

        \begin{eqnarray}
            {\pmax} \approx {\avgrms} \approx {\shannonmax}
        \end{eqnarray}

        \subidx{logarithmic}{returns}
        \subidx{returns}{logarithmic}
        In addition, the different methods of calculating the
        logarithmic returns, presented in Section~\ref{\SETLABEL:FS},
        should be compared. The four methods used were the mean of
        Figure~\ref{\SETLABEL:TF}, the constant in the least squares
        approximation to Figure~\ref{\SETLABEL:TF}, the least squares
        exponential approximation to Figure~\ref{\SETLABEL:TS}, and
        the logarithmic returns of Figure~\ref{\SETLABEL:TS}, derived
        as the mean of the logarithms of the quotients of the
        increments. The values for each of the methods are,
        respectively:

        \begin{equation}
            \datafractionmeanbits \approx \datafractionconstantbits \approx \datatslsqepbits \approx \logreturns
        \end{equation}

        It is implied in Section~\ref{\SETLABEL:FS},
        Subsection~\ref{\SETLABEL:SP} and in
        Section~\ref{\SETLABEL:TSUNFAIRBROWNIAN} that, a Brownian
        motion with fixed increments fractal may ``model'' the
        {\market}. Using Equation~\ref{stddev9} from
        Chapter~\ref{general}, Section~\ref{abmfi}:

        \begin{eqnarray}
                                    rms \left(2P - 1\right) & \approx & \frac{\sigma \left(2P - 1\right)}{2 \sqrt{P\left(1 - P\right)}}\\
            \datafractionrms \left(2 \cdot \pmax - 1\right) & \approx & \frac{\datafractionstddev \left(2 \cdot \pmax - 1\right)}{2\sqrt{\pmax \left(1 - \pmax\right)}}\\
                       \datafractionrms \cdot \twopminusone & \approx & \datafractionstddev \cdot \twopx\\
                                                      \rmsp & \approx & \sigmap
        \end{eqnarray}

        \noindent and, equating to the mean:

        \begin{equation}
            \datafractionmean \approx \rmsp \approx \sigmap
        \end{equation}

        \subidx{Shannon}{probability}
        \subidx{probability}{Shannon}
        \noindent where, as in Equation~\ref{\SETLABEL:AVGS} using the
        mean, root mean square, and standard deviation values of the
        normalized increments of the time series data presented in
        Figure~\ref{\SETLABEL:TS} from Figure~\ref{\SETLABEL:TF}, and
        the Shannon probability as calculated by counting the total
        number of {\timescale}s that the {\market} movement was
        positive, as presented in Section~\ref{\SETLABEL:MAXSHANNON}.

        As a final qualitative comparison, the absolute value of the
        normalized increments should be the same as the root mean
        square value\footnote{The absolute value of the normalized
        increments, when averaged, is related to the root mean square
        of the increments by a constant. If the normalized increments
        are a fixed increment, the constant is unity. If the
        normalized increments have a Gaussian distribution, the
        constant is $\approx 0.8$ depending on the accuracy of of
        ``fit'' to a Gaussian distribution.}, where the absolute value
        is presented in Figure~\ref{\SETLABEL:TFA}, and the root mean
        square value is presented in Figure~\ref{\SETLABEL:TF}:

        \begin{equation}
            \datafractionabsmean \approx \datafractionrms
        \end{equation}

        Note, that if the {\market} could be ``modeled'' as a Brownian
        motion with fixed increments fractal, then the standard
        deviation of the absolute value of the normalized increments
        of the time series data presented in Figure~\ref{\SETLABEL:TS}
        from Figure~\ref{\SETLABEL:TF} should be zero. It is
        $\datafractionabsstddev$.

% Local Variables:
% TeX-parse-self: t
% TeX-auto-save: t
% TeX-master: "fractal.tex"
% End:


    \renewcommand{\market}{Coin Tossing Game}
    \renewcommand{\directory}{../markets/tscoin}
    \renewcommand{\datafractionmean}{0.008052}
\renewcommand{\datafractionmeanbits}{0.011570}
\renewcommand{\datafractionmeanq}{0.002684}
\renewcommand{\datafractionmeanbitsq}{0.003867}
\renewcommand{\datafractionstddev}{0.038579}
\renewcommand{\datafractionrms}{0.039311}
\renewcommand{\avgrms}{0.602414}
\renewcommand{\ncompanies}{5.210454}
\renewcommand{\pncompanies}{0.544866}
\renewcommand{\datafractionabsmean}{0.029745}
\renewcommand{\datafractionabsstddev}{0.025769}
\renewcommand{\datafractionconstant}{0.010041}
\renewcommand{\datafractionconstantbits}{0.014414}
\renewcommand{\datafractionconstantq}{0.003347}
\renewcommand{\datafractionconstantbitsq}{0.004821}
\renewcommand{\datafractionslope}{-0.000021}
\renewcommand{\datafractionabsconstant}{0.035145}
\renewcommand{\datafractionabsslope}{-0.000057}
\renewcommand{\hurstall}{0.659558}
\renewcommand{\hurstlow}{0.707509}
\renewcommand{\hurstlowtwo}{1.415018}
\renewcommand{\hurstlowhundred}{70.750900}
\renewcommand{\hcalcall}{0.184942}
\renewcommand{\hcalclow}{0.102042}
\renewcommand{\shannonmax}{0.604167}
\renewcommand{\twoponemax}{0.208334}
\renewcommand{\logreturns}{0.010456}
\renewcommand{\twologreturns}{1.007274}
\renewcommand{\twologreturnshundred}{0.727387}
\renewcommand{\oneoverlogreturns}{95.638868}
\renewcommand{\pmax}{0.602094}
\renewcommand{\twopminusone}{0.204188}
\renewcommand{\rmsp}{0.008027}
\renewcommand{\twopx}{0.208583}
\renewcommand{\sigmap}{0.008047}
\renewcommand{\tsunfairbrownianfractionmean}{0.007862}
\renewcommand{\tsunfairbrownianfractionstddev}{0.038619}
\renewcommand{\shannonlogreturns}{0.560125}
\renewcommand{\shannonlogreturnshundred}{56.012500}
\renewcommand{\twopone}{0.120250}
\renewcommand{\twoponehundred}{12.025000}
\renewcommand{\hundredtwoponehundred}{87.975000}
\renewcommand{\hundredshannonlogreturnshundred}{43.987500}
\renewcommand{\datatslsqepbits}{0.007623}
\renewcommand{\thurstall}{0.633980}
\renewcommand{\thurstlow}{0.710108}
\renewcommand{\thurstlowtwo}{1.420216}
\renewcommand{\thurstlowhundred}{71.010800}
\renewcommand{\thcalcall}{0.247886}
\renewcommand{\thcalclow}{0.171737}
\renewcommand{\chisquared}{2.862000}
\renewcommand{\critical}{42.557000}

    \renewcommand{\timescale}{tosses}
    \subidx{market}{\market}
    \idx{\market}

    \section{\market}

        \renewcommand{\SETLABEL}{\LABPRE:CT}
        \renewcommand{\SETLABELQ}{\LABPRE:CTQ}
        \label{\SETLABEL}
        \renewcommand{\SETLABELREF}{\LABPREREF:CT}

        \subidx{tscoin}{program}
        \subidx{programs}{tscoin}
        For the analysis, the data was in the directory
        {\directory}\footnote{As a simulation model, the program {\it
        tscoin}\/ was run to make a time series data file, with the
        following parameters:

        \vspace{0.1in}
        {\noindent}tscoin -p 0.6 300 > data
        \vspace{0.1in}

        \noindent to make a time series of 300 elements, with a
        Shannon probability of 0.6.  The data is by {\timescale}.}.

        The data in this section is presented in tabular form in
        Section~\ref{\SETLABELREF}. Note that in this analysis, the
        rate of revenue returns means the increase or decrease in the
        cumulative sum of the {\market}. This is included for
        ``theoretical'' comparative purposes.

        %
% -----------------------------------------------------------------------------
%
% A license is hereby granted to reproduce this software source code and
% to create executable versions from this source code for personal,
% non-commercial use.  The copyright notice included with the software
% must be maintained in all copies produced.
%
% THIS PROGRAM IS PROVIDED "AS IS". THE AUTHOR PROVIDES NO WARRANTIES
% WHATSOEVER, EXPRESSED OR IMPLIED, INCLUDING WARRANTIES OF
% MERCHANTABILITY, TITLE, OR FITNESS FOR ANY PARTICULAR PURPOSE.  THE
% AUTHOR DOES NOT WARRANT THAT USE OF THIS PROGRAM DOES NOT INFRINGE THE
% INTELLECTUAL PROPERTY RIGHTS OF ANY THIRD PARTY IN ANY COUNTRY.
%
% Copyright (c) 1994-2006, John Conover, All Rights Reserved.
%
% Comments and/or bug reports should be addressed to:
%
%     john@email.johncon.com (John Conover)
%
% -----------------------------------------------------------------------------
%
% Revision: \RCSRevision \\
% Revision Time: \RCSTime UMT \\
% Revision Date: \RCSDate \\
% Revision Id: \RCSId \\
% Revision File: \RCSLog \\
\RCS $Revision: 0.0 $
\RCS $Date: 2006/01/20 04:38:13 $
\RCS $Id: fraction.tex,v 0.0 2006/01/20 04:38:13 john Exp $
% $Log: fraction.tex,v $
% Revision 0.0  2006/01/20 04:38:13  john
% Initial version
%
%
    \subsection{Time Series Increments Analysis}
        \label{\SETLABEL:TSA}

        \subidx{\market}{Time series analysis}
        \subidx{time series}{increments}
        \subidx{time series}{analysis}
        \subidx{cumulative sum}{analysis}
        \subidx{analysis}{cumulative sum}
        \subidx{analysis}{random process}
        \subidx{random process}{analysis}
        \subidx{Gaussian}{increments}
        \subidx{increments}{Gaussian}
        \subidx{Brownian}{motion, fractional}
        \subidx{fractional}{Brownian motion}
        \subidx{fractal}{Brownian motion}
        The data in this section is presented in tabular form in
        Section~\ref{\SETLABELREF:TSA}.  Figure~\ref{\SETLABEL:TS} is
        a graph of the time series data for the {\market}.

        \subidx{increments}{normalized}
        \subidx{normalized}{increments}
        \subidx{programs}{tsfraction}
        \subidx{tsfraction}{program}
        Figure~\ref{\SETLABEL:TF} is a graph of the normalized
        increments of the time series data presented in
        Figure~\ref{\SETLABEL:TS}. The data presented was made by
        running the program {\it tsfraction}\/ on the time series
        data. The program {\it tsfraction}\/ is described briefly in
        Appendix~\ref{programs}, and subtracts the previous value from
        the next value, dividing this difference by the previous
        value, for each element in the time series data. The new time
        series contains the instantaneous change in the rate of
        revenue returns, divided by the magnitude of the instantaneous
        rate of revenue returns.

        \subidx{mean}{standard deviation}
        \subidx{standard deviation}{mean}
        \idx{root mean square}
        \idx{least squares approximation}
        \begin{figure}[ht]
            \begin{center}
                \begin{minipage}[t]{0.45\textwidth}
                    \epsfxsize=1.0\linewidth
                    \epsffile{\directory/data.eps}
                    \caption{{\market}, time series data.}
                    \label{\SETLABEL:TS}
                    \label{\SETLABELQ:TS}
                \end{minipage}
                \hfill
                \begin{minipage}[t]{0.45\textwidth}
                    \epsfxsize=1.0\linewidth
                    \epsffile{\directory/data.tsfraction.eps}
                    \caption[{\market}, normalized
                        increments]{{\market}, normalized increments
                        of the time series data presented in
                        Figure~\ref{\SETLABEL:TS}. The mean is
                        {\datafractionmean} with a standard deviation
                        of {\datafractionstddev}. The formula for the
                        least squares approximation is
                        ${\datafractionconstant} +
                        {\datafractionslope}t$, and the root mean
                        squared value is {\datafractionrms}. The
                        graph, labeled ``data\-.tsfraction\-.tsrms,''
                        is the running root mean square, and
                        ``data\-.tsfraction\-.tsavg'' is the running
                        average of the normalized increments.  This
                        graph is the fraction of change in the time
                        series, as a function of time. Note that the
                        slope of the mean, {\datafractionslope}, is
                        the coefficient of the nonlinearity term in
                        the normalized increments. See
                        Chapter~\ref{general}, Section~\ref{nlextend}
                        for a possible application of the logistic
                        function to this data set.}
                    \label{\SETLABEL:TF}
                    \label{\SETLABELQ:TF}
                \end{minipage}
            \end{center}
        \end{figure}

        \subidx{absolute value}{increments}
        \subidx{increments}{absolute value}

        Figure~\ref{\SETLABEL:TFA} is a graph of the absolute value of
        the normalized increments of the time series data presented in
        Figure~\ref{\SETLABEL:TF}. The data presented was made by
        running the Unix utility sed(1) on the normalized increments
        time series data to remove the negative signs. This is an
        absolute value procedure.  The resulting time series contains
        the absolute value of the instantaneous change in the rate of
        revenue returns, divided by the magnitude of the instantaneous
        rate of revenue returns\footnote{The absolute value of the
        normalized increments, when averaged, is related to the root
        mean square of the increments by a constant. If the normalized
        increments are a fixed increment, the constant is unity. If
        the normalized increments have a Gaussian distribution, the
        constant is $\approx 0.8$ depending on the accuracy of of
        ``fit'' to a Gaussian distribution.}.

        \subidx{histogram}{normalized}
        \subidx{normalized}{histogram}
        \subidx{programs}{tsnormal}
        \subidx{tsnormal}{program}
        \subidx{mean}{standard deviation}
        \subidx{standard deviation}{mean}
        \idx{root mean square}
        \idx{least squares approximation}
        \subidx{\market}{analysis of increments}
        Figure~\ref{\SETLABEL:NH} is the normalized histogram of the
        normalized increments of the time series data shown in
        Figure~\ref{\SETLABEL:TF}. The abscissa is 3 $\sigma$ limits,
        and the area under the two curves is identical. The data for
        this figure was produced by the program {\it tsnormal}\/,
        which is described briefly in Appendix~\ref{programs}.

        \begin{figure}[ht]
            \begin{center}
                \begin{minipage}[t]{0.45\textwidth}
                    \epsfxsize=1.0\linewidth
                    \epsffile{\directory/data.tsfraction.abs.eps}
                    \caption[{\market}, absolute value of the
                        normalized increments]{{\market}, absolute
                        value of the normalized increments of the time
                        series data presented in
                        Figure~\ref{\SETLABEL:TF}.  The mean is
                        {\datafractionabsmean} with a standard
                        deviation of {\datafractionabsstddev}. The
                        formula for the least squares approximation is
                        ${\datafractionabsconstant} +
                        {\datafractionabsslope}t$, and the root mean
                        square value, from Figure~\ref{\SETLABEL:TF},
                        is {\datafractionrms}.  The graph, labeled
                        ``data\-.tsfraction\-.tsrms,'' is the running
                        root mean square, and
                        ``data\-.tsfraction\-.tsavg'' is the running
                        average of the normalized increments presented
                        in Figure~\ref{\SETLABEL:TF}, superimposed
                        here for convenience. This graph is the
                        absolute value of the fraction of change in
                        the time series, as a function of time.}
                    \label{\SETLABEL:TFA}
                    \label{\SETLABELQ:TFA}
                \end{minipage}
                \hfill
                \begin{minipage}[t]{0.45\textwidth}
                    \epsfxsize=1.0\linewidth
                    \epsffile{\directory/data.tsfraction.tsnormal-s30.eps}
                    \caption[{\market}, normalized histogram of the
                        normalized increments]{{\market}, normalized
                        histogram of the normalized increments of the
                        time series data shown in
                        Figure~\ref{\SETLABEL:TF}.  The data has a
                        mean of {\datafractionmean}, with a standard
                        deviation of {\datafractionstddev}.  The area
                        under the two curves is identical. The
                        $\chi^2$ value of the observed and expected
                        values of the two curves is {\chisquared},
                        with a critical value of {\critical}.}
                    \label{\SETLABEL:NH}
                \end{minipage}
            \end{center}
        \end{figure}

        \subidx{programs}{tsXsquared}
        \subidx{tsXsquared}{program}
        \subidx{\market}{chi-squared values of increments}
        The program {\it tsXsquared}\/, which is briefly described in
        appendix~\ref{programs}, was used to derive the $\chi^2$
        statistics for the data presented in
        Figure~\ref{\SETLABEL:NH}.

        \subidx{programs}{tsstatest}
        \subidx{tsstatest}{program}
        \subidx{\market}{statistical estimates}

        Figure~\ref{\SETLABEL:SE} is the statistical estimate for the
        data presented in Figure~\ref{\SETLABEL:TF}, as derived by the
        program {\it tsstatest}\/, which is briefly described in
        appendix~\ref{programs}.

        \begin{figure}[ht]
            \begin{center}
                \begin{minipage}[t]{\textwidth}
                    \center{\fbox{\parbox{0.9\textwidth}{\XXX{\directory/data.tsstatest-f0.1-c0.9-i.tex}}}}
                    \caption[{\market}, statistical estimates of the
                        normalized increments]{{\market}, statistical
                        estimates of the normalized increments of the
                        time series shown in Figure~\ref{\SETLABEL:TF}.
                        The table was produced with the {\it
                        tsstatest}\/ program, and illustrates the
                        size of the data set required for a confidence
                        level of 90\%, with an error estimate of $\pm$
                        10\%, or alternately, the error estimate on
                        the time series shown in Figure~\ref{\SETLABEL:TF}.}
                    \label{\SETLABEL:SE}
                \end{minipage}
            \end{center}
        \end{figure}

        Note that the data set size estimations, as produced by the
        {\it tsstatest}\/ program, are probably very conservative,
        depending on the magnitude of the Shannon probability, $P =
        \shannonlogreturns$, as derived in
        Section~\ref{\SETLABEL:SP}. See Chapter~\ref{general},
        Section~\ref{serdss} for possible alternative methodologies
        for addressing the analysis of fractal time series with
        limited data set sizes. Depending on the magnitude of the
        Shannon probability, $P$, these estimates can be several
        orders of magnitude too high.

        \subidx{derivative of increments}{normalized}
        \subidx{normalized}{derivative of increments}
        \subidx{programs}{tsderivative}
        \subidx{tsderivative}{program}
        Figure~\ref{\SETLABEL:TF1} is the normalized histogram of the
        first derivative of the normalized increments of the time
        series data shown in Figure~\ref{\SETLABEL:TF}. In principle,
        if the distribution of the normalized increments presented in
        Figure~\ref{\SETLABEL:NH} is Gaussian in nature, this
        distribution would be similar to ``white noise,'' as presented
        in appendix~\ref{programs}, Figure~\ref{whiteexample}. The
        data was generated by the {\it tsderivative}\/ program, which
        is briefly described in
        appendix~\ref{programs}. Figure~\ref{\SETLABEL:TF2} is the
        normalized histogram of the second derivative of the
        normalized increments of the time series data shown in
        Figure~\ref{\SETLABEL:TF}. In principle, if the distribution
        of the normalized increments presented in
        Figure~\ref{\SETLABEL:NH} is an integrated Gaussian
        distribution in nature, this distribution would be similar to
        ``white noise,'' as presented in appendix~\ref{programs},
        Figure~\ref{whiteexample}.

        \begin{figure}[ht]
            \begin{center}
                \begin{minipage}[t]{0.45\textwidth}
                    \epsfxsize=1.0\linewidth
                    \epsffile{\directory/data.tsfraction.tsderivative.tsnormal-s30.eps}
                    \caption[{\market}, histogram of the first
                        derivative of the increments]{{\market},
                        normalized histogram of the first derivative
                        of the normalized increments of the time
                        series data shown in
                        Figure~\ref{\SETLABEL:TF}.}
                    \label{\SETLABEL:TF1}
                \end{minipage}
                \hfill
                \begin{minipage}[t]{0.45\textwidth}
                    \epsfxsize=1.0\linewidth
                    \epsffile{\directory/data.tsfraction.2tsderivative.tsnormal-s30.eps}
                    \caption[{\market}, histogram of the second
                        derivative of the increments]{{\market},
                        normalized histogram of second derivative of
                        the the normalized increments of the time
                        series data shown in
                        Figure~\ref{\SETLABEL:TF}.}
                    \label{\SETLABEL:TF2}
                \end{minipage}
            \end{center}
        \end{figure}

        \subidx{fractal}{range}
        \subidx{fractal}{R/S analysis}
        \subidx{\market}{rate of revenue returns, range}
        \subidx{\market}{deterministic mechanism}
        \subidx{deterministic}{mechanism}
        \subidx{mechanism}{deterministic}
        Figure~\ref{\SETLABEL:TR} is the range of values of the time
        series shown in Figure~\ref{\SETLABEL:TS}. The horizontal axis
        is time into the future. In principle, if the time series was
        characterized as fractional Brownian motion the graph in
        Figure~\ref{\SETLABEL:TR} would be a square root
        function\footnote{Note that the ``roughness,'' or ``sawtooth''
        characteristics of the graph in Figure~\ref{\SETLABEL:TR} are
        a computational artifact---caused by not using the -m option
        to the program {\it tshurst}\/, which is computationally
        inefficient.}. Figure~\ref{\SETLABEL:TD} is the deterministic
        map of the normalized increments of the time series data shown
        in Figure~\ref{\SETLABEL:TF}. The deterministic map is useful
        for determining if a time series was created by a
        deterministic mechanism. This, essentially, maps each element
        in the time series with the previous element in the time
        series.  See,~\cite[pp. 745]{Peitgen}.

        \begin{figure}[ht]
            \begin{center}
                \begin{minipage}[t]{0.45\textwidth}
                    \epsfxsize=1.0\linewidth
                    \epsffile{\directory/data.tshurst-f.eps}
                    \caption[{\market}, range]{{\market}, range of the
                        time series data shown in
                        Figure~\ref{\SETLABEL:TS}.}
                    \label{\SETLABEL:TR}
                \end{minipage}
                \hfill
                \begin{minipage}[t]{0.45\textwidth}
                    \epsfxsize=1.0\linewidth
                    \epsffile{\directory/data.tsfraction.tsdeterministic.eps}
                    \caption[{\market}, deterministic map]{{\market},
                        deterministic map of the normalized increments
                        of the time series data shown in
                        Figure~\ref{\SETLABEL:TF}.}
                    \label{\SETLABEL:TD}
                \end{minipage}
            \end{center}
        \end{figure}

% Local Variables:
% TeX-parse-self: t
% TeX-auto-save: t
% TeX-master: "fractal.tex"
% End:


        %
% -----------------------------------------------------------------------------
%
% A license is hereby granted to reproduce this software source code and
% to create executable versions from this source code for personal,
% non-commercial use.  The copyright notice included with the software
% must be maintained in all copies produced.
%
% THIS PROGRAM IS PROVIDED "AS IS". THE AUTHOR PROVIDES NO WARRANTIES
% WHATSOEVER, EXPRESSED OR IMPLIED, INCLUDING WARRANTIES OF
% MERCHANTABILITY, TITLE, OR FITNESS FOR ANY PARTICULAR PURPOSE.  THE
% AUTHOR DOES NOT WARRANT THAT USE OF THIS PROGRAM DOES NOT INFRINGE THE
% INTELLECTUAL PROPERTY RIGHTS OF ANY THIRD PARTY IN ANY COUNTRY.
%
% Copyright (c) 1994-2006, John Conover, All Rights Reserved.
%
% Comments and/or bug reports should be addressed to:
%
%     john@email.johncon.com (John Conover)
%
% -----------------------------------------------------------------------------
%
% Revision: \RCSRevision \\
% Revision Time: \RCSTime UMT \\
% Revision Date: \RCSDate \\
% Revision Id: \RCSId \\
% Revision File: \RCSLog \\
\RCS $Revision: 0.0 $
\RCS $Date: 2006/01/20 04:38:13 $
\RCS $Id: instant.tex,v 0.0 2006/01/20 04:38:13 john Exp $
% $Log: instant.tex,v $
% Revision 0.0  2006/01/20 04:38:13  john
% Initial version
%
%
    \subsection{Instantaneous Analysis of Normalized Increments}
        \label{\SETLABEL:IA}

        \subidx{\market}{instantaneous analysis of normalized increments}
        \idx{average of normalized increments}
        \idx{root mean square of normalized increments}
        \subidx{Shannon probability}{instantaneous computation of}
        \subidx{average of normalized increments}{instantaneous computation of}
        \subidx{root mean square of normalized increments}{instantaneous computation of}
        \subidx{instantaneous computation}{Shannon probability}
        \subidx{instantaneous computation}{average of normalized increments}
        \subidx{instantaneous computation}{root mean square of normalized increments}
        \idx{time series}
        \subidx{time series}{instantaneous analysis}
        \subidx{instantaneous analysis}{time series}
        \subidx{time series}{increments}
        \subidx{time series}{analysis}
        \subidx{Shannon}{probability}
        \subidx{probability}{Shannon}
        \subidx{normalized}{increments}
        \subidx{increments}{normalized}

        The program {\it tsinstant}\/, which is briefly described in
        Appendix~\ref{programs}, is for finding the instantaneous
        fraction of change in a time series. The value of a sample in
        the time series is subtracted from the previous sample in the
        time series, and divided by the value of the previous sample.
        As explained in Chapter~\ref{general},
        Sections~\ref{derivation},~\ref{GA},~\ref{abmfi},~\ref{aftsma}
        and,~\ref{ompl} for Brownian motion, random walk fractals, the
        absolute value of the instantaneous fraction of change is also
        the root mean square of the instantaneous fraction of
        change\footnote{The absolute value of the normalized
        increments, when averaged, is related to the root mean square
        of the increments by a constant. If the normalized increments
        are a fixed increment, the constant is unity. If the
        normalized increments have a Gaussian distribution, the
        constant is $\approx 0.8$ depending on the accuracy of of
        ``fit'' to a Gaussian distribution.}. Squaring this value is
        the average of the instantaneous fraction of change, and
        adding unity to the absolute value of the instantaneous
        fraction of change, and dividing by two, is the Shannon
        probability of the instantaneous fraction of change.

        Figure~\ref{\SETLABEL:IA1} is the instantaneous value of the
        root mean square of the normalized increments for the
        {\market}, and Figure~\ref{\SETLABEL:IA2} is the instantaneous
        Shannon probability for the normalized increments.

        \begin{figure}[ht]
            \begin{center}
                \begin{minipage}[t]{0.45\textwidth}
                    \epsfxsize=1.0\linewidth
                    \epsffile{\directory/data.tsinstant-r.eps}
                    \caption[{\market}, instantaneous value of
                        rms.]{{\market}, instantaneous value of the
                        root mean square of the normalized increments,
                        provided by running the program {\it
                        tsinstant}\/ with the -r option on the data
                        presented in Figure~\ref{\SETLABEL:TS}.}
                    \label{\SETLABEL:IA1}
                    \label{\SETLABELQ:IA1}
                \end{minipage}
                \hfill
                \begin{minipage}[t]{0.45\textwidth}
                    \epsfxsize=1.0\linewidth
                    \epsffile{\directory/data.tsinstant-s.eps}
                    \caption[{\market}, instantaneous value of
                        Shannon probability.]{{\market}, instantaneous
                        value of the Shannon probability of the
                        normalized increments, provided by running the
                        program {\it tsinstant}\/ with the -s option
                        on the data presented in
                        Figure~\ref{\SETLABEL:TS}.}
                    \label{\SETLABEL:IA2}
                    \label{\SETLABELQ:IA2}
                \end{minipage}
            \end{center}
        \end{figure}

% Local Variables:
% TeX-parse-self: t
% TeX-auto-save: t
% TeX-master: "fractal.tex"
% End:


        %
% -----------------------------------------------------------------------------
%
% A license is hereby granted to reproduce this software source code and
% to create executable versions from this source code for personal,
% non-commercial use.  The copyright notice included with the software
% must be maintained in all copies produced.
%
% THIS PROGRAM IS PROVIDED "AS IS". THE AUTHOR PROVIDES NO WARRANTIES
% WHATSOEVER, EXPRESSED OR IMPLIED, INCLUDING WARRANTIES OF
% MERCHANTABILITY, TITLE, OR FITNESS FOR ANY PARTICULAR PURPOSE.  THE
% AUTHOR DOES NOT WARRANT THAT USE OF THIS PROGRAM DOES NOT INFRINGE THE
% INTELLECTUAL PROPERTY RIGHTS OF ANY THIRD PARTY IN ANY COUNTRY.
%
% Copyright (c) 1994-2006, John Conover, All Rights Reserved.
%
% Comments and/or bug reports should be addressed to:
%
%     john@email.johncon.com (John Conover)
%
% -----------------------------------------------------------------------------
%
% Revision: \RCSRevision \\
% Revision Time: \RCSTime UMT \\
% Revision Date: \RCSDate \\
% Revision Id: \RCSId \\
% Revision File: \RCSLog \\
\RCS $Revision: 0.0 $
\RCS $Date: 2006/01/20 04:38:13 $
\RCS $Id: logistic.tex,v 0.0 2006/01/20 04:38:13 john Exp $
% $Log: logistic.tex,v $
% Revision 0.0  2006/01/20 04:38:13  john
% Initial version
%
%
    \subsection{Logistic Analysis}
        \label{\SETLABEL:LA}

        \subidx{\market}{Logistic function analysis}
        \subidx{time series}{logistic function}
        \subidx{logistic function}{time series}
        \subidx{time series}{increments}
        \subidx{time series}{analysis}
        \subidx{cumulative sum}{analysis}
        \subidx{analysis}{cumulative sum}
        \subidx{analysis}{random process}
        \subidx{random process}{analysis}
        The data in this section is presented in tabular form in
        Section~\ref{\SETLABELREF:LAA}.  Figure~\ref{\SETLABEL:LA1} is
        a graph of the logistic function estimates of the time series
        data for the {\market}. The reader is cautioned that these
        graphs are constructed using the method suggested in
        Chapter~\ref{general}, Section~\ref{nlextend} and enormous
        precision is required for adequate prediction of the logistic
        function,~\cite{Modis}. Particularly, the non-linear term will
        usually require intervention to produce a practical fit to the
        data. In addition, there are numerical stability issues with
        logistic function methodologies\footnote{For example, in
        Figures~\ref{\SETLABEL:LA1} and~\ref{\SETLABEL:LA2}, if the
        non-linear term, $b$, was greater than zero, it was set to
        zero to produce the graphs. See Section~\ref{\SETLABELREF:LAA}
        for the actual derived values. In other cases, the magnitude
        of $b$ was too large, resulting in a graph that was decreasing
        as a function of time}.  The methodology should be regarded as
        ``fragile.'' It is included for completeness.

        \idx{least squares approximation}
        Figure~\ref{\SETLABEL:LA1} is a graph of the logistic function
        for the time series data presented in
        Figure~\ref{\SETLABEL:TS}. The data presented was made by
        running the program {\it tsdlogistic}\/, which is described
        briefly in Appendix~\ref{programs}, on the parameters
        extracted from the time series data as suggested in
        Figure~\ref{\SETLABEL:TF}. The program {\it tslsq}\/ was used
        to derive the constant and the slope of the normalized
        increments of the data presented in Figure~\ref{\SETLABEL:TF}.
        Figure~\ref{\SETLABEL:LA2} is the same graph, but with the
        time scale expanded by a factor of two.

        \begin{figure}[ht]
            \begin{center}
                \begin{minipage}[t]{0.45\textwidth}
                    \epsfxsize=1.0\linewidth
                    \epsffile{\directory/data.tsfraction.tslsq-p.tsdlogistic.eps}
                    \caption[{\market}, logistic function
                        estimates.]{{\market}, logistic function
                        estimates, provided by running the {\it
                        tslsq}\/ program on the normalized increments
                        presented in Figure~\ref{\SETLABEL:TF} with
                        the -p option. These parameters were used as
                        arguments to the {\it tsdlogistic}\/ program.}
                    \label{\SETLABEL:LA1}
                    \label{\SETLABELQ:LA1}
                \end{minipage}
                \hfill
                \begin{minipage}[t]{0.45\textwidth}
                    \epsfxsize=1.0\linewidth
                    \epsffile{\directory/data.tsfraction.tslsq-p.tsdlogistic2.eps}
                    \caption[{\market}, logistic function
                        estimates.]{{\market}, logistic function
                        estimates of Figure~\ref{\SETLABEL:LA1} with
                        the time scale expanded by a factor of two.}
                    \label{\SETLABEL:LA2}
                    \label{\SETLABELQ:LA2}
                \end{minipage}
            \end{center}
        \end{figure}

% Local Variables:
% TeX-parse-self: t
% TeX-auto-save: t
% TeX-master: "fractal.tex"
% End:


        %
% -----------------------------------------------------------------------------
%
% A license is hereby granted to reproduce this software source code and
% to create executable versions from this source code for personal,
% non-commercial use.  The copyright notice included with the software
% must be maintained in all copies produced.
%
% THIS PROGRAM IS PROVIDED "AS IS". THE AUTHOR PROVIDES NO WARRANTIES
% WHATSOEVER, EXPRESSED OR IMPLIED, INCLUDING WARRANTIES OF
% MERCHANTABILITY, TITLE, OR FITNESS FOR ANY PARTICULAR PURPOSE.  THE
% AUTHOR DOES NOT WARRANT THAT USE OF THIS PROGRAM DOES NOT INFRINGE THE
% INTELLECTUAL PROPERTY RIGHTS OF ANY THIRD PARTY IN ANY COUNTRY.
%
% Copyright (c) 1994-2006, John Conover, All Rights Reserved.
%
% Comments and/or bug reports should be addressed to:
%
%     john@email.johncon.com (John Conover)
%
% -----------------------------------------------------------------------------
%
% Revision: \RCSRevision \\
% Revision Time: \RCSTime UMT \\
% Revision Date: \RCSDate \\
% Revision Id: \RCSId \\
% Revision File: \RCSLog \\
\RCS $Revision: 0.0 $
\RCS $Date: 2006/01/20 04:38:13 $
\RCS $Id: hurst.tex,v 0.0 2006/01/20 04:38:13 john Exp $
% $Log: hurst.tex,v $
% Revision 0.0  2006/01/20 04:38:13  john
% Initial version
%
%
    \subsection{Hurst Coefficient Analysis}
        \label{\SETLABEL:H}

        \subidx{\market}{Hurst coefficient analysis}
        \subidx{Hurst coefficient}{analysis}
        \subidx{increments}{normalized}
        \subidx{normalized}{increments}
        \subidx{programs}{tshurst}
        \subidx{tshurst}{program}
        The data in this section is presented in tabular form in
        Section~\ref{\SETLABELREF:HCHP}. Figure~\ref{\SETLABEL:HC} is
        a graph of the Hurst coefficient data time series data shown
        in Figure~\ref{\SETLABEL:TS}. The slope of the graph is the
        Hurst coefficient.  The data for this figure was produced by
        the program {\it tshurst}\/, which is described briefly in
        Appendix~\ref{programs}.

        \subidx{\market}{H parameter analysis}
        \subidx{H parameter}{analysis}
        \subidx{programs}{tshcalc}
        \subidx{tshcalc}{program}
        Figure~\ref{\SETLABEL:HP} is a graph of the H parameter data
        for the normalized increments of the time series data shown in
        Figure~\ref{\SETLABEL:TF}. The data for this figure was
        produced by the program {\it tshcalc}\/, which is described
        briefly in Appendix~\ref{programs}.

        \begin{figure}[ht]
            \begin{center}
                \begin{minipage}[t]{0.45\textwidth}
                    \epsfxsize=1.0\linewidth
                    \epsffile{\directory/data.tshurst.eps}
                    \caption[{\market}, Hurst coefficient data]{{\market},
                        Hurst coefficient data for the normalized
                        increments of the time series data shown in
                        Figure~\ref{\SETLABEL:TF}.  The slope of the graph
                        is the Hurst coefficient.}
                    \label{\SETLABEL:HC}
                \end{minipage}
                \hfill
                \begin{minipage}[t]{0.45\textwidth}
                    \epsfxsize=1.0\linewidth
                    \epsffile{\directory/data.tshcalc.eps}
                    \caption[{\market}, H parameter data]{{\market}, H
                        parameter data for the normalized increments of
                        the time series data shown in
                        Figure~\ref{\SETLABEL:TF} The slope of the graph
                        is the H parameter.}
                    \label{\SETLABEL:HP}
                \end{minipage}
            \end{center}
        \end{figure}

        \subidx{revenue}{See, rate of revenue returns}
        \subidx{returns}{See, rate of revenue returns}
        \subidx{\market}{revenues}
        \subidx{Hurst coefficient}{analysis}
        \subidx{\market}{Hurst coefficient analysis}
        \subidx{\market}{rate of change}
        \subidx{\market}{windows of opportunity}
        \subidx{rate of revenue returns}{forecast}
        \subidx{forecast}{rate of revenue returns}
        \idx{windows of opportunity}
        \subidx{programs}{tslsq}
        \subidx{tslsq}{program}

        The approximately linear slope of the graph in
        Figure~\ref{\SETLABEL:HC} implies that the variance of the
        rate of revenue returns, (per {\timescale},) in the {\market},
        $V(t_2 - t_1)$, over a period of time is proportional to the
        period of time raised to twice the Hurst
        coefficient~\cite[pp. 180]{Feder},~\cite[pp. 246]{Crownover}.
        This seems to be a quantitative statement concerning how fast,
        and to what degree, the rate of revenue returns' state of
        affairs can change over a period of time.  An additional
        implication, for Hurst coefficients sufficiently close to 0.5,
        is that the probability of the state of affairs repeating
        sometime in the future goes down with increasing
        time\footnote{It can be shown that the number of expected
        market ``high'' and ``low'' transitions, $N$, scales with the
        square root of time, or $N \propto \sqrt {t}$, meaning that
        the cumulative distribution of the probability, $P$, of the
        duration of a market's ``high'' or ``low'' exceeding a given
        time interval, $t$, is proportional to the reciprocal of the
        square root of the time interval, $P \propto 1 / \sqrt {t}$,
        (or, conversely, that the probability of the duration of a
        market's ``high'' or ``low'' exceeding a given time interval
        is proportional to the reciprocal of the time interval raised
        to the power $3 / 2$, ie., $P \propto 1 / t^{3 /
        2}$,~\cite[pp. 153]{Schroeder}. What this means is that a
        histogram of the ``zero free'' run-lengths of a market being
        ``high'' or ``low,'' over a long time, would have a $1 / t^{3
        / 2}$ characteristic.)}, $t$, $p(t) = erf (1/\sqrt{2t})$ which
        is approximately $1/\sqrt{t}$ for $t \gg
        1$~\cite[pp. 160]{Schroeder}. Figures~\ref{\SETLABEL:FN},
        and,~\ref{\SETLABEL:FF} compare methods of approximation of
        the ``forecastability'' of the rate of revenue returns in the
        {\market} for the near term and far term,
        respectively~\cite[pp. 83-84]{Peters:CAOITCM}\footnote{The
        author is not comfortable with Peters' interpretation. For
        example, if the algorithm explained
        in~\cite[pp. 82]{Peters:CAOITCM} is used on ``white noise''
        which, by definition, never has any correlations, the short
        term Hurst coefficient, and thus the ``forecastability,'' is
        still near unity---a bit of an enigma. This can be verified
        with the {\it tswhite}\/ and {\it tshurst}\/ programs, which
        are briefly described in Appendix~\ref{programs}.}.  This
        seems to be a quantitative statement concerning ``windows of
        opportunity'' in the rate of revenue returns, (per
        {\timescale}.)  The program {\it tslsq}\/ was used on the
        Hurst coefficient data, presented in
        Figure~\ref{\SETLABEL:HC}, to provide a least squares
        approximation to the Hurst coefficient. The superimposed least
        squares approximation with on original Hurst coefficient data
        is presented.  The time series data has a Hurst coefficient of
        {\thurstlow}, so that:

        \subidx{\market}{Hurst coefficient analysis}
        \begin{eqnarray}
            V\left(t_2 - t_1\right) & \propto & \left(t_2 - t_1\right)^{2 \cdot H}\\
            V\left(t_2 - t_1\right) & \propto & \left(t_2 - t_1\right)^{2 \cdot {\thurstlow}}\\
                                    & \propto & \left(t_2 - t_1\right)^{\thurstlowtwo}
            \label{\SETLABEL:V}
        \end{eqnarray}

        \subidx{fractional}{Brownian motion}
        \subidx{Brownian motion}{fractional}
        \idx{fractal}
        \noindent where $V(t_2 - t_1)$ is the variance of the
        increments of the rate of revenue returns, (per {\timescale},)
        over the time interval $t_2 -
        t_1$,~\cite[pp. 177]{Feder},~\cite[pp. 494]{Peitgen}. If $H >
        \frac{1}{2}$, then the time series is termed as being
        characterized by ``fractional Brownian
        motion~\cite[pp. 170]{Feder}.''

        \subidx{rate of revenue returns}{predictability}
        \subidx{rate of revenue returns}{forecastability}
        \subidx{rate of revenue returns}{consistency}
        \subidx{predictability}{rate of revenue returns}
        \subidx{forecastability}{rate of revenue returns}
        \subidx{consistency}{rate of revenue returns}
        \subidx{\market}{rate of revenue returns, predictability}
        \subidx{\market}{rate of revenue returns, forecastability}
        \subidx{\market}{rate of revenue returns, consistency}
        \subidx{Hurst coefficient}{analysis}
        \subidx{\market}{Hurst coefficient analysis}
        \subidx{\market}{rate of change}

        In some sense, the Hurst coefficient is a quantitative
        expression of the ``forecastability'' of the future based on
        the past\footnote{Actually, in general, when summing fractal
        entities, the method used should be a root mean square
        process, dependent on the Hurst Coefficient, $H$, where
        $P_{total}^H = P_1^H + P_2^H + \cdots$, where $P_n$ is the
        fractal entities. For a Brownian motion, or random walk type
        of fractal the Hurst Coefficient is a function of time into
        the future. For the ``near term,'' the Hurst coefficient is
        very near unity, meaning the summation process is linear. For
        the ``long term,'' $H \approx 0.5$, or a standard root mean
        square summation process should be used. If $H$ is $0.5$ then
        the market is termed a Brownian motion, or random walk
        process. If it is larger than 0.5, it is termed fractional
        Brownian motion process. For a random walk process, ``near
        term'' and ``far term'' are quantitatively differentiated on
        the Hurst Coefficient graph where $1 - \ln (t) = 0.5 \cdot \ln
        (t)$, or when $\ln (t) = 2$, or $t = 7.389\ldots$ See
        Section~\ref{\SETLABEL:FS} for the particulars on using Hurst
        Coefficient to sum fractal process' for the {\market}. See
        also~\cite[pp. 67, 83-84]{Peters:CAOITCM} and~\cite[pp. 129,
        159]{Schroeder} for particulars on the implications of the
        Hurst Coefficient and root mean square summation issues.}.  A
        Hurst coefficient of {\thurstlow}, (for the near future, and
        {\thurstall} for the distant future.) implies that the
        likelihood of the rate of revenue returns, (per {\timescale},)
        for any two consecutive {\timescale}s being the same is
        {\thurstlowhundred}\%~\cite[pp. 66]{Peters:CAOITCM} for the
        near future, and {\thurstall} for the distant
        future. Likewise, there is a {\thurstlowhundred}\% chance of
        the rate of revenue returns, (per {\timescale},) movements
        being the same in consecutive time periods---ie., if, in a
        given {\timescale}, the rate of revenue returns, (per
        {\timescale},) is increasing, there is a {\thurstlowhundred}\%
        that the rate of revenue returns, (per {\timescale},) will
        increase in the following period, also. In some sense, this is
        a quantitative statement on how ``predictable,'' or
        ``forecastable'' the rate of revenue returns, (per
        {\timescale},) for the {\market} are over time, since the
        probability of having $n$ many consecutive {\timescale}s of
        the same agenda is $H^n$ where $H$ is the Hurst coefficient,
        or, letting the short term probability of having $n$ many
        {\timescale}s of the same market agenda, $p_a$, is:

        \begin{eqnarray}
            p_a\left(n\right) & = & H^{n}\\
                              & = & {\thurstlow}^{n}
            \label{\SETLABEL:MA}
        \end{eqnarray}

        \subidx{rate of revenue returns}{predictability}
        \subidx{rate of revenue returns}{forecastability}
        \subidx{rate of revenue returns}{consistency}
        \subidx{predictability}{rate of revenue returns}
        \subidx{forecastability}{rate of revenue returns}
        \subidx{consistency}{rate of revenue returns}
        As an interesting interpretation of the normalized increments
        of the time series data presented in
        Figure~\ref{\SETLABEL:TF}, if the vertical axis is multiplied
        by 100, to convert to percent, then the graph represents the
        error, in percent, that would be made by forecasting, month by
        month, that the next {\timescale}'s rate of revenue returns
        would be the same as the current {\timescale}'s revenue
        rate. Interestingly, it is $\datafractionmean \cdot 100$
        percent, on the average, with a standard deviation of
        $\datafractionstddev \cdot 100$ percent, and a root mean
        square error value of $\datafractionrms \cdot 100$
        percent---small values for such a simple forecasting
        mechanism.

        \subidx{\market}{rate of revenue returns, range}
        \subidx{Hurst coefficient}{analysis}
        \subidx{\market}{Hurst coefficient analysis}
        \subidx{\market}{rate of change}

        This is, essentially, a statement of the range of values, in
        the increments of the rate of revenue returns, (per
        {\timescale},) that is to be expected over the time interval,
        $t_2 - t_1$,
        $R_v$,~\cite[pp. 178]{Feder},~\cite[pp. 172]{Cambel}:

        \begin{eqnarray}
            R_v\left(t_2 - t_1\right) & \propto & \left(t_2 - t_1\right)^{H}\\
                                      & \propto & \left(t_2 - t_1\right)^{\thurstlow}
            \label{\SETLABEL:R}
        \end{eqnarray}

        \subidx{\market}{rate of revenue returns, range}
        \subidx{Hurst coefficient}{analysis}
        \subidx{\market}{Hurst coefficient analysis}
        \subidx{\market}{rate of change}
        \subidx{Markov}{statistics}
        \subidx{statistics}{Markov}
        \noindent where $R$ is the range of values in the increments
        of the rate of revenue returns, (per {\timescale}.) A Hurst
        coefficient, $H$, that is much larger than $\frac{1}{2}$, (but
        less than 1,) implies a strongly non-Gaussian distribution in
        the increments of the rate of revenue returns, (per
        {\timescale},)~\cite[pp. 152, 194]{Feder}, and a Hurst
        coefficient near $\frac{1}{2}$ implies that the increments of
        the rate of revenue returns, (per {\timescale}) is
        characteristic of an independent
        process~\cite[pp. 195]{Feder}. Extreme caution should be
        exercised in using Markov statistics in any analysis where the
        Hurst coefficient is not
        $\frac{1}{2}$,~\cite[pp. 124]{Crownover},~\cite[pp. 106]{Peters:CAOITCM}.


        As a useful approximation, if $H$, is approximately
        $\frac{1}{2}$, Equation~\ref{\SETLABEL:R} reduces
        to,~\cite[pp. 129]{Schroeder}:

        \begin{eqnarray}
            R\left(t_2 - t_1\right) & \propto & (t_2 - t_1)^{\frac{1}{2}}\\
                                    & \propto & \sqrt{\left(t_2 - t_1\right)}
        \end{eqnarray}

        \subidx{\market}{rate of revenue returns, range}
        \subidx{\market}{rate of revenue returns, increase and decrease}
        \subidx{Hurst coefficient}{analysis}
        \subidx{\market}{Hurst coefficient analysis}
        \subidx{\market}{rate of change}
        \subidx{Markov}{statistics}
        \subidx{statistics}{Markov}

        In the case where the Hurst coefficient, $H$, is
        $\frac{1}{2}$, the range of values in the increments of the
        rate of revenue returns, (per {\timescale},) divided by the
        standard deviation of these values, $S$, can be anticipated to
        increase over time according to the following
        relation,~\cite[pp. 154]{Feder},~\cite[pp. 129]{Schroeder}:

        \begin{equation}
            \frac{R\left(t_2 - t_1\right)}{S} \propto \left(t_2 - t_1\right)^{\frac{1}{2}}
        \end{equation}

        \subidx{\market}{rate of revenue returns, range}
        \subidx{\market}{rate of revenue returns, increase and decrease}
        \subidx{Hurst coefficient}{analysis}
        \subidx{\market}{Hurst coefficient analysis}
        \subidx{\market}{rate of change}
        \noindent which is a useful conceptual approximation, since it
        involves only the square root function---if the range and the
        standard deviation of the increments of the rate of revenue
        returns, (per {\timescale},) are known, (and $H \approx
        \frac{1}{2}$,) then the expected change in $\frac{R}{S}$, will
        increase with the square root of time\footnote{To be precise,
        it is actually asymptotically proportional to
        $\tau^{\frac{1}{2}}$}.

        Another useful approximation when rescaling processes that are
        characterize by Brownian motion, (ie., when $H \approx
        \frac{1}{2}$,) is that:

        \begin{eqnarray}
            X\left(t\right) & \propto & \frac{X\left(rt\right)}{r^{H}}\\
                            & \propto & \frac{X\left(rt\right)}{r^{\thurstlow}}
        \end{eqnarray}

        \idx{Brownian motion}
        \idx{fractal}
        Where $X(t)$ is the process characterized by Brownian motion,
        and $r$ is a scaling factor,~\cite[pp. 494]{Peitgen}.

        \subidx{programs}{tslsq}
        \subidx{tslsq}{program}
        The program {\it tslsq}\/ was used on the H parameter data,
        presented in Figure~\ref{\SETLABEL:HP}, to provide a least
        squares approximation to the H parameter for the
        {\market}. The superimposed least squares approximation on the
        original H parameter data is presented.  By contrast, the H
        parameter, as derived by the methodology outlined
        in~\cite[pp. 249]{Crownover}, is {\thcalclow} for the near
        future, and {\thcalcall} for the distant future.

        \subidx{\market}{Hurst coefficient analysis}
        \subidx{Hurst coefficient}{analysis}
        \subidx{increments}{normalized}
        \subidx{normalized}{increments}
        \subidx{programs}{tshurst}
        \subidx{tshurst}{program}
        \subidx{\market}{H parameter analysis}
        \subidx{H parameter}{analysis}
        \subidx{programs}{tshcalc}
        \subidx{tshcalc}{program}
        Figures~\ref{\SETLABEL:HC} and~\ref{\SETLABEL:HP} represent
        Hurst coefficient and H parameter data that are derived from
        the normalized increments, shown in
        Figure~\ref{\SETLABEL:TF}. In this case, the data is
        considered a normalized derivative of the time series data
        presented in Figure~\ref{\SETLABEL:TF}, instead of a
        cumulative sum.  The program, {\it tshurst}\/, is described
        briefly in appendix~\ref{programs}, and the data for
        figures~\ref{\SETLABEL:THC} and~\ref{\SETLABEL:THP} was made
        using the -d option.

        \begin{figure}[ht]
            \begin{center}
                \begin{minipage}[t]{0.45\textwidth}
                    \epsfxsize=1.0\linewidth
                    \epsffile{\directory/data.tsfraction.tshurst-d.eps}
                    \caption[{\market}, traditional Hurst coefficient
                        data]{{\market}, traditional Hurst coefficient
                        data for the time series data shown in
                        Figure~\ref{\SETLABEL:TS}.  The slope of the
                        graph is the Hurst coefficient, and is
                        {\hurstlow} for the near term, and
                        {\hurstall} for the far term.}
                    \label{\SETLABEL:THC}
                \end{minipage}
                \hfill
                \begin{minipage}[t]{0.45\textwidth}
                    \epsfxsize=1.0\linewidth
                    \epsffile{\directory/data.tsfraction.tshcalc-d.eps}
                    \caption[{\market}, traditional H parameter
                        data]{{\market}, traditional H parameter data
                        for the time series data shown in
                        Figure~\ref{\SETLABEL:TS} The slope of the
                        graph is the H parameter, and is {\hcalclow}
                        for the near term, and {\hcalcall} for the
                        far term.}
                    \label{\SETLABEL:THP}
                \end{minipage}
            \end{center}
        \end{figure}

% Local Variables:
% TeX-parse-self: t
% TeX-auto-save: t
% TeX-master: "fractal.tex"
% End:


        %
% -----------------------------------------------------------------------------
%
% A license is hereby granted to reproduce this software source code and
% to create executable versions from this source code for personal,
% non-commercial use.  The copyright notice included with the software
% must be maintained in all copies produced.
%
% THIS PROGRAM IS PROVIDED "AS IS". THE AUTHOR PROVIDES NO WARRANTIES
% WHATSOEVER, EXPRESSED OR IMPLIED, INCLUDING WARRANTIES OF
% MERCHANTABILITY, TITLE, OR FITNESS FOR ANY PARTICULAR PURPOSE.  THE
% AUTHOR DOES NOT WARRANT THAT USE OF THIS PROGRAM DOES NOT INFRINGE THE
% INTELLECTUAL PROPERTY RIGHTS OF ANY THIRD PARTY IN ANY COUNTRY.
%
% Copyright (c) 1994-2006, John Conover, All Rights Reserved.
%
% Comments and/or bug reports should be addressed to:
%
%     john@email.johncon.com (John Conover)
%
% -----------------------------------------------------------------------------
%
% Revision: \RCSRevision \\
% Revision Time: \RCSTime UMT \\
% Revision Date: \RCSDate \\
% Revision Id: \RCSId \\
% Revision File: \RCSLog \\
\RCS $Revision: 0.0 $
\RCS $Date: 2006/01/20 04:38:13 $
\RCS $Id: fiscal.tex,v 0.0 2006/01/20 04:38:13 john Exp $
% $Log: fiscal.tex,v $
% Revision 0.0  2006/01/20 04:38:13  john
% Initial version
%
%
    \subsection{Fixed Increment Approximation for Fiscal Strategy}
        \label{\SETLABEL:FS}

        \subidx{\market}{fiscal strategy}
        \subidx{markets}{analysis}
        \subidx{analysis}{markets}
        \subidx{strategy}{fiscal}
        \subidx{fiscal}{strategy}
        The data in this section is presented in tabular form in
        Section~\ref{\SETLABELREF:LR}. This section derives various
        values based on the ``average'' of the normalized increments
        presented in Figure~\ref{\SETLABEL:TFA}. These values are an
        approximation to a, probably, complex process with a
        distribution shown in Figure~\ref{\SETLABEL:TF}. These values
        will be used in a fixed increment Brownian fractal analysis
        and simulation of the {\market}, and may, or may not, provide
        adequate accuracy for projections.

        For an organization operating in the {\market}, the fiscal
        strategy, commensurate with the aggregate environment, can be
        derived as follows~\cite[pp. 128, pp
        151]{Schroeder},~\cite[pp. 450]{Reza},~\cite[pp. 270]{Pierce}:
        \vspace{0.15in}

        \subsubsection{Logarithmic Returns}
            \label{\SETLABEL:LR}

            \subidx{logarithmic}{returns}
            \subidx{returns}{logarithmic}
            \subidx{\market}{logarithmic returns}
            The logarithmic returns can be calculated by various
            means. Four will be presented here, for comparison.

            \subidx{programs}{tsnormal}
            \subidx{tsnormal}{program}
            \subidx{logarithmic}{returns}
            \subidx{returns}{logarithmic}
            The logarithmic returns, in bits, $bits$, as computed from
            the mean, by the program {\it tsnormal}\/, which is
            described in Chapter~\ref{programs}, and is presented in
            Figure~\ref{\SETLABEL:TF}, and Equation~\ref{abits} from
            Section~\ref{ereturns} in Chapter~\ref{general}:

            \begin{equation}
                bits = \frac{\ln \left({\datafractionmean} + 1\right)}{\ln \left(2\right)} = \datafractionmeanbits
            \end{equation}

            \subidx{programs}{tslsq}
            \subidx{tslsq}{program}
            \subidx{logarithmic}{returns}
            \subidx{returns}{logarithmic}
            \noindent By comparison, the logarithmic returns, in bits,
            $bits$, as computed from the constant in the least squares
            approximation, using the program {\it tslsq}\/, which is briefly
            described in Chapter~\ref{programs}, as presented in
            Figure~\ref{\SETLABEL:TF}, and Equation~\ref{abits} from
            Section~\ref{ereturns} in Chapter~\ref{general}:

            \begin{equation}
                bits = \frac{\ln \left({\datafractionconstant} + 1\right)}{\ln \left(2\right)} = \datafractionconstantbits
            \end{equation}

            Note that if the mean is not constant in
            Figure~\ref{\SETLABEL:TF}, this method will not provide
            accurate results.

            \subidx{programs}{tslsq}
            \subidx{tslsq}{program}
            \subidx{logarithmic}{returns}
            \subidx{returns}{logarithmic}
            \noindent And by yet another comparison, using the program
            {\it tslsq}\/, which is briefly described in
            Chapter~\ref{programs}, with the -e -p options, to provide
            a formula for the least squares exponential fit to the
            time series data set presented in
            Figure~\ref{\SETLABEL:TS}:

            \begin{equation}
                bits = {\datatslsqepbits}
            \end{equation}

            \subidx{programs}{tslogreturns}
            \subidx{tslogreturns}{program}
            \subidx{logarithmic}{returns}
            \subidx{returns}{logarithmic}
            \noindent And finally, by comparison, from the
            {\it tslogreturns}\/ program, which is briefly described
            in Chapter~\ref{programs}, with the -p option, to provide
            a formula for the logarithmic returns of the time series
            data set presented in Figure~\ref{\SETLABEL:TS}:

            \begin{equation}
                bits = {\logreturns}
            \end{equation}

        \subsubsection{Calculation of Shannon Probability}
            \label{\SETLABEL:SP}

            \subidx{\market}{Shannon probability}
            Ideally, all of the values presented in
            Section~\ref{\SETLABEL:LR} would be equal. Using the
            logarithmic returns provided by the {\it tslogreturns}\/
            program, to be consistent
            with~\cite[pp. 81]{Peters:CAOITCM}

            \subidx{programs}{tslogreturns}
            \subidx{tslogreturns}{program}
            \begin{equation}
                2^{{\logreturns}t}
            \end{equation}

            \noindent therefore:
            \begin{equation}
                C\left(p\right) = {\logreturns}
            \end{equation}
            \subidx{programs}{tsshannon}
            \subidx{tsshannon}{program}
            \subidx{Shannon}{probability}
            \subidx{probability}{Shannon}
            \noindent and, {\it tsshannon}\/ {\logreturns} gives:
            \begin{equation}
                \label{\SETLABEL:F0}
                C\left({\shannonlogreturns}\right) = {\logreturns}
            \end{equation}
            \noindent therefore:
            \begin{eqnarray}
                2^{C\left({\shannonlogreturns}\right)} & = & 2^{\logreturns}\\
                                                       & = & {\twologreturns}\\
                                                       & = & {\twologreturnshundred}\%
            \end{eqnarray}
            \noindent and:
            \begin{eqnarray}
                2p - 1 & = & \left(2 \cdot {\shannonlogreturns}\right) - 1\\
                       & = & {\twopone}\\
                       \label{\SETLABEL:F1}
                       & = & {\twoponehundred}\%
            \end{eqnarray}

            \subidx{\market}{fiscal strategy}
            \subidx{markets}{analysis}
            \subidx{analysis}{markets}
            \subidx{strategy}{fiscal}
            \subidx{fiscal}{strategy}
            \subidx{\market}{fiscal strategy}
            \subidx{\market}{growth rate}
            Presuming the simplified assumptions outlined in
            Section~\ref{assumptions}, the ``typical'' organization
            operating in the {\market} executes a long term fiscal
            strategy, commensurate with the aggregate environment,
            that is to invest, every {\timescale}, in sufficient
            additional resources and infrastructure, to increase the
            manufacturing of goods and services by {\twoponehundred}\%
            of its rate of revenue returns, (per {\timescale}.) As a
            conceptual model, the remaining {\hundredtwoponehundred}\%
            will be held in ``reserve'' with a
            {\shannonlogreturnshundred}\% chance of making twice the
            {\twoponehundred}\% back, (and a
            {\hundredshannonlogreturnshundred}\% chance of making
            0.0,) in one {\timescale}, on the average, for an average
            growth in its rate of revenue returns, (per {\timescale},)
            of {\twologreturnshundred}\%, or a doubling of its rate of
            revenue returns, (per {\timescale},) in
            {\oneoverlogreturns} {\timescale}s.

        \subsubsection{Example Fixed Increment Approximation Fiscal Strategies}

            \subidx{\market}{fiscal strategy}
            \subidx{markets}{analysis}
            \subidx{analysis}{markets}
            \subidx{strategy}{fiscal}
            \subidx{fiscal}{strategy}
            \subidx{\market}{fiscal strategy}
            \subidx{\market}{growth rate}
            \subidx{\market}{management metric}
            \idx{management metric}
            A possible metric on the effectiveness of long term fiscal
            management could possibly be that if an investment of
            {\twoponehundred}\% per {\timescale} of the rate of
            revenue returns, (per {\timescale},) is made in resources
            and infrastructure, then the rate of revenue returns would
            be expected to increase by {\twologreturnshundred}\%, per
            {\timescale}, on average.

            Note that the metrics presented in this section are
            representative of the {\market} as an aggregate whole, and
            may or may not be accurate representations for any
            particular participant in the environment. Of interest to
            the participants in the environment would be a similar
            analysis of each product or service rendered in the
            marketplace.

            \subidx{\market}{fiscal strategy}
            \subidx{markets}{analysis}
            \subidx{analysis}{markets}
            \subidx{strategy}{fiscal}
            \subidx{fiscal}{strategy}
            \subidx{\market}{fiscal strategy}
            As a simple illustrative example, a company operating in
            this environment might obtain a credit line from a bank
            that is equal to {\twoponehundred}\% of its rate of
            revenue returns, (per {\timescale},) to finance additional
            operations. In this simple scenario, the company would use
            its revenue base as collateral for the loan. Some
            {\timescale}s, depending on the {\market}'s environment,
            the company's rate of revenue returns exceeds what was
            borrowed from the bank, and the loan is repaid in
            full. Other {\timescale}s, the company must default, and
            the bank seizes a portion of the company's revenue base to
            pay the delinquent loan. However, on the average, the
            company will expand its rate of revenue returns at
            {\twologreturnshundred}\% per {\timescale}.

            \subidx{\market}{fiscal strategy}
            \subidx{markets}{analysis}
            \subidx{analysis}{markets}
            \subidx{strategy}{fiscal}
            \subidx{fiscal}{strategy}
            \subidx{\market}{fiscal strategy}
            As another simple example, a company re-invests
            {\twoponehundred}\% of its rate of revenue returns, (per
            {\timescale},) in development, marketing, sales, and
            distribution of new products.  Although some products will
            be successful and the return on the investment will exceed
            the {\twoponehundred}\% per {\timescale} investment,
            others will not. However, on the average, the company will
            expand it gross rate of revenue returns at
            {\twologreturnshundred}\% per {\timescale}.

            \subidx{\market}{fiscal strategy}
            \subidx{markets}{analysis}
            \subidx{analysis}{markets}
            \subidx{strategy}{fiscal}
            \subidx{fiscal}{strategy}
            \subidx{\market}{fiscal strategy}
            \subidx{\market}{product portfolio}
            \subidx{\market}{product diversity}
            \subidx{\market}{product mix}
            \subidx{\market}{optimum number of products}
            \idx{product portfolio}
            \idx{product diversity}
            \idx{optimum number of products}
            \idx{product mix}

            As an example of ``product portfolio'' management, suppose
            a company re-invests {\twoponehundred}\% of its rate of
            revenue returns, (per {\timescale},) in development,
            marketing, sales, and distribution of new products.
            Further suppose that the company has two products, and a
            fractal analysis of the individual product rate of revenue
            return time series indicates that one product has a
            Shannon probability of 0.65, and the other has a Shannon
            probability of 0.55. Then the percentage of re-investment
            in the first product would be $(2 \cdot 0.65 - 1) \cdot
            {\twoponehundred}$, percent of the rate of revenue
            returns, and $(2 \cdot 0.55 - 1) \cdot {\twoponehundred}$
            percent for the second product, implying that the company
            should diversify its product line\footnote{The astute
            reader would note that the linear addition was used to add
            the contribution to development of each product. This is a
            ``near term'' interpretation. Actually, in general, the
            method used should be a root mean square process,
            dependent on the Hurst Coefficient, $H$, where
            $P_{total}^H = P_1^H + P_2^H + \cdots$, where $P_n$ is the
            contribution to each individual product. For a Brownian
            motion, or random walk type of fractal the Hurst
            Coefficient is a function of time into the future. For the
            ``near term,'' the Hurst coefficient is very near unity,
            meaning the summation process is linear. For the ``long
            term,'' $H \approx 0.5$, or a standard root mean square
            summation process should be used. If $H$ is $0.5$ then the
            market is termed a Brownian motion, or random walk
            process. If it is larger than 0.5, it is termed fractional
            Brownian motion process. For a random walk process, ``near
            term'' and ``far term'' are quantitatively differentiated
            on the Hurst Coefficient graph where $1 - \ln (t) = 0.5
            \cdot \ln (t)$, or when $\ln (t) = 2$, or $t =
            7.389\ldots$ See~\cite[pp. 67, 83-84]{Peters:CAOITCM}
            and~\cite[pp. 129, 159]{Schroeder} for particulars on the
            implications of the Hurst Coefficient and root mean square
            summation issues.}.  Note that this is a ``bet hedging''
            metric methodology, and assumes that the products have
            uncorrelated revenue return rates. If this re-investment
            methodology is not feasible, perhaps for strategic
            financial reasons, then the re-investment in both products
            should total the ${\twoponehundred}$\%, and the investment
            in each product should be made at a ratio of $\frac{(2
            \cdot 0.65 - 1)}{(2 \cdot 0.55 - 1)} = 3 : 1$,
            respectively. Note that this ``bet hedging'' can be used
            to define the optimal number of products that can be
            supported on the rate of revenue returns. If it assumed
            that all products are ``typical'' for the {\market}, as a
            standard bench mark, then the optimal number will be
            $\frac{1}{{\twopone}}$. Note that this is a
            ``theoretical'' value, since not all products are
            ``typical,'' and there may be strategic reasons, for
            example product leveraging, that may increase the number
            of products above the optimum. However, most of the
            revenue should come from the optimal number of products,
            since having more products will decrease the amount of the
            potential investment in each product, and having less than
            the optimum number of products will increase the risk that
            many of the products could suffer a ``down market''
            concurrently, impacting the rate of revenue returns.  As
            another interesting interpretation of the optimal
            ``hedging of bets,'' in product portfolio strategy, and
            considering the graph of the normalized increments
            presented in Figure~\ref{\SETLABEL:TF}, if the
            organization is running optimally, then these products
            will generate, at least in principle, one standard
            deviation, approximately $0.8413 = 84.13$\% of the future
            growth in rate of revenue returns. Naturally, these are
            approximations, and the values are an approximation to a,
            probably, complex process, and appropriate scrutiny should
            be exercised before making specific projections.  As yet
            another example of ``product portfolio'' management,
            consider the issue of product mix. In this interpretation,
            {\twoponehundred}\% of the product manufactured should be
            ``proprietary,'' while the rest is ``industry standard.''
            As yet another possibility, {\twoponehundred}\% of the
            product manufactured should be predatory into new markets,
            and the remainder in markets that are ``traditional'' for
            the company.

% Local Variables:
% TeX-parse-self: t
% TeX-auto-save: t
% TeX-master: "fractal.tex"
% End:


        %
% -----------------------------------------------------------------------------
%
% A license is hereby granted to reproduce this software source code and
% to create executable versions from this source code for personal,
% non-commercial use.  The copyright notice included with the software
% must be maintained in all copies produced.
%
% THIS PROGRAM IS PROVIDED "AS IS". THE AUTHOR PROVIDES NO WARRANTIES
% WHATSOEVER, EXPRESSED OR IMPLIED, INCLUDING WARRANTIES OF
% MERCHANTABILITY, TITLE, OR FITNESS FOR ANY PARTICULAR PURPOSE.  THE
% AUTHOR DOES NOT WARRANT THAT USE OF THIS PROGRAM DOES NOT INFRINGE THE
% INTELLECTUAL PROPERTY RIGHTS OF ANY THIRD PARTY IN ANY COUNTRY.
%
% Copyright (c) 1994-2006, John Conover, All Rights Reserved.
%
% Comments and/or bug reports should be addressed to:
%
%     john@email.johncon.com (John Conover)
%
% -----------------------------------------------------------------------------
%
% Revision: \RCSRevision \\
% Revision Time: \RCSTime UMT \\
% Revision Date: \RCSDate \\
% Revision Id: \RCSId \\
% Revision File: \RCSLog \\
\RCS $Revision: 0.0 $
\RCS $Date: 2006/01/20 04:38:13 $
\RCS $Id: companies.tex,v 0.0 2006/01/20 04:38:13 john Exp $
% $Log: companies.tex,v $
% Revision 0.0  2006/01/20 04:38:13  john
% Initial version
%
%
    \subsection{Number of Companies}
        \label{\SETLABEL:QNC}

        \subidx{\market}{number of companies}
        \subidx{number of companies}{analysis}
        \subidx{analysis}{number of companies}
        \subidx{Shannon}{probability}
        \subidx{probability}{Shannon}
        This section evaluates the approximate, or ``average,'' number
        of companies in the {\market}, and uses the method outlined in
        Chapter~\ref{general}, Section~\ref{aftsma}. Since the
        average, $avg_{ind}$, and the root mean square, $rms_{ind}$,
        of the normalized increments of the {\market} time series is
        \datafractionmean, and \datafractionrms respectively, the
        number of companies participating in the market can be
        calculated by Equation~\ref{ncompanies} to be {\ncompanies}.

        If this value seems consistent number of companies in the
        {\market}, within the assumptions outlined in
        Chapter~\ref{general}, Section~\ref{aftsma}, then it would
        seem that there is some circumstantial or indirect evidence
        that the companies participating in the {\market} are
        operating optimally, and the ``average'' Shannon probability,
        $P$ for each participating company would be, using
        Equation~\ref{pncompanies}, {\pncompanies}, which would be the
        value which should be used in Section~\ref{\SETLABEL:FS} for
        each participating company if market expansion was to be
        consistent with the rest of the industry. However, if the
        Shannon probability derived in Section~\ref{\SETLABEL:FS} is
        greater than the average Shannon probability for the companies
        participating in the {\market}, as derived in this section,
        then the market would, possibly, be exploitable with the
        fiscal strategy outlined in Section~\ref{\SETLABEL:FS}. The
        maximum exploitability for the {\market} is derived in
        Section~\ref{\SETLABEL:MAXSHANNON}, but it is probably of
        doubtful practicality.

        Note that these optimizations would maximize a company's
        market growth. Since there are probably many companies
        competing in the market place, this would not necessarily
        maximize a company's P\&L, as described in
        Chapter~\ref{general}, Section~\ref{ompl}. The Shannon
        probability that maximizes market share in the {\market} is
        \pncompanies, with several alternative solutions listed in the
        previous paragraph. However, these should be contrasted to the
        Shannon probability that maximizes a company's P\&L which is
        \avgrms~in the {\market}. In all cases, the fraction of the
        P\&L that should be ``wagered'' on the future, $f$, should be:

        \begin{equation}
            f = 2P - 1
        \end{equation}

        \noindent where $P$ is the particular Shannon probability
        chosen optimize a particular fiscal strategy. Interestingly,
        the measured Shannon probability of the {\market} would tend
        to indicate that the companies participating in the market
        have chosen a fiscal strategy that optimizes market growth, as
        opposed to capital growth.

        \subidx{\market}{increasing returns}
        \subidx{economic increasing returns}{\market}
        As interesting interpretation of these exploitive issues,
        since all three fiscal strategies will result in exponential
        market growth for every company participating in the market,
        is that they may represent, perhaps, an example of
        ``increasing returns.''

% Local Variables:
% TeX-parse-self: t
% TeX-auto-save: t
% TeX-master: "fractal.tex"
% End:


        %
% -----------------------------------------------------------------------------
%
% A license is hereby granted to reproduce this software source code and
% to create executable versions from this source code for personal,
% non-commercial use.  The copyright notice included with the software
% must be maintained in all copies produced.
%
% THIS PROGRAM IS PROVIDED "AS IS". THE AUTHOR PROVIDES NO WARRANTIES
% WHATSOEVER, EXPRESSED OR IMPLIED, INCLUDING WARRANTIES OF
% MERCHANTABILITY, TITLE, OR FITNESS FOR ANY PARTICULAR PURPOSE.  THE
% AUTHOR DOES NOT WARRANT THAT USE OF THIS PROGRAM DOES NOT INFRINGE THE
% INTELLECTUAL PROPERTY RIGHTS OF ANY THIRD PARTY IN ANY COUNTRY.
%
% Copyright (c) 1994-2006, John Conover, All Rights Reserved.
%
% Comments and/or bug reports should be addressed to:
%
%     john@email.johncon.com (John Conover)
%
% -----------------------------------------------------------------------------
%
% Revision: \RCSRevision \\
% Revision Time: \RCSTime UMT \\
% Revision Date: \RCSDate \\
% Revision Id: \RCSId \\
% Revision File: \RCSLog \\
\RCS $Revision: 0.0 $
\RCS $Date: 2006/01/20 04:38:13 $
\RCS $Id: operations.tex,v 0.0 2006/01/20 04:38:13 john Exp $
% $Log: operations.tex,v $
% Revision 0.0  2006/01/20 04:38:13  john
% Initial version
%
%
    \subsection{Fixed Increment Approximation for Operational Strategy}
        \label{\SETLABEL:OPS}.

        This section derives various values based on the ``average''
        of the normalized increments presented in
        Figure~\ref{\SETLABEL:TFA}. These values are an approximation
        to a, probably, complex process with a distribution shown in
        Figure~\ref{\SETLABEL:TF}. These values will be used in a
        fixed increment Brownian fractal analysis and simulation of
        the {\market}, and may, or may not, provide adequate accuracy
        for projections.

        \subidx{\market}{fiscal strategy}
        \subidx{\market}{Shannon probability}
        \subidx{strategy}{fiscal}
        \subidx{fiscal}{strategy}
        \subidx{Shannon}{probability}
        \subidx{probability}{Shannon}
        It should be noted that the analysis of fiscal strategy,
        presented in Section~\ref{\SETLABEL:FS}, is derived from the
        {\market} metrics and may, or may not, be maximally
        optimal. For the optimal fiscal strategy, which may be
        exploitable, see Section~\ref{\SETLABEL:MAXSHANNON}.

        \subidx{strategy}{exploitable}
        \subidx{exploitable}{strategy}
        \subidx{\market}{windows of opportunity}
        \idx{windows of opportunity}
        \subidx{decision}{obsolete}
        \subidx{obsolete}{decision}
        \subidx{decision}{timeliness}
        \subidx{timeliness}{decision}
        \subidx{rate of revenue returns}{forecast}
        \subidx{forecast}{rate of revenue returns}
        An additional exploitable strategy may be time itself.
        Equations~\ref{\SETLABEL:V},~\ref{\SETLABEL:R},
        and,~\ref{\SETLABEL:MA}, are, essentially, metrics on how fast
        a decision, which is based on information concerning the
        current status of the {\market}, becomes obsolete. Obviously,
        how long a decision is expected to remain relevant should be
        addressed as an operational necessity in strategic planning
        and project management. Figures~\ref{\SETLABEL:FN},
        and,~\ref{\SETLABEL:FF} compare methods of approximation of
        the ``forecastability'' of rate of revenue returns in the
        {\market} for the near term and far
        term~\cite[pp. 83-84]{Peters:CAOITCM}, respectively. As a
        general rule, caution must be exercised when making decisions
        that will span a time interval larger than the time interval
        where the ``forecastability'' of rate of revenue returns drops
        below 50\%. Beyond this time interval, the chances increase
        that the competitive and market forces will alter the market
        environment in a possibly detrimental unanticipated
        fashion. Obviously, there is significant advantage in
        ``timeliness'' of development, manufacturing, and distribution
        of products and services that are consistent with this
        temporal agenda. Automation of these processes, if executed
        consistently with this agenda, should be considered a
        competitive advantage.

        \subidx{strategy}{exploitable}
        \subidx{exploitable}{strategy}
        \subidx{rate of revenue returns}{forecast}
        \subidx{forecast}{rate of revenue returns}
        \idx{product life cycle}
        \idx{life cycle, product}
        In some sense, this temporal agenda defines the ``average''
        product or service life cycle in the {\market}. When the
        ``forecastability'' of rate of revenue returns drops below
        50\%, there is an even chance that the rate of revenue returns
        for the product or service will change in a detrimental
        fashion. If it is assumed that a product or service life cycle
        consists of a ramp up, a maintenence interval, and a ramp
        down, then, if all three life cycle intervals are equal, the
        product life cycle will be, approximately, three times the
        time interval where the ``forecastability'' of rate of revenue
        returns drops below 50\%. Although probably not an accurate
        prediction of product or service life cycle, the technique may
        be used as a conceptual approximation to the dynamics of
        ``market windows.\footnote{For example, consider the market
        for table salt. Since it has inelastic supply and demand
        curves, and is a necessary requirement for life, it would be
        expected that the Hurst coefficient would be very near
        unity---ignoring competitive pressures in the market. The
        predictability of the table salt market would, therefore, be
        expected to be relatively good, over time.}''  The conceptual
        approximation will probably predict a ``conservative'' or
        ``pessimistic'' value in relation to actual markets.

        \begin{figure}[ht]
            \begin{center}
                \begin{minipage}[t]{0.45\textwidth}
                    \epsfxsize=1.0\linewidth
                    \epsffile{\directory/datahurstlownear.eps}
                    \caption[{\market}, ``forecastability'' of near
                        term rate of revenue returns]{{\market},
                        ``forecastability'' of near term rate of
                        revenue returns. Although the error function
                        is the most accurate, for the near term,
                        $H^{t} = \thurstlow^{t}$ may be used as a
                        reliable metric of ``forecastability'' of the
                        rate of revenue returns.}
                    \label{\SETLABEL:FN}
                \end{minipage}
                \hfill
                \begin{minipage}[t]{0.45\textwidth}
                    \epsfxsize=1.0\linewidth
                    \epsffile{\directory/datahurstlowfar.eps}
                    \caption[{\market}, ``forecastability'' of far
                        term rate of revenue returns]{{\market},
                        ``forecastability'' of far term rate of
                        revenue returns. Although the error function
                        is the most accurate, for the far term,
                        $\frac{1}{\sqrt{t}}$ may be used as a reliable
                        metric of ``forecastability'' of the rate of
                        revenue returns.}
                    \label{\SETLABEL:FF}
                \end{minipage}
            \end{center}
        \end{figure}

        \idx{operations research}
        As an interesting interpretation of the data presented in
        Figure~\ref{\SETLABEL:FN}, there may be, perhaps, some
        applicability to such operational agendas as inventory
        control. Maintaining too little inventory, obviously, will
        create a situation where the organization can not exploit
        market expansion, and maintaining too much inventory,
        likewise, would over extend the company, creating unnecessary
        losses when the market contracts. The company should maintain
        inventory levels that do not exceed, from
        Equation~\ref{\SETLABEL:MA}, ${\thurstlow}^{n} = 0.5$
        {\timescale}s of operations. Since the optimal amount of
        inventory and, from Equation~\ref{\SETLABEL:V}, the variance
        of change in the rate of revenue returns in the future can be
        calculated, there may, perhaps, be some applicability to a
        forecasting methodology that can be incorporated into other
        areas of operations research, for example the linear algebras
        using simplex methodologies for optimization of manufacturing
        processes. Traditionally, these forecasts are made by the
        sales department, and are subject to various subjective
        biases.

% Local Variables:
% TeX-parse-self: t
% TeX-auto-save: t
% TeX-master: "fractal.tex"
% End:


        %
% -----------------------------------------------------------------------------
%
% A license is hereby granted to reproduce this software source code and
% to create executable versions from this source code for personal,
% non-commercial use.  The copyright notice included with the software
% must be maintained in all copies produced.
%
% THIS PROGRAM IS PROVIDED "AS IS". THE AUTHOR PROVIDES NO WARRANTIES
% WHATSOEVER, EXPRESSED OR IMPLIED, INCLUDING WARRANTIES OF
% MERCHANTABILITY, TITLE, OR FITNESS FOR ANY PARTICULAR PURPOSE.  THE
% AUTHOR DOES NOT WARRANT THAT USE OF THIS PROGRAM DOES NOT INFRINGE THE
% INTELLECTUAL PROPERTY RIGHTS OF ANY THIRD PARTY IN ANY COUNTRY.
%
% Copyright (c) 1994-2006, John Conover, All Rights Reserved.
%
% Comments and/or bug reports should be addressed to:
%
%     john@email.johncon.com (John Conover)
%
% -----------------------------------------------------------------------------
%
% Revision: \RCSRevision \\
% Revision Time: \RCSTime UMT \\
% Revision Date: \RCSDate \\
% Revision Id: \RCSId \\
% Revision File: \RCSLog \\
\RCS $Revision: 0.0 $
\RCS $Date: 2006/01/20 04:38:13 $
\RCS $Id: simulation.tex,v 0.0 2006/01/20 04:38:13 john Exp $
% $Log: simulation.tex,v $
% Revision 0.0  2006/01/20 04:38:13  john
% Initial version
%
%
    \subsection{Simulation of Fixed Increment Approximation for Fiscal Strategy}
        \label{\SETLABEL:TSUNFAIRBROWNIAN}

        \subidx{\market}{market simulation}
        The data in this section is presented in tabular form in
        Section~\ref{\SETLABELREF:SIM}.
        Figure~\ref{\SETLABEL:TSUNFAIRBROWNIAN0} represents a
        constructional simulation of the time series data presented in
        Figure~\ref{\SETLABEL:TS}. The program {\it
        tsunfairbrownian}\/, which is briefly described in
        appendix~\ref{programs}, was used in the reconstruction. The
        reconstructed data is superimposed on the original time series
        data.  The program, {\it tsunfairbrownian}\/, essentially,
        constructs the new time series as a Brownian fractal with
        fixed increments---the value of the fixed increment is derived
        from the root mean square average of the normalized increments
        presented in Figure~\ref{\SETLABEL:TF}. The ``quality'' of
        such a reconstruction should be subject to adequate scepticism
        and scrutiny since, in all probability, the normalized
        increments presented in Figure~\ref{\SETLABEL:TF} represent a
        relatively complex process, that may not be ``modeled'' with
        such a simple methodology.

        As a further comparison of the the constructional simulation
        with the original time series data,
        Figure~\ref{\SETLABEL:TSUNFAIRBROWNIAN1} presents a normalized
        histogram of the normalized increments of the reconstructed
        time series, superimposed on the normalized histogram
        presented in Figure~\ref{\SETLABEL:NH}.

        \subidx{\market}{fiscal strategy, simulation}
        \subidx{markets}{simulation}
        \subidx{simulation}{markets}
        \subidx{strategy}{fiscal, simulation}
        \subidx{fiscal}{strategy, simulation}
        \subidx{programs}{tsunfairbrownian}
        \subidx{tsunfairbrownian}{program}
        \begin{figure}[ht]
            \begin{center}
                \begin{minipage}[t]{0.45\textwidth}
                    \epsfxsize=1.0\linewidth
                    \epsffile{\directory/tsunfairbrownian-f.eps}
                    \caption[{\market}, Time series data, empirical and
                        simulated]{{\market}, Time series data, empirical
                        and simulated, using the program {\it tsunfairbrownian}\/
                        with f = {\datafractionrms}. This data is
                        superimposed on the data presented in
                        Figure~\ref{\SETLABEL:TS}.}
                    \label{\SETLABEL:TSUNFAIRBROWNIAN0}
                \end{minipage}
                \hfill
                \begin{minipage}[t]{0.45\textwidth}
                    \epsfxsize=1.0\linewidth
                    \epsffile{\directory/tsunfairbrownian-f.tsfraction.tsnormal-s30.eps}
                    \caption[{\market}, normalized histogram,
                        empirical and simulated]{{\market}, normalized
                        histogram of the normalized increments of the
                        time series data shown in
                        Figure~\ref{\SETLABEL:TSUNFAIRBROWNIAN0},
                        empirical and simulated.  The empirical data
                        has a mean of {\datafractionmean}, with a
                        standard deviation of {\datafractionstddev}.
                        By comparison, the simulated data has a mean
                        of {\tsunfairbrownianfractionmean} with a
                        standard deviation of
                        {\tsunfairbrownianfractionstddev}. This data
                        is superimposed on the data presented in
                        Figure~\ref{\SETLABEL:NH}. The area under the
                        four curves is identical.}
                    \label{\SETLABEL:TSUNFAIRBROWNIAN1}
                \end{minipage}
            \end{center}
        \end{figure}

% Local Variables:
% TeX-parse-self: t
% TeX-auto-save: t
% TeX-master: "fractal.tex"
% End:


        %
% -----------------------------------------------------------------------------
%
% A license is hereby granted to reproduce this software source code and
% to create executable versions from this source code for personal,
% non-commercial use.  The copyright notice included with the software
% must be maintained in all copies produced.
%
% THIS PROGRAM IS PROVIDED "AS IS". THE AUTHOR PROVIDES NO WARRANTIES
% WHATSOEVER, EXPRESSED OR IMPLIED, INCLUDING WARRANTIES OF
% MERCHANTABILITY, TITLE, OR FITNESS FOR ANY PARTICULAR PURPOSE.  THE
% AUTHOR DOES NOT WARRANT THAT USE OF THIS PROGRAM DOES NOT INFRINGE THE
% INTELLECTUAL PROPERTY RIGHTS OF ANY THIRD PARTY IN ANY COUNTRY.
%
% Copyright (c) 1994-2006, John Conover, All Rights Reserved.
%
% Comments and/or bug reports should be addressed to:
%
%     john@email.johncon.com (John Conover)
%
% -----------------------------------------------------------------------------
%
% Revision: \RCSRevision \\
% Revision Time: \RCSTime UMT \\
% Revision Date: \RCSDate \\
% Revision Id: \RCSId \\
% Revision File: \RCSLog \\
\RCS $Revision: 0.0 $
\RCS $Date: 2006/01/20 04:38:13 $
\RCS $Id: maximum.tex,v 0.0 2006/01/20 04:38:13 john Exp $
% $Log: maximum.tex,v $
% Revision 0.0  2006/01/20 04:38:13  john
% Initial version
%
%
    \subsection{Simulation of Fixed Increment Approximation for Optimally Maximal Fiscal Strategy}
        \label{\SETLABEL:MAXSHANNON}
        \subidx{\market}{fiscal strategy, simulation}
        \subidx{\market}{maximum Shannon probability}
        \subidx{markets}{simulation}
        \subidx{simulation}{markets}
        \subidx{strategy}{optimum fiscal, simulation}
        \subidx{fiscal}{optimum strategy, simulation}
        \subidx{programs}{tsunfairbrownian}
        \subidx{tsunfairbrownian}{program}
        \subidx{Shannon}{probability}
        \subidx{probability}{Shannon}

        \subidx{strategy}{exploitable}
        \subidx{exploitable}{strategy}
        \subidx{programs}{tsshannonmax}
        \subidx{tsshannonmax}{program}
        \subidx{programs}{tsunfairbrownian}
        \subidx{tsunfairbrownian}{program}
        \subidx{strategy}{fiscal}
        \subidx{fiscal}{strategy}
        The data in this section is presented in tabular form in
        Section~\ref{\SETLABELREF:MAXSHANNON}. One of the issues of
        analysis, as mentioned in Section~\ref{\SETLABEL:OPS}, is to
        determine the maximum Shannon probability for the time series
        presented in Figure~\ref{\SETLABEL:TS}. Potentially, this
        could be exploited with an aggressive fiscal
        strategy. Figure~\ref{\SETLABEL:SHANNONMAX0} is a graph of the
        output of the {\it tsshannonmax}\/ program, which is described
        briefly in appendix~\ref{programs}. The maximum of this
        function is the maximum Shannon probability for the time
        series data presented in Figure~\ref{\SETLABEL:TS}.
        Figure~\ref{\SETLABEL:SHANNONMAX1} was constructed using {\it
        tsunfairbrownian}\/ program, which is also described in
        appendix~\ref{programs}, with the maximum Shannon probability,
        and the time series data presented in
        Figure~\ref{\SETLABEL:TS}. This represents a ``what if'' the
        investment strategy was changed from a Shannon probability of
        {\shannonlogreturns}, as derived in Section~\ref{\SETLABEL:SP}
        to {\shannonmax}. This process, essentially, extracts the
        random statistical data from the time series presented in
        Figure~\ref{\SETLABEL:TS}, and constructs a new time series,
        using the random statistical data, with a different investment
        strategy.  The program, {\it tsunfairbrownian}\/, essentially,
        constructs the new time series as a Brownian fractal with
        fixed increments.  The ``quality'' of such a reconstruction
        should be subject to adequate scepticism and scrutiny since,
        in all probability, the increments in the original data
        represent a relatively complex process, that may not be
        ``modeled'' with such a simple methodology.

        \begin{figure}[ht]
            \begin{center}
                \begin{minipage}[t]{0.45\textwidth}
                    \epsfxsize=1.0\linewidth
                    \epsffile{\directory/data.tsshannonmax.eps}
                    \caption[{\market}, maximum rate of revenue
                        returns] {{\market}, maximum rate of revenue
                        returns, per {\timescale}, vs. Shannon
                        probability. The maximum rate of revenue
                        returns, per {\timescale}, occurs at a Shannon
                        probability of {\shannonmax}.}
                    \label{\SETLABEL:SHANNONMAX0}
                \end{minipage}
                \hfill
                \begin{minipage}[t]{0.45\textwidth}
                    \epsfxsize=1.0\linewidth
                    \epsffile{\directory/data.tsshannonmax-p.tsunfairbrownian-p.eps}
                    \caption[{\market}, maximum rate of revenue
                        returns] {{\market}, maximum rate of revenue
                        returns, per {\timescale}, at a Shannon
                        probability, of {\shannonmax}, corresponding
                        to a ``wager'' fraction of {\twoponemax}.}
                    \label{\SETLABEL:SHANNONMAX1}
                \end{minipage}
            \end{center}
        \end{figure}

        \subidx{fractional}{Brownian motion}
        \subidx{Brownian motion}{fractional}
        \subidx{Shannon}{probability}
        \subidx{probability}{Shannon}
        \subidx{programs}{tsshannonmax}
        \subidx{tsshannonmax}{program}
        If it is assumed that the time series data set, presented in
        Figure~\ref{\SETLABEL:TS}, constitutes classical Brownian
        motion, then the Shannon probability can be calculated by
        counting the total number of {\timescale}s that the {\market}
        movement was positive, and dividing by the total number of
        {timescale}s represented in the time series. This quotient is
        {\pmax}, as compared with the predicted value from the program
        {\it tsshannonmax}\/ of {\shannonmax}.

% Local Variables:
% TeX-parse-self: t
% TeX-auto-save: t
% TeX-master: "fractal.tex"
% End:


        %
% -----------------------------------------------------------------------------
%
% A license is hereby granted to reproduce this software source code and
% to create executable versions from this source code for personal,
% non-commercial use.  The copyright notice included with the software
% must be maintained in all copies produced.
%
% THIS PROGRAM IS PROVIDED "AS IS". THE AUTHOR PROVIDES NO WARRANTIES
% WHATSOEVER, EXPRESSED OR IMPLIED, INCLUDING WARRANTIES OF
% MERCHANTABILITY, TITLE, OR FITNESS FOR ANY PARTICULAR PURPOSE.  THE
% AUTHOR DOES NOT WARRANT THAT USE OF THIS PROGRAM DOES NOT INFRINGE THE
% INTELLECTUAL PROPERTY RIGHTS OF ANY THIRD PARTY IN ANY COUNTRY.
%
% Copyright (c) 1994-2006, John Conover, All Rights Reserved.
%
% Comments and/or bug reports should be addressed to:
%
%     john@email.johncon.com (John Conover)
%
% -----------------------------------------------------------------------------
%
% Revision: \RCSRevision \\
% Revision Time: \RCSTime UMT \\
% Revision Date: \RCSDate \\
% Revision Id: \RCSId \\
% Revision File: \RCSLog \\
\RCS $Revision: 0.0 $
\RCS $Date: 2006/01/20 04:38:13 $
\RCS $Id: verification.tex,v 0.0 2006/01/20 04:38:13 john Exp $
% $Log: verification.tex,v $
% Revision 0.0  2006/01/20 04:38:13  john
% Initial version
%
%
    \subsection{Qualitative Verification of Fixed Increment Approximation Analysis}
        \label{\SETLABEL:QVA}

        \subidx{\market}{verification of analysis}
        \subidx{verification}{analysis}
        \subidx{analysis}{verification}
        \subidx{quality}{of analysis}
        \subidx{verification}{of methodology}
        \subidx{methodology}{verification of}
        \subidx{Shannon}{probability}
        \subidx{probability}{Shannon}

        This section evaluates various values based on the ``average''
        of the normalized increments presented in
        Figure~\ref{\SETLABEL:TFA}. These values are an approximation
        to a, probably, complex process with a distribution shown in
        Figure~\ref{\SETLABEL:TF}. These values will be used in a
        fixed increment Brownian fractal analysis of the {\market},
        and may, or may not, provide adequate accuracy for
        projections.

        The data in this section is presented in tabular form in
        sections~\ref{\SETLABELREF:VI1} and~\ref{\SETLABELREF:VI2}.
        As a subjective evaluation of the ``quality'' of the analysis
        of the {\market}, from Chapter~\ref{methodology},
        Equation~\ref{metricvalues1}, and using the mean and root mean
        square values of the normalized increments of the time series
        data presented in Figure~\ref{\SETLABEL:TS} from
        Figure~\ref{\SETLABEL:TF}, and the Shannon probability as
        calculated by counting the total number of {\timescale}s that
        the {\market} movement was positive, as presented in
        Section~\ref{\SETLABEL:MAXSHANNON}:

        \begin{eqnarray}
                  P & \approx & \frac{\frac{avg}{rms} + 1}{2}\\
            {\pmax} & \approx & \frac{\frac{\datafractionmean}{\datafractionrms} + 1}{2}\\
            {\pmax} & \approx & {\avgrms}
            \label{\SETLABEL:AVGS}
        \end{eqnarray}

        \subidx{Shannon}{probability}
        \subidx{probability}{Shannon}
        \noindent and comparing these values to the Shannon
        probability, as found by the {\it tsshannonmax}\/ program, which
        iterates for a maximum:

        \begin{eqnarray}
            {\pmax} \approx {\avgrms} \approx {\shannonmax}
        \end{eqnarray}

        \subidx{logarithmic}{returns}
        \subidx{returns}{logarithmic}
        In addition, the different methods of calculating the
        logarithmic returns, presented in Section~\ref{\SETLABEL:FS},
        should be compared. The four methods used were the mean of
        Figure~\ref{\SETLABEL:TF}, the constant in the least squares
        approximation to Figure~\ref{\SETLABEL:TF}, the least squares
        exponential approximation to Figure~\ref{\SETLABEL:TS}, and
        the logarithmic returns of Figure~\ref{\SETLABEL:TS}, derived
        as the mean of the logarithms of the quotients of the
        increments. The values for each of the methods are,
        respectively:

        \begin{equation}
            \datafractionmeanbits \approx \datafractionconstantbits \approx \datatslsqepbits \approx \logreturns
        \end{equation}

        It is implied in Section~\ref{\SETLABEL:FS},
        Subsection~\ref{\SETLABEL:SP} and in
        Section~\ref{\SETLABEL:TSUNFAIRBROWNIAN} that, a Brownian
        motion with fixed increments fractal may ``model'' the
        {\market}. Using Equation~\ref{stddev9} from
        Chapter~\ref{general}, Section~\ref{abmfi}:

        \begin{eqnarray}
                                    rms \left(2P - 1\right) & \approx & \frac{\sigma \left(2P - 1\right)}{2 \sqrt{P\left(1 - P\right)}}\\
            \datafractionrms \left(2 \cdot \pmax - 1\right) & \approx & \frac{\datafractionstddev \left(2 \cdot \pmax - 1\right)}{2\sqrt{\pmax \left(1 - \pmax\right)}}\\
                       \datafractionrms \cdot \twopminusone & \approx & \datafractionstddev \cdot \twopx\\
                                                      \rmsp & \approx & \sigmap
        \end{eqnarray}

        \noindent and, equating to the mean:

        \begin{equation}
            \datafractionmean \approx \rmsp \approx \sigmap
        \end{equation}

        \subidx{Shannon}{probability}
        \subidx{probability}{Shannon}
        \noindent where, as in Equation~\ref{\SETLABEL:AVGS} using the
        mean, root mean square, and standard deviation values of the
        normalized increments of the time series data presented in
        Figure~\ref{\SETLABEL:TS} from Figure~\ref{\SETLABEL:TF}, and
        the Shannon probability as calculated by counting the total
        number of {\timescale}s that the {\market} movement was
        positive, as presented in Section~\ref{\SETLABEL:MAXSHANNON}.

        As a final qualitative comparison, the absolute value of the
        normalized increments should be the same as the root mean
        square value\footnote{The absolute value of the normalized
        increments, when averaged, is related to the root mean square
        of the increments by a constant. If the normalized increments
        are a fixed increment, the constant is unity. If the
        normalized increments have a Gaussian distribution, the
        constant is $\approx 0.8$ depending on the accuracy of of
        ``fit'' to a Gaussian distribution.}, where the absolute value
        is presented in Figure~\ref{\SETLABEL:TFA}, and the root mean
        square value is presented in Figure~\ref{\SETLABEL:TF}:

        \begin{equation}
            \datafractionabsmean \approx \datafractionrms
        \end{equation}

        Note, that if the {\market} could be ``modeled'' as a Brownian
        motion with fixed increments fractal, then the standard
        deviation of the absolute value of the normalized increments
        of the time series data presented in Figure~\ref{\SETLABEL:TS}
        from Figure~\ref{\SETLABEL:TF} should be zero. It is
        $\datafractionabsstddev$.

% Local Variables:
% TeX-parse-self: t
% TeX-auto-save: t
% TeX-master: "fractal.tex"
% End:


    \renewcommand{\market}{Non-optimal Coin Tossing Game}
    \renewcommand{\directory}{../markets/tscoin.tsunfairbrownian}
    \renewcommand{\datafractionmean}{0.008052}
\renewcommand{\datafractionmeanbits}{0.011570}
\renewcommand{\datafractionmeanq}{0.002684}
\renewcommand{\datafractionmeanbitsq}{0.003867}
\renewcommand{\datafractionstddev}{0.038579}
\renewcommand{\datafractionrms}{0.039311}
\renewcommand{\avgrms}{0.602414}
\renewcommand{\ncompanies}{5.210454}
\renewcommand{\pncompanies}{0.544866}
\renewcommand{\datafractionabsmean}{0.029745}
\renewcommand{\datafractionabsstddev}{0.025769}
\renewcommand{\datafractionconstant}{0.010041}
\renewcommand{\datafractionconstantbits}{0.014414}
\renewcommand{\datafractionconstantq}{0.003347}
\renewcommand{\datafractionconstantbitsq}{0.004821}
\renewcommand{\datafractionslope}{-0.000021}
\renewcommand{\datafractionabsconstant}{0.035145}
\renewcommand{\datafractionabsslope}{-0.000057}
\renewcommand{\hurstall}{0.659558}
\renewcommand{\hurstlow}{0.707509}
\renewcommand{\hurstlowtwo}{1.415018}
\renewcommand{\hurstlowhundred}{70.750900}
\renewcommand{\hcalcall}{0.184942}
\renewcommand{\hcalclow}{0.102042}
\renewcommand{\shannonmax}{0.604167}
\renewcommand{\twoponemax}{0.208334}
\renewcommand{\logreturns}{0.010456}
\renewcommand{\twologreturns}{1.007274}
\renewcommand{\twologreturnshundred}{0.727387}
\renewcommand{\oneoverlogreturns}{95.638868}
\renewcommand{\pmax}{0.602094}
\renewcommand{\twopminusone}{0.204188}
\renewcommand{\rmsp}{0.008027}
\renewcommand{\twopx}{0.208583}
\renewcommand{\sigmap}{0.008047}
\renewcommand{\tsunfairbrownianfractionmean}{0.007862}
\renewcommand{\tsunfairbrownianfractionstddev}{0.038619}
\renewcommand{\shannonlogreturns}{0.560125}
\renewcommand{\shannonlogreturnshundred}{56.012500}
\renewcommand{\twopone}{0.120250}
\renewcommand{\twoponehundred}{12.025000}
\renewcommand{\hundredtwoponehundred}{87.975000}
\renewcommand{\hundredshannonlogreturnshundred}{43.987500}
\renewcommand{\datatslsqepbits}{0.007623}
\renewcommand{\thurstall}{0.633980}
\renewcommand{\thurstlow}{0.710108}
\renewcommand{\thurstlowtwo}{1.420216}
\renewcommand{\thurstlowhundred}{71.010800}
\renewcommand{\thcalcall}{0.247886}
\renewcommand{\thcalclow}{0.171737}
\renewcommand{\chisquared}{2.862000}
\renewcommand{\critical}{42.557000}

    \renewcommand{\timescale}{tosses}
    \subidx{market}{\market}
    \idx{\market}

    \section{\market}

        \renewcommand{\SETLABEL}{\LABPRE:NOCT}
        \renewcommand{\SETLABELQ}{\LABPRE:NOCTQ}
        \label{\SETLABEL}
        \renewcommand{\SETLABELREF}{\LABPREREF:NOCT}

        \idx{tscoin}
        \idx{tsunfairbrownian}
        \subidx{programs}{tscoin}
        \subidx{tscoin}{program}
        \subidx{programs}{tsunfairbrownian}
        \subidx{tsunfairbrownian}{program}
        For the analysis, the data was in the directory
        {\directory}\footnote{As a simulation model, the program {\it
        tscoin}\/ was run to make a time series data file, with the
        following parameters:

        \vspace{0.1in}
        {\noindent}tscoin -p 0.7 300 > data.1
        \vspace{0.1in}

        \noindent to make a time series of 300 elements, with a
        Shannon probability of 0.7.  In addition, the program {\it
        tsunfairbrownian}\/ was run on the data file with the
        following parameters:

        \vspace{0.1in}
        {\noindent}tsunfairbrownian -f 0.03 data.1 > data
        \vspace{0.1in}

        \noindent to make a time series with a known non-optimal
        investment strategy.  The data is by {\timescale}.}.

        The data in this section is presented in tabular form in
        Section~\ref{\SETLABELREF}. Note that in this analysis, the
        rate of revenue returns means the increase or decrease in the
        cumulative sum of the {\market}. This is included for
        ``theoretical'' comparative purposes.

        %
% -----------------------------------------------------------------------------
%
% A license is hereby granted to reproduce this software source code and
% to create executable versions from this source code for personal,
% non-commercial use.  The copyright notice included with the software
% must be maintained in all copies produced.
%
% THIS PROGRAM IS PROVIDED "AS IS". THE AUTHOR PROVIDES NO WARRANTIES
% WHATSOEVER, EXPRESSED OR IMPLIED, INCLUDING WARRANTIES OF
% MERCHANTABILITY, TITLE, OR FITNESS FOR ANY PARTICULAR PURPOSE.  THE
% AUTHOR DOES NOT WARRANT THAT USE OF THIS PROGRAM DOES NOT INFRINGE THE
% INTELLECTUAL PROPERTY RIGHTS OF ANY THIRD PARTY IN ANY COUNTRY.
%
% Copyright (c) 1994-2006, John Conover, All Rights Reserved.
%
% Comments and/or bug reports should be addressed to:
%
%     john@email.johncon.com (John Conover)
%
% -----------------------------------------------------------------------------
%
% Revision: \RCSRevision \\
% Revision Time: \RCSTime UMT \\
% Revision Date: \RCSDate \\
% Revision Id: \RCSId \\
% Revision File: \RCSLog \\
\RCS $Revision: 0.0 $
\RCS $Date: 2006/01/20 04:38:13 $
\RCS $Id: fraction.tex,v 0.0 2006/01/20 04:38:13 john Exp $
% $Log: fraction.tex,v $
% Revision 0.0  2006/01/20 04:38:13  john
% Initial version
%
%
    \subsection{Time Series Increments Analysis}
        \label{\SETLABEL:TSA}

        \subidx{\market}{Time series analysis}
        \subidx{time series}{increments}
        \subidx{time series}{analysis}
        \subidx{cumulative sum}{analysis}
        \subidx{analysis}{cumulative sum}
        \subidx{analysis}{random process}
        \subidx{random process}{analysis}
        \subidx{Gaussian}{increments}
        \subidx{increments}{Gaussian}
        \subidx{Brownian}{motion, fractional}
        \subidx{fractional}{Brownian motion}
        \subidx{fractal}{Brownian motion}
        The data in this section is presented in tabular form in
        Section~\ref{\SETLABELREF:TSA}.  Figure~\ref{\SETLABEL:TS} is
        a graph of the time series data for the {\market}.

        \subidx{increments}{normalized}
        \subidx{normalized}{increments}
        \subidx{programs}{tsfraction}
        \subidx{tsfraction}{program}
        Figure~\ref{\SETLABEL:TF} is a graph of the normalized
        increments of the time series data presented in
        Figure~\ref{\SETLABEL:TS}. The data presented was made by
        running the program {\it tsfraction}\/ on the time series
        data. The program {\it tsfraction}\/ is described briefly in
        Appendix~\ref{programs}, and subtracts the previous value from
        the next value, dividing this difference by the previous
        value, for each element in the time series data. The new time
        series contains the instantaneous change in the rate of
        revenue returns, divided by the magnitude of the instantaneous
        rate of revenue returns.

        \subidx{mean}{standard deviation}
        \subidx{standard deviation}{mean}
        \idx{root mean square}
        \idx{least squares approximation}
        \begin{figure}[ht]
            \begin{center}
                \begin{minipage}[t]{0.45\textwidth}
                    \epsfxsize=1.0\linewidth
                    \epsffile{\directory/data.eps}
                    \caption{{\market}, time series data.}
                    \label{\SETLABEL:TS}
                    \label{\SETLABELQ:TS}
                \end{minipage}
                \hfill
                \begin{minipage}[t]{0.45\textwidth}
                    \epsfxsize=1.0\linewidth
                    \epsffile{\directory/data.tsfraction.eps}
                    \caption[{\market}, normalized
                        increments]{{\market}, normalized increments
                        of the time series data presented in
                        Figure~\ref{\SETLABEL:TS}. The mean is
                        {\datafractionmean} with a standard deviation
                        of {\datafractionstddev}. The formula for the
                        least squares approximation is
                        ${\datafractionconstant} +
                        {\datafractionslope}t$, and the root mean
                        squared value is {\datafractionrms}. The
                        graph, labeled ``data\-.tsfraction\-.tsrms,''
                        is the running root mean square, and
                        ``data\-.tsfraction\-.tsavg'' is the running
                        average of the normalized increments.  This
                        graph is the fraction of change in the time
                        series, as a function of time. Note that the
                        slope of the mean, {\datafractionslope}, is
                        the coefficient of the nonlinearity term in
                        the normalized increments. See
                        Chapter~\ref{general}, Section~\ref{nlextend}
                        for a possible application of the logistic
                        function to this data set.}
                    \label{\SETLABEL:TF}
                    \label{\SETLABELQ:TF}
                \end{minipage}
            \end{center}
        \end{figure}

        \subidx{absolute value}{increments}
        \subidx{increments}{absolute value}

        Figure~\ref{\SETLABEL:TFA} is a graph of the absolute value of
        the normalized increments of the time series data presented in
        Figure~\ref{\SETLABEL:TF}. The data presented was made by
        running the Unix utility sed(1) on the normalized increments
        time series data to remove the negative signs. This is an
        absolute value procedure.  The resulting time series contains
        the absolute value of the instantaneous change in the rate of
        revenue returns, divided by the magnitude of the instantaneous
        rate of revenue returns\footnote{The absolute value of the
        normalized increments, when averaged, is related to the root
        mean square of the increments by a constant. If the normalized
        increments are a fixed increment, the constant is unity. If
        the normalized increments have a Gaussian distribution, the
        constant is $\approx 0.8$ depending on the accuracy of of
        ``fit'' to a Gaussian distribution.}.

        \subidx{histogram}{normalized}
        \subidx{normalized}{histogram}
        \subidx{programs}{tsnormal}
        \subidx{tsnormal}{program}
        \subidx{mean}{standard deviation}
        \subidx{standard deviation}{mean}
        \idx{root mean square}
        \idx{least squares approximation}
        \subidx{\market}{analysis of increments}
        Figure~\ref{\SETLABEL:NH} is the normalized histogram of the
        normalized increments of the time series data shown in
        Figure~\ref{\SETLABEL:TF}. The abscissa is 3 $\sigma$ limits,
        and the area under the two curves is identical. The data for
        this figure was produced by the program {\it tsnormal}\/,
        which is described briefly in Appendix~\ref{programs}.

        \begin{figure}[ht]
            \begin{center}
                \begin{minipage}[t]{0.45\textwidth}
                    \epsfxsize=1.0\linewidth
                    \epsffile{\directory/data.tsfraction.abs.eps}
                    \caption[{\market}, absolute value of the
                        normalized increments]{{\market}, absolute
                        value of the normalized increments of the time
                        series data presented in
                        Figure~\ref{\SETLABEL:TF}.  The mean is
                        {\datafractionabsmean} with a standard
                        deviation of {\datafractionabsstddev}. The
                        formula for the least squares approximation is
                        ${\datafractionabsconstant} +
                        {\datafractionabsslope}t$, and the root mean
                        square value, from Figure~\ref{\SETLABEL:TF},
                        is {\datafractionrms}.  The graph, labeled
                        ``data\-.tsfraction\-.tsrms,'' is the running
                        root mean square, and
                        ``data\-.tsfraction\-.tsavg'' is the running
                        average of the normalized increments presented
                        in Figure~\ref{\SETLABEL:TF}, superimposed
                        here for convenience. This graph is the
                        absolute value of the fraction of change in
                        the time series, as a function of time.}
                    \label{\SETLABEL:TFA}
                    \label{\SETLABELQ:TFA}
                \end{minipage}
                \hfill
                \begin{minipage}[t]{0.45\textwidth}
                    \epsfxsize=1.0\linewidth
                    \epsffile{\directory/data.tsfraction.tsnormal-s30.eps}
                    \caption[{\market}, normalized histogram of the
                        normalized increments]{{\market}, normalized
                        histogram of the normalized increments of the
                        time series data shown in
                        Figure~\ref{\SETLABEL:TF}.  The data has a
                        mean of {\datafractionmean}, with a standard
                        deviation of {\datafractionstddev}.  The area
                        under the two curves is identical. The
                        $\chi^2$ value of the observed and expected
                        values of the two curves is {\chisquared},
                        with a critical value of {\critical}.}
                    \label{\SETLABEL:NH}
                \end{minipage}
            \end{center}
        \end{figure}

        \subidx{programs}{tsXsquared}
        \subidx{tsXsquared}{program}
        \subidx{\market}{chi-squared values of increments}
        The program {\it tsXsquared}\/, which is briefly described in
        appendix~\ref{programs}, was used to derive the $\chi^2$
        statistics for the data presented in
        Figure~\ref{\SETLABEL:NH}.

        \subidx{programs}{tsstatest}
        \subidx{tsstatest}{program}
        \subidx{\market}{statistical estimates}

        Figure~\ref{\SETLABEL:SE} is the statistical estimate for the
        data presented in Figure~\ref{\SETLABEL:TF}, as derived by the
        program {\it tsstatest}\/, which is briefly described in
        appendix~\ref{programs}.

        \begin{figure}[ht]
            \begin{center}
                \begin{minipage}[t]{\textwidth}
                    \center{\fbox{\parbox{0.9\textwidth}{\XXX{\directory/data.tsstatest-f0.1-c0.9-i.tex}}}}
                    \caption[{\market}, statistical estimates of the
                        normalized increments]{{\market}, statistical
                        estimates of the normalized increments of the
                        time series shown in Figure~\ref{\SETLABEL:TF}.
                        The table was produced with the {\it
                        tsstatest}\/ program, and illustrates the
                        size of the data set required for a confidence
                        level of 90\%, with an error estimate of $\pm$
                        10\%, or alternately, the error estimate on
                        the time series shown in Figure~\ref{\SETLABEL:TF}.}
                    \label{\SETLABEL:SE}
                \end{minipage}
            \end{center}
        \end{figure}

        Note that the data set size estimations, as produced by the
        {\it tsstatest}\/ program, are probably very conservative,
        depending on the magnitude of the Shannon probability, $P =
        \shannonlogreturns$, as derived in
        Section~\ref{\SETLABEL:SP}. See Chapter~\ref{general},
        Section~\ref{serdss} for possible alternative methodologies
        for addressing the analysis of fractal time series with
        limited data set sizes. Depending on the magnitude of the
        Shannon probability, $P$, these estimates can be several
        orders of magnitude too high.

        \subidx{derivative of increments}{normalized}
        \subidx{normalized}{derivative of increments}
        \subidx{programs}{tsderivative}
        \subidx{tsderivative}{program}
        Figure~\ref{\SETLABEL:TF1} is the normalized histogram of the
        first derivative of the normalized increments of the time
        series data shown in Figure~\ref{\SETLABEL:TF}. In principle,
        if the distribution of the normalized increments presented in
        Figure~\ref{\SETLABEL:NH} is Gaussian in nature, this
        distribution would be similar to ``white noise,'' as presented
        in appendix~\ref{programs}, Figure~\ref{whiteexample}. The
        data was generated by the {\it tsderivative}\/ program, which
        is briefly described in
        appendix~\ref{programs}. Figure~\ref{\SETLABEL:TF2} is the
        normalized histogram of the second derivative of the
        normalized increments of the time series data shown in
        Figure~\ref{\SETLABEL:TF}. In principle, if the distribution
        of the normalized increments presented in
        Figure~\ref{\SETLABEL:NH} is an integrated Gaussian
        distribution in nature, this distribution would be similar to
        ``white noise,'' as presented in appendix~\ref{programs},
        Figure~\ref{whiteexample}.

        \begin{figure}[ht]
            \begin{center}
                \begin{minipage}[t]{0.45\textwidth}
                    \epsfxsize=1.0\linewidth
                    \epsffile{\directory/data.tsfraction.tsderivative.tsnormal-s30.eps}
                    \caption[{\market}, histogram of the first
                        derivative of the increments]{{\market},
                        normalized histogram of the first derivative
                        of the normalized increments of the time
                        series data shown in
                        Figure~\ref{\SETLABEL:TF}.}
                    \label{\SETLABEL:TF1}
                \end{minipage}
                \hfill
                \begin{minipage}[t]{0.45\textwidth}
                    \epsfxsize=1.0\linewidth
                    \epsffile{\directory/data.tsfraction.2tsderivative.tsnormal-s30.eps}
                    \caption[{\market}, histogram of the second
                        derivative of the increments]{{\market},
                        normalized histogram of second derivative of
                        the the normalized increments of the time
                        series data shown in
                        Figure~\ref{\SETLABEL:TF}.}
                    \label{\SETLABEL:TF2}
                \end{minipage}
            \end{center}
        \end{figure}

        \subidx{fractal}{range}
        \subidx{fractal}{R/S analysis}
        \subidx{\market}{rate of revenue returns, range}
        \subidx{\market}{deterministic mechanism}
        \subidx{deterministic}{mechanism}
        \subidx{mechanism}{deterministic}
        Figure~\ref{\SETLABEL:TR} is the range of values of the time
        series shown in Figure~\ref{\SETLABEL:TS}. The horizontal axis
        is time into the future. In principle, if the time series was
        characterized as fractional Brownian motion the graph in
        Figure~\ref{\SETLABEL:TR} would be a square root
        function\footnote{Note that the ``roughness,'' or ``sawtooth''
        characteristics of the graph in Figure~\ref{\SETLABEL:TR} are
        a computational artifact---caused by not using the -m option
        to the program {\it tshurst}\/, which is computationally
        inefficient.}. Figure~\ref{\SETLABEL:TD} is the deterministic
        map of the normalized increments of the time series data shown
        in Figure~\ref{\SETLABEL:TF}. The deterministic map is useful
        for determining if a time series was created by a
        deterministic mechanism. This, essentially, maps each element
        in the time series with the previous element in the time
        series.  See,~\cite[pp. 745]{Peitgen}.

        \begin{figure}[ht]
            \begin{center}
                \begin{minipage}[t]{0.45\textwidth}
                    \epsfxsize=1.0\linewidth
                    \epsffile{\directory/data.tshurst-f.eps}
                    \caption[{\market}, range]{{\market}, range of the
                        time series data shown in
                        Figure~\ref{\SETLABEL:TS}.}
                    \label{\SETLABEL:TR}
                \end{minipage}
                \hfill
                \begin{minipage}[t]{0.45\textwidth}
                    \epsfxsize=1.0\linewidth
                    \epsffile{\directory/data.tsfraction.tsdeterministic.eps}
                    \caption[{\market}, deterministic map]{{\market},
                        deterministic map of the normalized increments
                        of the time series data shown in
                        Figure~\ref{\SETLABEL:TF}.}
                    \label{\SETLABEL:TD}
                \end{minipage}
            \end{center}
        \end{figure}

% Local Variables:
% TeX-parse-self: t
% TeX-auto-save: t
% TeX-master: "fractal.tex"
% End:


        %
% -----------------------------------------------------------------------------
%
% A license is hereby granted to reproduce this software source code and
% to create executable versions from this source code for personal,
% non-commercial use.  The copyright notice included with the software
% must be maintained in all copies produced.
%
% THIS PROGRAM IS PROVIDED "AS IS". THE AUTHOR PROVIDES NO WARRANTIES
% WHATSOEVER, EXPRESSED OR IMPLIED, INCLUDING WARRANTIES OF
% MERCHANTABILITY, TITLE, OR FITNESS FOR ANY PARTICULAR PURPOSE.  THE
% AUTHOR DOES NOT WARRANT THAT USE OF THIS PROGRAM DOES NOT INFRINGE THE
% INTELLECTUAL PROPERTY RIGHTS OF ANY THIRD PARTY IN ANY COUNTRY.
%
% Copyright (c) 1994-2006, John Conover, All Rights Reserved.
%
% Comments and/or bug reports should be addressed to:
%
%     john@email.johncon.com (John Conover)
%
% -----------------------------------------------------------------------------
%
% Revision: \RCSRevision \\
% Revision Time: \RCSTime UMT \\
% Revision Date: \RCSDate \\
% Revision Id: \RCSId \\
% Revision File: \RCSLog \\
\RCS $Revision: 0.0 $
\RCS $Date: 2006/01/20 04:38:13 $
\RCS $Id: instant.tex,v 0.0 2006/01/20 04:38:13 john Exp $
% $Log: instant.tex,v $
% Revision 0.0  2006/01/20 04:38:13  john
% Initial version
%
%
    \subsection{Instantaneous Analysis of Normalized Increments}
        \label{\SETLABEL:IA}

        \subidx{\market}{instantaneous analysis of normalized increments}
        \idx{average of normalized increments}
        \idx{root mean square of normalized increments}
        \subidx{Shannon probability}{instantaneous computation of}
        \subidx{average of normalized increments}{instantaneous computation of}
        \subidx{root mean square of normalized increments}{instantaneous computation of}
        \subidx{instantaneous computation}{Shannon probability}
        \subidx{instantaneous computation}{average of normalized increments}
        \subidx{instantaneous computation}{root mean square of normalized increments}
        \idx{time series}
        \subidx{time series}{instantaneous analysis}
        \subidx{instantaneous analysis}{time series}
        \subidx{time series}{increments}
        \subidx{time series}{analysis}
        \subidx{Shannon}{probability}
        \subidx{probability}{Shannon}
        \subidx{normalized}{increments}
        \subidx{increments}{normalized}

        The program {\it tsinstant}\/, which is briefly described in
        Appendix~\ref{programs}, is for finding the instantaneous
        fraction of change in a time series. The value of a sample in
        the time series is subtracted from the previous sample in the
        time series, and divided by the value of the previous sample.
        As explained in Chapter~\ref{general},
        Sections~\ref{derivation},~\ref{GA},~\ref{abmfi},~\ref{aftsma}
        and,~\ref{ompl} for Brownian motion, random walk fractals, the
        absolute value of the instantaneous fraction of change is also
        the root mean square of the instantaneous fraction of
        change\footnote{The absolute value of the normalized
        increments, when averaged, is related to the root mean square
        of the increments by a constant. If the normalized increments
        are a fixed increment, the constant is unity. If the
        normalized increments have a Gaussian distribution, the
        constant is $\approx 0.8$ depending on the accuracy of of
        ``fit'' to a Gaussian distribution.}. Squaring this value is
        the average of the instantaneous fraction of change, and
        adding unity to the absolute value of the instantaneous
        fraction of change, and dividing by two, is the Shannon
        probability of the instantaneous fraction of change.

        Figure~\ref{\SETLABEL:IA1} is the instantaneous value of the
        root mean square of the normalized increments for the
        {\market}, and Figure~\ref{\SETLABEL:IA2} is the instantaneous
        Shannon probability for the normalized increments.

        \begin{figure}[ht]
            \begin{center}
                \begin{minipage}[t]{0.45\textwidth}
                    \epsfxsize=1.0\linewidth
                    \epsffile{\directory/data.tsinstant-r.eps}
                    \caption[{\market}, instantaneous value of
                        rms.]{{\market}, instantaneous value of the
                        root mean square of the normalized increments,
                        provided by running the program {\it
                        tsinstant}\/ with the -r option on the data
                        presented in Figure~\ref{\SETLABEL:TS}.}
                    \label{\SETLABEL:IA1}
                    \label{\SETLABELQ:IA1}
                \end{minipage}
                \hfill
                \begin{minipage}[t]{0.45\textwidth}
                    \epsfxsize=1.0\linewidth
                    \epsffile{\directory/data.tsinstant-s.eps}
                    \caption[{\market}, instantaneous value of
                        Shannon probability.]{{\market}, instantaneous
                        value of the Shannon probability of the
                        normalized increments, provided by running the
                        program {\it tsinstant}\/ with the -s option
                        on the data presented in
                        Figure~\ref{\SETLABEL:TS}.}
                    \label{\SETLABEL:IA2}
                    \label{\SETLABELQ:IA2}
                \end{minipage}
            \end{center}
        \end{figure}

% Local Variables:
% TeX-parse-self: t
% TeX-auto-save: t
% TeX-master: "fractal.tex"
% End:


        %
% -----------------------------------------------------------------------------
%
% A license is hereby granted to reproduce this software source code and
% to create executable versions from this source code for personal,
% non-commercial use.  The copyright notice included with the software
% must be maintained in all copies produced.
%
% THIS PROGRAM IS PROVIDED "AS IS". THE AUTHOR PROVIDES NO WARRANTIES
% WHATSOEVER, EXPRESSED OR IMPLIED, INCLUDING WARRANTIES OF
% MERCHANTABILITY, TITLE, OR FITNESS FOR ANY PARTICULAR PURPOSE.  THE
% AUTHOR DOES NOT WARRANT THAT USE OF THIS PROGRAM DOES NOT INFRINGE THE
% INTELLECTUAL PROPERTY RIGHTS OF ANY THIRD PARTY IN ANY COUNTRY.
%
% Copyright (c) 1994-2006, John Conover, All Rights Reserved.
%
% Comments and/or bug reports should be addressed to:
%
%     john@email.johncon.com (John Conover)
%
% -----------------------------------------------------------------------------
%
% Revision: \RCSRevision \\
% Revision Time: \RCSTime UMT \\
% Revision Date: \RCSDate \\
% Revision Id: \RCSId \\
% Revision File: \RCSLog \\
\RCS $Revision: 0.0 $
\RCS $Date: 2006/01/20 04:38:13 $
\RCS $Id: logistic.tex,v 0.0 2006/01/20 04:38:13 john Exp $
% $Log: logistic.tex,v $
% Revision 0.0  2006/01/20 04:38:13  john
% Initial version
%
%
    \subsection{Logistic Analysis}
        \label{\SETLABEL:LA}

        \subidx{\market}{Logistic function analysis}
        \subidx{time series}{logistic function}
        \subidx{logistic function}{time series}
        \subidx{time series}{increments}
        \subidx{time series}{analysis}
        \subidx{cumulative sum}{analysis}
        \subidx{analysis}{cumulative sum}
        \subidx{analysis}{random process}
        \subidx{random process}{analysis}
        The data in this section is presented in tabular form in
        Section~\ref{\SETLABELREF:LAA}.  Figure~\ref{\SETLABEL:LA1} is
        a graph of the logistic function estimates of the time series
        data for the {\market}. The reader is cautioned that these
        graphs are constructed using the method suggested in
        Chapter~\ref{general}, Section~\ref{nlextend} and enormous
        precision is required for adequate prediction of the logistic
        function,~\cite{Modis}. Particularly, the non-linear term will
        usually require intervention to produce a practical fit to the
        data. In addition, there are numerical stability issues with
        logistic function methodologies\footnote{For example, in
        Figures~\ref{\SETLABEL:LA1} and~\ref{\SETLABEL:LA2}, if the
        non-linear term, $b$, was greater than zero, it was set to
        zero to produce the graphs. See Section~\ref{\SETLABELREF:LAA}
        for the actual derived values. In other cases, the magnitude
        of $b$ was too large, resulting in a graph that was decreasing
        as a function of time}.  The methodology should be regarded as
        ``fragile.'' It is included for completeness.

        \idx{least squares approximation}
        Figure~\ref{\SETLABEL:LA1} is a graph of the logistic function
        for the time series data presented in
        Figure~\ref{\SETLABEL:TS}. The data presented was made by
        running the program {\it tsdlogistic}\/, which is described
        briefly in Appendix~\ref{programs}, on the parameters
        extracted from the time series data as suggested in
        Figure~\ref{\SETLABEL:TF}. The program {\it tslsq}\/ was used
        to derive the constant and the slope of the normalized
        increments of the data presented in Figure~\ref{\SETLABEL:TF}.
        Figure~\ref{\SETLABEL:LA2} is the same graph, but with the
        time scale expanded by a factor of two.

        \begin{figure}[ht]
            \begin{center}
                \begin{minipage}[t]{0.45\textwidth}
                    \epsfxsize=1.0\linewidth
                    \epsffile{\directory/data.tsfraction.tslsq-p.tsdlogistic.eps}
                    \caption[{\market}, logistic function
                        estimates.]{{\market}, logistic function
                        estimates, provided by running the {\it
                        tslsq}\/ program on the normalized increments
                        presented in Figure~\ref{\SETLABEL:TF} with
                        the -p option. These parameters were used as
                        arguments to the {\it tsdlogistic}\/ program.}
                    \label{\SETLABEL:LA1}
                    \label{\SETLABELQ:LA1}
                \end{minipage}
                \hfill
                \begin{minipage}[t]{0.45\textwidth}
                    \epsfxsize=1.0\linewidth
                    \epsffile{\directory/data.tsfraction.tslsq-p.tsdlogistic2.eps}
                    \caption[{\market}, logistic function
                        estimates.]{{\market}, logistic function
                        estimates of Figure~\ref{\SETLABEL:LA1} with
                        the time scale expanded by a factor of two.}
                    \label{\SETLABEL:LA2}
                    \label{\SETLABELQ:LA2}
                \end{minipage}
            \end{center}
        \end{figure}

% Local Variables:
% TeX-parse-self: t
% TeX-auto-save: t
% TeX-master: "fractal.tex"
% End:


        %
% -----------------------------------------------------------------------------
%
% A license is hereby granted to reproduce this software source code and
% to create executable versions from this source code for personal,
% non-commercial use.  The copyright notice included with the software
% must be maintained in all copies produced.
%
% THIS PROGRAM IS PROVIDED "AS IS". THE AUTHOR PROVIDES NO WARRANTIES
% WHATSOEVER, EXPRESSED OR IMPLIED, INCLUDING WARRANTIES OF
% MERCHANTABILITY, TITLE, OR FITNESS FOR ANY PARTICULAR PURPOSE.  THE
% AUTHOR DOES NOT WARRANT THAT USE OF THIS PROGRAM DOES NOT INFRINGE THE
% INTELLECTUAL PROPERTY RIGHTS OF ANY THIRD PARTY IN ANY COUNTRY.
%
% Copyright (c) 1994-2006, John Conover, All Rights Reserved.
%
% Comments and/or bug reports should be addressed to:
%
%     john@email.johncon.com (John Conover)
%
% -----------------------------------------------------------------------------
%
% Revision: \RCSRevision \\
% Revision Time: \RCSTime UMT \\
% Revision Date: \RCSDate \\
% Revision Id: \RCSId \\
% Revision File: \RCSLog \\
\RCS $Revision: 0.0 $
\RCS $Date: 2006/01/20 04:38:13 $
\RCS $Id: hurst.tex,v 0.0 2006/01/20 04:38:13 john Exp $
% $Log: hurst.tex,v $
% Revision 0.0  2006/01/20 04:38:13  john
% Initial version
%
%
    \subsection{Hurst Coefficient Analysis}
        \label{\SETLABEL:H}

        \subidx{\market}{Hurst coefficient analysis}
        \subidx{Hurst coefficient}{analysis}
        \subidx{increments}{normalized}
        \subidx{normalized}{increments}
        \subidx{programs}{tshurst}
        \subidx{tshurst}{program}
        The data in this section is presented in tabular form in
        Section~\ref{\SETLABELREF:HCHP}. Figure~\ref{\SETLABEL:HC} is
        a graph of the Hurst coefficient data time series data shown
        in Figure~\ref{\SETLABEL:TS}. The slope of the graph is the
        Hurst coefficient.  The data for this figure was produced by
        the program {\it tshurst}\/, which is described briefly in
        Appendix~\ref{programs}.

        \subidx{\market}{H parameter analysis}
        \subidx{H parameter}{analysis}
        \subidx{programs}{tshcalc}
        \subidx{tshcalc}{program}
        Figure~\ref{\SETLABEL:HP} is a graph of the H parameter data
        for the normalized increments of the time series data shown in
        Figure~\ref{\SETLABEL:TF}. The data for this figure was
        produced by the program {\it tshcalc}\/, which is described
        briefly in Appendix~\ref{programs}.

        \begin{figure}[ht]
            \begin{center}
                \begin{minipage}[t]{0.45\textwidth}
                    \epsfxsize=1.0\linewidth
                    \epsffile{\directory/data.tshurst.eps}
                    \caption[{\market}, Hurst coefficient data]{{\market},
                        Hurst coefficient data for the normalized
                        increments of the time series data shown in
                        Figure~\ref{\SETLABEL:TF}.  The slope of the graph
                        is the Hurst coefficient.}
                    \label{\SETLABEL:HC}
                \end{minipage}
                \hfill
                \begin{minipage}[t]{0.45\textwidth}
                    \epsfxsize=1.0\linewidth
                    \epsffile{\directory/data.tshcalc.eps}
                    \caption[{\market}, H parameter data]{{\market}, H
                        parameter data for the normalized increments of
                        the time series data shown in
                        Figure~\ref{\SETLABEL:TF} The slope of the graph
                        is the H parameter.}
                    \label{\SETLABEL:HP}
                \end{minipage}
            \end{center}
        \end{figure}

        \subidx{revenue}{See, rate of revenue returns}
        \subidx{returns}{See, rate of revenue returns}
        \subidx{\market}{revenues}
        \subidx{Hurst coefficient}{analysis}
        \subidx{\market}{Hurst coefficient analysis}
        \subidx{\market}{rate of change}
        \subidx{\market}{windows of opportunity}
        \subidx{rate of revenue returns}{forecast}
        \subidx{forecast}{rate of revenue returns}
        \idx{windows of opportunity}
        \subidx{programs}{tslsq}
        \subidx{tslsq}{program}

        The approximately linear slope of the graph in
        Figure~\ref{\SETLABEL:HC} implies that the variance of the
        rate of revenue returns, (per {\timescale},) in the {\market},
        $V(t_2 - t_1)$, over a period of time is proportional to the
        period of time raised to twice the Hurst
        coefficient~\cite[pp. 180]{Feder},~\cite[pp. 246]{Crownover}.
        This seems to be a quantitative statement concerning how fast,
        and to what degree, the rate of revenue returns' state of
        affairs can change over a period of time.  An additional
        implication, for Hurst coefficients sufficiently close to 0.5,
        is that the probability of the state of affairs repeating
        sometime in the future goes down with increasing
        time\footnote{It can be shown that the number of expected
        market ``high'' and ``low'' transitions, $N$, scales with the
        square root of time, or $N \propto \sqrt {t}$, meaning that
        the cumulative distribution of the probability, $P$, of the
        duration of a market's ``high'' or ``low'' exceeding a given
        time interval, $t$, is proportional to the reciprocal of the
        square root of the time interval, $P \propto 1 / \sqrt {t}$,
        (or, conversely, that the probability of the duration of a
        market's ``high'' or ``low'' exceeding a given time interval
        is proportional to the reciprocal of the time interval raised
        to the power $3 / 2$, ie., $P \propto 1 / t^{3 /
        2}$,~\cite[pp. 153]{Schroeder}. What this means is that a
        histogram of the ``zero free'' run-lengths of a market being
        ``high'' or ``low,'' over a long time, would have a $1 / t^{3
        / 2}$ characteristic.)}, $t$, $p(t) = erf (1/\sqrt{2t})$ which
        is approximately $1/\sqrt{t}$ for $t \gg
        1$~\cite[pp. 160]{Schroeder}. Figures~\ref{\SETLABEL:FN},
        and,~\ref{\SETLABEL:FF} compare methods of approximation of
        the ``forecastability'' of the rate of revenue returns in the
        {\market} for the near term and far term,
        respectively~\cite[pp. 83-84]{Peters:CAOITCM}\footnote{The
        author is not comfortable with Peters' interpretation. For
        example, if the algorithm explained
        in~\cite[pp. 82]{Peters:CAOITCM} is used on ``white noise''
        which, by definition, never has any correlations, the short
        term Hurst coefficient, and thus the ``forecastability,'' is
        still near unity---a bit of an enigma. This can be verified
        with the {\it tswhite}\/ and {\it tshurst}\/ programs, which
        are briefly described in Appendix~\ref{programs}.}.  This
        seems to be a quantitative statement concerning ``windows of
        opportunity'' in the rate of revenue returns, (per
        {\timescale}.)  The program {\it tslsq}\/ was used on the
        Hurst coefficient data, presented in
        Figure~\ref{\SETLABEL:HC}, to provide a least squares
        approximation to the Hurst coefficient. The superimposed least
        squares approximation with on original Hurst coefficient data
        is presented.  The time series data has a Hurst coefficient of
        {\thurstlow}, so that:

        \subidx{\market}{Hurst coefficient analysis}
        \begin{eqnarray}
            V\left(t_2 - t_1\right) & \propto & \left(t_2 - t_1\right)^{2 \cdot H}\\
            V\left(t_2 - t_1\right) & \propto & \left(t_2 - t_1\right)^{2 \cdot {\thurstlow}}\\
                                    & \propto & \left(t_2 - t_1\right)^{\thurstlowtwo}
            \label{\SETLABEL:V}
        \end{eqnarray}

        \subidx{fractional}{Brownian motion}
        \subidx{Brownian motion}{fractional}
        \idx{fractal}
        \noindent where $V(t_2 - t_1)$ is the variance of the
        increments of the rate of revenue returns, (per {\timescale},)
        over the time interval $t_2 -
        t_1$,~\cite[pp. 177]{Feder},~\cite[pp. 494]{Peitgen}. If $H >
        \frac{1}{2}$, then the time series is termed as being
        characterized by ``fractional Brownian
        motion~\cite[pp. 170]{Feder}.''

        \subidx{rate of revenue returns}{predictability}
        \subidx{rate of revenue returns}{forecastability}
        \subidx{rate of revenue returns}{consistency}
        \subidx{predictability}{rate of revenue returns}
        \subidx{forecastability}{rate of revenue returns}
        \subidx{consistency}{rate of revenue returns}
        \subidx{\market}{rate of revenue returns, predictability}
        \subidx{\market}{rate of revenue returns, forecastability}
        \subidx{\market}{rate of revenue returns, consistency}
        \subidx{Hurst coefficient}{analysis}
        \subidx{\market}{Hurst coefficient analysis}
        \subidx{\market}{rate of change}

        In some sense, the Hurst coefficient is a quantitative
        expression of the ``forecastability'' of the future based on
        the past\footnote{Actually, in general, when summing fractal
        entities, the method used should be a root mean square
        process, dependent on the Hurst Coefficient, $H$, where
        $P_{total}^H = P_1^H + P_2^H + \cdots$, where $P_n$ is the
        fractal entities. For a Brownian motion, or random walk type
        of fractal the Hurst Coefficient is a function of time into
        the future. For the ``near term,'' the Hurst coefficient is
        very near unity, meaning the summation process is linear. For
        the ``long term,'' $H \approx 0.5$, or a standard root mean
        square summation process should be used. If $H$ is $0.5$ then
        the market is termed a Brownian motion, or random walk
        process. If it is larger than 0.5, it is termed fractional
        Brownian motion process. For a random walk process, ``near
        term'' and ``far term'' are quantitatively differentiated on
        the Hurst Coefficient graph where $1 - \ln (t) = 0.5 \cdot \ln
        (t)$, or when $\ln (t) = 2$, or $t = 7.389\ldots$ See
        Section~\ref{\SETLABEL:FS} for the particulars on using Hurst
        Coefficient to sum fractal process' for the {\market}. See
        also~\cite[pp. 67, 83-84]{Peters:CAOITCM} and~\cite[pp. 129,
        159]{Schroeder} for particulars on the implications of the
        Hurst Coefficient and root mean square summation issues.}.  A
        Hurst coefficient of {\thurstlow}, (for the near future, and
        {\thurstall} for the distant future.) implies that the
        likelihood of the rate of revenue returns, (per {\timescale},)
        for any two consecutive {\timescale}s being the same is
        {\thurstlowhundred}\%~\cite[pp. 66]{Peters:CAOITCM} for the
        near future, and {\thurstall} for the distant
        future. Likewise, there is a {\thurstlowhundred}\% chance of
        the rate of revenue returns, (per {\timescale},) movements
        being the same in consecutive time periods---ie., if, in a
        given {\timescale}, the rate of revenue returns, (per
        {\timescale},) is increasing, there is a {\thurstlowhundred}\%
        that the rate of revenue returns, (per {\timescale},) will
        increase in the following period, also. In some sense, this is
        a quantitative statement on how ``predictable,'' or
        ``forecastable'' the rate of revenue returns, (per
        {\timescale},) for the {\market} are over time, since the
        probability of having $n$ many consecutive {\timescale}s of
        the same agenda is $H^n$ where $H$ is the Hurst coefficient,
        or, letting the short term probability of having $n$ many
        {\timescale}s of the same market agenda, $p_a$, is:

        \begin{eqnarray}
            p_a\left(n\right) & = & H^{n}\\
                              & = & {\thurstlow}^{n}
            \label{\SETLABEL:MA}
        \end{eqnarray}

        \subidx{rate of revenue returns}{predictability}
        \subidx{rate of revenue returns}{forecastability}
        \subidx{rate of revenue returns}{consistency}
        \subidx{predictability}{rate of revenue returns}
        \subidx{forecastability}{rate of revenue returns}
        \subidx{consistency}{rate of revenue returns}
        As an interesting interpretation of the normalized increments
        of the time series data presented in
        Figure~\ref{\SETLABEL:TF}, if the vertical axis is multiplied
        by 100, to convert to percent, then the graph represents the
        error, in percent, that would be made by forecasting, month by
        month, that the next {\timescale}'s rate of revenue returns
        would be the same as the current {\timescale}'s revenue
        rate. Interestingly, it is $\datafractionmean \cdot 100$
        percent, on the average, with a standard deviation of
        $\datafractionstddev \cdot 100$ percent, and a root mean
        square error value of $\datafractionrms \cdot 100$
        percent---small values for such a simple forecasting
        mechanism.

        \subidx{\market}{rate of revenue returns, range}
        \subidx{Hurst coefficient}{analysis}
        \subidx{\market}{Hurst coefficient analysis}
        \subidx{\market}{rate of change}

        This is, essentially, a statement of the range of values, in
        the increments of the rate of revenue returns, (per
        {\timescale},) that is to be expected over the time interval,
        $t_2 - t_1$,
        $R_v$,~\cite[pp. 178]{Feder},~\cite[pp. 172]{Cambel}:

        \begin{eqnarray}
            R_v\left(t_2 - t_1\right) & \propto & \left(t_2 - t_1\right)^{H}\\
                                      & \propto & \left(t_2 - t_1\right)^{\thurstlow}
            \label{\SETLABEL:R}
        \end{eqnarray}

        \subidx{\market}{rate of revenue returns, range}
        \subidx{Hurst coefficient}{analysis}
        \subidx{\market}{Hurst coefficient analysis}
        \subidx{\market}{rate of change}
        \subidx{Markov}{statistics}
        \subidx{statistics}{Markov}
        \noindent where $R$ is the range of values in the increments
        of the rate of revenue returns, (per {\timescale}.) A Hurst
        coefficient, $H$, that is much larger than $\frac{1}{2}$, (but
        less than 1,) implies a strongly non-Gaussian distribution in
        the increments of the rate of revenue returns, (per
        {\timescale},)~\cite[pp. 152, 194]{Feder}, and a Hurst
        coefficient near $\frac{1}{2}$ implies that the increments of
        the rate of revenue returns, (per {\timescale}) is
        characteristic of an independent
        process~\cite[pp. 195]{Feder}. Extreme caution should be
        exercised in using Markov statistics in any analysis where the
        Hurst coefficient is not
        $\frac{1}{2}$,~\cite[pp. 124]{Crownover},~\cite[pp. 106]{Peters:CAOITCM}.


        As a useful approximation, if $H$, is approximately
        $\frac{1}{2}$, Equation~\ref{\SETLABEL:R} reduces
        to,~\cite[pp. 129]{Schroeder}:

        \begin{eqnarray}
            R\left(t_2 - t_1\right) & \propto & (t_2 - t_1)^{\frac{1}{2}}\\
                                    & \propto & \sqrt{\left(t_2 - t_1\right)}
        \end{eqnarray}

        \subidx{\market}{rate of revenue returns, range}
        \subidx{\market}{rate of revenue returns, increase and decrease}
        \subidx{Hurst coefficient}{analysis}
        \subidx{\market}{Hurst coefficient analysis}
        \subidx{\market}{rate of change}
        \subidx{Markov}{statistics}
        \subidx{statistics}{Markov}

        In the case where the Hurst coefficient, $H$, is
        $\frac{1}{2}$, the range of values in the increments of the
        rate of revenue returns, (per {\timescale},) divided by the
        standard deviation of these values, $S$, can be anticipated to
        increase over time according to the following
        relation,~\cite[pp. 154]{Feder},~\cite[pp. 129]{Schroeder}:

        \begin{equation}
            \frac{R\left(t_2 - t_1\right)}{S} \propto \left(t_2 - t_1\right)^{\frac{1}{2}}
        \end{equation}

        \subidx{\market}{rate of revenue returns, range}
        \subidx{\market}{rate of revenue returns, increase and decrease}
        \subidx{Hurst coefficient}{analysis}
        \subidx{\market}{Hurst coefficient analysis}
        \subidx{\market}{rate of change}
        \noindent which is a useful conceptual approximation, since it
        involves only the square root function---if the range and the
        standard deviation of the increments of the rate of revenue
        returns, (per {\timescale},) are known, (and $H \approx
        \frac{1}{2}$,) then the expected change in $\frac{R}{S}$, will
        increase with the square root of time\footnote{To be precise,
        it is actually asymptotically proportional to
        $\tau^{\frac{1}{2}}$}.

        Another useful approximation when rescaling processes that are
        characterize by Brownian motion, (ie., when $H \approx
        \frac{1}{2}$,) is that:

        \begin{eqnarray}
            X\left(t\right) & \propto & \frac{X\left(rt\right)}{r^{H}}\\
                            & \propto & \frac{X\left(rt\right)}{r^{\thurstlow}}
        \end{eqnarray}

        \idx{Brownian motion}
        \idx{fractal}
        Where $X(t)$ is the process characterized by Brownian motion,
        and $r$ is a scaling factor,~\cite[pp. 494]{Peitgen}.

        \subidx{programs}{tslsq}
        \subidx{tslsq}{program}
        The program {\it tslsq}\/ was used on the H parameter data,
        presented in Figure~\ref{\SETLABEL:HP}, to provide a least
        squares approximation to the H parameter for the
        {\market}. The superimposed least squares approximation on the
        original H parameter data is presented.  By contrast, the H
        parameter, as derived by the methodology outlined
        in~\cite[pp. 249]{Crownover}, is {\thcalclow} for the near
        future, and {\thcalcall} for the distant future.

        \subidx{\market}{Hurst coefficient analysis}
        \subidx{Hurst coefficient}{analysis}
        \subidx{increments}{normalized}
        \subidx{normalized}{increments}
        \subidx{programs}{tshurst}
        \subidx{tshurst}{program}
        \subidx{\market}{H parameter analysis}
        \subidx{H parameter}{analysis}
        \subidx{programs}{tshcalc}
        \subidx{tshcalc}{program}
        Figures~\ref{\SETLABEL:HC} and~\ref{\SETLABEL:HP} represent
        Hurst coefficient and H parameter data that are derived from
        the normalized increments, shown in
        Figure~\ref{\SETLABEL:TF}. In this case, the data is
        considered a normalized derivative of the time series data
        presented in Figure~\ref{\SETLABEL:TF}, instead of a
        cumulative sum.  The program, {\it tshurst}\/, is described
        briefly in appendix~\ref{programs}, and the data for
        figures~\ref{\SETLABEL:THC} and~\ref{\SETLABEL:THP} was made
        using the -d option.

        \begin{figure}[ht]
            \begin{center}
                \begin{minipage}[t]{0.45\textwidth}
                    \epsfxsize=1.0\linewidth
                    \epsffile{\directory/data.tsfraction.tshurst-d.eps}
                    \caption[{\market}, traditional Hurst coefficient
                        data]{{\market}, traditional Hurst coefficient
                        data for the time series data shown in
                        Figure~\ref{\SETLABEL:TS}.  The slope of the
                        graph is the Hurst coefficient, and is
                        {\hurstlow} for the near term, and
                        {\hurstall} for the far term.}
                    \label{\SETLABEL:THC}
                \end{minipage}
                \hfill
                \begin{minipage}[t]{0.45\textwidth}
                    \epsfxsize=1.0\linewidth
                    \epsffile{\directory/data.tsfraction.tshcalc-d.eps}
                    \caption[{\market}, traditional H parameter
                        data]{{\market}, traditional H parameter data
                        for the time series data shown in
                        Figure~\ref{\SETLABEL:TS} The slope of the
                        graph is the H parameter, and is {\hcalclow}
                        for the near term, and {\hcalcall} for the
                        far term.}
                    \label{\SETLABEL:THP}
                \end{minipage}
            \end{center}
        \end{figure}

% Local Variables:
% TeX-parse-self: t
% TeX-auto-save: t
% TeX-master: "fractal.tex"
% End:


        %
% -----------------------------------------------------------------------------
%
% A license is hereby granted to reproduce this software source code and
% to create executable versions from this source code for personal,
% non-commercial use.  The copyright notice included with the software
% must be maintained in all copies produced.
%
% THIS PROGRAM IS PROVIDED "AS IS". THE AUTHOR PROVIDES NO WARRANTIES
% WHATSOEVER, EXPRESSED OR IMPLIED, INCLUDING WARRANTIES OF
% MERCHANTABILITY, TITLE, OR FITNESS FOR ANY PARTICULAR PURPOSE.  THE
% AUTHOR DOES NOT WARRANT THAT USE OF THIS PROGRAM DOES NOT INFRINGE THE
% INTELLECTUAL PROPERTY RIGHTS OF ANY THIRD PARTY IN ANY COUNTRY.
%
% Copyright (c) 1994-2006, John Conover, All Rights Reserved.
%
% Comments and/or bug reports should be addressed to:
%
%     john@email.johncon.com (John Conover)
%
% -----------------------------------------------------------------------------
%
% Revision: \RCSRevision \\
% Revision Time: \RCSTime UMT \\
% Revision Date: \RCSDate \\
% Revision Id: \RCSId \\
% Revision File: \RCSLog \\
\RCS $Revision: 0.0 $
\RCS $Date: 2006/01/20 04:38:13 $
\RCS $Id: fiscal.tex,v 0.0 2006/01/20 04:38:13 john Exp $
% $Log: fiscal.tex,v $
% Revision 0.0  2006/01/20 04:38:13  john
% Initial version
%
%
    \subsection{Fixed Increment Approximation for Fiscal Strategy}
        \label{\SETLABEL:FS}

        \subidx{\market}{fiscal strategy}
        \subidx{markets}{analysis}
        \subidx{analysis}{markets}
        \subidx{strategy}{fiscal}
        \subidx{fiscal}{strategy}
        The data in this section is presented in tabular form in
        Section~\ref{\SETLABELREF:LR}. This section derives various
        values based on the ``average'' of the normalized increments
        presented in Figure~\ref{\SETLABEL:TFA}. These values are an
        approximation to a, probably, complex process with a
        distribution shown in Figure~\ref{\SETLABEL:TF}. These values
        will be used in a fixed increment Brownian fractal analysis
        and simulation of the {\market}, and may, or may not, provide
        adequate accuracy for projections.

        For an organization operating in the {\market}, the fiscal
        strategy, commensurate with the aggregate environment, can be
        derived as follows~\cite[pp. 128, pp
        151]{Schroeder},~\cite[pp. 450]{Reza},~\cite[pp. 270]{Pierce}:
        \vspace{0.15in}

        \subsubsection{Logarithmic Returns}
            \label{\SETLABEL:LR}

            \subidx{logarithmic}{returns}
            \subidx{returns}{logarithmic}
            \subidx{\market}{logarithmic returns}
            The logarithmic returns can be calculated by various
            means. Four will be presented here, for comparison.

            \subidx{programs}{tsnormal}
            \subidx{tsnormal}{program}
            \subidx{logarithmic}{returns}
            \subidx{returns}{logarithmic}
            The logarithmic returns, in bits, $bits$, as computed from
            the mean, by the program {\it tsnormal}\/, which is
            described in Chapter~\ref{programs}, and is presented in
            Figure~\ref{\SETLABEL:TF}, and Equation~\ref{abits} from
            Section~\ref{ereturns} in Chapter~\ref{general}:

            \begin{equation}
                bits = \frac{\ln \left({\datafractionmean} + 1\right)}{\ln \left(2\right)} = \datafractionmeanbits
            \end{equation}

            \subidx{programs}{tslsq}
            \subidx{tslsq}{program}
            \subidx{logarithmic}{returns}
            \subidx{returns}{logarithmic}
            \noindent By comparison, the logarithmic returns, in bits,
            $bits$, as computed from the constant in the least squares
            approximation, using the program {\it tslsq}\/, which is briefly
            described in Chapter~\ref{programs}, as presented in
            Figure~\ref{\SETLABEL:TF}, and Equation~\ref{abits} from
            Section~\ref{ereturns} in Chapter~\ref{general}:

            \begin{equation}
                bits = \frac{\ln \left({\datafractionconstant} + 1\right)}{\ln \left(2\right)} = \datafractionconstantbits
            \end{equation}

            Note that if the mean is not constant in
            Figure~\ref{\SETLABEL:TF}, this method will not provide
            accurate results.

            \subidx{programs}{tslsq}
            \subidx{tslsq}{program}
            \subidx{logarithmic}{returns}
            \subidx{returns}{logarithmic}
            \noindent And by yet another comparison, using the program
            {\it tslsq}\/, which is briefly described in
            Chapter~\ref{programs}, with the -e -p options, to provide
            a formula for the least squares exponential fit to the
            time series data set presented in
            Figure~\ref{\SETLABEL:TS}:

            \begin{equation}
                bits = {\datatslsqepbits}
            \end{equation}

            \subidx{programs}{tslogreturns}
            \subidx{tslogreturns}{program}
            \subidx{logarithmic}{returns}
            \subidx{returns}{logarithmic}
            \noindent And finally, by comparison, from the
            {\it tslogreturns}\/ program, which is briefly described
            in Chapter~\ref{programs}, with the -p option, to provide
            a formula for the logarithmic returns of the time series
            data set presented in Figure~\ref{\SETLABEL:TS}:

            \begin{equation}
                bits = {\logreturns}
            \end{equation}

        \subsubsection{Calculation of Shannon Probability}
            \label{\SETLABEL:SP}

            \subidx{\market}{Shannon probability}
            Ideally, all of the values presented in
            Section~\ref{\SETLABEL:LR} would be equal. Using the
            logarithmic returns provided by the {\it tslogreturns}\/
            program, to be consistent
            with~\cite[pp. 81]{Peters:CAOITCM}

            \subidx{programs}{tslogreturns}
            \subidx{tslogreturns}{program}
            \begin{equation}
                2^{{\logreturns}t}
            \end{equation}

            \noindent therefore:
            \begin{equation}
                C\left(p\right) = {\logreturns}
            \end{equation}
            \subidx{programs}{tsshannon}
            \subidx{tsshannon}{program}
            \subidx{Shannon}{probability}
            \subidx{probability}{Shannon}
            \noindent and, {\it tsshannon}\/ {\logreturns} gives:
            \begin{equation}
                \label{\SETLABEL:F0}
                C\left({\shannonlogreturns}\right) = {\logreturns}
            \end{equation}
            \noindent therefore:
            \begin{eqnarray}
                2^{C\left({\shannonlogreturns}\right)} & = & 2^{\logreturns}\\
                                                       & = & {\twologreturns}\\
                                                       & = & {\twologreturnshundred}\%
            \end{eqnarray}
            \noindent and:
            \begin{eqnarray}
                2p - 1 & = & \left(2 \cdot {\shannonlogreturns}\right) - 1\\
                       & = & {\twopone}\\
                       \label{\SETLABEL:F1}
                       & = & {\twoponehundred}\%
            \end{eqnarray}

            \subidx{\market}{fiscal strategy}
            \subidx{markets}{analysis}
            \subidx{analysis}{markets}
            \subidx{strategy}{fiscal}
            \subidx{fiscal}{strategy}
            \subidx{\market}{fiscal strategy}
            \subidx{\market}{growth rate}
            Presuming the simplified assumptions outlined in
            Section~\ref{assumptions}, the ``typical'' organization
            operating in the {\market} executes a long term fiscal
            strategy, commensurate with the aggregate environment,
            that is to invest, every {\timescale}, in sufficient
            additional resources and infrastructure, to increase the
            manufacturing of goods and services by {\twoponehundred}\%
            of its rate of revenue returns, (per {\timescale}.) As a
            conceptual model, the remaining {\hundredtwoponehundred}\%
            will be held in ``reserve'' with a
            {\shannonlogreturnshundred}\% chance of making twice the
            {\twoponehundred}\% back, (and a
            {\hundredshannonlogreturnshundred}\% chance of making
            0.0,) in one {\timescale}, on the average, for an average
            growth in its rate of revenue returns, (per {\timescale},)
            of {\twologreturnshundred}\%, or a doubling of its rate of
            revenue returns, (per {\timescale},) in
            {\oneoverlogreturns} {\timescale}s.

        \subsubsection{Example Fixed Increment Approximation Fiscal Strategies}

            \subidx{\market}{fiscal strategy}
            \subidx{markets}{analysis}
            \subidx{analysis}{markets}
            \subidx{strategy}{fiscal}
            \subidx{fiscal}{strategy}
            \subidx{\market}{fiscal strategy}
            \subidx{\market}{growth rate}
            \subidx{\market}{management metric}
            \idx{management metric}
            A possible metric on the effectiveness of long term fiscal
            management could possibly be that if an investment of
            {\twoponehundred}\% per {\timescale} of the rate of
            revenue returns, (per {\timescale},) is made in resources
            and infrastructure, then the rate of revenue returns would
            be expected to increase by {\twologreturnshundred}\%, per
            {\timescale}, on average.

            Note that the metrics presented in this section are
            representative of the {\market} as an aggregate whole, and
            may or may not be accurate representations for any
            particular participant in the environment. Of interest to
            the participants in the environment would be a similar
            analysis of each product or service rendered in the
            marketplace.

            \subidx{\market}{fiscal strategy}
            \subidx{markets}{analysis}
            \subidx{analysis}{markets}
            \subidx{strategy}{fiscal}
            \subidx{fiscal}{strategy}
            \subidx{\market}{fiscal strategy}
            As a simple illustrative example, a company operating in
            this environment might obtain a credit line from a bank
            that is equal to {\twoponehundred}\% of its rate of
            revenue returns, (per {\timescale},) to finance additional
            operations. In this simple scenario, the company would use
            its revenue base as collateral for the loan. Some
            {\timescale}s, depending on the {\market}'s environment,
            the company's rate of revenue returns exceeds what was
            borrowed from the bank, and the loan is repaid in
            full. Other {\timescale}s, the company must default, and
            the bank seizes a portion of the company's revenue base to
            pay the delinquent loan. However, on the average, the
            company will expand its rate of revenue returns at
            {\twologreturnshundred}\% per {\timescale}.

            \subidx{\market}{fiscal strategy}
            \subidx{markets}{analysis}
            \subidx{analysis}{markets}
            \subidx{strategy}{fiscal}
            \subidx{fiscal}{strategy}
            \subidx{\market}{fiscal strategy}
            As another simple example, a company re-invests
            {\twoponehundred}\% of its rate of revenue returns, (per
            {\timescale},) in development, marketing, sales, and
            distribution of new products.  Although some products will
            be successful and the return on the investment will exceed
            the {\twoponehundred}\% per {\timescale} investment,
            others will not. However, on the average, the company will
            expand it gross rate of revenue returns at
            {\twologreturnshundred}\% per {\timescale}.

            \subidx{\market}{fiscal strategy}
            \subidx{markets}{analysis}
            \subidx{analysis}{markets}
            \subidx{strategy}{fiscal}
            \subidx{fiscal}{strategy}
            \subidx{\market}{fiscal strategy}
            \subidx{\market}{product portfolio}
            \subidx{\market}{product diversity}
            \subidx{\market}{product mix}
            \subidx{\market}{optimum number of products}
            \idx{product portfolio}
            \idx{product diversity}
            \idx{optimum number of products}
            \idx{product mix}

            As an example of ``product portfolio'' management, suppose
            a company re-invests {\twoponehundred}\% of its rate of
            revenue returns, (per {\timescale},) in development,
            marketing, sales, and distribution of new products.
            Further suppose that the company has two products, and a
            fractal analysis of the individual product rate of revenue
            return time series indicates that one product has a
            Shannon probability of 0.65, and the other has a Shannon
            probability of 0.55. Then the percentage of re-investment
            in the first product would be $(2 \cdot 0.65 - 1) \cdot
            {\twoponehundred}$, percent of the rate of revenue
            returns, and $(2 \cdot 0.55 - 1) \cdot {\twoponehundred}$
            percent for the second product, implying that the company
            should diversify its product line\footnote{The astute
            reader would note that the linear addition was used to add
            the contribution to development of each product. This is a
            ``near term'' interpretation. Actually, in general, the
            method used should be a root mean square process,
            dependent on the Hurst Coefficient, $H$, where
            $P_{total}^H = P_1^H + P_2^H + \cdots$, where $P_n$ is the
            contribution to each individual product. For a Brownian
            motion, or random walk type of fractal the Hurst
            Coefficient is a function of time into the future. For the
            ``near term,'' the Hurst coefficient is very near unity,
            meaning the summation process is linear. For the ``long
            term,'' $H \approx 0.5$, or a standard root mean square
            summation process should be used. If $H$ is $0.5$ then the
            market is termed a Brownian motion, or random walk
            process. If it is larger than 0.5, it is termed fractional
            Brownian motion process. For a random walk process, ``near
            term'' and ``far term'' are quantitatively differentiated
            on the Hurst Coefficient graph where $1 - \ln (t) = 0.5
            \cdot \ln (t)$, or when $\ln (t) = 2$, or $t =
            7.389\ldots$ See~\cite[pp. 67, 83-84]{Peters:CAOITCM}
            and~\cite[pp. 129, 159]{Schroeder} for particulars on the
            implications of the Hurst Coefficient and root mean square
            summation issues.}.  Note that this is a ``bet hedging''
            metric methodology, and assumes that the products have
            uncorrelated revenue return rates. If this re-investment
            methodology is not feasible, perhaps for strategic
            financial reasons, then the re-investment in both products
            should total the ${\twoponehundred}$\%, and the investment
            in each product should be made at a ratio of $\frac{(2
            \cdot 0.65 - 1)}{(2 \cdot 0.55 - 1)} = 3 : 1$,
            respectively. Note that this ``bet hedging'' can be used
            to define the optimal number of products that can be
            supported on the rate of revenue returns. If it assumed
            that all products are ``typical'' for the {\market}, as a
            standard bench mark, then the optimal number will be
            $\frac{1}{{\twopone}}$. Note that this is a
            ``theoretical'' value, since not all products are
            ``typical,'' and there may be strategic reasons, for
            example product leveraging, that may increase the number
            of products above the optimum. However, most of the
            revenue should come from the optimal number of products,
            since having more products will decrease the amount of the
            potential investment in each product, and having less than
            the optimum number of products will increase the risk that
            many of the products could suffer a ``down market''
            concurrently, impacting the rate of revenue returns.  As
            another interesting interpretation of the optimal
            ``hedging of bets,'' in product portfolio strategy, and
            considering the graph of the normalized increments
            presented in Figure~\ref{\SETLABEL:TF}, if the
            organization is running optimally, then these products
            will generate, at least in principle, one standard
            deviation, approximately $0.8413 = 84.13$\% of the future
            growth in rate of revenue returns. Naturally, these are
            approximations, and the values are an approximation to a,
            probably, complex process, and appropriate scrutiny should
            be exercised before making specific projections.  As yet
            another example of ``product portfolio'' management,
            consider the issue of product mix. In this interpretation,
            {\twoponehundred}\% of the product manufactured should be
            ``proprietary,'' while the rest is ``industry standard.''
            As yet another possibility, {\twoponehundred}\% of the
            product manufactured should be predatory into new markets,
            and the remainder in markets that are ``traditional'' for
            the company.

% Local Variables:
% TeX-parse-self: t
% TeX-auto-save: t
% TeX-master: "fractal.tex"
% End:


        %
% -----------------------------------------------------------------------------
%
% A license is hereby granted to reproduce this software source code and
% to create executable versions from this source code for personal,
% non-commercial use.  The copyright notice included with the software
% must be maintained in all copies produced.
%
% THIS PROGRAM IS PROVIDED "AS IS". THE AUTHOR PROVIDES NO WARRANTIES
% WHATSOEVER, EXPRESSED OR IMPLIED, INCLUDING WARRANTIES OF
% MERCHANTABILITY, TITLE, OR FITNESS FOR ANY PARTICULAR PURPOSE.  THE
% AUTHOR DOES NOT WARRANT THAT USE OF THIS PROGRAM DOES NOT INFRINGE THE
% INTELLECTUAL PROPERTY RIGHTS OF ANY THIRD PARTY IN ANY COUNTRY.
%
% Copyright (c) 1994-2006, John Conover, All Rights Reserved.
%
% Comments and/or bug reports should be addressed to:
%
%     john@email.johncon.com (John Conover)
%
% -----------------------------------------------------------------------------
%
% Revision: \RCSRevision \\
% Revision Time: \RCSTime UMT \\
% Revision Date: \RCSDate \\
% Revision Id: \RCSId \\
% Revision File: \RCSLog \\
\RCS $Revision: 0.0 $
\RCS $Date: 2006/01/20 04:38:13 $
\RCS $Id: companies.tex,v 0.0 2006/01/20 04:38:13 john Exp $
% $Log: companies.tex,v $
% Revision 0.0  2006/01/20 04:38:13  john
% Initial version
%
%
    \subsection{Number of Companies}
        \label{\SETLABEL:QNC}

        \subidx{\market}{number of companies}
        \subidx{number of companies}{analysis}
        \subidx{analysis}{number of companies}
        \subidx{Shannon}{probability}
        \subidx{probability}{Shannon}
        This section evaluates the approximate, or ``average,'' number
        of companies in the {\market}, and uses the method outlined in
        Chapter~\ref{general}, Section~\ref{aftsma}. Since the
        average, $avg_{ind}$, and the root mean square, $rms_{ind}$,
        of the normalized increments of the {\market} time series is
        \datafractionmean, and \datafractionrms respectively, the
        number of companies participating in the market can be
        calculated by Equation~\ref{ncompanies} to be {\ncompanies}.

        If this value seems consistent number of companies in the
        {\market}, within the assumptions outlined in
        Chapter~\ref{general}, Section~\ref{aftsma}, then it would
        seem that there is some circumstantial or indirect evidence
        that the companies participating in the {\market} are
        operating optimally, and the ``average'' Shannon probability,
        $P$ for each participating company would be, using
        Equation~\ref{pncompanies}, {\pncompanies}, which would be the
        value which should be used in Section~\ref{\SETLABEL:FS} for
        each participating company if market expansion was to be
        consistent with the rest of the industry. However, if the
        Shannon probability derived in Section~\ref{\SETLABEL:FS} is
        greater than the average Shannon probability for the companies
        participating in the {\market}, as derived in this section,
        then the market would, possibly, be exploitable with the
        fiscal strategy outlined in Section~\ref{\SETLABEL:FS}. The
        maximum exploitability for the {\market} is derived in
        Section~\ref{\SETLABEL:MAXSHANNON}, but it is probably of
        doubtful practicality.

        Note that these optimizations would maximize a company's
        market growth. Since there are probably many companies
        competing in the market place, this would not necessarily
        maximize a company's P\&L, as described in
        Chapter~\ref{general}, Section~\ref{ompl}. The Shannon
        probability that maximizes market share in the {\market} is
        \pncompanies, with several alternative solutions listed in the
        previous paragraph. However, these should be contrasted to the
        Shannon probability that maximizes a company's P\&L which is
        \avgrms~in the {\market}. In all cases, the fraction of the
        P\&L that should be ``wagered'' on the future, $f$, should be:

        \begin{equation}
            f = 2P - 1
        \end{equation}

        \noindent where $P$ is the particular Shannon probability
        chosen optimize a particular fiscal strategy. Interestingly,
        the measured Shannon probability of the {\market} would tend
        to indicate that the companies participating in the market
        have chosen a fiscal strategy that optimizes market growth, as
        opposed to capital growth.

        \subidx{\market}{increasing returns}
        \subidx{economic increasing returns}{\market}
        As interesting interpretation of these exploitive issues,
        since all three fiscal strategies will result in exponential
        market growth for every company participating in the market,
        is that they may represent, perhaps, an example of
        ``increasing returns.''

% Local Variables:
% TeX-parse-self: t
% TeX-auto-save: t
% TeX-master: "fractal.tex"
% End:


        %
% -----------------------------------------------------------------------------
%
% A license is hereby granted to reproduce this software source code and
% to create executable versions from this source code for personal,
% non-commercial use.  The copyright notice included with the software
% must be maintained in all copies produced.
%
% THIS PROGRAM IS PROVIDED "AS IS". THE AUTHOR PROVIDES NO WARRANTIES
% WHATSOEVER, EXPRESSED OR IMPLIED, INCLUDING WARRANTIES OF
% MERCHANTABILITY, TITLE, OR FITNESS FOR ANY PARTICULAR PURPOSE.  THE
% AUTHOR DOES NOT WARRANT THAT USE OF THIS PROGRAM DOES NOT INFRINGE THE
% INTELLECTUAL PROPERTY RIGHTS OF ANY THIRD PARTY IN ANY COUNTRY.
%
% Copyright (c) 1994-2006, John Conover, All Rights Reserved.
%
% Comments and/or bug reports should be addressed to:
%
%     john@email.johncon.com (John Conover)
%
% -----------------------------------------------------------------------------
%
% Revision: \RCSRevision \\
% Revision Time: \RCSTime UMT \\
% Revision Date: \RCSDate \\
% Revision Id: \RCSId \\
% Revision File: \RCSLog \\
\RCS $Revision: 0.0 $
\RCS $Date: 2006/01/20 04:38:13 $
\RCS $Id: operations.tex,v 0.0 2006/01/20 04:38:13 john Exp $
% $Log: operations.tex,v $
% Revision 0.0  2006/01/20 04:38:13  john
% Initial version
%
%
    \subsection{Fixed Increment Approximation for Operational Strategy}
        \label{\SETLABEL:OPS}.

        This section derives various values based on the ``average''
        of the normalized increments presented in
        Figure~\ref{\SETLABEL:TFA}. These values are an approximation
        to a, probably, complex process with a distribution shown in
        Figure~\ref{\SETLABEL:TF}. These values will be used in a
        fixed increment Brownian fractal analysis and simulation of
        the {\market}, and may, or may not, provide adequate accuracy
        for projections.

        \subidx{\market}{fiscal strategy}
        \subidx{\market}{Shannon probability}
        \subidx{strategy}{fiscal}
        \subidx{fiscal}{strategy}
        \subidx{Shannon}{probability}
        \subidx{probability}{Shannon}
        It should be noted that the analysis of fiscal strategy,
        presented in Section~\ref{\SETLABEL:FS}, is derived from the
        {\market} metrics and may, or may not, be maximally
        optimal. For the optimal fiscal strategy, which may be
        exploitable, see Section~\ref{\SETLABEL:MAXSHANNON}.

        \subidx{strategy}{exploitable}
        \subidx{exploitable}{strategy}
        \subidx{\market}{windows of opportunity}
        \idx{windows of opportunity}
        \subidx{decision}{obsolete}
        \subidx{obsolete}{decision}
        \subidx{decision}{timeliness}
        \subidx{timeliness}{decision}
        \subidx{rate of revenue returns}{forecast}
        \subidx{forecast}{rate of revenue returns}
        An additional exploitable strategy may be time itself.
        Equations~\ref{\SETLABEL:V},~\ref{\SETLABEL:R},
        and,~\ref{\SETLABEL:MA}, are, essentially, metrics on how fast
        a decision, which is based on information concerning the
        current status of the {\market}, becomes obsolete. Obviously,
        how long a decision is expected to remain relevant should be
        addressed as an operational necessity in strategic planning
        and project management. Figures~\ref{\SETLABEL:FN},
        and,~\ref{\SETLABEL:FF} compare methods of approximation of
        the ``forecastability'' of rate of revenue returns in the
        {\market} for the near term and far
        term~\cite[pp. 83-84]{Peters:CAOITCM}, respectively. As a
        general rule, caution must be exercised when making decisions
        that will span a time interval larger than the time interval
        where the ``forecastability'' of rate of revenue returns drops
        below 50\%. Beyond this time interval, the chances increase
        that the competitive and market forces will alter the market
        environment in a possibly detrimental unanticipated
        fashion. Obviously, there is significant advantage in
        ``timeliness'' of development, manufacturing, and distribution
        of products and services that are consistent with this
        temporal agenda. Automation of these processes, if executed
        consistently with this agenda, should be considered a
        competitive advantage.

        \subidx{strategy}{exploitable}
        \subidx{exploitable}{strategy}
        \subidx{rate of revenue returns}{forecast}
        \subidx{forecast}{rate of revenue returns}
        \idx{product life cycle}
        \idx{life cycle, product}
        In some sense, this temporal agenda defines the ``average''
        product or service life cycle in the {\market}. When the
        ``forecastability'' of rate of revenue returns drops below
        50\%, there is an even chance that the rate of revenue returns
        for the product or service will change in a detrimental
        fashion. If it is assumed that a product or service life cycle
        consists of a ramp up, a maintenence interval, and a ramp
        down, then, if all three life cycle intervals are equal, the
        product life cycle will be, approximately, three times the
        time interval where the ``forecastability'' of rate of revenue
        returns drops below 50\%. Although probably not an accurate
        prediction of product or service life cycle, the technique may
        be used as a conceptual approximation to the dynamics of
        ``market windows.\footnote{For example, consider the market
        for table salt. Since it has inelastic supply and demand
        curves, and is a necessary requirement for life, it would be
        expected that the Hurst coefficient would be very near
        unity---ignoring competitive pressures in the market. The
        predictability of the table salt market would, therefore, be
        expected to be relatively good, over time.}''  The conceptual
        approximation will probably predict a ``conservative'' or
        ``pessimistic'' value in relation to actual markets.

        \begin{figure}[ht]
            \begin{center}
                \begin{minipage}[t]{0.45\textwidth}
                    \epsfxsize=1.0\linewidth
                    \epsffile{\directory/datahurstlownear.eps}
                    \caption[{\market}, ``forecastability'' of near
                        term rate of revenue returns]{{\market},
                        ``forecastability'' of near term rate of
                        revenue returns. Although the error function
                        is the most accurate, for the near term,
                        $H^{t} = \thurstlow^{t}$ may be used as a
                        reliable metric of ``forecastability'' of the
                        rate of revenue returns.}
                    \label{\SETLABEL:FN}
                \end{minipage}
                \hfill
                \begin{minipage}[t]{0.45\textwidth}
                    \epsfxsize=1.0\linewidth
                    \epsffile{\directory/datahurstlowfar.eps}
                    \caption[{\market}, ``forecastability'' of far
                        term rate of revenue returns]{{\market},
                        ``forecastability'' of far term rate of
                        revenue returns. Although the error function
                        is the most accurate, for the far term,
                        $\frac{1}{\sqrt{t}}$ may be used as a reliable
                        metric of ``forecastability'' of the rate of
                        revenue returns.}
                    \label{\SETLABEL:FF}
                \end{minipage}
            \end{center}
        \end{figure}

        \idx{operations research}
        As an interesting interpretation of the data presented in
        Figure~\ref{\SETLABEL:FN}, there may be, perhaps, some
        applicability to such operational agendas as inventory
        control. Maintaining too little inventory, obviously, will
        create a situation where the organization can not exploit
        market expansion, and maintaining too much inventory,
        likewise, would over extend the company, creating unnecessary
        losses when the market contracts. The company should maintain
        inventory levels that do not exceed, from
        Equation~\ref{\SETLABEL:MA}, ${\thurstlow}^{n} = 0.5$
        {\timescale}s of operations. Since the optimal amount of
        inventory and, from Equation~\ref{\SETLABEL:V}, the variance
        of change in the rate of revenue returns in the future can be
        calculated, there may, perhaps, be some applicability to a
        forecasting methodology that can be incorporated into other
        areas of operations research, for example the linear algebras
        using simplex methodologies for optimization of manufacturing
        processes. Traditionally, these forecasts are made by the
        sales department, and are subject to various subjective
        biases.

% Local Variables:
% TeX-parse-self: t
% TeX-auto-save: t
% TeX-master: "fractal.tex"
% End:


        %
% -----------------------------------------------------------------------------
%
% A license is hereby granted to reproduce this software source code and
% to create executable versions from this source code for personal,
% non-commercial use.  The copyright notice included with the software
% must be maintained in all copies produced.
%
% THIS PROGRAM IS PROVIDED "AS IS". THE AUTHOR PROVIDES NO WARRANTIES
% WHATSOEVER, EXPRESSED OR IMPLIED, INCLUDING WARRANTIES OF
% MERCHANTABILITY, TITLE, OR FITNESS FOR ANY PARTICULAR PURPOSE.  THE
% AUTHOR DOES NOT WARRANT THAT USE OF THIS PROGRAM DOES NOT INFRINGE THE
% INTELLECTUAL PROPERTY RIGHTS OF ANY THIRD PARTY IN ANY COUNTRY.
%
% Copyright (c) 1994-2006, John Conover, All Rights Reserved.
%
% Comments and/or bug reports should be addressed to:
%
%     john@email.johncon.com (John Conover)
%
% -----------------------------------------------------------------------------
%
% Revision: \RCSRevision \\
% Revision Time: \RCSTime UMT \\
% Revision Date: \RCSDate \\
% Revision Id: \RCSId \\
% Revision File: \RCSLog \\
\RCS $Revision: 0.0 $
\RCS $Date: 2006/01/20 04:38:13 $
\RCS $Id: simulation.tex,v 0.0 2006/01/20 04:38:13 john Exp $
% $Log: simulation.tex,v $
% Revision 0.0  2006/01/20 04:38:13  john
% Initial version
%
%
    \subsection{Simulation of Fixed Increment Approximation for Fiscal Strategy}
        \label{\SETLABEL:TSUNFAIRBROWNIAN}

        \subidx{\market}{market simulation}
        The data in this section is presented in tabular form in
        Section~\ref{\SETLABELREF:SIM}.
        Figure~\ref{\SETLABEL:TSUNFAIRBROWNIAN0} represents a
        constructional simulation of the time series data presented in
        Figure~\ref{\SETLABEL:TS}. The program {\it
        tsunfairbrownian}\/, which is briefly described in
        appendix~\ref{programs}, was used in the reconstruction. The
        reconstructed data is superimposed on the original time series
        data.  The program, {\it tsunfairbrownian}\/, essentially,
        constructs the new time series as a Brownian fractal with
        fixed increments---the value of the fixed increment is derived
        from the root mean square average of the normalized increments
        presented in Figure~\ref{\SETLABEL:TF}. The ``quality'' of
        such a reconstruction should be subject to adequate scepticism
        and scrutiny since, in all probability, the normalized
        increments presented in Figure~\ref{\SETLABEL:TF} represent a
        relatively complex process, that may not be ``modeled'' with
        such a simple methodology.

        As a further comparison of the the constructional simulation
        with the original time series data,
        Figure~\ref{\SETLABEL:TSUNFAIRBROWNIAN1} presents a normalized
        histogram of the normalized increments of the reconstructed
        time series, superimposed on the normalized histogram
        presented in Figure~\ref{\SETLABEL:NH}.

        \subidx{\market}{fiscal strategy, simulation}
        \subidx{markets}{simulation}
        \subidx{simulation}{markets}
        \subidx{strategy}{fiscal, simulation}
        \subidx{fiscal}{strategy, simulation}
        \subidx{programs}{tsunfairbrownian}
        \subidx{tsunfairbrownian}{program}
        \begin{figure}[ht]
            \begin{center}
                \begin{minipage}[t]{0.45\textwidth}
                    \epsfxsize=1.0\linewidth
                    \epsffile{\directory/tsunfairbrownian-f.eps}
                    \caption[{\market}, Time series data, empirical and
                        simulated]{{\market}, Time series data, empirical
                        and simulated, using the program {\it tsunfairbrownian}\/
                        with f = {\datafractionrms}. This data is
                        superimposed on the data presented in
                        Figure~\ref{\SETLABEL:TS}.}
                    \label{\SETLABEL:TSUNFAIRBROWNIAN0}
                \end{minipage}
                \hfill
                \begin{minipage}[t]{0.45\textwidth}
                    \epsfxsize=1.0\linewidth
                    \epsffile{\directory/tsunfairbrownian-f.tsfraction.tsnormal-s30.eps}
                    \caption[{\market}, normalized histogram,
                        empirical and simulated]{{\market}, normalized
                        histogram of the normalized increments of the
                        time series data shown in
                        Figure~\ref{\SETLABEL:TSUNFAIRBROWNIAN0},
                        empirical and simulated.  The empirical data
                        has a mean of {\datafractionmean}, with a
                        standard deviation of {\datafractionstddev}.
                        By comparison, the simulated data has a mean
                        of {\tsunfairbrownianfractionmean} with a
                        standard deviation of
                        {\tsunfairbrownianfractionstddev}. This data
                        is superimposed on the data presented in
                        Figure~\ref{\SETLABEL:NH}. The area under the
                        four curves is identical.}
                    \label{\SETLABEL:TSUNFAIRBROWNIAN1}
                \end{minipage}
            \end{center}
        \end{figure}

% Local Variables:
% TeX-parse-self: t
% TeX-auto-save: t
% TeX-master: "fractal.tex"
% End:


        %
% -----------------------------------------------------------------------------
%
% A license is hereby granted to reproduce this software source code and
% to create executable versions from this source code for personal,
% non-commercial use.  The copyright notice included with the software
% must be maintained in all copies produced.
%
% THIS PROGRAM IS PROVIDED "AS IS". THE AUTHOR PROVIDES NO WARRANTIES
% WHATSOEVER, EXPRESSED OR IMPLIED, INCLUDING WARRANTIES OF
% MERCHANTABILITY, TITLE, OR FITNESS FOR ANY PARTICULAR PURPOSE.  THE
% AUTHOR DOES NOT WARRANT THAT USE OF THIS PROGRAM DOES NOT INFRINGE THE
% INTELLECTUAL PROPERTY RIGHTS OF ANY THIRD PARTY IN ANY COUNTRY.
%
% Copyright (c) 1994-2006, John Conover, All Rights Reserved.
%
% Comments and/or bug reports should be addressed to:
%
%     john@email.johncon.com (John Conover)
%
% -----------------------------------------------------------------------------
%
% Revision: \RCSRevision \\
% Revision Time: \RCSTime UMT \\
% Revision Date: \RCSDate \\
% Revision Id: \RCSId \\
% Revision File: \RCSLog \\
\RCS $Revision: 0.0 $
\RCS $Date: 2006/01/20 04:38:13 $
\RCS $Id: maximum.tex,v 0.0 2006/01/20 04:38:13 john Exp $
% $Log: maximum.tex,v $
% Revision 0.0  2006/01/20 04:38:13  john
% Initial version
%
%
    \subsection{Simulation of Fixed Increment Approximation for Optimally Maximal Fiscal Strategy}
        \label{\SETLABEL:MAXSHANNON}
        \subidx{\market}{fiscal strategy, simulation}
        \subidx{\market}{maximum Shannon probability}
        \subidx{markets}{simulation}
        \subidx{simulation}{markets}
        \subidx{strategy}{optimum fiscal, simulation}
        \subidx{fiscal}{optimum strategy, simulation}
        \subidx{programs}{tsunfairbrownian}
        \subidx{tsunfairbrownian}{program}
        \subidx{Shannon}{probability}
        \subidx{probability}{Shannon}

        \subidx{strategy}{exploitable}
        \subidx{exploitable}{strategy}
        \subidx{programs}{tsshannonmax}
        \subidx{tsshannonmax}{program}
        \subidx{programs}{tsunfairbrownian}
        \subidx{tsunfairbrownian}{program}
        \subidx{strategy}{fiscal}
        \subidx{fiscal}{strategy}
        The data in this section is presented in tabular form in
        Section~\ref{\SETLABELREF:MAXSHANNON}. One of the issues of
        analysis, as mentioned in Section~\ref{\SETLABEL:OPS}, is to
        determine the maximum Shannon probability for the time series
        presented in Figure~\ref{\SETLABEL:TS}. Potentially, this
        could be exploited with an aggressive fiscal
        strategy. Figure~\ref{\SETLABEL:SHANNONMAX0} is a graph of the
        output of the {\it tsshannonmax}\/ program, which is described
        briefly in appendix~\ref{programs}. The maximum of this
        function is the maximum Shannon probability for the time
        series data presented in Figure~\ref{\SETLABEL:TS}.
        Figure~\ref{\SETLABEL:SHANNONMAX1} was constructed using {\it
        tsunfairbrownian}\/ program, which is also described in
        appendix~\ref{programs}, with the maximum Shannon probability,
        and the time series data presented in
        Figure~\ref{\SETLABEL:TS}. This represents a ``what if'' the
        investment strategy was changed from a Shannon probability of
        {\shannonlogreturns}, as derived in Section~\ref{\SETLABEL:SP}
        to {\shannonmax}. This process, essentially, extracts the
        random statistical data from the time series presented in
        Figure~\ref{\SETLABEL:TS}, and constructs a new time series,
        using the random statistical data, with a different investment
        strategy.  The program, {\it tsunfairbrownian}\/, essentially,
        constructs the new time series as a Brownian fractal with
        fixed increments.  The ``quality'' of such a reconstruction
        should be subject to adequate scepticism and scrutiny since,
        in all probability, the increments in the original data
        represent a relatively complex process, that may not be
        ``modeled'' with such a simple methodology.

        \begin{figure}[ht]
            \begin{center}
                \begin{minipage}[t]{0.45\textwidth}
                    \epsfxsize=1.0\linewidth
                    \epsffile{\directory/data.tsshannonmax.eps}
                    \caption[{\market}, maximum rate of revenue
                        returns] {{\market}, maximum rate of revenue
                        returns, per {\timescale}, vs. Shannon
                        probability. The maximum rate of revenue
                        returns, per {\timescale}, occurs at a Shannon
                        probability of {\shannonmax}.}
                    \label{\SETLABEL:SHANNONMAX0}
                \end{minipage}
                \hfill
                \begin{minipage}[t]{0.45\textwidth}
                    \epsfxsize=1.0\linewidth
                    \epsffile{\directory/data.tsshannonmax-p.tsunfairbrownian-p.eps}
                    \caption[{\market}, maximum rate of revenue
                        returns] {{\market}, maximum rate of revenue
                        returns, per {\timescale}, at a Shannon
                        probability, of {\shannonmax}, corresponding
                        to a ``wager'' fraction of {\twoponemax}.}
                    \label{\SETLABEL:SHANNONMAX1}
                \end{minipage}
            \end{center}
        \end{figure}

        \subidx{fractional}{Brownian motion}
        \subidx{Brownian motion}{fractional}
        \subidx{Shannon}{probability}
        \subidx{probability}{Shannon}
        \subidx{programs}{tsshannonmax}
        \subidx{tsshannonmax}{program}
        If it is assumed that the time series data set, presented in
        Figure~\ref{\SETLABEL:TS}, constitutes classical Brownian
        motion, then the Shannon probability can be calculated by
        counting the total number of {\timescale}s that the {\market}
        movement was positive, and dividing by the total number of
        {timescale}s represented in the time series. This quotient is
        {\pmax}, as compared with the predicted value from the program
        {\it tsshannonmax}\/ of {\shannonmax}.

% Local Variables:
% TeX-parse-self: t
% TeX-auto-save: t
% TeX-master: "fractal.tex"
% End:


        %
% -----------------------------------------------------------------------------
%
% A license is hereby granted to reproduce this software source code and
% to create executable versions from this source code for personal,
% non-commercial use.  The copyright notice included with the software
% must be maintained in all copies produced.
%
% THIS PROGRAM IS PROVIDED "AS IS". THE AUTHOR PROVIDES NO WARRANTIES
% WHATSOEVER, EXPRESSED OR IMPLIED, INCLUDING WARRANTIES OF
% MERCHANTABILITY, TITLE, OR FITNESS FOR ANY PARTICULAR PURPOSE.  THE
% AUTHOR DOES NOT WARRANT THAT USE OF THIS PROGRAM DOES NOT INFRINGE THE
% INTELLECTUAL PROPERTY RIGHTS OF ANY THIRD PARTY IN ANY COUNTRY.
%
% Copyright (c) 1994-2006, John Conover, All Rights Reserved.
%
% Comments and/or bug reports should be addressed to:
%
%     john@email.johncon.com (John Conover)
%
% -----------------------------------------------------------------------------
%
% Revision: \RCSRevision \\
% Revision Time: \RCSTime UMT \\
% Revision Date: \RCSDate \\
% Revision Id: \RCSId \\
% Revision File: \RCSLog \\
\RCS $Revision: 0.0 $
\RCS $Date: 2006/01/20 04:38:13 $
\RCS $Id: verification.tex,v 0.0 2006/01/20 04:38:13 john Exp $
% $Log: verification.tex,v $
% Revision 0.0  2006/01/20 04:38:13  john
% Initial version
%
%
    \subsection{Qualitative Verification of Fixed Increment Approximation Analysis}
        \label{\SETLABEL:QVA}

        \subidx{\market}{verification of analysis}
        \subidx{verification}{analysis}
        \subidx{analysis}{verification}
        \subidx{quality}{of analysis}
        \subidx{verification}{of methodology}
        \subidx{methodology}{verification of}
        \subidx{Shannon}{probability}
        \subidx{probability}{Shannon}

        This section evaluates various values based on the ``average''
        of the normalized increments presented in
        Figure~\ref{\SETLABEL:TFA}. These values are an approximation
        to a, probably, complex process with a distribution shown in
        Figure~\ref{\SETLABEL:TF}. These values will be used in a
        fixed increment Brownian fractal analysis of the {\market},
        and may, or may not, provide adequate accuracy for
        projections.

        The data in this section is presented in tabular form in
        sections~\ref{\SETLABELREF:VI1} and~\ref{\SETLABELREF:VI2}.
        As a subjective evaluation of the ``quality'' of the analysis
        of the {\market}, from Chapter~\ref{methodology},
        Equation~\ref{metricvalues1}, and using the mean and root mean
        square values of the normalized increments of the time series
        data presented in Figure~\ref{\SETLABEL:TS} from
        Figure~\ref{\SETLABEL:TF}, and the Shannon probability as
        calculated by counting the total number of {\timescale}s that
        the {\market} movement was positive, as presented in
        Section~\ref{\SETLABEL:MAXSHANNON}:

        \begin{eqnarray}
                  P & \approx & \frac{\frac{avg}{rms} + 1}{2}\\
            {\pmax} & \approx & \frac{\frac{\datafractionmean}{\datafractionrms} + 1}{2}\\
            {\pmax} & \approx & {\avgrms}
            \label{\SETLABEL:AVGS}
        \end{eqnarray}

        \subidx{Shannon}{probability}
        \subidx{probability}{Shannon}
        \noindent and comparing these values to the Shannon
        probability, as found by the {\it tsshannonmax}\/ program, which
        iterates for a maximum:

        \begin{eqnarray}
            {\pmax} \approx {\avgrms} \approx {\shannonmax}
        \end{eqnarray}

        \subidx{logarithmic}{returns}
        \subidx{returns}{logarithmic}
        In addition, the different methods of calculating the
        logarithmic returns, presented in Section~\ref{\SETLABEL:FS},
        should be compared. The four methods used were the mean of
        Figure~\ref{\SETLABEL:TF}, the constant in the least squares
        approximation to Figure~\ref{\SETLABEL:TF}, the least squares
        exponential approximation to Figure~\ref{\SETLABEL:TS}, and
        the logarithmic returns of Figure~\ref{\SETLABEL:TS}, derived
        as the mean of the logarithms of the quotients of the
        increments. The values for each of the methods are,
        respectively:

        \begin{equation}
            \datafractionmeanbits \approx \datafractionconstantbits \approx \datatslsqepbits \approx \logreturns
        \end{equation}

        It is implied in Section~\ref{\SETLABEL:FS},
        Subsection~\ref{\SETLABEL:SP} and in
        Section~\ref{\SETLABEL:TSUNFAIRBROWNIAN} that, a Brownian
        motion with fixed increments fractal may ``model'' the
        {\market}. Using Equation~\ref{stddev9} from
        Chapter~\ref{general}, Section~\ref{abmfi}:

        \begin{eqnarray}
                                    rms \left(2P - 1\right) & \approx & \frac{\sigma \left(2P - 1\right)}{2 \sqrt{P\left(1 - P\right)}}\\
            \datafractionrms \left(2 \cdot \pmax - 1\right) & \approx & \frac{\datafractionstddev \left(2 \cdot \pmax - 1\right)}{2\sqrt{\pmax \left(1 - \pmax\right)}}\\
                       \datafractionrms \cdot \twopminusone & \approx & \datafractionstddev \cdot \twopx\\
                                                      \rmsp & \approx & \sigmap
        \end{eqnarray}

        \noindent and, equating to the mean:

        \begin{equation}
            \datafractionmean \approx \rmsp \approx \sigmap
        \end{equation}

        \subidx{Shannon}{probability}
        \subidx{probability}{Shannon}
        \noindent where, as in Equation~\ref{\SETLABEL:AVGS} using the
        mean, root mean square, and standard deviation values of the
        normalized increments of the time series data presented in
        Figure~\ref{\SETLABEL:TS} from Figure~\ref{\SETLABEL:TF}, and
        the Shannon probability as calculated by counting the total
        number of {\timescale}s that the {\market} movement was
        positive, as presented in Section~\ref{\SETLABEL:MAXSHANNON}.

        As a final qualitative comparison, the absolute value of the
        normalized increments should be the same as the root mean
        square value\footnote{The absolute value of the normalized
        increments, when averaged, is related to the root mean square
        of the increments by a constant. If the normalized increments
        are a fixed increment, the constant is unity. If the
        normalized increments have a Gaussian distribution, the
        constant is $\approx 0.8$ depending on the accuracy of of
        ``fit'' to a Gaussian distribution.}, where the absolute value
        is presented in Figure~\ref{\SETLABEL:TFA}, and the root mean
        square value is presented in Figure~\ref{\SETLABEL:TF}:

        \begin{equation}
            \datafractionabsmean \approx \datafractionrms
        \end{equation}

        Note, that if the {\market} could be ``modeled'' as a Brownian
        motion with fixed increments fractal, then the standard
        deviation of the absolute value of the normalized increments
        of the time series data presented in Figure~\ref{\SETLABEL:TS}
        from Figure~\ref{\SETLABEL:TF} should be zero. It is
        $\datafractionabsstddev$.

% Local Variables:
% TeX-parse-self: t
% TeX-auto-save: t
% TeX-master: "fractal.tex"
% End:


    \renewcommand{\market}{Time Sampled Non-optimal Coin Tossing Game}
    \renewcommand{\directory}{../markets/tscoin.tsunfairbrownian.tssample}
    \renewcommand{\datafractionmean}{0.008052}
\renewcommand{\datafractionmeanbits}{0.011570}
\renewcommand{\datafractionmeanq}{0.002684}
\renewcommand{\datafractionmeanbitsq}{0.003867}
\renewcommand{\datafractionstddev}{0.038579}
\renewcommand{\datafractionrms}{0.039311}
\renewcommand{\avgrms}{0.602414}
\renewcommand{\ncompanies}{5.210454}
\renewcommand{\pncompanies}{0.544866}
\renewcommand{\datafractionabsmean}{0.029745}
\renewcommand{\datafractionabsstddev}{0.025769}
\renewcommand{\datafractionconstant}{0.010041}
\renewcommand{\datafractionconstantbits}{0.014414}
\renewcommand{\datafractionconstantq}{0.003347}
\renewcommand{\datafractionconstantbitsq}{0.004821}
\renewcommand{\datafractionslope}{-0.000021}
\renewcommand{\datafractionabsconstant}{0.035145}
\renewcommand{\datafractionabsslope}{-0.000057}
\renewcommand{\hurstall}{0.659558}
\renewcommand{\hurstlow}{0.707509}
\renewcommand{\hurstlowtwo}{1.415018}
\renewcommand{\hurstlowhundred}{70.750900}
\renewcommand{\hcalcall}{0.184942}
\renewcommand{\hcalclow}{0.102042}
\renewcommand{\shannonmax}{0.604167}
\renewcommand{\twoponemax}{0.208334}
\renewcommand{\logreturns}{0.010456}
\renewcommand{\twologreturns}{1.007274}
\renewcommand{\twologreturnshundred}{0.727387}
\renewcommand{\oneoverlogreturns}{95.638868}
\renewcommand{\pmax}{0.602094}
\renewcommand{\twopminusone}{0.204188}
\renewcommand{\rmsp}{0.008027}
\renewcommand{\twopx}{0.208583}
\renewcommand{\sigmap}{0.008047}
\renewcommand{\tsunfairbrownianfractionmean}{0.007862}
\renewcommand{\tsunfairbrownianfractionstddev}{0.038619}
\renewcommand{\shannonlogreturns}{0.560125}
\renewcommand{\shannonlogreturnshundred}{56.012500}
\renewcommand{\twopone}{0.120250}
\renewcommand{\twoponehundred}{12.025000}
\renewcommand{\hundredtwoponehundred}{87.975000}
\renewcommand{\hundredshannonlogreturnshundred}{43.987500}
\renewcommand{\datatslsqepbits}{0.007623}
\renewcommand{\thurstall}{0.633980}
\renewcommand{\thurstlow}{0.710108}
\renewcommand{\thurstlowtwo}{1.420216}
\renewcommand{\thurstlowhundred}{71.010800}
\renewcommand{\thcalcall}{0.247886}
\renewcommand{\thcalclow}{0.171737}
\renewcommand{\chisquared}{2.862000}
\renewcommand{\critical}{42.557000}

    \renewcommand{\timescale}{tosses}
    \subidx{market}{\market}
    \idx{\market}

    \section{\market}

        \renewcommand{\SETLABEL}{\LABPRE:TSNOCT}
        \renewcommand{\SETLABELQ}{\LABPRE:TSNOCTQ}
        \label{\SETLABEL}
        \renewcommand{\SETLABELREF}{\LABPREREF:TSNOCT}

        \idx{tscoin}
        \idx{tsunfairbrownian}
        \idx{tssample}
        \subidx{programs}{tscoin}
        \subidx{tscoin}{program}
        \subidx{programs}{tsunfairbrownian}
        \subidx{tsunfairbrownian}{program}
        \subidx{programs}{tssample}
        \subidx{tssample}{program}
        For the analysis, the data was in the directory
        {\directory}\footnote{As a simulation model, the program {\it
        tscoin}\/ was run to make a time series data file, with the
        following parameters:

        \vspace{0.1in}
        {\noindent}tscoin -p 0.70 1500 > data.1
        \vspace{0.1in}

        \noindent to make a time series of 1500 elements, with a
        Shannon probability of 0.70.  In addition, the program {\it
        tsunfairbrownian}\/ was run on the data file with the
        following parameters:

        \vspace{0.1in}
        {\noindent}tsunfairbrownian -f 0.0894 data.1 > data.2
        \vspace{0.1in}

        \noindent to make a time series with a known non-optimal
        investment strategy. The value, 0.0894 was calculated by
        reducing the desired value, 0.2, by a factor of
        $\frac{1}{\sqrt{5}}$, where the sampling occurs every fifth
        time series element. Then the program {\it tssample}\/ was run
        with the following parameters:

        \vspace{0.1in}
        {\noindent}tssample -i 5  data.2 > data
        \vspace{0.1in}

        \noindent to time sample every fifth element in the time
        series to make a time sampled time series with a known non
        optimal investment strategy. The data is by {\timescale}.}.

        The data in this section is presented in tabular form in
        Section~\ref{\SETLABELREF}. Note that in this analysis, the
        rate of revenue returns means the increase or decrease in the
        cumulative sum of the {\market}. This is included for
        ``theoretical'' comparative purposes.

        %
% -----------------------------------------------------------------------------
%
% A license is hereby granted to reproduce this software source code and
% to create executable versions from this source code for personal,
% non-commercial use.  The copyright notice included with the software
% must be maintained in all copies produced.
%
% THIS PROGRAM IS PROVIDED "AS IS". THE AUTHOR PROVIDES NO WARRANTIES
% WHATSOEVER, EXPRESSED OR IMPLIED, INCLUDING WARRANTIES OF
% MERCHANTABILITY, TITLE, OR FITNESS FOR ANY PARTICULAR PURPOSE.  THE
% AUTHOR DOES NOT WARRANT THAT USE OF THIS PROGRAM DOES NOT INFRINGE THE
% INTELLECTUAL PROPERTY RIGHTS OF ANY THIRD PARTY IN ANY COUNTRY.
%
% Copyright (c) 1994-2006, John Conover, All Rights Reserved.
%
% Comments and/or bug reports should be addressed to:
%
%     john@email.johncon.com (John Conover)
%
% -----------------------------------------------------------------------------
%
% Revision: \RCSRevision \\
% Revision Time: \RCSTime UMT \\
% Revision Date: \RCSDate \\
% Revision Id: \RCSId \\
% Revision File: \RCSLog \\
\RCS $Revision: 0.0 $
\RCS $Date: 2006/01/20 04:38:13 $
\RCS $Id: fraction.tex,v 0.0 2006/01/20 04:38:13 john Exp $
% $Log: fraction.tex,v $
% Revision 0.0  2006/01/20 04:38:13  john
% Initial version
%
%
    \subsection{Time Series Increments Analysis}
        \label{\SETLABEL:TSA}

        \subidx{\market}{Time series analysis}
        \subidx{time series}{increments}
        \subidx{time series}{analysis}
        \subidx{cumulative sum}{analysis}
        \subidx{analysis}{cumulative sum}
        \subidx{analysis}{random process}
        \subidx{random process}{analysis}
        \subidx{Gaussian}{increments}
        \subidx{increments}{Gaussian}
        \subidx{Brownian}{motion, fractional}
        \subidx{fractional}{Brownian motion}
        \subidx{fractal}{Brownian motion}
        The data in this section is presented in tabular form in
        Section~\ref{\SETLABELREF:TSA}.  Figure~\ref{\SETLABEL:TS} is
        a graph of the time series data for the {\market}.

        \subidx{increments}{normalized}
        \subidx{normalized}{increments}
        \subidx{programs}{tsfraction}
        \subidx{tsfraction}{program}
        Figure~\ref{\SETLABEL:TF} is a graph of the normalized
        increments of the time series data presented in
        Figure~\ref{\SETLABEL:TS}. The data presented was made by
        running the program {\it tsfraction}\/ on the time series
        data. The program {\it tsfraction}\/ is described briefly in
        Appendix~\ref{programs}, and subtracts the previous value from
        the next value, dividing this difference by the previous
        value, for each element in the time series data. The new time
        series contains the instantaneous change in the rate of
        revenue returns, divided by the magnitude of the instantaneous
        rate of revenue returns.

        \subidx{mean}{standard deviation}
        \subidx{standard deviation}{mean}
        \idx{root mean square}
        \idx{least squares approximation}
        \begin{figure}[ht]
            \begin{center}
                \begin{minipage}[t]{0.45\textwidth}
                    \epsfxsize=1.0\linewidth
                    \epsffile{\directory/data.eps}
                    \caption{{\market}, time series data.}
                    \label{\SETLABEL:TS}
                    \label{\SETLABELQ:TS}
                \end{minipage}
                \hfill
                \begin{minipage}[t]{0.45\textwidth}
                    \epsfxsize=1.0\linewidth
                    \epsffile{\directory/data.tsfraction.eps}
                    \caption[{\market}, normalized
                        increments]{{\market}, normalized increments
                        of the time series data presented in
                        Figure~\ref{\SETLABEL:TS}. The mean is
                        {\datafractionmean} with a standard deviation
                        of {\datafractionstddev}. The formula for the
                        least squares approximation is
                        ${\datafractionconstant} +
                        {\datafractionslope}t$, and the root mean
                        squared value is {\datafractionrms}. The
                        graph, labeled ``data\-.tsfraction\-.tsrms,''
                        is the running root mean square, and
                        ``data\-.tsfraction\-.tsavg'' is the running
                        average of the normalized increments.  This
                        graph is the fraction of change in the time
                        series, as a function of time. Note that the
                        slope of the mean, {\datafractionslope}, is
                        the coefficient of the nonlinearity term in
                        the normalized increments. See
                        Chapter~\ref{general}, Section~\ref{nlextend}
                        for a possible application of the logistic
                        function to this data set.}
                    \label{\SETLABEL:TF}
                    \label{\SETLABELQ:TF}
                \end{minipage}
            \end{center}
        \end{figure}

        \subidx{absolute value}{increments}
        \subidx{increments}{absolute value}

        Figure~\ref{\SETLABEL:TFA} is a graph of the absolute value of
        the normalized increments of the time series data presented in
        Figure~\ref{\SETLABEL:TF}. The data presented was made by
        running the Unix utility sed(1) on the normalized increments
        time series data to remove the negative signs. This is an
        absolute value procedure.  The resulting time series contains
        the absolute value of the instantaneous change in the rate of
        revenue returns, divided by the magnitude of the instantaneous
        rate of revenue returns\footnote{The absolute value of the
        normalized increments, when averaged, is related to the root
        mean square of the increments by a constant. If the normalized
        increments are a fixed increment, the constant is unity. If
        the normalized increments have a Gaussian distribution, the
        constant is $\approx 0.8$ depending on the accuracy of of
        ``fit'' to a Gaussian distribution.}.

        \subidx{histogram}{normalized}
        \subidx{normalized}{histogram}
        \subidx{programs}{tsnormal}
        \subidx{tsnormal}{program}
        \subidx{mean}{standard deviation}
        \subidx{standard deviation}{mean}
        \idx{root mean square}
        \idx{least squares approximation}
        \subidx{\market}{analysis of increments}
        Figure~\ref{\SETLABEL:NH} is the normalized histogram of the
        normalized increments of the time series data shown in
        Figure~\ref{\SETLABEL:TF}. The abscissa is 3 $\sigma$ limits,
        and the area under the two curves is identical. The data for
        this figure was produced by the program {\it tsnormal}\/,
        which is described briefly in Appendix~\ref{programs}.

        \begin{figure}[ht]
            \begin{center}
                \begin{minipage}[t]{0.45\textwidth}
                    \epsfxsize=1.0\linewidth
                    \epsffile{\directory/data.tsfraction.abs.eps}
                    \caption[{\market}, absolute value of the
                        normalized increments]{{\market}, absolute
                        value of the normalized increments of the time
                        series data presented in
                        Figure~\ref{\SETLABEL:TF}.  The mean is
                        {\datafractionabsmean} with a standard
                        deviation of {\datafractionabsstddev}. The
                        formula for the least squares approximation is
                        ${\datafractionabsconstant} +
                        {\datafractionabsslope}t$, and the root mean
                        square value, from Figure~\ref{\SETLABEL:TF},
                        is {\datafractionrms}.  The graph, labeled
                        ``data\-.tsfraction\-.tsrms,'' is the running
                        root mean square, and
                        ``data\-.tsfraction\-.tsavg'' is the running
                        average of the normalized increments presented
                        in Figure~\ref{\SETLABEL:TF}, superimposed
                        here for convenience. This graph is the
                        absolute value of the fraction of change in
                        the time series, as a function of time.}
                    \label{\SETLABEL:TFA}
                    \label{\SETLABELQ:TFA}
                \end{minipage}
                \hfill
                \begin{minipage}[t]{0.45\textwidth}
                    \epsfxsize=1.0\linewidth
                    \epsffile{\directory/data.tsfraction.tsnormal-s30.eps}
                    \caption[{\market}, normalized histogram of the
                        normalized increments]{{\market}, normalized
                        histogram of the normalized increments of the
                        time series data shown in
                        Figure~\ref{\SETLABEL:TF}.  The data has a
                        mean of {\datafractionmean}, with a standard
                        deviation of {\datafractionstddev}.  The area
                        under the two curves is identical. The
                        $\chi^2$ value of the observed and expected
                        values of the two curves is {\chisquared},
                        with a critical value of {\critical}.}
                    \label{\SETLABEL:NH}
                \end{minipage}
            \end{center}
        \end{figure}

        \subidx{programs}{tsXsquared}
        \subidx{tsXsquared}{program}
        \subidx{\market}{chi-squared values of increments}
        The program {\it tsXsquared}\/, which is briefly described in
        appendix~\ref{programs}, was used to derive the $\chi^2$
        statistics for the data presented in
        Figure~\ref{\SETLABEL:NH}.

        \subidx{programs}{tsstatest}
        \subidx{tsstatest}{program}
        \subidx{\market}{statistical estimates}

        Figure~\ref{\SETLABEL:SE} is the statistical estimate for the
        data presented in Figure~\ref{\SETLABEL:TF}, as derived by the
        program {\it tsstatest}\/, which is briefly described in
        appendix~\ref{programs}.

        \begin{figure}[ht]
            \begin{center}
                \begin{minipage}[t]{\textwidth}
                    \center{\fbox{\parbox{0.9\textwidth}{\XXX{\directory/data.tsstatest-f0.1-c0.9-i.tex}}}}
                    \caption[{\market}, statistical estimates of the
                        normalized increments]{{\market}, statistical
                        estimates of the normalized increments of the
                        time series shown in Figure~\ref{\SETLABEL:TF}.
                        The table was produced with the {\it
                        tsstatest}\/ program, and illustrates the
                        size of the data set required for a confidence
                        level of 90\%, with an error estimate of $\pm$
                        10\%, or alternately, the error estimate on
                        the time series shown in Figure~\ref{\SETLABEL:TF}.}
                    \label{\SETLABEL:SE}
                \end{minipage}
            \end{center}
        \end{figure}

        Note that the data set size estimations, as produced by the
        {\it tsstatest}\/ program, are probably very conservative,
        depending on the magnitude of the Shannon probability, $P =
        \shannonlogreturns$, as derived in
        Section~\ref{\SETLABEL:SP}. See Chapter~\ref{general},
        Section~\ref{serdss} for possible alternative methodologies
        for addressing the analysis of fractal time series with
        limited data set sizes. Depending on the magnitude of the
        Shannon probability, $P$, these estimates can be several
        orders of magnitude too high.

        \subidx{derivative of increments}{normalized}
        \subidx{normalized}{derivative of increments}
        \subidx{programs}{tsderivative}
        \subidx{tsderivative}{program}
        Figure~\ref{\SETLABEL:TF1} is the normalized histogram of the
        first derivative of the normalized increments of the time
        series data shown in Figure~\ref{\SETLABEL:TF}. In principle,
        if the distribution of the normalized increments presented in
        Figure~\ref{\SETLABEL:NH} is Gaussian in nature, this
        distribution would be similar to ``white noise,'' as presented
        in appendix~\ref{programs}, Figure~\ref{whiteexample}. The
        data was generated by the {\it tsderivative}\/ program, which
        is briefly described in
        appendix~\ref{programs}. Figure~\ref{\SETLABEL:TF2} is the
        normalized histogram of the second derivative of the
        normalized increments of the time series data shown in
        Figure~\ref{\SETLABEL:TF}. In principle, if the distribution
        of the normalized increments presented in
        Figure~\ref{\SETLABEL:NH} is an integrated Gaussian
        distribution in nature, this distribution would be similar to
        ``white noise,'' as presented in appendix~\ref{programs},
        Figure~\ref{whiteexample}.

        \begin{figure}[ht]
            \begin{center}
                \begin{minipage}[t]{0.45\textwidth}
                    \epsfxsize=1.0\linewidth
                    \epsffile{\directory/data.tsfraction.tsderivative.tsnormal-s30.eps}
                    \caption[{\market}, histogram of the first
                        derivative of the increments]{{\market},
                        normalized histogram of the first derivative
                        of the normalized increments of the time
                        series data shown in
                        Figure~\ref{\SETLABEL:TF}.}
                    \label{\SETLABEL:TF1}
                \end{minipage}
                \hfill
                \begin{minipage}[t]{0.45\textwidth}
                    \epsfxsize=1.0\linewidth
                    \epsffile{\directory/data.tsfraction.2tsderivative.tsnormal-s30.eps}
                    \caption[{\market}, histogram of the second
                        derivative of the increments]{{\market},
                        normalized histogram of second derivative of
                        the the normalized increments of the time
                        series data shown in
                        Figure~\ref{\SETLABEL:TF}.}
                    \label{\SETLABEL:TF2}
                \end{minipage}
            \end{center}
        \end{figure}

        \subidx{fractal}{range}
        \subidx{fractal}{R/S analysis}
        \subidx{\market}{rate of revenue returns, range}
        \subidx{\market}{deterministic mechanism}
        \subidx{deterministic}{mechanism}
        \subidx{mechanism}{deterministic}
        Figure~\ref{\SETLABEL:TR} is the range of values of the time
        series shown in Figure~\ref{\SETLABEL:TS}. The horizontal axis
        is time into the future. In principle, if the time series was
        characterized as fractional Brownian motion the graph in
        Figure~\ref{\SETLABEL:TR} would be a square root
        function\footnote{Note that the ``roughness,'' or ``sawtooth''
        characteristics of the graph in Figure~\ref{\SETLABEL:TR} are
        a computational artifact---caused by not using the -m option
        to the program {\it tshurst}\/, which is computationally
        inefficient.}. Figure~\ref{\SETLABEL:TD} is the deterministic
        map of the normalized increments of the time series data shown
        in Figure~\ref{\SETLABEL:TF}. The deterministic map is useful
        for determining if a time series was created by a
        deterministic mechanism. This, essentially, maps each element
        in the time series with the previous element in the time
        series.  See,~\cite[pp. 745]{Peitgen}.

        \begin{figure}[ht]
            \begin{center}
                \begin{minipage}[t]{0.45\textwidth}
                    \epsfxsize=1.0\linewidth
                    \epsffile{\directory/data.tshurst-f.eps}
                    \caption[{\market}, range]{{\market}, range of the
                        time series data shown in
                        Figure~\ref{\SETLABEL:TS}.}
                    \label{\SETLABEL:TR}
                \end{minipage}
                \hfill
                \begin{minipage}[t]{0.45\textwidth}
                    \epsfxsize=1.0\linewidth
                    \epsffile{\directory/data.tsfraction.tsdeterministic.eps}
                    \caption[{\market}, deterministic map]{{\market},
                        deterministic map of the normalized increments
                        of the time series data shown in
                        Figure~\ref{\SETLABEL:TF}.}
                    \label{\SETLABEL:TD}
                \end{minipage}
            \end{center}
        \end{figure}

% Local Variables:
% TeX-parse-self: t
% TeX-auto-save: t
% TeX-master: "fractal.tex"
% End:


        %
% -----------------------------------------------------------------------------
%
% A license is hereby granted to reproduce this software source code and
% to create executable versions from this source code for personal,
% non-commercial use.  The copyright notice included with the software
% must be maintained in all copies produced.
%
% THIS PROGRAM IS PROVIDED "AS IS". THE AUTHOR PROVIDES NO WARRANTIES
% WHATSOEVER, EXPRESSED OR IMPLIED, INCLUDING WARRANTIES OF
% MERCHANTABILITY, TITLE, OR FITNESS FOR ANY PARTICULAR PURPOSE.  THE
% AUTHOR DOES NOT WARRANT THAT USE OF THIS PROGRAM DOES NOT INFRINGE THE
% INTELLECTUAL PROPERTY RIGHTS OF ANY THIRD PARTY IN ANY COUNTRY.
%
% Copyright (c) 1994-2006, John Conover, All Rights Reserved.
%
% Comments and/or bug reports should be addressed to:
%
%     john@email.johncon.com (John Conover)
%
% -----------------------------------------------------------------------------
%
% Revision: \RCSRevision \\
% Revision Time: \RCSTime UMT \\
% Revision Date: \RCSDate \\
% Revision Id: \RCSId \\
% Revision File: \RCSLog \\
\RCS $Revision: 0.0 $
\RCS $Date: 2006/01/20 04:38:13 $
\RCS $Id: instant.tex,v 0.0 2006/01/20 04:38:13 john Exp $
% $Log: instant.tex,v $
% Revision 0.0  2006/01/20 04:38:13  john
% Initial version
%
%
    \subsection{Instantaneous Analysis of Normalized Increments}
        \label{\SETLABEL:IA}

        \subidx{\market}{instantaneous analysis of normalized increments}
        \idx{average of normalized increments}
        \idx{root mean square of normalized increments}
        \subidx{Shannon probability}{instantaneous computation of}
        \subidx{average of normalized increments}{instantaneous computation of}
        \subidx{root mean square of normalized increments}{instantaneous computation of}
        \subidx{instantaneous computation}{Shannon probability}
        \subidx{instantaneous computation}{average of normalized increments}
        \subidx{instantaneous computation}{root mean square of normalized increments}
        \idx{time series}
        \subidx{time series}{instantaneous analysis}
        \subidx{instantaneous analysis}{time series}
        \subidx{time series}{increments}
        \subidx{time series}{analysis}
        \subidx{Shannon}{probability}
        \subidx{probability}{Shannon}
        \subidx{normalized}{increments}
        \subidx{increments}{normalized}

        The program {\it tsinstant}\/, which is briefly described in
        Appendix~\ref{programs}, is for finding the instantaneous
        fraction of change in a time series. The value of a sample in
        the time series is subtracted from the previous sample in the
        time series, and divided by the value of the previous sample.
        As explained in Chapter~\ref{general},
        Sections~\ref{derivation},~\ref{GA},~\ref{abmfi},~\ref{aftsma}
        and,~\ref{ompl} for Brownian motion, random walk fractals, the
        absolute value of the instantaneous fraction of change is also
        the root mean square of the instantaneous fraction of
        change\footnote{The absolute value of the normalized
        increments, when averaged, is related to the root mean square
        of the increments by a constant. If the normalized increments
        are a fixed increment, the constant is unity. If the
        normalized increments have a Gaussian distribution, the
        constant is $\approx 0.8$ depending on the accuracy of of
        ``fit'' to a Gaussian distribution.}. Squaring this value is
        the average of the instantaneous fraction of change, and
        adding unity to the absolute value of the instantaneous
        fraction of change, and dividing by two, is the Shannon
        probability of the instantaneous fraction of change.

        Figure~\ref{\SETLABEL:IA1} is the instantaneous value of the
        root mean square of the normalized increments for the
        {\market}, and Figure~\ref{\SETLABEL:IA2} is the instantaneous
        Shannon probability for the normalized increments.

        \begin{figure}[ht]
            \begin{center}
                \begin{minipage}[t]{0.45\textwidth}
                    \epsfxsize=1.0\linewidth
                    \epsffile{\directory/data.tsinstant-r.eps}
                    \caption[{\market}, instantaneous value of
                        rms.]{{\market}, instantaneous value of the
                        root mean square of the normalized increments,
                        provided by running the program {\it
                        tsinstant}\/ with the -r option on the data
                        presented in Figure~\ref{\SETLABEL:TS}.}
                    \label{\SETLABEL:IA1}
                    \label{\SETLABELQ:IA1}
                \end{minipage}
                \hfill
                \begin{minipage}[t]{0.45\textwidth}
                    \epsfxsize=1.0\linewidth
                    \epsffile{\directory/data.tsinstant-s.eps}
                    \caption[{\market}, instantaneous value of
                        Shannon probability.]{{\market}, instantaneous
                        value of the Shannon probability of the
                        normalized increments, provided by running the
                        program {\it tsinstant}\/ with the -s option
                        on the data presented in
                        Figure~\ref{\SETLABEL:TS}.}
                    \label{\SETLABEL:IA2}
                    \label{\SETLABELQ:IA2}
                \end{minipage}
            \end{center}
        \end{figure}

% Local Variables:
% TeX-parse-self: t
% TeX-auto-save: t
% TeX-master: "fractal.tex"
% End:


        %
% -----------------------------------------------------------------------------
%
% A license is hereby granted to reproduce this software source code and
% to create executable versions from this source code for personal,
% non-commercial use.  The copyright notice included with the software
% must be maintained in all copies produced.
%
% THIS PROGRAM IS PROVIDED "AS IS". THE AUTHOR PROVIDES NO WARRANTIES
% WHATSOEVER, EXPRESSED OR IMPLIED, INCLUDING WARRANTIES OF
% MERCHANTABILITY, TITLE, OR FITNESS FOR ANY PARTICULAR PURPOSE.  THE
% AUTHOR DOES NOT WARRANT THAT USE OF THIS PROGRAM DOES NOT INFRINGE THE
% INTELLECTUAL PROPERTY RIGHTS OF ANY THIRD PARTY IN ANY COUNTRY.
%
% Copyright (c) 1994-2006, John Conover, All Rights Reserved.
%
% Comments and/or bug reports should be addressed to:
%
%     john@email.johncon.com (John Conover)
%
% -----------------------------------------------------------------------------
%
% Revision: \RCSRevision \\
% Revision Time: \RCSTime UMT \\
% Revision Date: \RCSDate \\
% Revision Id: \RCSId \\
% Revision File: \RCSLog \\
\RCS $Revision: 0.0 $
\RCS $Date: 2006/01/20 04:38:13 $
\RCS $Id: logistic.tex,v 0.0 2006/01/20 04:38:13 john Exp $
% $Log: logistic.tex,v $
% Revision 0.0  2006/01/20 04:38:13  john
% Initial version
%
%
    \subsection{Logistic Analysis}
        \label{\SETLABEL:LA}

        \subidx{\market}{Logistic function analysis}
        \subidx{time series}{logistic function}
        \subidx{logistic function}{time series}
        \subidx{time series}{increments}
        \subidx{time series}{analysis}
        \subidx{cumulative sum}{analysis}
        \subidx{analysis}{cumulative sum}
        \subidx{analysis}{random process}
        \subidx{random process}{analysis}
        The data in this section is presented in tabular form in
        Section~\ref{\SETLABELREF:LAA}.  Figure~\ref{\SETLABEL:LA1} is
        a graph of the logistic function estimates of the time series
        data for the {\market}. The reader is cautioned that these
        graphs are constructed using the method suggested in
        Chapter~\ref{general}, Section~\ref{nlextend} and enormous
        precision is required for adequate prediction of the logistic
        function,~\cite{Modis}. Particularly, the non-linear term will
        usually require intervention to produce a practical fit to the
        data. In addition, there are numerical stability issues with
        logistic function methodologies\footnote{For example, in
        Figures~\ref{\SETLABEL:LA1} and~\ref{\SETLABEL:LA2}, if the
        non-linear term, $b$, was greater than zero, it was set to
        zero to produce the graphs. See Section~\ref{\SETLABELREF:LAA}
        for the actual derived values. In other cases, the magnitude
        of $b$ was too large, resulting in a graph that was decreasing
        as a function of time}.  The methodology should be regarded as
        ``fragile.'' It is included for completeness.

        \idx{least squares approximation}
        Figure~\ref{\SETLABEL:LA1} is a graph of the logistic function
        for the time series data presented in
        Figure~\ref{\SETLABEL:TS}. The data presented was made by
        running the program {\it tsdlogistic}\/, which is described
        briefly in Appendix~\ref{programs}, on the parameters
        extracted from the time series data as suggested in
        Figure~\ref{\SETLABEL:TF}. The program {\it tslsq}\/ was used
        to derive the constant and the slope of the normalized
        increments of the data presented in Figure~\ref{\SETLABEL:TF}.
        Figure~\ref{\SETLABEL:LA2} is the same graph, but with the
        time scale expanded by a factor of two.

        \begin{figure}[ht]
            \begin{center}
                \begin{minipage}[t]{0.45\textwidth}
                    \epsfxsize=1.0\linewidth
                    \epsffile{\directory/data.tsfraction.tslsq-p.tsdlogistic.eps}
                    \caption[{\market}, logistic function
                        estimates.]{{\market}, logistic function
                        estimates, provided by running the {\it
                        tslsq}\/ program on the normalized increments
                        presented in Figure~\ref{\SETLABEL:TF} with
                        the -p option. These parameters were used as
                        arguments to the {\it tsdlogistic}\/ program.}
                    \label{\SETLABEL:LA1}
                    \label{\SETLABELQ:LA1}
                \end{minipage}
                \hfill
                \begin{minipage}[t]{0.45\textwidth}
                    \epsfxsize=1.0\linewidth
                    \epsffile{\directory/data.tsfraction.tslsq-p.tsdlogistic2.eps}
                    \caption[{\market}, logistic function
                        estimates.]{{\market}, logistic function
                        estimates of Figure~\ref{\SETLABEL:LA1} with
                        the time scale expanded by a factor of two.}
                    \label{\SETLABEL:LA2}
                    \label{\SETLABELQ:LA2}
                \end{minipage}
            \end{center}
        \end{figure}

% Local Variables:
% TeX-parse-self: t
% TeX-auto-save: t
% TeX-master: "fractal.tex"
% End:


        %
% -----------------------------------------------------------------------------
%
% A license is hereby granted to reproduce this software source code and
% to create executable versions from this source code for personal,
% non-commercial use.  The copyright notice included with the software
% must be maintained in all copies produced.
%
% THIS PROGRAM IS PROVIDED "AS IS". THE AUTHOR PROVIDES NO WARRANTIES
% WHATSOEVER, EXPRESSED OR IMPLIED, INCLUDING WARRANTIES OF
% MERCHANTABILITY, TITLE, OR FITNESS FOR ANY PARTICULAR PURPOSE.  THE
% AUTHOR DOES NOT WARRANT THAT USE OF THIS PROGRAM DOES NOT INFRINGE THE
% INTELLECTUAL PROPERTY RIGHTS OF ANY THIRD PARTY IN ANY COUNTRY.
%
% Copyright (c) 1994-2006, John Conover, All Rights Reserved.
%
% Comments and/or bug reports should be addressed to:
%
%     john@email.johncon.com (John Conover)
%
% -----------------------------------------------------------------------------
%
% Revision: \RCSRevision \\
% Revision Time: \RCSTime UMT \\
% Revision Date: \RCSDate \\
% Revision Id: \RCSId \\
% Revision File: \RCSLog \\
\RCS $Revision: 0.0 $
\RCS $Date: 2006/01/20 04:38:13 $
\RCS $Id: hurst.tex,v 0.0 2006/01/20 04:38:13 john Exp $
% $Log: hurst.tex,v $
% Revision 0.0  2006/01/20 04:38:13  john
% Initial version
%
%
    \subsection{Hurst Coefficient Analysis}
        \label{\SETLABEL:H}

        \subidx{\market}{Hurst coefficient analysis}
        \subidx{Hurst coefficient}{analysis}
        \subidx{increments}{normalized}
        \subidx{normalized}{increments}
        \subidx{programs}{tshurst}
        \subidx{tshurst}{program}
        The data in this section is presented in tabular form in
        Section~\ref{\SETLABELREF:HCHP}. Figure~\ref{\SETLABEL:HC} is
        a graph of the Hurst coefficient data time series data shown
        in Figure~\ref{\SETLABEL:TS}. The slope of the graph is the
        Hurst coefficient.  The data for this figure was produced by
        the program {\it tshurst}\/, which is described briefly in
        Appendix~\ref{programs}.

        \subidx{\market}{H parameter analysis}
        \subidx{H parameter}{analysis}
        \subidx{programs}{tshcalc}
        \subidx{tshcalc}{program}
        Figure~\ref{\SETLABEL:HP} is a graph of the H parameter data
        for the normalized increments of the time series data shown in
        Figure~\ref{\SETLABEL:TF}. The data for this figure was
        produced by the program {\it tshcalc}\/, which is described
        briefly in Appendix~\ref{programs}.

        \begin{figure}[ht]
            \begin{center}
                \begin{minipage}[t]{0.45\textwidth}
                    \epsfxsize=1.0\linewidth
                    \epsffile{\directory/data.tshurst.eps}
                    \caption[{\market}, Hurst coefficient data]{{\market},
                        Hurst coefficient data for the normalized
                        increments of the time series data shown in
                        Figure~\ref{\SETLABEL:TF}.  The slope of the graph
                        is the Hurst coefficient.}
                    \label{\SETLABEL:HC}
                \end{minipage}
                \hfill
                \begin{minipage}[t]{0.45\textwidth}
                    \epsfxsize=1.0\linewidth
                    \epsffile{\directory/data.tshcalc.eps}
                    \caption[{\market}, H parameter data]{{\market}, H
                        parameter data for the normalized increments of
                        the time series data shown in
                        Figure~\ref{\SETLABEL:TF} The slope of the graph
                        is the H parameter.}
                    \label{\SETLABEL:HP}
                \end{minipage}
            \end{center}
        \end{figure}

        \subidx{revenue}{See, rate of revenue returns}
        \subidx{returns}{See, rate of revenue returns}
        \subidx{\market}{revenues}
        \subidx{Hurst coefficient}{analysis}
        \subidx{\market}{Hurst coefficient analysis}
        \subidx{\market}{rate of change}
        \subidx{\market}{windows of opportunity}
        \subidx{rate of revenue returns}{forecast}
        \subidx{forecast}{rate of revenue returns}
        \idx{windows of opportunity}
        \subidx{programs}{tslsq}
        \subidx{tslsq}{program}

        The approximately linear slope of the graph in
        Figure~\ref{\SETLABEL:HC} implies that the variance of the
        rate of revenue returns, (per {\timescale},) in the {\market},
        $V(t_2 - t_1)$, over a period of time is proportional to the
        period of time raised to twice the Hurst
        coefficient~\cite[pp. 180]{Feder},~\cite[pp. 246]{Crownover}.
        This seems to be a quantitative statement concerning how fast,
        and to what degree, the rate of revenue returns' state of
        affairs can change over a period of time.  An additional
        implication, for Hurst coefficients sufficiently close to 0.5,
        is that the probability of the state of affairs repeating
        sometime in the future goes down with increasing
        time\footnote{It can be shown that the number of expected
        market ``high'' and ``low'' transitions, $N$, scales with the
        square root of time, or $N \propto \sqrt {t}$, meaning that
        the cumulative distribution of the probability, $P$, of the
        duration of a market's ``high'' or ``low'' exceeding a given
        time interval, $t$, is proportional to the reciprocal of the
        square root of the time interval, $P \propto 1 / \sqrt {t}$,
        (or, conversely, that the probability of the duration of a
        market's ``high'' or ``low'' exceeding a given time interval
        is proportional to the reciprocal of the time interval raised
        to the power $3 / 2$, ie., $P \propto 1 / t^{3 /
        2}$,~\cite[pp. 153]{Schroeder}. What this means is that a
        histogram of the ``zero free'' run-lengths of a market being
        ``high'' or ``low,'' over a long time, would have a $1 / t^{3
        / 2}$ characteristic.)}, $t$, $p(t) = erf (1/\sqrt{2t})$ which
        is approximately $1/\sqrt{t}$ for $t \gg
        1$~\cite[pp. 160]{Schroeder}. Figures~\ref{\SETLABEL:FN},
        and,~\ref{\SETLABEL:FF} compare methods of approximation of
        the ``forecastability'' of the rate of revenue returns in the
        {\market} for the near term and far term,
        respectively~\cite[pp. 83-84]{Peters:CAOITCM}\footnote{The
        author is not comfortable with Peters' interpretation. For
        example, if the algorithm explained
        in~\cite[pp. 82]{Peters:CAOITCM} is used on ``white noise''
        which, by definition, never has any correlations, the short
        term Hurst coefficient, and thus the ``forecastability,'' is
        still near unity---a bit of an enigma. This can be verified
        with the {\it tswhite}\/ and {\it tshurst}\/ programs, which
        are briefly described in Appendix~\ref{programs}.}.  This
        seems to be a quantitative statement concerning ``windows of
        opportunity'' in the rate of revenue returns, (per
        {\timescale}.)  The program {\it tslsq}\/ was used on the
        Hurst coefficient data, presented in
        Figure~\ref{\SETLABEL:HC}, to provide a least squares
        approximation to the Hurst coefficient. The superimposed least
        squares approximation with on original Hurst coefficient data
        is presented.  The time series data has a Hurst coefficient of
        {\thurstlow}, so that:

        \subidx{\market}{Hurst coefficient analysis}
        \begin{eqnarray}
            V\left(t_2 - t_1\right) & \propto & \left(t_2 - t_1\right)^{2 \cdot H}\\
            V\left(t_2 - t_1\right) & \propto & \left(t_2 - t_1\right)^{2 \cdot {\thurstlow}}\\
                                    & \propto & \left(t_2 - t_1\right)^{\thurstlowtwo}
            \label{\SETLABEL:V}
        \end{eqnarray}

        \subidx{fractional}{Brownian motion}
        \subidx{Brownian motion}{fractional}
        \idx{fractal}
        \noindent where $V(t_2 - t_1)$ is the variance of the
        increments of the rate of revenue returns, (per {\timescale},)
        over the time interval $t_2 -
        t_1$,~\cite[pp. 177]{Feder},~\cite[pp. 494]{Peitgen}. If $H >
        \frac{1}{2}$, then the time series is termed as being
        characterized by ``fractional Brownian
        motion~\cite[pp. 170]{Feder}.''

        \subidx{rate of revenue returns}{predictability}
        \subidx{rate of revenue returns}{forecastability}
        \subidx{rate of revenue returns}{consistency}
        \subidx{predictability}{rate of revenue returns}
        \subidx{forecastability}{rate of revenue returns}
        \subidx{consistency}{rate of revenue returns}
        \subidx{\market}{rate of revenue returns, predictability}
        \subidx{\market}{rate of revenue returns, forecastability}
        \subidx{\market}{rate of revenue returns, consistency}
        \subidx{Hurst coefficient}{analysis}
        \subidx{\market}{Hurst coefficient analysis}
        \subidx{\market}{rate of change}

        In some sense, the Hurst coefficient is a quantitative
        expression of the ``forecastability'' of the future based on
        the past\footnote{Actually, in general, when summing fractal
        entities, the method used should be a root mean square
        process, dependent on the Hurst Coefficient, $H$, where
        $P_{total}^H = P_1^H + P_2^H + \cdots$, where $P_n$ is the
        fractal entities. For a Brownian motion, or random walk type
        of fractal the Hurst Coefficient is a function of time into
        the future. For the ``near term,'' the Hurst coefficient is
        very near unity, meaning the summation process is linear. For
        the ``long term,'' $H \approx 0.5$, or a standard root mean
        square summation process should be used. If $H$ is $0.5$ then
        the market is termed a Brownian motion, or random walk
        process. If it is larger than 0.5, it is termed fractional
        Brownian motion process. For a random walk process, ``near
        term'' and ``far term'' are quantitatively differentiated on
        the Hurst Coefficient graph where $1 - \ln (t) = 0.5 \cdot \ln
        (t)$, or when $\ln (t) = 2$, or $t = 7.389\ldots$ See
        Section~\ref{\SETLABEL:FS} for the particulars on using Hurst
        Coefficient to sum fractal process' for the {\market}. See
        also~\cite[pp. 67, 83-84]{Peters:CAOITCM} and~\cite[pp. 129,
        159]{Schroeder} for particulars on the implications of the
        Hurst Coefficient and root mean square summation issues.}.  A
        Hurst coefficient of {\thurstlow}, (for the near future, and
        {\thurstall} for the distant future.) implies that the
        likelihood of the rate of revenue returns, (per {\timescale},)
        for any two consecutive {\timescale}s being the same is
        {\thurstlowhundred}\%~\cite[pp. 66]{Peters:CAOITCM} for the
        near future, and {\thurstall} for the distant
        future. Likewise, there is a {\thurstlowhundred}\% chance of
        the rate of revenue returns, (per {\timescale},) movements
        being the same in consecutive time periods---ie., if, in a
        given {\timescale}, the rate of revenue returns, (per
        {\timescale},) is increasing, there is a {\thurstlowhundred}\%
        that the rate of revenue returns, (per {\timescale},) will
        increase in the following period, also. In some sense, this is
        a quantitative statement on how ``predictable,'' or
        ``forecastable'' the rate of revenue returns, (per
        {\timescale},) for the {\market} are over time, since the
        probability of having $n$ many consecutive {\timescale}s of
        the same agenda is $H^n$ where $H$ is the Hurst coefficient,
        or, letting the short term probability of having $n$ many
        {\timescale}s of the same market agenda, $p_a$, is:

        \begin{eqnarray}
            p_a\left(n\right) & = & H^{n}\\
                              & = & {\thurstlow}^{n}
            \label{\SETLABEL:MA}
        \end{eqnarray}

        \subidx{rate of revenue returns}{predictability}
        \subidx{rate of revenue returns}{forecastability}
        \subidx{rate of revenue returns}{consistency}
        \subidx{predictability}{rate of revenue returns}
        \subidx{forecastability}{rate of revenue returns}
        \subidx{consistency}{rate of revenue returns}
        As an interesting interpretation of the normalized increments
        of the time series data presented in
        Figure~\ref{\SETLABEL:TF}, if the vertical axis is multiplied
        by 100, to convert to percent, then the graph represents the
        error, in percent, that would be made by forecasting, month by
        month, that the next {\timescale}'s rate of revenue returns
        would be the same as the current {\timescale}'s revenue
        rate. Interestingly, it is $\datafractionmean \cdot 100$
        percent, on the average, with a standard deviation of
        $\datafractionstddev \cdot 100$ percent, and a root mean
        square error value of $\datafractionrms \cdot 100$
        percent---small values for such a simple forecasting
        mechanism.

        \subidx{\market}{rate of revenue returns, range}
        \subidx{Hurst coefficient}{analysis}
        \subidx{\market}{Hurst coefficient analysis}
        \subidx{\market}{rate of change}

        This is, essentially, a statement of the range of values, in
        the increments of the rate of revenue returns, (per
        {\timescale},) that is to be expected over the time interval,
        $t_2 - t_1$,
        $R_v$,~\cite[pp. 178]{Feder},~\cite[pp. 172]{Cambel}:

        \begin{eqnarray}
            R_v\left(t_2 - t_1\right) & \propto & \left(t_2 - t_1\right)^{H}\\
                                      & \propto & \left(t_2 - t_1\right)^{\thurstlow}
            \label{\SETLABEL:R}
        \end{eqnarray}

        \subidx{\market}{rate of revenue returns, range}
        \subidx{Hurst coefficient}{analysis}
        \subidx{\market}{Hurst coefficient analysis}
        \subidx{\market}{rate of change}
        \subidx{Markov}{statistics}
        \subidx{statistics}{Markov}
        \noindent where $R$ is the range of values in the increments
        of the rate of revenue returns, (per {\timescale}.) A Hurst
        coefficient, $H$, that is much larger than $\frac{1}{2}$, (but
        less than 1,) implies a strongly non-Gaussian distribution in
        the increments of the rate of revenue returns, (per
        {\timescale},)~\cite[pp. 152, 194]{Feder}, and a Hurst
        coefficient near $\frac{1}{2}$ implies that the increments of
        the rate of revenue returns, (per {\timescale}) is
        characteristic of an independent
        process~\cite[pp. 195]{Feder}. Extreme caution should be
        exercised in using Markov statistics in any analysis where the
        Hurst coefficient is not
        $\frac{1}{2}$,~\cite[pp. 124]{Crownover},~\cite[pp. 106]{Peters:CAOITCM}.


        As a useful approximation, if $H$, is approximately
        $\frac{1}{2}$, Equation~\ref{\SETLABEL:R} reduces
        to,~\cite[pp. 129]{Schroeder}:

        \begin{eqnarray}
            R\left(t_2 - t_1\right) & \propto & (t_2 - t_1)^{\frac{1}{2}}\\
                                    & \propto & \sqrt{\left(t_2 - t_1\right)}
        \end{eqnarray}

        \subidx{\market}{rate of revenue returns, range}
        \subidx{\market}{rate of revenue returns, increase and decrease}
        \subidx{Hurst coefficient}{analysis}
        \subidx{\market}{Hurst coefficient analysis}
        \subidx{\market}{rate of change}
        \subidx{Markov}{statistics}
        \subidx{statistics}{Markov}

        In the case where the Hurst coefficient, $H$, is
        $\frac{1}{2}$, the range of values in the increments of the
        rate of revenue returns, (per {\timescale},) divided by the
        standard deviation of these values, $S$, can be anticipated to
        increase over time according to the following
        relation,~\cite[pp. 154]{Feder},~\cite[pp. 129]{Schroeder}:

        \begin{equation}
            \frac{R\left(t_2 - t_1\right)}{S} \propto \left(t_2 - t_1\right)^{\frac{1}{2}}
        \end{equation}

        \subidx{\market}{rate of revenue returns, range}
        \subidx{\market}{rate of revenue returns, increase and decrease}
        \subidx{Hurst coefficient}{analysis}
        \subidx{\market}{Hurst coefficient analysis}
        \subidx{\market}{rate of change}
        \noindent which is a useful conceptual approximation, since it
        involves only the square root function---if the range and the
        standard deviation of the increments of the rate of revenue
        returns, (per {\timescale},) are known, (and $H \approx
        \frac{1}{2}$,) then the expected change in $\frac{R}{S}$, will
        increase with the square root of time\footnote{To be precise,
        it is actually asymptotically proportional to
        $\tau^{\frac{1}{2}}$}.

        Another useful approximation when rescaling processes that are
        characterize by Brownian motion, (ie., when $H \approx
        \frac{1}{2}$,) is that:

        \begin{eqnarray}
            X\left(t\right) & \propto & \frac{X\left(rt\right)}{r^{H}}\\
                            & \propto & \frac{X\left(rt\right)}{r^{\thurstlow}}
        \end{eqnarray}

        \idx{Brownian motion}
        \idx{fractal}
        Where $X(t)$ is the process characterized by Brownian motion,
        and $r$ is a scaling factor,~\cite[pp. 494]{Peitgen}.

        \subidx{programs}{tslsq}
        \subidx{tslsq}{program}
        The program {\it tslsq}\/ was used on the H parameter data,
        presented in Figure~\ref{\SETLABEL:HP}, to provide a least
        squares approximation to the H parameter for the
        {\market}. The superimposed least squares approximation on the
        original H parameter data is presented.  By contrast, the H
        parameter, as derived by the methodology outlined
        in~\cite[pp. 249]{Crownover}, is {\thcalclow} for the near
        future, and {\thcalcall} for the distant future.

        \subidx{\market}{Hurst coefficient analysis}
        \subidx{Hurst coefficient}{analysis}
        \subidx{increments}{normalized}
        \subidx{normalized}{increments}
        \subidx{programs}{tshurst}
        \subidx{tshurst}{program}
        \subidx{\market}{H parameter analysis}
        \subidx{H parameter}{analysis}
        \subidx{programs}{tshcalc}
        \subidx{tshcalc}{program}
        Figures~\ref{\SETLABEL:HC} and~\ref{\SETLABEL:HP} represent
        Hurst coefficient and H parameter data that are derived from
        the normalized increments, shown in
        Figure~\ref{\SETLABEL:TF}. In this case, the data is
        considered a normalized derivative of the time series data
        presented in Figure~\ref{\SETLABEL:TF}, instead of a
        cumulative sum.  The program, {\it tshurst}\/, is described
        briefly in appendix~\ref{programs}, and the data for
        figures~\ref{\SETLABEL:THC} and~\ref{\SETLABEL:THP} was made
        using the -d option.

        \begin{figure}[ht]
            \begin{center}
                \begin{minipage}[t]{0.45\textwidth}
                    \epsfxsize=1.0\linewidth
                    \epsffile{\directory/data.tsfraction.tshurst-d.eps}
                    \caption[{\market}, traditional Hurst coefficient
                        data]{{\market}, traditional Hurst coefficient
                        data for the time series data shown in
                        Figure~\ref{\SETLABEL:TS}.  The slope of the
                        graph is the Hurst coefficient, and is
                        {\hurstlow} for the near term, and
                        {\hurstall} for the far term.}
                    \label{\SETLABEL:THC}
                \end{minipage}
                \hfill
                \begin{minipage}[t]{0.45\textwidth}
                    \epsfxsize=1.0\linewidth
                    \epsffile{\directory/data.tsfraction.tshcalc-d.eps}
                    \caption[{\market}, traditional H parameter
                        data]{{\market}, traditional H parameter data
                        for the time series data shown in
                        Figure~\ref{\SETLABEL:TS} The slope of the
                        graph is the H parameter, and is {\hcalclow}
                        for the near term, and {\hcalcall} for the
                        far term.}
                    \label{\SETLABEL:THP}
                \end{minipage}
            \end{center}
        \end{figure}

% Local Variables:
% TeX-parse-self: t
% TeX-auto-save: t
% TeX-master: "fractal.tex"
% End:


        %
% -----------------------------------------------------------------------------
%
% A license is hereby granted to reproduce this software source code and
% to create executable versions from this source code for personal,
% non-commercial use.  The copyright notice included with the software
% must be maintained in all copies produced.
%
% THIS PROGRAM IS PROVIDED "AS IS". THE AUTHOR PROVIDES NO WARRANTIES
% WHATSOEVER, EXPRESSED OR IMPLIED, INCLUDING WARRANTIES OF
% MERCHANTABILITY, TITLE, OR FITNESS FOR ANY PARTICULAR PURPOSE.  THE
% AUTHOR DOES NOT WARRANT THAT USE OF THIS PROGRAM DOES NOT INFRINGE THE
% INTELLECTUAL PROPERTY RIGHTS OF ANY THIRD PARTY IN ANY COUNTRY.
%
% Copyright (c) 1994-2006, John Conover, All Rights Reserved.
%
% Comments and/or bug reports should be addressed to:
%
%     john@email.johncon.com (John Conover)
%
% -----------------------------------------------------------------------------
%
% Revision: \RCSRevision \\
% Revision Time: \RCSTime UMT \\
% Revision Date: \RCSDate \\
% Revision Id: \RCSId \\
% Revision File: \RCSLog \\
\RCS $Revision: 0.0 $
\RCS $Date: 2006/01/20 04:38:13 $
\RCS $Id: fiscal.tex,v 0.0 2006/01/20 04:38:13 john Exp $
% $Log: fiscal.tex,v $
% Revision 0.0  2006/01/20 04:38:13  john
% Initial version
%
%
    \subsection{Fixed Increment Approximation for Fiscal Strategy}
        \label{\SETLABEL:FS}

        \subidx{\market}{fiscal strategy}
        \subidx{markets}{analysis}
        \subidx{analysis}{markets}
        \subidx{strategy}{fiscal}
        \subidx{fiscal}{strategy}
        The data in this section is presented in tabular form in
        Section~\ref{\SETLABELREF:LR}. This section derives various
        values based on the ``average'' of the normalized increments
        presented in Figure~\ref{\SETLABEL:TFA}. These values are an
        approximation to a, probably, complex process with a
        distribution shown in Figure~\ref{\SETLABEL:TF}. These values
        will be used in a fixed increment Brownian fractal analysis
        and simulation of the {\market}, and may, or may not, provide
        adequate accuracy for projections.

        For an organization operating in the {\market}, the fiscal
        strategy, commensurate with the aggregate environment, can be
        derived as follows~\cite[pp. 128, pp
        151]{Schroeder},~\cite[pp. 450]{Reza},~\cite[pp. 270]{Pierce}:
        \vspace{0.15in}

        \subsubsection{Logarithmic Returns}
            \label{\SETLABEL:LR}

            \subidx{logarithmic}{returns}
            \subidx{returns}{logarithmic}
            \subidx{\market}{logarithmic returns}
            The logarithmic returns can be calculated by various
            means. Four will be presented here, for comparison.

            \subidx{programs}{tsnormal}
            \subidx{tsnormal}{program}
            \subidx{logarithmic}{returns}
            \subidx{returns}{logarithmic}
            The logarithmic returns, in bits, $bits$, as computed from
            the mean, by the program {\it tsnormal}\/, which is
            described in Chapter~\ref{programs}, and is presented in
            Figure~\ref{\SETLABEL:TF}, and Equation~\ref{abits} from
            Section~\ref{ereturns} in Chapter~\ref{general}:

            \begin{equation}
                bits = \frac{\ln \left({\datafractionmean} + 1\right)}{\ln \left(2\right)} = \datafractionmeanbits
            \end{equation}

            \subidx{programs}{tslsq}
            \subidx{tslsq}{program}
            \subidx{logarithmic}{returns}
            \subidx{returns}{logarithmic}
            \noindent By comparison, the logarithmic returns, in bits,
            $bits$, as computed from the constant in the least squares
            approximation, using the program {\it tslsq}\/, which is briefly
            described in Chapter~\ref{programs}, as presented in
            Figure~\ref{\SETLABEL:TF}, and Equation~\ref{abits} from
            Section~\ref{ereturns} in Chapter~\ref{general}:

            \begin{equation}
                bits = \frac{\ln \left({\datafractionconstant} + 1\right)}{\ln \left(2\right)} = \datafractionconstantbits
            \end{equation}

            Note that if the mean is not constant in
            Figure~\ref{\SETLABEL:TF}, this method will not provide
            accurate results.

            \subidx{programs}{tslsq}
            \subidx{tslsq}{program}
            \subidx{logarithmic}{returns}
            \subidx{returns}{logarithmic}
            \noindent And by yet another comparison, using the program
            {\it tslsq}\/, which is briefly described in
            Chapter~\ref{programs}, with the -e -p options, to provide
            a formula for the least squares exponential fit to the
            time series data set presented in
            Figure~\ref{\SETLABEL:TS}:

            \begin{equation}
                bits = {\datatslsqepbits}
            \end{equation}

            \subidx{programs}{tslogreturns}
            \subidx{tslogreturns}{program}
            \subidx{logarithmic}{returns}
            \subidx{returns}{logarithmic}
            \noindent And finally, by comparison, from the
            {\it tslogreturns}\/ program, which is briefly described
            in Chapter~\ref{programs}, with the -p option, to provide
            a formula for the logarithmic returns of the time series
            data set presented in Figure~\ref{\SETLABEL:TS}:

            \begin{equation}
                bits = {\logreturns}
            \end{equation}

        \subsubsection{Calculation of Shannon Probability}
            \label{\SETLABEL:SP}

            \subidx{\market}{Shannon probability}
            Ideally, all of the values presented in
            Section~\ref{\SETLABEL:LR} would be equal. Using the
            logarithmic returns provided by the {\it tslogreturns}\/
            program, to be consistent
            with~\cite[pp. 81]{Peters:CAOITCM}

            \subidx{programs}{tslogreturns}
            \subidx{tslogreturns}{program}
            \begin{equation}
                2^{{\logreturns}t}
            \end{equation}

            \noindent therefore:
            \begin{equation}
                C\left(p\right) = {\logreturns}
            \end{equation}
            \subidx{programs}{tsshannon}
            \subidx{tsshannon}{program}
            \subidx{Shannon}{probability}
            \subidx{probability}{Shannon}
            \noindent and, {\it tsshannon}\/ {\logreturns} gives:
            \begin{equation}
                \label{\SETLABEL:F0}
                C\left({\shannonlogreturns}\right) = {\logreturns}
            \end{equation}
            \noindent therefore:
            \begin{eqnarray}
                2^{C\left({\shannonlogreturns}\right)} & = & 2^{\logreturns}\\
                                                       & = & {\twologreturns}\\
                                                       & = & {\twologreturnshundred}\%
            \end{eqnarray}
            \noindent and:
            \begin{eqnarray}
                2p - 1 & = & \left(2 \cdot {\shannonlogreturns}\right) - 1\\
                       & = & {\twopone}\\
                       \label{\SETLABEL:F1}
                       & = & {\twoponehundred}\%
            \end{eqnarray}

            \subidx{\market}{fiscal strategy}
            \subidx{markets}{analysis}
            \subidx{analysis}{markets}
            \subidx{strategy}{fiscal}
            \subidx{fiscal}{strategy}
            \subidx{\market}{fiscal strategy}
            \subidx{\market}{growth rate}
            Presuming the simplified assumptions outlined in
            Section~\ref{assumptions}, the ``typical'' organization
            operating in the {\market} executes a long term fiscal
            strategy, commensurate with the aggregate environment,
            that is to invest, every {\timescale}, in sufficient
            additional resources and infrastructure, to increase the
            manufacturing of goods and services by {\twoponehundred}\%
            of its rate of revenue returns, (per {\timescale}.) As a
            conceptual model, the remaining {\hundredtwoponehundred}\%
            will be held in ``reserve'' with a
            {\shannonlogreturnshundred}\% chance of making twice the
            {\twoponehundred}\% back, (and a
            {\hundredshannonlogreturnshundred}\% chance of making
            0.0,) in one {\timescale}, on the average, for an average
            growth in its rate of revenue returns, (per {\timescale},)
            of {\twologreturnshundred}\%, or a doubling of its rate of
            revenue returns, (per {\timescale},) in
            {\oneoverlogreturns} {\timescale}s.

        \subsubsection{Example Fixed Increment Approximation Fiscal Strategies}

            \subidx{\market}{fiscal strategy}
            \subidx{markets}{analysis}
            \subidx{analysis}{markets}
            \subidx{strategy}{fiscal}
            \subidx{fiscal}{strategy}
            \subidx{\market}{fiscal strategy}
            \subidx{\market}{growth rate}
            \subidx{\market}{management metric}
            \idx{management metric}
            A possible metric on the effectiveness of long term fiscal
            management could possibly be that if an investment of
            {\twoponehundred}\% per {\timescale} of the rate of
            revenue returns, (per {\timescale},) is made in resources
            and infrastructure, then the rate of revenue returns would
            be expected to increase by {\twologreturnshundred}\%, per
            {\timescale}, on average.

            Note that the metrics presented in this section are
            representative of the {\market} as an aggregate whole, and
            may or may not be accurate representations for any
            particular participant in the environment. Of interest to
            the participants in the environment would be a similar
            analysis of each product or service rendered in the
            marketplace.

            \subidx{\market}{fiscal strategy}
            \subidx{markets}{analysis}
            \subidx{analysis}{markets}
            \subidx{strategy}{fiscal}
            \subidx{fiscal}{strategy}
            \subidx{\market}{fiscal strategy}
            As a simple illustrative example, a company operating in
            this environment might obtain a credit line from a bank
            that is equal to {\twoponehundred}\% of its rate of
            revenue returns, (per {\timescale},) to finance additional
            operations. In this simple scenario, the company would use
            its revenue base as collateral for the loan. Some
            {\timescale}s, depending on the {\market}'s environment,
            the company's rate of revenue returns exceeds what was
            borrowed from the bank, and the loan is repaid in
            full. Other {\timescale}s, the company must default, and
            the bank seizes a portion of the company's revenue base to
            pay the delinquent loan. However, on the average, the
            company will expand its rate of revenue returns at
            {\twologreturnshundred}\% per {\timescale}.

            \subidx{\market}{fiscal strategy}
            \subidx{markets}{analysis}
            \subidx{analysis}{markets}
            \subidx{strategy}{fiscal}
            \subidx{fiscal}{strategy}
            \subidx{\market}{fiscal strategy}
            As another simple example, a company re-invests
            {\twoponehundred}\% of its rate of revenue returns, (per
            {\timescale},) in development, marketing, sales, and
            distribution of new products.  Although some products will
            be successful and the return on the investment will exceed
            the {\twoponehundred}\% per {\timescale} investment,
            others will not. However, on the average, the company will
            expand it gross rate of revenue returns at
            {\twologreturnshundred}\% per {\timescale}.

            \subidx{\market}{fiscal strategy}
            \subidx{markets}{analysis}
            \subidx{analysis}{markets}
            \subidx{strategy}{fiscal}
            \subidx{fiscal}{strategy}
            \subidx{\market}{fiscal strategy}
            \subidx{\market}{product portfolio}
            \subidx{\market}{product diversity}
            \subidx{\market}{product mix}
            \subidx{\market}{optimum number of products}
            \idx{product portfolio}
            \idx{product diversity}
            \idx{optimum number of products}
            \idx{product mix}

            As an example of ``product portfolio'' management, suppose
            a company re-invests {\twoponehundred}\% of its rate of
            revenue returns, (per {\timescale},) in development,
            marketing, sales, and distribution of new products.
            Further suppose that the company has two products, and a
            fractal analysis of the individual product rate of revenue
            return time series indicates that one product has a
            Shannon probability of 0.65, and the other has a Shannon
            probability of 0.55. Then the percentage of re-investment
            in the first product would be $(2 \cdot 0.65 - 1) \cdot
            {\twoponehundred}$, percent of the rate of revenue
            returns, and $(2 \cdot 0.55 - 1) \cdot {\twoponehundred}$
            percent for the second product, implying that the company
            should diversify its product line\footnote{The astute
            reader would note that the linear addition was used to add
            the contribution to development of each product. This is a
            ``near term'' interpretation. Actually, in general, the
            method used should be a root mean square process,
            dependent on the Hurst Coefficient, $H$, where
            $P_{total}^H = P_1^H + P_2^H + \cdots$, where $P_n$ is the
            contribution to each individual product. For a Brownian
            motion, or random walk type of fractal the Hurst
            Coefficient is a function of time into the future. For the
            ``near term,'' the Hurst coefficient is very near unity,
            meaning the summation process is linear. For the ``long
            term,'' $H \approx 0.5$, or a standard root mean square
            summation process should be used. If $H$ is $0.5$ then the
            market is termed a Brownian motion, or random walk
            process. If it is larger than 0.5, it is termed fractional
            Brownian motion process. For a random walk process, ``near
            term'' and ``far term'' are quantitatively differentiated
            on the Hurst Coefficient graph where $1 - \ln (t) = 0.5
            \cdot \ln (t)$, or when $\ln (t) = 2$, or $t =
            7.389\ldots$ See~\cite[pp. 67, 83-84]{Peters:CAOITCM}
            and~\cite[pp. 129, 159]{Schroeder} for particulars on the
            implications of the Hurst Coefficient and root mean square
            summation issues.}.  Note that this is a ``bet hedging''
            metric methodology, and assumes that the products have
            uncorrelated revenue return rates. If this re-investment
            methodology is not feasible, perhaps for strategic
            financial reasons, then the re-investment in both products
            should total the ${\twoponehundred}$\%, and the investment
            in each product should be made at a ratio of $\frac{(2
            \cdot 0.65 - 1)}{(2 \cdot 0.55 - 1)} = 3 : 1$,
            respectively. Note that this ``bet hedging'' can be used
            to define the optimal number of products that can be
            supported on the rate of revenue returns. If it assumed
            that all products are ``typical'' for the {\market}, as a
            standard bench mark, then the optimal number will be
            $\frac{1}{{\twopone}}$. Note that this is a
            ``theoretical'' value, since not all products are
            ``typical,'' and there may be strategic reasons, for
            example product leveraging, that may increase the number
            of products above the optimum. However, most of the
            revenue should come from the optimal number of products,
            since having more products will decrease the amount of the
            potential investment in each product, and having less than
            the optimum number of products will increase the risk that
            many of the products could suffer a ``down market''
            concurrently, impacting the rate of revenue returns.  As
            another interesting interpretation of the optimal
            ``hedging of bets,'' in product portfolio strategy, and
            considering the graph of the normalized increments
            presented in Figure~\ref{\SETLABEL:TF}, if the
            organization is running optimally, then these products
            will generate, at least in principle, one standard
            deviation, approximately $0.8413 = 84.13$\% of the future
            growth in rate of revenue returns. Naturally, these are
            approximations, and the values are an approximation to a,
            probably, complex process, and appropriate scrutiny should
            be exercised before making specific projections.  As yet
            another example of ``product portfolio'' management,
            consider the issue of product mix. In this interpretation,
            {\twoponehundred}\% of the product manufactured should be
            ``proprietary,'' while the rest is ``industry standard.''
            As yet another possibility, {\twoponehundred}\% of the
            product manufactured should be predatory into new markets,
            and the remainder in markets that are ``traditional'' for
            the company.

% Local Variables:
% TeX-parse-self: t
% TeX-auto-save: t
% TeX-master: "fractal.tex"
% End:


        %
% -----------------------------------------------------------------------------
%
% A license is hereby granted to reproduce this software source code and
% to create executable versions from this source code for personal,
% non-commercial use.  The copyright notice included with the software
% must be maintained in all copies produced.
%
% THIS PROGRAM IS PROVIDED "AS IS". THE AUTHOR PROVIDES NO WARRANTIES
% WHATSOEVER, EXPRESSED OR IMPLIED, INCLUDING WARRANTIES OF
% MERCHANTABILITY, TITLE, OR FITNESS FOR ANY PARTICULAR PURPOSE.  THE
% AUTHOR DOES NOT WARRANT THAT USE OF THIS PROGRAM DOES NOT INFRINGE THE
% INTELLECTUAL PROPERTY RIGHTS OF ANY THIRD PARTY IN ANY COUNTRY.
%
% Copyright (c) 1994-2006, John Conover, All Rights Reserved.
%
% Comments and/or bug reports should be addressed to:
%
%     john@email.johncon.com (John Conover)
%
% -----------------------------------------------------------------------------
%
% Revision: \RCSRevision \\
% Revision Time: \RCSTime UMT \\
% Revision Date: \RCSDate \\
% Revision Id: \RCSId \\
% Revision File: \RCSLog \\
\RCS $Revision: 0.0 $
\RCS $Date: 2006/01/20 04:38:13 $
\RCS $Id: companies.tex,v 0.0 2006/01/20 04:38:13 john Exp $
% $Log: companies.tex,v $
% Revision 0.0  2006/01/20 04:38:13  john
% Initial version
%
%
    \subsection{Number of Companies}
        \label{\SETLABEL:QNC}

        \subidx{\market}{number of companies}
        \subidx{number of companies}{analysis}
        \subidx{analysis}{number of companies}
        \subidx{Shannon}{probability}
        \subidx{probability}{Shannon}
        This section evaluates the approximate, or ``average,'' number
        of companies in the {\market}, and uses the method outlined in
        Chapter~\ref{general}, Section~\ref{aftsma}. Since the
        average, $avg_{ind}$, and the root mean square, $rms_{ind}$,
        of the normalized increments of the {\market} time series is
        \datafractionmean, and \datafractionrms respectively, the
        number of companies participating in the market can be
        calculated by Equation~\ref{ncompanies} to be {\ncompanies}.

        If this value seems consistent number of companies in the
        {\market}, within the assumptions outlined in
        Chapter~\ref{general}, Section~\ref{aftsma}, then it would
        seem that there is some circumstantial or indirect evidence
        that the companies participating in the {\market} are
        operating optimally, and the ``average'' Shannon probability,
        $P$ for each participating company would be, using
        Equation~\ref{pncompanies}, {\pncompanies}, which would be the
        value which should be used in Section~\ref{\SETLABEL:FS} for
        each participating company if market expansion was to be
        consistent with the rest of the industry. However, if the
        Shannon probability derived in Section~\ref{\SETLABEL:FS} is
        greater than the average Shannon probability for the companies
        participating in the {\market}, as derived in this section,
        then the market would, possibly, be exploitable with the
        fiscal strategy outlined in Section~\ref{\SETLABEL:FS}. The
        maximum exploitability for the {\market} is derived in
        Section~\ref{\SETLABEL:MAXSHANNON}, but it is probably of
        doubtful practicality.

        Note that these optimizations would maximize a company's
        market growth. Since there are probably many companies
        competing in the market place, this would not necessarily
        maximize a company's P\&L, as described in
        Chapter~\ref{general}, Section~\ref{ompl}. The Shannon
        probability that maximizes market share in the {\market} is
        \pncompanies, with several alternative solutions listed in the
        previous paragraph. However, these should be contrasted to the
        Shannon probability that maximizes a company's P\&L which is
        \avgrms~in the {\market}. In all cases, the fraction of the
        P\&L that should be ``wagered'' on the future, $f$, should be:

        \begin{equation}
            f = 2P - 1
        \end{equation}

        \noindent where $P$ is the particular Shannon probability
        chosen optimize a particular fiscal strategy. Interestingly,
        the measured Shannon probability of the {\market} would tend
        to indicate that the companies participating in the market
        have chosen a fiscal strategy that optimizes market growth, as
        opposed to capital growth.

        \subidx{\market}{increasing returns}
        \subidx{economic increasing returns}{\market}
        As interesting interpretation of these exploitive issues,
        since all three fiscal strategies will result in exponential
        market growth for every company participating in the market,
        is that they may represent, perhaps, an example of
        ``increasing returns.''

% Local Variables:
% TeX-parse-self: t
% TeX-auto-save: t
% TeX-master: "fractal.tex"
% End:


        %
% -----------------------------------------------------------------------------
%
% A license is hereby granted to reproduce this software source code and
% to create executable versions from this source code for personal,
% non-commercial use.  The copyright notice included with the software
% must be maintained in all copies produced.
%
% THIS PROGRAM IS PROVIDED "AS IS". THE AUTHOR PROVIDES NO WARRANTIES
% WHATSOEVER, EXPRESSED OR IMPLIED, INCLUDING WARRANTIES OF
% MERCHANTABILITY, TITLE, OR FITNESS FOR ANY PARTICULAR PURPOSE.  THE
% AUTHOR DOES NOT WARRANT THAT USE OF THIS PROGRAM DOES NOT INFRINGE THE
% INTELLECTUAL PROPERTY RIGHTS OF ANY THIRD PARTY IN ANY COUNTRY.
%
% Copyright (c) 1994-2006, John Conover, All Rights Reserved.
%
% Comments and/or bug reports should be addressed to:
%
%     john@email.johncon.com (John Conover)
%
% -----------------------------------------------------------------------------
%
% Revision: \RCSRevision \\
% Revision Time: \RCSTime UMT \\
% Revision Date: \RCSDate \\
% Revision Id: \RCSId \\
% Revision File: \RCSLog \\
\RCS $Revision: 0.0 $
\RCS $Date: 2006/01/20 04:38:13 $
\RCS $Id: operations.tex,v 0.0 2006/01/20 04:38:13 john Exp $
% $Log: operations.tex,v $
% Revision 0.0  2006/01/20 04:38:13  john
% Initial version
%
%
    \subsection{Fixed Increment Approximation for Operational Strategy}
        \label{\SETLABEL:OPS}.

        This section derives various values based on the ``average''
        of the normalized increments presented in
        Figure~\ref{\SETLABEL:TFA}. These values are an approximation
        to a, probably, complex process with a distribution shown in
        Figure~\ref{\SETLABEL:TF}. These values will be used in a
        fixed increment Brownian fractal analysis and simulation of
        the {\market}, and may, or may not, provide adequate accuracy
        for projections.

        \subidx{\market}{fiscal strategy}
        \subidx{\market}{Shannon probability}
        \subidx{strategy}{fiscal}
        \subidx{fiscal}{strategy}
        \subidx{Shannon}{probability}
        \subidx{probability}{Shannon}
        It should be noted that the analysis of fiscal strategy,
        presented in Section~\ref{\SETLABEL:FS}, is derived from the
        {\market} metrics and may, or may not, be maximally
        optimal. For the optimal fiscal strategy, which may be
        exploitable, see Section~\ref{\SETLABEL:MAXSHANNON}.

        \subidx{strategy}{exploitable}
        \subidx{exploitable}{strategy}
        \subidx{\market}{windows of opportunity}
        \idx{windows of opportunity}
        \subidx{decision}{obsolete}
        \subidx{obsolete}{decision}
        \subidx{decision}{timeliness}
        \subidx{timeliness}{decision}
        \subidx{rate of revenue returns}{forecast}
        \subidx{forecast}{rate of revenue returns}
        An additional exploitable strategy may be time itself.
        Equations~\ref{\SETLABEL:V},~\ref{\SETLABEL:R},
        and,~\ref{\SETLABEL:MA}, are, essentially, metrics on how fast
        a decision, which is based on information concerning the
        current status of the {\market}, becomes obsolete. Obviously,
        how long a decision is expected to remain relevant should be
        addressed as an operational necessity in strategic planning
        and project management. Figures~\ref{\SETLABEL:FN},
        and,~\ref{\SETLABEL:FF} compare methods of approximation of
        the ``forecastability'' of rate of revenue returns in the
        {\market} for the near term and far
        term~\cite[pp. 83-84]{Peters:CAOITCM}, respectively. As a
        general rule, caution must be exercised when making decisions
        that will span a time interval larger than the time interval
        where the ``forecastability'' of rate of revenue returns drops
        below 50\%. Beyond this time interval, the chances increase
        that the competitive and market forces will alter the market
        environment in a possibly detrimental unanticipated
        fashion. Obviously, there is significant advantage in
        ``timeliness'' of development, manufacturing, and distribution
        of products and services that are consistent with this
        temporal agenda. Automation of these processes, if executed
        consistently with this agenda, should be considered a
        competitive advantage.

        \subidx{strategy}{exploitable}
        \subidx{exploitable}{strategy}
        \subidx{rate of revenue returns}{forecast}
        \subidx{forecast}{rate of revenue returns}
        \idx{product life cycle}
        \idx{life cycle, product}
        In some sense, this temporal agenda defines the ``average''
        product or service life cycle in the {\market}. When the
        ``forecastability'' of rate of revenue returns drops below
        50\%, there is an even chance that the rate of revenue returns
        for the product or service will change in a detrimental
        fashion. If it is assumed that a product or service life cycle
        consists of a ramp up, a maintenence interval, and a ramp
        down, then, if all three life cycle intervals are equal, the
        product life cycle will be, approximately, three times the
        time interval where the ``forecastability'' of rate of revenue
        returns drops below 50\%. Although probably not an accurate
        prediction of product or service life cycle, the technique may
        be used as a conceptual approximation to the dynamics of
        ``market windows.\footnote{For example, consider the market
        for table salt. Since it has inelastic supply and demand
        curves, and is a necessary requirement for life, it would be
        expected that the Hurst coefficient would be very near
        unity---ignoring competitive pressures in the market. The
        predictability of the table salt market would, therefore, be
        expected to be relatively good, over time.}''  The conceptual
        approximation will probably predict a ``conservative'' or
        ``pessimistic'' value in relation to actual markets.

        \begin{figure}[ht]
            \begin{center}
                \begin{minipage}[t]{0.45\textwidth}
                    \epsfxsize=1.0\linewidth
                    \epsffile{\directory/datahurstlownear.eps}
                    \caption[{\market}, ``forecastability'' of near
                        term rate of revenue returns]{{\market},
                        ``forecastability'' of near term rate of
                        revenue returns. Although the error function
                        is the most accurate, for the near term,
                        $H^{t} = \thurstlow^{t}$ may be used as a
                        reliable metric of ``forecastability'' of the
                        rate of revenue returns.}
                    \label{\SETLABEL:FN}
                \end{minipage}
                \hfill
                \begin{minipage}[t]{0.45\textwidth}
                    \epsfxsize=1.0\linewidth
                    \epsffile{\directory/datahurstlowfar.eps}
                    \caption[{\market}, ``forecastability'' of far
                        term rate of revenue returns]{{\market},
                        ``forecastability'' of far term rate of
                        revenue returns. Although the error function
                        is the most accurate, for the far term,
                        $\frac{1}{\sqrt{t}}$ may be used as a reliable
                        metric of ``forecastability'' of the rate of
                        revenue returns.}
                    \label{\SETLABEL:FF}
                \end{minipage}
            \end{center}
        \end{figure}

        \idx{operations research}
        As an interesting interpretation of the data presented in
        Figure~\ref{\SETLABEL:FN}, there may be, perhaps, some
        applicability to such operational agendas as inventory
        control. Maintaining too little inventory, obviously, will
        create a situation where the organization can not exploit
        market expansion, and maintaining too much inventory,
        likewise, would over extend the company, creating unnecessary
        losses when the market contracts. The company should maintain
        inventory levels that do not exceed, from
        Equation~\ref{\SETLABEL:MA}, ${\thurstlow}^{n} = 0.5$
        {\timescale}s of operations. Since the optimal amount of
        inventory and, from Equation~\ref{\SETLABEL:V}, the variance
        of change in the rate of revenue returns in the future can be
        calculated, there may, perhaps, be some applicability to a
        forecasting methodology that can be incorporated into other
        areas of operations research, for example the linear algebras
        using simplex methodologies for optimization of manufacturing
        processes. Traditionally, these forecasts are made by the
        sales department, and are subject to various subjective
        biases.

% Local Variables:
% TeX-parse-self: t
% TeX-auto-save: t
% TeX-master: "fractal.tex"
% End:


        %
% -----------------------------------------------------------------------------
%
% A license is hereby granted to reproduce this software source code and
% to create executable versions from this source code for personal,
% non-commercial use.  The copyright notice included with the software
% must be maintained in all copies produced.
%
% THIS PROGRAM IS PROVIDED "AS IS". THE AUTHOR PROVIDES NO WARRANTIES
% WHATSOEVER, EXPRESSED OR IMPLIED, INCLUDING WARRANTIES OF
% MERCHANTABILITY, TITLE, OR FITNESS FOR ANY PARTICULAR PURPOSE.  THE
% AUTHOR DOES NOT WARRANT THAT USE OF THIS PROGRAM DOES NOT INFRINGE THE
% INTELLECTUAL PROPERTY RIGHTS OF ANY THIRD PARTY IN ANY COUNTRY.
%
% Copyright (c) 1994-2006, John Conover, All Rights Reserved.
%
% Comments and/or bug reports should be addressed to:
%
%     john@email.johncon.com (John Conover)
%
% -----------------------------------------------------------------------------
%
% Revision: \RCSRevision \\
% Revision Time: \RCSTime UMT \\
% Revision Date: \RCSDate \\
% Revision Id: \RCSId \\
% Revision File: \RCSLog \\
\RCS $Revision: 0.0 $
\RCS $Date: 2006/01/20 04:38:13 $
\RCS $Id: simulation.tex,v 0.0 2006/01/20 04:38:13 john Exp $
% $Log: simulation.tex,v $
% Revision 0.0  2006/01/20 04:38:13  john
% Initial version
%
%
    \subsection{Simulation of Fixed Increment Approximation for Fiscal Strategy}
        \label{\SETLABEL:TSUNFAIRBROWNIAN}

        \subidx{\market}{market simulation}
        The data in this section is presented in tabular form in
        Section~\ref{\SETLABELREF:SIM}.
        Figure~\ref{\SETLABEL:TSUNFAIRBROWNIAN0} represents a
        constructional simulation of the time series data presented in
        Figure~\ref{\SETLABEL:TS}. The program {\it
        tsunfairbrownian}\/, which is briefly described in
        appendix~\ref{programs}, was used in the reconstruction. The
        reconstructed data is superimposed on the original time series
        data.  The program, {\it tsunfairbrownian}\/, essentially,
        constructs the new time series as a Brownian fractal with
        fixed increments---the value of the fixed increment is derived
        from the root mean square average of the normalized increments
        presented in Figure~\ref{\SETLABEL:TF}. The ``quality'' of
        such a reconstruction should be subject to adequate scepticism
        and scrutiny since, in all probability, the normalized
        increments presented in Figure~\ref{\SETLABEL:TF} represent a
        relatively complex process, that may not be ``modeled'' with
        such a simple methodology.

        As a further comparison of the the constructional simulation
        with the original time series data,
        Figure~\ref{\SETLABEL:TSUNFAIRBROWNIAN1} presents a normalized
        histogram of the normalized increments of the reconstructed
        time series, superimposed on the normalized histogram
        presented in Figure~\ref{\SETLABEL:NH}.

        \subidx{\market}{fiscal strategy, simulation}
        \subidx{markets}{simulation}
        \subidx{simulation}{markets}
        \subidx{strategy}{fiscal, simulation}
        \subidx{fiscal}{strategy, simulation}
        \subidx{programs}{tsunfairbrownian}
        \subidx{tsunfairbrownian}{program}
        \begin{figure}[ht]
            \begin{center}
                \begin{minipage}[t]{0.45\textwidth}
                    \epsfxsize=1.0\linewidth
                    \epsffile{\directory/tsunfairbrownian-f.eps}
                    \caption[{\market}, Time series data, empirical and
                        simulated]{{\market}, Time series data, empirical
                        and simulated, using the program {\it tsunfairbrownian}\/
                        with f = {\datafractionrms}. This data is
                        superimposed on the data presented in
                        Figure~\ref{\SETLABEL:TS}.}
                    \label{\SETLABEL:TSUNFAIRBROWNIAN0}
                \end{minipage}
                \hfill
                \begin{minipage}[t]{0.45\textwidth}
                    \epsfxsize=1.0\linewidth
                    \epsffile{\directory/tsunfairbrownian-f.tsfraction.tsnormal-s30.eps}
                    \caption[{\market}, normalized histogram,
                        empirical and simulated]{{\market}, normalized
                        histogram of the normalized increments of the
                        time series data shown in
                        Figure~\ref{\SETLABEL:TSUNFAIRBROWNIAN0},
                        empirical and simulated.  The empirical data
                        has a mean of {\datafractionmean}, with a
                        standard deviation of {\datafractionstddev}.
                        By comparison, the simulated data has a mean
                        of {\tsunfairbrownianfractionmean} with a
                        standard deviation of
                        {\tsunfairbrownianfractionstddev}. This data
                        is superimposed on the data presented in
                        Figure~\ref{\SETLABEL:NH}. The area under the
                        four curves is identical.}
                    \label{\SETLABEL:TSUNFAIRBROWNIAN1}
                \end{minipage}
            \end{center}
        \end{figure}

% Local Variables:
% TeX-parse-self: t
% TeX-auto-save: t
% TeX-master: "fractal.tex"
% End:


        %
% -----------------------------------------------------------------------------
%
% A license is hereby granted to reproduce this software source code and
% to create executable versions from this source code for personal,
% non-commercial use.  The copyright notice included with the software
% must be maintained in all copies produced.
%
% THIS PROGRAM IS PROVIDED "AS IS". THE AUTHOR PROVIDES NO WARRANTIES
% WHATSOEVER, EXPRESSED OR IMPLIED, INCLUDING WARRANTIES OF
% MERCHANTABILITY, TITLE, OR FITNESS FOR ANY PARTICULAR PURPOSE.  THE
% AUTHOR DOES NOT WARRANT THAT USE OF THIS PROGRAM DOES NOT INFRINGE THE
% INTELLECTUAL PROPERTY RIGHTS OF ANY THIRD PARTY IN ANY COUNTRY.
%
% Copyright (c) 1994-2006, John Conover, All Rights Reserved.
%
% Comments and/or bug reports should be addressed to:
%
%     john@email.johncon.com (John Conover)
%
% -----------------------------------------------------------------------------
%
% Revision: \RCSRevision \\
% Revision Time: \RCSTime UMT \\
% Revision Date: \RCSDate \\
% Revision Id: \RCSId \\
% Revision File: \RCSLog \\
\RCS $Revision: 0.0 $
\RCS $Date: 2006/01/20 04:38:13 $
\RCS $Id: maximum.tex,v 0.0 2006/01/20 04:38:13 john Exp $
% $Log: maximum.tex,v $
% Revision 0.0  2006/01/20 04:38:13  john
% Initial version
%
%
    \subsection{Simulation of Fixed Increment Approximation for Optimally Maximal Fiscal Strategy}
        \label{\SETLABEL:MAXSHANNON}
        \subidx{\market}{fiscal strategy, simulation}
        \subidx{\market}{maximum Shannon probability}
        \subidx{markets}{simulation}
        \subidx{simulation}{markets}
        \subidx{strategy}{optimum fiscal, simulation}
        \subidx{fiscal}{optimum strategy, simulation}
        \subidx{programs}{tsunfairbrownian}
        \subidx{tsunfairbrownian}{program}
        \subidx{Shannon}{probability}
        \subidx{probability}{Shannon}

        \subidx{strategy}{exploitable}
        \subidx{exploitable}{strategy}
        \subidx{programs}{tsshannonmax}
        \subidx{tsshannonmax}{program}
        \subidx{programs}{tsunfairbrownian}
        \subidx{tsunfairbrownian}{program}
        \subidx{strategy}{fiscal}
        \subidx{fiscal}{strategy}
        The data in this section is presented in tabular form in
        Section~\ref{\SETLABELREF:MAXSHANNON}. One of the issues of
        analysis, as mentioned in Section~\ref{\SETLABEL:OPS}, is to
        determine the maximum Shannon probability for the time series
        presented in Figure~\ref{\SETLABEL:TS}. Potentially, this
        could be exploited with an aggressive fiscal
        strategy. Figure~\ref{\SETLABEL:SHANNONMAX0} is a graph of the
        output of the {\it tsshannonmax}\/ program, which is described
        briefly in appendix~\ref{programs}. The maximum of this
        function is the maximum Shannon probability for the time
        series data presented in Figure~\ref{\SETLABEL:TS}.
        Figure~\ref{\SETLABEL:SHANNONMAX1} was constructed using {\it
        tsunfairbrownian}\/ program, which is also described in
        appendix~\ref{programs}, with the maximum Shannon probability,
        and the time series data presented in
        Figure~\ref{\SETLABEL:TS}. This represents a ``what if'' the
        investment strategy was changed from a Shannon probability of
        {\shannonlogreturns}, as derived in Section~\ref{\SETLABEL:SP}
        to {\shannonmax}. This process, essentially, extracts the
        random statistical data from the time series presented in
        Figure~\ref{\SETLABEL:TS}, and constructs a new time series,
        using the random statistical data, with a different investment
        strategy.  The program, {\it tsunfairbrownian}\/, essentially,
        constructs the new time series as a Brownian fractal with
        fixed increments.  The ``quality'' of such a reconstruction
        should be subject to adequate scepticism and scrutiny since,
        in all probability, the increments in the original data
        represent a relatively complex process, that may not be
        ``modeled'' with such a simple methodology.

        \begin{figure}[ht]
            \begin{center}
                \begin{minipage}[t]{0.45\textwidth}
                    \epsfxsize=1.0\linewidth
                    \epsffile{\directory/data.tsshannonmax.eps}
                    \caption[{\market}, maximum rate of revenue
                        returns] {{\market}, maximum rate of revenue
                        returns, per {\timescale}, vs. Shannon
                        probability. The maximum rate of revenue
                        returns, per {\timescale}, occurs at a Shannon
                        probability of {\shannonmax}.}
                    \label{\SETLABEL:SHANNONMAX0}
                \end{minipage}
                \hfill
                \begin{minipage}[t]{0.45\textwidth}
                    \epsfxsize=1.0\linewidth
                    \epsffile{\directory/data.tsshannonmax-p.tsunfairbrownian-p.eps}
                    \caption[{\market}, maximum rate of revenue
                        returns] {{\market}, maximum rate of revenue
                        returns, per {\timescale}, at a Shannon
                        probability, of {\shannonmax}, corresponding
                        to a ``wager'' fraction of {\twoponemax}.}
                    \label{\SETLABEL:SHANNONMAX1}
                \end{minipage}
            \end{center}
        \end{figure}

        \subidx{fractional}{Brownian motion}
        \subidx{Brownian motion}{fractional}
        \subidx{Shannon}{probability}
        \subidx{probability}{Shannon}
        \subidx{programs}{tsshannonmax}
        \subidx{tsshannonmax}{program}
        If it is assumed that the time series data set, presented in
        Figure~\ref{\SETLABEL:TS}, constitutes classical Brownian
        motion, then the Shannon probability can be calculated by
        counting the total number of {\timescale}s that the {\market}
        movement was positive, and dividing by the total number of
        {timescale}s represented in the time series. This quotient is
        {\pmax}, as compared with the predicted value from the program
        {\it tsshannonmax}\/ of {\shannonmax}.

% Local Variables:
% TeX-parse-self: t
% TeX-auto-save: t
% TeX-master: "fractal.tex"
% End:


        %
% -----------------------------------------------------------------------------
%
% A license is hereby granted to reproduce this software source code and
% to create executable versions from this source code for personal,
% non-commercial use.  The copyright notice included with the software
% must be maintained in all copies produced.
%
% THIS PROGRAM IS PROVIDED "AS IS". THE AUTHOR PROVIDES NO WARRANTIES
% WHATSOEVER, EXPRESSED OR IMPLIED, INCLUDING WARRANTIES OF
% MERCHANTABILITY, TITLE, OR FITNESS FOR ANY PARTICULAR PURPOSE.  THE
% AUTHOR DOES NOT WARRANT THAT USE OF THIS PROGRAM DOES NOT INFRINGE THE
% INTELLECTUAL PROPERTY RIGHTS OF ANY THIRD PARTY IN ANY COUNTRY.
%
% Copyright (c) 1994-2006, John Conover, All Rights Reserved.
%
% Comments and/or bug reports should be addressed to:
%
%     john@email.johncon.com (John Conover)
%
% -----------------------------------------------------------------------------
%
% Revision: \RCSRevision \\
% Revision Time: \RCSTime UMT \\
% Revision Date: \RCSDate \\
% Revision Id: \RCSId \\
% Revision File: \RCSLog \\
\RCS $Revision: 0.0 $
\RCS $Date: 2006/01/20 04:38:13 $
\RCS $Id: verification.tex,v 0.0 2006/01/20 04:38:13 john Exp $
% $Log: verification.tex,v $
% Revision 0.0  2006/01/20 04:38:13  john
% Initial version
%
%
    \subsection{Qualitative Verification of Fixed Increment Approximation Analysis}
        \label{\SETLABEL:QVA}

        \subidx{\market}{verification of analysis}
        \subidx{verification}{analysis}
        \subidx{analysis}{verification}
        \subidx{quality}{of analysis}
        \subidx{verification}{of methodology}
        \subidx{methodology}{verification of}
        \subidx{Shannon}{probability}
        \subidx{probability}{Shannon}

        This section evaluates various values based on the ``average''
        of the normalized increments presented in
        Figure~\ref{\SETLABEL:TFA}. These values are an approximation
        to a, probably, complex process with a distribution shown in
        Figure~\ref{\SETLABEL:TF}. These values will be used in a
        fixed increment Brownian fractal analysis of the {\market},
        and may, or may not, provide adequate accuracy for
        projections.

        The data in this section is presented in tabular form in
        sections~\ref{\SETLABELREF:VI1} and~\ref{\SETLABELREF:VI2}.
        As a subjective evaluation of the ``quality'' of the analysis
        of the {\market}, from Chapter~\ref{methodology},
        Equation~\ref{metricvalues1}, and using the mean and root mean
        square values of the normalized increments of the time series
        data presented in Figure~\ref{\SETLABEL:TS} from
        Figure~\ref{\SETLABEL:TF}, and the Shannon probability as
        calculated by counting the total number of {\timescale}s that
        the {\market} movement was positive, as presented in
        Section~\ref{\SETLABEL:MAXSHANNON}:

        \begin{eqnarray}
                  P & \approx & \frac{\frac{avg}{rms} + 1}{2}\\
            {\pmax} & \approx & \frac{\frac{\datafractionmean}{\datafractionrms} + 1}{2}\\
            {\pmax} & \approx & {\avgrms}
            \label{\SETLABEL:AVGS}
        \end{eqnarray}

        \subidx{Shannon}{probability}
        \subidx{probability}{Shannon}
        \noindent and comparing these values to the Shannon
        probability, as found by the {\it tsshannonmax}\/ program, which
        iterates for a maximum:

        \begin{eqnarray}
            {\pmax} \approx {\avgrms} \approx {\shannonmax}
        \end{eqnarray}

        \subidx{logarithmic}{returns}
        \subidx{returns}{logarithmic}
        In addition, the different methods of calculating the
        logarithmic returns, presented in Section~\ref{\SETLABEL:FS},
        should be compared. The four methods used were the mean of
        Figure~\ref{\SETLABEL:TF}, the constant in the least squares
        approximation to Figure~\ref{\SETLABEL:TF}, the least squares
        exponential approximation to Figure~\ref{\SETLABEL:TS}, and
        the logarithmic returns of Figure~\ref{\SETLABEL:TS}, derived
        as the mean of the logarithms of the quotients of the
        increments. The values for each of the methods are,
        respectively:

        \begin{equation}
            \datafractionmeanbits \approx \datafractionconstantbits \approx \datatslsqepbits \approx \logreturns
        \end{equation}

        It is implied in Section~\ref{\SETLABEL:FS},
        Subsection~\ref{\SETLABEL:SP} and in
        Section~\ref{\SETLABEL:TSUNFAIRBROWNIAN} that, a Brownian
        motion with fixed increments fractal may ``model'' the
        {\market}. Using Equation~\ref{stddev9} from
        Chapter~\ref{general}, Section~\ref{abmfi}:

        \begin{eqnarray}
                                    rms \left(2P - 1\right) & \approx & \frac{\sigma \left(2P - 1\right)}{2 \sqrt{P\left(1 - P\right)}}\\
            \datafractionrms \left(2 \cdot \pmax - 1\right) & \approx & \frac{\datafractionstddev \left(2 \cdot \pmax - 1\right)}{2\sqrt{\pmax \left(1 - \pmax\right)}}\\
                       \datafractionrms \cdot \twopminusone & \approx & \datafractionstddev \cdot \twopx\\
                                                      \rmsp & \approx & \sigmap
        \end{eqnarray}

        \noindent and, equating to the mean:

        \begin{equation}
            \datafractionmean \approx \rmsp \approx \sigmap
        \end{equation}

        \subidx{Shannon}{probability}
        \subidx{probability}{Shannon}
        \noindent where, as in Equation~\ref{\SETLABEL:AVGS} using the
        mean, root mean square, and standard deviation values of the
        normalized increments of the time series data presented in
        Figure~\ref{\SETLABEL:TS} from Figure~\ref{\SETLABEL:TF}, and
        the Shannon probability as calculated by counting the total
        number of {\timescale}s that the {\market} movement was
        positive, as presented in Section~\ref{\SETLABEL:MAXSHANNON}.

        As a final qualitative comparison, the absolute value of the
        normalized increments should be the same as the root mean
        square value\footnote{The absolute value of the normalized
        increments, when averaged, is related to the root mean square
        of the increments by a constant. If the normalized increments
        are a fixed increment, the constant is unity. If the
        normalized increments have a Gaussian distribution, the
        constant is $\approx 0.8$ depending on the accuracy of of
        ``fit'' to a Gaussian distribution.}, where the absolute value
        is presented in Figure~\ref{\SETLABEL:TFA}, and the root mean
        square value is presented in Figure~\ref{\SETLABEL:TF}:

        \begin{equation}
            \datafractionabsmean \approx \datafractionrms
        \end{equation}

        Note, that if the {\market} could be ``modeled'' as a Brownian
        motion with fixed increments fractal, then the standard
        deviation of the absolute value of the normalized increments
        of the time series data presented in Figure~\ref{\SETLABEL:TS}
        from Figure~\ref{\SETLABEL:TF} should be zero. It is
        $\datafractionabsstddev$.

% Local Variables:
% TeX-parse-self: t
% TeX-auto-save: t
% TeX-master: "fractal.tex"
% End:


    \renewcommand{\market}{Time Sampled Coin Tossing Game}
    \renewcommand{\directory}{../markets/tscoin.tssample}
    \renewcommand{\datafractionmean}{0.008052}
\renewcommand{\datafractionmeanbits}{0.011570}
\renewcommand{\datafractionmeanq}{0.002684}
\renewcommand{\datafractionmeanbitsq}{0.003867}
\renewcommand{\datafractionstddev}{0.038579}
\renewcommand{\datafractionrms}{0.039311}
\renewcommand{\avgrms}{0.602414}
\renewcommand{\ncompanies}{5.210454}
\renewcommand{\pncompanies}{0.544866}
\renewcommand{\datafractionabsmean}{0.029745}
\renewcommand{\datafractionabsstddev}{0.025769}
\renewcommand{\datafractionconstant}{0.010041}
\renewcommand{\datafractionconstantbits}{0.014414}
\renewcommand{\datafractionconstantq}{0.003347}
\renewcommand{\datafractionconstantbitsq}{0.004821}
\renewcommand{\datafractionslope}{-0.000021}
\renewcommand{\datafractionabsconstant}{0.035145}
\renewcommand{\datafractionabsslope}{-0.000057}
\renewcommand{\hurstall}{0.659558}
\renewcommand{\hurstlow}{0.707509}
\renewcommand{\hurstlowtwo}{1.415018}
\renewcommand{\hurstlowhundred}{70.750900}
\renewcommand{\hcalcall}{0.184942}
\renewcommand{\hcalclow}{0.102042}
\renewcommand{\shannonmax}{0.604167}
\renewcommand{\twoponemax}{0.208334}
\renewcommand{\logreturns}{0.010456}
\renewcommand{\twologreturns}{1.007274}
\renewcommand{\twologreturnshundred}{0.727387}
\renewcommand{\oneoverlogreturns}{95.638868}
\renewcommand{\pmax}{0.602094}
\renewcommand{\twopminusone}{0.204188}
\renewcommand{\rmsp}{0.008027}
\renewcommand{\twopx}{0.208583}
\renewcommand{\sigmap}{0.008047}
\renewcommand{\tsunfairbrownianfractionmean}{0.007862}
\renewcommand{\tsunfairbrownianfractionstddev}{0.038619}
\renewcommand{\shannonlogreturns}{0.560125}
\renewcommand{\shannonlogreturnshundred}{56.012500}
\renewcommand{\twopone}{0.120250}
\renewcommand{\twoponehundred}{12.025000}
\renewcommand{\hundredtwoponehundred}{87.975000}
\renewcommand{\hundredshannonlogreturnshundred}{43.987500}
\renewcommand{\datatslsqepbits}{0.007623}
\renewcommand{\thurstall}{0.633980}
\renewcommand{\thurstlow}{0.710108}
\renewcommand{\thurstlowtwo}{1.420216}
\renewcommand{\thurstlowhundred}{71.010800}
\renewcommand{\thcalcall}{0.247886}
\renewcommand{\thcalclow}{0.171737}
\renewcommand{\chisquared}{2.862000}
\renewcommand{\critical}{42.557000}

    \renewcommand{\timescale}{tosses}
    \subidx{market}{\market}
    \idx{\market}

    \section{\market}

        \renewcommand{\SETLABEL}{\LABPRE:TSCT}
        \renewcommand{\SETLABELQ}{\LABPRE:TSCTQ}
        \label{\SETLABEL}
        \renewcommand{\SETLABELREF}{\LABPREREF:TSCT}

        \idx{tscoin}
        \idx{tsunfairbrownian}
        \idx{tssample}
        \subidx{programs}{tscoin}
        \subidx{tscoin}{program}
        \subidx{tsunfairbrownian}{program}
        \subidx{programs}{tsunfairbrownian}
        \subidx{programs}{tssample}
        \subidx{tssample}{program}
        For the analysis, the data was in the directory
        {\directory}\footnote{As a simulation model, the program {\it
        tscoin}\/ was run to make a time series data file, with the
        following parameters:

        \vspace{0.1in}
        {\noindent}tscoin -p 0.5447 1500 > data.1
        \vspace{0.1in}

        \noindent to make a time series of 1500 elements, with a
        Shannon probability of 0.5447. Since $f = 2P - 1$, where the
        desired Shannon probability, $P$, is $0.6$, $f$ must be
        reduced by a factor of $\frac{1}{\sqrt{5}}$. Reducing $f$ from
        $0.2$ to $0.0894$, and recalculating $P$ to be $0.5447$.  Then
        the program {\it tssample}\/ was run with the following
        parameters:

        \vspace{0.1in}
        {\noindent}tssample -i 5 data.1 > data
        \vspace{0.1in}

        \noindent to time sample every fifth element in the time
        series to make a time sampled time series.  The data is by
        {\timescale}.}.

        The data in this section is presented in tabular form in
        Section~\ref{\SETLABELREF}. Note that in this analysis, the
        rate of revenue returns means the increase or decrease in the
        cumulative sum of the {\market}. This is included for
        ``theoretical'' comparative purposes.

        %
% -----------------------------------------------------------------------------
%
% A license is hereby granted to reproduce this software source code and
% to create executable versions from this source code for personal,
% non-commercial use.  The copyright notice included with the software
% must be maintained in all copies produced.
%
% THIS PROGRAM IS PROVIDED "AS IS". THE AUTHOR PROVIDES NO WARRANTIES
% WHATSOEVER, EXPRESSED OR IMPLIED, INCLUDING WARRANTIES OF
% MERCHANTABILITY, TITLE, OR FITNESS FOR ANY PARTICULAR PURPOSE.  THE
% AUTHOR DOES NOT WARRANT THAT USE OF THIS PROGRAM DOES NOT INFRINGE THE
% INTELLECTUAL PROPERTY RIGHTS OF ANY THIRD PARTY IN ANY COUNTRY.
%
% Copyright (c) 1994-2006, John Conover, All Rights Reserved.
%
% Comments and/or bug reports should be addressed to:
%
%     john@email.johncon.com (John Conover)
%
% -----------------------------------------------------------------------------
%
% Revision: \RCSRevision \\
% Revision Time: \RCSTime UMT \\
% Revision Date: \RCSDate \\
% Revision Id: \RCSId \\
% Revision File: \RCSLog \\
\RCS $Revision: 0.0 $
\RCS $Date: 2006/01/20 04:38:13 $
\RCS $Id: fraction.tex,v 0.0 2006/01/20 04:38:13 john Exp $
% $Log: fraction.tex,v $
% Revision 0.0  2006/01/20 04:38:13  john
% Initial version
%
%
    \subsection{Time Series Increments Analysis}
        \label{\SETLABEL:TSA}

        \subidx{\market}{Time series analysis}
        \subidx{time series}{increments}
        \subidx{time series}{analysis}
        \subidx{cumulative sum}{analysis}
        \subidx{analysis}{cumulative sum}
        \subidx{analysis}{random process}
        \subidx{random process}{analysis}
        \subidx{Gaussian}{increments}
        \subidx{increments}{Gaussian}
        \subidx{Brownian}{motion, fractional}
        \subidx{fractional}{Brownian motion}
        \subidx{fractal}{Brownian motion}
        The data in this section is presented in tabular form in
        Section~\ref{\SETLABELREF:TSA}.  Figure~\ref{\SETLABEL:TS} is
        a graph of the time series data for the {\market}.

        \subidx{increments}{normalized}
        \subidx{normalized}{increments}
        \subidx{programs}{tsfraction}
        \subidx{tsfraction}{program}
        Figure~\ref{\SETLABEL:TF} is a graph of the normalized
        increments of the time series data presented in
        Figure~\ref{\SETLABEL:TS}. The data presented was made by
        running the program {\it tsfraction}\/ on the time series
        data. The program {\it tsfraction}\/ is described briefly in
        Appendix~\ref{programs}, and subtracts the previous value from
        the next value, dividing this difference by the previous
        value, for each element in the time series data. The new time
        series contains the instantaneous change in the rate of
        revenue returns, divided by the magnitude of the instantaneous
        rate of revenue returns.

        \subidx{mean}{standard deviation}
        \subidx{standard deviation}{mean}
        \idx{root mean square}
        \idx{least squares approximation}
        \begin{figure}[ht]
            \begin{center}
                \begin{minipage}[t]{0.45\textwidth}
                    \epsfxsize=1.0\linewidth
                    \epsffile{\directory/data.eps}
                    \caption{{\market}, time series data.}
                    \label{\SETLABEL:TS}
                    \label{\SETLABELQ:TS}
                \end{minipage}
                \hfill
                \begin{minipage}[t]{0.45\textwidth}
                    \epsfxsize=1.0\linewidth
                    \epsffile{\directory/data.tsfraction.eps}
                    \caption[{\market}, normalized
                        increments]{{\market}, normalized increments
                        of the time series data presented in
                        Figure~\ref{\SETLABEL:TS}. The mean is
                        {\datafractionmean} with a standard deviation
                        of {\datafractionstddev}. The formula for the
                        least squares approximation is
                        ${\datafractionconstant} +
                        {\datafractionslope}t$, and the root mean
                        squared value is {\datafractionrms}. The
                        graph, labeled ``data\-.tsfraction\-.tsrms,''
                        is the running root mean square, and
                        ``data\-.tsfraction\-.tsavg'' is the running
                        average of the normalized increments.  This
                        graph is the fraction of change in the time
                        series, as a function of time. Note that the
                        slope of the mean, {\datafractionslope}, is
                        the coefficient of the nonlinearity term in
                        the normalized increments. See
                        Chapter~\ref{general}, Section~\ref{nlextend}
                        for a possible application of the logistic
                        function to this data set.}
                    \label{\SETLABEL:TF}
                    \label{\SETLABELQ:TF}
                \end{minipage}
            \end{center}
        \end{figure}

        \subidx{absolute value}{increments}
        \subidx{increments}{absolute value}

        Figure~\ref{\SETLABEL:TFA} is a graph of the absolute value of
        the normalized increments of the time series data presented in
        Figure~\ref{\SETLABEL:TF}. The data presented was made by
        running the Unix utility sed(1) on the normalized increments
        time series data to remove the negative signs. This is an
        absolute value procedure.  The resulting time series contains
        the absolute value of the instantaneous change in the rate of
        revenue returns, divided by the magnitude of the instantaneous
        rate of revenue returns\footnote{The absolute value of the
        normalized increments, when averaged, is related to the root
        mean square of the increments by a constant. If the normalized
        increments are a fixed increment, the constant is unity. If
        the normalized increments have a Gaussian distribution, the
        constant is $\approx 0.8$ depending on the accuracy of of
        ``fit'' to a Gaussian distribution.}.

        \subidx{histogram}{normalized}
        \subidx{normalized}{histogram}
        \subidx{programs}{tsnormal}
        \subidx{tsnormal}{program}
        \subidx{mean}{standard deviation}
        \subidx{standard deviation}{mean}
        \idx{root mean square}
        \idx{least squares approximation}
        \subidx{\market}{analysis of increments}
        Figure~\ref{\SETLABEL:NH} is the normalized histogram of the
        normalized increments of the time series data shown in
        Figure~\ref{\SETLABEL:TF}. The abscissa is 3 $\sigma$ limits,
        and the area under the two curves is identical. The data for
        this figure was produced by the program {\it tsnormal}\/,
        which is described briefly in Appendix~\ref{programs}.

        \begin{figure}[ht]
            \begin{center}
                \begin{minipage}[t]{0.45\textwidth}
                    \epsfxsize=1.0\linewidth
                    \epsffile{\directory/data.tsfraction.abs.eps}
                    \caption[{\market}, absolute value of the
                        normalized increments]{{\market}, absolute
                        value of the normalized increments of the time
                        series data presented in
                        Figure~\ref{\SETLABEL:TF}.  The mean is
                        {\datafractionabsmean} with a standard
                        deviation of {\datafractionabsstddev}. The
                        formula for the least squares approximation is
                        ${\datafractionabsconstant} +
                        {\datafractionabsslope}t$, and the root mean
                        square value, from Figure~\ref{\SETLABEL:TF},
                        is {\datafractionrms}.  The graph, labeled
                        ``data\-.tsfraction\-.tsrms,'' is the running
                        root mean square, and
                        ``data\-.tsfraction\-.tsavg'' is the running
                        average of the normalized increments presented
                        in Figure~\ref{\SETLABEL:TF}, superimposed
                        here for convenience. This graph is the
                        absolute value of the fraction of change in
                        the time series, as a function of time.}
                    \label{\SETLABEL:TFA}
                    \label{\SETLABELQ:TFA}
                \end{minipage}
                \hfill
                \begin{minipage}[t]{0.45\textwidth}
                    \epsfxsize=1.0\linewidth
                    \epsffile{\directory/data.tsfraction.tsnormal-s30.eps}
                    \caption[{\market}, normalized histogram of the
                        normalized increments]{{\market}, normalized
                        histogram of the normalized increments of the
                        time series data shown in
                        Figure~\ref{\SETLABEL:TF}.  The data has a
                        mean of {\datafractionmean}, with a standard
                        deviation of {\datafractionstddev}.  The area
                        under the two curves is identical. The
                        $\chi^2$ value of the observed and expected
                        values of the two curves is {\chisquared},
                        with a critical value of {\critical}.}
                    \label{\SETLABEL:NH}
                \end{minipage}
            \end{center}
        \end{figure}

        \subidx{programs}{tsXsquared}
        \subidx{tsXsquared}{program}
        \subidx{\market}{chi-squared values of increments}
        The program {\it tsXsquared}\/, which is briefly described in
        appendix~\ref{programs}, was used to derive the $\chi^2$
        statistics for the data presented in
        Figure~\ref{\SETLABEL:NH}.

        \subidx{programs}{tsstatest}
        \subidx{tsstatest}{program}
        \subidx{\market}{statistical estimates}

        Figure~\ref{\SETLABEL:SE} is the statistical estimate for the
        data presented in Figure~\ref{\SETLABEL:TF}, as derived by the
        program {\it tsstatest}\/, which is briefly described in
        appendix~\ref{programs}.

        \begin{figure}[ht]
            \begin{center}
                \begin{minipage}[t]{\textwidth}
                    \center{\fbox{\parbox{0.9\textwidth}{\XXX{\directory/data.tsstatest-f0.1-c0.9-i.tex}}}}
                    \caption[{\market}, statistical estimates of the
                        normalized increments]{{\market}, statistical
                        estimates of the normalized increments of the
                        time series shown in Figure~\ref{\SETLABEL:TF}.
                        The table was produced with the {\it
                        tsstatest}\/ program, and illustrates the
                        size of the data set required for a confidence
                        level of 90\%, with an error estimate of $\pm$
                        10\%, or alternately, the error estimate on
                        the time series shown in Figure~\ref{\SETLABEL:TF}.}
                    \label{\SETLABEL:SE}
                \end{minipage}
            \end{center}
        \end{figure}

        Note that the data set size estimations, as produced by the
        {\it tsstatest}\/ program, are probably very conservative,
        depending on the magnitude of the Shannon probability, $P =
        \shannonlogreturns$, as derived in
        Section~\ref{\SETLABEL:SP}. See Chapter~\ref{general},
        Section~\ref{serdss} for possible alternative methodologies
        for addressing the analysis of fractal time series with
        limited data set sizes. Depending on the magnitude of the
        Shannon probability, $P$, these estimates can be several
        orders of magnitude too high.

        \subidx{derivative of increments}{normalized}
        \subidx{normalized}{derivative of increments}
        \subidx{programs}{tsderivative}
        \subidx{tsderivative}{program}
        Figure~\ref{\SETLABEL:TF1} is the normalized histogram of the
        first derivative of the normalized increments of the time
        series data shown in Figure~\ref{\SETLABEL:TF}. In principle,
        if the distribution of the normalized increments presented in
        Figure~\ref{\SETLABEL:NH} is Gaussian in nature, this
        distribution would be similar to ``white noise,'' as presented
        in appendix~\ref{programs}, Figure~\ref{whiteexample}. The
        data was generated by the {\it tsderivative}\/ program, which
        is briefly described in
        appendix~\ref{programs}. Figure~\ref{\SETLABEL:TF2} is the
        normalized histogram of the second derivative of the
        normalized increments of the time series data shown in
        Figure~\ref{\SETLABEL:TF}. In principle, if the distribution
        of the normalized increments presented in
        Figure~\ref{\SETLABEL:NH} is an integrated Gaussian
        distribution in nature, this distribution would be similar to
        ``white noise,'' as presented in appendix~\ref{programs},
        Figure~\ref{whiteexample}.

        \begin{figure}[ht]
            \begin{center}
                \begin{minipage}[t]{0.45\textwidth}
                    \epsfxsize=1.0\linewidth
                    \epsffile{\directory/data.tsfraction.tsderivative.tsnormal-s30.eps}
                    \caption[{\market}, histogram of the first
                        derivative of the increments]{{\market},
                        normalized histogram of the first derivative
                        of the normalized increments of the time
                        series data shown in
                        Figure~\ref{\SETLABEL:TF}.}
                    \label{\SETLABEL:TF1}
                \end{minipage}
                \hfill
                \begin{minipage}[t]{0.45\textwidth}
                    \epsfxsize=1.0\linewidth
                    \epsffile{\directory/data.tsfraction.2tsderivative.tsnormal-s30.eps}
                    \caption[{\market}, histogram of the second
                        derivative of the increments]{{\market},
                        normalized histogram of second derivative of
                        the the normalized increments of the time
                        series data shown in
                        Figure~\ref{\SETLABEL:TF}.}
                    \label{\SETLABEL:TF2}
                \end{minipage}
            \end{center}
        \end{figure}

        \subidx{fractal}{range}
        \subidx{fractal}{R/S analysis}
        \subidx{\market}{rate of revenue returns, range}
        \subidx{\market}{deterministic mechanism}
        \subidx{deterministic}{mechanism}
        \subidx{mechanism}{deterministic}
        Figure~\ref{\SETLABEL:TR} is the range of values of the time
        series shown in Figure~\ref{\SETLABEL:TS}. The horizontal axis
        is time into the future. In principle, if the time series was
        characterized as fractional Brownian motion the graph in
        Figure~\ref{\SETLABEL:TR} would be a square root
        function\footnote{Note that the ``roughness,'' or ``sawtooth''
        characteristics of the graph in Figure~\ref{\SETLABEL:TR} are
        a computational artifact---caused by not using the -m option
        to the program {\it tshurst}\/, which is computationally
        inefficient.}. Figure~\ref{\SETLABEL:TD} is the deterministic
        map of the normalized increments of the time series data shown
        in Figure~\ref{\SETLABEL:TF}. The deterministic map is useful
        for determining if a time series was created by a
        deterministic mechanism. This, essentially, maps each element
        in the time series with the previous element in the time
        series.  See,~\cite[pp. 745]{Peitgen}.

        \begin{figure}[ht]
            \begin{center}
                \begin{minipage}[t]{0.45\textwidth}
                    \epsfxsize=1.0\linewidth
                    \epsffile{\directory/data.tshurst-f.eps}
                    \caption[{\market}, range]{{\market}, range of the
                        time series data shown in
                        Figure~\ref{\SETLABEL:TS}.}
                    \label{\SETLABEL:TR}
                \end{minipage}
                \hfill
                \begin{minipage}[t]{0.45\textwidth}
                    \epsfxsize=1.0\linewidth
                    \epsffile{\directory/data.tsfraction.tsdeterministic.eps}
                    \caption[{\market}, deterministic map]{{\market},
                        deterministic map of the normalized increments
                        of the time series data shown in
                        Figure~\ref{\SETLABEL:TF}.}
                    \label{\SETLABEL:TD}
                \end{minipage}
            \end{center}
        \end{figure}

% Local Variables:
% TeX-parse-self: t
% TeX-auto-save: t
% TeX-master: "fractal.tex"
% End:


        %
% -----------------------------------------------------------------------------
%
% A license is hereby granted to reproduce this software source code and
% to create executable versions from this source code for personal,
% non-commercial use.  The copyright notice included with the software
% must be maintained in all copies produced.
%
% THIS PROGRAM IS PROVIDED "AS IS". THE AUTHOR PROVIDES NO WARRANTIES
% WHATSOEVER, EXPRESSED OR IMPLIED, INCLUDING WARRANTIES OF
% MERCHANTABILITY, TITLE, OR FITNESS FOR ANY PARTICULAR PURPOSE.  THE
% AUTHOR DOES NOT WARRANT THAT USE OF THIS PROGRAM DOES NOT INFRINGE THE
% INTELLECTUAL PROPERTY RIGHTS OF ANY THIRD PARTY IN ANY COUNTRY.
%
% Copyright (c) 1994-2006, John Conover, All Rights Reserved.
%
% Comments and/or bug reports should be addressed to:
%
%     john@email.johncon.com (John Conover)
%
% -----------------------------------------------------------------------------
%
% Revision: \RCSRevision \\
% Revision Time: \RCSTime UMT \\
% Revision Date: \RCSDate \\
% Revision Id: \RCSId \\
% Revision File: \RCSLog \\
\RCS $Revision: 0.0 $
\RCS $Date: 2006/01/20 04:38:13 $
\RCS $Id: instant.tex,v 0.0 2006/01/20 04:38:13 john Exp $
% $Log: instant.tex,v $
% Revision 0.0  2006/01/20 04:38:13  john
% Initial version
%
%
    \subsection{Instantaneous Analysis of Normalized Increments}
        \label{\SETLABEL:IA}

        \subidx{\market}{instantaneous analysis of normalized increments}
        \idx{average of normalized increments}
        \idx{root mean square of normalized increments}
        \subidx{Shannon probability}{instantaneous computation of}
        \subidx{average of normalized increments}{instantaneous computation of}
        \subidx{root mean square of normalized increments}{instantaneous computation of}
        \subidx{instantaneous computation}{Shannon probability}
        \subidx{instantaneous computation}{average of normalized increments}
        \subidx{instantaneous computation}{root mean square of normalized increments}
        \idx{time series}
        \subidx{time series}{instantaneous analysis}
        \subidx{instantaneous analysis}{time series}
        \subidx{time series}{increments}
        \subidx{time series}{analysis}
        \subidx{Shannon}{probability}
        \subidx{probability}{Shannon}
        \subidx{normalized}{increments}
        \subidx{increments}{normalized}

        The program {\it tsinstant}\/, which is briefly described in
        Appendix~\ref{programs}, is for finding the instantaneous
        fraction of change in a time series. The value of a sample in
        the time series is subtracted from the previous sample in the
        time series, and divided by the value of the previous sample.
        As explained in Chapter~\ref{general},
        Sections~\ref{derivation},~\ref{GA},~\ref{abmfi},~\ref{aftsma}
        and,~\ref{ompl} for Brownian motion, random walk fractals, the
        absolute value of the instantaneous fraction of change is also
        the root mean square of the instantaneous fraction of
        change\footnote{The absolute value of the normalized
        increments, when averaged, is related to the root mean square
        of the increments by a constant. If the normalized increments
        are a fixed increment, the constant is unity. If the
        normalized increments have a Gaussian distribution, the
        constant is $\approx 0.8$ depending on the accuracy of of
        ``fit'' to a Gaussian distribution.}. Squaring this value is
        the average of the instantaneous fraction of change, and
        adding unity to the absolute value of the instantaneous
        fraction of change, and dividing by two, is the Shannon
        probability of the instantaneous fraction of change.

        Figure~\ref{\SETLABEL:IA1} is the instantaneous value of the
        root mean square of the normalized increments for the
        {\market}, and Figure~\ref{\SETLABEL:IA2} is the instantaneous
        Shannon probability for the normalized increments.

        \begin{figure}[ht]
            \begin{center}
                \begin{minipage}[t]{0.45\textwidth}
                    \epsfxsize=1.0\linewidth
                    \epsffile{\directory/data.tsinstant-r.eps}
                    \caption[{\market}, instantaneous value of
                        rms.]{{\market}, instantaneous value of the
                        root mean square of the normalized increments,
                        provided by running the program {\it
                        tsinstant}\/ with the -r option on the data
                        presented in Figure~\ref{\SETLABEL:TS}.}
                    \label{\SETLABEL:IA1}
                    \label{\SETLABELQ:IA1}
                \end{minipage}
                \hfill
                \begin{minipage}[t]{0.45\textwidth}
                    \epsfxsize=1.0\linewidth
                    \epsffile{\directory/data.tsinstant-s.eps}
                    \caption[{\market}, instantaneous value of
                        Shannon probability.]{{\market}, instantaneous
                        value of the Shannon probability of the
                        normalized increments, provided by running the
                        program {\it tsinstant}\/ with the -s option
                        on the data presented in
                        Figure~\ref{\SETLABEL:TS}.}
                    \label{\SETLABEL:IA2}
                    \label{\SETLABELQ:IA2}
                \end{minipage}
            \end{center}
        \end{figure}

% Local Variables:
% TeX-parse-self: t
% TeX-auto-save: t
% TeX-master: "fractal.tex"
% End:


        %
% -----------------------------------------------------------------------------
%
% A license is hereby granted to reproduce this software source code and
% to create executable versions from this source code for personal,
% non-commercial use.  The copyright notice included with the software
% must be maintained in all copies produced.
%
% THIS PROGRAM IS PROVIDED "AS IS". THE AUTHOR PROVIDES NO WARRANTIES
% WHATSOEVER, EXPRESSED OR IMPLIED, INCLUDING WARRANTIES OF
% MERCHANTABILITY, TITLE, OR FITNESS FOR ANY PARTICULAR PURPOSE.  THE
% AUTHOR DOES NOT WARRANT THAT USE OF THIS PROGRAM DOES NOT INFRINGE THE
% INTELLECTUAL PROPERTY RIGHTS OF ANY THIRD PARTY IN ANY COUNTRY.
%
% Copyright (c) 1994-2006, John Conover, All Rights Reserved.
%
% Comments and/or bug reports should be addressed to:
%
%     john@email.johncon.com (John Conover)
%
% -----------------------------------------------------------------------------
%
% Revision: \RCSRevision \\
% Revision Time: \RCSTime UMT \\
% Revision Date: \RCSDate \\
% Revision Id: \RCSId \\
% Revision File: \RCSLog \\
\RCS $Revision: 0.0 $
\RCS $Date: 2006/01/20 04:38:13 $
\RCS $Id: logistic.tex,v 0.0 2006/01/20 04:38:13 john Exp $
% $Log: logistic.tex,v $
% Revision 0.0  2006/01/20 04:38:13  john
% Initial version
%
%
    \subsection{Logistic Analysis}
        \label{\SETLABEL:LA}

        \subidx{\market}{Logistic function analysis}
        \subidx{time series}{logistic function}
        \subidx{logistic function}{time series}
        \subidx{time series}{increments}
        \subidx{time series}{analysis}
        \subidx{cumulative sum}{analysis}
        \subidx{analysis}{cumulative sum}
        \subidx{analysis}{random process}
        \subidx{random process}{analysis}
        The data in this section is presented in tabular form in
        Section~\ref{\SETLABELREF:LAA}.  Figure~\ref{\SETLABEL:LA1} is
        a graph of the logistic function estimates of the time series
        data for the {\market}. The reader is cautioned that these
        graphs are constructed using the method suggested in
        Chapter~\ref{general}, Section~\ref{nlextend} and enormous
        precision is required for adequate prediction of the logistic
        function,~\cite{Modis}. Particularly, the non-linear term will
        usually require intervention to produce a practical fit to the
        data. In addition, there are numerical stability issues with
        logistic function methodologies\footnote{For example, in
        Figures~\ref{\SETLABEL:LA1} and~\ref{\SETLABEL:LA2}, if the
        non-linear term, $b$, was greater than zero, it was set to
        zero to produce the graphs. See Section~\ref{\SETLABELREF:LAA}
        for the actual derived values. In other cases, the magnitude
        of $b$ was too large, resulting in a graph that was decreasing
        as a function of time}.  The methodology should be regarded as
        ``fragile.'' It is included for completeness.

        \idx{least squares approximation}
        Figure~\ref{\SETLABEL:LA1} is a graph of the logistic function
        for the time series data presented in
        Figure~\ref{\SETLABEL:TS}. The data presented was made by
        running the program {\it tsdlogistic}\/, which is described
        briefly in Appendix~\ref{programs}, on the parameters
        extracted from the time series data as suggested in
        Figure~\ref{\SETLABEL:TF}. The program {\it tslsq}\/ was used
        to derive the constant and the slope of the normalized
        increments of the data presented in Figure~\ref{\SETLABEL:TF}.
        Figure~\ref{\SETLABEL:LA2} is the same graph, but with the
        time scale expanded by a factor of two.

        \begin{figure}[ht]
            \begin{center}
                \begin{minipage}[t]{0.45\textwidth}
                    \epsfxsize=1.0\linewidth
                    \epsffile{\directory/data.tsfraction.tslsq-p.tsdlogistic.eps}
                    \caption[{\market}, logistic function
                        estimates.]{{\market}, logistic function
                        estimates, provided by running the {\it
                        tslsq}\/ program on the normalized increments
                        presented in Figure~\ref{\SETLABEL:TF} with
                        the -p option. These parameters were used as
                        arguments to the {\it tsdlogistic}\/ program.}
                    \label{\SETLABEL:LA1}
                    \label{\SETLABELQ:LA1}
                \end{minipage}
                \hfill
                \begin{minipage}[t]{0.45\textwidth}
                    \epsfxsize=1.0\linewidth
                    \epsffile{\directory/data.tsfraction.tslsq-p.tsdlogistic2.eps}
                    \caption[{\market}, logistic function
                        estimates.]{{\market}, logistic function
                        estimates of Figure~\ref{\SETLABEL:LA1} with
                        the time scale expanded by a factor of two.}
                    \label{\SETLABEL:LA2}
                    \label{\SETLABELQ:LA2}
                \end{minipage}
            \end{center}
        \end{figure}

% Local Variables:
% TeX-parse-self: t
% TeX-auto-save: t
% TeX-master: "fractal.tex"
% End:


        %
% -----------------------------------------------------------------------------
%
% A license is hereby granted to reproduce this software source code and
% to create executable versions from this source code for personal,
% non-commercial use.  The copyright notice included with the software
% must be maintained in all copies produced.
%
% THIS PROGRAM IS PROVIDED "AS IS". THE AUTHOR PROVIDES NO WARRANTIES
% WHATSOEVER, EXPRESSED OR IMPLIED, INCLUDING WARRANTIES OF
% MERCHANTABILITY, TITLE, OR FITNESS FOR ANY PARTICULAR PURPOSE.  THE
% AUTHOR DOES NOT WARRANT THAT USE OF THIS PROGRAM DOES NOT INFRINGE THE
% INTELLECTUAL PROPERTY RIGHTS OF ANY THIRD PARTY IN ANY COUNTRY.
%
% Copyright (c) 1994-2006, John Conover, All Rights Reserved.
%
% Comments and/or bug reports should be addressed to:
%
%     john@email.johncon.com (John Conover)
%
% -----------------------------------------------------------------------------
%
% Revision: \RCSRevision \\
% Revision Time: \RCSTime UMT \\
% Revision Date: \RCSDate \\
% Revision Id: \RCSId \\
% Revision File: \RCSLog \\
\RCS $Revision: 0.0 $
\RCS $Date: 2006/01/20 04:38:13 $
\RCS $Id: hurst.tex,v 0.0 2006/01/20 04:38:13 john Exp $
% $Log: hurst.tex,v $
% Revision 0.0  2006/01/20 04:38:13  john
% Initial version
%
%
    \subsection{Hurst Coefficient Analysis}
        \label{\SETLABEL:H}

        \subidx{\market}{Hurst coefficient analysis}
        \subidx{Hurst coefficient}{analysis}
        \subidx{increments}{normalized}
        \subidx{normalized}{increments}
        \subidx{programs}{tshurst}
        \subidx{tshurst}{program}
        The data in this section is presented in tabular form in
        Section~\ref{\SETLABELREF:HCHP}. Figure~\ref{\SETLABEL:HC} is
        a graph of the Hurst coefficient data time series data shown
        in Figure~\ref{\SETLABEL:TS}. The slope of the graph is the
        Hurst coefficient.  The data for this figure was produced by
        the program {\it tshurst}\/, which is described briefly in
        Appendix~\ref{programs}.

        \subidx{\market}{H parameter analysis}
        \subidx{H parameter}{analysis}
        \subidx{programs}{tshcalc}
        \subidx{tshcalc}{program}
        Figure~\ref{\SETLABEL:HP} is a graph of the H parameter data
        for the normalized increments of the time series data shown in
        Figure~\ref{\SETLABEL:TF}. The data for this figure was
        produced by the program {\it tshcalc}\/, which is described
        briefly in Appendix~\ref{programs}.

        \begin{figure}[ht]
            \begin{center}
                \begin{minipage}[t]{0.45\textwidth}
                    \epsfxsize=1.0\linewidth
                    \epsffile{\directory/data.tshurst.eps}
                    \caption[{\market}, Hurst coefficient data]{{\market},
                        Hurst coefficient data for the normalized
                        increments of the time series data shown in
                        Figure~\ref{\SETLABEL:TF}.  The slope of the graph
                        is the Hurst coefficient.}
                    \label{\SETLABEL:HC}
                \end{minipage}
                \hfill
                \begin{minipage}[t]{0.45\textwidth}
                    \epsfxsize=1.0\linewidth
                    \epsffile{\directory/data.tshcalc.eps}
                    \caption[{\market}, H parameter data]{{\market}, H
                        parameter data for the normalized increments of
                        the time series data shown in
                        Figure~\ref{\SETLABEL:TF} The slope of the graph
                        is the H parameter.}
                    \label{\SETLABEL:HP}
                \end{minipage}
            \end{center}
        \end{figure}

        \subidx{revenue}{See, rate of revenue returns}
        \subidx{returns}{See, rate of revenue returns}
        \subidx{\market}{revenues}
        \subidx{Hurst coefficient}{analysis}
        \subidx{\market}{Hurst coefficient analysis}
        \subidx{\market}{rate of change}
        \subidx{\market}{windows of opportunity}
        \subidx{rate of revenue returns}{forecast}
        \subidx{forecast}{rate of revenue returns}
        \idx{windows of opportunity}
        \subidx{programs}{tslsq}
        \subidx{tslsq}{program}

        The approximately linear slope of the graph in
        Figure~\ref{\SETLABEL:HC} implies that the variance of the
        rate of revenue returns, (per {\timescale},) in the {\market},
        $V(t_2 - t_1)$, over a period of time is proportional to the
        period of time raised to twice the Hurst
        coefficient~\cite[pp. 180]{Feder},~\cite[pp. 246]{Crownover}.
        This seems to be a quantitative statement concerning how fast,
        and to what degree, the rate of revenue returns' state of
        affairs can change over a period of time.  An additional
        implication, for Hurst coefficients sufficiently close to 0.5,
        is that the probability of the state of affairs repeating
        sometime in the future goes down with increasing
        time\footnote{It can be shown that the number of expected
        market ``high'' and ``low'' transitions, $N$, scales with the
        square root of time, or $N \propto \sqrt {t}$, meaning that
        the cumulative distribution of the probability, $P$, of the
        duration of a market's ``high'' or ``low'' exceeding a given
        time interval, $t$, is proportional to the reciprocal of the
        square root of the time interval, $P \propto 1 / \sqrt {t}$,
        (or, conversely, that the probability of the duration of a
        market's ``high'' or ``low'' exceeding a given time interval
        is proportional to the reciprocal of the time interval raised
        to the power $3 / 2$, ie., $P \propto 1 / t^{3 /
        2}$,~\cite[pp. 153]{Schroeder}. What this means is that a
        histogram of the ``zero free'' run-lengths of a market being
        ``high'' or ``low,'' over a long time, would have a $1 / t^{3
        / 2}$ characteristic.)}, $t$, $p(t) = erf (1/\sqrt{2t})$ which
        is approximately $1/\sqrt{t}$ for $t \gg
        1$~\cite[pp. 160]{Schroeder}. Figures~\ref{\SETLABEL:FN},
        and,~\ref{\SETLABEL:FF} compare methods of approximation of
        the ``forecastability'' of the rate of revenue returns in the
        {\market} for the near term and far term,
        respectively~\cite[pp. 83-84]{Peters:CAOITCM}\footnote{The
        author is not comfortable with Peters' interpretation. For
        example, if the algorithm explained
        in~\cite[pp. 82]{Peters:CAOITCM} is used on ``white noise''
        which, by definition, never has any correlations, the short
        term Hurst coefficient, and thus the ``forecastability,'' is
        still near unity---a bit of an enigma. This can be verified
        with the {\it tswhite}\/ and {\it tshurst}\/ programs, which
        are briefly described in Appendix~\ref{programs}.}.  This
        seems to be a quantitative statement concerning ``windows of
        opportunity'' in the rate of revenue returns, (per
        {\timescale}.)  The program {\it tslsq}\/ was used on the
        Hurst coefficient data, presented in
        Figure~\ref{\SETLABEL:HC}, to provide a least squares
        approximation to the Hurst coefficient. The superimposed least
        squares approximation with on original Hurst coefficient data
        is presented.  The time series data has a Hurst coefficient of
        {\thurstlow}, so that:

        \subidx{\market}{Hurst coefficient analysis}
        \begin{eqnarray}
            V\left(t_2 - t_1\right) & \propto & \left(t_2 - t_1\right)^{2 \cdot H}\\
            V\left(t_2 - t_1\right) & \propto & \left(t_2 - t_1\right)^{2 \cdot {\thurstlow}}\\
                                    & \propto & \left(t_2 - t_1\right)^{\thurstlowtwo}
            \label{\SETLABEL:V}
        \end{eqnarray}

        \subidx{fractional}{Brownian motion}
        \subidx{Brownian motion}{fractional}
        \idx{fractal}
        \noindent where $V(t_2 - t_1)$ is the variance of the
        increments of the rate of revenue returns, (per {\timescale},)
        over the time interval $t_2 -
        t_1$,~\cite[pp. 177]{Feder},~\cite[pp. 494]{Peitgen}. If $H >
        \frac{1}{2}$, then the time series is termed as being
        characterized by ``fractional Brownian
        motion~\cite[pp. 170]{Feder}.''

        \subidx{rate of revenue returns}{predictability}
        \subidx{rate of revenue returns}{forecastability}
        \subidx{rate of revenue returns}{consistency}
        \subidx{predictability}{rate of revenue returns}
        \subidx{forecastability}{rate of revenue returns}
        \subidx{consistency}{rate of revenue returns}
        \subidx{\market}{rate of revenue returns, predictability}
        \subidx{\market}{rate of revenue returns, forecastability}
        \subidx{\market}{rate of revenue returns, consistency}
        \subidx{Hurst coefficient}{analysis}
        \subidx{\market}{Hurst coefficient analysis}
        \subidx{\market}{rate of change}

        In some sense, the Hurst coefficient is a quantitative
        expression of the ``forecastability'' of the future based on
        the past\footnote{Actually, in general, when summing fractal
        entities, the method used should be a root mean square
        process, dependent on the Hurst Coefficient, $H$, where
        $P_{total}^H = P_1^H + P_2^H + \cdots$, where $P_n$ is the
        fractal entities. For a Brownian motion, or random walk type
        of fractal the Hurst Coefficient is a function of time into
        the future. For the ``near term,'' the Hurst coefficient is
        very near unity, meaning the summation process is linear. For
        the ``long term,'' $H \approx 0.5$, or a standard root mean
        square summation process should be used. If $H$ is $0.5$ then
        the market is termed a Brownian motion, or random walk
        process. If it is larger than 0.5, it is termed fractional
        Brownian motion process. For a random walk process, ``near
        term'' and ``far term'' are quantitatively differentiated on
        the Hurst Coefficient graph where $1 - \ln (t) = 0.5 \cdot \ln
        (t)$, or when $\ln (t) = 2$, or $t = 7.389\ldots$ See
        Section~\ref{\SETLABEL:FS} for the particulars on using Hurst
        Coefficient to sum fractal process' for the {\market}. See
        also~\cite[pp. 67, 83-84]{Peters:CAOITCM} and~\cite[pp. 129,
        159]{Schroeder} for particulars on the implications of the
        Hurst Coefficient and root mean square summation issues.}.  A
        Hurst coefficient of {\thurstlow}, (for the near future, and
        {\thurstall} for the distant future.) implies that the
        likelihood of the rate of revenue returns, (per {\timescale},)
        for any two consecutive {\timescale}s being the same is
        {\thurstlowhundred}\%~\cite[pp. 66]{Peters:CAOITCM} for the
        near future, and {\thurstall} for the distant
        future. Likewise, there is a {\thurstlowhundred}\% chance of
        the rate of revenue returns, (per {\timescale},) movements
        being the same in consecutive time periods---ie., if, in a
        given {\timescale}, the rate of revenue returns, (per
        {\timescale},) is increasing, there is a {\thurstlowhundred}\%
        that the rate of revenue returns, (per {\timescale},) will
        increase in the following period, also. In some sense, this is
        a quantitative statement on how ``predictable,'' or
        ``forecastable'' the rate of revenue returns, (per
        {\timescale},) for the {\market} are over time, since the
        probability of having $n$ many consecutive {\timescale}s of
        the same agenda is $H^n$ where $H$ is the Hurst coefficient,
        or, letting the short term probability of having $n$ many
        {\timescale}s of the same market agenda, $p_a$, is:

        \begin{eqnarray}
            p_a\left(n\right) & = & H^{n}\\
                              & = & {\thurstlow}^{n}
            \label{\SETLABEL:MA}
        \end{eqnarray}

        \subidx{rate of revenue returns}{predictability}
        \subidx{rate of revenue returns}{forecastability}
        \subidx{rate of revenue returns}{consistency}
        \subidx{predictability}{rate of revenue returns}
        \subidx{forecastability}{rate of revenue returns}
        \subidx{consistency}{rate of revenue returns}
        As an interesting interpretation of the normalized increments
        of the time series data presented in
        Figure~\ref{\SETLABEL:TF}, if the vertical axis is multiplied
        by 100, to convert to percent, then the graph represents the
        error, in percent, that would be made by forecasting, month by
        month, that the next {\timescale}'s rate of revenue returns
        would be the same as the current {\timescale}'s revenue
        rate. Interestingly, it is $\datafractionmean \cdot 100$
        percent, on the average, with a standard deviation of
        $\datafractionstddev \cdot 100$ percent, and a root mean
        square error value of $\datafractionrms \cdot 100$
        percent---small values for such a simple forecasting
        mechanism.

        \subidx{\market}{rate of revenue returns, range}
        \subidx{Hurst coefficient}{analysis}
        \subidx{\market}{Hurst coefficient analysis}
        \subidx{\market}{rate of change}

        This is, essentially, a statement of the range of values, in
        the increments of the rate of revenue returns, (per
        {\timescale},) that is to be expected over the time interval,
        $t_2 - t_1$,
        $R_v$,~\cite[pp. 178]{Feder},~\cite[pp. 172]{Cambel}:

        \begin{eqnarray}
            R_v\left(t_2 - t_1\right) & \propto & \left(t_2 - t_1\right)^{H}\\
                                      & \propto & \left(t_2 - t_1\right)^{\thurstlow}
            \label{\SETLABEL:R}
        \end{eqnarray}

        \subidx{\market}{rate of revenue returns, range}
        \subidx{Hurst coefficient}{analysis}
        \subidx{\market}{Hurst coefficient analysis}
        \subidx{\market}{rate of change}
        \subidx{Markov}{statistics}
        \subidx{statistics}{Markov}
        \noindent where $R$ is the range of values in the increments
        of the rate of revenue returns, (per {\timescale}.) A Hurst
        coefficient, $H$, that is much larger than $\frac{1}{2}$, (but
        less than 1,) implies a strongly non-Gaussian distribution in
        the increments of the rate of revenue returns, (per
        {\timescale},)~\cite[pp. 152, 194]{Feder}, and a Hurst
        coefficient near $\frac{1}{2}$ implies that the increments of
        the rate of revenue returns, (per {\timescale}) is
        characteristic of an independent
        process~\cite[pp. 195]{Feder}. Extreme caution should be
        exercised in using Markov statistics in any analysis where the
        Hurst coefficient is not
        $\frac{1}{2}$,~\cite[pp. 124]{Crownover},~\cite[pp. 106]{Peters:CAOITCM}.


        As a useful approximation, if $H$, is approximately
        $\frac{1}{2}$, Equation~\ref{\SETLABEL:R} reduces
        to,~\cite[pp. 129]{Schroeder}:

        \begin{eqnarray}
            R\left(t_2 - t_1\right) & \propto & (t_2 - t_1)^{\frac{1}{2}}\\
                                    & \propto & \sqrt{\left(t_2 - t_1\right)}
        \end{eqnarray}

        \subidx{\market}{rate of revenue returns, range}
        \subidx{\market}{rate of revenue returns, increase and decrease}
        \subidx{Hurst coefficient}{analysis}
        \subidx{\market}{Hurst coefficient analysis}
        \subidx{\market}{rate of change}
        \subidx{Markov}{statistics}
        \subidx{statistics}{Markov}

        In the case where the Hurst coefficient, $H$, is
        $\frac{1}{2}$, the range of values in the increments of the
        rate of revenue returns, (per {\timescale},) divided by the
        standard deviation of these values, $S$, can be anticipated to
        increase over time according to the following
        relation,~\cite[pp. 154]{Feder},~\cite[pp. 129]{Schroeder}:

        \begin{equation}
            \frac{R\left(t_2 - t_1\right)}{S} \propto \left(t_2 - t_1\right)^{\frac{1}{2}}
        \end{equation}

        \subidx{\market}{rate of revenue returns, range}
        \subidx{\market}{rate of revenue returns, increase and decrease}
        \subidx{Hurst coefficient}{analysis}
        \subidx{\market}{Hurst coefficient analysis}
        \subidx{\market}{rate of change}
        \noindent which is a useful conceptual approximation, since it
        involves only the square root function---if the range and the
        standard deviation of the increments of the rate of revenue
        returns, (per {\timescale},) are known, (and $H \approx
        \frac{1}{2}$,) then the expected change in $\frac{R}{S}$, will
        increase with the square root of time\footnote{To be precise,
        it is actually asymptotically proportional to
        $\tau^{\frac{1}{2}}$}.

        Another useful approximation when rescaling processes that are
        characterize by Brownian motion, (ie., when $H \approx
        \frac{1}{2}$,) is that:

        \begin{eqnarray}
            X\left(t\right) & \propto & \frac{X\left(rt\right)}{r^{H}}\\
                            & \propto & \frac{X\left(rt\right)}{r^{\thurstlow}}
        \end{eqnarray}

        \idx{Brownian motion}
        \idx{fractal}
        Where $X(t)$ is the process characterized by Brownian motion,
        and $r$ is a scaling factor,~\cite[pp. 494]{Peitgen}.

        \subidx{programs}{tslsq}
        \subidx{tslsq}{program}
        The program {\it tslsq}\/ was used on the H parameter data,
        presented in Figure~\ref{\SETLABEL:HP}, to provide a least
        squares approximation to the H parameter for the
        {\market}. The superimposed least squares approximation on the
        original H parameter data is presented.  By contrast, the H
        parameter, as derived by the methodology outlined
        in~\cite[pp. 249]{Crownover}, is {\thcalclow} for the near
        future, and {\thcalcall} for the distant future.

        \subidx{\market}{Hurst coefficient analysis}
        \subidx{Hurst coefficient}{analysis}
        \subidx{increments}{normalized}
        \subidx{normalized}{increments}
        \subidx{programs}{tshurst}
        \subidx{tshurst}{program}
        \subidx{\market}{H parameter analysis}
        \subidx{H parameter}{analysis}
        \subidx{programs}{tshcalc}
        \subidx{tshcalc}{program}
        Figures~\ref{\SETLABEL:HC} and~\ref{\SETLABEL:HP} represent
        Hurst coefficient and H parameter data that are derived from
        the normalized increments, shown in
        Figure~\ref{\SETLABEL:TF}. In this case, the data is
        considered a normalized derivative of the time series data
        presented in Figure~\ref{\SETLABEL:TF}, instead of a
        cumulative sum.  The program, {\it tshurst}\/, is described
        briefly in appendix~\ref{programs}, and the data for
        figures~\ref{\SETLABEL:THC} and~\ref{\SETLABEL:THP} was made
        using the -d option.

        \begin{figure}[ht]
            \begin{center}
                \begin{minipage}[t]{0.45\textwidth}
                    \epsfxsize=1.0\linewidth
                    \epsffile{\directory/data.tsfraction.tshurst-d.eps}
                    \caption[{\market}, traditional Hurst coefficient
                        data]{{\market}, traditional Hurst coefficient
                        data for the time series data shown in
                        Figure~\ref{\SETLABEL:TS}.  The slope of the
                        graph is the Hurst coefficient, and is
                        {\hurstlow} for the near term, and
                        {\hurstall} for the far term.}
                    \label{\SETLABEL:THC}
                \end{minipage}
                \hfill
                \begin{minipage}[t]{0.45\textwidth}
                    \epsfxsize=1.0\linewidth
                    \epsffile{\directory/data.tsfraction.tshcalc-d.eps}
                    \caption[{\market}, traditional H parameter
                        data]{{\market}, traditional H parameter data
                        for the time series data shown in
                        Figure~\ref{\SETLABEL:TS} The slope of the
                        graph is the H parameter, and is {\hcalclow}
                        for the near term, and {\hcalcall} for the
                        far term.}
                    \label{\SETLABEL:THP}
                \end{minipage}
            \end{center}
        \end{figure}

% Local Variables:
% TeX-parse-self: t
% TeX-auto-save: t
% TeX-master: "fractal.tex"
% End:


        %
% -----------------------------------------------------------------------------
%
% A license is hereby granted to reproduce this software source code and
% to create executable versions from this source code for personal,
% non-commercial use.  The copyright notice included with the software
% must be maintained in all copies produced.
%
% THIS PROGRAM IS PROVIDED "AS IS". THE AUTHOR PROVIDES NO WARRANTIES
% WHATSOEVER, EXPRESSED OR IMPLIED, INCLUDING WARRANTIES OF
% MERCHANTABILITY, TITLE, OR FITNESS FOR ANY PARTICULAR PURPOSE.  THE
% AUTHOR DOES NOT WARRANT THAT USE OF THIS PROGRAM DOES NOT INFRINGE THE
% INTELLECTUAL PROPERTY RIGHTS OF ANY THIRD PARTY IN ANY COUNTRY.
%
% Copyright (c) 1994-2006, John Conover, All Rights Reserved.
%
% Comments and/or bug reports should be addressed to:
%
%     john@email.johncon.com (John Conover)
%
% -----------------------------------------------------------------------------
%
% Revision: \RCSRevision \\
% Revision Time: \RCSTime UMT \\
% Revision Date: \RCSDate \\
% Revision Id: \RCSId \\
% Revision File: \RCSLog \\
\RCS $Revision: 0.0 $
\RCS $Date: 2006/01/20 04:38:13 $
\RCS $Id: fiscal.tex,v 0.0 2006/01/20 04:38:13 john Exp $
% $Log: fiscal.tex,v $
% Revision 0.0  2006/01/20 04:38:13  john
% Initial version
%
%
    \subsection{Fixed Increment Approximation for Fiscal Strategy}
        \label{\SETLABEL:FS}

        \subidx{\market}{fiscal strategy}
        \subidx{markets}{analysis}
        \subidx{analysis}{markets}
        \subidx{strategy}{fiscal}
        \subidx{fiscal}{strategy}
        The data in this section is presented in tabular form in
        Section~\ref{\SETLABELREF:LR}. This section derives various
        values based on the ``average'' of the normalized increments
        presented in Figure~\ref{\SETLABEL:TFA}. These values are an
        approximation to a, probably, complex process with a
        distribution shown in Figure~\ref{\SETLABEL:TF}. These values
        will be used in a fixed increment Brownian fractal analysis
        and simulation of the {\market}, and may, or may not, provide
        adequate accuracy for projections.

        For an organization operating in the {\market}, the fiscal
        strategy, commensurate with the aggregate environment, can be
        derived as follows~\cite[pp. 128, pp
        151]{Schroeder},~\cite[pp. 450]{Reza},~\cite[pp. 270]{Pierce}:
        \vspace{0.15in}

        \subsubsection{Logarithmic Returns}
            \label{\SETLABEL:LR}

            \subidx{logarithmic}{returns}
            \subidx{returns}{logarithmic}
            \subidx{\market}{logarithmic returns}
            The logarithmic returns can be calculated by various
            means. Four will be presented here, for comparison.

            \subidx{programs}{tsnormal}
            \subidx{tsnormal}{program}
            \subidx{logarithmic}{returns}
            \subidx{returns}{logarithmic}
            The logarithmic returns, in bits, $bits$, as computed from
            the mean, by the program {\it tsnormal}\/, which is
            described in Chapter~\ref{programs}, and is presented in
            Figure~\ref{\SETLABEL:TF}, and Equation~\ref{abits} from
            Section~\ref{ereturns} in Chapter~\ref{general}:

            \begin{equation}
                bits = \frac{\ln \left({\datafractionmean} + 1\right)}{\ln \left(2\right)} = \datafractionmeanbits
            \end{equation}

            \subidx{programs}{tslsq}
            \subidx{tslsq}{program}
            \subidx{logarithmic}{returns}
            \subidx{returns}{logarithmic}
            \noindent By comparison, the logarithmic returns, in bits,
            $bits$, as computed from the constant in the least squares
            approximation, using the program {\it tslsq}\/, which is briefly
            described in Chapter~\ref{programs}, as presented in
            Figure~\ref{\SETLABEL:TF}, and Equation~\ref{abits} from
            Section~\ref{ereturns} in Chapter~\ref{general}:

            \begin{equation}
                bits = \frac{\ln \left({\datafractionconstant} + 1\right)}{\ln \left(2\right)} = \datafractionconstantbits
            \end{equation}

            Note that if the mean is not constant in
            Figure~\ref{\SETLABEL:TF}, this method will not provide
            accurate results.

            \subidx{programs}{tslsq}
            \subidx{tslsq}{program}
            \subidx{logarithmic}{returns}
            \subidx{returns}{logarithmic}
            \noindent And by yet another comparison, using the program
            {\it tslsq}\/, which is briefly described in
            Chapter~\ref{programs}, with the -e -p options, to provide
            a formula for the least squares exponential fit to the
            time series data set presented in
            Figure~\ref{\SETLABEL:TS}:

            \begin{equation}
                bits = {\datatslsqepbits}
            \end{equation}

            \subidx{programs}{tslogreturns}
            \subidx{tslogreturns}{program}
            \subidx{logarithmic}{returns}
            \subidx{returns}{logarithmic}
            \noindent And finally, by comparison, from the
            {\it tslogreturns}\/ program, which is briefly described
            in Chapter~\ref{programs}, with the -p option, to provide
            a formula for the logarithmic returns of the time series
            data set presented in Figure~\ref{\SETLABEL:TS}:

            \begin{equation}
                bits = {\logreturns}
            \end{equation}

        \subsubsection{Calculation of Shannon Probability}
            \label{\SETLABEL:SP}

            \subidx{\market}{Shannon probability}
            Ideally, all of the values presented in
            Section~\ref{\SETLABEL:LR} would be equal. Using the
            logarithmic returns provided by the {\it tslogreturns}\/
            program, to be consistent
            with~\cite[pp. 81]{Peters:CAOITCM}

            \subidx{programs}{tslogreturns}
            \subidx{tslogreturns}{program}
            \begin{equation}
                2^{{\logreturns}t}
            \end{equation}

            \noindent therefore:
            \begin{equation}
                C\left(p\right) = {\logreturns}
            \end{equation}
            \subidx{programs}{tsshannon}
            \subidx{tsshannon}{program}
            \subidx{Shannon}{probability}
            \subidx{probability}{Shannon}
            \noindent and, {\it tsshannon}\/ {\logreturns} gives:
            \begin{equation}
                \label{\SETLABEL:F0}
                C\left({\shannonlogreturns}\right) = {\logreturns}
            \end{equation}
            \noindent therefore:
            \begin{eqnarray}
                2^{C\left({\shannonlogreturns}\right)} & = & 2^{\logreturns}\\
                                                       & = & {\twologreturns}\\
                                                       & = & {\twologreturnshundred}\%
            \end{eqnarray}
            \noindent and:
            \begin{eqnarray}
                2p - 1 & = & \left(2 \cdot {\shannonlogreturns}\right) - 1\\
                       & = & {\twopone}\\
                       \label{\SETLABEL:F1}
                       & = & {\twoponehundred}\%
            \end{eqnarray}

            \subidx{\market}{fiscal strategy}
            \subidx{markets}{analysis}
            \subidx{analysis}{markets}
            \subidx{strategy}{fiscal}
            \subidx{fiscal}{strategy}
            \subidx{\market}{fiscal strategy}
            \subidx{\market}{growth rate}
            Presuming the simplified assumptions outlined in
            Section~\ref{assumptions}, the ``typical'' organization
            operating in the {\market} executes a long term fiscal
            strategy, commensurate with the aggregate environment,
            that is to invest, every {\timescale}, in sufficient
            additional resources and infrastructure, to increase the
            manufacturing of goods and services by {\twoponehundred}\%
            of its rate of revenue returns, (per {\timescale}.) As a
            conceptual model, the remaining {\hundredtwoponehundred}\%
            will be held in ``reserve'' with a
            {\shannonlogreturnshundred}\% chance of making twice the
            {\twoponehundred}\% back, (and a
            {\hundredshannonlogreturnshundred}\% chance of making
            0.0,) in one {\timescale}, on the average, for an average
            growth in its rate of revenue returns, (per {\timescale},)
            of {\twologreturnshundred}\%, or a doubling of its rate of
            revenue returns, (per {\timescale},) in
            {\oneoverlogreturns} {\timescale}s.

        \subsubsection{Example Fixed Increment Approximation Fiscal Strategies}

            \subidx{\market}{fiscal strategy}
            \subidx{markets}{analysis}
            \subidx{analysis}{markets}
            \subidx{strategy}{fiscal}
            \subidx{fiscal}{strategy}
            \subidx{\market}{fiscal strategy}
            \subidx{\market}{growth rate}
            \subidx{\market}{management metric}
            \idx{management metric}
            A possible metric on the effectiveness of long term fiscal
            management could possibly be that if an investment of
            {\twoponehundred}\% per {\timescale} of the rate of
            revenue returns, (per {\timescale},) is made in resources
            and infrastructure, then the rate of revenue returns would
            be expected to increase by {\twologreturnshundred}\%, per
            {\timescale}, on average.

            Note that the metrics presented in this section are
            representative of the {\market} as an aggregate whole, and
            may or may not be accurate representations for any
            particular participant in the environment. Of interest to
            the participants in the environment would be a similar
            analysis of each product or service rendered in the
            marketplace.

            \subidx{\market}{fiscal strategy}
            \subidx{markets}{analysis}
            \subidx{analysis}{markets}
            \subidx{strategy}{fiscal}
            \subidx{fiscal}{strategy}
            \subidx{\market}{fiscal strategy}
            As a simple illustrative example, a company operating in
            this environment might obtain a credit line from a bank
            that is equal to {\twoponehundred}\% of its rate of
            revenue returns, (per {\timescale},) to finance additional
            operations. In this simple scenario, the company would use
            its revenue base as collateral for the loan. Some
            {\timescale}s, depending on the {\market}'s environment,
            the company's rate of revenue returns exceeds what was
            borrowed from the bank, and the loan is repaid in
            full. Other {\timescale}s, the company must default, and
            the bank seizes a portion of the company's revenue base to
            pay the delinquent loan. However, on the average, the
            company will expand its rate of revenue returns at
            {\twologreturnshundred}\% per {\timescale}.

            \subidx{\market}{fiscal strategy}
            \subidx{markets}{analysis}
            \subidx{analysis}{markets}
            \subidx{strategy}{fiscal}
            \subidx{fiscal}{strategy}
            \subidx{\market}{fiscal strategy}
            As another simple example, a company re-invests
            {\twoponehundred}\% of its rate of revenue returns, (per
            {\timescale},) in development, marketing, sales, and
            distribution of new products.  Although some products will
            be successful and the return on the investment will exceed
            the {\twoponehundred}\% per {\timescale} investment,
            others will not. However, on the average, the company will
            expand it gross rate of revenue returns at
            {\twologreturnshundred}\% per {\timescale}.

            \subidx{\market}{fiscal strategy}
            \subidx{markets}{analysis}
            \subidx{analysis}{markets}
            \subidx{strategy}{fiscal}
            \subidx{fiscal}{strategy}
            \subidx{\market}{fiscal strategy}
            \subidx{\market}{product portfolio}
            \subidx{\market}{product diversity}
            \subidx{\market}{product mix}
            \subidx{\market}{optimum number of products}
            \idx{product portfolio}
            \idx{product diversity}
            \idx{optimum number of products}
            \idx{product mix}

            As an example of ``product portfolio'' management, suppose
            a company re-invests {\twoponehundred}\% of its rate of
            revenue returns, (per {\timescale},) in development,
            marketing, sales, and distribution of new products.
            Further suppose that the company has two products, and a
            fractal analysis of the individual product rate of revenue
            return time series indicates that one product has a
            Shannon probability of 0.65, and the other has a Shannon
            probability of 0.55. Then the percentage of re-investment
            in the first product would be $(2 \cdot 0.65 - 1) \cdot
            {\twoponehundred}$, percent of the rate of revenue
            returns, and $(2 \cdot 0.55 - 1) \cdot {\twoponehundred}$
            percent for the second product, implying that the company
            should diversify its product line\footnote{The astute
            reader would note that the linear addition was used to add
            the contribution to development of each product. This is a
            ``near term'' interpretation. Actually, in general, the
            method used should be a root mean square process,
            dependent on the Hurst Coefficient, $H$, where
            $P_{total}^H = P_1^H + P_2^H + \cdots$, where $P_n$ is the
            contribution to each individual product. For a Brownian
            motion, or random walk type of fractal the Hurst
            Coefficient is a function of time into the future. For the
            ``near term,'' the Hurst coefficient is very near unity,
            meaning the summation process is linear. For the ``long
            term,'' $H \approx 0.5$, or a standard root mean square
            summation process should be used. If $H$ is $0.5$ then the
            market is termed a Brownian motion, or random walk
            process. If it is larger than 0.5, it is termed fractional
            Brownian motion process. For a random walk process, ``near
            term'' and ``far term'' are quantitatively differentiated
            on the Hurst Coefficient graph where $1 - \ln (t) = 0.5
            \cdot \ln (t)$, or when $\ln (t) = 2$, or $t =
            7.389\ldots$ See~\cite[pp. 67, 83-84]{Peters:CAOITCM}
            and~\cite[pp. 129, 159]{Schroeder} for particulars on the
            implications of the Hurst Coefficient and root mean square
            summation issues.}.  Note that this is a ``bet hedging''
            metric methodology, and assumes that the products have
            uncorrelated revenue return rates. If this re-investment
            methodology is not feasible, perhaps for strategic
            financial reasons, then the re-investment in both products
            should total the ${\twoponehundred}$\%, and the investment
            in each product should be made at a ratio of $\frac{(2
            \cdot 0.65 - 1)}{(2 \cdot 0.55 - 1)} = 3 : 1$,
            respectively. Note that this ``bet hedging'' can be used
            to define the optimal number of products that can be
            supported on the rate of revenue returns. If it assumed
            that all products are ``typical'' for the {\market}, as a
            standard bench mark, then the optimal number will be
            $\frac{1}{{\twopone}}$. Note that this is a
            ``theoretical'' value, since not all products are
            ``typical,'' and there may be strategic reasons, for
            example product leveraging, that may increase the number
            of products above the optimum. However, most of the
            revenue should come from the optimal number of products,
            since having more products will decrease the amount of the
            potential investment in each product, and having less than
            the optimum number of products will increase the risk that
            many of the products could suffer a ``down market''
            concurrently, impacting the rate of revenue returns.  As
            another interesting interpretation of the optimal
            ``hedging of bets,'' in product portfolio strategy, and
            considering the graph of the normalized increments
            presented in Figure~\ref{\SETLABEL:TF}, if the
            organization is running optimally, then these products
            will generate, at least in principle, one standard
            deviation, approximately $0.8413 = 84.13$\% of the future
            growth in rate of revenue returns. Naturally, these are
            approximations, and the values are an approximation to a,
            probably, complex process, and appropriate scrutiny should
            be exercised before making specific projections.  As yet
            another example of ``product portfolio'' management,
            consider the issue of product mix. In this interpretation,
            {\twoponehundred}\% of the product manufactured should be
            ``proprietary,'' while the rest is ``industry standard.''
            As yet another possibility, {\twoponehundred}\% of the
            product manufactured should be predatory into new markets,
            and the remainder in markets that are ``traditional'' for
            the company.

% Local Variables:
% TeX-parse-self: t
% TeX-auto-save: t
% TeX-master: "fractal.tex"
% End:


        %
% -----------------------------------------------------------------------------
%
% A license is hereby granted to reproduce this software source code and
% to create executable versions from this source code for personal,
% non-commercial use.  The copyright notice included with the software
% must be maintained in all copies produced.
%
% THIS PROGRAM IS PROVIDED "AS IS". THE AUTHOR PROVIDES NO WARRANTIES
% WHATSOEVER, EXPRESSED OR IMPLIED, INCLUDING WARRANTIES OF
% MERCHANTABILITY, TITLE, OR FITNESS FOR ANY PARTICULAR PURPOSE.  THE
% AUTHOR DOES NOT WARRANT THAT USE OF THIS PROGRAM DOES NOT INFRINGE THE
% INTELLECTUAL PROPERTY RIGHTS OF ANY THIRD PARTY IN ANY COUNTRY.
%
% Copyright (c) 1994-2006, John Conover, All Rights Reserved.
%
% Comments and/or bug reports should be addressed to:
%
%     john@email.johncon.com (John Conover)
%
% -----------------------------------------------------------------------------
%
% Revision: \RCSRevision \\
% Revision Time: \RCSTime UMT \\
% Revision Date: \RCSDate \\
% Revision Id: \RCSId \\
% Revision File: \RCSLog \\
\RCS $Revision: 0.0 $
\RCS $Date: 2006/01/20 04:38:13 $
\RCS $Id: companies.tex,v 0.0 2006/01/20 04:38:13 john Exp $
% $Log: companies.tex,v $
% Revision 0.0  2006/01/20 04:38:13  john
% Initial version
%
%
    \subsection{Number of Companies}
        \label{\SETLABEL:QNC}

        \subidx{\market}{number of companies}
        \subidx{number of companies}{analysis}
        \subidx{analysis}{number of companies}
        \subidx{Shannon}{probability}
        \subidx{probability}{Shannon}
        This section evaluates the approximate, or ``average,'' number
        of companies in the {\market}, and uses the method outlined in
        Chapter~\ref{general}, Section~\ref{aftsma}. Since the
        average, $avg_{ind}$, and the root mean square, $rms_{ind}$,
        of the normalized increments of the {\market} time series is
        \datafractionmean, and \datafractionrms respectively, the
        number of companies participating in the market can be
        calculated by Equation~\ref{ncompanies} to be {\ncompanies}.

        If this value seems consistent number of companies in the
        {\market}, within the assumptions outlined in
        Chapter~\ref{general}, Section~\ref{aftsma}, then it would
        seem that there is some circumstantial or indirect evidence
        that the companies participating in the {\market} are
        operating optimally, and the ``average'' Shannon probability,
        $P$ for each participating company would be, using
        Equation~\ref{pncompanies}, {\pncompanies}, which would be the
        value which should be used in Section~\ref{\SETLABEL:FS} for
        each participating company if market expansion was to be
        consistent with the rest of the industry. However, if the
        Shannon probability derived in Section~\ref{\SETLABEL:FS} is
        greater than the average Shannon probability for the companies
        participating in the {\market}, as derived in this section,
        then the market would, possibly, be exploitable with the
        fiscal strategy outlined in Section~\ref{\SETLABEL:FS}. The
        maximum exploitability for the {\market} is derived in
        Section~\ref{\SETLABEL:MAXSHANNON}, but it is probably of
        doubtful practicality.

        Note that these optimizations would maximize a company's
        market growth. Since there are probably many companies
        competing in the market place, this would not necessarily
        maximize a company's P\&L, as described in
        Chapter~\ref{general}, Section~\ref{ompl}. The Shannon
        probability that maximizes market share in the {\market} is
        \pncompanies, with several alternative solutions listed in the
        previous paragraph. However, these should be contrasted to the
        Shannon probability that maximizes a company's P\&L which is
        \avgrms~in the {\market}. In all cases, the fraction of the
        P\&L that should be ``wagered'' on the future, $f$, should be:

        \begin{equation}
            f = 2P - 1
        \end{equation}

        \noindent where $P$ is the particular Shannon probability
        chosen optimize a particular fiscal strategy. Interestingly,
        the measured Shannon probability of the {\market} would tend
        to indicate that the companies participating in the market
        have chosen a fiscal strategy that optimizes market growth, as
        opposed to capital growth.

        \subidx{\market}{increasing returns}
        \subidx{economic increasing returns}{\market}
        As interesting interpretation of these exploitive issues,
        since all three fiscal strategies will result in exponential
        market growth for every company participating in the market,
        is that they may represent, perhaps, an example of
        ``increasing returns.''

% Local Variables:
% TeX-parse-self: t
% TeX-auto-save: t
% TeX-master: "fractal.tex"
% End:


        %
% -----------------------------------------------------------------------------
%
% A license is hereby granted to reproduce this software source code and
% to create executable versions from this source code for personal,
% non-commercial use.  The copyright notice included with the software
% must be maintained in all copies produced.
%
% THIS PROGRAM IS PROVIDED "AS IS". THE AUTHOR PROVIDES NO WARRANTIES
% WHATSOEVER, EXPRESSED OR IMPLIED, INCLUDING WARRANTIES OF
% MERCHANTABILITY, TITLE, OR FITNESS FOR ANY PARTICULAR PURPOSE.  THE
% AUTHOR DOES NOT WARRANT THAT USE OF THIS PROGRAM DOES NOT INFRINGE THE
% INTELLECTUAL PROPERTY RIGHTS OF ANY THIRD PARTY IN ANY COUNTRY.
%
% Copyright (c) 1994-2006, John Conover, All Rights Reserved.
%
% Comments and/or bug reports should be addressed to:
%
%     john@email.johncon.com (John Conover)
%
% -----------------------------------------------------------------------------
%
% Revision: \RCSRevision \\
% Revision Time: \RCSTime UMT \\
% Revision Date: \RCSDate \\
% Revision Id: \RCSId \\
% Revision File: \RCSLog \\
\RCS $Revision: 0.0 $
\RCS $Date: 2006/01/20 04:38:13 $
\RCS $Id: operations.tex,v 0.0 2006/01/20 04:38:13 john Exp $
% $Log: operations.tex,v $
% Revision 0.0  2006/01/20 04:38:13  john
% Initial version
%
%
    \subsection{Fixed Increment Approximation for Operational Strategy}
        \label{\SETLABEL:OPS}.

        This section derives various values based on the ``average''
        of the normalized increments presented in
        Figure~\ref{\SETLABEL:TFA}. These values are an approximation
        to a, probably, complex process with a distribution shown in
        Figure~\ref{\SETLABEL:TF}. These values will be used in a
        fixed increment Brownian fractal analysis and simulation of
        the {\market}, and may, or may not, provide adequate accuracy
        for projections.

        \subidx{\market}{fiscal strategy}
        \subidx{\market}{Shannon probability}
        \subidx{strategy}{fiscal}
        \subidx{fiscal}{strategy}
        \subidx{Shannon}{probability}
        \subidx{probability}{Shannon}
        It should be noted that the analysis of fiscal strategy,
        presented in Section~\ref{\SETLABEL:FS}, is derived from the
        {\market} metrics and may, or may not, be maximally
        optimal. For the optimal fiscal strategy, which may be
        exploitable, see Section~\ref{\SETLABEL:MAXSHANNON}.

        \subidx{strategy}{exploitable}
        \subidx{exploitable}{strategy}
        \subidx{\market}{windows of opportunity}
        \idx{windows of opportunity}
        \subidx{decision}{obsolete}
        \subidx{obsolete}{decision}
        \subidx{decision}{timeliness}
        \subidx{timeliness}{decision}
        \subidx{rate of revenue returns}{forecast}
        \subidx{forecast}{rate of revenue returns}
        An additional exploitable strategy may be time itself.
        Equations~\ref{\SETLABEL:V},~\ref{\SETLABEL:R},
        and,~\ref{\SETLABEL:MA}, are, essentially, metrics on how fast
        a decision, which is based on information concerning the
        current status of the {\market}, becomes obsolete. Obviously,
        how long a decision is expected to remain relevant should be
        addressed as an operational necessity in strategic planning
        and project management. Figures~\ref{\SETLABEL:FN},
        and,~\ref{\SETLABEL:FF} compare methods of approximation of
        the ``forecastability'' of rate of revenue returns in the
        {\market} for the near term and far
        term~\cite[pp. 83-84]{Peters:CAOITCM}, respectively. As a
        general rule, caution must be exercised when making decisions
        that will span a time interval larger than the time interval
        where the ``forecastability'' of rate of revenue returns drops
        below 50\%. Beyond this time interval, the chances increase
        that the competitive and market forces will alter the market
        environment in a possibly detrimental unanticipated
        fashion. Obviously, there is significant advantage in
        ``timeliness'' of development, manufacturing, and distribution
        of products and services that are consistent with this
        temporal agenda. Automation of these processes, if executed
        consistently with this agenda, should be considered a
        competitive advantage.

        \subidx{strategy}{exploitable}
        \subidx{exploitable}{strategy}
        \subidx{rate of revenue returns}{forecast}
        \subidx{forecast}{rate of revenue returns}
        \idx{product life cycle}
        \idx{life cycle, product}
        In some sense, this temporal agenda defines the ``average''
        product or service life cycle in the {\market}. When the
        ``forecastability'' of rate of revenue returns drops below
        50\%, there is an even chance that the rate of revenue returns
        for the product or service will change in a detrimental
        fashion. If it is assumed that a product or service life cycle
        consists of a ramp up, a maintenence interval, and a ramp
        down, then, if all three life cycle intervals are equal, the
        product life cycle will be, approximately, three times the
        time interval where the ``forecastability'' of rate of revenue
        returns drops below 50\%. Although probably not an accurate
        prediction of product or service life cycle, the technique may
        be used as a conceptual approximation to the dynamics of
        ``market windows.\footnote{For example, consider the market
        for table salt. Since it has inelastic supply and demand
        curves, and is a necessary requirement for life, it would be
        expected that the Hurst coefficient would be very near
        unity---ignoring competitive pressures in the market. The
        predictability of the table salt market would, therefore, be
        expected to be relatively good, over time.}''  The conceptual
        approximation will probably predict a ``conservative'' or
        ``pessimistic'' value in relation to actual markets.

        \begin{figure}[ht]
            \begin{center}
                \begin{minipage}[t]{0.45\textwidth}
                    \epsfxsize=1.0\linewidth
                    \epsffile{\directory/datahurstlownear.eps}
                    \caption[{\market}, ``forecastability'' of near
                        term rate of revenue returns]{{\market},
                        ``forecastability'' of near term rate of
                        revenue returns. Although the error function
                        is the most accurate, for the near term,
                        $H^{t} = \thurstlow^{t}$ may be used as a
                        reliable metric of ``forecastability'' of the
                        rate of revenue returns.}
                    \label{\SETLABEL:FN}
                \end{minipage}
                \hfill
                \begin{minipage}[t]{0.45\textwidth}
                    \epsfxsize=1.0\linewidth
                    \epsffile{\directory/datahurstlowfar.eps}
                    \caption[{\market}, ``forecastability'' of far
                        term rate of revenue returns]{{\market},
                        ``forecastability'' of far term rate of
                        revenue returns. Although the error function
                        is the most accurate, for the far term,
                        $\frac{1}{\sqrt{t}}$ may be used as a reliable
                        metric of ``forecastability'' of the rate of
                        revenue returns.}
                    \label{\SETLABEL:FF}
                \end{minipage}
            \end{center}
        \end{figure}

        \idx{operations research}
        As an interesting interpretation of the data presented in
        Figure~\ref{\SETLABEL:FN}, there may be, perhaps, some
        applicability to such operational agendas as inventory
        control. Maintaining too little inventory, obviously, will
        create a situation where the organization can not exploit
        market expansion, and maintaining too much inventory,
        likewise, would over extend the company, creating unnecessary
        losses when the market contracts. The company should maintain
        inventory levels that do not exceed, from
        Equation~\ref{\SETLABEL:MA}, ${\thurstlow}^{n} = 0.5$
        {\timescale}s of operations. Since the optimal amount of
        inventory and, from Equation~\ref{\SETLABEL:V}, the variance
        of change in the rate of revenue returns in the future can be
        calculated, there may, perhaps, be some applicability to a
        forecasting methodology that can be incorporated into other
        areas of operations research, for example the linear algebras
        using simplex methodologies for optimization of manufacturing
        processes. Traditionally, these forecasts are made by the
        sales department, and are subject to various subjective
        biases.

% Local Variables:
% TeX-parse-self: t
% TeX-auto-save: t
% TeX-master: "fractal.tex"
% End:


        %
% -----------------------------------------------------------------------------
%
% A license is hereby granted to reproduce this software source code and
% to create executable versions from this source code for personal,
% non-commercial use.  The copyright notice included with the software
% must be maintained in all copies produced.
%
% THIS PROGRAM IS PROVIDED "AS IS". THE AUTHOR PROVIDES NO WARRANTIES
% WHATSOEVER, EXPRESSED OR IMPLIED, INCLUDING WARRANTIES OF
% MERCHANTABILITY, TITLE, OR FITNESS FOR ANY PARTICULAR PURPOSE.  THE
% AUTHOR DOES NOT WARRANT THAT USE OF THIS PROGRAM DOES NOT INFRINGE THE
% INTELLECTUAL PROPERTY RIGHTS OF ANY THIRD PARTY IN ANY COUNTRY.
%
% Copyright (c) 1994-2006, John Conover, All Rights Reserved.
%
% Comments and/or bug reports should be addressed to:
%
%     john@email.johncon.com (John Conover)
%
% -----------------------------------------------------------------------------
%
% Revision: \RCSRevision \\
% Revision Time: \RCSTime UMT \\
% Revision Date: \RCSDate \\
% Revision Id: \RCSId \\
% Revision File: \RCSLog \\
\RCS $Revision: 0.0 $
\RCS $Date: 2006/01/20 04:38:13 $
\RCS $Id: simulation.tex,v 0.0 2006/01/20 04:38:13 john Exp $
% $Log: simulation.tex,v $
% Revision 0.0  2006/01/20 04:38:13  john
% Initial version
%
%
    \subsection{Simulation of Fixed Increment Approximation for Fiscal Strategy}
        \label{\SETLABEL:TSUNFAIRBROWNIAN}

        \subidx{\market}{market simulation}
        The data in this section is presented in tabular form in
        Section~\ref{\SETLABELREF:SIM}.
        Figure~\ref{\SETLABEL:TSUNFAIRBROWNIAN0} represents a
        constructional simulation of the time series data presented in
        Figure~\ref{\SETLABEL:TS}. The program {\it
        tsunfairbrownian}\/, which is briefly described in
        appendix~\ref{programs}, was used in the reconstruction. The
        reconstructed data is superimposed on the original time series
        data.  The program, {\it tsunfairbrownian}\/, essentially,
        constructs the new time series as a Brownian fractal with
        fixed increments---the value of the fixed increment is derived
        from the root mean square average of the normalized increments
        presented in Figure~\ref{\SETLABEL:TF}. The ``quality'' of
        such a reconstruction should be subject to adequate scepticism
        and scrutiny since, in all probability, the normalized
        increments presented in Figure~\ref{\SETLABEL:TF} represent a
        relatively complex process, that may not be ``modeled'' with
        such a simple methodology.

        As a further comparison of the the constructional simulation
        with the original time series data,
        Figure~\ref{\SETLABEL:TSUNFAIRBROWNIAN1} presents a normalized
        histogram of the normalized increments of the reconstructed
        time series, superimposed on the normalized histogram
        presented in Figure~\ref{\SETLABEL:NH}.

        \subidx{\market}{fiscal strategy, simulation}
        \subidx{markets}{simulation}
        \subidx{simulation}{markets}
        \subidx{strategy}{fiscal, simulation}
        \subidx{fiscal}{strategy, simulation}
        \subidx{programs}{tsunfairbrownian}
        \subidx{tsunfairbrownian}{program}
        \begin{figure}[ht]
            \begin{center}
                \begin{minipage}[t]{0.45\textwidth}
                    \epsfxsize=1.0\linewidth
                    \epsffile{\directory/tsunfairbrownian-f.eps}
                    \caption[{\market}, Time series data, empirical and
                        simulated]{{\market}, Time series data, empirical
                        and simulated, using the program {\it tsunfairbrownian}\/
                        with f = {\datafractionrms}. This data is
                        superimposed on the data presented in
                        Figure~\ref{\SETLABEL:TS}.}
                    \label{\SETLABEL:TSUNFAIRBROWNIAN0}
                \end{minipage}
                \hfill
                \begin{minipage}[t]{0.45\textwidth}
                    \epsfxsize=1.0\linewidth
                    \epsffile{\directory/tsunfairbrownian-f.tsfraction.tsnormal-s30.eps}
                    \caption[{\market}, normalized histogram,
                        empirical and simulated]{{\market}, normalized
                        histogram of the normalized increments of the
                        time series data shown in
                        Figure~\ref{\SETLABEL:TSUNFAIRBROWNIAN0},
                        empirical and simulated.  The empirical data
                        has a mean of {\datafractionmean}, with a
                        standard deviation of {\datafractionstddev}.
                        By comparison, the simulated data has a mean
                        of {\tsunfairbrownianfractionmean} with a
                        standard deviation of
                        {\tsunfairbrownianfractionstddev}. This data
                        is superimposed on the data presented in
                        Figure~\ref{\SETLABEL:NH}. The area under the
                        four curves is identical.}
                    \label{\SETLABEL:TSUNFAIRBROWNIAN1}
                \end{minipage}
            \end{center}
        \end{figure}

% Local Variables:
% TeX-parse-self: t
% TeX-auto-save: t
% TeX-master: "fractal.tex"
% End:


        %
% -----------------------------------------------------------------------------
%
% A license is hereby granted to reproduce this software source code and
% to create executable versions from this source code for personal,
% non-commercial use.  The copyright notice included with the software
% must be maintained in all copies produced.
%
% THIS PROGRAM IS PROVIDED "AS IS". THE AUTHOR PROVIDES NO WARRANTIES
% WHATSOEVER, EXPRESSED OR IMPLIED, INCLUDING WARRANTIES OF
% MERCHANTABILITY, TITLE, OR FITNESS FOR ANY PARTICULAR PURPOSE.  THE
% AUTHOR DOES NOT WARRANT THAT USE OF THIS PROGRAM DOES NOT INFRINGE THE
% INTELLECTUAL PROPERTY RIGHTS OF ANY THIRD PARTY IN ANY COUNTRY.
%
% Copyright (c) 1994-2006, John Conover, All Rights Reserved.
%
% Comments and/or bug reports should be addressed to:
%
%     john@email.johncon.com (John Conover)
%
% -----------------------------------------------------------------------------
%
% Revision: \RCSRevision \\
% Revision Time: \RCSTime UMT \\
% Revision Date: \RCSDate \\
% Revision Id: \RCSId \\
% Revision File: \RCSLog \\
\RCS $Revision: 0.0 $
\RCS $Date: 2006/01/20 04:38:13 $
\RCS $Id: maximum.tex,v 0.0 2006/01/20 04:38:13 john Exp $
% $Log: maximum.tex,v $
% Revision 0.0  2006/01/20 04:38:13  john
% Initial version
%
%
    \subsection{Simulation of Fixed Increment Approximation for Optimally Maximal Fiscal Strategy}
        \label{\SETLABEL:MAXSHANNON}
        \subidx{\market}{fiscal strategy, simulation}
        \subidx{\market}{maximum Shannon probability}
        \subidx{markets}{simulation}
        \subidx{simulation}{markets}
        \subidx{strategy}{optimum fiscal, simulation}
        \subidx{fiscal}{optimum strategy, simulation}
        \subidx{programs}{tsunfairbrownian}
        \subidx{tsunfairbrownian}{program}
        \subidx{Shannon}{probability}
        \subidx{probability}{Shannon}

        \subidx{strategy}{exploitable}
        \subidx{exploitable}{strategy}
        \subidx{programs}{tsshannonmax}
        \subidx{tsshannonmax}{program}
        \subidx{programs}{tsunfairbrownian}
        \subidx{tsunfairbrownian}{program}
        \subidx{strategy}{fiscal}
        \subidx{fiscal}{strategy}
        The data in this section is presented in tabular form in
        Section~\ref{\SETLABELREF:MAXSHANNON}. One of the issues of
        analysis, as mentioned in Section~\ref{\SETLABEL:OPS}, is to
        determine the maximum Shannon probability for the time series
        presented in Figure~\ref{\SETLABEL:TS}. Potentially, this
        could be exploited with an aggressive fiscal
        strategy. Figure~\ref{\SETLABEL:SHANNONMAX0} is a graph of the
        output of the {\it tsshannonmax}\/ program, which is described
        briefly in appendix~\ref{programs}. The maximum of this
        function is the maximum Shannon probability for the time
        series data presented in Figure~\ref{\SETLABEL:TS}.
        Figure~\ref{\SETLABEL:SHANNONMAX1} was constructed using {\it
        tsunfairbrownian}\/ program, which is also described in
        appendix~\ref{programs}, with the maximum Shannon probability,
        and the time series data presented in
        Figure~\ref{\SETLABEL:TS}. This represents a ``what if'' the
        investment strategy was changed from a Shannon probability of
        {\shannonlogreturns}, as derived in Section~\ref{\SETLABEL:SP}
        to {\shannonmax}. This process, essentially, extracts the
        random statistical data from the time series presented in
        Figure~\ref{\SETLABEL:TS}, and constructs a new time series,
        using the random statistical data, with a different investment
        strategy.  The program, {\it tsunfairbrownian}\/, essentially,
        constructs the new time series as a Brownian fractal with
        fixed increments.  The ``quality'' of such a reconstruction
        should be subject to adequate scepticism and scrutiny since,
        in all probability, the increments in the original data
        represent a relatively complex process, that may not be
        ``modeled'' with such a simple methodology.

        \begin{figure}[ht]
            \begin{center}
                \begin{minipage}[t]{0.45\textwidth}
                    \epsfxsize=1.0\linewidth
                    \epsffile{\directory/data.tsshannonmax.eps}
                    \caption[{\market}, maximum rate of revenue
                        returns] {{\market}, maximum rate of revenue
                        returns, per {\timescale}, vs. Shannon
                        probability. The maximum rate of revenue
                        returns, per {\timescale}, occurs at a Shannon
                        probability of {\shannonmax}.}
                    \label{\SETLABEL:SHANNONMAX0}
                \end{minipage}
                \hfill
                \begin{minipage}[t]{0.45\textwidth}
                    \epsfxsize=1.0\linewidth
                    \epsffile{\directory/data.tsshannonmax-p.tsunfairbrownian-p.eps}
                    \caption[{\market}, maximum rate of revenue
                        returns] {{\market}, maximum rate of revenue
                        returns, per {\timescale}, at a Shannon
                        probability, of {\shannonmax}, corresponding
                        to a ``wager'' fraction of {\twoponemax}.}
                    \label{\SETLABEL:SHANNONMAX1}
                \end{minipage}
            \end{center}
        \end{figure}

        \subidx{fractional}{Brownian motion}
        \subidx{Brownian motion}{fractional}
        \subidx{Shannon}{probability}
        \subidx{probability}{Shannon}
        \subidx{programs}{tsshannonmax}
        \subidx{tsshannonmax}{program}
        If it is assumed that the time series data set, presented in
        Figure~\ref{\SETLABEL:TS}, constitutes classical Brownian
        motion, then the Shannon probability can be calculated by
        counting the total number of {\timescale}s that the {\market}
        movement was positive, and dividing by the total number of
        {timescale}s represented in the time series. This quotient is
        {\pmax}, as compared with the predicted value from the program
        {\it tsshannonmax}\/ of {\shannonmax}.

% Local Variables:
% TeX-parse-self: t
% TeX-auto-save: t
% TeX-master: "fractal.tex"
% End:


        %
% -----------------------------------------------------------------------------
%
% A license is hereby granted to reproduce this software source code and
% to create executable versions from this source code for personal,
% non-commercial use.  The copyright notice included with the software
% must be maintained in all copies produced.
%
% THIS PROGRAM IS PROVIDED "AS IS". THE AUTHOR PROVIDES NO WARRANTIES
% WHATSOEVER, EXPRESSED OR IMPLIED, INCLUDING WARRANTIES OF
% MERCHANTABILITY, TITLE, OR FITNESS FOR ANY PARTICULAR PURPOSE.  THE
% AUTHOR DOES NOT WARRANT THAT USE OF THIS PROGRAM DOES NOT INFRINGE THE
% INTELLECTUAL PROPERTY RIGHTS OF ANY THIRD PARTY IN ANY COUNTRY.
%
% Copyright (c) 1994-2006, John Conover, All Rights Reserved.
%
% Comments and/or bug reports should be addressed to:
%
%     john@email.johncon.com (John Conover)
%
% -----------------------------------------------------------------------------
%
% Revision: \RCSRevision \\
% Revision Time: \RCSTime UMT \\
% Revision Date: \RCSDate \\
% Revision Id: \RCSId \\
% Revision File: \RCSLog \\
\RCS $Revision: 0.0 $
\RCS $Date: 2006/01/20 04:38:13 $
\RCS $Id: verification.tex,v 0.0 2006/01/20 04:38:13 john Exp $
% $Log: verification.tex,v $
% Revision 0.0  2006/01/20 04:38:13  john
% Initial version
%
%
    \subsection{Qualitative Verification of Fixed Increment Approximation Analysis}
        \label{\SETLABEL:QVA}

        \subidx{\market}{verification of analysis}
        \subidx{verification}{analysis}
        \subidx{analysis}{verification}
        \subidx{quality}{of analysis}
        \subidx{verification}{of methodology}
        \subidx{methodology}{verification of}
        \subidx{Shannon}{probability}
        \subidx{probability}{Shannon}

        This section evaluates various values based on the ``average''
        of the normalized increments presented in
        Figure~\ref{\SETLABEL:TFA}. These values are an approximation
        to a, probably, complex process with a distribution shown in
        Figure~\ref{\SETLABEL:TF}. These values will be used in a
        fixed increment Brownian fractal analysis of the {\market},
        and may, or may not, provide adequate accuracy for
        projections.

        The data in this section is presented in tabular form in
        sections~\ref{\SETLABELREF:VI1} and~\ref{\SETLABELREF:VI2}.
        As a subjective evaluation of the ``quality'' of the analysis
        of the {\market}, from Chapter~\ref{methodology},
        Equation~\ref{metricvalues1}, and using the mean and root mean
        square values of the normalized increments of the time series
        data presented in Figure~\ref{\SETLABEL:TS} from
        Figure~\ref{\SETLABEL:TF}, and the Shannon probability as
        calculated by counting the total number of {\timescale}s that
        the {\market} movement was positive, as presented in
        Section~\ref{\SETLABEL:MAXSHANNON}:

        \begin{eqnarray}
                  P & \approx & \frac{\frac{avg}{rms} + 1}{2}\\
            {\pmax} & \approx & \frac{\frac{\datafractionmean}{\datafractionrms} + 1}{2}\\
            {\pmax} & \approx & {\avgrms}
            \label{\SETLABEL:AVGS}
        \end{eqnarray}

        \subidx{Shannon}{probability}
        \subidx{probability}{Shannon}
        \noindent and comparing these values to the Shannon
        probability, as found by the {\it tsshannonmax}\/ program, which
        iterates for a maximum:

        \begin{eqnarray}
            {\pmax} \approx {\avgrms} \approx {\shannonmax}
        \end{eqnarray}

        \subidx{logarithmic}{returns}
        \subidx{returns}{logarithmic}
        In addition, the different methods of calculating the
        logarithmic returns, presented in Section~\ref{\SETLABEL:FS},
        should be compared. The four methods used were the mean of
        Figure~\ref{\SETLABEL:TF}, the constant in the least squares
        approximation to Figure~\ref{\SETLABEL:TF}, the least squares
        exponential approximation to Figure~\ref{\SETLABEL:TS}, and
        the logarithmic returns of Figure~\ref{\SETLABEL:TS}, derived
        as the mean of the logarithms of the quotients of the
        increments. The values for each of the methods are,
        respectively:

        \begin{equation}
            \datafractionmeanbits \approx \datafractionconstantbits \approx \datatslsqepbits \approx \logreturns
        \end{equation}

        It is implied in Section~\ref{\SETLABEL:FS},
        Subsection~\ref{\SETLABEL:SP} and in
        Section~\ref{\SETLABEL:TSUNFAIRBROWNIAN} that, a Brownian
        motion with fixed increments fractal may ``model'' the
        {\market}. Using Equation~\ref{stddev9} from
        Chapter~\ref{general}, Section~\ref{abmfi}:

        \begin{eqnarray}
                                    rms \left(2P - 1\right) & \approx & \frac{\sigma \left(2P - 1\right)}{2 \sqrt{P\left(1 - P\right)}}\\
            \datafractionrms \left(2 \cdot \pmax - 1\right) & \approx & \frac{\datafractionstddev \left(2 \cdot \pmax - 1\right)}{2\sqrt{\pmax \left(1 - \pmax\right)}}\\
                       \datafractionrms \cdot \twopminusone & \approx & \datafractionstddev \cdot \twopx\\
                                                      \rmsp & \approx & \sigmap
        \end{eqnarray}

        \noindent and, equating to the mean:

        \begin{equation}
            \datafractionmean \approx \rmsp \approx \sigmap
        \end{equation}

        \subidx{Shannon}{probability}
        \subidx{probability}{Shannon}
        \noindent where, as in Equation~\ref{\SETLABEL:AVGS} using the
        mean, root mean square, and standard deviation values of the
        normalized increments of the time series data presented in
        Figure~\ref{\SETLABEL:TS} from Figure~\ref{\SETLABEL:TF}, and
        the Shannon probability as calculated by counting the total
        number of {\timescale}s that the {\market} movement was
        positive, as presented in Section~\ref{\SETLABEL:MAXSHANNON}.

        As a final qualitative comparison, the absolute value of the
        normalized increments should be the same as the root mean
        square value\footnote{The absolute value of the normalized
        increments, when averaged, is related to the root mean square
        of the increments by a constant. If the normalized increments
        are a fixed increment, the constant is unity. If the
        normalized increments have a Gaussian distribution, the
        constant is $\approx 0.8$ depending on the accuracy of of
        ``fit'' to a Gaussian distribution.}, where the absolute value
        is presented in Figure~\ref{\SETLABEL:TFA}, and the root mean
        square value is presented in Figure~\ref{\SETLABEL:TF}:

        \begin{equation}
            \datafractionabsmean \approx \datafractionrms
        \end{equation}

        Note, that if the {\market} could be ``modeled'' as a Brownian
        motion with fixed increments fractal, then the standard
        deviation of the absolute value of the normalized increments
        of the time series data presented in Figure~\ref{\SETLABEL:TS}
        from Figure~\ref{\SETLABEL:TF} should be zero. It is
        $\datafractionabsstddev$.

% Local Variables:
% TeX-parse-self: t
% TeX-auto-save: t
% TeX-master: "fractal.tex"
% End:


    \renewcommand{\market}{Simulated Shannon Probability of 0.6 Game}
    \renewcommand{\directory}{../markets/tsunfairbrownian.exponential}
    \renewcommand{\datafractionmean}{0.008052}
\renewcommand{\datafractionmeanbits}{0.011570}
\renewcommand{\datafractionmeanq}{0.002684}
\renewcommand{\datafractionmeanbitsq}{0.003867}
\renewcommand{\datafractionstddev}{0.038579}
\renewcommand{\datafractionrms}{0.039311}
\renewcommand{\avgrms}{0.602414}
\renewcommand{\ncompanies}{5.210454}
\renewcommand{\pncompanies}{0.544866}
\renewcommand{\datafractionabsmean}{0.029745}
\renewcommand{\datafractionabsstddev}{0.025769}
\renewcommand{\datafractionconstant}{0.010041}
\renewcommand{\datafractionconstantbits}{0.014414}
\renewcommand{\datafractionconstantq}{0.003347}
\renewcommand{\datafractionconstantbitsq}{0.004821}
\renewcommand{\datafractionslope}{-0.000021}
\renewcommand{\datafractionabsconstant}{0.035145}
\renewcommand{\datafractionabsslope}{-0.000057}
\renewcommand{\hurstall}{0.659558}
\renewcommand{\hurstlow}{0.707509}
\renewcommand{\hurstlowtwo}{1.415018}
\renewcommand{\hurstlowhundred}{70.750900}
\renewcommand{\hcalcall}{0.184942}
\renewcommand{\hcalclow}{0.102042}
\renewcommand{\shannonmax}{0.604167}
\renewcommand{\twoponemax}{0.208334}
\renewcommand{\logreturns}{0.010456}
\renewcommand{\twologreturns}{1.007274}
\renewcommand{\twologreturnshundred}{0.727387}
\renewcommand{\oneoverlogreturns}{95.638868}
\renewcommand{\pmax}{0.602094}
\renewcommand{\twopminusone}{0.204188}
\renewcommand{\rmsp}{0.008027}
\renewcommand{\twopx}{0.208583}
\renewcommand{\sigmap}{0.008047}
\renewcommand{\tsunfairbrownianfractionmean}{0.007862}
\renewcommand{\tsunfairbrownianfractionstddev}{0.038619}
\renewcommand{\shannonlogreturns}{0.560125}
\renewcommand{\shannonlogreturnshundred}{56.012500}
\renewcommand{\twopone}{0.120250}
\renewcommand{\twoponehundred}{12.025000}
\renewcommand{\hundredtwoponehundred}{87.975000}
\renewcommand{\hundredshannonlogreturnshundred}{43.987500}
\renewcommand{\datatslsqepbits}{0.007623}
\renewcommand{\thurstall}{0.633980}
\renewcommand{\thurstlow}{0.710108}
\renewcommand{\thurstlowtwo}{1.420216}
\renewcommand{\thurstlowhundred}{71.010800}
\renewcommand{\thcalcall}{0.247886}
\renewcommand{\thcalclow}{0.171737}
\renewcommand{\chisquared}{2.862000}
\renewcommand{\critical}{42.557000}

    \renewcommand{\timescale}{time units}
    \subidx{market}{\market}
    \idx{\market}

    \section{\market}

        \renewcommand{\SETLABEL}{\LABPRE:TSTE}
        \renewcommand{\SETLABELQ}{\LABPRE:TSTEQ}
        \label{\SETLABEL}
        \renewcommand{\SETLABELREF}{\LABPREREF:TSTE}

        \subidx{tscoin}{program}
        \subidx{programs}{tscoin}
        \subidx{tsunfairbrownian}{program}
        \subidx{programs}{tsunfairbrownian}
        \subidx{programs}{tscoin}
        \subidx{tscoin}{program}
        For the analysis, the data was in the directory
        {\directory}\footnote{As a simulation model, the program {\it
        tsunfairbrownian}\/ was run on the time series,
        ``data.original,'' constructed with a text editor, by
        replicating the following fragment 1000 times:

            \vspace{0.1in}
            \noindent\hspace*{1.0in}0.2\\
            \noindent\hspace*{1.0in}-0.2\\
            \noindent\hspace*{1.0in}0.2\\
            \noindent\hspace*{1.0in}-0.2\\
            \noindent\hspace*{1.0in}0.2\\
            \vspace{0.1in}

        \noindent to produce a time series data file of 5000 records
        that ``oscillates,'' on a period of 5, with a Shannon
        probability of 3 / 5 = 0.6. A data file was made by running:

        \vspace{0.1in}
        \noindent tsunfairbrownian -d -i 1.0 -f 0.2 data.original > data
        \vspace{0.1in}

        \noindent since $f = 2P - 1$, where $P = 0.6, f = 0.2$. An $i$
        of $1.0$ was used simulate an exponential beginning with
        $e^{0}$. After running the {\it tsunfairbrownian}\/ program to
        make the data time series, and the program {\it tsfraction}\/,
        the sequence will be:

            \vspace{0.1in}
            \noindent\hspace*{1.0in}0.2\\
            \noindent\hspace*{1.0in}-0.2\\
            \noindent\hspace*{1.0in}0.2\\
            \noindent\hspace*{1.0in}-0.2\\
            \noindent\hspace*{1.0in}0.2\\
            \vspace{0.1in}

        \noindent Note that there are $3 +0.2$'s for every $2 -.2$s in
        $5$ {\timescale}s, for an average of $+0.2 / 5 = 0.04$.  The
        rationale for the numbers, $+0.2$ and $-0.2$, is that it is
        the optimum for a Shannon probability of $P = 0.6$, since $0.2
        = 2P - 1$, (which also equals $P - (1 - P)$,) where $2 \cdot
        0.6 - 1 = 0.2$, which is the optimal amount of the cumulative
        returns to wager with an unfair coin that has a probability of
        $0.6$ of a win, ie., $3$ out of $5$. If the $n-1$'th value in
        the time series is subtracted from the $n$'th value, and the
        value of this subtraction is then divided by the $n-1$'th
        value, then this quotient should be either $+0.2$ or $-0.2$
        depending on the whether the wager was won or lost.

        \noindent Under this scenario, $P = 0.6$, and the returns are:

        \begin{equation}
            2^{0.029049406} = e^{0.020135514}
        \end{equation}

        \subidx{programs}{tsshannon}
        \subidx{tsshannon}{program}
        \noindent which can be verified with the program {\it
        tsshannon}\/, and are consistent
        with~\cite[pp. 128]{Schroeder}.

        Using tsunfairbrownian -f 0.2 will construct an exponential
        data time series that is known to be optimum, ie., a Shannon
        probability of 0.6 with an optimal wager fraction of 0.2, with
        an ``approximate'' Brownian motion noise content-albeit not
        random. For an analytical insight, see
        appendix~\ref{tutorial}, Section~\ref{simple}. It is useful
        for evaluating methodologies.  The data is by {\timescale}.}.

        The data in this section is presented in tabular form in
        Section~\ref{\SETLABELREF}. Note that in this analysis, the
        rate of revenue returns means the increase or decrease in the
        cumulative sum of the {\market}. This is included for
        ``theoretical'' comparative purposes.

        %
% -----------------------------------------------------------------------------
%
% A license is hereby granted to reproduce this software source code and
% to create executable versions from this source code for personal,
% non-commercial use.  The copyright notice included with the software
% must be maintained in all copies produced.
%
% THIS PROGRAM IS PROVIDED "AS IS". THE AUTHOR PROVIDES NO WARRANTIES
% WHATSOEVER, EXPRESSED OR IMPLIED, INCLUDING WARRANTIES OF
% MERCHANTABILITY, TITLE, OR FITNESS FOR ANY PARTICULAR PURPOSE.  THE
% AUTHOR DOES NOT WARRANT THAT USE OF THIS PROGRAM DOES NOT INFRINGE THE
% INTELLECTUAL PROPERTY RIGHTS OF ANY THIRD PARTY IN ANY COUNTRY.
%
% Copyright (c) 1994-2006, John Conover, All Rights Reserved.
%
% Comments and/or bug reports should be addressed to:
%
%     john@email.johncon.com (John Conover)
%
% -----------------------------------------------------------------------------
%
% Revision: \RCSRevision \\
% Revision Time: \RCSTime UMT \\
% Revision Date: \RCSDate \\
% Revision Id: \RCSId \\
% Revision File: \RCSLog \\
\RCS $Revision: 0.0 $
\RCS $Date: 2006/01/20 04:38:13 $
\RCS $Id: fraction.tex,v 0.0 2006/01/20 04:38:13 john Exp $
% $Log: fraction.tex,v $
% Revision 0.0  2006/01/20 04:38:13  john
% Initial version
%
%
    \subsection{Time Series Increments Analysis}
        \label{\SETLABEL:TSA}

        \subidx{\market}{Time series analysis}
        \subidx{time series}{increments}
        \subidx{time series}{analysis}
        \subidx{cumulative sum}{analysis}
        \subidx{analysis}{cumulative sum}
        \subidx{analysis}{random process}
        \subidx{random process}{analysis}
        \subidx{Gaussian}{increments}
        \subidx{increments}{Gaussian}
        \subidx{Brownian}{motion, fractional}
        \subidx{fractional}{Brownian motion}
        \subidx{fractal}{Brownian motion}
        The data in this section is presented in tabular form in
        Section~\ref{\SETLABELREF:TSA}.  Figure~\ref{\SETLABEL:TS} is
        a graph of the time series data for the {\market}.

        \subidx{increments}{normalized}
        \subidx{normalized}{increments}
        \subidx{programs}{tsfraction}
        \subidx{tsfraction}{program}
        Figure~\ref{\SETLABEL:TF} is a graph of the normalized
        increments of the time series data presented in
        Figure~\ref{\SETLABEL:TS}. The data presented was made by
        running the program {\it tsfraction}\/ on the time series
        data. The program {\it tsfraction}\/ is described briefly in
        Appendix~\ref{programs}, and subtracts the previous value from
        the next value, dividing this difference by the previous
        value, for each element in the time series data. The new time
        series contains the instantaneous change in the rate of
        revenue returns, divided by the magnitude of the instantaneous
        rate of revenue returns.

        \subidx{mean}{standard deviation}
        \subidx{standard deviation}{mean}
        \idx{root mean square}
        \idx{least squares approximation}
        \begin{figure}[ht]
            \begin{center}
                \begin{minipage}[t]{0.45\textwidth}
                    \epsfxsize=1.0\linewidth
                    \epsffile{\directory/data.eps}
                    \caption{{\market}, time series data.}
                    \label{\SETLABEL:TS}
                    \label{\SETLABELQ:TS}
                \end{minipage}
                \hfill
                \begin{minipage}[t]{0.45\textwidth}
                    \epsfxsize=1.0\linewidth
                    \epsffile{\directory/data.tsfraction.eps}
                    \caption[{\market}, normalized
                        increments]{{\market}, normalized increments
                        of the time series data presented in
                        Figure~\ref{\SETLABEL:TS}. The mean is
                        {\datafractionmean} with a standard deviation
                        of {\datafractionstddev}. The formula for the
                        least squares approximation is
                        ${\datafractionconstant} +
                        {\datafractionslope}t$, and the root mean
                        squared value is {\datafractionrms}. The
                        graph, labeled ``data\-.tsfraction\-.tsrms,''
                        is the running root mean square, and
                        ``data\-.tsfraction\-.tsavg'' is the running
                        average of the normalized increments.  This
                        graph is the fraction of change in the time
                        series, as a function of time. Note that the
                        slope of the mean, {\datafractionslope}, is
                        the coefficient of the nonlinearity term in
                        the normalized increments. See
                        Chapter~\ref{general}, Section~\ref{nlextend}
                        for a possible application of the logistic
                        function to this data set.}
                    \label{\SETLABEL:TF}
                    \label{\SETLABELQ:TF}
                \end{minipage}
            \end{center}
        \end{figure}

        \subidx{absolute value}{increments}
        \subidx{increments}{absolute value}

        Figure~\ref{\SETLABEL:TFA} is a graph of the absolute value of
        the normalized increments of the time series data presented in
        Figure~\ref{\SETLABEL:TF}. The data presented was made by
        running the Unix utility sed(1) on the normalized increments
        time series data to remove the negative signs. This is an
        absolute value procedure.  The resulting time series contains
        the absolute value of the instantaneous change in the rate of
        revenue returns, divided by the magnitude of the instantaneous
        rate of revenue returns\footnote{The absolute value of the
        normalized increments, when averaged, is related to the root
        mean square of the increments by a constant. If the normalized
        increments are a fixed increment, the constant is unity. If
        the normalized increments have a Gaussian distribution, the
        constant is $\approx 0.8$ depending on the accuracy of of
        ``fit'' to a Gaussian distribution.}.

        \subidx{histogram}{normalized}
        \subidx{normalized}{histogram}
        \subidx{programs}{tsnormal}
        \subidx{tsnormal}{program}
        \subidx{mean}{standard deviation}
        \subidx{standard deviation}{mean}
        \idx{root mean square}
        \idx{least squares approximation}
        \subidx{\market}{analysis of increments}
        Figure~\ref{\SETLABEL:NH} is the normalized histogram of the
        normalized increments of the time series data shown in
        Figure~\ref{\SETLABEL:TF}. The abscissa is 3 $\sigma$ limits,
        and the area under the two curves is identical. The data for
        this figure was produced by the program {\it tsnormal}\/,
        which is described briefly in Appendix~\ref{programs}.

        \begin{figure}[ht]
            \begin{center}
                \begin{minipage}[t]{0.45\textwidth}
                    \epsfxsize=1.0\linewidth
                    \epsffile{\directory/data.tsfraction.abs.eps}
                    \caption[{\market}, absolute value of the
                        normalized increments]{{\market}, absolute
                        value of the normalized increments of the time
                        series data presented in
                        Figure~\ref{\SETLABEL:TF}.  The mean is
                        {\datafractionabsmean} with a standard
                        deviation of {\datafractionabsstddev}. The
                        formula for the least squares approximation is
                        ${\datafractionabsconstant} +
                        {\datafractionabsslope}t$, and the root mean
                        square value, from Figure~\ref{\SETLABEL:TF},
                        is {\datafractionrms}.  The graph, labeled
                        ``data\-.tsfraction\-.tsrms,'' is the running
                        root mean square, and
                        ``data\-.tsfraction\-.tsavg'' is the running
                        average of the normalized increments presented
                        in Figure~\ref{\SETLABEL:TF}, superimposed
                        here for convenience. This graph is the
                        absolute value of the fraction of change in
                        the time series, as a function of time.}
                    \label{\SETLABEL:TFA}
                    \label{\SETLABELQ:TFA}
                \end{minipage}
                \hfill
                \begin{minipage}[t]{0.45\textwidth}
                    \epsfxsize=1.0\linewidth
                    \epsffile{\directory/data.tsfraction.tsnormal-s30.eps}
                    \caption[{\market}, normalized histogram of the
                        normalized increments]{{\market}, normalized
                        histogram of the normalized increments of the
                        time series data shown in
                        Figure~\ref{\SETLABEL:TF}.  The data has a
                        mean of {\datafractionmean}, with a standard
                        deviation of {\datafractionstddev}.  The area
                        under the two curves is identical. The
                        $\chi^2$ value of the observed and expected
                        values of the two curves is {\chisquared},
                        with a critical value of {\critical}.}
                    \label{\SETLABEL:NH}
                \end{minipage}
            \end{center}
        \end{figure}

        \subidx{programs}{tsXsquared}
        \subidx{tsXsquared}{program}
        \subidx{\market}{chi-squared values of increments}
        The program {\it tsXsquared}\/, which is briefly described in
        appendix~\ref{programs}, was used to derive the $\chi^2$
        statistics for the data presented in
        Figure~\ref{\SETLABEL:NH}.

        \subidx{programs}{tsstatest}
        \subidx{tsstatest}{program}
        \subidx{\market}{statistical estimates}

        Figure~\ref{\SETLABEL:SE} is the statistical estimate for the
        data presented in Figure~\ref{\SETLABEL:TF}, as derived by the
        program {\it tsstatest}\/, which is briefly described in
        appendix~\ref{programs}.

        \begin{figure}[ht]
            \begin{center}
                \begin{minipage}[t]{\textwidth}
                    \center{\fbox{\parbox{0.9\textwidth}{\XXX{\directory/data.tsstatest-f0.1-c0.9-i.tex}}}}
                    \caption[{\market}, statistical estimates of the
                        normalized increments]{{\market}, statistical
                        estimates of the normalized increments of the
                        time series shown in Figure~\ref{\SETLABEL:TF}.
                        The table was produced with the {\it
                        tsstatest}\/ program, and illustrates the
                        size of the data set required for a confidence
                        level of 90\%, with an error estimate of $\pm$
                        10\%, or alternately, the error estimate on
                        the time series shown in Figure~\ref{\SETLABEL:TF}.}
                    \label{\SETLABEL:SE}
                \end{minipage}
            \end{center}
        \end{figure}

        Note that the data set size estimations, as produced by the
        {\it tsstatest}\/ program, are probably very conservative,
        depending on the magnitude of the Shannon probability, $P =
        \shannonlogreturns$, as derived in
        Section~\ref{\SETLABEL:SP}. See Chapter~\ref{general},
        Section~\ref{serdss} for possible alternative methodologies
        for addressing the analysis of fractal time series with
        limited data set sizes. Depending on the magnitude of the
        Shannon probability, $P$, these estimates can be several
        orders of magnitude too high.

        \subidx{derivative of increments}{normalized}
        \subidx{normalized}{derivative of increments}
        \subidx{programs}{tsderivative}
        \subidx{tsderivative}{program}
        Figure~\ref{\SETLABEL:TF1} is the normalized histogram of the
        first derivative of the normalized increments of the time
        series data shown in Figure~\ref{\SETLABEL:TF}. In principle,
        if the distribution of the normalized increments presented in
        Figure~\ref{\SETLABEL:NH} is Gaussian in nature, this
        distribution would be similar to ``white noise,'' as presented
        in appendix~\ref{programs}, Figure~\ref{whiteexample}. The
        data was generated by the {\it tsderivative}\/ program, which
        is briefly described in
        appendix~\ref{programs}. Figure~\ref{\SETLABEL:TF2} is the
        normalized histogram of the second derivative of the
        normalized increments of the time series data shown in
        Figure~\ref{\SETLABEL:TF}. In principle, if the distribution
        of the normalized increments presented in
        Figure~\ref{\SETLABEL:NH} is an integrated Gaussian
        distribution in nature, this distribution would be similar to
        ``white noise,'' as presented in appendix~\ref{programs},
        Figure~\ref{whiteexample}.

        \begin{figure}[ht]
            \begin{center}
                \begin{minipage}[t]{0.45\textwidth}
                    \epsfxsize=1.0\linewidth
                    \epsffile{\directory/data.tsfraction.tsderivative.tsnormal-s30.eps}
                    \caption[{\market}, histogram of the first
                        derivative of the increments]{{\market},
                        normalized histogram of the first derivative
                        of the normalized increments of the time
                        series data shown in
                        Figure~\ref{\SETLABEL:TF}.}
                    \label{\SETLABEL:TF1}
                \end{minipage}
                \hfill
                \begin{minipage}[t]{0.45\textwidth}
                    \epsfxsize=1.0\linewidth
                    \epsffile{\directory/data.tsfraction.2tsderivative.tsnormal-s30.eps}
                    \caption[{\market}, histogram of the second
                        derivative of the increments]{{\market},
                        normalized histogram of second derivative of
                        the the normalized increments of the time
                        series data shown in
                        Figure~\ref{\SETLABEL:TF}.}
                    \label{\SETLABEL:TF2}
                \end{minipage}
            \end{center}
        \end{figure}

        \subidx{fractal}{range}
        \subidx{fractal}{R/S analysis}
        \subidx{\market}{rate of revenue returns, range}
        \subidx{\market}{deterministic mechanism}
        \subidx{deterministic}{mechanism}
        \subidx{mechanism}{deterministic}
        Figure~\ref{\SETLABEL:TR} is the range of values of the time
        series shown in Figure~\ref{\SETLABEL:TS}. The horizontal axis
        is time into the future. In principle, if the time series was
        characterized as fractional Brownian motion the graph in
        Figure~\ref{\SETLABEL:TR} would be a square root
        function\footnote{Note that the ``roughness,'' or ``sawtooth''
        characteristics of the graph in Figure~\ref{\SETLABEL:TR} are
        a computational artifact---caused by not using the -m option
        to the program {\it tshurst}\/, which is computationally
        inefficient.}. Figure~\ref{\SETLABEL:TD} is the deterministic
        map of the normalized increments of the time series data shown
        in Figure~\ref{\SETLABEL:TF}. The deterministic map is useful
        for determining if a time series was created by a
        deterministic mechanism. This, essentially, maps each element
        in the time series with the previous element in the time
        series.  See,~\cite[pp. 745]{Peitgen}.

        \begin{figure}[ht]
            \begin{center}
                \begin{minipage}[t]{0.45\textwidth}
                    \epsfxsize=1.0\linewidth
                    \epsffile{\directory/data.tshurst-f.eps}
                    \caption[{\market}, range]{{\market}, range of the
                        time series data shown in
                        Figure~\ref{\SETLABEL:TS}.}
                    \label{\SETLABEL:TR}
                \end{minipage}
                \hfill
                \begin{minipage}[t]{0.45\textwidth}
                    \epsfxsize=1.0\linewidth
                    \epsffile{\directory/data.tsfraction.tsdeterministic.eps}
                    \caption[{\market}, deterministic map]{{\market},
                        deterministic map of the normalized increments
                        of the time series data shown in
                        Figure~\ref{\SETLABEL:TF}.}
                    \label{\SETLABEL:TD}
                \end{minipage}
            \end{center}
        \end{figure}

% Local Variables:
% TeX-parse-self: t
% TeX-auto-save: t
% TeX-master: "fractal.tex"
% End:


        %
% -----------------------------------------------------------------------------
%
% A license is hereby granted to reproduce this software source code and
% to create executable versions from this source code for personal,
% non-commercial use.  The copyright notice included with the software
% must be maintained in all copies produced.
%
% THIS PROGRAM IS PROVIDED "AS IS". THE AUTHOR PROVIDES NO WARRANTIES
% WHATSOEVER, EXPRESSED OR IMPLIED, INCLUDING WARRANTIES OF
% MERCHANTABILITY, TITLE, OR FITNESS FOR ANY PARTICULAR PURPOSE.  THE
% AUTHOR DOES NOT WARRANT THAT USE OF THIS PROGRAM DOES NOT INFRINGE THE
% INTELLECTUAL PROPERTY RIGHTS OF ANY THIRD PARTY IN ANY COUNTRY.
%
% Copyright (c) 1994-2006, John Conover, All Rights Reserved.
%
% Comments and/or bug reports should be addressed to:
%
%     john@email.johncon.com (John Conover)
%
% -----------------------------------------------------------------------------
%
% Revision: \RCSRevision \\
% Revision Time: \RCSTime UMT \\
% Revision Date: \RCSDate \\
% Revision Id: \RCSId \\
% Revision File: \RCSLog \\
\RCS $Revision: 0.0 $
\RCS $Date: 2006/01/20 04:38:13 $
\RCS $Id: instant.tex,v 0.0 2006/01/20 04:38:13 john Exp $
% $Log: instant.tex,v $
% Revision 0.0  2006/01/20 04:38:13  john
% Initial version
%
%
    \subsection{Instantaneous Analysis of Normalized Increments}
        \label{\SETLABEL:IA}

        \subidx{\market}{instantaneous analysis of normalized increments}
        \idx{average of normalized increments}
        \idx{root mean square of normalized increments}
        \subidx{Shannon probability}{instantaneous computation of}
        \subidx{average of normalized increments}{instantaneous computation of}
        \subidx{root mean square of normalized increments}{instantaneous computation of}
        \subidx{instantaneous computation}{Shannon probability}
        \subidx{instantaneous computation}{average of normalized increments}
        \subidx{instantaneous computation}{root mean square of normalized increments}
        \idx{time series}
        \subidx{time series}{instantaneous analysis}
        \subidx{instantaneous analysis}{time series}
        \subidx{time series}{increments}
        \subidx{time series}{analysis}
        \subidx{Shannon}{probability}
        \subidx{probability}{Shannon}
        \subidx{normalized}{increments}
        \subidx{increments}{normalized}

        The program {\it tsinstant}\/, which is briefly described in
        Appendix~\ref{programs}, is for finding the instantaneous
        fraction of change in a time series. The value of a sample in
        the time series is subtracted from the previous sample in the
        time series, and divided by the value of the previous sample.
        As explained in Chapter~\ref{general},
        Sections~\ref{derivation},~\ref{GA},~\ref{abmfi},~\ref{aftsma}
        and,~\ref{ompl} for Brownian motion, random walk fractals, the
        absolute value of the instantaneous fraction of change is also
        the root mean square of the instantaneous fraction of
        change\footnote{The absolute value of the normalized
        increments, when averaged, is related to the root mean square
        of the increments by a constant. If the normalized increments
        are a fixed increment, the constant is unity. If the
        normalized increments have a Gaussian distribution, the
        constant is $\approx 0.8$ depending on the accuracy of of
        ``fit'' to a Gaussian distribution.}. Squaring this value is
        the average of the instantaneous fraction of change, and
        adding unity to the absolute value of the instantaneous
        fraction of change, and dividing by two, is the Shannon
        probability of the instantaneous fraction of change.

        Figure~\ref{\SETLABEL:IA1} is the instantaneous value of the
        root mean square of the normalized increments for the
        {\market}, and Figure~\ref{\SETLABEL:IA2} is the instantaneous
        Shannon probability for the normalized increments.

        \begin{figure}[ht]
            \begin{center}
                \begin{minipage}[t]{0.45\textwidth}
                    \epsfxsize=1.0\linewidth
                    \epsffile{\directory/data.tsinstant-r.eps}
                    \caption[{\market}, instantaneous value of
                        rms.]{{\market}, instantaneous value of the
                        root mean square of the normalized increments,
                        provided by running the program {\it
                        tsinstant}\/ with the -r option on the data
                        presented in Figure~\ref{\SETLABEL:TS}.}
                    \label{\SETLABEL:IA1}
                    \label{\SETLABELQ:IA1}
                \end{minipage}
                \hfill
                \begin{minipage}[t]{0.45\textwidth}
                    \epsfxsize=1.0\linewidth
                    \epsffile{\directory/data.tsinstant-s.eps}
                    \caption[{\market}, instantaneous value of
                        Shannon probability.]{{\market}, instantaneous
                        value of the Shannon probability of the
                        normalized increments, provided by running the
                        program {\it tsinstant}\/ with the -s option
                        on the data presented in
                        Figure~\ref{\SETLABEL:TS}.}
                    \label{\SETLABEL:IA2}
                    \label{\SETLABELQ:IA2}
                \end{minipage}
            \end{center}
        \end{figure}

% Local Variables:
% TeX-parse-self: t
% TeX-auto-save: t
% TeX-master: "fractal.tex"
% End:


        %
% -----------------------------------------------------------------------------
%
% A license is hereby granted to reproduce this software source code and
% to create executable versions from this source code for personal,
% non-commercial use.  The copyright notice included with the software
% must be maintained in all copies produced.
%
% THIS PROGRAM IS PROVIDED "AS IS". THE AUTHOR PROVIDES NO WARRANTIES
% WHATSOEVER, EXPRESSED OR IMPLIED, INCLUDING WARRANTIES OF
% MERCHANTABILITY, TITLE, OR FITNESS FOR ANY PARTICULAR PURPOSE.  THE
% AUTHOR DOES NOT WARRANT THAT USE OF THIS PROGRAM DOES NOT INFRINGE THE
% INTELLECTUAL PROPERTY RIGHTS OF ANY THIRD PARTY IN ANY COUNTRY.
%
% Copyright (c) 1994-2006, John Conover, All Rights Reserved.
%
% Comments and/or bug reports should be addressed to:
%
%     john@email.johncon.com (John Conover)
%
% -----------------------------------------------------------------------------
%
% Revision: \RCSRevision \\
% Revision Time: \RCSTime UMT \\
% Revision Date: \RCSDate \\
% Revision Id: \RCSId \\
% Revision File: \RCSLog \\
\RCS $Revision: 0.0 $
\RCS $Date: 2006/01/20 04:38:13 $
\RCS $Id: logistic.tex,v 0.0 2006/01/20 04:38:13 john Exp $
% $Log: logistic.tex,v $
% Revision 0.0  2006/01/20 04:38:13  john
% Initial version
%
%
    \subsection{Logistic Analysis}
        \label{\SETLABEL:LA}

        \subidx{\market}{Logistic function analysis}
        \subidx{time series}{logistic function}
        \subidx{logistic function}{time series}
        \subidx{time series}{increments}
        \subidx{time series}{analysis}
        \subidx{cumulative sum}{analysis}
        \subidx{analysis}{cumulative sum}
        \subidx{analysis}{random process}
        \subidx{random process}{analysis}
        The data in this section is presented in tabular form in
        Section~\ref{\SETLABELREF:LAA}.  Figure~\ref{\SETLABEL:LA1} is
        a graph of the logistic function estimates of the time series
        data for the {\market}. The reader is cautioned that these
        graphs are constructed using the method suggested in
        Chapter~\ref{general}, Section~\ref{nlextend} and enormous
        precision is required for adequate prediction of the logistic
        function,~\cite{Modis}. Particularly, the non-linear term will
        usually require intervention to produce a practical fit to the
        data. In addition, there are numerical stability issues with
        logistic function methodologies\footnote{For example, in
        Figures~\ref{\SETLABEL:LA1} and~\ref{\SETLABEL:LA2}, if the
        non-linear term, $b$, was greater than zero, it was set to
        zero to produce the graphs. See Section~\ref{\SETLABELREF:LAA}
        for the actual derived values. In other cases, the magnitude
        of $b$ was too large, resulting in a graph that was decreasing
        as a function of time}.  The methodology should be regarded as
        ``fragile.'' It is included for completeness.

        \idx{least squares approximation}
        Figure~\ref{\SETLABEL:LA1} is a graph of the logistic function
        for the time series data presented in
        Figure~\ref{\SETLABEL:TS}. The data presented was made by
        running the program {\it tsdlogistic}\/, which is described
        briefly in Appendix~\ref{programs}, on the parameters
        extracted from the time series data as suggested in
        Figure~\ref{\SETLABEL:TF}. The program {\it tslsq}\/ was used
        to derive the constant and the slope of the normalized
        increments of the data presented in Figure~\ref{\SETLABEL:TF}.
        Figure~\ref{\SETLABEL:LA2} is the same graph, but with the
        time scale expanded by a factor of two.

        \begin{figure}[ht]
            \begin{center}
                \begin{minipage}[t]{0.45\textwidth}
                    \epsfxsize=1.0\linewidth
                    \epsffile{\directory/data.tsfraction.tslsq-p.tsdlogistic.eps}
                    \caption[{\market}, logistic function
                        estimates.]{{\market}, logistic function
                        estimates, provided by running the {\it
                        tslsq}\/ program on the normalized increments
                        presented in Figure~\ref{\SETLABEL:TF} with
                        the -p option. These parameters were used as
                        arguments to the {\it tsdlogistic}\/ program.}
                    \label{\SETLABEL:LA1}
                    \label{\SETLABELQ:LA1}
                \end{minipage}
                \hfill
                \begin{minipage}[t]{0.45\textwidth}
                    \epsfxsize=1.0\linewidth
                    \epsffile{\directory/data.tsfraction.tslsq-p.tsdlogistic2.eps}
                    \caption[{\market}, logistic function
                        estimates.]{{\market}, logistic function
                        estimates of Figure~\ref{\SETLABEL:LA1} with
                        the time scale expanded by a factor of two.}
                    \label{\SETLABEL:LA2}
                    \label{\SETLABELQ:LA2}
                \end{minipage}
            \end{center}
        \end{figure}

% Local Variables:
% TeX-parse-self: t
% TeX-auto-save: t
% TeX-master: "fractal.tex"
% End:


        %
% -----------------------------------------------------------------------------
%
% A license is hereby granted to reproduce this software source code and
% to create executable versions from this source code for personal,
% non-commercial use.  The copyright notice included with the software
% must be maintained in all copies produced.
%
% THIS PROGRAM IS PROVIDED "AS IS". THE AUTHOR PROVIDES NO WARRANTIES
% WHATSOEVER, EXPRESSED OR IMPLIED, INCLUDING WARRANTIES OF
% MERCHANTABILITY, TITLE, OR FITNESS FOR ANY PARTICULAR PURPOSE.  THE
% AUTHOR DOES NOT WARRANT THAT USE OF THIS PROGRAM DOES NOT INFRINGE THE
% INTELLECTUAL PROPERTY RIGHTS OF ANY THIRD PARTY IN ANY COUNTRY.
%
% Copyright (c) 1994-2006, John Conover, All Rights Reserved.
%
% Comments and/or bug reports should be addressed to:
%
%     john@email.johncon.com (John Conover)
%
% -----------------------------------------------------------------------------
%
% Revision: \RCSRevision \\
% Revision Time: \RCSTime UMT \\
% Revision Date: \RCSDate \\
% Revision Id: \RCSId \\
% Revision File: \RCSLog \\
\RCS $Revision: 0.0 $
\RCS $Date: 2006/01/20 04:38:13 $
\RCS $Id: hurst.tex,v 0.0 2006/01/20 04:38:13 john Exp $
% $Log: hurst.tex,v $
% Revision 0.0  2006/01/20 04:38:13  john
% Initial version
%
%
    \subsection{Hurst Coefficient Analysis}
        \label{\SETLABEL:H}

        \subidx{\market}{Hurst coefficient analysis}
        \subidx{Hurst coefficient}{analysis}
        \subidx{increments}{normalized}
        \subidx{normalized}{increments}
        \subidx{programs}{tshurst}
        \subidx{tshurst}{program}
        The data in this section is presented in tabular form in
        Section~\ref{\SETLABELREF:HCHP}. Figure~\ref{\SETLABEL:HC} is
        a graph of the Hurst coefficient data time series data shown
        in Figure~\ref{\SETLABEL:TS}. The slope of the graph is the
        Hurst coefficient.  The data for this figure was produced by
        the program {\it tshurst}\/, which is described briefly in
        Appendix~\ref{programs}.

        \subidx{\market}{H parameter analysis}
        \subidx{H parameter}{analysis}
        \subidx{programs}{tshcalc}
        \subidx{tshcalc}{program}
        Figure~\ref{\SETLABEL:HP} is a graph of the H parameter data
        for the normalized increments of the time series data shown in
        Figure~\ref{\SETLABEL:TF}. The data for this figure was
        produced by the program {\it tshcalc}\/, which is described
        briefly in Appendix~\ref{programs}.

        \begin{figure}[ht]
            \begin{center}
                \begin{minipage}[t]{0.45\textwidth}
                    \epsfxsize=1.0\linewidth
                    \epsffile{\directory/data.tshurst.eps}
                    \caption[{\market}, Hurst coefficient data]{{\market},
                        Hurst coefficient data for the normalized
                        increments of the time series data shown in
                        Figure~\ref{\SETLABEL:TF}.  The slope of the graph
                        is the Hurst coefficient.}
                    \label{\SETLABEL:HC}
                \end{minipage}
                \hfill
                \begin{minipage}[t]{0.45\textwidth}
                    \epsfxsize=1.0\linewidth
                    \epsffile{\directory/data.tshcalc.eps}
                    \caption[{\market}, H parameter data]{{\market}, H
                        parameter data for the normalized increments of
                        the time series data shown in
                        Figure~\ref{\SETLABEL:TF} The slope of the graph
                        is the H parameter.}
                    \label{\SETLABEL:HP}
                \end{minipage}
            \end{center}
        \end{figure}

        \subidx{revenue}{See, rate of revenue returns}
        \subidx{returns}{See, rate of revenue returns}
        \subidx{\market}{revenues}
        \subidx{Hurst coefficient}{analysis}
        \subidx{\market}{Hurst coefficient analysis}
        \subidx{\market}{rate of change}
        \subidx{\market}{windows of opportunity}
        \subidx{rate of revenue returns}{forecast}
        \subidx{forecast}{rate of revenue returns}
        \idx{windows of opportunity}
        \subidx{programs}{tslsq}
        \subidx{tslsq}{program}

        The approximately linear slope of the graph in
        Figure~\ref{\SETLABEL:HC} implies that the variance of the
        rate of revenue returns, (per {\timescale},) in the {\market},
        $V(t_2 - t_1)$, over a period of time is proportional to the
        period of time raised to twice the Hurst
        coefficient~\cite[pp. 180]{Feder},~\cite[pp. 246]{Crownover}.
        This seems to be a quantitative statement concerning how fast,
        and to what degree, the rate of revenue returns' state of
        affairs can change over a period of time.  An additional
        implication, for Hurst coefficients sufficiently close to 0.5,
        is that the probability of the state of affairs repeating
        sometime in the future goes down with increasing
        time\footnote{It can be shown that the number of expected
        market ``high'' and ``low'' transitions, $N$, scales with the
        square root of time, or $N \propto \sqrt {t}$, meaning that
        the cumulative distribution of the probability, $P$, of the
        duration of a market's ``high'' or ``low'' exceeding a given
        time interval, $t$, is proportional to the reciprocal of the
        square root of the time interval, $P \propto 1 / \sqrt {t}$,
        (or, conversely, that the probability of the duration of a
        market's ``high'' or ``low'' exceeding a given time interval
        is proportional to the reciprocal of the time interval raised
        to the power $3 / 2$, ie., $P \propto 1 / t^{3 /
        2}$,~\cite[pp. 153]{Schroeder}. What this means is that a
        histogram of the ``zero free'' run-lengths of a market being
        ``high'' or ``low,'' over a long time, would have a $1 / t^{3
        / 2}$ characteristic.)}, $t$, $p(t) = erf (1/\sqrt{2t})$ which
        is approximately $1/\sqrt{t}$ for $t \gg
        1$~\cite[pp. 160]{Schroeder}. Figures~\ref{\SETLABEL:FN},
        and,~\ref{\SETLABEL:FF} compare methods of approximation of
        the ``forecastability'' of the rate of revenue returns in the
        {\market} for the near term and far term,
        respectively~\cite[pp. 83-84]{Peters:CAOITCM}\footnote{The
        author is not comfortable with Peters' interpretation. For
        example, if the algorithm explained
        in~\cite[pp. 82]{Peters:CAOITCM} is used on ``white noise''
        which, by definition, never has any correlations, the short
        term Hurst coefficient, and thus the ``forecastability,'' is
        still near unity---a bit of an enigma. This can be verified
        with the {\it tswhite}\/ and {\it tshurst}\/ programs, which
        are briefly described in Appendix~\ref{programs}.}.  This
        seems to be a quantitative statement concerning ``windows of
        opportunity'' in the rate of revenue returns, (per
        {\timescale}.)  The program {\it tslsq}\/ was used on the
        Hurst coefficient data, presented in
        Figure~\ref{\SETLABEL:HC}, to provide a least squares
        approximation to the Hurst coefficient. The superimposed least
        squares approximation with on original Hurst coefficient data
        is presented.  The time series data has a Hurst coefficient of
        {\thurstlow}, so that:

        \subidx{\market}{Hurst coefficient analysis}
        \begin{eqnarray}
            V\left(t_2 - t_1\right) & \propto & \left(t_2 - t_1\right)^{2 \cdot H}\\
            V\left(t_2 - t_1\right) & \propto & \left(t_2 - t_1\right)^{2 \cdot {\thurstlow}}\\
                                    & \propto & \left(t_2 - t_1\right)^{\thurstlowtwo}
            \label{\SETLABEL:V}
        \end{eqnarray}

        \subidx{fractional}{Brownian motion}
        \subidx{Brownian motion}{fractional}
        \idx{fractal}
        \noindent where $V(t_2 - t_1)$ is the variance of the
        increments of the rate of revenue returns, (per {\timescale},)
        over the time interval $t_2 -
        t_1$,~\cite[pp. 177]{Feder},~\cite[pp. 494]{Peitgen}. If $H >
        \frac{1}{2}$, then the time series is termed as being
        characterized by ``fractional Brownian
        motion~\cite[pp. 170]{Feder}.''

        \subidx{rate of revenue returns}{predictability}
        \subidx{rate of revenue returns}{forecastability}
        \subidx{rate of revenue returns}{consistency}
        \subidx{predictability}{rate of revenue returns}
        \subidx{forecastability}{rate of revenue returns}
        \subidx{consistency}{rate of revenue returns}
        \subidx{\market}{rate of revenue returns, predictability}
        \subidx{\market}{rate of revenue returns, forecastability}
        \subidx{\market}{rate of revenue returns, consistency}
        \subidx{Hurst coefficient}{analysis}
        \subidx{\market}{Hurst coefficient analysis}
        \subidx{\market}{rate of change}

        In some sense, the Hurst coefficient is a quantitative
        expression of the ``forecastability'' of the future based on
        the past\footnote{Actually, in general, when summing fractal
        entities, the method used should be a root mean square
        process, dependent on the Hurst Coefficient, $H$, where
        $P_{total}^H = P_1^H + P_2^H + \cdots$, where $P_n$ is the
        fractal entities. For a Brownian motion, or random walk type
        of fractal the Hurst Coefficient is a function of time into
        the future. For the ``near term,'' the Hurst coefficient is
        very near unity, meaning the summation process is linear. For
        the ``long term,'' $H \approx 0.5$, or a standard root mean
        square summation process should be used. If $H$ is $0.5$ then
        the market is termed a Brownian motion, or random walk
        process. If it is larger than 0.5, it is termed fractional
        Brownian motion process. For a random walk process, ``near
        term'' and ``far term'' are quantitatively differentiated on
        the Hurst Coefficient graph where $1 - \ln (t) = 0.5 \cdot \ln
        (t)$, or when $\ln (t) = 2$, or $t = 7.389\ldots$ See
        Section~\ref{\SETLABEL:FS} for the particulars on using Hurst
        Coefficient to sum fractal process' for the {\market}. See
        also~\cite[pp. 67, 83-84]{Peters:CAOITCM} and~\cite[pp. 129,
        159]{Schroeder} for particulars on the implications of the
        Hurst Coefficient and root mean square summation issues.}.  A
        Hurst coefficient of {\thurstlow}, (for the near future, and
        {\thurstall} for the distant future.) implies that the
        likelihood of the rate of revenue returns, (per {\timescale},)
        for any two consecutive {\timescale}s being the same is
        {\thurstlowhundred}\%~\cite[pp. 66]{Peters:CAOITCM} for the
        near future, and {\thurstall} for the distant
        future. Likewise, there is a {\thurstlowhundred}\% chance of
        the rate of revenue returns, (per {\timescale},) movements
        being the same in consecutive time periods---ie., if, in a
        given {\timescale}, the rate of revenue returns, (per
        {\timescale},) is increasing, there is a {\thurstlowhundred}\%
        that the rate of revenue returns, (per {\timescale},) will
        increase in the following period, also. In some sense, this is
        a quantitative statement on how ``predictable,'' or
        ``forecastable'' the rate of revenue returns, (per
        {\timescale},) for the {\market} are over time, since the
        probability of having $n$ many consecutive {\timescale}s of
        the same agenda is $H^n$ where $H$ is the Hurst coefficient,
        or, letting the short term probability of having $n$ many
        {\timescale}s of the same market agenda, $p_a$, is:

        \begin{eqnarray}
            p_a\left(n\right) & = & H^{n}\\
                              & = & {\thurstlow}^{n}
            \label{\SETLABEL:MA}
        \end{eqnarray}

        \subidx{rate of revenue returns}{predictability}
        \subidx{rate of revenue returns}{forecastability}
        \subidx{rate of revenue returns}{consistency}
        \subidx{predictability}{rate of revenue returns}
        \subidx{forecastability}{rate of revenue returns}
        \subidx{consistency}{rate of revenue returns}
        As an interesting interpretation of the normalized increments
        of the time series data presented in
        Figure~\ref{\SETLABEL:TF}, if the vertical axis is multiplied
        by 100, to convert to percent, then the graph represents the
        error, in percent, that would be made by forecasting, month by
        month, that the next {\timescale}'s rate of revenue returns
        would be the same as the current {\timescale}'s revenue
        rate. Interestingly, it is $\datafractionmean \cdot 100$
        percent, on the average, with a standard deviation of
        $\datafractionstddev \cdot 100$ percent, and a root mean
        square error value of $\datafractionrms \cdot 100$
        percent---small values for such a simple forecasting
        mechanism.

        \subidx{\market}{rate of revenue returns, range}
        \subidx{Hurst coefficient}{analysis}
        \subidx{\market}{Hurst coefficient analysis}
        \subidx{\market}{rate of change}

        This is, essentially, a statement of the range of values, in
        the increments of the rate of revenue returns, (per
        {\timescale},) that is to be expected over the time interval,
        $t_2 - t_1$,
        $R_v$,~\cite[pp. 178]{Feder},~\cite[pp. 172]{Cambel}:

        \begin{eqnarray}
            R_v\left(t_2 - t_1\right) & \propto & \left(t_2 - t_1\right)^{H}\\
                                      & \propto & \left(t_2 - t_1\right)^{\thurstlow}
            \label{\SETLABEL:R}
        \end{eqnarray}

        \subidx{\market}{rate of revenue returns, range}
        \subidx{Hurst coefficient}{analysis}
        \subidx{\market}{Hurst coefficient analysis}
        \subidx{\market}{rate of change}
        \subidx{Markov}{statistics}
        \subidx{statistics}{Markov}
        \noindent where $R$ is the range of values in the increments
        of the rate of revenue returns, (per {\timescale}.) A Hurst
        coefficient, $H$, that is much larger than $\frac{1}{2}$, (but
        less than 1,) implies a strongly non-Gaussian distribution in
        the increments of the rate of revenue returns, (per
        {\timescale},)~\cite[pp. 152, 194]{Feder}, and a Hurst
        coefficient near $\frac{1}{2}$ implies that the increments of
        the rate of revenue returns, (per {\timescale}) is
        characteristic of an independent
        process~\cite[pp. 195]{Feder}. Extreme caution should be
        exercised in using Markov statistics in any analysis where the
        Hurst coefficient is not
        $\frac{1}{2}$,~\cite[pp. 124]{Crownover},~\cite[pp. 106]{Peters:CAOITCM}.


        As a useful approximation, if $H$, is approximately
        $\frac{1}{2}$, Equation~\ref{\SETLABEL:R} reduces
        to,~\cite[pp. 129]{Schroeder}:

        \begin{eqnarray}
            R\left(t_2 - t_1\right) & \propto & (t_2 - t_1)^{\frac{1}{2}}\\
                                    & \propto & \sqrt{\left(t_2 - t_1\right)}
        \end{eqnarray}

        \subidx{\market}{rate of revenue returns, range}
        \subidx{\market}{rate of revenue returns, increase and decrease}
        \subidx{Hurst coefficient}{analysis}
        \subidx{\market}{Hurst coefficient analysis}
        \subidx{\market}{rate of change}
        \subidx{Markov}{statistics}
        \subidx{statistics}{Markov}

        In the case where the Hurst coefficient, $H$, is
        $\frac{1}{2}$, the range of values in the increments of the
        rate of revenue returns, (per {\timescale},) divided by the
        standard deviation of these values, $S$, can be anticipated to
        increase over time according to the following
        relation,~\cite[pp. 154]{Feder},~\cite[pp. 129]{Schroeder}:

        \begin{equation}
            \frac{R\left(t_2 - t_1\right)}{S} \propto \left(t_2 - t_1\right)^{\frac{1}{2}}
        \end{equation}

        \subidx{\market}{rate of revenue returns, range}
        \subidx{\market}{rate of revenue returns, increase and decrease}
        \subidx{Hurst coefficient}{analysis}
        \subidx{\market}{Hurst coefficient analysis}
        \subidx{\market}{rate of change}
        \noindent which is a useful conceptual approximation, since it
        involves only the square root function---if the range and the
        standard deviation of the increments of the rate of revenue
        returns, (per {\timescale},) are known, (and $H \approx
        \frac{1}{2}$,) then the expected change in $\frac{R}{S}$, will
        increase with the square root of time\footnote{To be precise,
        it is actually asymptotically proportional to
        $\tau^{\frac{1}{2}}$}.

        Another useful approximation when rescaling processes that are
        characterize by Brownian motion, (ie., when $H \approx
        \frac{1}{2}$,) is that:

        \begin{eqnarray}
            X\left(t\right) & \propto & \frac{X\left(rt\right)}{r^{H}}\\
                            & \propto & \frac{X\left(rt\right)}{r^{\thurstlow}}
        \end{eqnarray}

        \idx{Brownian motion}
        \idx{fractal}
        Where $X(t)$ is the process characterized by Brownian motion,
        and $r$ is a scaling factor,~\cite[pp. 494]{Peitgen}.

        \subidx{programs}{tslsq}
        \subidx{tslsq}{program}
        The program {\it tslsq}\/ was used on the H parameter data,
        presented in Figure~\ref{\SETLABEL:HP}, to provide a least
        squares approximation to the H parameter for the
        {\market}. The superimposed least squares approximation on the
        original H parameter data is presented.  By contrast, the H
        parameter, as derived by the methodology outlined
        in~\cite[pp. 249]{Crownover}, is {\thcalclow} for the near
        future, and {\thcalcall} for the distant future.

        \subidx{\market}{Hurst coefficient analysis}
        \subidx{Hurst coefficient}{analysis}
        \subidx{increments}{normalized}
        \subidx{normalized}{increments}
        \subidx{programs}{tshurst}
        \subidx{tshurst}{program}
        \subidx{\market}{H parameter analysis}
        \subidx{H parameter}{analysis}
        \subidx{programs}{tshcalc}
        \subidx{tshcalc}{program}
        Figures~\ref{\SETLABEL:HC} and~\ref{\SETLABEL:HP} represent
        Hurst coefficient and H parameter data that are derived from
        the normalized increments, shown in
        Figure~\ref{\SETLABEL:TF}. In this case, the data is
        considered a normalized derivative of the time series data
        presented in Figure~\ref{\SETLABEL:TF}, instead of a
        cumulative sum.  The program, {\it tshurst}\/, is described
        briefly in appendix~\ref{programs}, and the data for
        figures~\ref{\SETLABEL:THC} and~\ref{\SETLABEL:THP} was made
        using the -d option.

        \begin{figure}[ht]
            \begin{center}
                \begin{minipage}[t]{0.45\textwidth}
                    \epsfxsize=1.0\linewidth
                    \epsffile{\directory/data.tsfraction.tshurst-d.eps}
                    \caption[{\market}, traditional Hurst coefficient
                        data]{{\market}, traditional Hurst coefficient
                        data for the time series data shown in
                        Figure~\ref{\SETLABEL:TS}.  The slope of the
                        graph is the Hurst coefficient, and is
                        {\hurstlow} for the near term, and
                        {\hurstall} for the far term.}
                    \label{\SETLABEL:THC}
                \end{minipage}
                \hfill
                \begin{minipage}[t]{0.45\textwidth}
                    \epsfxsize=1.0\linewidth
                    \epsffile{\directory/data.tsfraction.tshcalc-d.eps}
                    \caption[{\market}, traditional H parameter
                        data]{{\market}, traditional H parameter data
                        for the time series data shown in
                        Figure~\ref{\SETLABEL:TS} The slope of the
                        graph is the H parameter, and is {\hcalclow}
                        for the near term, and {\hcalcall} for the
                        far term.}
                    \label{\SETLABEL:THP}
                \end{minipage}
            \end{center}
        \end{figure}

% Local Variables:
% TeX-parse-self: t
% TeX-auto-save: t
% TeX-master: "fractal.tex"
% End:


        \subsubsection{Observations on the Hurst Coefficient Analysis}

            Note that both the Hurst coefficient and H parameter
            graphs indicate that the time series data set does not
            contain a random process---which is to be anticipated,
            since the data set is periodic.

        %
% -----------------------------------------------------------------------------
%
% A license is hereby granted to reproduce this software source code and
% to create executable versions from this source code for personal,
% non-commercial use.  The copyright notice included with the software
% must be maintained in all copies produced.
%
% THIS PROGRAM IS PROVIDED "AS IS". THE AUTHOR PROVIDES NO WARRANTIES
% WHATSOEVER, EXPRESSED OR IMPLIED, INCLUDING WARRANTIES OF
% MERCHANTABILITY, TITLE, OR FITNESS FOR ANY PARTICULAR PURPOSE.  THE
% AUTHOR DOES NOT WARRANT THAT USE OF THIS PROGRAM DOES NOT INFRINGE THE
% INTELLECTUAL PROPERTY RIGHTS OF ANY THIRD PARTY IN ANY COUNTRY.
%
% Copyright (c) 1994-2006, John Conover, All Rights Reserved.
%
% Comments and/or bug reports should be addressed to:
%
%     john@email.johncon.com (John Conover)
%
% -----------------------------------------------------------------------------
%
% Revision: \RCSRevision \\
% Revision Time: \RCSTime UMT \\
% Revision Date: \RCSDate \\
% Revision Id: \RCSId \\
% Revision File: \RCSLog \\
\RCS $Revision: 0.0 $
\RCS $Date: 2006/01/20 04:38:13 $
\RCS $Id: fiscal.tex,v 0.0 2006/01/20 04:38:13 john Exp $
% $Log: fiscal.tex,v $
% Revision 0.0  2006/01/20 04:38:13  john
% Initial version
%
%
    \subsection{Fixed Increment Approximation for Fiscal Strategy}
        \label{\SETLABEL:FS}

        \subidx{\market}{fiscal strategy}
        \subidx{markets}{analysis}
        \subidx{analysis}{markets}
        \subidx{strategy}{fiscal}
        \subidx{fiscal}{strategy}
        The data in this section is presented in tabular form in
        Section~\ref{\SETLABELREF:LR}. This section derives various
        values based on the ``average'' of the normalized increments
        presented in Figure~\ref{\SETLABEL:TFA}. These values are an
        approximation to a, probably, complex process with a
        distribution shown in Figure~\ref{\SETLABEL:TF}. These values
        will be used in a fixed increment Brownian fractal analysis
        and simulation of the {\market}, and may, or may not, provide
        adequate accuracy for projections.

        For an organization operating in the {\market}, the fiscal
        strategy, commensurate with the aggregate environment, can be
        derived as follows~\cite[pp. 128, pp
        151]{Schroeder},~\cite[pp. 450]{Reza},~\cite[pp. 270]{Pierce}:
        \vspace{0.15in}

        \subsubsection{Logarithmic Returns}
            \label{\SETLABEL:LR}

            \subidx{logarithmic}{returns}
            \subidx{returns}{logarithmic}
            \subidx{\market}{logarithmic returns}
            The logarithmic returns can be calculated by various
            means. Four will be presented here, for comparison.

            \subidx{programs}{tsnormal}
            \subidx{tsnormal}{program}
            \subidx{logarithmic}{returns}
            \subidx{returns}{logarithmic}
            The logarithmic returns, in bits, $bits$, as computed from
            the mean, by the program {\it tsnormal}\/, which is
            described in Chapter~\ref{programs}, and is presented in
            Figure~\ref{\SETLABEL:TF}, and Equation~\ref{abits} from
            Section~\ref{ereturns} in Chapter~\ref{general}:

            \begin{equation}
                bits = \frac{\ln \left({\datafractionmean} + 1\right)}{\ln \left(2\right)} = \datafractionmeanbits
            \end{equation}

            \subidx{programs}{tslsq}
            \subidx{tslsq}{program}
            \subidx{logarithmic}{returns}
            \subidx{returns}{logarithmic}
            \noindent By comparison, the logarithmic returns, in bits,
            $bits$, as computed from the constant in the least squares
            approximation, using the program {\it tslsq}\/, which is briefly
            described in Chapter~\ref{programs}, as presented in
            Figure~\ref{\SETLABEL:TF}, and Equation~\ref{abits} from
            Section~\ref{ereturns} in Chapter~\ref{general}:

            \begin{equation}
                bits = \frac{\ln \left({\datafractionconstant} + 1\right)}{\ln \left(2\right)} = \datafractionconstantbits
            \end{equation}

            Note that if the mean is not constant in
            Figure~\ref{\SETLABEL:TF}, this method will not provide
            accurate results.

            \subidx{programs}{tslsq}
            \subidx{tslsq}{program}
            \subidx{logarithmic}{returns}
            \subidx{returns}{logarithmic}
            \noindent And by yet another comparison, using the program
            {\it tslsq}\/, which is briefly described in
            Chapter~\ref{programs}, with the -e -p options, to provide
            a formula for the least squares exponential fit to the
            time series data set presented in
            Figure~\ref{\SETLABEL:TS}:

            \begin{equation}
                bits = {\datatslsqepbits}
            \end{equation}

            \subidx{programs}{tslogreturns}
            \subidx{tslogreturns}{program}
            \subidx{logarithmic}{returns}
            \subidx{returns}{logarithmic}
            \noindent And finally, by comparison, from the
            {\it tslogreturns}\/ program, which is briefly described
            in Chapter~\ref{programs}, with the -p option, to provide
            a formula for the logarithmic returns of the time series
            data set presented in Figure~\ref{\SETLABEL:TS}:

            \begin{equation}
                bits = {\logreturns}
            \end{equation}

        \subsubsection{Calculation of Shannon Probability}
            \label{\SETLABEL:SP}

            \subidx{\market}{Shannon probability}
            Ideally, all of the values presented in
            Section~\ref{\SETLABEL:LR} would be equal. Using the
            logarithmic returns provided by the {\it tslogreturns}\/
            program, to be consistent
            with~\cite[pp. 81]{Peters:CAOITCM}

            \subidx{programs}{tslogreturns}
            \subidx{tslogreturns}{program}
            \begin{equation}
                2^{{\logreturns}t}
            \end{equation}

            \noindent therefore:
            \begin{equation}
                C\left(p\right) = {\logreturns}
            \end{equation}
            \subidx{programs}{tsshannon}
            \subidx{tsshannon}{program}
            \subidx{Shannon}{probability}
            \subidx{probability}{Shannon}
            \noindent and, {\it tsshannon}\/ {\logreturns} gives:
            \begin{equation}
                \label{\SETLABEL:F0}
                C\left({\shannonlogreturns}\right) = {\logreturns}
            \end{equation}
            \noindent therefore:
            \begin{eqnarray}
                2^{C\left({\shannonlogreturns}\right)} & = & 2^{\logreturns}\\
                                                       & = & {\twologreturns}\\
                                                       & = & {\twologreturnshundred}\%
            \end{eqnarray}
            \noindent and:
            \begin{eqnarray}
                2p - 1 & = & \left(2 \cdot {\shannonlogreturns}\right) - 1\\
                       & = & {\twopone}\\
                       \label{\SETLABEL:F1}
                       & = & {\twoponehundred}\%
            \end{eqnarray}

            \subidx{\market}{fiscal strategy}
            \subidx{markets}{analysis}
            \subidx{analysis}{markets}
            \subidx{strategy}{fiscal}
            \subidx{fiscal}{strategy}
            \subidx{\market}{fiscal strategy}
            \subidx{\market}{growth rate}
            Presuming the simplified assumptions outlined in
            Section~\ref{assumptions}, the ``typical'' organization
            operating in the {\market} executes a long term fiscal
            strategy, commensurate with the aggregate environment,
            that is to invest, every {\timescale}, in sufficient
            additional resources and infrastructure, to increase the
            manufacturing of goods and services by {\twoponehundred}\%
            of its rate of revenue returns, (per {\timescale}.) As a
            conceptual model, the remaining {\hundredtwoponehundred}\%
            will be held in ``reserve'' with a
            {\shannonlogreturnshundred}\% chance of making twice the
            {\twoponehundred}\% back, (and a
            {\hundredshannonlogreturnshundred}\% chance of making
            0.0,) in one {\timescale}, on the average, for an average
            growth in its rate of revenue returns, (per {\timescale},)
            of {\twologreturnshundred}\%, or a doubling of its rate of
            revenue returns, (per {\timescale},) in
            {\oneoverlogreturns} {\timescale}s.

        \subsubsection{Example Fixed Increment Approximation Fiscal Strategies}

            \subidx{\market}{fiscal strategy}
            \subidx{markets}{analysis}
            \subidx{analysis}{markets}
            \subidx{strategy}{fiscal}
            \subidx{fiscal}{strategy}
            \subidx{\market}{fiscal strategy}
            \subidx{\market}{growth rate}
            \subidx{\market}{management metric}
            \idx{management metric}
            A possible metric on the effectiveness of long term fiscal
            management could possibly be that if an investment of
            {\twoponehundred}\% per {\timescale} of the rate of
            revenue returns, (per {\timescale},) is made in resources
            and infrastructure, then the rate of revenue returns would
            be expected to increase by {\twologreturnshundred}\%, per
            {\timescale}, on average.

            Note that the metrics presented in this section are
            representative of the {\market} as an aggregate whole, and
            may or may not be accurate representations for any
            particular participant in the environment. Of interest to
            the participants in the environment would be a similar
            analysis of each product or service rendered in the
            marketplace.

            \subidx{\market}{fiscal strategy}
            \subidx{markets}{analysis}
            \subidx{analysis}{markets}
            \subidx{strategy}{fiscal}
            \subidx{fiscal}{strategy}
            \subidx{\market}{fiscal strategy}
            As a simple illustrative example, a company operating in
            this environment might obtain a credit line from a bank
            that is equal to {\twoponehundred}\% of its rate of
            revenue returns, (per {\timescale},) to finance additional
            operations. In this simple scenario, the company would use
            its revenue base as collateral for the loan. Some
            {\timescale}s, depending on the {\market}'s environment,
            the company's rate of revenue returns exceeds what was
            borrowed from the bank, and the loan is repaid in
            full. Other {\timescale}s, the company must default, and
            the bank seizes a portion of the company's revenue base to
            pay the delinquent loan. However, on the average, the
            company will expand its rate of revenue returns at
            {\twologreturnshundred}\% per {\timescale}.

            \subidx{\market}{fiscal strategy}
            \subidx{markets}{analysis}
            \subidx{analysis}{markets}
            \subidx{strategy}{fiscal}
            \subidx{fiscal}{strategy}
            \subidx{\market}{fiscal strategy}
            As another simple example, a company re-invests
            {\twoponehundred}\% of its rate of revenue returns, (per
            {\timescale},) in development, marketing, sales, and
            distribution of new products.  Although some products will
            be successful and the return on the investment will exceed
            the {\twoponehundred}\% per {\timescale} investment,
            others will not. However, on the average, the company will
            expand it gross rate of revenue returns at
            {\twologreturnshundred}\% per {\timescale}.

            \subidx{\market}{fiscal strategy}
            \subidx{markets}{analysis}
            \subidx{analysis}{markets}
            \subidx{strategy}{fiscal}
            \subidx{fiscal}{strategy}
            \subidx{\market}{fiscal strategy}
            \subidx{\market}{product portfolio}
            \subidx{\market}{product diversity}
            \subidx{\market}{product mix}
            \subidx{\market}{optimum number of products}
            \idx{product portfolio}
            \idx{product diversity}
            \idx{optimum number of products}
            \idx{product mix}

            As an example of ``product portfolio'' management, suppose
            a company re-invests {\twoponehundred}\% of its rate of
            revenue returns, (per {\timescale},) in development,
            marketing, sales, and distribution of new products.
            Further suppose that the company has two products, and a
            fractal analysis of the individual product rate of revenue
            return time series indicates that one product has a
            Shannon probability of 0.65, and the other has a Shannon
            probability of 0.55. Then the percentage of re-investment
            in the first product would be $(2 \cdot 0.65 - 1) \cdot
            {\twoponehundred}$, percent of the rate of revenue
            returns, and $(2 \cdot 0.55 - 1) \cdot {\twoponehundred}$
            percent for the second product, implying that the company
            should diversify its product line\footnote{The astute
            reader would note that the linear addition was used to add
            the contribution to development of each product. This is a
            ``near term'' interpretation. Actually, in general, the
            method used should be a root mean square process,
            dependent on the Hurst Coefficient, $H$, where
            $P_{total}^H = P_1^H + P_2^H + \cdots$, where $P_n$ is the
            contribution to each individual product. For a Brownian
            motion, or random walk type of fractal the Hurst
            Coefficient is a function of time into the future. For the
            ``near term,'' the Hurst coefficient is very near unity,
            meaning the summation process is linear. For the ``long
            term,'' $H \approx 0.5$, or a standard root mean square
            summation process should be used. If $H$ is $0.5$ then the
            market is termed a Brownian motion, or random walk
            process. If it is larger than 0.5, it is termed fractional
            Brownian motion process. For a random walk process, ``near
            term'' and ``far term'' are quantitatively differentiated
            on the Hurst Coefficient graph where $1 - \ln (t) = 0.5
            \cdot \ln (t)$, or when $\ln (t) = 2$, or $t =
            7.389\ldots$ See~\cite[pp. 67, 83-84]{Peters:CAOITCM}
            and~\cite[pp. 129, 159]{Schroeder} for particulars on the
            implications of the Hurst Coefficient and root mean square
            summation issues.}.  Note that this is a ``bet hedging''
            metric methodology, and assumes that the products have
            uncorrelated revenue return rates. If this re-investment
            methodology is not feasible, perhaps for strategic
            financial reasons, then the re-investment in both products
            should total the ${\twoponehundred}$\%, and the investment
            in each product should be made at a ratio of $\frac{(2
            \cdot 0.65 - 1)}{(2 \cdot 0.55 - 1)} = 3 : 1$,
            respectively. Note that this ``bet hedging'' can be used
            to define the optimal number of products that can be
            supported on the rate of revenue returns. If it assumed
            that all products are ``typical'' for the {\market}, as a
            standard bench mark, then the optimal number will be
            $\frac{1}{{\twopone}}$. Note that this is a
            ``theoretical'' value, since not all products are
            ``typical,'' and there may be strategic reasons, for
            example product leveraging, that may increase the number
            of products above the optimum. However, most of the
            revenue should come from the optimal number of products,
            since having more products will decrease the amount of the
            potential investment in each product, and having less than
            the optimum number of products will increase the risk that
            many of the products could suffer a ``down market''
            concurrently, impacting the rate of revenue returns.  As
            another interesting interpretation of the optimal
            ``hedging of bets,'' in product portfolio strategy, and
            considering the graph of the normalized increments
            presented in Figure~\ref{\SETLABEL:TF}, if the
            organization is running optimally, then these products
            will generate, at least in principle, one standard
            deviation, approximately $0.8413 = 84.13$\% of the future
            growth in rate of revenue returns. Naturally, these are
            approximations, and the values are an approximation to a,
            probably, complex process, and appropriate scrutiny should
            be exercised before making specific projections.  As yet
            another example of ``product portfolio'' management,
            consider the issue of product mix. In this interpretation,
            {\twoponehundred}\% of the product manufactured should be
            ``proprietary,'' while the rest is ``industry standard.''
            As yet another possibility, {\twoponehundred}\% of the
            product manufactured should be predatory into new markets,
            and the remainder in markets that are ``traditional'' for
            the company.

% Local Variables:
% TeX-parse-self: t
% TeX-auto-save: t
% TeX-master: "fractal.tex"
% End:


        %
% -----------------------------------------------------------------------------
%
% A license is hereby granted to reproduce this software source code and
% to create executable versions from this source code for personal,
% non-commercial use.  The copyright notice included with the software
% must be maintained in all copies produced.
%
% THIS PROGRAM IS PROVIDED "AS IS". THE AUTHOR PROVIDES NO WARRANTIES
% WHATSOEVER, EXPRESSED OR IMPLIED, INCLUDING WARRANTIES OF
% MERCHANTABILITY, TITLE, OR FITNESS FOR ANY PARTICULAR PURPOSE.  THE
% AUTHOR DOES NOT WARRANT THAT USE OF THIS PROGRAM DOES NOT INFRINGE THE
% INTELLECTUAL PROPERTY RIGHTS OF ANY THIRD PARTY IN ANY COUNTRY.
%
% Copyright (c) 1994-2006, John Conover, All Rights Reserved.
%
% Comments and/or bug reports should be addressed to:
%
%     john@email.johncon.com (John Conover)
%
% -----------------------------------------------------------------------------
%
% Revision: \RCSRevision \\
% Revision Time: \RCSTime UMT \\
% Revision Date: \RCSDate \\
% Revision Id: \RCSId \\
% Revision File: \RCSLog \\
\RCS $Revision: 0.0 $
\RCS $Date: 2006/01/20 04:38:13 $
\RCS $Id: companies.tex,v 0.0 2006/01/20 04:38:13 john Exp $
% $Log: companies.tex,v $
% Revision 0.0  2006/01/20 04:38:13  john
% Initial version
%
%
    \subsection{Number of Companies}
        \label{\SETLABEL:QNC}

        \subidx{\market}{number of companies}
        \subidx{number of companies}{analysis}
        \subidx{analysis}{number of companies}
        \subidx{Shannon}{probability}
        \subidx{probability}{Shannon}
        This section evaluates the approximate, or ``average,'' number
        of companies in the {\market}, and uses the method outlined in
        Chapter~\ref{general}, Section~\ref{aftsma}. Since the
        average, $avg_{ind}$, and the root mean square, $rms_{ind}$,
        of the normalized increments of the {\market} time series is
        \datafractionmean, and \datafractionrms respectively, the
        number of companies participating in the market can be
        calculated by Equation~\ref{ncompanies} to be {\ncompanies}.

        If this value seems consistent number of companies in the
        {\market}, within the assumptions outlined in
        Chapter~\ref{general}, Section~\ref{aftsma}, then it would
        seem that there is some circumstantial or indirect evidence
        that the companies participating in the {\market} are
        operating optimally, and the ``average'' Shannon probability,
        $P$ for each participating company would be, using
        Equation~\ref{pncompanies}, {\pncompanies}, which would be the
        value which should be used in Section~\ref{\SETLABEL:FS} for
        each participating company if market expansion was to be
        consistent with the rest of the industry. However, if the
        Shannon probability derived in Section~\ref{\SETLABEL:FS} is
        greater than the average Shannon probability for the companies
        participating in the {\market}, as derived in this section,
        then the market would, possibly, be exploitable with the
        fiscal strategy outlined in Section~\ref{\SETLABEL:FS}. The
        maximum exploitability for the {\market} is derived in
        Section~\ref{\SETLABEL:MAXSHANNON}, but it is probably of
        doubtful practicality.

        Note that these optimizations would maximize a company's
        market growth. Since there are probably many companies
        competing in the market place, this would not necessarily
        maximize a company's P\&L, as described in
        Chapter~\ref{general}, Section~\ref{ompl}. The Shannon
        probability that maximizes market share in the {\market} is
        \pncompanies, with several alternative solutions listed in the
        previous paragraph. However, these should be contrasted to the
        Shannon probability that maximizes a company's P\&L which is
        \avgrms~in the {\market}. In all cases, the fraction of the
        P\&L that should be ``wagered'' on the future, $f$, should be:

        \begin{equation}
            f = 2P - 1
        \end{equation}

        \noindent where $P$ is the particular Shannon probability
        chosen optimize a particular fiscal strategy. Interestingly,
        the measured Shannon probability of the {\market} would tend
        to indicate that the companies participating in the market
        have chosen a fiscal strategy that optimizes market growth, as
        opposed to capital growth.

        \subidx{\market}{increasing returns}
        \subidx{economic increasing returns}{\market}
        As interesting interpretation of these exploitive issues,
        since all three fiscal strategies will result in exponential
        market growth for every company participating in the market,
        is that they may represent, perhaps, an example of
        ``increasing returns.''

% Local Variables:
% TeX-parse-self: t
% TeX-auto-save: t
% TeX-master: "fractal.tex"
% End:


        %
% -----------------------------------------------------------------------------
%
% A license is hereby granted to reproduce this software source code and
% to create executable versions from this source code for personal,
% non-commercial use.  The copyright notice included with the software
% must be maintained in all copies produced.
%
% THIS PROGRAM IS PROVIDED "AS IS". THE AUTHOR PROVIDES NO WARRANTIES
% WHATSOEVER, EXPRESSED OR IMPLIED, INCLUDING WARRANTIES OF
% MERCHANTABILITY, TITLE, OR FITNESS FOR ANY PARTICULAR PURPOSE.  THE
% AUTHOR DOES NOT WARRANT THAT USE OF THIS PROGRAM DOES NOT INFRINGE THE
% INTELLECTUAL PROPERTY RIGHTS OF ANY THIRD PARTY IN ANY COUNTRY.
%
% Copyright (c) 1994-2006, John Conover, All Rights Reserved.
%
% Comments and/or bug reports should be addressed to:
%
%     john@email.johncon.com (John Conover)
%
% -----------------------------------------------------------------------------
%
% Revision: \RCSRevision \\
% Revision Time: \RCSTime UMT \\
% Revision Date: \RCSDate \\
% Revision Id: \RCSId \\
% Revision File: \RCSLog \\
\RCS $Revision: 0.0 $
\RCS $Date: 2006/01/20 04:38:13 $
\RCS $Id: operations.tex,v 0.0 2006/01/20 04:38:13 john Exp $
% $Log: operations.tex,v $
% Revision 0.0  2006/01/20 04:38:13  john
% Initial version
%
%
    \subsection{Fixed Increment Approximation for Operational Strategy}
        \label{\SETLABEL:OPS}.

        This section derives various values based on the ``average''
        of the normalized increments presented in
        Figure~\ref{\SETLABEL:TFA}. These values are an approximation
        to a, probably, complex process with a distribution shown in
        Figure~\ref{\SETLABEL:TF}. These values will be used in a
        fixed increment Brownian fractal analysis and simulation of
        the {\market}, and may, or may not, provide adequate accuracy
        for projections.

        \subidx{\market}{fiscal strategy}
        \subidx{\market}{Shannon probability}
        \subidx{strategy}{fiscal}
        \subidx{fiscal}{strategy}
        \subidx{Shannon}{probability}
        \subidx{probability}{Shannon}
        It should be noted that the analysis of fiscal strategy,
        presented in Section~\ref{\SETLABEL:FS}, is derived from the
        {\market} metrics and may, or may not, be maximally
        optimal. For the optimal fiscal strategy, which may be
        exploitable, see Section~\ref{\SETLABEL:MAXSHANNON}.

        \subidx{strategy}{exploitable}
        \subidx{exploitable}{strategy}
        \subidx{\market}{windows of opportunity}
        \idx{windows of opportunity}
        \subidx{decision}{obsolete}
        \subidx{obsolete}{decision}
        \subidx{decision}{timeliness}
        \subidx{timeliness}{decision}
        \subidx{rate of revenue returns}{forecast}
        \subidx{forecast}{rate of revenue returns}
        An additional exploitable strategy may be time itself.
        Equations~\ref{\SETLABEL:V},~\ref{\SETLABEL:R},
        and,~\ref{\SETLABEL:MA}, are, essentially, metrics on how fast
        a decision, which is based on information concerning the
        current status of the {\market}, becomes obsolete. Obviously,
        how long a decision is expected to remain relevant should be
        addressed as an operational necessity in strategic planning
        and project management. Figures~\ref{\SETLABEL:FN},
        and,~\ref{\SETLABEL:FF} compare methods of approximation of
        the ``forecastability'' of rate of revenue returns in the
        {\market} for the near term and far
        term~\cite[pp. 83-84]{Peters:CAOITCM}, respectively. As a
        general rule, caution must be exercised when making decisions
        that will span a time interval larger than the time interval
        where the ``forecastability'' of rate of revenue returns drops
        below 50\%. Beyond this time interval, the chances increase
        that the competitive and market forces will alter the market
        environment in a possibly detrimental unanticipated
        fashion. Obviously, there is significant advantage in
        ``timeliness'' of development, manufacturing, and distribution
        of products and services that are consistent with this
        temporal agenda. Automation of these processes, if executed
        consistently with this agenda, should be considered a
        competitive advantage.

        \subidx{strategy}{exploitable}
        \subidx{exploitable}{strategy}
        \subidx{rate of revenue returns}{forecast}
        \subidx{forecast}{rate of revenue returns}
        \idx{product life cycle}
        \idx{life cycle, product}
        In some sense, this temporal agenda defines the ``average''
        product or service life cycle in the {\market}. When the
        ``forecastability'' of rate of revenue returns drops below
        50\%, there is an even chance that the rate of revenue returns
        for the product or service will change in a detrimental
        fashion. If it is assumed that a product or service life cycle
        consists of a ramp up, a maintenence interval, and a ramp
        down, then, if all three life cycle intervals are equal, the
        product life cycle will be, approximately, three times the
        time interval where the ``forecastability'' of rate of revenue
        returns drops below 50\%. Although probably not an accurate
        prediction of product or service life cycle, the technique may
        be used as a conceptual approximation to the dynamics of
        ``market windows.\footnote{For example, consider the market
        for table salt. Since it has inelastic supply and demand
        curves, and is a necessary requirement for life, it would be
        expected that the Hurst coefficient would be very near
        unity---ignoring competitive pressures in the market. The
        predictability of the table salt market would, therefore, be
        expected to be relatively good, over time.}''  The conceptual
        approximation will probably predict a ``conservative'' or
        ``pessimistic'' value in relation to actual markets.

        \begin{figure}[ht]
            \begin{center}
                \begin{minipage}[t]{0.45\textwidth}
                    \epsfxsize=1.0\linewidth
                    \epsffile{\directory/datahurstlownear.eps}
                    \caption[{\market}, ``forecastability'' of near
                        term rate of revenue returns]{{\market},
                        ``forecastability'' of near term rate of
                        revenue returns. Although the error function
                        is the most accurate, for the near term,
                        $H^{t} = \thurstlow^{t}$ may be used as a
                        reliable metric of ``forecastability'' of the
                        rate of revenue returns.}
                    \label{\SETLABEL:FN}
                \end{minipage}
                \hfill
                \begin{minipage}[t]{0.45\textwidth}
                    \epsfxsize=1.0\linewidth
                    \epsffile{\directory/datahurstlowfar.eps}
                    \caption[{\market}, ``forecastability'' of far
                        term rate of revenue returns]{{\market},
                        ``forecastability'' of far term rate of
                        revenue returns. Although the error function
                        is the most accurate, for the far term,
                        $\frac{1}{\sqrt{t}}$ may be used as a reliable
                        metric of ``forecastability'' of the rate of
                        revenue returns.}
                    \label{\SETLABEL:FF}
                \end{minipage}
            \end{center}
        \end{figure}

        \idx{operations research}
        As an interesting interpretation of the data presented in
        Figure~\ref{\SETLABEL:FN}, there may be, perhaps, some
        applicability to such operational agendas as inventory
        control. Maintaining too little inventory, obviously, will
        create a situation where the organization can not exploit
        market expansion, and maintaining too much inventory,
        likewise, would over extend the company, creating unnecessary
        losses when the market contracts. The company should maintain
        inventory levels that do not exceed, from
        Equation~\ref{\SETLABEL:MA}, ${\thurstlow}^{n} = 0.5$
        {\timescale}s of operations. Since the optimal amount of
        inventory and, from Equation~\ref{\SETLABEL:V}, the variance
        of change in the rate of revenue returns in the future can be
        calculated, there may, perhaps, be some applicability to a
        forecasting methodology that can be incorporated into other
        areas of operations research, for example the linear algebras
        using simplex methodologies for optimization of manufacturing
        processes. Traditionally, these forecasts are made by the
        sales department, and are subject to various subjective
        biases.

% Local Variables:
% TeX-parse-self: t
% TeX-auto-save: t
% TeX-master: "fractal.tex"
% End:


        %
% -----------------------------------------------------------------------------
%
% A license is hereby granted to reproduce this software source code and
% to create executable versions from this source code for personal,
% non-commercial use.  The copyright notice included with the software
% must be maintained in all copies produced.
%
% THIS PROGRAM IS PROVIDED "AS IS". THE AUTHOR PROVIDES NO WARRANTIES
% WHATSOEVER, EXPRESSED OR IMPLIED, INCLUDING WARRANTIES OF
% MERCHANTABILITY, TITLE, OR FITNESS FOR ANY PARTICULAR PURPOSE.  THE
% AUTHOR DOES NOT WARRANT THAT USE OF THIS PROGRAM DOES NOT INFRINGE THE
% INTELLECTUAL PROPERTY RIGHTS OF ANY THIRD PARTY IN ANY COUNTRY.
%
% Copyright (c) 1994-2006, John Conover, All Rights Reserved.
%
% Comments and/or bug reports should be addressed to:
%
%     john@email.johncon.com (John Conover)
%
% -----------------------------------------------------------------------------
%
% Revision: \RCSRevision \\
% Revision Time: \RCSTime UMT \\
% Revision Date: \RCSDate \\
% Revision Id: \RCSId \\
% Revision File: \RCSLog \\
\RCS $Revision: 0.0 $
\RCS $Date: 2006/01/20 04:38:13 $
\RCS $Id: simulation.tex,v 0.0 2006/01/20 04:38:13 john Exp $
% $Log: simulation.tex,v $
% Revision 0.0  2006/01/20 04:38:13  john
% Initial version
%
%
    \subsection{Simulation of Fixed Increment Approximation for Fiscal Strategy}
        \label{\SETLABEL:TSUNFAIRBROWNIAN}

        \subidx{\market}{market simulation}
        The data in this section is presented in tabular form in
        Section~\ref{\SETLABELREF:SIM}.
        Figure~\ref{\SETLABEL:TSUNFAIRBROWNIAN0} represents a
        constructional simulation of the time series data presented in
        Figure~\ref{\SETLABEL:TS}. The program {\it
        tsunfairbrownian}\/, which is briefly described in
        appendix~\ref{programs}, was used in the reconstruction. The
        reconstructed data is superimposed on the original time series
        data.  The program, {\it tsunfairbrownian}\/, essentially,
        constructs the new time series as a Brownian fractal with
        fixed increments---the value of the fixed increment is derived
        from the root mean square average of the normalized increments
        presented in Figure~\ref{\SETLABEL:TF}. The ``quality'' of
        such a reconstruction should be subject to adequate scepticism
        and scrutiny since, in all probability, the normalized
        increments presented in Figure~\ref{\SETLABEL:TF} represent a
        relatively complex process, that may not be ``modeled'' with
        such a simple methodology.

        As a further comparison of the the constructional simulation
        with the original time series data,
        Figure~\ref{\SETLABEL:TSUNFAIRBROWNIAN1} presents a normalized
        histogram of the normalized increments of the reconstructed
        time series, superimposed on the normalized histogram
        presented in Figure~\ref{\SETLABEL:NH}.

        \subidx{\market}{fiscal strategy, simulation}
        \subidx{markets}{simulation}
        \subidx{simulation}{markets}
        \subidx{strategy}{fiscal, simulation}
        \subidx{fiscal}{strategy, simulation}
        \subidx{programs}{tsunfairbrownian}
        \subidx{tsunfairbrownian}{program}
        \begin{figure}[ht]
            \begin{center}
                \begin{minipage}[t]{0.45\textwidth}
                    \epsfxsize=1.0\linewidth
                    \epsffile{\directory/tsunfairbrownian-f.eps}
                    \caption[{\market}, Time series data, empirical and
                        simulated]{{\market}, Time series data, empirical
                        and simulated, using the program {\it tsunfairbrownian}\/
                        with f = {\datafractionrms}. This data is
                        superimposed on the data presented in
                        Figure~\ref{\SETLABEL:TS}.}
                    \label{\SETLABEL:TSUNFAIRBROWNIAN0}
                \end{minipage}
                \hfill
                \begin{minipage}[t]{0.45\textwidth}
                    \epsfxsize=1.0\linewidth
                    \epsffile{\directory/tsunfairbrownian-f.tsfraction.tsnormal-s30.eps}
                    \caption[{\market}, normalized histogram,
                        empirical and simulated]{{\market}, normalized
                        histogram of the normalized increments of the
                        time series data shown in
                        Figure~\ref{\SETLABEL:TSUNFAIRBROWNIAN0},
                        empirical and simulated.  The empirical data
                        has a mean of {\datafractionmean}, with a
                        standard deviation of {\datafractionstddev}.
                        By comparison, the simulated data has a mean
                        of {\tsunfairbrownianfractionmean} with a
                        standard deviation of
                        {\tsunfairbrownianfractionstddev}. This data
                        is superimposed on the data presented in
                        Figure~\ref{\SETLABEL:NH}. The area under the
                        four curves is identical.}
                    \label{\SETLABEL:TSUNFAIRBROWNIAN1}
                \end{minipage}
            \end{center}
        \end{figure}

% Local Variables:
% TeX-parse-self: t
% TeX-auto-save: t
% TeX-master: "fractal.tex"
% End:


        %
% -----------------------------------------------------------------------------
%
% A license is hereby granted to reproduce this software source code and
% to create executable versions from this source code for personal,
% non-commercial use.  The copyright notice included with the software
% must be maintained in all copies produced.
%
% THIS PROGRAM IS PROVIDED "AS IS". THE AUTHOR PROVIDES NO WARRANTIES
% WHATSOEVER, EXPRESSED OR IMPLIED, INCLUDING WARRANTIES OF
% MERCHANTABILITY, TITLE, OR FITNESS FOR ANY PARTICULAR PURPOSE.  THE
% AUTHOR DOES NOT WARRANT THAT USE OF THIS PROGRAM DOES NOT INFRINGE THE
% INTELLECTUAL PROPERTY RIGHTS OF ANY THIRD PARTY IN ANY COUNTRY.
%
% Copyright (c) 1994-2006, John Conover, All Rights Reserved.
%
% Comments and/or bug reports should be addressed to:
%
%     john@email.johncon.com (John Conover)
%
% -----------------------------------------------------------------------------
%
% Revision: \RCSRevision \\
% Revision Time: \RCSTime UMT \\
% Revision Date: \RCSDate \\
% Revision Id: \RCSId \\
% Revision File: \RCSLog \\
\RCS $Revision: 0.0 $
\RCS $Date: 2006/01/20 04:38:13 $
\RCS $Id: maximum.tex,v 0.0 2006/01/20 04:38:13 john Exp $
% $Log: maximum.tex,v $
% Revision 0.0  2006/01/20 04:38:13  john
% Initial version
%
%
    \subsection{Simulation of Fixed Increment Approximation for Optimally Maximal Fiscal Strategy}
        \label{\SETLABEL:MAXSHANNON}
        \subidx{\market}{fiscal strategy, simulation}
        \subidx{\market}{maximum Shannon probability}
        \subidx{markets}{simulation}
        \subidx{simulation}{markets}
        \subidx{strategy}{optimum fiscal, simulation}
        \subidx{fiscal}{optimum strategy, simulation}
        \subidx{programs}{tsunfairbrownian}
        \subidx{tsunfairbrownian}{program}
        \subidx{Shannon}{probability}
        \subidx{probability}{Shannon}

        \subidx{strategy}{exploitable}
        \subidx{exploitable}{strategy}
        \subidx{programs}{tsshannonmax}
        \subidx{tsshannonmax}{program}
        \subidx{programs}{tsunfairbrownian}
        \subidx{tsunfairbrownian}{program}
        \subidx{strategy}{fiscal}
        \subidx{fiscal}{strategy}
        The data in this section is presented in tabular form in
        Section~\ref{\SETLABELREF:MAXSHANNON}. One of the issues of
        analysis, as mentioned in Section~\ref{\SETLABEL:OPS}, is to
        determine the maximum Shannon probability for the time series
        presented in Figure~\ref{\SETLABEL:TS}. Potentially, this
        could be exploited with an aggressive fiscal
        strategy. Figure~\ref{\SETLABEL:SHANNONMAX0} is a graph of the
        output of the {\it tsshannonmax}\/ program, which is described
        briefly in appendix~\ref{programs}. The maximum of this
        function is the maximum Shannon probability for the time
        series data presented in Figure~\ref{\SETLABEL:TS}.
        Figure~\ref{\SETLABEL:SHANNONMAX1} was constructed using {\it
        tsunfairbrownian}\/ program, which is also described in
        appendix~\ref{programs}, with the maximum Shannon probability,
        and the time series data presented in
        Figure~\ref{\SETLABEL:TS}. This represents a ``what if'' the
        investment strategy was changed from a Shannon probability of
        {\shannonlogreturns}, as derived in Section~\ref{\SETLABEL:SP}
        to {\shannonmax}. This process, essentially, extracts the
        random statistical data from the time series presented in
        Figure~\ref{\SETLABEL:TS}, and constructs a new time series,
        using the random statistical data, with a different investment
        strategy.  The program, {\it tsunfairbrownian}\/, essentially,
        constructs the new time series as a Brownian fractal with
        fixed increments.  The ``quality'' of such a reconstruction
        should be subject to adequate scepticism and scrutiny since,
        in all probability, the increments in the original data
        represent a relatively complex process, that may not be
        ``modeled'' with such a simple methodology.

        \begin{figure}[ht]
            \begin{center}
                \begin{minipage}[t]{0.45\textwidth}
                    \epsfxsize=1.0\linewidth
                    \epsffile{\directory/data.tsshannonmax.eps}
                    \caption[{\market}, maximum rate of revenue
                        returns] {{\market}, maximum rate of revenue
                        returns, per {\timescale}, vs. Shannon
                        probability. The maximum rate of revenue
                        returns, per {\timescale}, occurs at a Shannon
                        probability of {\shannonmax}.}
                    \label{\SETLABEL:SHANNONMAX0}
                \end{minipage}
                \hfill
                \begin{minipage}[t]{0.45\textwidth}
                    \epsfxsize=1.0\linewidth
                    \epsffile{\directory/data.tsshannonmax-p.tsunfairbrownian-p.eps}
                    \caption[{\market}, maximum rate of revenue
                        returns] {{\market}, maximum rate of revenue
                        returns, per {\timescale}, at a Shannon
                        probability, of {\shannonmax}, corresponding
                        to a ``wager'' fraction of {\twoponemax}.}
                    \label{\SETLABEL:SHANNONMAX1}
                \end{minipage}
            \end{center}
        \end{figure}

        \subidx{fractional}{Brownian motion}
        \subidx{Brownian motion}{fractional}
        \subidx{Shannon}{probability}
        \subidx{probability}{Shannon}
        \subidx{programs}{tsshannonmax}
        \subidx{tsshannonmax}{program}
        If it is assumed that the time series data set, presented in
        Figure~\ref{\SETLABEL:TS}, constitutes classical Brownian
        motion, then the Shannon probability can be calculated by
        counting the total number of {\timescale}s that the {\market}
        movement was positive, and dividing by the total number of
        {timescale}s represented in the time series. This quotient is
        {\pmax}, as compared with the predicted value from the program
        {\it tsshannonmax}\/ of {\shannonmax}.

% Local Variables:
% TeX-parse-self: t
% TeX-auto-save: t
% TeX-master: "fractal.tex"
% End:


        %
% -----------------------------------------------------------------------------
%
% A license is hereby granted to reproduce this software source code and
% to create executable versions from this source code for personal,
% non-commercial use.  The copyright notice included with the software
% must be maintained in all copies produced.
%
% THIS PROGRAM IS PROVIDED "AS IS". THE AUTHOR PROVIDES NO WARRANTIES
% WHATSOEVER, EXPRESSED OR IMPLIED, INCLUDING WARRANTIES OF
% MERCHANTABILITY, TITLE, OR FITNESS FOR ANY PARTICULAR PURPOSE.  THE
% AUTHOR DOES NOT WARRANT THAT USE OF THIS PROGRAM DOES NOT INFRINGE THE
% INTELLECTUAL PROPERTY RIGHTS OF ANY THIRD PARTY IN ANY COUNTRY.
%
% Copyright (c) 1994-2006, John Conover, All Rights Reserved.
%
% Comments and/or bug reports should be addressed to:
%
%     john@email.johncon.com (John Conover)
%
% -----------------------------------------------------------------------------
%
% Revision: \RCSRevision \\
% Revision Time: \RCSTime UMT \\
% Revision Date: \RCSDate \\
% Revision Id: \RCSId \\
% Revision File: \RCSLog \\
\RCS $Revision: 0.0 $
\RCS $Date: 2006/01/20 04:38:13 $
\RCS $Id: verification.tex,v 0.0 2006/01/20 04:38:13 john Exp $
% $Log: verification.tex,v $
% Revision 0.0  2006/01/20 04:38:13  john
% Initial version
%
%
    \subsection{Qualitative Verification of Fixed Increment Approximation Analysis}
        \label{\SETLABEL:QVA}

        \subidx{\market}{verification of analysis}
        \subidx{verification}{analysis}
        \subidx{analysis}{verification}
        \subidx{quality}{of analysis}
        \subidx{verification}{of methodology}
        \subidx{methodology}{verification of}
        \subidx{Shannon}{probability}
        \subidx{probability}{Shannon}

        This section evaluates various values based on the ``average''
        of the normalized increments presented in
        Figure~\ref{\SETLABEL:TFA}. These values are an approximation
        to a, probably, complex process with a distribution shown in
        Figure~\ref{\SETLABEL:TF}. These values will be used in a
        fixed increment Brownian fractal analysis of the {\market},
        and may, or may not, provide adequate accuracy for
        projections.

        The data in this section is presented in tabular form in
        sections~\ref{\SETLABELREF:VI1} and~\ref{\SETLABELREF:VI2}.
        As a subjective evaluation of the ``quality'' of the analysis
        of the {\market}, from Chapter~\ref{methodology},
        Equation~\ref{metricvalues1}, and using the mean and root mean
        square values of the normalized increments of the time series
        data presented in Figure~\ref{\SETLABEL:TS} from
        Figure~\ref{\SETLABEL:TF}, and the Shannon probability as
        calculated by counting the total number of {\timescale}s that
        the {\market} movement was positive, as presented in
        Section~\ref{\SETLABEL:MAXSHANNON}:

        \begin{eqnarray}
                  P & \approx & \frac{\frac{avg}{rms} + 1}{2}\\
            {\pmax} & \approx & \frac{\frac{\datafractionmean}{\datafractionrms} + 1}{2}\\
            {\pmax} & \approx & {\avgrms}
            \label{\SETLABEL:AVGS}
        \end{eqnarray}

        \subidx{Shannon}{probability}
        \subidx{probability}{Shannon}
        \noindent and comparing these values to the Shannon
        probability, as found by the {\it tsshannonmax}\/ program, which
        iterates for a maximum:

        \begin{eqnarray}
            {\pmax} \approx {\avgrms} \approx {\shannonmax}
        \end{eqnarray}

        \subidx{logarithmic}{returns}
        \subidx{returns}{logarithmic}
        In addition, the different methods of calculating the
        logarithmic returns, presented in Section~\ref{\SETLABEL:FS},
        should be compared. The four methods used were the mean of
        Figure~\ref{\SETLABEL:TF}, the constant in the least squares
        approximation to Figure~\ref{\SETLABEL:TF}, the least squares
        exponential approximation to Figure~\ref{\SETLABEL:TS}, and
        the logarithmic returns of Figure~\ref{\SETLABEL:TS}, derived
        as the mean of the logarithms of the quotients of the
        increments. The values for each of the methods are,
        respectively:

        \begin{equation}
            \datafractionmeanbits \approx \datafractionconstantbits \approx \datatslsqepbits \approx \logreturns
        \end{equation}

        It is implied in Section~\ref{\SETLABEL:FS},
        Subsection~\ref{\SETLABEL:SP} and in
        Section~\ref{\SETLABEL:TSUNFAIRBROWNIAN} that, a Brownian
        motion with fixed increments fractal may ``model'' the
        {\market}. Using Equation~\ref{stddev9} from
        Chapter~\ref{general}, Section~\ref{abmfi}:

        \begin{eqnarray}
                                    rms \left(2P - 1\right) & \approx & \frac{\sigma \left(2P - 1\right)}{2 \sqrt{P\left(1 - P\right)}}\\
            \datafractionrms \left(2 \cdot \pmax - 1\right) & \approx & \frac{\datafractionstddev \left(2 \cdot \pmax - 1\right)}{2\sqrt{\pmax \left(1 - \pmax\right)}}\\
                       \datafractionrms \cdot \twopminusone & \approx & \datafractionstddev \cdot \twopx\\
                                                      \rmsp & \approx & \sigmap
        \end{eqnarray}

        \noindent and, equating to the mean:

        \begin{equation}
            \datafractionmean \approx \rmsp \approx \sigmap
        \end{equation}

        \subidx{Shannon}{probability}
        \subidx{probability}{Shannon}
        \noindent where, as in Equation~\ref{\SETLABEL:AVGS} using the
        mean, root mean square, and standard deviation values of the
        normalized increments of the time series data presented in
        Figure~\ref{\SETLABEL:TS} from Figure~\ref{\SETLABEL:TF}, and
        the Shannon probability as calculated by counting the total
        number of {\timescale}s that the {\market} movement was
        positive, as presented in Section~\ref{\SETLABEL:MAXSHANNON}.

        As a final qualitative comparison, the absolute value of the
        normalized increments should be the same as the root mean
        square value\footnote{The absolute value of the normalized
        increments, when averaged, is related to the root mean square
        of the increments by a constant. If the normalized increments
        are a fixed increment, the constant is unity. If the
        normalized increments have a Gaussian distribution, the
        constant is $\approx 0.8$ depending on the accuracy of of
        ``fit'' to a Gaussian distribution.}, where the absolute value
        is presented in Figure~\ref{\SETLABEL:TFA}, and the root mean
        square value is presented in Figure~\ref{\SETLABEL:TF}:

        \begin{equation}
            \datafractionabsmean \approx \datafractionrms
        \end{equation}

        Note, that if the {\market} could be ``modeled'' as a Brownian
        motion with fixed increments fractal, then the standard
        deviation of the absolute value of the normalized increments
        of the time series data presented in Figure~\ref{\SETLABEL:TS}
        from Figure~\ref{\SETLABEL:TF} should be zero. It is
        $\datafractionabsstddev$.

% Local Variables:
% TeX-parse-self: t
% TeX-auto-save: t
% TeX-master: "fractal.tex"
% End:


    \renewcommand{\market}{Coins Tossing Game}
    \renewcommand{\directory}{../markets/tscoins}
    \renewcommand{\datafractionmean}{0.008052}
\renewcommand{\datafractionmeanbits}{0.011570}
\renewcommand{\datafractionmeanq}{0.002684}
\renewcommand{\datafractionmeanbitsq}{0.003867}
\renewcommand{\datafractionstddev}{0.038579}
\renewcommand{\datafractionrms}{0.039311}
\renewcommand{\avgrms}{0.602414}
\renewcommand{\ncompanies}{5.210454}
\renewcommand{\pncompanies}{0.544866}
\renewcommand{\datafractionabsmean}{0.029745}
\renewcommand{\datafractionabsstddev}{0.025769}
\renewcommand{\datafractionconstant}{0.010041}
\renewcommand{\datafractionconstantbits}{0.014414}
\renewcommand{\datafractionconstantq}{0.003347}
\renewcommand{\datafractionconstantbitsq}{0.004821}
\renewcommand{\datafractionslope}{-0.000021}
\renewcommand{\datafractionabsconstant}{0.035145}
\renewcommand{\datafractionabsslope}{-0.000057}
\renewcommand{\hurstall}{0.659558}
\renewcommand{\hurstlow}{0.707509}
\renewcommand{\hurstlowtwo}{1.415018}
\renewcommand{\hurstlowhundred}{70.750900}
\renewcommand{\hcalcall}{0.184942}
\renewcommand{\hcalclow}{0.102042}
\renewcommand{\shannonmax}{0.604167}
\renewcommand{\twoponemax}{0.208334}
\renewcommand{\logreturns}{0.010456}
\renewcommand{\twologreturns}{1.007274}
\renewcommand{\twologreturnshundred}{0.727387}
\renewcommand{\oneoverlogreturns}{95.638868}
\renewcommand{\pmax}{0.602094}
\renewcommand{\twopminusone}{0.204188}
\renewcommand{\rmsp}{0.008027}
\renewcommand{\twopx}{0.208583}
\renewcommand{\sigmap}{0.008047}
\renewcommand{\tsunfairbrownianfractionmean}{0.007862}
\renewcommand{\tsunfairbrownianfractionstddev}{0.038619}
\renewcommand{\shannonlogreturns}{0.560125}
\renewcommand{\shannonlogreturnshundred}{56.012500}
\renewcommand{\twopone}{0.120250}
\renewcommand{\twoponehundred}{12.025000}
\renewcommand{\hundredtwoponehundred}{87.975000}
\renewcommand{\hundredshannonlogreturnshundred}{43.987500}
\renewcommand{\datatslsqepbits}{0.007623}
\renewcommand{\thurstall}{0.633980}
\renewcommand{\thurstlow}{0.710108}
\renewcommand{\thurstlowtwo}{1.420216}
\renewcommand{\thurstlowhundred}{71.010800}
\renewcommand{\thcalcall}{0.247886}
\renewcommand{\thcalclow}{0.171737}
\renewcommand{\chisquared}{2.862000}
\renewcommand{\critical}{42.557000}

    \renewcommand{\timescale}{tosses}
    \subidx{market}{\market}
    \idx{\market}

    \section{\market}

        \renewcommand{\SETLABEL}{\LABPRE:CST}
        \renewcommand{\SETLABELQ}{\LABPRE:CSTQ}
        \label{\SETLABEL}
        \renewcommand{\SETLABELREF}{\LABPREREF:CST}

        \subidx{tscoins}{program}
        \subidx{programs}{tscoins}
        For the analysis, the data was in the directory
        {\directory}\footnote{As a simulation model, the program {\it
        tscoins}\/ was run to make a time series data file, with the
        following parameters:

        \vspace{0.1in}
        {\noindent}tscoins -p 0.6 300 > data
        \vspace{0.1in}

        \noindent to make a time series of 300 elements, with a
        Shannon probability of 0.6.  The data is by {\timescale}.}.

        The data in this section is presented in tabular form in
        Section~\ref{\SETLABELREF}. Note that in this analysis, the
        rate of revenue returns means the increase or decrease in the
        cumulative sum of the {\market}. This is included for
        ``theoretical'' comparative purposes.

        %
% -----------------------------------------------------------------------------
%
% A license is hereby granted to reproduce this software source code and
% to create executable versions from this source code for personal,
% non-commercial use.  The copyright notice included with the software
% must be maintained in all copies produced.
%
% THIS PROGRAM IS PROVIDED "AS IS". THE AUTHOR PROVIDES NO WARRANTIES
% WHATSOEVER, EXPRESSED OR IMPLIED, INCLUDING WARRANTIES OF
% MERCHANTABILITY, TITLE, OR FITNESS FOR ANY PARTICULAR PURPOSE.  THE
% AUTHOR DOES NOT WARRANT THAT USE OF THIS PROGRAM DOES NOT INFRINGE THE
% INTELLECTUAL PROPERTY RIGHTS OF ANY THIRD PARTY IN ANY COUNTRY.
%
% Copyright (c) 1994-2006, John Conover, All Rights Reserved.
%
% Comments and/or bug reports should be addressed to:
%
%     john@email.johncon.com (John Conover)
%
% -----------------------------------------------------------------------------
%
% Revision: \RCSRevision \\
% Revision Time: \RCSTime UMT \\
% Revision Date: \RCSDate \\
% Revision Id: \RCSId \\
% Revision File: \RCSLog \\
\RCS $Revision: 0.0 $
\RCS $Date: 2006/01/20 04:38:13 $
\RCS $Id: fraction.tex,v 0.0 2006/01/20 04:38:13 john Exp $
% $Log: fraction.tex,v $
% Revision 0.0  2006/01/20 04:38:13  john
% Initial version
%
%
    \subsection{Time Series Increments Analysis}
        \label{\SETLABEL:TSA}

        \subidx{\market}{Time series analysis}
        \subidx{time series}{increments}
        \subidx{time series}{analysis}
        \subidx{cumulative sum}{analysis}
        \subidx{analysis}{cumulative sum}
        \subidx{analysis}{random process}
        \subidx{random process}{analysis}
        \subidx{Gaussian}{increments}
        \subidx{increments}{Gaussian}
        \subidx{Brownian}{motion, fractional}
        \subidx{fractional}{Brownian motion}
        \subidx{fractal}{Brownian motion}
        The data in this section is presented in tabular form in
        Section~\ref{\SETLABELREF:TSA}.  Figure~\ref{\SETLABEL:TS} is
        a graph of the time series data for the {\market}.

        \subidx{increments}{normalized}
        \subidx{normalized}{increments}
        \subidx{programs}{tsfraction}
        \subidx{tsfraction}{program}
        Figure~\ref{\SETLABEL:TF} is a graph of the normalized
        increments of the time series data presented in
        Figure~\ref{\SETLABEL:TS}. The data presented was made by
        running the program {\it tsfraction}\/ on the time series
        data. The program {\it tsfraction}\/ is described briefly in
        Appendix~\ref{programs}, and subtracts the previous value from
        the next value, dividing this difference by the previous
        value, for each element in the time series data. The new time
        series contains the instantaneous change in the rate of
        revenue returns, divided by the magnitude of the instantaneous
        rate of revenue returns.

        \subidx{mean}{standard deviation}
        \subidx{standard deviation}{mean}
        \idx{root mean square}
        \idx{least squares approximation}
        \begin{figure}[ht]
            \begin{center}
                \begin{minipage}[t]{0.45\textwidth}
                    \epsfxsize=1.0\linewidth
                    \epsffile{\directory/data.eps}
                    \caption{{\market}, time series data.}
                    \label{\SETLABEL:TS}
                    \label{\SETLABELQ:TS}
                \end{minipage}
                \hfill
                \begin{minipage}[t]{0.45\textwidth}
                    \epsfxsize=1.0\linewidth
                    \epsffile{\directory/data.tsfraction.eps}
                    \caption[{\market}, normalized
                        increments]{{\market}, normalized increments
                        of the time series data presented in
                        Figure~\ref{\SETLABEL:TS}. The mean is
                        {\datafractionmean} with a standard deviation
                        of {\datafractionstddev}. The formula for the
                        least squares approximation is
                        ${\datafractionconstant} +
                        {\datafractionslope}t$, and the root mean
                        squared value is {\datafractionrms}. The
                        graph, labeled ``data\-.tsfraction\-.tsrms,''
                        is the running root mean square, and
                        ``data\-.tsfraction\-.tsavg'' is the running
                        average of the normalized increments.  This
                        graph is the fraction of change in the time
                        series, as a function of time. Note that the
                        slope of the mean, {\datafractionslope}, is
                        the coefficient of the nonlinearity term in
                        the normalized increments. See
                        Chapter~\ref{general}, Section~\ref{nlextend}
                        for a possible application of the logistic
                        function to this data set.}
                    \label{\SETLABEL:TF}
                    \label{\SETLABELQ:TF}
                \end{minipage}
            \end{center}
        \end{figure}

        \subidx{absolute value}{increments}
        \subidx{increments}{absolute value}

        Figure~\ref{\SETLABEL:TFA} is a graph of the absolute value of
        the normalized increments of the time series data presented in
        Figure~\ref{\SETLABEL:TF}. The data presented was made by
        running the Unix utility sed(1) on the normalized increments
        time series data to remove the negative signs. This is an
        absolute value procedure.  The resulting time series contains
        the absolute value of the instantaneous change in the rate of
        revenue returns, divided by the magnitude of the instantaneous
        rate of revenue returns\footnote{The absolute value of the
        normalized increments, when averaged, is related to the root
        mean square of the increments by a constant. If the normalized
        increments are a fixed increment, the constant is unity. If
        the normalized increments have a Gaussian distribution, the
        constant is $\approx 0.8$ depending on the accuracy of of
        ``fit'' to a Gaussian distribution.}.

        \subidx{histogram}{normalized}
        \subidx{normalized}{histogram}
        \subidx{programs}{tsnormal}
        \subidx{tsnormal}{program}
        \subidx{mean}{standard deviation}
        \subidx{standard deviation}{mean}
        \idx{root mean square}
        \idx{least squares approximation}
        \subidx{\market}{analysis of increments}
        Figure~\ref{\SETLABEL:NH} is the normalized histogram of the
        normalized increments of the time series data shown in
        Figure~\ref{\SETLABEL:TF}. The abscissa is 3 $\sigma$ limits,
        and the area under the two curves is identical. The data for
        this figure was produced by the program {\it tsnormal}\/,
        which is described briefly in Appendix~\ref{programs}.

        \begin{figure}[ht]
            \begin{center}
                \begin{minipage}[t]{0.45\textwidth}
                    \epsfxsize=1.0\linewidth
                    \epsffile{\directory/data.tsfraction.abs.eps}
                    \caption[{\market}, absolute value of the
                        normalized increments]{{\market}, absolute
                        value of the normalized increments of the time
                        series data presented in
                        Figure~\ref{\SETLABEL:TF}.  The mean is
                        {\datafractionabsmean} with a standard
                        deviation of {\datafractionabsstddev}. The
                        formula for the least squares approximation is
                        ${\datafractionabsconstant} +
                        {\datafractionabsslope}t$, and the root mean
                        square value, from Figure~\ref{\SETLABEL:TF},
                        is {\datafractionrms}.  The graph, labeled
                        ``data\-.tsfraction\-.tsrms,'' is the running
                        root mean square, and
                        ``data\-.tsfraction\-.tsavg'' is the running
                        average of the normalized increments presented
                        in Figure~\ref{\SETLABEL:TF}, superimposed
                        here for convenience. This graph is the
                        absolute value of the fraction of change in
                        the time series, as a function of time.}
                    \label{\SETLABEL:TFA}
                    \label{\SETLABELQ:TFA}
                \end{minipage}
                \hfill
                \begin{minipage}[t]{0.45\textwidth}
                    \epsfxsize=1.0\linewidth
                    \epsffile{\directory/data.tsfraction.tsnormal-s30.eps}
                    \caption[{\market}, normalized histogram of the
                        normalized increments]{{\market}, normalized
                        histogram of the normalized increments of the
                        time series data shown in
                        Figure~\ref{\SETLABEL:TF}.  The data has a
                        mean of {\datafractionmean}, with a standard
                        deviation of {\datafractionstddev}.  The area
                        under the two curves is identical. The
                        $\chi^2$ value of the observed and expected
                        values of the two curves is {\chisquared},
                        with a critical value of {\critical}.}
                    \label{\SETLABEL:NH}
                \end{minipage}
            \end{center}
        \end{figure}

        \subidx{programs}{tsXsquared}
        \subidx{tsXsquared}{program}
        \subidx{\market}{chi-squared values of increments}
        The program {\it tsXsquared}\/, which is briefly described in
        appendix~\ref{programs}, was used to derive the $\chi^2$
        statistics for the data presented in
        Figure~\ref{\SETLABEL:NH}.

        \subidx{programs}{tsstatest}
        \subidx{tsstatest}{program}
        \subidx{\market}{statistical estimates}

        Figure~\ref{\SETLABEL:SE} is the statistical estimate for the
        data presented in Figure~\ref{\SETLABEL:TF}, as derived by the
        program {\it tsstatest}\/, which is briefly described in
        appendix~\ref{programs}.

        \begin{figure}[ht]
            \begin{center}
                \begin{minipage}[t]{\textwidth}
                    \center{\fbox{\parbox{0.9\textwidth}{\XXX{\directory/data.tsstatest-f0.1-c0.9-i.tex}}}}
                    \caption[{\market}, statistical estimates of the
                        normalized increments]{{\market}, statistical
                        estimates of the normalized increments of the
                        time series shown in Figure~\ref{\SETLABEL:TF}.
                        The table was produced with the {\it
                        tsstatest}\/ program, and illustrates the
                        size of the data set required for a confidence
                        level of 90\%, with an error estimate of $\pm$
                        10\%, or alternately, the error estimate on
                        the time series shown in Figure~\ref{\SETLABEL:TF}.}
                    \label{\SETLABEL:SE}
                \end{minipage}
            \end{center}
        \end{figure}

        Note that the data set size estimations, as produced by the
        {\it tsstatest}\/ program, are probably very conservative,
        depending on the magnitude of the Shannon probability, $P =
        \shannonlogreturns$, as derived in
        Section~\ref{\SETLABEL:SP}. See Chapter~\ref{general},
        Section~\ref{serdss} for possible alternative methodologies
        for addressing the analysis of fractal time series with
        limited data set sizes. Depending on the magnitude of the
        Shannon probability, $P$, these estimates can be several
        orders of magnitude too high.

        \subidx{derivative of increments}{normalized}
        \subidx{normalized}{derivative of increments}
        \subidx{programs}{tsderivative}
        \subidx{tsderivative}{program}
        Figure~\ref{\SETLABEL:TF1} is the normalized histogram of the
        first derivative of the normalized increments of the time
        series data shown in Figure~\ref{\SETLABEL:TF}. In principle,
        if the distribution of the normalized increments presented in
        Figure~\ref{\SETLABEL:NH} is Gaussian in nature, this
        distribution would be similar to ``white noise,'' as presented
        in appendix~\ref{programs}, Figure~\ref{whiteexample}. The
        data was generated by the {\it tsderivative}\/ program, which
        is briefly described in
        appendix~\ref{programs}. Figure~\ref{\SETLABEL:TF2} is the
        normalized histogram of the second derivative of the
        normalized increments of the time series data shown in
        Figure~\ref{\SETLABEL:TF}. In principle, if the distribution
        of the normalized increments presented in
        Figure~\ref{\SETLABEL:NH} is an integrated Gaussian
        distribution in nature, this distribution would be similar to
        ``white noise,'' as presented in appendix~\ref{programs},
        Figure~\ref{whiteexample}.

        \begin{figure}[ht]
            \begin{center}
                \begin{minipage}[t]{0.45\textwidth}
                    \epsfxsize=1.0\linewidth
                    \epsffile{\directory/data.tsfraction.tsderivative.tsnormal-s30.eps}
                    \caption[{\market}, histogram of the first
                        derivative of the increments]{{\market},
                        normalized histogram of the first derivative
                        of the normalized increments of the time
                        series data shown in
                        Figure~\ref{\SETLABEL:TF}.}
                    \label{\SETLABEL:TF1}
                \end{minipage}
                \hfill
                \begin{minipage}[t]{0.45\textwidth}
                    \epsfxsize=1.0\linewidth
                    \epsffile{\directory/data.tsfraction.2tsderivative.tsnormal-s30.eps}
                    \caption[{\market}, histogram of the second
                        derivative of the increments]{{\market},
                        normalized histogram of second derivative of
                        the the normalized increments of the time
                        series data shown in
                        Figure~\ref{\SETLABEL:TF}.}
                    \label{\SETLABEL:TF2}
                \end{minipage}
            \end{center}
        \end{figure}

        \subidx{fractal}{range}
        \subidx{fractal}{R/S analysis}
        \subidx{\market}{rate of revenue returns, range}
        \subidx{\market}{deterministic mechanism}
        \subidx{deterministic}{mechanism}
        \subidx{mechanism}{deterministic}
        Figure~\ref{\SETLABEL:TR} is the range of values of the time
        series shown in Figure~\ref{\SETLABEL:TS}. The horizontal axis
        is time into the future. In principle, if the time series was
        characterized as fractional Brownian motion the graph in
        Figure~\ref{\SETLABEL:TR} would be a square root
        function\footnote{Note that the ``roughness,'' or ``sawtooth''
        characteristics of the graph in Figure~\ref{\SETLABEL:TR} are
        a computational artifact---caused by not using the -m option
        to the program {\it tshurst}\/, which is computationally
        inefficient.}. Figure~\ref{\SETLABEL:TD} is the deterministic
        map of the normalized increments of the time series data shown
        in Figure~\ref{\SETLABEL:TF}. The deterministic map is useful
        for determining if a time series was created by a
        deterministic mechanism. This, essentially, maps each element
        in the time series with the previous element in the time
        series.  See,~\cite[pp. 745]{Peitgen}.

        \begin{figure}[ht]
            \begin{center}
                \begin{minipage}[t]{0.45\textwidth}
                    \epsfxsize=1.0\linewidth
                    \epsffile{\directory/data.tshurst-f.eps}
                    \caption[{\market}, range]{{\market}, range of the
                        time series data shown in
                        Figure~\ref{\SETLABEL:TS}.}
                    \label{\SETLABEL:TR}
                \end{minipage}
                \hfill
                \begin{minipage}[t]{0.45\textwidth}
                    \epsfxsize=1.0\linewidth
                    \epsffile{\directory/data.tsfraction.tsdeterministic.eps}
                    \caption[{\market}, deterministic map]{{\market},
                        deterministic map of the normalized increments
                        of the time series data shown in
                        Figure~\ref{\SETLABEL:TF}.}
                    \label{\SETLABEL:TD}
                \end{minipage}
            \end{center}
        \end{figure}

% Local Variables:
% TeX-parse-self: t
% TeX-auto-save: t
% TeX-master: "fractal.tex"
% End:


            Figure~\ref{\SETLABEL:NH} would seem to indicate that the
            time series data for the {\market} represents a cumulative
            sum/integration of a random process that has a Gaussian
            distribution, (ie., satisfies the Gaussian increments
            property of fractional Brownian
            motion~\cite[pp. 250]{Crownover},) tending to justify the
            assumption that the time series data represents fractional
            Brownian motion.

        %
% -----------------------------------------------------------------------------
%
% A license is hereby granted to reproduce this software source code and
% to create executable versions from this source code for personal,
% non-commercial use.  The copyright notice included with the software
% must be maintained in all copies produced.
%
% THIS PROGRAM IS PROVIDED "AS IS". THE AUTHOR PROVIDES NO WARRANTIES
% WHATSOEVER, EXPRESSED OR IMPLIED, INCLUDING WARRANTIES OF
% MERCHANTABILITY, TITLE, OR FITNESS FOR ANY PARTICULAR PURPOSE.  THE
% AUTHOR DOES NOT WARRANT THAT USE OF THIS PROGRAM DOES NOT INFRINGE THE
% INTELLECTUAL PROPERTY RIGHTS OF ANY THIRD PARTY IN ANY COUNTRY.
%
% Copyright (c) 1994-2006, John Conover, All Rights Reserved.
%
% Comments and/or bug reports should be addressed to:
%
%     john@email.johncon.com (John Conover)
%
% -----------------------------------------------------------------------------
%
% Revision: \RCSRevision \\
% Revision Time: \RCSTime UMT \\
% Revision Date: \RCSDate \\
% Revision Id: \RCSId \\
% Revision File: \RCSLog \\
\RCS $Revision: 0.0 $
\RCS $Date: 2006/01/20 04:38:13 $
\RCS $Id: instant.tex,v 0.0 2006/01/20 04:38:13 john Exp $
% $Log: instant.tex,v $
% Revision 0.0  2006/01/20 04:38:13  john
% Initial version
%
%
    \subsection{Instantaneous Analysis of Normalized Increments}
        \label{\SETLABEL:IA}

        \subidx{\market}{instantaneous analysis of normalized increments}
        \idx{average of normalized increments}
        \idx{root mean square of normalized increments}
        \subidx{Shannon probability}{instantaneous computation of}
        \subidx{average of normalized increments}{instantaneous computation of}
        \subidx{root mean square of normalized increments}{instantaneous computation of}
        \subidx{instantaneous computation}{Shannon probability}
        \subidx{instantaneous computation}{average of normalized increments}
        \subidx{instantaneous computation}{root mean square of normalized increments}
        \idx{time series}
        \subidx{time series}{instantaneous analysis}
        \subidx{instantaneous analysis}{time series}
        \subidx{time series}{increments}
        \subidx{time series}{analysis}
        \subidx{Shannon}{probability}
        \subidx{probability}{Shannon}
        \subidx{normalized}{increments}
        \subidx{increments}{normalized}

        The program {\it tsinstant}\/, which is briefly described in
        Appendix~\ref{programs}, is for finding the instantaneous
        fraction of change in a time series. The value of a sample in
        the time series is subtracted from the previous sample in the
        time series, and divided by the value of the previous sample.
        As explained in Chapter~\ref{general},
        Sections~\ref{derivation},~\ref{GA},~\ref{abmfi},~\ref{aftsma}
        and,~\ref{ompl} for Brownian motion, random walk fractals, the
        absolute value of the instantaneous fraction of change is also
        the root mean square of the instantaneous fraction of
        change\footnote{The absolute value of the normalized
        increments, when averaged, is related to the root mean square
        of the increments by a constant. If the normalized increments
        are a fixed increment, the constant is unity. If the
        normalized increments have a Gaussian distribution, the
        constant is $\approx 0.8$ depending on the accuracy of of
        ``fit'' to a Gaussian distribution.}. Squaring this value is
        the average of the instantaneous fraction of change, and
        adding unity to the absolute value of the instantaneous
        fraction of change, and dividing by two, is the Shannon
        probability of the instantaneous fraction of change.

        Figure~\ref{\SETLABEL:IA1} is the instantaneous value of the
        root mean square of the normalized increments for the
        {\market}, and Figure~\ref{\SETLABEL:IA2} is the instantaneous
        Shannon probability for the normalized increments.

        \begin{figure}[ht]
            \begin{center}
                \begin{minipage}[t]{0.45\textwidth}
                    \epsfxsize=1.0\linewidth
                    \epsffile{\directory/data.tsinstant-r.eps}
                    \caption[{\market}, instantaneous value of
                        rms.]{{\market}, instantaneous value of the
                        root mean square of the normalized increments,
                        provided by running the program {\it
                        tsinstant}\/ with the -r option on the data
                        presented in Figure~\ref{\SETLABEL:TS}.}
                    \label{\SETLABEL:IA1}
                    \label{\SETLABELQ:IA1}
                \end{minipage}
                \hfill
                \begin{minipage}[t]{0.45\textwidth}
                    \epsfxsize=1.0\linewidth
                    \epsffile{\directory/data.tsinstant-s.eps}
                    \caption[{\market}, instantaneous value of
                        Shannon probability.]{{\market}, instantaneous
                        value of the Shannon probability of the
                        normalized increments, provided by running the
                        program {\it tsinstant}\/ with the -s option
                        on the data presented in
                        Figure~\ref{\SETLABEL:TS}.}
                    \label{\SETLABEL:IA2}
                    \label{\SETLABELQ:IA2}
                \end{minipage}
            \end{center}
        \end{figure}

% Local Variables:
% TeX-parse-self: t
% TeX-auto-save: t
% TeX-master: "fractal.tex"
% End:


        %
% -----------------------------------------------------------------------------
%
% A license is hereby granted to reproduce this software source code and
% to create executable versions from this source code for personal,
% non-commercial use.  The copyright notice included with the software
% must be maintained in all copies produced.
%
% THIS PROGRAM IS PROVIDED "AS IS". THE AUTHOR PROVIDES NO WARRANTIES
% WHATSOEVER, EXPRESSED OR IMPLIED, INCLUDING WARRANTIES OF
% MERCHANTABILITY, TITLE, OR FITNESS FOR ANY PARTICULAR PURPOSE.  THE
% AUTHOR DOES NOT WARRANT THAT USE OF THIS PROGRAM DOES NOT INFRINGE THE
% INTELLECTUAL PROPERTY RIGHTS OF ANY THIRD PARTY IN ANY COUNTRY.
%
% Copyright (c) 1994-2006, John Conover, All Rights Reserved.
%
% Comments and/or bug reports should be addressed to:
%
%     john@email.johncon.com (John Conover)
%
% -----------------------------------------------------------------------------
%
% Revision: \RCSRevision \\
% Revision Time: \RCSTime UMT \\
% Revision Date: \RCSDate \\
% Revision Id: \RCSId \\
% Revision File: \RCSLog \\
\RCS $Revision: 0.0 $
\RCS $Date: 2006/01/20 04:38:13 $
\RCS $Id: logistic.tex,v 0.0 2006/01/20 04:38:13 john Exp $
% $Log: logistic.tex,v $
% Revision 0.0  2006/01/20 04:38:13  john
% Initial version
%
%
    \subsection{Logistic Analysis}
        \label{\SETLABEL:LA}

        \subidx{\market}{Logistic function analysis}
        \subidx{time series}{logistic function}
        \subidx{logistic function}{time series}
        \subidx{time series}{increments}
        \subidx{time series}{analysis}
        \subidx{cumulative sum}{analysis}
        \subidx{analysis}{cumulative sum}
        \subidx{analysis}{random process}
        \subidx{random process}{analysis}
        The data in this section is presented in tabular form in
        Section~\ref{\SETLABELREF:LAA}.  Figure~\ref{\SETLABEL:LA1} is
        a graph of the logistic function estimates of the time series
        data for the {\market}. The reader is cautioned that these
        graphs are constructed using the method suggested in
        Chapter~\ref{general}, Section~\ref{nlextend} and enormous
        precision is required for adequate prediction of the logistic
        function,~\cite{Modis}. Particularly, the non-linear term will
        usually require intervention to produce a practical fit to the
        data. In addition, there are numerical stability issues with
        logistic function methodologies\footnote{For example, in
        Figures~\ref{\SETLABEL:LA1} and~\ref{\SETLABEL:LA2}, if the
        non-linear term, $b$, was greater than zero, it was set to
        zero to produce the graphs. See Section~\ref{\SETLABELREF:LAA}
        for the actual derived values. In other cases, the magnitude
        of $b$ was too large, resulting in a graph that was decreasing
        as a function of time}.  The methodology should be regarded as
        ``fragile.'' It is included for completeness.

        \idx{least squares approximation}
        Figure~\ref{\SETLABEL:LA1} is a graph of the logistic function
        for the time series data presented in
        Figure~\ref{\SETLABEL:TS}. The data presented was made by
        running the program {\it tsdlogistic}\/, which is described
        briefly in Appendix~\ref{programs}, on the parameters
        extracted from the time series data as suggested in
        Figure~\ref{\SETLABEL:TF}. The program {\it tslsq}\/ was used
        to derive the constant and the slope of the normalized
        increments of the data presented in Figure~\ref{\SETLABEL:TF}.
        Figure~\ref{\SETLABEL:LA2} is the same graph, but with the
        time scale expanded by a factor of two.

        \begin{figure}[ht]
            \begin{center}
                \begin{minipage}[t]{0.45\textwidth}
                    \epsfxsize=1.0\linewidth
                    \epsffile{\directory/data.tsfraction.tslsq-p.tsdlogistic.eps}
                    \caption[{\market}, logistic function
                        estimates.]{{\market}, logistic function
                        estimates, provided by running the {\it
                        tslsq}\/ program on the normalized increments
                        presented in Figure~\ref{\SETLABEL:TF} with
                        the -p option. These parameters were used as
                        arguments to the {\it tsdlogistic}\/ program.}
                    \label{\SETLABEL:LA1}
                    \label{\SETLABELQ:LA1}
                \end{minipage}
                \hfill
                \begin{minipage}[t]{0.45\textwidth}
                    \epsfxsize=1.0\linewidth
                    \epsffile{\directory/data.tsfraction.tslsq-p.tsdlogistic2.eps}
                    \caption[{\market}, logistic function
                        estimates.]{{\market}, logistic function
                        estimates of Figure~\ref{\SETLABEL:LA1} with
                        the time scale expanded by a factor of two.}
                    \label{\SETLABEL:LA2}
                    \label{\SETLABELQ:LA2}
                \end{minipage}
            \end{center}
        \end{figure}

% Local Variables:
% TeX-parse-self: t
% TeX-auto-save: t
% TeX-master: "fractal.tex"
% End:


        %
% -----------------------------------------------------------------------------
%
% A license is hereby granted to reproduce this software source code and
% to create executable versions from this source code for personal,
% non-commercial use.  The copyright notice included with the software
% must be maintained in all copies produced.
%
% THIS PROGRAM IS PROVIDED "AS IS". THE AUTHOR PROVIDES NO WARRANTIES
% WHATSOEVER, EXPRESSED OR IMPLIED, INCLUDING WARRANTIES OF
% MERCHANTABILITY, TITLE, OR FITNESS FOR ANY PARTICULAR PURPOSE.  THE
% AUTHOR DOES NOT WARRANT THAT USE OF THIS PROGRAM DOES NOT INFRINGE THE
% INTELLECTUAL PROPERTY RIGHTS OF ANY THIRD PARTY IN ANY COUNTRY.
%
% Copyright (c) 1994-2006, John Conover, All Rights Reserved.
%
% Comments and/or bug reports should be addressed to:
%
%     john@email.johncon.com (John Conover)
%
% -----------------------------------------------------------------------------
%
% Revision: \RCSRevision \\
% Revision Time: \RCSTime UMT \\
% Revision Date: \RCSDate \\
% Revision Id: \RCSId \\
% Revision File: \RCSLog \\
\RCS $Revision: 0.0 $
\RCS $Date: 2006/01/20 04:38:13 $
\RCS $Id: hurst.tex,v 0.0 2006/01/20 04:38:13 john Exp $
% $Log: hurst.tex,v $
% Revision 0.0  2006/01/20 04:38:13  john
% Initial version
%
%
    \subsection{Hurst Coefficient Analysis}
        \label{\SETLABEL:H}

        \subidx{\market}{Hurst coefficient analysis}
        \subidx{Hurst coefficient}{analysis}
        \subidx{increments}{normalized}
        \subidx{normalized}{increments}
        \subidx{programs}{tshurst}
        \subidx{tshurst}{program}
        The data in this section is presented in tabular form in
        Section~\ref{\SETLABELREF:HCHP}. Figure~\ref{\SETLABEL:HC} is
        a graph of the Hurst coefficient data time series data shown
        in Figure~\ref{\SETLABEL:TS}. The slope of the graph is the
        Hurst coefficient.  The data for this figure was produced by
        the program {\it tshurst}\/, which is described briefly in
        Appendix~\ref{programs}.

        \subidx{\market}{H parameter analysis}
        \subidx{H parameter}{analysis}
        \subidx{programs}{tshcalc}
        \subidx{tshcalc}{program}
        Figure~\ref{\SETLABEL:HP} is a graph of the H parameter data
        for the normalized increments of the time series data shown in
        Figure~\ref{\SETLABEL:TF}. The data for this figure was
        produced by the program {\it tshcalc}\/, which is described
        briefly in Appendix~\ref{programs}.

        \begin{figure}[ht]
            \begin{center}
                \begin{minipage}[t]{0.45\textwidth}
                    \epsfxsize=1.0\linewidth
                    \epsffile{\directory/data.tshurst.eps}
                    \caption[{\market}, Hurst coefficient data]{{\market},
                        Hurst coefficient data for the normalized
                        increments of the time series data shown in
                        Figure~\ref{\SETLABEL:TF}.  The slope of the graph
                        is the Hurst coefficient.}
                    \label{\SETLABEL:HC}
                \end{minipage}
                \hfill
                \begin{minipage}[t]{0.45\textwidth}
                    \epsfxsize=1.0\linewidth
                    \epsffile{\directory/data.tshcalc.eps}
                    \caption[{\market}, H parameter data]{{\market}, H
                        parameter data for the normalized increments of
                        the time series data shown in
                        Figure~\ref{\SETLABEL:TF} The slope of the graph
                        is the H parameter.}
                    \label{\SETLABEL:HP}
                \end{minipage}
            \end{center}
        \end{figure}

        \subidx{revenue}{See, rate of revenue returns}
        \subidx{returns}{See, rate of revenue returns}
        \subidx{\market}{revenues}
        \subidx{Hurst coefficient}{analysis}
        \subidx{\market}{Hurst coefficient analysis}
        \subidx{\market}{rate of change}
        \subidx{\market}{windows of opportunity}
        \subidx{rate of revenue returns}{forecast}
        \subidx{forecast}{rate of revenue returns}
        \idx{windows of opportunity}
        \subidx{programs}{tslsq}
        \subidx{tslsq}{program}

        The approximately linear slope of the graph in
        Figure~\ref{\SETLABEL:HC} implies that the variance of the
        rate of revenue returns, (per {\timescale},) in the {\market},
        $V(t_2 - t_1)$, over a period of time is proportional to the
        period of time raised to twice the Hurst
        coefficient~\cite[pp. 180]{Feder},~\cite[pp. 246]{Crownover}.
        This seems to be a quantitative statement concerning how fast,
        and to what degree, the rate of revenue returns' state of
        affairs can change over a period of time.  An additional
        implication, for Hurst coefficients sufficiently close to 0.5,
        is that the probability of the state of affairs repeating
        sometime in the future goes down with increasing
        time\footnote{It can be shown that the number of expected
        market ``high'' and ``low'' transitions, $N$, scales with the
        square root of time, or $N \propto \sqrt {t}$, meaning that
        the cumulative distribution of the probability, $P$, of the
        duration of a market's ``high'' or ``low'' exceeding a given
        time interval, $t$, is proportional to the reciprocal of the
        square root of the time interval, $P \propto 1 / \sqrt {t}$,
        (or, conversely, that the probability of the duration of a
        market's ``high'' or ``low'' exceeding a given time interval
        is proportional to the reciprocal of the time interval raised
        to the power $3 / 2$, ie., $P \propto 1 / t^{3 /
        2}$,~\cite[pp. 153]{Schroeder}. What this means is that a
        histogram of the ``zero free'' run-lengths of a market being
        ``high'' or ``low,'' over a long time, would have a $1 / t^{3
        / 2}$ characteristic.)}, $t$, $p(t) = erf (1/\sqrt{2t})$ which
        is approximately $1/\sqrt{t}$ for $t \gg
        1$~\cite[pp. 160]{Schroeder}. Figures~\ref{\SETLABEL:FN},
        and,~\ref{\SETLABEL:FF} compare methods of approximation of
        the ``forecastability'' of the rate of revenue returns in the
        {\market} for the near term and far term,
        respectively~\cite[pp. 83-84]{Peters:CAOITCM}\footnote{The
        author is not comfortable with Peters' interpretation. For
        example, if the algorithm explained
        in~\cite[pp. 82]{Peters:CAOITCM} is used on ``white noise''
        which, by definition, never has any correlations, the short
        term Hurst coefficient, and thus the ``forecastability,'' is
        still near unity---a bit of an enigma. This can be verified
        with the {\it tswhite}\/ and {\it tshurst}\/ programs, which
        are briefly described in Appendix~\ref{programs}.}.  This
        seems to be a quantitative statement concerning ``windows of
        opportunity'' in the rate of revenue returns, (per
        {\timescale}.)  The program {\it tslsq}\/ was used on the
        Hurst coefficient data, presented in
        Figure~\ref{\SETLABEL:HC}, to provide a least squares
        approximation to the Hurst coefficient. The superimposed least
        squares approximation with on original Hurst coefficient data
        is presented.  The time series data has a Hurst coefficient of
        {\thurstlow}, so that:

        \subidx{\market}{Hurst coefficient analysis}
        \begin{eqnarray}
            V\left(t_2 - t_1\right) & \propto & \left(t_2 - t_1\right)^{2 \cdot H}\\
            V\left(t_2 - t_1\right) & \propto & \left(t_2 - t_1\right)^{2 \cdot {\thurstlow}}\\
                                    & \propto & \left(t_2 - t_1\right)^{\thurstlowtwo}
            \label{\SETLABEL:V}
        \end{eqnarray}

        \subidx{fractional}{Brownian motion}
        \subidx{Brownian motion}{fractional}
        \idx{fractal}
        \noindent where $V(t_2 - t_1)$ is the variance of the
        increments of the rate of revenue returns, (per {\timescale},)
        over the time interval $t_2 -
        t_1$,~\cite[pp. 177]{Feder},~\cite[pp. 494]{Peitgen}. If $H >
        \frac{1}{2}$, then the time series is termed as being
        characterized by ``fractional Brownian
        motion~\cite[pp. 170]{Feder}.''

        \subidx{rate of revenue returns}{predictability}
        \subidx{rate of revenue returns}{forecastability}
        \subidx{rate of revenue returns}{consistency}
        \subidx{predictability}{rate of revenue returns}
        \subidx{forecastability}{rate of revenue returns}
        \subidx{consistency}{rate of revenue returns}
        \subidx{\market}{rate of revenue returns, predictability}
        \subidx{\market}{rate of revenue returns, forecastability}
        \subidx{\market}{rate of revenue returns, consistency}
        \subidx{Hurst coefficient}{analysis}
        \subidx{\market}{Hurst coefficient analysis}
        \subidx{\market}{rate of change}

        In some sense, the Hurst coefficient is a quantitative
        expression of the ``forecastability'' of the future based on
        the past\footnote{Actually, in general, when summing fractal
        entities, the method used should be a root mean square
        process, dependent on the Hurst Coefficient, $H$, where
        $P_{total}^H = P_1^H + P_2^H + \cdots$, where $P_n$ is the
        fractal entities. For a Brownian motion, or random walk type
        of fractal the Hurst Coefficient is a function of time into
        the future. For the ``near term,'' the Hurst coefficient is
        very near unity, meaning the summation process is linear. For
        the ``long term,'' $H \approx 0.5$, or a standard root mean
        square summation process should be used. If $H$ is $0.5$ then
        the market is termed a Brownian motion, or random walk
        process. If it is larger than 0.5, it is termed fractional
        Brownian motion process. For a random walk process, ``near
        term'' and ``far term'' are quantitatively differentiated on
        the Hurst Coefficient graph where $1 - \ln (t) = 0.5 \cdot \ln
        (t)$, or when $\ln (t) = 2$, or $t = 7.389\ldots$ See
        Section~\ref{\SETLABEL:FS} for the particulars on using Hurst
        Coefficient to sum fractal process' for the {\market}. See
        also~\cite[pp. 67, 83-84]{Peters:CAOITCM} and~\cite[pp. 129,
        159]{Schroeder} for particulars on the implications of the
        Hurst Coefficient and root mean square summation issues.}.  A
        Hurst coefficient of {\thurstlow}, (for the near future, and
        {\thurstall} for the distant future.) implies that the
        likelihood of the rate of revenue returns, (per {\timescale},)
        for any two consecutive {\timescale}s being the same is
        {\thurstlowhundred}\%~\cite[pp. 66]{Peters:CAOITCM} for the
        near future, and {\thurstall} for the distant
        future. Likewise, there is a {\thurstlowhundred}\% chance of
        the rate of revenue returns, (per {\timescale},) movements
        being the same in consecutive time periods---ie., if, in a
        given {\timescale}, the rate of revenue returns, (per
        {\timescale},) is increasing, there is a {\thurstlowhundred}\%
        that the rate of revenue returns, (per {\timescale},) will
        increase in the following period, also. In some sense, this is
        a quantitative statement on how ``predictable,'' or
        ``forecastable'' the rate of revenue returns, (per
        {\timescale},) for the {\market} are over time, since the
        probability of having $n$ many consecutive {\timescale}s of
        the same agenda is $H^n$ where $H$ is the Hurst coefficient,
        or, letting the short term probability of having $n$ many
        {\timescale}s of the same market agenda, $p_a$, is:

        \begin{eqnarray}
            p_a\left(n\right) & = & H^{n}\\
                              & = & {\thurstlow}^{n}
            \label{\SETLABEL:MA}
        \end{eqnarray}

        \subidx{rate of revenue returns}{predictability}
        \subidx{rate of revenue returns}{forecastability}
        \subidx{rate of revenue returns}{consistency}
        \subidx{predictability}{rate of revenue returns}
        \subidx{forecastability}{rate of revenue returns}
        \subidx{consistency}{rate of revenue returns}
        As an interesting interpretation of the normalized increments
        of the time series data presented in
        Figure~\ref{\SETLABEL:TF}, if the vertical axis is multiplied
        by 100, to convert to percent, then the graph represents the
        error, in percent, that would be made by forecasting, month by
        month, that the next {\timescale}'s rate of revenue returns
        would be the same as the current {\timescale}'s revenue
        rate. Interestingly, it is $\datafractionmean \cdot 100$
        percent, on the average, with a standard deviation of
        $\datafractionstddev \cdot 100$ percent, and a root mean
        square error value of $\datafractionrms \cdot 100$
        percent---small values for such a simple forecasting
        mechanism.

        \subidx{\market}{rate of revenue returns, range}
        \subidx{Hurst coefficient}{analysis}
        \subidx{\market}{Hurst coefficient analysis}
        \subidx{\market}{rate of change}

        This is, essentially, a statement of the range of values, in
        the increments of the rate of revenue returns, (per
        {\timescale},) that is to be expected over the time interval,
        $t_2 - t_1$,
        $R_v$,~\cite[pp. 178]{Feder},~\cite[pp. 172]{Cambel}:

        \begin{eqnarray}
            R_v\left(t_2 - t_1\right) & \propto & \left(t_2 - t_1\right)^{H}\\
                                      & \propto & \left(t_2 - t_1\right)^{\thurstlow}
            \label{\SETLABEL:R}
        \end{eqnarray}

        \subidx{\market}{rate of revenue returns, range}
        \subidx{Hurst coefficient}{analysis}
        \subidx{\market}{Hurst coefficient analysis}
        \subidx{\market}{rate of change}
        \subidx{Markov}{statistics}
        \subidx{statistics}{Markov}
        \noindent where $R$ is the range of values in the increments
        of the rate of revenue returns, (per {\timescale}.) A Hurst
        coefficient, $H$, that is much larger than $\frac{1}{2}$, (but
        less than 1,) implies a strongly non-Gaussian distribution in
        the increments of the rate of revenue returns, (per
        {\timescale},)~\cite[pp. 152, 194]{Feder}, and a Hurst
        coefficient near $\frac{1}{2}$ implies that the increments of
        the rate of revenue returns, (per {\timescale}) is
        characteristic of an independent
        process~\cite[pp. 195]{Feder}. Extreme caution should be
        exercised in using Markov statistics in any analysis where the
        Hurst coefficient is not
        $\frac{1}{2}$,~\cite[pp. 124]{Crownover},~\cite[pp. 106]{Peters:CAOITCM}.


        As a useful approximation, if $H$, is approximately
        $\frac{1}{2}$, Equation~\ref{\SETLABEL:R} reduces
        to,~\cite[pp. 129]{Schroeder}:

        \begin{eqnarray}
            R\left(t_2 - t_1\right) & \propto & (t_2 - t_1)^{\frac{1}{2}}\\
                                    & \propto & \sqrt{\left(t_2 - t_1\right)}
        \end{eqnarray}

        \subidx{\market}{rate of revenue returns, range}
        \subidx{\market}{rate of revenue returns, increase and decrease}
        \subidx{Hurst coefficient}{analysis}
        \subidx{\market}{Hurst coefficient analysis}
        \subidx{\market}{rate of change}
        \subidx{Markov}{statistics}
        \subidx{statistics}{Markov}

        In the case where the Hurst coefficient, $H$, is
        $\frac{1}{2}$, the range of values in the increments of the
        rate of revenue returns, (per {\timescale},) divided by the
        standard deviation of these values, $S$, can be anticipated to
        increase over time according to the following
        relation,~\cite[pp. 154]{Feder},~\cite[pp. 129]{Schroeder}:

        \begin{equation}
            \frac{R\left(t_2 - t_1\right)}{S} \propto \left(t_2 - t_1\right)^{\frac{1}{2}}
        \end{equation}

        \subidx{\market}{rate of revenue returns, range}
        \subidx{\market}{rate of revenue returns, increase and decrease}
        \subidx{Hurst coefficient}{analysis}
        \subidx{\market}{Hurst coefficient analysis}
        \subidx{\market}{rate of change}
        \noindent which is a useful conceptual approximation, since it
        involves only the square root function---if the range and the
        standard deviation of the increments of the rate of revenue
        returns, (per {\timescale},) are known, (and $H \approx
        \frac{1}{2}$,) then the expected change in $\frac{R}{S}$, will
        increase with the square root of time\footnote{To be precise,
        it is actually asymptotically proportional to
        $\tau^{\frac{1}{2}}$}.

        Another useful approximation when rescaling processes that are
        characterize by Brownian motion, (ie., when $H \approx
        \frac{1}{2}$,) is that:

        \begin{eqnarray}
            X\left(t\right) & \propto & \frac{X\left(rt\right)}{r^{H}}\\
                            & \propto & \frac{X\left(rt\right)}{r^{\thurstlow}}
        \end{eqnarray}

        \idx{Brownian motion}
        \idx{fractal}
        Where $X(t)$ is the process characterized by Brownian motion,
        and $r$ is a scaling factor,~\cite[pp. 494]{Peitgen}.

        \subidx{programs}{tslsq}
        \subidx{tslsq}{program}
        The program {\it tslsq}\/ was used on the H parameter data,
        presented in Figure~\ref{\SETLABEL:HP}, to provide a least
        squares approximation to the H parameter for the
        {\market}. The superimposed least squares approximation on the
        original H parameter data is presented.  By contrast, the H
        parameter, as derived by the methodology outlined
        in~\cite[pp. 249]{Crownover}, is {\thcalclow} for the near
        future, and {\thcalcall} for the distant future.

        \subidx{\market}{Hurst coefficient analysis}
        \subidx{Hurst coefficient}{analysis}
        \subidx{increments}{normalized}
        \subidx{normalized}{increments}
        \subidx{programs}{tshurst}
        \subidx{tshurst}{program}
        \subidx{\market}{H parameter analysis}
        \subidx{H parameter}{analysis}
        \subidx{programs}{tshcalc}
        \subidx{tshcalc}{program}
        Figures~\ref{\SETLABEL:HC} and~\ref{\SETLABEL:HP} represent
        Hurst coefficient and H parameter data that are derived from
        the normalized increments, shown in
        Figure~\ref{\SETLABEL:TF}. In this case, the data is
        considered a normalized derivative of the time series data
        presented in Figure~\ref{\SETLABEL:TF}, instead of a
        cumulative sum.  The program, {\it tshurst}\/, is described
        briefly in appendix~\ref{programs}, and the data for
        figures~\ref{\SETLABEL:THC} and~\ref{\SETLABEL:THP} was made
        using the -d option.

        \begin{figure}[ht]
            \begin{center}
                \begin{minipage}[t]{0.45\textwidth}
                    \epsfxsize=1.0\linewidth
                    \epsffile{\directory/data.tsfraction.tshurst-d.eps}
                    \caption[{\market}, traditional Hurst coefficient
                        data]{{\market}, traditional Hurst coefficient
                        data for the time series data shown in
                        Figure~\ref{\SETLABEL:TS}.  The slope of the
                        graph is the Hurst coefficient, and is
                        {\hurstlow} for the near term, and
                        {\hurstall} for the far term.}
                    \label{\SETLABEL:THC}
                \end{minipage}
                \hfill
                \begin{minipage}[t]{0.45\textwidth}
                    \epsfxsize=1.0\linewidth
                    \epsffile{\directory/data.tsfraction.tshcalc-d.eps}
                    \caption[{\market}, traditional H parameter
                        data]{{\market}, traditional H parameter data
                        for the time series data shown in
                        Figure~\ref{\SETLABEL:TS} The slope of the
                        graph is the H parameter, and is {\hcalclow}
                        for the near term, and {\hcalcall} for the
                        far term.}
                    \label{\SETLABEL:THP}
                \end{minipage}
            \end{center}
        \end{figure}

% Local Variables:
% TeX-parse-self: t
% TeX-auto-save: t
% TeX-master: "fractal.tex"
% End:


        %
% -----------------------------------------------------------------------------
%
% A license is hereby granted to reproduce this software source code and
% to create executable versions from this source code for personal,
% non-commercial use.  The copyright notice included with the software
% must be maintained in all copies produced.
%
% THIS PROGRAM IS PROVIDED "AS IS". THE AUTHOR PROVIDES NO WARRANTIES
% WHATSOEVER, EXPRESSED OR IMPLIED, INCLUDING WARRANTIES OF
% MERCHANTABILITY, TITLE, OR FITNESS FOR ANY PARTICULAR PURPOSE.  THE
% AUTHOR DOES NOT WARRANT THAT USE OF THIS PROGRAM DOES NOT INFRINGE THE
% INTELLECTUAL PROPERTY RIGHTS OF ANY THIRD PARTY IN ANY COUNTRY.
%
% Copyright (c) 1994-2006, John Conover, All Rights Reserved.
%
% Comments and/or bug reports should be addressed to:
%
%     john@email.johncon.com (John Conover)
%
% -----------------------------------------------------------------------------
%
% Revision: \RCSRevision \\
% Revision Time: \RCSTime UMT \\
% Revision Date: \RCSDate \\
% Revision Id: \RCSId \\
% Revision File: \RCSLog \\
\RCS $Revision: 0.0 $
\RCS $Date: 2006/01/20 04:38:13 $
\RCS $Id: fiscal.tex,v 0.0 2006/01/20 04:38:13 john Exp $
% $Log: fiscal.tex,v $
% Revision 0.0  2006/01/20 04:38:13  john
% Initial version
%
%
    \subsection{Fixed Increment Approximation for Fiscal Strategy}
        \label{\SETLABEL:FS}

        \subidx{\market}{fiscal strategy}
        \subidx{markets}{analysis}
        \subidx{analysis}{markets}
        \subidx{strategy}{fiscal}
        \subidx{fiscal}{strategy}
        The data in this section is presented in tabular form in
        Section~\ref{\SETLABELREF:LR}. This section derives various
        values based on the ``average'' of the normalized increments
        presented in Figure~\ref{\SETLABEL:TFA}. These values are an
        approximation to a, probably, complex process with a
        distribution shown in Figure~\ref{\SETLABEL:TF}. These values
        will be used in a fixed increment Brownian fractal analysis
        and simulation of the {\market}, and may, or may not, provide
        adequate accuracy for projections.

        For an organization operating in the {\market}, the fiscal
        strategy, commensurate with the aggregate environment, can be
        derived as follows~\cite[pp. 128, pp
        151]{Schroeder},~\cite[pp. 450]{Reza},~\cite[pp. 270]{Pierce}:
        \vspace{0.15in}

        \subsubsection{Logarithmic Returns}
            \label{\SETLABEL:LR}

            \subidx{logarithmic}{returns}
            \subidx{returns}{logarithmic}
            \subidx{\market}{logarithmic returns}
            The logarithmic returns can be calculated by various
            means. Four will be presented here, for comparison.

            \subidx{programs}{tsnormal}
            \subidx{tsnormal}{program}
            \subidx{logarithmic}{returns}
            \subidx{returns}{logarithmic}
            The logarithmic returns, in bits, $bits$, as computed from
            the mean, by the program {\it tsnormal}\/, which is
            described in Chapter~\ref{programs}, and is presented in
            Figure~\ref{\SETLABEL:TF}, and Equation~\ref{abits} from
            Section~\ref{ereturns} in Chapter~\ref{general}:

            \begin{equation}
                bits = \frac{\ln \left({\datafractionmean} + 1\right)}{\ln \left(2\right)} = \datafractionmeanbits
            \end{equation}

            \subidx{programs}{tslsq}
            \subidx{tslsq}{program}
            \subidx{logarithmic}{returns}
            \subidx{returns}{logarithmic}
            \noindent By comparison, the logarithmic returns, in bits,
            $bits$, as computed from the constant in the least squares
            approximation, using the program {\it tslsq}\/, which is briefly
            described in Chapter~\ref{programs}, as presented in
            Figure~\ref{\SETLABEL:TF}, and Equation~\ref{abits} from
            Section~\ref{ereturns} in Chapter~\ref{general}:

            \begin{equation}
                bits = \frac{\ln \left({\datafractionconstant} + 1\right)}{\ln \left(2\right)} = \datafractionconstantbits
            \end{equation}

            Note that if the mean is not constant in
            Figure~\ref{\SETLABEL:TF}, this method will not provide
            accurate results.

            \subidx{programs}{tslsq}
            \subidx{tslsq}{program}
            \subidx{logarithmic}{returns}
            \subidx{returns}{logarithmic}
            \noindent And by yet another comparison, using the program
            {\it tslsq}\/, which is briefly described in
            Chapter~\ref{programs}, with the -e -p options, to provide
            a formula for the least squares exponential fit to the
            time series data set presented in
            Figure~\ref{\SETLABEL:TS}:

            \begin{equation}
                bits = {\datatslsqepbits}
            \end{equation}

            \subidx{programs}{tslogreturns}
            \subidx{tslogreturns}{program}
            \subidx{logarithmic}{returns}
            \subidx{returns}{logarithmic}
            \noindent And finally, by comparison, from the
            {\it tslogreturns}\/ program, which is briefly described
            in Chapter~\ref{programs}, with the -p option, to provide
            a formula for the logarithmic returns of the time series
            data set presented in Figure~\ref{\SETLABEL:TS}:

            \begin{equation}
                bits = {\logreturns}
            \end{equation}

        \subsubsection{Calculation of Shannon Probability}
            \label{\SETLABEL:SP}

            \subidx{\market}{Shannon probability}
            Ideally, all of the values presented in
            Section~\ref{\SETLABEL:LR} would be equal. Using the
            logarithmic returns provided by the {\it tslogreturns}\/
            program, to be consistent
            with~\cite[pp. 81]{Peters:CAOITCM}

            \subidx{programs}{tslogreturns}
            \subidx{tslogreturns}{program}
            \begin{equation}
                2^{{\logreturns}t}
            \end{equation}

            \noindent therefore:
            \begin{equation}
                C\left(p\right) = {\logreturns}
            \end{equation}
            \subidx{programs}{tsshannon}
            \subidx{tsshannon}{program}
            \subidx{Shannon}{probability}
            \subidx{probability}{Shannon}
            \noindent and, {\it tsshannon}\/ {\logreturns} gives:
            \begin{equation}
                \label{\SETLABEL:F0}
                C\left({\shannonlogreturns}\right) = {\logreturns}
            \end{equation}
            \noindent therefore:
            \begin{eqnarray}
                2^{C\left({\shannonlogreturns}\right)} & = & 2^{\logreturns}\\
                                                       & = & {\twologreturns}\\
                                                       & = & {\twologreturnshundred}\%
            \end{eqnarray}
            \noindent and:
            \begin{eqnarray}
                2p - 1 & = & \left(2 \cdot {\shannonlogreturns}\right) - 1\\
                       & = & {\twopone}\\
                       \label{\SETLABEL:F1}
                       & = & {\twoponehundred}\%
            \end{eqnarray}

            \subidx{\market}{fiscal strategy}
            \subidx{markets}{analysis}
            \subidx{analysis}{markets}
            \subidx{strategy}{fiscal}
            \subidx{fiscal}{strategy}
            \subidx{\market}{fiscal strategy}
            \subidx{\market}{growth rate}
            Presuming the simplified assumptions outlined in
            Section~\ref{assumptions}, the ``typical'' organization
            operating in the {\market} executes a long term fiscal
            strategy, commensurate with the aggregate environment,
            that is to invest, every {\timescale}, in sufficient
            additional resources and infrastructure, to increase the
            manufacturing of goods and services by {\twoponehundred}\%
            of its rate of revenue returns, (per {\timescale}.) As a
            conceptual model, the remaining {\hundredtwoponehundred}\%
            will be held in ``reserve'' with a
            {\shannonlogreturnshundred}\% chance of making twice the
            {\twoponehundred}\% back, (and a
            {\hundredshannonlogreturnshundred}\% chance of making
            0.0,) in one {\timescale}, on the average, for an average
            growth in its rate of revenue returns, (per {\timescale},)
            of {\twologreturnshundred}\%, or a doubling of its rate of
            revenue returns, (per {\timescale},) in
            {\oneoverlogreturns} {\timescale}s.

        \subsubsection{Example Fixed Increment Approximation Fiscal Strategies}

            \subidx{\market}{fiscal strategy}
            \subidx{markets}{analysis}
            \subidx{analysis}{markets}
            \subidx{strategy}{fiscal}
            \subidx{fiscal}{strategy}
            \subidx{\market}{fiscal strategy}
            \subidx{\market}{growth rate}
            \subidx{\market}{management metric}
            \idx{management metric}
            A possible metric on the effectiveness of long term fiscal
            management could possibly be that if an investment of
            {\twoponehundred}\% per {\timescale} of the rate of
            revenue returns, (per {\timescale},) is made in resources
            and infrastructure, then the rate of revenue returns would
            be expected to increase by {\twologreturnshundred}\%, per
            {\timescale}, on average.

            Note that the metrics presented in this section are
            representative of the {\market} as an aggregate whole, and
            may or may not be accurate representations for any
            particular participant in the environment. Of interest to
            the participants in the environment would be a similar
            analysis of each product or service rendered in the
            marketplace.

            \subidx{\market}{fiscal strategy}
            \subidx{markets}{analysis}
            \subidx{analysis}{markets}
            \subidx{strategy}{fiscal}
            \subidx{fiscal}{strategy}
            \subidx{\market}{fiscal strategy}
            As a simple illustrative example, a company operating in
            this environment might obtain a credit line from a bank
            that is equal to {\twoponehundred}\% of its rate of
            revenue returns, (per {\timescale},) to finance additional
            operations. In this simple scenario, the company would use
            its revenue base as collateral for the loan. Some
            {\timescale}s, depending on the {\market}'s environment,
            the company's rate of revenue returns exceeds what was
            borrowed from the bank, and the loan is repaid in
            full. Other {\timescale}s, the company must default, and
            the bank seizes a portion of the company's revenue base to
            pay the delinquent loan. However, on the average, the
            company will expand its rate of revenue returns at
            {\twologreturnshundred}\% per {\timescale}.

            \subidx{\market}{fiscal strategy}
            \subidx{markets}{analysis}
            \subidx{analysis}{markets}
            \subidx{strategy}{fiscal}
            \subidx{fiscal}{strategy}
            \subidx{\market}{fiscal strategy}
            As another simple example, a company re-invests
            {\twoponehundred}\% of its rate of revenue returns, (per
            {\timescale},) in development, marketing, sales, and
            distribution of new products.  Although some products will
            be successful and the return on the investment will exceed
            the {\twoponehundred}\% per {\timescale} investment,
            others will not. However, on the average, the company will
            expand it gross rate of revenue returns at
            {\twologreturnshundred}\% per {\timescale}.

            \subidx{\market}{fiscal strategy}
            \subidx{markets}{analysis}
            \subidx{analysis}{markets}
            \subidx{strategy}{fiscal}
            \subidx{fiscal}{strategy}
            \subidx{\market}{fiscal strategy}
            \subidx{\market}{product portfolio}
            \subidx{\market}{product diversity}
            \subidx{\market}{product mix}
            \subidx{\market}{optimum number of products}
            \idx{product portfolio}
            \idx{product diversity}
            \idx{optimum number of products}
            \idx{product mix}

            As an example of ``product portfolio'' management, suppose
            a company re-invests {\twoponehundred}\% of its rate of
            revenue returns, (per {\timescale},) in development,
            marketing, sales, and distribution of new products.
            Further suppose that the company has two products, and a
            fractal analysis of the individual product rate of revenue
            return time series indicates that one product has a
            Shannon probability of 0.65, and the other has a Shannon
            probability of 0.55. Then the percentage of re-investment
            in the first product would be $(2 \cdot 0.65 - 1) \cdot
            {\twoponehundred}$, percent of the rate of revenue
            returns, and $(2 \cdot 0.55 - 1) \cdot {\twoponehundred}$
            percent for the second product, implying that the company
            should diversify its product line\footnote{The astute
            reader would note that the linear addition was used to add
            the contribution to development of each product. This is a
            ``near term'' interpretation. Actually, in general, the
            method used should be a root mean square process,
            dependent on the Hurst Coefficient, $H$, where
            $P_{total}^H = P_1^H + P_2^H + \cdots$, where $P_n$ is the
            contribution to each individual product. For a Brownian
            motion, or random walk type of fractal the Hurst
            Coefficient is a function of time into the future. For the
            ``near term,'' the Hurst coefficient is very near unity,
            meaning the summation process is linear. For the ``long
            term,'' $H \approx 0.5$, or a standard root mean square
            summation process should be used. If $H$ is $0.5$ then the
            market is termed a Brownian motion, or random walk
            process. If it is larger than 0.5, it is termed fractional
            Brownian motion process. For a random walk process, ``near
            term'' and ``far term'' are quantitatively differentiated
            on the Hurst Coefficient graph where $1 - \ln (t) = 0.5
            \cdot \ln (t)$, or when $\ln (t) = 2$, or $t =
            7.389\ldots$ See~\cite[pp. 67, 83-84]{Peters:CAOITCM}
            and~\cite[pp. 129, 159]{Schroeder} for particulars on the
            implications of the Hurst Coefficient and root mean square
            summation issues.}.  Note that this is a ``bet hedging''
            metric methodology, and assumes that the products have
            uncorrelated revenue return rates. If this re-investment
            methodology is not feasible, perhaps for strategic
            financial reasons, then the re-investment in both products
            should total the ${\twoponehundred}$\%, and the investment
            in each product should be made at a ratio of $\frac{(2
            \cdot 0.65 - 1)}{(2 \cdot 0.55 - 1)} = 3 : 1$,
            respectively. Note that this ``bet hedging'' can be used
            to define the optimal number of products that can be
            supported on the rate of revenue returns. If it assumed
            that all products are ``typical'' for the {\market}, as a
            standard bench mark, then the optimal number will be
            $\frac{1}{{\twopone}}$. Note that this is a
            ``theoretical'' value, since not all products are
            ``typical,'' and there may be strategic reasons, for
            example product leveraging, that may increase the number
            of products above the optimum. However, most of the
            revenue should come from the optimal number of products,
            since having more products will decrease the amount of the
            potential investment in each product, and having less than
            the optimum number of products will increase the risk that
            many of the products could suffer a ``down market''
            concurrently, impacting the rate of revenue returns.  As
            another interesting interpretation of the optimal
            ``hedging of bets,'' in product portfolio strategy, and
            considering the graph of the normalized increments
            presented in Figure~\ref{\SETLABEL:TF}, if the
            organization is running optimally, then these products
            will generate, at least in principle, one standard
            deviation, approximately $0.8413 = 84.13$\% of the future
            growth in rate of revenue returns. Naturally, these are
            approximations, and the values are an approximation to a,
            probably, complex process, and appropriate scrutiny should
            be exercised before making specific projections.  As yet
            another example of ``product portfolio'' management,
            consider the issue of product mix. In this interpretation,
            {\twoponehundred}\% of the product manufactured should be
            ``proprietary,'' while the rest is ``industry standard.''
            As yet another possibility, {\twoponehundred}\% of the
            product manufactured should be predatory into new markets,
            and the remainder in markets that are ``traditional'' for
            the company.

% Local Variables:
% TeX-parse-self: t
% TeX-auto-save: t
% TeX-master: "fractal.tex"
% End:


        %
% -----------------------------------------------------------------------------
%
% A license is hereby granted to reproduce this software source code and
% to create executable versions from this source code for personal,
% non-commercial use.  The copyright notice included with the software
% must be maintained in all copies produced.
%
% THIS PROGRAM IS PROVIDED "AS IS". THE AUTHOR PROVIDES NO WARRANTIES
% WHATSOEVER, EXPRESSED OR IMPLIED, INCLUDING WARRANTIES OF
% MERCHANTABILITY, TITLE, OR FITNESS FOR ANY PARTICULAR PURPOSE.  THE
% AUTHOR DOES NOT WARRANT THAT USE OF THIS PROGRAM DOES NOT INFRINGE THE
% INTELLECTUAL PROPERTY RIGHTS OF ANY THIRD PARTY IN ANY COUNTRY.
%
% Copyright (c) 1994-2006, John Conover, All Rights Reserved.
%
% Comments and/or bug reports should be addressed to:
%
%     john@email.johncon.com (John Conover)
%
% -----------------------------------------------------------------------------
%
% Revision: \RCSRevision \\
% Revision Time: \RCSTime UMT \\
% Revision Date: \RCSDate \\
% Revision Id: \RCSId \\
% Revision File: \RCSLog \\
\RCS $Revision: 0.0 $
\RCS $Date: 2006/01/20 04:38:13 $
\RCS $Id: companies.tex,v 0.0 2006/01/20 04:38:13 john Exp $
% $Log: companies.tex,v $
% Revision 0.0  2006/01/20 04:38:13  john
% Initial version
%
%
    \subsection{Number of Companies}
        \label{\SETLABEL:QNC}

        \subidx{\market}{number of companies}
        \subidx{number of companies}{analysis}
        \subidx{analysis}{number of companies}
        \subidx{Shannon}{probability}
        \subidx{probability}{Shannon}
        This section evaluates the approximate, or ``average,'' number
        of companies in the {\market}, and uses the method outlined in
        Chapter~\ref{general}, Section~\ref{aftsma}. Since the
        average, $avg_{ind}$, and the root mean square, $rms_{ind}$,
        of the normalized increments of the {\market} time series is
        \datafractionmean, and \datafractionrms respectively, the
        number of companies participating in the market can be
        calculated by Equation~\ref{ncompanies} to be {\ncompanies}.

        If this value seems consistent number of companies in the
        {\market}, within the assumptions outlined in
        Chapter~\ref{general}, Section~\ref{aftsma}, then it would
        seem that there is some circumstantial or indirect evidence
        that the companies participating in the {\market} are
        operating optimally, and the ``average'' Shannon probability,
        $P$ for each participating company would be, using
        Equation~\ref{pncompanies}, {\pncompanies}, which would be the
        value which should be used in Section~\ref{\SETLABEL:FS} for
        each participating company if market expansion was to be
        consistent with the rest of the industry. However, if the
        Shannon probability derived in Section~\ref{\SETLABEL:FS} is
        greater than the average Shannon probability for the companies
        participating in the {\market}, as derived in this section,
        then the market would, possibly, be exploitable with the
        fiscal strategy outlined in Section~\ref{\SETLABEL:FS}. The
        maximum exploitability for the {\market} is derived in
        Section~\ref{\SETLABEL:MAXSHANNON}, but it is probably of
        doubtful practicality.

        Note that these optimizations would maximize a company's
        market growth. Since there are probably many companies
        competing in the market place, this would not necessarily
        maximize a company's P\&L, as described in
        Chapter~\ref{general}, Section~\ref{ompl}. The Shannon
        probability that maximizes market share in the {\market} is
        \pncompanies, with several alternative solutions listed in the
        previous paragraph. However, these should be contrasted to the
        Shannon probability that maximizes a company's P\&L which is
        \avgrms~in the {\market}. In all cases, the fraction of the
        P\&L that should be ``wagered'' on the future, $f$, should be:

        \begin{equation}
            f = 2P - 1
        \end{equation}

        \noindent where $P$ is the particular Shannon probability
        chosen optimize a particular fiscal strategy. Interestingly,
        the measured Shannon probability of the {\market} would tend
        to indicate that the companies participating in the market
        have chosen a fiscal strategy that optimizes market growth, as
        opposed to capital growth.

        \subidx{\market}{increasing returns}
        \subidx{economic increasing returns}{\market}
        As interesting interpretation of these exploitive issues,
        since all three fiscal strategies will result in exponential
        market growth for every company participating in the market,
        is that they may represent, perhaps, an example of
        ``increasing returns.''

% Local Variables:
% TeX-parse-self: t
% TeX-auto-save: t
% TeX-master: "fractal.tex"
% End:


        %
% -----------------------------------------------------------------------------
%
% A license is hereby granted to reproduce this software source code and
% to create executable versions from this source code for personal,
% non-commercial use.  The copyright notice included with the software
% must be maintained in all copies produced.
%
% THIS PROGRAM IS PROVIDED "AS IS". THE AUTHOR PROVIDES NO WARRANTIES
% WHATSOEVER, EXPRESSED OR IMPLIED, INCLUDING WARRANTIES OF
% MERCHANTABILITY, TITLE, OR FITNESS FOR ANY PARTICULAR PURPOSE.  THE
% AUTHOR DOES NOT WARRANT THAT USE OF THIS PROGRAM DOES NOT INFRINGE THE
% INTELLECTUAL PROPERTY RIGHTS OF ANY THIRD PARTY IN ANY COUNTRY.
%
% Copyright (c) 1994-2006, John Conover, All Rights Reserved.
%
% Comments and/or bug reports should be addressed to:
%
%     john@email.johncon.com (John Conover)
%
% -----------------------------------------------------------------------------
%
% Revision: \RCSRevision \\
% Revision Time: \RCSTime UMT \\
% Revision Date: \RCSDate \\
% Revision Id: \RCSId \\
% Revision File: \RCSLog \\
\RCS $Revision: 0.0 $
\RCS $Date: 2006/01/20 04:38:13 $
\RCS $Id: operations.tex,v 0.0 2006/01/20 04:38:13 john Exp $
% $Log: operations.tex,v $
% Revision 0.0  2006/01/20 04:38:13  john
% Initial version
%
%
    \subsection{Fixed Increment Approximation for Operational Strategy}
        \label{\SETLABEL:OPS}.

        This section derives various values based on the ``average''
        of the normalized increments presented in
        Figure~\ref{\SETLABEL:TFA}. These values are an approximation
        to a, probably, complex process with a distribution shown in
        Figure~\ref{\SETLABEL:TF}. These values will be used in a
        fixed increment Brownian fractal analysis and simulation of
        the {\market}, and may, or may not, provide adequate accuracy
        for projections.

        \subidx{\market}{fiscal strategy}
        \subidx{\market}{Shannon probability}
        \subidx{strategy}{fiscal}
        \subidx{fiscal}{strategy}
        \subidx{Shannon}{probability}
        \subidx{probability}{Shannon}
        It should be noted that the analysis of fiscal strategy,
        presented in Section~\ref{\SETLABEL:FS}, is derived from the
        {\market} metrics and may, or may not, be maximally
        optimal. For the optimal fiscal strategy, which may be
        exploitable, see Section~\ref{\SETLABEL:MAXSHANNON}.

        \subidx{strategy}{exploitable}
        \subidx{exploitable}{strategy}
        \subidx{\market}{windows of opportunity}
        \idx{windows of opportunity}
        \subidx{decision}{obsolete}
        \subidx{obsolete}{decision}
        \subidx{decision}{timeliness}
        \subidx{timeliness}{decision}
        \subidx{rate of revenue returns}{forecast}
        \subidx{forecast}{rate of revenue returns}
        An additional exploitable strategy may be time itself.
        Equations~\ref{\SETLABEL:V},~\ref{\SETLABEL:R},
        and,~\ref{\SETLABEL:MA}, are, essentially, metrics on how fast
        a decision, which is based on information concerning the
        current status of the {\market}, becomes obsolete. Obviously,
        how long a decision is expected to remain relevant should be
        addressed as an operational necessity in strategic planning
        and project management. Figures~\ref{\SETLABEL:FN},
        and,~\ref{\SETLABEL:FF} compare methods of approximation of
        the ``forecastability'' of rate of revenue returns in the
        {\market} for the near term and far
        term~\cite[pp. 83-84]{Peters:CAOITCM}, respectively. As a
        general rule, caution must be exercised when making decisions
        that will span a time interval larger than the time interval
        where the ``forecastability'' of rate of revenue returns drops
        below 50\%. Beyond this time interval, the chances increase
        that the competitive and market forces will alter the market
        environment in a possibly detrimental unanticipated
        fashion. Obviously, there is significant advantage in
        ``timeliness'' of development, manufacturing, and distribution
        of products and services that are consistent with this
        temporal agenda. Automation of these processes, if executed
        consistently with this agenda, should be considered a
        competitive advantage.

        \subidx{strategy}{exploitable}
        \subidx{exploitable}{strategy}
        \subidx{rate of revenue returns}{forecast}
        \subidx{forecast}{rate of revenue returns}
        \idx{product life cycle}
        \idx{life cycle, product}
        In some sense, this temporal agenda defines the ``average''
        product or service life cycle in the {\market}. When the
        ``forecastability'' of rate of revenue returns drops below
        50\%, there is an even chance that the rate of revenue returns
        for the product or service will change in a detrimental
        fashion. If it is assumed that a product or service life cycle
        consists of a ramp up, a maintenence interval, and a ramp
        down, then, if all three life cycle intervals are equal, the
        product life cycle will be, approximately, three times the
        time interval where the ``forecastability'' of rate of revenue
        returns drops below 50\%. Although probably not an accurate
        prediction of product or service life cycle, the technique may
        be used as a conceptual approximation to the dynamics of
        ``market windows.\footnote{For example, consider the market
        for table salt. Since it has inelastic supply and demand
        curves, and is a necessary requirement for life, it would be
        expected that the Hurst coefficient would be very near
        unity---ignoring competitive pressures in the market. The
        predictability of the table salt market would, therefore, be
        expected to be relatively good, over time.}''  The conceptual
        approximation will probably predict a ``conservative'' or
        ``pessimistic'' value in relation to actual markets.

        \begin{figure}[ht]
            \begin{center}
                \begin{minipage}[t]{0.45\textwidth}
                    \epsfxsize=1.0\linewidth
                    \epsffile{\directory/datahurstlownear.eps}
                    \caption[{\market}, ``forecastability'' of near
                        term rate of revenue returns]{{\market},
                        ``forecastability'' of near term rate of
                        revenue returns. Although the error function
                        is the most accurate, for the near term,
                        $H^{t} = \thurstlow^{t}$ may be used as a
                        reliable metric of ``forecastability'' of the
                        rate of revenue returns.}
                    \label{\SETLABEL:FN}
                \end{minipage}
                \hfill
                \begin{minipage}[t]{0.45\textwidth}
                    \epsfxsize=1.0\linewidth
                    \epsffile{\directory/datahurstlowfar.eps}
                    \caption[{\market}, ``forecastability'' of far
                        term rate of revenue returns]{{\market},
                        ``forecastability'' of far term rate of
                        revenue returns. Although the error function
                        is the most accurate, for the far term,
                        $\frac{1}{\sqrt{t}}$ may be used as a reliable
                        metric of ``forecastability'' of the rate of
                        revenue returns.}
                    \label{\SETLABEL:FF}
                \end{minipage}
            \end{center}
        \end{figure}

        \idx{operations research}
        As an interesting interpretation of the data presented in
        Figure~\ref{\SETLABEL:FN}, there may be, perhaps, some
        applicability to such operational agendas as inventory
        control. Maintaining too little inventory, obviously, will
        create a situation where the organization can not exploit
        market expansion, and maintaining too much inventory,
        likewise, would over extend the company, creating unnecessary
        losses when the market contracts. The company should maintain
        inventory levels that do not exceed, from
        Equation~\ref{\SETLABEL:MA}, ${\thurstlow}^{n} = 0.5$
        {\timescale}s of operations. Since the optimal amount of
        inventory and, from Equation~\ref{\SETLABEL:V}, the variance
        of change in the rate of revenue returns in the future can be
        calculated, there may, perhaps, be some applicability to a
        forecasting methodology that can be incorporated into other
        areas of operations research, for example the linear algebras
        using simplex methodologies for optimization of manufacturing
        processes. Traditionally, these forecasts are made by the
        sales department, and are subject to various subjective
        biases.

% Local Variables:
% TeX-parse-self: t
% TeX-auto-save: t
% TeX-master: "fractal.tex"
% End:


        %
% -----------------------------------------------------------------------------
%
% A license is hereby granted to reproduce this software source code and
% to create executable versions from this source code for personal,
% non-commercial use.  The copyright notice included with the software
% must be maintained in all copies produced.
%
% THIS PROGRAM IS PROVIDED "AS IS". THE AUTHOR PROVIDES NO WARRANTIES
% WHATSOEVER, EXPRESSED OR IMPLIED, INCLUDING WARRANTIES OF
% MERCHANTABILITY, TITLE, OR FITNESS FOR ANY PARTICULAR PURPOSE.  THE
% AUTHOR DOES NOT WARRANT THAT USE OF THIS PROGRAM DOES NOT INFRINGE THE
% INTELLECTUAL PROPERTY RIGHTS OF ANY THIRD PARTY IN ANY COUNTRY.
%
% Copyright (c) 1994-2006, John Conover, All Rights Reserved.
%
% Comments and/or bug reports should be addressed to:
%
%     john@email.johncon.com (John Conover)
%
% -----------------------------------------------------------------------------
%
% Revision: \RCSRevision \\
% Revision Time: \RCSTime UMT \\
% Revision Date: \RCSDate \\
% Revision Id: \RCSId \\
% Revision File: \RCSLog \\
\RCS $Revision: 0.0 $
\RCS $Date: 2006/01/20 04:38:13 $
\RCS $Id: simulation.tex,v 0.0 2006/01/20 04:38:13 john Exp $
% $Log: simulation.tex,v $
% Revision 0.0  2006/01/20 04:38:13  john
% Initial version
%
%
    \subsection{Simulation of Fixed Increment Approximation for Fiscal Strategy}
        \label{\SETLABEL:TSUNFAIRBROWNIAN}

        \subidx{\market}{market simulation}
        The data in this section is presented in tabular form in
        Section~\ref{\SETLABELREF:SIM}.
        Figure~\ref{\SETLABEL:TSUNFAIRBROWNIAN0} represents a
        constructional simulation of the time series data presented in
        Figure~\ref{\SETLABEL:TS}. The program {\it
        tsunfairbrownian}\/, which is briefly described in
        appendix~\ref{programs}, was used in the reconstruction. The
        reconstructed data is superimposed on the original time series
        data.  The program, {\it tsunfairbrownian}\/, essentially,
        constructs the new time series as a Brownian fractal with
        fixed increments---the value of the fixed increment is derived
        from the root mean square average of the normalized increments
        presented in Figure~\ref{\SETLABEL:TF}. The ``quality'' of
        such a reconstruction should be subject to adequate scepticism
        and scrutiny since, in all probability, the normalized
        increments presented in Figure~\ref{\SETLABEL:TF} represent a
        relatively complex process, that may not be ``modeled'' with
        such a simple methodology.

        As a further comparison of the the constructional simulation
        with the original time series data,
        Figure~\ref{\SETLABEL:TSUNFAIRBROWNIAN1} presents a normalized
        histogram of the normalized increments of the reconstructed
        time series, superimposed on the normalized histogram
        presented in Figure~\ref{\SETLABEL:NH}.

        \subidx{\market}{fiscal strategy, simulation}
        \subidx{markets}{simulation}
        \subidx{simulation}{markets}
        \subidx{strategy}{fiscal, simulation}
        \subidx{fiscal}{strategy, simulation}
        \subidx{programs}{tsunfairbrownian}
        \subidx{tsunfairbrownian}{program}
        \begin{figure}[ht]
            \begin{center}
                \begin{minipage}[t]{0.45\textwidth}
                    \epsfxsize=1.0\linewidth
                    \epsffile{\directory/tsunfairbrownian-f.eps}
                    \caption[{\market}, Time series data, empirical and
                        simulated]{{\market}, Time series data, empirical
                        and simulated, using the program {\it tsunfairbrownian}\/
                        with f = {\datafractionrms}. This data is
                        superimposed on the data presented in
                        Figure~\ref{\SETLABEL:TS}.}
                    \label{\SETLABEL:TSUNFAIRBROWNIAN0}
                \end{minipage}
                \hfill
                \begin{minipage}[t]{0.45\textwidth}
                    \epsfxsize=1.0\linewidth
                    \epsffile{\directory/tsunfairbrownian-f.tsfraction.tsnormal-s30.eps}
                    \caption[{\market}, normalized histogram,
                        empirical and simulated]{{\market}, normalized
                        histogram of the normalized increments of the
                        time series data shown in
                        Figure~\ref{\SETLABEL:TSUNFAIRBROWNIAN0},
                        empirical and simulated.  The empirical data
                        has a mean of {\datafractionmean}, with a
                        standard deviation of {\datafractionstddev}.
                        By comparison, the simulated data has a mean
                        of {\tsunfairbrownianfractionmean} with a
                        standard deviation of
                        {\tsunfairbrownianfractionstddev}. This data
                        is superimposed on the data presented in
                        Figure~\ref{\SETLABEL:NH}. The area under the
                        four curves is identical.}
                    \label{\SETLABEL:TSUNFAIRBROWNIAN1}
                \end{minipage}
            \end{center}
        \end{figure}

% Local Variables:
% TeX-parse-self: t
% TeX-auto-save: t
% TeX-master: "fractal.tex"
% End:


        %
% -----------------------------------------------------------------------------
%
% A license is hereby granted to reproduce this software source code and
% to create executable versions from this source code for personal,
% non-commercial use.  The copyright notice included with the software
% must be maintained in all copies produced.
%
% THIS PROGRAM IS PROVIDED "AS IS". THE AUTHOR PROVIDES NO WARRANTIES
% WHATSOEVER, EXPRESSED OR IMPLIED, INCLUDING WARRANTIES OF
% MERCHANTABILITY, TITLE, OR FITNESS FOR ANY PARTICULAR PURPOSE.  THE
% AUTHOR DOES NOT WARRANT THAT USE OF THIS PROGRAM DOES NOT INFRINGE THE
% INTELLECTUAL PROPERTY RIGHTS OF ANY THIRD PARTY IN ANY COUNTRY.
%
% Copyright (c) 1994-2006, John Conover, All Rights Reserved.
%
% Comments and/or bug reports should be addressed to:
%
%     john@email.johncon.com (John Conover)
%
% -----------------------------------------------------------------------------
%
% Revision: \RCSRevision \\
% Revision Time: \RCSTime UMT \\
% Revision Date: \RCSDate \\
% Revision Id: \RCSId \\
% Revision File: \RCSLog \\
\RCS $Revision: 0.0 $
\RCS $Date: 2006/01/20 04:38:13 $
\RCS $Id: maximum.tex,v 0.0 2006/01/20 04:38:13 john Exp $
% $Log: maximum.tex,v $
% Revision 0.0  2006/01/20 04:38:13  john
% Initial version
%
%
    \subsection{Simulation of Fixed Increment Approximation for Optimally Maximal Fiscal Strategy}
        \label{\SETLABEL:MAXSHANNON}
        \subidx{\market}{fiscal strategy, simulation}
        \subidx{\market}{maximum Shannon probability}
        \subidx{markets}{simulation}
        \subidx{simulation}{markets}
        \subidx{strategy}{optimum fiscal, simulation}
        \subidx{fiscal}{optimum strategy, simulation}
        \subidx{programs}{tsunfairbrownian}
        \subidx{tsunfairbrownian}{program}
        \subidx{Shannon}{probability}
        \subidx{probability}{Shannon}

        \subidx{strategy}{exploitable}
        \subidx{exploitable}{strategy}
        \subidx{programs}{tsshannonmax}
        \subidx{tsshannonmax}{program}
        \subidx{programs}{tsunfairbrownian}
        \subidx{tsunfairbrownian}{program}
        \subidx{strategy}{fiscal}
        \subidx{fiscal}{strategy}
        The data in this section is presented in tabular form in
        Section~\ref{\SETLABELREF:MAXSHANNON}. One of the issues of
        analysis, as mentioned in Section~\ref{\SETLABEL:OPS}, is to
        determine the maximum Shannon probability for the time series
        presented in Figure~\ref{\SETLABEL:TS}. Potentially, this
        could be exploited with an aggressive fiscal
        strategy. Figure~\ref{\SETLABEL:SHANNONMAX0} is a graph of the
        output of the {\it tsshannonmax}\/ program, which is described
        briefly in appendix~\ref{programs}. The maximum of this
        function is the maximum Shannon probability for the time
        series data presented in Figure~\ref{\SETLABEL:TS}.
        Figure~\ref{\SETLABEL:SHANNONMAX1} was constructed using {\it
        tsunfairbrownian}\/ program, which is also described in
        appendix~\ref{programs}, with the maximum Shannon probability,
        and the time series data presented in
        Figure~\ref{\SETLABEL:TS}. This represents a ``what if'' the
        investment strategy was changed from a Shannon probability of
        {\shannonlogreturns}, as derived in Section~\ref{\SETLABEL:SP}
        to {\shannonmax}. This process, essentially, extracts the
        random statistical data from the time series presented in
        Figure~\ref{\SETLABEL:TS}, and constructs a new time series,
        using the random statistical data, with a different investment
        strategy.  The program, {\it tsunfairbrownian}\/, essentially,
        constructs the new time series as a Brownian fractal with
        fixed increments.  The ``quality'' of such a reconstruction
        should be subject to adequate scepticism and scrutiny since,
        in all probability, the increments in the original data
        represent a relatively complex process, that may not be
        ``modeled'' with such a simple methodology.

        \begin{figure}[ht]
            \begin{center}
                \begin{minipage}[t]{0.45\textwidth}
                    \epsfxsize=1.0\linewidth
                    \epsffile{\directory/data.tsshannonmax.eps}
                    \caption[{\market}, maximum rate of revenue
                        returns] {{\market}, maximum rate of revenue
                        returns, per {\timescale}, vs. Shannon
                        probability. The maximum rate of revenue
                        returns, per {\timescale}, occurs at a Shannon
                        probability of {\shannonmax}.}
                    \label{\SETLABEL:SHANNONMAX0}
                \end{minipage}
                \hfill
                \begin{minipage}[t]{0.45\textwidth}
                    \epsfxsize=1.0\linewidth
                    \epsffile{\directory/data.tsshannonmax-p.tsunfairbrownian-p.eps}
                    \caption[{\market}, maximum rate of revenue
                        returns] {{\market}, maximum rate of revenue
                        returns, per {\timescale}, at a Shannon
                        probability, of {\shannonmax}, corresponding
                        to a ``wager'' fraction of {\twoponemax}.}
                    \label{\SETLABEL:SHANNONMAX1}
                \end{minipage}
            \end{center}
        \end{figure}

        \subidx{fractional}{Brownian motion}
        \subidx{Brownian motion}{fractional}
        \subidx{Shannon}{probability}
        \subidx{probability}{Shannon}
        \subidx{programs}{tsshannonmax}
        \subidx{tsshannonmax}{program}
        If it is assumed that the time series data set, presented in
        Figure~\ref{\SETLABEL:TS}, constitutes classical Brownian
        motion, then the Shannon probability can be calculated by
        counting the total number of {\timescale}s that the {\market}
        movement was positive, and dividing by the total number of
        {timescale}s represented in the time series. This quotient is
        {\pmax}, as compared with the predicted value from the program
        {\it tsshannonmax}\/ of {\shannonmax}.

% Local Variables:
% TeX-parse-self: t
% TeX-auto-save: t
% TeX-master: "fractal.tex"
% End:


        %
% -----------------------------------------------------------------------------
%
% A license is hereby granted to reproduce this software source code and
% to create executable versions from this source code for personal,
% non-commercial use.  The copyright notice included with the software
% must be maintained in all copies produced.
%
% THIS PROGRAM IS PROVIDED "AS IS". THE AUTHOR PROVIDES NO WARRANTIES
% WHATSOEVER, EXPRESSED OR IMPLIED, INCLUDING WARRANTIES OF
% MERCHANTABILITY, TITLE, OR FITNESS FOR ANY PARTICULAR PURPOSE.  THE
% AUTHOR DOES NOT WARRANT THAT USE OF THIS PROGRAM DOES NOT INFRINGE THE
% INTELLECTUAL PROPERTY RIGHTS OF ANY THIRD PARTY IN ANY COUNTRY.
%
% Copyright (c) 1994-2006, John Conover, All Rights Reserved.
%
% Comments and/or bug reports should be addressed to:
%
%     john@email.johncon.com (John Conover)
%
% -----------------------------------------------------------------------------
%
% Revision: \RCSRevision \\
% Revision Time: \RCSTime UMT \\
% Revision Date: \RCSDate \\
% Revision Id: \RCSId \\
% Revision File: \RCSLog \\
\RCS $Revision: 0.0 $
\RCS $Date: 2006/01/20 04:38:13 $
\RCS $Id: verification.tex,v 0.0 2006/01/20 04:38:13 john Exp $
% $Log: verification.tex,v $
% Revision 0.0  2006/01/20 04:38:13  john
% Initial version
%
%
    \subsection{Qualitative Verification of Fixed Increment Approximation Analysis}
        \label{\SETLABEL:QVA}

        \subidx{\market}{verification of analysis}
        \subidx{verification}{analysis}
        \subidx{analysis}{verification}
        \subidx{quality}{of analysis}
        \subidx{verification}{of methodology}
        \subidx{methodology}{verification of}
        \subidx{Shannon}{probability}
        \subidx{probability}{Shannon}

        This section evaluates various values based on the ``average''
        of the normalized increments presented in
        Figure~\ref{\SETLABEL:TFA}. These values are an approximation
        to a, probably, complex process with a distribution shown in
        Figure~\ref{\SETLABEL:TF}. These values will be used in a
        fixed increment Brownian fractal analysis of the {\market},
        and may, or may not, provide adequate accuracy for
        projections.

        The data in this section is presented in tabular form in
        sections~\ref{\SETLABELREF:VI1} and~\ref{\SETLABELREF:VI2}.
        As a subjective evaluation of the ``quality'' of the analysis
        of the {\market}, from Chapter~\ref{methodology},
        Equation~\ref{metricvalues1}, and using the mean and root mean
        square values of the normalized increments of the time series
        data presented in Figure~\ref{\SETLABEL:TS} from
        Figure~\ref{\SETLABEL:TF}, and the Shannon probability as
        calculated by counting the total number of {\timescale}s that
        the {\market} movement was positive, as presented in
        Section~\ref{\SETLABEL:MAXSHANNON}:

        \begin{eqnarray}
                  P & \approx & \frac{\frac{avg}{rms} + 1}{2}\\
            {\pmax} & \approx & \frac{\frac{\datafractionmean}{\datafractionrms} + 1}{2}\\
            {\pmax} & \approx & {\avgrms}
            \label{\SETLABEL:AVGS}
        \end{eqnarray}

        \subidx{Shannon}{probability}
        \subidx{probability}{Shannon}
        \noindent and comparing these values to the Shannon
        probability, as found by the {\it tsshannonmax}\/ program, which
        iterates for a maximum:

        \begin{eqnarray}
            {\pmax} \approx {\avgrms} \approx {\shannonmax}
        \end{eqnarray}

        \subidx{logarithmic}{returns}
        \subidx{returns}{logarithmic}
        In addition, the different methods of calculating the
        logarithmic returns, presented in Section~\ref{\SETLABEL:FS},
        should be compared. The four methods used were the mean of
        Figure~\ref{\SETLABEL:TF}, the constant in the least squares
        approximation to Figure~\ref{\SETLABEL:TF}, the least squares
        exponential approximation to Figure~\ref{\SETLABEL:TS}, and
        the logarithmic returns of Figure~\ref{\SETLABEL:TS}, derived
        as the mean of the logarithms of the quotients of the
        increments. The values for each of the methods are,
        respectively:

        \begin{equation}
            \datafractionmeanbits \approx \datafractionconstantbits \approx \datatslsqepbits \approx \logreturns
        \end{equation}

        It is implied in Section~\ref{\SETLABEL:FS},
        Subsection~\ref{\SETLABEL:SP} and in
        Section~\ref{\SETLABEL:TSUNFAIRBROWNIAN} that, a Brownian
        motion with fixed increments fractal may ``model'' the
        {\market}. Using Equation~\ref{stddev9} from
        Chapter~\ref{general}, Section~\ref{abmfi}:

        \begin{eqnarray}
                                    rms \left(2P - 1\right) & \approx & \frac{\sigma \left(2P - 1\right)}{2 \sqrt{P\left(1 - P\right)}}\\
            \datafractionrms \left(2 \cdot \pmax - 1\right) & \approx & \frac{\datafractionstddev \left(2 \cdot \pmax - 1\right)}{2\sqrt{\pmax \left(1 - \pmax\right)}}\\
                       \datafractionrms \cdot \twopminusone & \approx & \datafractionstddev \cdot \twopx\\
                                                      \rmsp & \approx & \sigmap
        \end{eqnarray}

        \noindent and, equating to the mean:

        \begin{equation}
            \datafractionmean \approx \rmsp \approx \sigmap
        \end{equation}

        \subidx{Shannon}{probability}
        \subidx{probability}{Shannon}
        \noindent where, as in Equation~\ref{\SETLABEL:AVGS} using the
        mean, root mean square, and standard deviation values of the
        normalized increments of the time series data presented in
        Figure~\ref{\SETLABEL:TS} from Figure~\ref{\SETLABEL:TF}, and
        the Shannon probability as calculated by counting the total
        number of {\timescale}s that the {\market} movement was
        positive, as presented in Section~\ref{\SETLABEL:MAXSHANNON}.

        As a final qualitative comparison, the absolute value of the
        normalized increments should be the same as the root mean
        square value\footnote{The absolute value of the normalized
        increments, when averaged, is related to the root mean square
        of the increments by a constant. If the normalized increments
        are a fixed increment, the constant is unity. If the
        normalized increments have a Gaussian distribution, the
        constant is $\approx 0.8$ depending on the accuracy of of
        ``fit'' to a Gaussian distribution.}, where the absolute value
        is presented in Figure~\ref{\SETLABEL:TFA}, and the root mean
        square value is presented in Figure~\ref{\SETLABEL:TF}:

        \begin{equation}
            \datafractionabsmean \approx \datafractionrms
        \end{equation}

        Note, that if the {\market} could be ``modeled'' as a Brownian
        motion with fixed increments fractal, then the standard
        deviation of the absolute value of the normalized increments
        of the time series data presented in Figure~\ref{\SETLABEL:TS}
        from Figure~\ref{\SETLABEL:TF} should be zero. It is
        $\datafractionabsstddev$.

% Local Variables:
% TeX-parse-self: t
% TeX-auto-save: t
% TeX-master: "fractal.tex"
% End:


    \renewcommand{\market}{Non-optimal Coins Tossing Game}
    \renewcommand{\directory}{../markets/tscoins-f}
    \renewcommand{\datafractionmean}{0.008052}
\renewcommand{\datafractionmeanbits}{0.011570}
\renewcommand{\datafractionmeanq}{0.002684}
\renewcommand{\datafractionmeanbitsq}{0.003867}
\renewcommand{\datafractionstddev}{0.038579}
\renewcommand{\datafractionrms}{0.039311}
\renewcommand{\avgrms}{0.602414}
\renewcommand{\ncompanies}{5.210454}
\renewcommand{\pncompanies}{0.544866}
\renewcommand{\datafractionabsmean}{0.029745}
\renewcommand{\datafractionabsstddev}{0.025769}
\renewcommand{\datafractionconstant}{0.010041}
\renewcommand{\datafractionconstantbits}{0.014414}
\renewcommand{\datafractionconstantq}{0.003347}
\renewcommand{\datafractionconstantbitsq}{0.004821}
\renewcommand{\datafractionslope}{-0.000021}
\renewcommand{\datafractionabsconstant}{0.035145}
\renewcommand{\datafractionabsslope}{-0.000057}
\renewcommand{\hurstall}{0.659558}
\renewcommand{\hurstlow}{0.707509}
\renewcommand{\hurstlowtwo}{1.415018}
\renewcommand{\hurstlowhundred}{70.750900}
\renewcommand{\hcalcall}{0.184942}
\renewcommand{\hcalclow}{0.102042}
\renewcommand{\shannonmax}{0.604167}
\renewcommand{\twoponemax}{0.208334}
\renewcommand{\logreturns}{0.010456}
\renewcommand{\twologreturns}{1.007274}
\renewcommand{\twologreturnshundred}{0.727387}
\renewcommand{\oneoverlogreturns}{95.638868}
\renewcommand{\pmax}{0.602094}
\renewcommand{\twopminusone}{0.204188}
\renewcommand{\rmsp}{0.008027}
\renewcommand{\twopx}{0.208583}
\renewcommand{\sigmap}{0.008047}
\renewcommand{\tsunfairbrownianfractionmean}{0.007862}
\renewcommand{\tsunfairbrownianfractionstddev}{0.038619}
\renewcommand{\shannonlogreturns}{0.560125}
\renewcommand{\shannonlogreturnshundred}{56.012500}
\renewcommand{\twopone}{0.120250}
\renewcommand{\twoponehundred}{12.025000}
\renewcommand{\hundredtwoponehundred}{87.975000}
\renewcommand{\hundredshannonlogreturnshundred}{43.987500}
\renewcommand{\datatslsqepbits}{0.007623}
\renewcommand{\thurstall}{0.633980}
\renewcommand{\thurstlow}{0.710108}
\renewcommand{\thurstlowtwo}{1.420216}
\renewcommand{\thurstlowhundred}{71.010800}
\renewcommand{\thcalcall}{0.247886}
\renewcommand{\thcalclow}{0.171737}
\renewcommand{\chisquared}{2.862000}
\renewcommand{\critical}{42.557000}

    \renewcommand{\timescale}{tosses}
    \subidx{market}{\market}
    \idx{\market}

    \section{\market}

        \renewcommand{\SETLABEL}{\LABPRE:NOCST}
        \renewcommand{\SETLABELQ}{\LABPRE:NOCSTQ}
        \label{\SETLABEL}
        \renewcommand{\SETLABELREF}{\LABPREREF:NOCST}

        \subidx{tscoins}{program}
        \subidx{programs}{tscoins}
        For the analysis, the data was in the directory
        {\directory}\footnote{As a simulation model, the program {\it
        tscoins}\/ was run to make a time series data file, with the
        following parameters:

        \vspace{0.1in}
        {\noindent}tscoins -p 0.6 -f 0.03 300 > data
        \vspace{0.1in}

        \noindent to make a time series of 300 elements, with a
        Shannon probability of 0.6 and a known non-optimal investment
        strategy.  The data is by {\timescale}.}.

        The data in this section is presented in tabular form in
        Section~\ref{\SETLABELREF}. Note that in this analysis, the
        rate of revenue returns means the increase or decrease in the
        cumulative sum of the {\market}. This is included for
        ``theoretical'' comparative purposes.

        %
% -----------------------------------------------------------------------------
%
% A license is hereby granted to reproduce this software source code and
% to create executable versions from this source code for personal,
% non-commercial use.  The copyright notice included with the software
% must be maintained in all copies produced.
%
% THIS PROGRAM IS PROVIDED "AS IS". THE AUTHOR PROVIDES NO WARRANTIES
% WHATSOEVER, EXPRESSED OR IMPLIED, INCLUDING WARRANTIES OF
% MERCHANTABILITY, TITLE, OR FITNESS FOR ANY PARTICULAR PURPOSE.  THE
% AUTHOR DOES NOT WARRANT THAT USE OF THIS PROGRAM DOES NOT INFRINGE THE
% INTELLECTUAL PROPERTY RIGHTS OF ANY THIRD PARTY IN ANY COUNTRY.
%
% Copyright (c) 1994-2006, John Conover, All Rights Reserved.
%
% Comments and/or bug reports should be addressed to:
%
%     john@email.johncon.com (John Conover)
%
% -----------------------------------------------------------------------------
%
% Revision: \RCSRevision \\
% Revision Time: \RCSTime UMT \\
% Revision Date: \RCSDate \\
% Revision Id: \RCSId \\
% Revision File: \RCSLog \\
\RCS $Revision: 0.0 $
\RCS $Date: 2006/01/20 04:38:13 $
\RCS $Id: fraction.tex,v 0.0 2006/01/20 04:38:13 john Exp $
% $Log: fraction.tex,v $
% Revision 0.0  2006/01/20 04:38:13  john
% Initial version
%
%
    \subsection{Time Series Increments Analysis}
        \label{\SETLABEL:TSA}

        \subidx{\market}{Time series analysis}
        \subidx{time series}{increments}
        \subidx{time series}{analysis}
        \subidx{cumulative sum}{analysis}
        \subidx{analysis}{cumulative sum}
        \subidx{analysis}{random process}
        \subidx{random process}{analysis}
        \subidx{Gaussian}{increments}
        \subidx{increments}{Gaussian}
        \subidx{Brownian}{motion, fractional}
        \subidx{fractional}{Brownian motion}
        \subidx{fractal}{Brownian motion}
        The data in this section is presented in tabular form in
        Section~\ref{\SETLABELREF:TSA}.  Figure~\ref{\SETLABEL:TS} is
        a graph of the time series data for the {\market}.

        \subidx{increments}{normalized}
        \subidx{normalized}{increments}
        \subidx{programs}{tsfraction}
        \subidx{tsfraction}{program}
        Figure~\ref{\SETLABEL:TF} is a graph of the normalized
        increments of the time series data presented in
        Figure~\ref{\SETLABEL:TS}. The data presented was made by
        running the program {\it tsfraction}\/ on the time series
        data. The program {\it tsfraction}\/ is described briefly in
        Appendix~\ref{programs}, and subtracts the previous value from
        the next value, dividing this difference by the previous
        value, for each element in the time series data. The new time
        series contains the instantaneous change in the rate of
        revenue returns, divided by the magnitude of the instantaneous
        rate of revenue returns.

        \subidx{mean}{standard deviation}
        \subidx{standard deviation}{mean}
        \idx{root mean square}
        \idx{least squares approximation}
        \begin{figure}[ht]
            \begin{center}
                \begin{minipage}[t]{0.45\textwidth}
                    \epsfxsize=1.0\linewidth
                    \epsffile{\directory/data.eps}
                    \caption{{\market}, time series data.}
                    \label{\SETLABEL:TS}
                    \label{\SETLABELQ:TS}
                \end{minipage}
                \hfill
                \begin{minipage}[t]{0.45\textwidth}
                    \epsfxsize=1.0\linewidth
                    \epsffile{\directory/data.tsfraction.eps}
                    \caption[{\market}, normalized
                        increments]{{\market}, normalized increments
                        of the time series data presented in
                        Figure~\ref{\SETLABEL:TS}. The mean is
                        {\datafractionmean} with a standard deviation
                        of {\datafractionstddev}. The formula for the
                        least squares approximation is
                        ${\datafractionconstant} +
                        {\datafractionslope}t$, and the root mean
                        squared value is {\datafractionrms}. The
                        graph, labeled ``data\-.tsfraction\-.tsrms,''
                        is the running root mean square, and
                        ``data\-.tsfraction\-.tsavg'' is the running
                        average of the normalized increments.  This
                        graph is the fraction of change in the time
                        series, as a function of time. Note that the
                        slope of the mean, {\datafractionslope}, is
                        the coefficient of the nonlinearity term in
                        the normalized increments. See
                        Chapter~\ref{general}, Section~\ref{nlextend}
                        for a possible application of the logistic
                        function to this data set.}
                    \label{\SETLABEL:TF}
                    \label{\SETLABELQ:TF}
                \end{minipage}
            \end{center}
        \end{figure}

        \subidx{absolute value}{increments}
        \subidx{increments}{absolute value}

        Figure~\ref{\SETLABEL:TFA} is a graph of the absolute value of
        the normalized increments of the time series data presented in
        Figure~\ref{\SETLABEL:TF}. The data presented was made by
        running the Unix utility sed(1) on the normalized increments
        time series data to remove the negative signs. This is an
        absolute value procedure.  The resulting time series contains
        the absolute value of the instantaneous change in the rate of
        revenue returns, divided by the magnitude of the instantaneous
        rate of revenue returns\footnote{The absolute value of the
        normalized increments, when averaged, is related to the root
        mean square of the increments by a constant. If the normalized
        increments are a fixed increment, the constant is unity. If
        the normalized increments have a Gaussian distribution, the
        constant is $\approx 0.8$ depending on the accuracy of of
        ``fit'' to a Gaussian distribution.}.

        \subidx{histogram}{normalized}
        \subidx{normalized}{histogram}
        \subidx{programs}{tsnormal}
        \subidx{tsnormal}{program}
        \subidx{mean}{standard deviation}
        \subidx{standard deviation}{mean}
        \idx{root mean square}
        \idx{least squares approximation}
        \subidx{\market}{analysis of increments}
        Figure~\ref{\SETLABEL:NH} is the normalized histogram of the
        normalized increments of the time series data shown in
        Figure~\ref{\SETLABEL:TF}. The abscissa is 3 $\sigma$ limits,
        and the area under the two curves is identical. The data for
        this figure was produced by the program {\it tsnormal}\/,
        which is described briefly in Appendix~\ref{programs}.

        \begin{figure}[ht]
            \begin{center}
                \begin{minipage}[t]{0.45\textwidth}
                    \epsfxsize=1.0\linewidth
                    \epsffile{\directory/data.tsfraction.abs.eps}
                    \caption[{\market}, absolute value of the
                        normalized increments]{{\market}, absolute
                        value of the normalized increments of the time
                        series data presented in
                        Figure~\ref{\SETLABEL:TF}.  The mean is
                        {\datafractionabsmean} with a standard
                        deviation of {\datafractionabsstddev}. The
                        formula for the least squares approximation is
                        ${\datafractionabsconstant} +
                        {\datafractionabsslope}t$, and the root mean
                        square value, from Figure~\ref{\SETLABEL:TF},
                        is {\datafractionrms}.  The graph, labeled
                        ``data\-.tsfraction\-.tsrms,'' is the running
                        root mean square, and
                        ``data\-.tsfraction\-.tsavg'' is the running
                        average of the normalized increments presented
                        in Figure~\ref{\SETLABEL:TF}, superimposed
                        here for convenience. This graph is the
                        absolute value of the fraction of change in
                        the time series, as a function of time.}
                    \label{\SETLABEL:TFA}
                    \label{\SETLABELQ:TFA}
                \end{minipage}
                \hfill
                \begin{minipage}[t]{0.45\textwidth}
                    \epsfxsize=1.0\linewidth
                    \epsffile{\directory/data.tsfraction.tsnormal-s30.eps}
                    \caption[{\market}, normalized histogram of the
                        normalized increments]{{\market}, normalized
                        histogram of the normalized increments of the
                        time series data shown in
                        Figure~\ref{\SETLABEL:TF}.  The data has a
                        mean of {\datafractionmean}, with a standard
                        deviation of {\datafractionstddev}.  The area
                        under the two curves is identical. The
                        $\chi^2$ value of the observed and expected
                        values of the two curves is {\chisquared},
                        with a critical value of {\critical}.}
                    \label{\SETLABEL:NH}
                \end{minipage}
            \end{center}
        \end{figure}

        \subidx{programs}{tsXsquared}
        \subidx{tsXsquared}{program}
        \subidx{\market}{chi-squared values of increments}
        The program {\it tsXsquared}\/, which is briefly described in
        appendix~\ref{programs}, was used to derive the $\chi^2$
        statistics for the data presented in
        Figure~\ref{\SETLABEL:NH}.

        \subidx{programs}{tsstatest}
        \subidx{tsstatest}{program}
        \subidx{\market}{statistical estimates}

        Figure~\ref{\SETLABEL:SE} is the statistical estimate for the
        data presented in Figure~\ref{\SETLABEL:TF}, as derived by the
        program {\it tsstatest}\/, which is briefly described in
        appendix~\ref{programs}.

        \begin{figure}[ht]
            \begin{center}
                \begin{minipage}[t]{\textwidth}
                    \center{\fbox{\parbox{0.9\textwidth}{\XXX{\directory/data.tsstatest-f0.1-c0.9-i.tex}}}}
                    \caption[{\market}, statistical estimates of the
                        normalized increments]{{\market}, statistical
                        estimates of the normalized increments of the
                        time series shown in Figure~\ref{\SETLABEL:TF}.
                        The table was produced with the {\it
                        tsstatest}\/ program, and illustrates the
                        size of the data set required for a confidence
                        level of 90\%, with an error estimate of $\pm$
                        10\%, or alternately, the error estimate on
                        the time series shown in Figure~\ref{\SETLABEL:TF}.}
                    \label{\SETLABEL:SE}
                \end{minipage}
            \end{center}
        \end{figure}

        Note that the data set size estimations, as produced by the
        {\it tsstatest}\/ program, are probably very conservative,
        depending on the magnitude of the Shannon probability, $P =
        \shannonlogreturns$, as derived in
        Section~\ref{\SETLABEL:SP}. See Chapter~\ref{general},
        Section~\ref{serdss} for possible alternative methodologies
        for addressing the analysis of fractal time series with
        limited data set sizes. Depending on the magnitude of the
        Shannon probability, $P$, these estimates can be several
        orders of magnitude too high.

        \subidx{derivative of increments}{normalized}
        \subidx{normalized}{derivative of increments}
        \subidx{programs}{tsderivative}
        \subidx{tsderivative}{program}
        Figure~\ref{\SETLABEL:TF1} is the normalized histogram of the
        first derivative of the normalized increments of the time
        series data shown in Figure~\ref{\SETLABEL:TF}. In principle,
        if the distribution of the normalized increments presented in
        Figure~\ref{\SETLABEL:NH} is Gaussian in nature, this
        distribution would be similar to ``white noise,'' as presented
        in appendix~\ref{programs}, Figure~\ref{whiteexample}. The
        data was generated by the {\it tsderivative}\/ program, which
        is briefly described in
        appendix~\ref{programs}. Figure~\ref{\SETLABEL:TF2} is the
        normalized histogram of the second derivative of the
        normalized increments of the time series data shown in
        Figure~\ref{\SETLABEL:TF}. In principle, if the distribution
        of the normalized increments presented in
        Figure~\ref{\SETLABEL:NH} is an integrated Gaussian
        distribution in nature, this distribution would be similar to
        ``white noise,'' as presented in appendix~\ref{programs},
        Figure~\ref{whiteexample}.

        \begin{figure}[ht]
            \begin{center}
                \begin{minipage}[t]{0.45\textwidth}
                    \epsfxsize=1.0\linewidth
                    \epsffile{\directory/data.tsfraction.tsderivative.tsnormal-s30.eps}
                    \caption[{\market}, histogram of the first
                        derivative of the increments]{{\market},
                        normalized histogram of the first derivative
                        of the normalized increments of the time
                        series data shown in
                        Figure~\ref{\SETLABEL:TF}.}
                    \label{\SETLABEL:TF1}
                \end{minipage}
                \hfill
                \begin{minipage}[t]{0.45\textwidth}
                    \epsfxsize=1.0\linewidth
                    \epsffile{\directory/data.tsfraction.2tsderivative.tsnormal-s30.eps}
                    \caption[{\market}, histogram of the second
                        derivative of the increments]{{\market},
                        normalized histogram of second derivative of
                        the the normalized increments of the time
                        series data shown in
                        Figure~\ref{\SETLABEL:TF}.}
                    \label{\SETLABEL:TF2}
                \end{minipage}
            \end{center}
        \end{figure}

        \subidx{fractal}{range}
        \subidx{fractal}{R/S analysis}
        \subidx{\market}{rate of revenue returns, range}
        \subidx{\market}{deterministic mechanism}
        \subidx{deterministic}{mechanism}
        \subidx{mechanism}{deterministic}
        Figure~\ref{\SETLABEL:TR} is the range of values of the time
        series shown in Figure~\ref{\SETLABEL:TS}. The horizontal axis
        is time into the future. In principle, if the time series was
        characterized as fractional Brownian motion the graph in
        Figure~\ref{\SETLABEL:TR} would be a square root
        function\footnote{Note that the ``roughness,'' or ``sawtooth''
        characteristics of the graph in Figure~\ref{\SETLABEL:TR} are
        a computational artifact---caused by not using the -m option
        to the program {\it tshurst}\/, which is computationally
        inefficient.}. Figure~\ref{\SETLABEL:TD} is the deterministic
        map of the normalized increments of the time series data shown
        in Figure~\ref{\SETLABEL:TF}. The deterministic map is useful
        for determining if a time series was created by a
        deterministic mechanism. This, essentially, maps each element
        in the time series with the previous element in the time
        series.  See,~\cite[pp. 745]{Peitgen}.

        \begin{figure}[ht]
            \begin{center}
                \begin{minipage}[t]{0.45\textwidth}
                    \epsfxsize=1.0\linewidth
                    \epsffile{\directory/data.tshurst-f.eps}
                    \caption[{\market}, range]{{\market}, range of the
                        time series data shown in
                        Figure~\ref{\SETLABEL:TS}.}
                    \label{\SETLABEL:TR}
                \end{minipage}
                \hfill
                \begin{minipage}[t]{0.45\textwidth}
                    \epsfxsize=1.0\linewidth
                    \epsffile{\directory/data.tsfraction.tsdeterministic.eps}
                    \caption[{\market}, deterministic map]{{\market},
                        deterministic map of the normalized increments
                        of the time series data shown in
                        Figure~\ref{\SETLABEL:TF}.}
                    \label{\SETLABEL:TD}
                \end{minipage}
            \end{center}
        \end{figure}

% Local Variables:
% TeX-parse-self: t
% TeX-auto-save: t
% TeX-master: "fractal.tex"
% End:


            Figure~\ref{\SETLABEL:NH} would seem to indicate that the
            time series data for the {\market} represents a cumulative
            sum/integration of a random process that has a Gaussian
            distribution, (ie., satisfies the Gaussian increments
            property of fractional Brownian
            motion~\cite[pp. 250]{Crownover},) tending to justify the
            assumption that the time series data represents fractional
            Brownian motion.

        %
% -----------------------------------------------------------------------------
%
% A license is hereby granted to reproduce this software source code and
% to create executable versions from this source code for personal,
% non-commercial use.  The copyright notice included with the software
% must be maintained in all copies produced.
%
% THIS PROGRAM IS PROVIDED "AS IS". THE AUTHOR PROVIDES NO WARRANTIES
% WHATSOEVER, EXPRESSED OR IMPLIED, INCLUDING WARRANTIES OF
% MERCHANTABILITY, TITLE, OR FITNESS FOR ANY PARTICULAR PURPOSE.  THE
% AUTHOR DOES NOT WARRANT THAT USE OF THIS PROGRAM DOES NOT INFRINGE THE
% INTELLECTUAL PROPERTY RIGHTS OF ANY THIRD PARTY IN ANY COUNTRY.
%
% Copyright (c) 1994-2006, John Conover, All Rights Reserved.
%
% Comments and/or bug reports should be addressed to:
%
%     john@email.johncon.com (John Conover)
%
% -----------------------------------------------------------------------------
%
% Revision: \RCSRevision \\
% Revision Time: \RCSTime UMT \\
% Revision Date: \RCSDate \\
% Revision Id: \RCSId \\
% Revision File: \RCSLog \\
\RCS $Revision: 0.0 $
\RCS $Date: 2006/01/20 04:38:13 $
\RCS $Id: instant.tex,v 0.0 2006/01/20 04:38:13 john Exp $
% $Log: instant.tex,v $
% Revision 0.0  2006/01/20 04:38:13  john
% Initial version
%
%
    \subsection{Instantaneous Analysis of Normalized Increments}
        \label{\SETLABEL:IA}

        \subidx{\market}{instantaneous analysis of normalized increments}
        \idx{average of normalized increments}
        \idx{root mean square of normalized increments}
        \subidx{Shannon probability}{instantaneous computation of}
        \subidx{average of normalized increments}{instantaneous computation of}
        \subidx{root mean square of normalized increments}{instantaneous computation of}
        \subidx{instantaneous computation}{Shannon probability}
        \subidx{instantaneous computation}{average of normalized increments}
        \subidx{instantaneous computation}{root mean square of normalized increments}
        \idx{time series}
        \subidx{time series}{instantaneous analysis}
        \subidx{instantaneous analysis}{time series}
        \subidx{time series}{increments}
        \subidx{time series}{analysis}
        \subidx{Shannon}{probability}
        \subidx{probability}{Shannon}
        \subidx{normalized}{increments}
        \subidx{increments}{normalized}

        The program {\it tsinstant}\/, which is briefly described in
        Appendix~\ref{programs}, is for finding the instantaneous
        fraction of change in a time series. The value of a sample in
        the time series is subtracted from the previous sample in the
        time series, and divided by the value of the previous sample.
        As explained in Chapter~\ref{general},
        Sections~\ref{derivation},~\ref{GA},~\ref{abmfi},~\ref{aftsma}
        and,~\ref{ompl} for Brownian motion, random walk fractals, the
        absolute value of the instantaneous fraction of change is also
        the root mean square of the instantaneous fraction of
        change\footnote{The absolute value of the normalized
        increments, when averaged, is related to the root mean square
        of the increments by a constant. If the normalized increments
        are a fixed increment, the constant is unity. If the
        normalized increments have a Gaussian distribution, the
        constant is $\approx 0.8$ depending on the accuracy of of
        ``fit'' to a Gaussian distribution.}. Squaring this value is
        the average of the instantaneous fraction of change, and
        adding unity to the absolute value of the instantaneous
        fraction of change, and dividing by two, is the Shannon
        probability of the instantaneous fraction of change.

        Figure~\ref{\SETLABEL:IA1} is the instantaneous value of the
        root mean square of the normalized increments for the
        {\market}, and Figure~\ref{\SETLABEL:IA2} is the instantaneous
        Shannon probability for the normalized increments.

        \begin{figure}[ht]
            \begin{center}
                \begin{minipage}[t]{0.45\textwidth}
                    \epsfxsize=1.0\linewidth
                    \epsffile{\directory/data.tsinstant-r.eps}
                    \caption[{\market}, instantaneous value of
                        rms.]{{\market}, instantaneous value of the
                        root mean square of the normalized increments,
                        provided by running the program {\it
                        tsinstant}\/ with the -r option on the data
                        presented in Figure~\ref{\SETLABEL:TS}.}
                    \label{\SETLABEL:IA1}
                    \label{\SETLABELQ:IA1}
                \end{minipage}
                \hfill
                \begin{minipage}[t]{0.45\textwidth}
                    \epsfxsize=1.0\linewidth
                    \epsffile{\directory/data.tsinstant-s.eps}
                    \caption[{\market}, instantaneous value of
                        Shannon probability.]{{\market}, instantaneous
                        value of the Shannon probability of the
                        normalized increments, provided by running the
                        program {\it tsinstant}\/ with the -s option
                        on the data presented in
                        Figure~\ref{\SETLABEL:TS}.}
                    \label{\SETLABEL:IA2}
                    \label{\SETLABELQ:IA2}
                \end{minipage}
            \end{center}
        \end{figure}

% Local Variables:
% TeX-parse-self: t
% TeX-auto-save: t
% TeX-master: "fractal.tex"
% End:


        %
% -----------------------------------------------------------------------------
%
% A license is hereby granted to reproduce this software source code and
% to create executable versions from this source code for personal,
% non-commercial use.  The copyright notice included with the software
% must be maintained in all copies produced.
%
% THIS PROGRAM IS PROVIDED "AS IS". THE AUTHOR PROVIDES NO WARRANTIES
% WHATSOEVER, EXPRESSED OR IMPLIED, INCLUDING WARRANTIES OF
% MERCHANTABILITY, TITLE, OR FITNESS FOR ANY PARTICULAR PURPOSE.  THE
% AUTHOR DOES NOT WARRANT THAT USE OF THIS PROGRAM DOES NOT INFRINGE THE
% INTELLECTUAL PROPERTY RIGHTS OF ANY THIRD PARTY IN ANY COUNTRY.
%
% Copyright (c) 1994-2006, John Conover, All Rights Reserved.
%
% Comments and/or bug reports should be addressed to:
%
%     john@email.johncon.com (John Conover)
%
% -----------------------------------------------------------------------------
%
% Revision: \RCSRevision \\
% Revision Time: \RCSTime UMT \\
% Revision Date: \RCSDate \\
% Revision Id: \RCSId \\
% Revision File: \RCSLog \\
\RCS $Revision: 0.0 $
\RCS $Date: 2006/01/20 04:38:13 $
\RCS $Id: logistic.tex,v 0.0 2006/01/20 04:38:13 john Exp $
% $Log: logistic.tex,v $
% Revision 0.0  2006/01/20 04:38:13  john
% Initial version
%
%
    \subsection{Logistic Analysis}
        \label{\SETLABEL:LA}

        \subidx{\market}{Logistic function analysis}
        \subidx{time series}{logistic function}
        \subidx{logistic function}{time series}
        \subidx{time series}{increments}
        \subidx{time series}{analysis}
        \subidx{cumulative sum}{analysis}
        \subidx{analysis}{cumulative sum}
        \subidx{analysis}{random process}
        \subidx{random process}{analysis}
        The data in this section is presented in tabular form in
        Section~\ref{\SETLABELREF:LAA}.  Figure~\ref{\SETLABEL:LA1} is
        a graph of the logistic function estimates of the time series
        data for the {\market}. The reader is cautioned that these
        graphs are constructed using the method suggested in
        Chapter~\ref{general}, Section~\ref{nlextend} and enormous
        precision is required for adequate prediction of the logistic
        function,~\cite{Modis}. Particularly, the non-linear term will
        usually require intervention to produce a practical fit to the
        data. In addition, there are numerical stability issues with
        logistic function methodologies\footnote{For example, in
        Figures~\ref{\SETLABEL:LA1} and~\ref{\SETLABEL:LA2}, if the
        non-linear term, $b$, was greater than zero, it was set to
        zero to produce the graphs. See Section~\ref{\SETLABELREF:LAA}
        for the actual derived values. In other cases, the magnitude
        of $b$ was too large, resulting in a graph that was decreasing
        as a function of time}.  The methodology should be regarded as
        ``fragile.'' It is included for completeness.

        \idx{least squares approximation}
        Figure~\ref{\SETLABEL:LA1} is a graph of the logistic function
        for the time series data presented in
        Figure~\ref{\SETLABEL:TS}. The data presented was made by
        running the program {\it tsdlogistic}\/, which is described
        briefly in Appendix~\ref{programs}, on the parameters
        extracted from the time series data as suggested in
        Figure~\ref{\SETLABEL:TF}. The program {\it tslsq}\/ was used
        to derive the constant and the slope of the normalized
        increments of the data presented in Figure~\ref{\SETLABEL:TF}.
        Figure~\ref{\SETLABEL:LA2} is the same graph, but with the
        time scale expanded by a factor of two.

        \begin{figure}[ht]
            \begin{center}
                \begin{minipage}[t]{0.45\textwidth}
                    \epsfxsize=1.0\linewidth
                    \epsffile{\directory/data.tsfraction.tslsq-p.tsdlogistic.eps}
                    \caption[{\market}, logistic function
                        estimates.]{{\market}, logistic function
                        estimates, provided by running the {\it
                        tslsq}\/ program on the normalized increments
                        presented in Figure~\ref{\SETLABEL:TF} with
                        the -p option. These parameters were used as
                        arguments to the {\it tsdlogistic}\/ program.}
                    \label{\SETLABEL:LA1}
                    \label{\SETLABELQ:LA1}
                \end{minipage}
                \hfill
                \begin{minipage}[t]{0.45\textwidth}
                    \epsfxsize=1.0\linewidth
                    \epsffile{\directory/data.tsfraction.tslsq-p.tsdlogistic2.eps}
                    \caption[{\market}, logistic function
                        estimates.]{{\market}, logistic function
                        estimates of Figure~\ref{\SETLABEL:LA1} with
                        the time scale expanded by a factor of two.}
                    \label{\SETLABEL:LA2}
                    \label{\SETLABELQ:LA2}
                \end{minipage}
            \end{center}
        \end{figure}

% Local Variables:
% TeX-parse-self: t
% TeX-auto-save: t
% TeX-master: "fractal.tex"
% End:


        %
% -----------------------------------------------------------------------------
%
% A license is hereby granted to reproduce this software source code and
% to create executable versions from this source code for personal,
% non-commercial use.  The copyright notice included with the software
% must be maintained in all copies produced.
%
% THIS PROGRAM IS PROVIDED "AS IS". THE AUTHOR PROVIDES NO WARRANTIES
% WHATSOEVER, EXPRESSED OR IMPLIED, INCLUDING WARRANTIES OF
% MERCHANTABILITY, TITLE, OR FITNESS FOR ANY PARTICULAR PURPOSE.  THE
% AUTHOR DOES NOT WARRANT THAT USE OF THIS PROGRAM DOES NOT INFRINGE THE
% INTELLECTUAL PROPERTY RIGHTS OF ANY THIRD PARTY IN ANY COUNTRY.
%
% Copyright (c) 1994-2006, John Conover, All Rights Reserved.
%
% Comments and/or bug reports should be addressed to:
%
%     john@email.johncon.com (John Conover)
%
% -----------------------------------------------------------------------------
%
% Revision: \RCSRevision \\
% Revision Time: \RCSTime UMT \\
% Revision Date: \RCSDate \\
% Revision Id: \RCSId \\
% Revision File: \RCSLog \\
\RCS $Revision: 0.0 $
\RCS $Date: 2006/01/20 04:38:13 $
\RCS $Id: hurst.tex,v 0.0 2006/01/20 04:38:13 john Exp $
% $Log: hurst.tex,v $
% Revision 0.0  2006/01/20 04:38:13  john
% Initial version
%
%
    \subsection{Hurst Coefficient Analysis}
        \label{\SETLABEL:H}

        \subidx{\market}{Hurst coefficient analysis}
        \subidx{Hurst coefficient}{analysis}
        \subidx{increments}{normalized}
        \subidx{normalized}{increments}
        \subidx{programs}{tshurst}
        \subidx{tshurst}{program}
        The data in this section is presented in tabular form in
        Section~\ref{\SETLABELREF:HCHP}. Figure~\ref{\SETLABEL:HC} is
        a graph of the Hurst coefficient data time series data shown
        in Figure~\ref{\SETLABEL:TS}. The slope of the graph is the
        Hurst coefficient.  The data for this figure was produced by
        the program {\it tshurst}\/, which is described briefly in
        Appendix~\ref{programs}.

        \subidx{\market}{H parameter analysis}
        \subidx{H parameter}{analysis}
        \subidx{programs}{tshcalc}
        \subidx{tshcalc}{program}
        Figure~\ref{\SETLABEL:HP} is a graph of the H parameter data
        for the normalized increments of the time series data shown in
        Figure~\ref{\SETLABEL:TF}. The data for this figure was
        produced by the program {\it tshcalc}\/, which is described
        briefly in Appendix~\ref{programs}.

        \begin{figure}[ht]
            \begin{center}
                \begin{minipage}[t]{0.45\textwidth}
                    \epsfxsize=1.0\linewidth
                    \epsffile{\directory/data.tshurst.eps}
                    \caption[{\market}, Hurst coefficient data]{{\market},
                        Hurst coefficient data for the normalized
                        increments of the time series data shown in
                        Figure~\ref{\SETLABEL:TF}.  The slope of the graph
                        is the Hurst coefficient.}
                    \label{\SETLABEL:HC}
                \end{minipage}
                \hfill
                \begin{minipage}[t]{0.45\textwidth}
                    \epsfxsize=1.0\linewidth
                    \epsffile{\directory/data.tshcalc.eps}
                    \caption[{\market}, H parameter data]{{\market}, H
                        parameter data for the normalized increments of
                        the time series data shown in
                        Figure~\ref{\SETLABEL:TF} The slope of the graph
                        is the H parameter.}
                    \label{\SETLABEL:HP}
                \end{minipage}
            \end{center}
        \end{figure}

        \subidx{revenue}{See, rate of revenue returns}
        \subidx{returns}{See, rate of revenue returns}
        \subidx{\market}{revenues}
        \subidx{Hurst coefficient}{analysis}
        \subidx{\market}{Hurst coefficient analysis}
        \subidx{\market}{rate of change}
        \subidx{\market}{windows of opportunity}
        \subidx{rate of revenue returns}{forecast}
        \subidx{forecast}{rate of revenue returns}
        \idx{windows of opportunity}
        \subidx{programs}{tslsq}
        \subidx{tslsq}{program}

        The approximately linear slope of the graph in
        Figure~\ref{\SETLABEL:HC} implies that the variance of the
        rate of revenue returns, (per {\timescale},) in the {\market},
        $V(t_2 - t_1)$, over a period of time is proportional to the
        period of time raised to twice the Hurst
        coefficient~\cite[pp. 180]{Feder},~\cite[pp. 246]{Crownover}.
        This seems to be a quantitative statement concerning how fast,
        and to what degree, the rate of revenue returns' state of
        affairs can change over a period of time.  An additional
        implication, for Hurst coefficients sufficiently close to 0.5,
        is that the probability of the state of affairs repeating
        sometime in the future goes down with increasing
        time\footnote{It can be shown that the number of expected
        market ``high'' and ``low'' transitions, $N$, scales with the
        square root of time, or $N \propto \sqrt {t}$, meaning that
        the cumulative distribution of the probability, $P$, of the
        duration of a market's ``high'' or ``low'' exceeding a given
        time interval, $t$, is proportional to the reciprocal of the
        square root of the time interval, $P \propto 1 / \sqrt {t}$,
        (or, conversely, that the probability of the duration of a
        market's ``high'' or ``low'' exceeding a given time interval
        is proportional to the reciprocal of the time interval raised
        to the power $3 / 2$, ie., $P \propto 1 / t^{3 /
        2}$,~\cite[pp. 153]{Schroeder}. What this means is that a
        histogram of the ``zero free'' run-lengths of a market being
        ``high'' or ``low,'' over a long time, would have a $1 / t^{3
        / 2}$ characteristic.)}, $t$, $p(t) = erf (1/\sqrt{2t})$ which
        is approximately $1/\sqrt{t}$ for $t \gg
        1$~\cite[pp. 160]{Schroeder}. Figures~\ref{\SETLABEL:FN},
        and,~\ref{\SETLABEL:FF} compare methods of approximation of
        the ``forecastability'' of the rate of revenue returns in the
        {\market} for the near term and far term,
        respectively~\cite[pp. 83-84]{Peters:CAOITCM}\footnote{The
        author is not comfortable with Peters' interpretation. For
        example, if the algorithm explained
        in~\cite[pp. 82]{Peters:CAOITCM} is used on ``white noise''
        which, by definition, never has any correlations, the short
        term Hurst coefficient, and thus the ``forecastability,'' is
        still near unity---a bit of an enigma. This can be verified
        with the {\it tswhite}\/ and {\it tshurst}\/ programs, which
        are briefly described in Appendix~\ref{programs}.}.  This
        seems to be a quantitative statement concerning ``windows of
        opportunity'' in the rate of revenue returns, (per
        {\timescale}.)  The program {\it tslsq}\/ was used on the
        Hurst coefficient data, presented in
        Figure~\ref{\SETLABEL:HC}, to provide a least squares
        approximation to the Hurst coefficient. The superimposed least
        squares approximation with on original Hurst coefficient data
        is presented.  The time series data has a Hurst coefficient of
        {\thurstlow}, so that:

        \subidx{\market}{Hurst coefficient analysis}
        \begin{eqnarray}
            V\left(t_2 - t_1\right) & \propto & \left(t_2 - t_1\right)^{2 \cdot H}\\
            V\left(t_2 - t_1\right) & \propto & \left(t_2 - t_1\right)^{2 \cdot {\thurstlow}}\\
                                    & \propto & \left(t_2 - t_1\right)^{\thurstlowtwo}
            \label{\SETLABEL:V}
        \end{eqnarray}

        \subidx{fractional}{Brownian motion}
        \subidx{Brownian motion}{fractional}
        \idx{fractal}
        \noindent where $V(t_2 - t_1)$ is the variance of the
        increments of the rate of revenue returns, (per {\timescale},)
        over the time interval $t_2 -
        t_1$,~\cite[pp. 177]{Feder},~\cite[pp. 494]{Peitgen}. If $H >
        \frac{1}{2}$, then the time series is termed as being
        characterized by ``fractional Brownian
        motion~\cite[pp. 170]{Feder}.''

        \subidx{rate of revenue returns}{predictability}
        \subidx{rate of revenue returns}{forecastability}
        \subidx{rate of revenue returns}{consistency}
        \subidx{predictability}{rate of revenue returns}
        \subidx{forecastability}{rate of revenue returns}
        \subidx{consistency}{rate of revenue returns}
        \subidx{\market}{rate of revenue returns, predictability}
        \subidx{\market}{rate of revenue returns, forecastability}
        \subidx{\market}{rate of revenue returns, consistency}
        \subidx{Hurst coefficient}{analysis}
        \subidx{\market}{Hurst coefficient analysis}
        \subidx{\market}{rate of change}

        In some sense, the Hurst coefficient is a quantitative
        expression of the ``forecastability'' of the future based on
        the past\footnote{Actually, in general, when summing fractal
        entities, the method used should be a root mean square
        process, dependent on the Hurst Coefficient, $H$, where
        $P_{total}^H = P_1^H + P_2^H + \cdots$, where $P_n$ is the
        fractal entities. For a Brownian motion, or random walk type
        of fractal the Hurst Coefficient is a function of time into
        the future. For the ``near term,'' the Hurst coefficient is
        very near unity, meaning the summation process is linear. For
        the ``long term,'' $H \approx 0.5$, or a standard root mean
        square summation process should be used. If $H$ is $0.5$ then
        the market is termed a Brownian motion, or random walk
        process. If it is larger than 0.5, it is termed fractional
        Brownian motion process. For a random walk process, ``near
        term'' and ``far term'' are quantitatively differentiated on
        the Hurst Coefficient graph where $1 - \ln (t) = 0.5 \cdot \ln
        (t)$, or when $\ln (t) = 2$, or $t = 7.389\ldots$ See
        Section~\ref{\SETLABEL:FS} for the particulars on using Hurst
        Coefficient to sum fractal process' for the {\market}. See
        also~\cite[pp. 67, 83-84]{Peters:CAOITCM} and~\cite[pp. 129,
        159]{Schroeder} for particulars on the implications of the
        Hurst Coefficient and root mean square summation issues.}.  A
        Hurst coefficient of {\thurstlow}, (for the near future, and
        {\thurstall} for the distant future.) implies that the
        likelihood of the rate of revenue returns, (per {\timescale},)
        for any two consecutive {\timescale}s being the same is
        {\thurstlowhundred}\%~\cite[pp. 66]{Peters:CAOITCM} for the
        near future, and {\thurstall} for the distant
        future. Likewise, there is a {\thurstlowhundred}\% chance of
        the rate of revenue returns, (per {\timescale},) movements
        being the same in consecutive time periods---ie., if, in a
        given {\timescale}, the rate of revenue returns, (per
        {\timescale},) is increasing, there is a {\thurstlowhundred}\%
        that the rate of revenue returns, (per {\timescale},) will
        increase in the following period, also. In some sense, this is
        a quantitative statement on how ``predictable,'' or
        ``forecastable'' the rate of revenue returns, (per
        {\timescale},) for the {\market} are over time, since the
        probability of having $n$ many consecutive {\timescale}s of
        the same agenda is $H^n$ where $H$ is the Hurst coefficient,
        or, letting the short term probability of having $n$ many
        {\timescale}s of the same market agenda, $p_a$, is:

        \begin{eqnarray}
            p_a\left(n\right) & = & H^{n}\\
                              & = & {\thurstlow}^{n}
            \label{\SETLABEL:MA}
        \end{eqnarray}

        \subidx{rate of revenue returns}{predictability}
        \subidx{rate of revenue returns}{forecastability}
        \subidx{rate of revenue returns}{consistency}
        \subidx{predictability}{rate of revenue returns}
        \subidx{forecastability}{rate of revenue returns}
        \subidx{consistency}{rate of revenue returns}
        As an interesting interpretation of the normalized increments
        of the time series data presented in
        Figure~\ref{\SETLABEL:TF}, if the vertical axis is multiplied
        by 100, to convert to percent, then the graph represents the
        error, in percent, that would be made by forecasting, month by
        month, that the next {\timescale}'s rate of revenue returns
        would be the same as the current {\timescale}'s revenue
        rate. Interestingly, it is $\datafractionmean \cdot 100$
        percent, on the average, with a standard deviation of
        $\datafractionstddev \cdot 100$ percent, and a root mean
        square error value of $\datafractionrms \cdot 100$
        percent---small values for such a simple forecasting
        mechanism.

        \subidx{\market}{rate of revenue returns, range}
        \subidx{Hurst coefficient}{analysis}
        \subidx{\market}{Hurst coefficient analysis}
        \subidx{\market}{rate of change}

        This is, essentially, a statement of the range of values, in
        the increments of the rate of revenue returns, (per
        {\timescale},) that is to be expected over the time interval,
        $t_2 - t_1$,
        $R_v$,~\cite[pp. 178]{Feder},~\cite[pp. 172]{Cambel}:

        \begin{eqnarray}
            R_v\left(t_2 - t_1\right) & \propto & \left(t_2 - t_1\right)^{H}\\
                                      & \propto & \left(t_2 - t_1\right)^{\thurstlow}
            \label{\SETLABEL:R}
        \end{eqnarray}

        \subidx{\market}{rate of revenue returns, range}
        \subidx{Hurst coefficient}{analysis}
        \subidx{\market}{Hurst coefficient analysis}
        \subidx{\market}{rate of change}
        \subidx{Markov}{statistics}
        \subidx{statistics}{Markov}
        \noindent where $R$ is the range of values in the increments
        of the rate of revenue returns, (per {\timescale}.) A Hurst
        coefficient, $H$, that is much larger than $\frac{1}{2}$, (but
        less than 1,) implies a strongly non-Gaussian distribution in
        the increments of the rate of revenue returns, (per
        {\timescale},)~\cite[pp. 152, 194]{Feder}, and a Hurst
        coefficient near $\frac{1}{2}$ implies that the increments of
        the rate of revenue returns, (per {\timescale}) is
        characteristic of an independent
        process~\cite[pp. 195]{Feder}. Extreme caution should be
        exercised in using Markov statistics in any analysis where the
        Hurst coefficient is not
        $\frac{1}{2}$,~\cite[pp. 124]{Crownover},~\cite[pp. 106]{Peters:CAOITCM}.


        As a useful approximation, if $H$, is approximately
        $\frac{1}{2}$, Equation~\ref{\SETLABEL:R} reduces
        to,~\cite[pp. 129]{Schroeder}:

        \begin{eqnarray}
            R\left(t_2 - t_1\right) & \propto & (t_2 - t_1)^{\frac{1}{2}}\\
                                    & \propto & \sqrt{\left(t_2 - t_1\right)}
        \end{eqnarray}

        \subidx{\market}{rate of revenue returns, range}
        \subidx{\market}{rate of revenue returns, increase and decrease}
        \subidx{Hurst coefficient}{analysis}
        \subidx{\market}{Hurst coefficient analysis}
        \subidx{\market}{rate of change}
        \subidx{Markov}{statistics}
        \subidx{statistics}{Markov}

        In the case where the Hurst coefficient, $H$, is
        $\frac{1}{2}$, the range of values in the increments of the
        rate of revenue returns, (per {\timescale},) divided by the
        standard deviation of these values, $S$, can be anticipated to
        increase over time according to the following
        relation,~\cite[pp. 154]{Feder},~\cite[pp. 129]{Schroeder}:

        \begin{equation}
            \frac{R\left(t_2 - t_1\right)}{S} \propto \left(t_2 - t_1\right)^{\frac{1}{2}}
        \end{equation}

        \subidx{\market}{rate of revenue returns, range}
        \subidx{\market}{rate of revenue returns, increase and decrease}
        \subidx{Hurst coefficient}{analysis}
        \subidx{\market}{Hurst coefficient analysis}
        \subidx{\market}{rate of change}
        \noindent which is a useful conceptual approximation, since it
        involves only the square root function---if the range and the
        standard deviation of the increments of the rate of revenue
        returns, (per {\timescale},) are known, (and $H \approx
        \frac{1}{2}$,) then the expected change in $\frac{R}{S}$, will
        increase with the square root of time\footnote{To be precise,
        it is actually asymptotically proportional to
        $\tau^{\frac{1}{2}}$}.

        Another useful approximation when rescaling processes that are
        characterize by Brownian motion, (ie., when $H \approx
        \frac{1}{2}$,) is that:

        \begin{eqnarray}
            X\left(t\right) & \propto & \frac{X\left(rt\right)}{r^{H}}\\
                            & \propto & \frac{X\left(rt\right)}{r^{\thurstlow}}
        \end{eqnarray}

        \idx{Brownian motion}
        \idx{fractal}
        Where $X(t)$ is the process characterized by Brownian motion,
        and $r$ is a scaling factor,~\cite[pp. 494]{Peitgen}.

        \subidx{programs}{tslsq}
        \subidx{tslsq}{program}
        The program {\it tslsq}\/ was used on the H parameter data,
        presented in Figure~\ref{\SETLABEL:HP}, to provide a least
        squares approximation to the H parameter for the
        {\market}. The superimposed least squares approximation on the
        original H parameter data is presented.  By contrast, the H
        parameter, as derived by the methodology outlined
        in~\cite[pp. 249]{Crownover}, is {\thcalclow} for the near
        future, and {\thcalcall} for the distant future.

        \subidx{\market}{Hurst coefficient analysis}
        \subidx{Hurst coefficient}{analysis}
        \subidx{increments}{normalized}
        \subidx{normalized}{increments}
        \subidx{programs}{tshurst}
        \subidx{tshurst}{program}
        \subidx{\market}{H parameter analysis}
        \subidx{H parameter}{analysis}
        \subidx{programs}{tshcalc}
        \subidx{tshcalc}{program}
        Figures~\ref{\SETLABEL:HC} and~\ref{\SETLABEL:HP} represent
        Hurst coefficient and H parameter data that are derived from
        the normalized increments, shown in
        Figure~\ref{\SETLABEL:TF}. In this case, the data is
        considered a normalized derivative of the time series data
        presented in Figure~\ref{\SETLABEL:TF}, instead of a
        cumulative sum.  The program, {\it tshurst}\/, is described
        briefly in appendix~\ref{programs}, and the data for
        figures~\ref{\SETLABEL:THC} and~\ref{\SETLABEL:THP} was made
        using the -d option.

        \begin{figure}[ht]
            \begin{center}
                \begin{minipage}[t]{0.45\textwidth}
                    \epsfxsize=1.0\linewidth
                    \epsffile{\directory/data.tsfraction.tshurst-d.eps}
                    \caption[{\market}, traditional Hurst coefficient
                        data]{{\market}, traditional Hurst coefficient
                        data for the time series data shown in
                        Figure~\ref{\SETLABEL:TS}.  The slope of the
                        graph is the Hurst coefficient, and is
                        {\hurstlow} for the near term, and
                        {\hurstall} for the far term.}
                    \label{\SETLABEL:THC}
                \end{minipage}
                \hfill
                \begin{minipage}[t]{0.45\textwidth}
                    \epsfxsize=1.0\linewidth
                    \epsffile{\directory/data.tsfraction.tshcalc-d.eps}
                    \caption[{\market}, traditional H parameter
                        data]{{\market}, traditional H parameter data
                        for the time series data shown in
                        Figure~\ref{\SETLABEL:TS} The slope of the
                        graph is the H parameter, and is {\hcalclow}
                        for the near term, and {\hcalcall} for the
                        far term.}
                    \label{\SETLABEL:THP}
                \end{minipage}
            \end{center}
        \end{figure}

% Local Variables:
% TeX-parse-self: t
% TeX-auto-save: t
% TeX-master: "fractal.tex"
% End:


        %
% -----------------------------------------------------------------------------
%
% A license is hereby granted to reproduce this software source code and
% to create executable versions from this source code for personal,
% non-commercial use.  The copyright notice included with the software
% must be maintained in all copies produced.
%
% THIS PROGRAM IS PROVIDED "AS IS". THE AUTHOR PROVIDES NO WARRANTIES
% WHATSOEVER, EXPRESSED OR IMPLIED, INCLUDING WARRANTIES OF
% MERCHANTABILITY, TITLE, OR FITNESS FOR ANY PARTICULAR PURPOSE.  THE
% AUTHOR DOES NOT WARRANT THAT USE OF THIS PROGRAM DOES NOT INFRINGE THE
% INTELLECTUAL PROPERTY RIGHTS OF ANY THIRD PARTY IN ANY COUNTRY.
%
% Copyright (c) 1994-2006, John Conover, All Rights Reserved.
%
% Comments and/or bug reports should be addressed to:
%
%     john@email.johncon.com (John Conover)
%
% -----------------------------------------------------------------------------
%
% Revision: \RCSRevision \\
% Revision Time: \RCSTime UMT \\
% Revision Date: \RCSDate \\
% Revision Id: \RCSId \\
% Revision File: \RCSLog \\
\RCS $Revision: 0.0 $
\RCS $Date: 2006/01/20 04:38:13 $
\RCS $Id: fiscal.tex,v 0.0 2006/01/20 04:38:13 john Exp $
% $Log: fiscal.tex,v $
% Revision 0.0  2006/01/20 04:38:13  john
% Initial version
%
%
    \subsection{Fixed Increment Approximation for Fiscal Strategy}
        \label{\SETLABEL:FS}

        \subidx{\market}{fiscal strategy}
        \subidx{markets}{analysis}
        \subidx{analysis}{markets}
        \subidx{strategy}{fiscal}
        \subidx{fiscal}{strategy}
        The data in this section is presented in tabular form in
        Section~\ref{\SETLABELREF:LR}. This section derives various
        values based on the ``average'' of the normalized increments
        presented in Figure~\ref{\SETLABEL:TFA}. These values are an
        approximation to a, probably, complex process with a
        distribution shown in Figure~\ref{\SETLABEL:TF}. These values
        will be used in a fixed increment Brownian fractal analysis
        and simulation of the {\market}, and may, or may not, provide
        adequate accuracy for projections.

        For an organization operating in the {\market}, the fiscal
        strategy, commensurate with the aggregate environment, can be
        derived as follows~\cite[pp. 128, pp
        151]{Schroeder},~\cite[pp. 450]{Reza},~\cite[pp. 270]{Pierce}:
        \vspace{0.15in}

        \subsubsection{Logarithmic Returns}
            \label{\SETLABEL:LR}

            \subidx{logarithmic}{returns}
            \subidx{returns}{logarithmic}
            \subidx{\market}{logarithmic returns}
            The logarithmic returns can be calculated by various
            means. Four will be presented here, for comparison.

            \subidx{programs}{tsnormal}
            \subidx{tsnormal}{program}
            \subidx{logarithmic}{returns}
            \subidx{returns}{logarithmic}
            The logarithmic returns, in bits, $bits$, as computed from
            the mean, by the program {\it tsnormal}\/, which is
            described in Chapter~\ref{programs}, and is presented in
            Figure~\ref{\SETLABEL:TF}, and Equation~\ref{abits} from
            Section~\ref{ereturns} in Chapter~\ref{general}:

            \begin{equation}
                bits = \frac{\ln \left({\datafractionmean} + 1\right)}{\ln \left(2\right)} = \datafractionmeanbits
            \end{equation}

            \subidx{programs}{tslsq}
            \subidx{tslsq}{program}
            \subidx{logarithmic}{returns}
            \subidx{returns}{logarithmic}
            \noindent By comparison, the logarithmic returns, in bits,
            $bits$, as computed from the constant in the least squares
            approximation, using the program {\it tslsq}\/, which is briefly
            described in Chapter~\ref{programs}, as presented in
            Figure~\ref{\SETLABEL:TF}, and Equation~\ref{abits} from
            Section~\ref{ereturns} in Chapter~\ref{general}:

            \begin{equation}
                bits = \frac{\ln \left({\datafractionconstant} + 1\right)}{\ln \left(2\right)} = \datafractionconstantbits
            \end{equation}

            Note that if the mean is not constant in
            Figure~\ref{\SETLABEL:TF}, this method will not provide
            accurate results.

            \subidx{programs}{tslsq}
            \subidx{tslsq}{program}
            \subidx{logarithmic}{returns}
            \subidx{returns}{logarithmic}
            \noindent And by yet another comparison, using the program
            {\it tslsq}\/, which is briefly described in
            Chapter~\ref{programs}, with the -e -p options, to provide
            a formula for the least squares exponential fit to the
            time series data set presented in
            Figure~\ref{\SETLABEL:TS}:

            \begin{equation}
                bits = {\datatslsqepbits}
            \end{equation}

            \subidx{programs}{tslogreturns}
            \subidx{tslogreturns}{program}
            \subidx{logarithmic}{returns}
            \subidx{returns}{logarithmic}
            \noindent And finally, by comparison, from the
            {\it tslogreturns}\/ program, which is briefly described
            in Chapter~\ref{programs}, with the -p option, to provide
            a formula for the logarithmic returns of the time series
            data set presented in Figure~\ref{\SETLABEL:TS}:

            \begin{equation}
                bits = {\logreturns}
            \end{equation}

        \subsubsection{Calculation of Shannon Probability}
            \label{\SETLABEL:SP}

            \subidx{\market}{Shannon probability}
            Ideally, all of the values presented in
            Section~\ref{\SETLABEL:LR} would be equal. Using the
            logarithmic returns provided by the {\it tslogreturns}\/
            program, to be consistent
            with~\cite[pp. 81]{Peters:CAOITCM}

            \subidx{programs}{tslogreturns}
            \subidx{tslogreturns}{program}
            \begin{equation}
                2^{{\logreturns}t}
            \end{equation}

            \noindent therefore:
            \begin{equation}
                C\left(p\right) = {\logreturns}
            \end{equation}
            \subidx{programs}{tsshannon}
            \subidx{tsshannon}{program}
            \subidx{Shannon}{probability}
            \subidx{probability}{Shannon}
            \noindent and, {\it tsshannon}\/ {\logreturns} gives:
            \begin{equation}
                \label{\SETLABEL:F0}
                C\left({\shannonlogreturns}\right) = {\logreturns}
            \end{equation}
            \noindent therefore:
            \begin{eqnarray}
                2^{C\left({\shannonlogreturns}\right)} & = & 2^{\logreturns}\\
                                                       & = & {\twologreturns}\\
                                                       & = & {\twologreturnshundred}\%
            \end{eqnarray}
            \noindent and:
            \begin{eqnarray}
                2p - 1 & = & \left(2 \cdot {\shannonlogreturns}\right) - 1\\
                       & = & {\twopone}\\
                       \label{\SETLABEL:F1}
                       & = & {\twoponehundred}\%
            \end{eqnarray}

            \subidx{\market}{fiscal strategy}
            \subidx{markets}{analysis}
            \subidx{analysis}{markets}
            \subidx{strategy}{fiscal}
            \subidx{fiscal}{strategy}
            \subidx{\market}{fiscal strategy}
            \subidx{\market}{growth rate}
            Presuming the simplified assumptions outlined in
            Section~\ref{assumptions}, the ``typical'' organization
            operating in the {\market} executes a long term fiscal
            strategy, commensurate with the aggregate environment,
            that is to invest, every {\timescale}, in sufficient
            additional resources and infrastructure, to increase the
            manufacturing of goods and services by {\twoponehundred}\%
            of its rate of revenue returns, (per {\timescale}.) As a
            conceptual model, the remaining {\hundredtwoponehundred}\%
            will be held in ``reserve'' with a
            {\shannonlogreturnshundred}\% chance of making twice the
            {\twoponehundred}\% back, (and a
            {\hundredshannonlogreturnshundred}\% chance of making
            0.0,) in one {\timescale}, on the average, for an average
            growth in its rate of revenue returns, (per {\timescale},)
            of {\twologreturnshundred}\%, or a doubling of its rate of
            revenue returns, (per {\timescale},) in
            {\oneoverlogreturns} {\timescale}s.

        \subsubsection{Example Fixed Increment Approximation Fiscal Strategies}

            \subidx{\market}{fiscal strategy}
            \subidx{markets}{analysis}
            \subidx{analysis}{markets}
            \subidx{strategy}{fiscal}
            \subidx{fiscal}{strategy}
            \subidx{\market}{fiscal strategy}
            \subidx{\market}{growth rate}
            \subidx{\market}{management metric}
            \idx{management metric}
            A possible metric on the effectiveness of long term fiscal
            management could possibly be that if an investment of
            {\twoponehundred}\% per {\timescale} of the rate of
            revenue returns, (per {\timescale},) is made in resources
            and infrastructure, then the rate of revenue returns would
            be expected to increase by {\twologreturnshundred}\%, per
            {\timescale}, on average.

            Note that the metrics presented in this section are
            representative of the {\market} as an aggregate whole, and
            may or may not be accurate representations for any
            particular participant in the environment. Of interest to
            the participants in the environment would be a similar
            analysis of each product or service rendered in the
            marketplace.

            \subidx{\market}{fiscal strategy}
            \subidx{markets}{analysis}
            \subidx{analysis}{markets}
            \subidx{strategy}{fiscal}
            \subidx{fiscal}{strategy}
            \subidx{\market}{fiscal strategy}
            As a simple illustrative example, a company operating in
            this environment might obtain a credit line from a bank
            that is equal to {\twoponehundred}\% of its rate of
            revenue returns, (per {\timescale},) to finance additional
            operations. In this simple scenario, the company would use
            its revenue base as collateral for the loan. Some
            {\timescale}s, depending on the {\market}'s environment,
            the company's rate of revenue returns exceeds what was
            borrowed from the bank, and the loan is repaid in
            full. Other {\timescale}s, the company must default, and
            the bank seizes a portion of the company's revenue base to
            pay the delinquent loan. However, on the average, the
            company will expand its rate of revenue returns at
            {\twologreturnshundred}\% per {\timescale}.

            \subidx{\market}{fiscal strategy}
            \subidx{markets}{analysis}
            \subidx{analysis}{markets}
            \subidx{strategy}{fiscal}
            \subidx{fiscal}{strategy}
            \subidx{\market}{fiscal strategy}
            As another simple example, a company re-invests
            {\twoponehundred}\% of its rate of revenue returns, (per
            {\timescale},) in development, marketing, sales, and
            distribution of new products.  Although some products will
            be successful and the return on the investment will exceed
            the {\twoponehundred}\% per {\timescale} investment,
            others will not. However, on the average, the company will
            expand it gross rate of revenue returns at
            {\twologreturnshundred}\% per {\timescale}.

            \subidx{\market}{fiscal strategy}
            \subidx{markets}{analysis}
            \subidx{analysis}{markets}
            \subidx{strategy}{fiscal}
            \subidx{fiscal}{strategy}
            \subidx{\market}{fiscal strategy}
            \subidx{\market}{product portfolio}
            \subidx{\market}{product diversity}
            \subidx{\market}{product mix}
            \subidx{\market}{optimum number of products}
            \idx{product portfolio}
            \idx{product diversity}
            \idx{optimum number of products}
            \idx{product mix}

            As an example of ``product portfolio'' management, suppose
            a company re-invests {\twoponehundred}\% of its rate of
            revenue returns, (per {\timescale},) in development,
            marketing, sales, and distribution of new products.
            Further suppose that the company has two products, and a
            fractal analysis of the individual product rate of revenue
            return time series indicates that one product has a
            Shannon probability of 0.65, and the other has a Shannon
            probability of 0.55. Then the percentage of re-investment
            in the first product would be $(2 \cdot 0.65 - 1) \cdot
            {\twoponehundred}$, percent of the rate of revenue
            returns, and $(2 \cdot 0.55 - 1) \cdot {\twoponehundred}$
            percent for the second product, implying that the company
            should diversify its product line\footnote{The astute
            reader would note that the linear addition was used to add
            the contribution to development of each product. This is a
            ``near term'' interpretation. Actually, in general, the
            method used should be a root mean square process,
            dependent on the Hurst Coefficient, $H$, where
            $P_{total}^H = P_1^H + P_2^H + \cdots$, where $P_n$ is the
            contribution to each individual product. For a Brownian
            motion, or random walk type of fractal the Hurst
            Coefficient is a function of time into the future. For the
            ``near term,'' the Hurst coefficient is very near unity,
            meaning the summation process is linear. For the ``long
            term,'' $H \approx 0.5$, or a standard root mean square
            summation process should be used. If $H$ is $0.5$ then the
            market is termed a Brownian motion, or random walk
            process. If it is larger than 0.5, it is termed fractional
            Brownian motion process. For a random walk process, ``near
            term'' and ``far term'' are quantitatively differentiated
            on the Hurst Coefficient graph where $1 - \ln (t) = 0.5
            \cdot \ln (t)$, or when $\ln (t) = 2$, or $t =
            7.389\ldots$ See~\cite[pp. 67, 83-84]{Peters:CAOITCM}
            and~\cite[pp. 129, 159]{Schroeder} for particulars on the
            implications of the Hurst Coefficient and root mean square
            summation issues.}.  Note that this is a ``bet hedging''
            metric methodology, and assumes that the products have
            uncorrelated revenue return rates. If this re-investment
            methodology is not feasible, perhaps for strategic
            financial reasons, then the re-investment in both products
            should total the ${\twoponehundred}$\%, and the investment
            in each product should be made at a ratio of $\frac{(2
            \cdot 0.65 - 1)}{(2 \cdot 0.55 - 1)} = 3 : 1$,
            respectively. Note that this ``bet hedging'' can be used
            to define the optimal number of products that can be
            supported on the rate of revenue returns. If it assumed
            that all products are ``typical'' for the {\market}, as a
            standard bench mark, then the optimal number will be
            $\frac{1}{{\twopone}}$. Note that this is a
            ``theoretical'' value, since not all products are
            ``typical,'' and there may be strategic reasons, for
            example product leveraging, that may increase the number
            of products above the optimum. However, most of the
            revenue should come from the optimal number of products,
            since having more products will decrease the amount of the
            potential investment in each product, and having less than
            the optimum number of products will increase the risk that
            many of the products could suffer a ``down market''
            concurrently, impacting the rate of revenue returns.  As
            another interesting interpretation of the optimal
            ``hedging of bets,'' in product portfolio strategy, and
            considering the graph of the normalized increments
            presented in Figure~\ref{\SETLABEL:TF}, if the
            organization is running optimally, then these products
            will generate, at least in principle, one standard
            deviation, approximately $0.8413 = 84.13$\% of the future
            growth in rate of revenue returns. Naturally, these are
            approximations, and the values are an approximation to a,
            probably, complex process, and appropriate scrutiny should
            be exercised before making specific projections.  As yet
            another example of ``product portfolio'' management,
            consider the issue of product mix. In this interpretation,
            {\twoponehundred}\% of the product manufactured should be
            ``proprietary,'' while the rest is ``industry standard.''
            As yet another possibility, {\twoponehundred}\% of the
            product manufactured should be predatory into new markets,
            and the remainder in markets that are ``traditional'' for
            the company.

% Local Variables:
% TeX-parse-self: t
% TeX-auto-save: t
% TeX-master: "fractal.tex"
% End:


        %
% -----------------------------------------------------------------------------
%
% A license is hereby granted to reproduce this software source code and
% to create executable versions from this source code for personal,
% non-commercial use.  The copyright notice included with the software
% must be maintained in all copies produced.
%
% THIS PROGRAM IS PROVIDED "AS IS". THE AUTHOR PROVIDES NO WARRANTIES
% WHATSOEVER, EXPRESSED OR IMPLIED, INCLUDING WARRANTIES OF
% MERCHANTABILITY, TITLE, OR FITNESS FOR ANY PARTICULAR PURPOSE.  THE
% AUTHOR DOES NOT WARRANT THAT USE OF THIS PROGRAM DOES NOT INFRINGE THE
% INTELLECTUAL PROPERTY RIGHTS OF ANY THIRD PARTY IN ANY COUNTRY.
%
% Copyright (c) 1994-2006, John Conover, All Rights Reserved.
%
% Comments and/or bug reports should be addressed to:
%
%     john@email.johncon.com (John Conover)
%
% -----------------------------------------------------------------------------
%
% Revision: \RCSRevision \\
% Revision Time: \RCSTime UMT \\
% Revision Date: \RCSDate \\
% Revision Id: \RCSId \\
% Revision File: \RCSLog \\
\RCS $Revision: 0.0 $
\RCS $Date: 2006/01/20 04:38:13 $
\RCS $Id: companies.tex,v 0.0 2006/01/20 04:38:13 john Exp $
% $Log: companies.tex,v $
% Revision 0.0  2006/01/20 04:38:13  john
% Initial version
%
%
    \subsection{Number of Companies}
        \label{\SETLABEL:QNC}

        \subidx{\market}{number of companies}
        \subidx{number of companies}{analysis}
        \subidx{analysis}{number of companies}
        \subidx{Shannon}{probability}
        \subidx{probability}{Shannon}
        This section evaluates the approximate, or ``average,'' number
        of companies in the {\market}, and uses the method outlined in
        Chapter~\ref{general}, Section~\ref{aftsma}. Since the
        average, $avg_{ind}$, and the root mean square, $rms_{ind}$,
        of the normalized increments of the {\market} time series is
        \datafractionmean, and \datafractionrms respectively, the
        number of companies participating in the market can be
        calculated by Equation~\ref{ncompanies} to be {\ncompanies}.

        If this value seems consistent number of companies in the
        {\market}, within the assumptions outlined in
        Chapter~\ref{general}, Section~\ref{aftsma}, then it would
        seem that there is some circumstantial or indirect evidence
        that the companies participating in the {\market} are
        operating optimally, and the ``average'' Shannon probability,
        $P$ for each participating company would be, using
        Equation~\ref{pncompanies}, {\pncompanies}, which would be the
        value which should be used in Section~\ref{\SETLABEL:FS} for
        each participating company if market expansion was to be
        consistent with the rest of the industry. However, if the
        Shannon probability derived in Section~\ref{\SETLABEL:FS} is
        greater than the average Shannon probability for the companies
        participating in the {\market}, as derived in this section,
        then the market would, possibly, be exploitable with the
        fiscal strategy outlined in Section~\ref{\SETLABEL:FS}. The
        maximum exploitability for the {\market} is derived in
        Section~\ref{\SETLABEL:MAXSHANNON}, but it is probably of
        doubtful practicality.

        Note that these optimizations would maximize a company's
        market growth. Since there are probably many companies
        competing in the market place, this would not necessarily
        maximize a company's P\&L, as described in
        Chapter~\ref{general}, Section~\ref{ompl}. The Shannon
        probability that maximizes market share in the {\market} is
        \pncompanies, with several alternative solutions listed in the
        previous paragraph. However, these should be contrasted to the
        Shannon probability that maximizes a company's P\&L which is
        \avgrms~in the {\market}. In all cases, the fraction of the
        P\&L that should be ``wagered'' on the future, $f$, should be:

        \begin{equation}
            f = 2P - 1
        \end{equation}

        \noindent where $P$ is the particular Shannon probability
        chosen optimize a particular fiscal strategy. Interestingly,
        the measured Shannon probability of the {\market} would tend
        to indicate that the companies participating in the market
        have chosen a fiscal strategy that optimizes market growth, as
        opposed to capital growth.

        \subidx{\market}{increasing returns}
        \subidx{economic increasing returns}{\market}
        As interesting interpretation of these exploitive issues,
        since all three fiscal strategies will result in exponential
        market growth for every company participating in the market,
        is that they may represent, perhaps, an example of
        ``increasing returns.''

% Local Variables:
% TeX-parse-self: t
% TeX-auto-save: t
% TeX-master: "fractal.tex"
% End:


        %
% -----------------------------------------------------------------------------
%
% A license is hereby granted to reproduce this software source code and
% to create executable versions from this source code for personal,
% non-commercial use.  The copyright notice included with the software
% must be maintained in all copies produced.
%
% THIS PROGRAM IS PROVIDED "AS IS". THE AUTHOR PROVIDES NO WARRANTIES
% WHATSOEVER, EXPRESSED OR IMPLIED, INCLUDING WARRANTIES OF
% MERCHANTABILITY, TITLE, OR FITNESS FOR ANY PARTICULAR PURPOSE.  THE
% AUTHOR DOES NOT WARRANT THAT USE OF THIS PROGRAM DOES NOT INFRINGE THE
% INTELLECTUAL PROPERTY RIGHTS OF ANY THIRD PARTY IN ANY COUNTRY.
%
% Copyright (c) 1994-2006, John Conover, All Rights Reserved.
%
% Comments and/or bug reports should be addressed to:
%
%     john@email.johncon.com (John Conover)
%
% -----------------------------------------------------------------------------
%
% Revision: \RCSRevision \\
% Revision Time: \RCSTime UMT \\
% Revision Date: \RCSDate \\
% Revision Id: \RCSId \\
% Revision File: \RCSLog \\
\RCS $Revision: 0.0 $
\RCS $Date: 2006/01/20 04:38:13 $
\RCS $Id: operations.tex,v 0.0 2006/01/20 04:38:13 john Exp $
% $Log: operations.tex,v $
% Revision 0.0  2006/01/20 04:38:13  john
% Initial version
%
%
    \subsection{Fixed Increment Approximation for Operational Strategy}
        \label{\SETLABEL:OPS}.

        This section derives various values based on the ``average''
        of the normalized increments presented in
        Figure~\ref{\SETLABEL:TFA}. These values are an approximation
        to a, probably, complex process with a distribution shown in
        Figure~\ref{\SETLABEL:TF}. These values will be used in a
        fixed increment Brownian fractal analysis and simulation of
        the {\market}, and may, or may not, provide adequate accuracy
        for projections.

        \subidx{\market}{fiscal strategy}
        \subidx{\market}{Shannon probability}
        \subidx{strategy}{fiscal}
        \subidx{fiscal}{strategy}
        \subidx{Shannon}{probability}
        \subidx{probability}{Shannon}
        It should be noted that the analysis of fiscal strategy,
        presented in Section~\ref{\SETLABEL:FS}, is derived from the
        {\market} metrics and may, or may not, be maximally
        optimal. For the optimal fiscal strategy, which may be
        exploitable, see Section~\ref{\SETLABEL:MAXSHANNON}.

        \subidx{strategy}{exploitable}
        \subidx{exploitable}{strategy}
        \subidx{\market}{windows of opportunity}
        \idx{windows of opportunity}
        \subidx{decision}{obsolete}
        \subidx{obsolete}{decision}
        \subidx{decision}{timeliness}
        \subidx{timeliness}{decision}
        \subidx{rate of revenue returns}{forecast}
        \subidx{forecast}{rate of revenue returns}
        An additional exploitable strategy may be time itself.
        Equations~\ref{\SETLABEL:V},~\ref{\SETLABEL:R},
        and,~\ref{\SETLABEL:MA}, are, essentially, metrics on how fast
        a decision, which is based on information concerning the
        current status of the {\market}, becomes obsolete. Obviously,
        how long a decision is expected to remain relevant should be
        addressed as an operational necessity in strategic planning
        and project management. Figures~\ref{\SETLABEL:FN},
        and,~\ref{\SETLABEL:FF} compare methods of approximation of
        the ``forecastability'' of rate of revenue returns in the
        {\market} for the near term and far
        term~\cite[pp. 83-84]{Peters:CAOITCM}, respectively. As a
        general rule, caution must be exercised when making decisions
        that will span a time interval larger than the time interval
        where the ``forecastability'' of rate of revenue returns drops
        below 50\%. Beyond this time interval, the chances increase
        that the competitive and market forces will alter the market
        environment in a possibly detrimental unanticipated
        fashion. Obviously, there is significant advantage in
        ``timeliness'' of development, manufacturing, and distribution
        of products and services that are consistent with this
        temporal agenda. Automation of these processes, if executed
        consistently with this agenda, should be considered a
        competitive advantage.

        \subidx{strategy}{exploitable}
        \subidx{exploitable}{strategy}
        \subidx{rate of revenue returns}{forecast}
        \subidx{forecast}{rate of revenue returns}
        \idx{product life cycle}
        \idx{life cycle, product}
        In some sense, this temporal agenda defines the ``average''
        product or service life cycle in the {\market}. When the
        ``forecastability'' of rate of revenue returns drops below
        50\%, there is an even chance that the rate of revenue returns
        for the product or service will change in a detrimental
        fashion. If it is assumed that a product or service life cycle
        consists of a ramp up, a maintenence interval, and a ramp
        down, then, if all three life cycle intervals are equal, the
        product life cycle will be, approximately, three times the
        time interval where the ``forecastability'' of rate of revenue
        returns drops below 50\%. Although probably not an accurate
        prediction of product or service life cycle, the technique may
        be used as a conceptual approximation to the dynamics of
        ``market windows.\footnote{For example, consider the market
        for table salt. Since it has inelastic supply and demand
        curves, and is a necessary requirement for life, it would be
        expected that the Hurst coefficient would be very near
        unity---ignoring competitive pressures in the market. The
        predictability of the table salt market would, therefore, be
        expected to be relatively good, over time.}''  The conceptual
        approximation will probably predict a ``conservative'' or
        ``pessimistic'' value in relation to actual markets.

        \begin{figure}[ht]
            \begin{center}
                \begin{minipage}[t]{0.45\textwidth}
                    \epsfxsize=1.0\linewidth
                    \epsffile{\directory/datahurstlownear.eps}
                    \caption[{\market}, ``forecastability'' of near
                        term rate of revenue returns]{{\market},
                        ``forecastability'' of near term rate of
                        revenue returns. Although the error function
                        is the most accurate, for the near term,
                        $H^{t} = \thurstlow^{t}$ may be used as a
                        reliable metric of ``forecastability'' of the
                        rate of revenue returns.}
                    \label{\SETLABEL:FN}
                \end{minipage}
                \hfill
                \begin{minipage}[t]{0.45\textwidth}
                    \epsfxsize=1.0\linewidth
                    \epsffile{\directory/datahurstlowfar.eps}
                    \caption[{\market}, ``forecastability'' of far
                        term rate of revenue returns]{{\market},
                        ``forecastability'' of far term rate of
                        revenue returns. Although the error function
                        is the most accurate, for the far term,
                        $\frac{1}{\sqrt{t}}$ may be used as a reliable
                        metric of ``forecastability'' of the rate of
                        revenue returns.}
                    \label{\SETLABEL:FF}
                \end{minipage}
            \end{center}
        \end{figure}

        \idx{operations research}
        As an interesting interpretation of the data presented in
        Figure~\ref{\SETLABEL:FN}, there may be, perhaps, some
        applicability to such operational agendas as inventory
        control. Maintaining too little inventory, obviously, will
        create a situation where the organization can not exploit
        market expansion, and maintaining too much inventory,
        likewise, would over extend the company, creating unnecessary
        losses when the market contracts. The company should maintain
        inventory levels that do not exceed, from
        Equation~\ref{\SETLABEL:MA}, ${\thurstlow}^{n} = 0.5$
        {\timescale}s of operations. Since the optimal amount of
        inventory and, from Equation~\ref{\SETLABEL:V}, the variance
        of change in the rate of revenue returns in the future can be
        calculated, there may, perhaps, be some applicability to a
        forecasting methodology that can be incorporated into other
        areas of operations research, for example the linear algebras
        using simplex methodologies for optimization of manufacturing
        processes. Traditionally, these forecasts are made by the
        sales department, and are subject to various subjective
        biases.

% Local Variables:
% TeX-parse-self: t
% TeX-auto-save: t
% TeX-master: "fractal.tex"
% End:


        %
% -----------------------------------------------------------------------------
%
% A license is hereby granted to reproduce this software source code and
% to create executable versions from this source code for personal,
% non-commercial use.  The copyright notice included with the software
% must be maintained in all copies produced.
%
% THIS PROGRAM IS PROVIDED "AS IS". THE AUTHOR PROVIDES NO WARRANTIES
% WHATSOEVER, EXPRESSED OR IMPLIED, INCLUDING WARRANTIES OF
% MERCHANTABILITY, TITLE, OR FITNESS FOR ANY PARTICULAR PURPOSE.  THE
% AUTHOR DOES NOT WARRANT THAT USE OF THIS PROGRAM DOES NOT INFRINGE THE
% INTELLECTUAL PROPERTY RIGHTS OF ANY THIRD PARTY IN ANY COUNTRY.
%
% Copyright (c) 1994-2006, John Conover, All Rights Reserved.
%
% Comments and/or bug reports should be addressed to:
%
%     john@email.johncon.com (John Conover)
%
% -----------------------------------------------------------------------------
%
% Revision: \RCSRevision \\
% Revision Time: \RCSTime UMT \\
% Revision Date: \RCSDate \\
% Revision Id: \RCSId \\
% Revision File: \RCSLog \\
\RCS $Revision: 0.0 $
\RCS $Date: 2006/01/20 04:38:13 $
\RCS $Id: simulation.tex,v 0.0 2006/01/20 04:38:13 john Exp $
% $Log: simulation.tex,v $
% Revision 0.0  2006/01/20 04:38:13  john
% Initial version
%
%
    \subsection{Simulation of Fixed Increment Approximation for Fiscal Strategy}
        \label{\SETLABEL:TSUNFAIRBROWNIAN}

        \subidx{\market}{market simulation}
        The data in this section is presented in tabular form in
        Section~\ref{\SETLABELREF:SIM}.
        Figure~\ref{\SETLABEL:TSUNFAIRBROWNIAN0} represents a
        constructional simulation of the time series data presented in
        Figure~\ref{\SETLABEL:TS}. The program {\it
        tsunfairbrownian}\/, which is briefly described in
        appendix~\ref{programs}, was used in the reconstruction. The
        reconstructed data is superimposed on the original time series
        data.  The program, {\it tsunfairbrownian}\/, essentially,
        constructs the new time series as a Brownian fractal with
        fixed increments---the value of the fixed increment is derived
        from the root mean square average of the normalized increments
        presented in Figure~\ref{\SETLABEL:TF}. The ``quality'' of
        such a reconstruction should be subject to adequate scepticism
        and scrutiny since, in all probability, the normalized
        increments presented in Figure~\ref{\SETLABEL:TF} represent a
        relatively complex process, that may not be ``modeled'' with
        such a simple methodology.

        As a further comparison of the the constructional simulation
        with the original time series data,
        Figure~\ref{\SETLABEL:TSUNFAIRBROWNIAN1} presents a normalized
        histogram of the normalized increments of the reconstructed
        time series, superimposed on the normalized histogram
        presented in Figure~\ref{\SETLABEL:NH}.

        \subidx{\market}{fiscal strategy, simulation}
        \subidx{markets}{simulation}
        \subidx{simulation}{markets}
        \subidx{strategy}{fiscal, simulation}
        \subidx{fiscal}{strategy, simulation}
        \subidx{programs}{tsunfairbrownian}
        \subidx{tsunfairbrownian}{program}
        \begin{figure}[ht]
            \begin{center}
                \begin{minipage}[t]{0.45\textwidth}
                    \epsfxsize=1.0\linewidth
                    \epsffile{\directory/tsunfairbrownian-f.eps}
                    \caption[{\market}, Time series data, empirical and
                        simulated]{{\market}, Time series data, empirical
                        and simulated, using the program {\it tsunfairbrownian}\/
                        with f = {\datafractionrms}. This data is
                        superimposed on the data presented in
                        Figure~\ref{\SETLABEL:TS}.}
                    \label{\SETLABEL:TSUNFAIRBROWNIAN0}
                \end{minipage}
                \hfill
                \begin{minipage}[t]{0.45\textwidth}
                    \epsfxsize=1.0\linewidth
                    \epsffile{\directory/tsunfairbrownian-f.tsfraction.tsnormal-s30.eps}
                    \caption[{\market}, normalized histogram,
                        empirical and simulated]{{\market}, normalized
                        histogram of the normalized increments of the
                        time series data shown in
                        Figure~\ref{\SETLABEL:TSUNFAIRBROWNIAN0},
                        empirical and simulated.  The empirical data
                        has a mean of {\datafractionmean}, with a
                        standard deviation of {\datafractionstddev}.
                        By comparison, the simulated data has a mean
                        of {\tsunfairbrownianfractionmean} with a
                        standard deviation of
                        {\tsunfairbrownianfractionstddev}. This data
                        is superimposed on the data presented in
                        Figure~\ref{\SETLABEL:NH}. The area under the
                        four curves is identical.}
                    \label{\SETLABEL:TSUNFAIRBROWNIAN1}
                \end{minipage}
            \end{center}
        \end{figure}

% Local Variables:
% TeX-parse-self: t
% TeX-auto-save: t
% TeX-master: "fractal.tex"
% End:


        %
% -----------------------------------------------------------------------------
%
% A license is hereby granted to reproduce this software source code and
% to create executable versions from this source code for personal,
% non-commercial use.  The copyright notice included with the software
% must be maintained in all copies produced.
%
% THIS PROGRAM IS PROVIDED "AS IS". THE AUTHOR PROVIDES NO WARRANTIES
% WHATSOEVER, EXPRESSED OR IMPLIED, INCLUDING WARRANTIES OF
% MERCHANTABILITY, TITLE, OR FITNESS FOR ANY PARTICULAR PURPOSE.  THE
% AUTHOR DOES NOT WARRANT THAT USE OF THIS PROGRAM DOES NOT INFRINGE THE
% INTELLECTUAL PROPERTY RIGHTS OF ANY THIRD PARTY IN ANY COUNTRY.
%
% Copyright (c) 1994-2006, John Conover, All Rights Reserved.
%
% Comments and/or bug reports should be addressed to:
%
%     john@email.johncon.com (John Conover)
%
% -----------------------------------------------------------------------------
%
% Revision: \RCSRevision \\
% Revision Time: \RCSTime UMT \\
% Revision Date: \RCSDate \\
% Revision Id: \RCSId \\
% Revision File: \RCSLog \\
\RCS $Revision: 0.0 $
\RCS $Date: 2006/01/20 04:38:13 $
\RCS $Id: maximum.tex,v 0.0 2006/01/20 04:38:13 john Exp $
% $Log: maximum.tex,v $
% Revision 0.0  2006/01/20 04:38:13  john
% Initial version
%
%
    \subsection{Simulation of Fixed Increment Approximation for Optimally Maximal Fiscal Strategy}
        \label{\SETLABEL:MAXSHANNON}
        \subidx{\market}{fiscal strategy, simulation}
        \subidx{\market}{maximum Shannon probability}
        \subidx{markets}{simulation}
        \subidx{simulation}{markets}
        \subidx{strategy}{optimum fiscal, simulation}
        \subidx{fiscal}{optimum strategy, simulation}
        \subidx{programs}{tsunfairbrownian}
        \subidx{tsunfairbrownian}{program}
        \subidx{Shannon}{probability}
        \subidx{probability}{Shannon}

        \subidx{strategy}{exploitable}
        \subidx{exploitable}{strategy}
        \subidx{programs}{tsshannonmax}
        \subidx{tsshannonmax}{program}
        \subidx{programs}{tsunfairbrownian}
        \subidx{tsunfairbrownian}{program}
        \subidx{strategy}{fiscal}
        \subidx{fiscal}{strategy}
        The data in this section is presented in tabular form in
        Section~\ref{\SETLABELREF:MAXSHANNON}. One of the issues of
        analysis, as mentioned in Section~\ref{\SETLABEL:OPS}, is to
        determine the maximum Shannon probability for the time series
        presented in Figure~\ref{\SETLABEL:TS}. Potentially, this
        could be exploited with an aggressive fiscal
        strategy. Figure~\ref{\SETLABEL:SHANNONMAX0} is a graph of the
        output of the {\it tsshannonmax}\/ program, which is described
        briefly in appendix~\ref{programs}. The maximum of this
        function is the maximum Shannon probability for the time
        series data presented in Figure~\ref{\SETLABEL:TS}.
        Figure~\ref{\SETLABEL:SHANNONMAX1} was constructed using {\it
        tsunfairbrownian}\/ program, which is also described in
        appendix~\ref{programs}, with the maximum Shannon probability,
        and the time series data presented in
        Figure~\ref{\SETLABEL:TS}. This represents a ``what if'' the
        investment strategy was changed from a Shannon probability of
        {\shannonlogreturns}, as derived in Section~\ref{\SETLABEL:SP}
        to {\shannonmax}. This process, essentially, extracts the
        random statistical data from the time series presented in
        Figure~\ref{\SETLABEL:TS}, and constructs a new time series,
        using the random statistical data, with a different investment
        strategy.  The program, {\it tsunfairbrownian}\/, essentially,
        constructs the new time series as a Brownian fractal with
        fixed increments.  The ``quality'' of such a reconstruction
        should be subject to adequate scepticism and scrutiny since,
        in all probability, the increments in the original data
        represent a relatively complex process, that may not be
        ``modeled'' with such a simple methodology.

        \begin{figure}[ht]
            \begin{center}
                \begin{minipage}[t]{0.45\textwidth}
                    \epsfxsize=1.0\linewidth
                    \epsffile{\directory/data.tsshannonmax.eps}
                    \caption[{\market}, maximum rate of revenue
                        returns] {{\market}, maximum rate of revenue
                        returns, per {\timescale}, vs. Shannon
                        probability. The maximum rate of revenue
                        returns, per {\timescale}, occurs at a Shannon
                        probability of {\shannonmax}.}
                    \label{\SETLABEL:SHANNONMAX0}
                \end{minipage}
                \hfill
                \begin{minipage}[t]{0.45\textwidth}
                    \epsfxsize=1.0\linewidth
                    \epsffile{\directory/data.tsshannonmax-p.tsunfairbrownian-p.eps}
                    \caption[{\market}, maximum rate of revenue
                        returns] {{\market}, maximum rate of revenue
                        returns, per {\timescale}, at a Shannon
                        probability, of {\shannonmax}, corresponding
                        to a ``wager'' fraction of {\twoponemax}.}
                    \label{\SETLABEL:SHANNONMAX1}
                \end{minipage}
            \end{center}
        \end{figure}

        \subidx{fractional}{Brownian motion}
        \subidx{Brownian motion}{fractional}
        \subidx{Shannon}{probability}
        \subidx{probability}{Shannon}
        \subidx{programs}{tsshannonmax}
        \subidx{tsshannonmax}{program}
        If it is assumed that the time series data set, presented in
        Figure~\ref{\SETLABEL:TS}, constitutes classical Brownian
        motion, then the Shannon probability can be calculated by
        counting the total number of {\timescale}s that the {\market}
        movement was positive, and dividing by the total number of
        {timescale}s represented in the time series. This quotient is
        {\pmax}, as compared with the predicted value from the program
        {\it tsshannonmax}\/ of {\shannonmax}.

% Local Variables:
% TeX-parse-self: t
% TeX-auto-save: t
% TeX-master: "fractal.tex"
% End:


        %
% -----------------------------------------------------------------------------
%
% A license is hereby granted to reproduce this software source code and
% to create executable versions from this source code for personal,
% non-commercial use.  The copyright notice included with the software
% must be maintained in all copies produced.
%
% THIS PROGRAM IS PROVIDED "AS IS". THE AUTHOR PROVIDES NO WARRANTIES
% WHATSOEVER, EXPRESSED OR IMPLIED, INCLUDING WARRANTIES OF
% MERCHANTABILITY, TITLE, OR FITNESS FOR ANY PARTICULAR PURPOSE.  THE
% AUTHOR DOES NOT WARRANT THAT USE OF THIS PROGRAM DOES NOT INFRINGE THE
% INTELLECTUAL PROPERTY RIGHTS OF ANY THIRD PARTY IN ANY COUNTRY.
%
% Copyright (c) 1994-2006, John Conover, All Rights Reserved.
%
% Comments and/or bug reports should be addressed to:
%
%     john@email.johncon.com (John Conover)
%
% -----------------------------------------------------------------------------
%
% Revision: \RCSRevision \\
% Revision Time: \RCSTime UMT \\
% Revision Date: \RCSDate \\
% Revision Id: \RCSId \\
% Revision File: \RCSLog \\
\RCS $Revision: 0.0 $
\RCS $Date: 2006/01/20 04:38:13 $
\RCS $Id: verification.tex,v 0.0 2006/01/20 04:38:13 john Exp $
% $Log: verification.tex,v $
% Revision 0.0  2006/01/20 04:38:13  john
% Initial version
%
%
    \subsection{Qualitative Verification of Fixed Increment Approximation Analysis}
        \label{\SETLABEL:QVA}

        \subidx{\market}{verification of analysis}
        \subidx{verification}{analysis}
        \subidx{analysis}{verification}
        \subidx{quality}{of analysis}
        \subidx{verification}{of methodology}
        \subidx{methodology}{verification of}
        \subidx{Shannon}{probability}
        \subidx{probability}{Shannon}

        This section evaluates various values based on the ``average''
        of the normalized increments presented in
        Figure~\ref{\SETLABEL:TFA}. These values are an approximation
        to a, probably, complex process with a distribution shown in
        Figure~\ref{\SETLABEL:TF}. These values will be used in a
        fixed increment Brownian fractal analysis of the {\market},
        and may, or may not, provide adequate accuracy for
        projections.

        The data in this section is presented in tabular form in
        sections~\ref{\SETLABELREF:VI1} and~\ref{\SETLABELREF:VI2}.
        As a subjective evaluation of the ``quality'' of the analysis
        of the {\market}, from Chapter~\ref{methodology},
        Equation~\ref{metricvalues1}, and using the mean and root mean
        square values of the normalized increments of the time series
        data presented in Figure~\ref{\SETLABEL:TS} from
        Figure~\ref{\SETLABEL:TF}, and the Shannon probability as
        calculated by counting the total number of {\timescale}s that
        the {\market} movement was positive, as presented in
        Section~\ref{\SETLABEL:MAXSHANNON}:

        \begin{eqnarray}
                  P & \approx & \frac{\frac{avg}{rms} + 1}{2}\\
            {\pmax} & \approx & \frac{\frac{\datafractionmean}{\datafractionrms} + 1}{2}\\
            {\pmax} & \approx & {\avgrms}
            \label{\SETLABEL:AVGS}
        \end{eqnarray}

        \subidx{Shannon}{probability}
        \subidx{probability}{Shannon}
        \noindent and comparing these values to the Shannon
        probability, as found by the {\it tsshannonmax}\/ program, which
        iterates for a maximum:

        \begin{eqnarray}
            {\pmax} \approx {\avgrms} \approx {\shannonmax}
        \end{eqnarray}

        \subidx{logarithmic}{returns}
        \subidx{returns}{logarithmic}
        In addition, the different methods of calculating the
        logarithmic returns, presented in Section~\ref{\SETLABEL:FS},
        should be compared. The four methods used were the mean of
        Figure~\ref{\SETLABEL:TF}, the constant in the least squares
        approximation to Figure~\ref{\SETLABEL:TF}, the least squares
        exponential approximation to Figure~\ref{\SETLABEL:TS}, and
        the logarithmic returns of Figure~\ref{\SETLABEL:TS}, derived
        as the mean of the logarithms of the quotients of the
        increments. The values for each of the methods are,
        respectively:

        \begin{equation}
            \datafractionmeanbits \approx \datafractionconstantbits \approx \datatslsqepbits \approx \logreturns
        \end{equation}

        It is implied in Section~\ref{\SETLABEL:FS},
        Subsection~\ref{\SETLABEL:SP} and in
        Section~\ref{\SETLABEL:TSUNFAIRBROWNIAN} that, a Brownian
        motion with fixed increments fractal may ``model'' the
        {\market}. Using Equation~\ref{stddev9} from
        Chapter~\ref{general}, Section~\ref{abmfi}:

        \begin{eqnarray}
                                    rms \left(2P - 1\right) & \approx & \frac{\sigma \left(2P - 1\right)}{2 \sqrt{P\left(1 - P\right)}}\\
            \datafractionrms \left(2 \cdot \pmax - 1\right) & \approx & \frac{\datafractionstddev \left(2 \cdot \pmax - 1\right)}{2\sqrt{\pmax \left(1 - \pmax\right)}}\\
                       \datafractionrms \cdot \twopminusone & \approx & \datafractionstddev \cdot \twopx\\
                                                      \rmsp & \approx & \sigmap
        \end{eqnarray}

        \noindent and, equating to the mean:

        \begin{equation}
            \datafractionmean \approx \rmsp \approx \sigmap
        \end{equation}

        \subidx{Shannon}{probability}
        \subidx{probability}{Shannon}
        \noindent where, as in Equation~\ref{\SETLABEL:AVGS} using the
        mean, root mean square, and standard deviation values of the
        normalized increments of the time series data presented in
        Figure~\ref{\SETLABEL:TS} from Figure~\ref{\SETLABEL:TF}, and
        the Shannon probability as calculated by counting the total
        number of {\timescale}s that the {\market} movement was
        positive, as presented in Section~\ref{\SETLABEL:MAXSHANNON}.

        As a final qualitative comparison, the absolute value of the
        normalized increments should be the same as the root mean
        square value\footnote{The absolute value of the normalized
        increments, when averaged, is related to the root mean square
        of the increments by a constant. If the normalized increments
        are a fixed increment, the constant is unity. If the
        normalized increments have a Gaussian distribution, the
        constant is $\approx 0.8$ depending on the accuracy of of
        ``fit'' to a Gaussian distribution.}, where the absolute value
        is presented in Figure~\ref{\SETLABEL:TFA}, and the root mean
        square value is presented in Figure~\ref{\SETLABEL:TF}:

        \begin{equation}
            \datafractionabsmean \approx \datafractionrms
        \end{equation}

        Note, that if the {\market} could be ``modeled'' as a Brownian
        motion with fixed increments fractal, then the standard
        deviation of the absolute value of the normalized increments
        of the time series data presented in Figure~\ref{\SETLABEL:TS}
        from Figure~\ref{\SETLABEL:TF} should be zero. It is
        $\datafractionabsstddev$.

% Local Variables:
% TeX-parse-self: t
% TeX-auto-save: t
% TeX-master: "fractal.tex"
% End:


    \renewcommand{\market}{Non-optimal Logistic Coins Tossing Game}
    \renewcommand{\directory}{../markets/tscoins-b}
    \renewcommand{\datafractionmean}{0.008052}
\renewcommand{\datafractionmeanbits}{0.011570}
\renewcommand{\datafractionmeanq}{0.002684}
\renewcommand{\datafractionmeanbitsq}{0.003867}
\renewcommand{\datafractionstddev}{0.038579}
\renewcommand{\datafractionrms}{0.039311}
\renewcommand{\avgrms}{0.602414}
\renewcommand{\ncompanies}{5.210454}
\renewcommand{\pncompanies}{0.544866}
\renewcommand{\datafractionabsmean}{0.029745}
\renewcommand{\datafractionabsstddev}{0.025769}
\renewcommand{\datafractionconstant}{0.010041}
\renewcommand{\datafractionconstantbits}{0.014414}
\renewcommand{\datafractionconstantq}{0.003347}
\renewcommand{\datafractionconstantbitsq}{0.004821}
\renewcommand{\datafractionslope}{-0.000021}
\renewcommand{\datafractionabsconstant}{0.035145}
\renewcommand{\datafractionabsslope}{-0.000057}
\renewcommand{\hurstall}{0.659558}
\renewcommand{\hurstlow}{0.707509}
\renewcommand{\hurstlowtwo}{1.415018}
\renewcommand{\hurstlowhundred}{70.750900}
\renewcommand{\hcalcall}{0.184942}
\renewcommand{\hcalclow}{0.102042}
\renewcommand{\shannonmax}{0.604167}
\renewcommand{\twoponemax}{0.208334}
\renewcommand{\logreturns}{0.010456}
\renewcommand{\twologreturns}{1.007274}
\renewcommand{\twologreturnshundred}{0.727387}
\renewcommand{\oneoverlogreturns}{95.638868}
\renewcommand{\pmax}{0.602094}
\renewcommand{\twopminusone}{0.204188}
\renewcommand{\rmsp}{0.008027}
\renewcommand{\twopx}{0.208583}
\renewcommand{\sigmap}{0.008047}
\renewcommand{\tsunfairbrownianfractionmean}{0.007862}
\renewcommand{\tsunfairbrownianfractionstddev}{0.038619}
\renewcommand{\shannonlogreturns}{0.560125}
\renewcommand{\shannonlogreturnshundred}{56.012500}
\renewcommand{\twopone}{0.120250}
\renewcommand{\twoponehundred}{12.025000}
\renewcommand{\hundredtwoponehundred}{87.975000}
\renewcommand{\hundredshannonlogreturnshundred}{43.987500}
\renewcommand{\datatslsqepbits}{0.007623}
\renewcommand{\thurstall}{0.633980}
\renewcommand{\thurstlow}{0.710108}
\renewcommand{\thurstlowtwo}{1.420216}
\renewcommand{\thurstlowhundred}{71.010800}
\renewcommand{\thcalcall}{0.247886}
\renewcommand{\thcalclow}{0.171737}
\renewcommand{\chisquared}{2.862000}
\renewcommand{\critical}{42.557000}

    \renewcommand{\timescale}{tosses}
    \subidx{market}{\market}
    \idx{\market}

    \section{\market}
        \subidx{market}{non-linearity}
        \subidx{non-linearity}{market}
        \subidx{logistic}{function}

        \renewcommand{\SETLABEL}{\LABPRE:NOLCST}
        \renewcommand{\SETLABELQ}{\LABPRE:NOLCSTQ}
        \label{\SETLABEL}
        \renewcommand{\SETLABELREF}{\LABPREREF:NOLCST}

        \subidx{tscoins}{program}
        \subidx{programs}{tscoins}
        For the analysis, the data was in the directory
        {\directory}\footnote{As a simulation model, the program {\it
        tscoins}\/ was run to make a time series data file, with the
        following parameters:

        \vspace{0.1in}
        {\noindent}tscoins -p -b 0.00000005 0.6 -b 0.03 1000 > data
        \vspace{0.1in}

        \noindent to make a time series of 1000 elements, with a
        Shannon probability of 0.6 and a known non-optimal investment
        strategy.  The non-linearity term of the logistic function is
        $0.00000005$. Otherwise, the first 300 elements of the
        simulation is approximately the same as in
        Section~\ref{\LABPRE:NOCST}.  Note that there is some
        possibility that the analytical techniques used could be used
        to determine the maturity of an industrial market.  See
        Chapter~\ref{general}, Section~\ref{nlextend}. The data is by
        {\timescale}.}.

        The data in this section is presented in tabular form in
        Section~\ref{\SETLABELREF}. Note that in this analysis, the
        rate of revenue returns means the increase or decrease in the
        cumulative sum of the {\market}. This is included for
        ``theoretical'' comparative purposes.

        %
% -----------------------------------------------------------------------------
%
% A license is hereby granted to reproduce this software source code and
% to create executable versions from this source code for personal,
% non-commercial use.  The copyright notice included with the software
% must be maintained in all copies produced.
%
% THIS PROGRAM IS PROVIDED "AS IS". THE AUTHOR PROVIDES NO WARRANTIES
% WHATSOEVER, EXPRESSED OR IMPLIED, INCLUDING WARRANTIES OF
% MERCHANTABILITY, TITLE, OR FITNESS FOR ANY PARTICULAR PURPOSE.  THE
% AUTHOR DOES NOT WARRANT THAT USE OF THIS PROGRAM DOES NOT INFRINGE THE
% INTELLECTUAL PROPERTY RIGHTS OF ANY THIRD PARTY IN ANY COUNTRY.
%
% Copyright (c) 1994-2006, John Conover, All Rights Reserved.
%
% Comments and/or bug reports should be addressed to:
%
%     john@email.johncon.com (John Conover)
%
% -----------------------------------------------------------------------------
%
% Revision: \RCSRevision \\
% Revision Time: \RCSTime UMT \\
% Revision Date: \RCSDate \\
% Revision Id: \RCSId \\
% Revision File: \RCSLog \\
\RCS $Revision: 0.0 $
\RCS $Date: 2006/01/20 04:38:13 $
\RCS $Id: fraction.tex,v 0.0 2006/01/20 04:38:13 john Exp $
% $Log: fraction.tex,v $
% Revision 0.0  2006/01/20 04:38:13  john
% Initial version
%
%
    \subsection{Time Series Increments Analysis}
        \label{\SETLABEL:TSA}

        \subidx{\market}{Time series analysis}
        \subidx{time series}{increments}
        \subidx{time series}{analysis}
        \subidx{cumulative sum}{analysis}
        \subidx{analysis}{cumulative sum}
        \subidx{analysis}{random process}
        \subidx{random process}{analysis}
        \subidx{Gaussian}{increments}
        \subidx{increments}{Gaussian}
        \subidx{Brownian}{motion, fractional}
        \subidx{fractional}{Brownian motion}
        \subidx{fractal}{Brownian motion}
        The data in this section is presented in tabular form in
        Section~\ref{\SETLABELREF:TSA}.  Figure~\ref{\SETLABEL:TS} is
        a graph of the time series data for the {\market}.

        \subidx{increments}{normalized}
        \subidx{normalized}{increments}
        \subidx{programs}{tsfraction}
        \subidx{tsfraction}{program}
        Figure~\ref{\SETLABEL:TF} is a graph of the normalized
        increments of the time series data presented in
        Figure~\ref{\SETLABEL:TS}. The data presented was made by
        running the program {\it tsfraction}\/ on the time series
        data. The program {\it tsfraction}\/ is described briefly in
        Appendix~\ref{programs}, and subtracts the previous value from
        the next value, dividing this difference by the previous
        value, for each element in the time series data. The new time
        series contains the instantaneous change in the rate of
        revenue returns, divided by the magnitude of the instantaneous
        rate of revenue returns.

        \subidx{mean}{standard deviation}
        \subidx{standard deviation}{mean}
        \idx{root mean square}
        \idx{least squares approximation}
        \begin{figure}[ht]
            \begin{center}
                \begin{minipage}[t]{0.45\textwidth}
                    \epsfxsize=1.0\linewidth
                    \epsffile{\directory/data.eps}
                    \caption{{\market}, time series data.}
                    \label{\SETLABEL:TS}
                    \label{\SETLABELQ:TS}
                \end{minipage}
                \hfill
                \begin{minipage}[t]{0.45\textwidth}
                    \epsfxsize=1.0\linewidth
                    \epsffile{\directory/data.tsfraction.eps}
                    \caption[{\market}, normalized
                        increments]{{\market}, normalized increments
                        of the time series data presented in
                        Figure~\ref{\SETLABEL:TS}. The mean is
                        {\datafractionmean} with a standard deviation
                        of {\datafractionstddev}. The formula for the
                        least squares approximation is
                        ${\datafractionconstant} +
                        {\datafractionslope}t$, and the root mean
                        squared value is {\datafractionrms}. The
                        graph, labeled ``data\-.tsfraction\-.tsrms,''
                        is the running root mean square, and
                        ``data\-.tsfraction\-.tsavg'' is the running
                        average of the normalized increments.  This
                        graph is the fraction of change in the time
                        series, as a function of time. Note that the
                        slope of the mean, {\datafractionslope}, is
                        the coefficient of the nonlinearity term in
                        the normalized increments. See
                        Chapter~\ref{general}, Section~\ref{nlextend}
                        for a possible application of the logistic
                        function to this data set.}
                    \label{\SETLABEL:TF}
                    \label{\SETLABELQ:TF}
                \end{minipage}
            \end{center}
        \end{figure}

        \subidx{absolute value}{increments}
        \subidx{increments}{absolute value}

        Figure~\ref{\SETLABEL:TFA} is a graph of the absolute value of
        the normalized increments of the time series data presented in
        Figure~\ref{\SETLABEL:TF}. The data presented was made by
        running the Unix utility sed(1) on the normalized increments
        time series data to remove the negative signs. This is an
        absolute value procedure.  The resulting time series contains
        the absolute value of the instantaneous change in the rate of
        revenue returns, divided by the magnitude of the instantaneous
        rate of revenue returns\footnote{The absolute value of the
        normalized increments, when averaged, is related to the root
        mean square of the increments by a constant. If the normalized
        increments are a fixed increment, the constant is unity. If
        the normalized increments have a Gaussian distribution, the
        constant is $\approx 0.8$ depending on the accuracy of of
        ``fit'' to a Gaussian distribution.}.

        \subidx{histogram}{normalized}
        \subidx{normalized}{histogram}
        \subidx{programs}{tsnormal}
        \subidx{tsnormal}{program}
        \subidx{mean}{standard deviation}
        \subidx{standard deviation}{mean}
        \idx{root mean square}
        \idx{least squares approximation}
        \subidx{\market}{analysis of increments}
        Figure~\ref{\SETLABEL:NH} is the normalized histogram of the
        normalized increments of the time series data shown in
        Figure~\ref{\SETLABEL:TF}. The abscissa is 3 $\sigma$ limits,
        and the area under the two curves is identical. The data for
        this figure was produced by the program {\it tsnormal}\/,
        which is described briefly in Appendix~\ref{programs}.

        \begin{figure}[ht]
            \begin{center}
                \begin{minipage}[t]{0.45\textwidth}
                    \epsfxsize=1.0\linewidth
                    \epsffile{\directory/data.tsfraction.abs.eps}
                    \caption[{\market}, absolute value of the
                        normalized increments]{{\market}, absolute
                        value of the normalized increments of the time
                        series data presented in
                        Figure~\ref{\SETLABEL:TF}.  The mean is
                        {\datafractionabsmean} with a standard
                        deviation of {\datafractionabsstddev}. The
                        formula for the least squares approximation is
                        ${\datafractionabsconstant} +
                        {\datafractionabsslope}t$, and the root mean
                        square value, from Figure~\ref{\SETLABEL:TF},
                        is {\datafractionrms}.  The graph, labeled
                        ``data\-.tsfraction\-.tsrms,'' is the running
                        root mean square, and
                        ``data\-.tsfraction\-.tsavg'' is the running
                        average of the normalized increments presented
                        in Figure~\ref{\SETLABEL:TF}, superimposed
                        here for convenience. This graph is the
                        absolute value of the fraction of change in
                        the time series, as a function of time.}
                    \label{\SETLABEL:TFA}
                    \label{\SETLABELQ:TFA}
                \end{minipage}
                \hfill
                \begin{minipage}[t]{0.45\textwidth}
                    \epsfxsize=1.0\linewidth
                    \epsffile{\directory/data.tsfraction.tsnormal-s30.eps}
                    \caption[{\market}, normalized histogram of the
                        normalized increments]{{\market}, normalized
                        histogram of the normalized increments of the
                        time series data shown in
                        Figure~\ref{\SETLABEL:TF}.  The data has a
                        mean of {\datafractionmean}, with a standard
                        deviation of {\datafractionstddev}.  The area
                        under the two curves is identical. The
                        $\chi^2$ value of the observed and expected
                        values of the two curves is {\chisquared},
                        with a critical value of {\critical}.}
                    \label{\SETLABEL:NH}
                \end{minipage}
            \end{center}
        \end{figure}

        \subidx{programs}{tsXsquared}
        \subidx{tsXsquared}{program}
        \subidx{\market}{chi-squared values of increments}
        The program {\it tsXsquared}\/, which is briefly described in
        appendix~\ref{programs}, was used to derive the $\chi^2$
        statistics for the data presented in
        Figure~\ref{\SETLABEL:NH}.

        \subidx{programs}{tsstatest}
        \subidx{tsstatest}{program}
        \subidx{\market}{statistical estimates}

        Figure~\ref{\SETLABEL:SE} is the statistical estimate for the
        data presented in Figure~\ref{\SETLABEL:TF}, as derived by the
        program {\it tsstatest}\/, which is briefly described in
        appendix~\ref{programs}.

        \begin{figure}[ht]
            \begin{center}
                \begin{minipage}[t]{\textwidth}
                    \center{\fbox{\parbox{0.9\textwidth}{\XXX{\directory/data.tsstatest-f0.1-c0.9-i.tex}}}}
                    \caption[{\market}, statistical estimates of the
                        normalized increments]{{\market}, statistical
                        estimates of the normalized increments of the
                        time series shown in Figure~\ref{\SETLABEL:TF}.
                        The table was produced with the {\it
                        tsstatest}\/ program, and illustrates the
                        size of the data set required for a confidence
                        level of 90\%, with an error estimate of $\pm$
                        10\%, or alternately, the error estimate on
                        the time series shown in Figure~\ref{\SETLABEL:TF}.}
                    \label{\SETLABEL:SE}
                \end{minipage}
            \end{center}
        \end{figure}

        Note that the data set size estimations, as produced by the
        {\it tsstatest}\/ program, are probably very conservative,
        depending on the magnitude of the Shannon probability, $P =
        \shannonlogreturns$, as derived in
        Section~\ref{\SETLABEL:SP}. See Chapter~\ref{general},
        Section~\ref{serdss} for possible alternative methodologies
        for addressing the analysis of fractal time series with
        limited data set sizes. Depending on the magnitude of the
        Shannon probability, $P$, these estimates can be several
        orders of magnitude too high.

        \subidx{derivative of increments}{normalized}
        \subidx{normalized}{derivative of increments}
        \subidx{programs}{tsderivative}
        \subidx{tsderivative}{program}
        Figure~\ref{\SETLABEL:TF1} is the normalized histogram of the
        first derivative of the normalized increments of the time
        series data shown in Figure~\ref{\SETLABEL:TF}. In principle,
        if the distribution of the normalized increments presented in
        Figure~\ref{\SETLABEL:NH} is Gaussian in nature, this
        distribution would be similar to ``white noise,'' as presented
        in appendix~\ref{programs}, Figure~\ref{whiteexample}. The
        data was generated by the {\it tsderivative}\/ program, which
        is briefly described in
        appendix~\ref{programs}. Figure~\ref{\SETLABEL:TF2} is the
        normalized histogram of the second derivative of the
        normalized increments of the time series data shown in
        Figure~\ref{\SETLABEL:TF}. In principle, if the distribution
        of the normalized increments presented in
        Figure~\ref{\SETLABEL:NH} is an integrated Gaussian
        distribution in nature, this distribution would be similar to
        ``white noise,'' as presented in appendix~\ref{programs},
        Figure~\ref{whiteexample}.

        \begin{figure}[ht]
            \begin{center}
                \begin{minipage}[t]{0.45\textwidth}
                    \epsfxsize=1.0\linewidth
                    \epsffile{\directory/data.tsfraction.tsderivative.tsnormal-s30.eps}
                    \caption[{\market}, histogram of the first
                        derivative of the increments]{{\market},
                        normalized histogram of the first derivative
                        of the normalized increments of the time
                        series data shown in
                        Figure~\ref{\SETLABEL:TF}.}
                    \label{\SETLABEL:TF1}
                \end{minipage}
                \hfill
                \begin{minipage}[t]{0.45\textwidth}
                    \epsfxsize=1.0\linewidth
                    \epsffile{\directory/data.tsfraction.2tsderivative.tsnormal-s30.eps}
                    \caption[{\market}, histogram of the second
                        derivative of the increments]{{\market},
                        normalized histogram of second derivative of
                        the the normalized increments of the time
                        series data shown in
                        Figure~\ref{\SETLABEL:TF}.}
                    \label{\SETLABEL:TF2}
                \end{minipage}
            \end{center}
        \end{figure}

        \subidx{fractal}{range}
        \subidx{fractal}{R/S analysis}
        \subidx{\market}{rate of revenue returns, range}
        \subidx{\market}{deterministic mechanism}
        \subidx{deterministic}{mechanism}
        \subidx{mechanism}{deterministic}
        Figure~\ref{\SETLABEL:TR} is the range of values of the time
        series shown in Figure~\ref{\SETLABEL:TS}. The horizontal axis
        is time into the future. In principle, if the time series was
        characterized as fractional Brownian motion the graph in
        Figure~\ref{\SETLABEL:TR} would be a square root
        function\footnote{Note that the ``roughness,'' or ``sawtooth''
        characteristics of the graph in Figure~\ref{\SETLABEL:TR} are
        a computational artifact---caused by not using the -m option
        to the program {\it tshurst}\/, which is computationally
        inefficient.}. Figure~\ref{\SETLABEL:TD} is the deterministic
        map of the normalized increments of the time series data shown
        in Figure~\ref{\SETLABEL:TF}. The deterministic map is useful
        for determining if a time series was created by a
        deterministic mechanism. This, essentially, maps each element
        in the time series with the previous element in the time
        series.  See,~\cite[pp. 745]{Peitgen}.

        \begin{figure}[ht]
            \begin{center}
                \begin{minipage}[t]{0.45\textwidth}
                    \epsfxsize=1.0\linewidth
                    \epsffile{\directory/data.tshurst-f.eps}
                    \caption[{\market}, range]{{\market}, range of the
                        time series data shown in
                        Figure~\ref{\SETLABEL:TS}.}
                    \label{\SETLABEL:TR}
                \end{minipage}
                \hfill
                \begin{minipage}[t]{0.45\textwidth}
                    \epsfxsize=1.0\linewidth
                    \epsffile{\directory/data.tsfraction.tsdeterministic.eps}
                    \caption[{\market}, deterministic map]{{\market},
                        deterministic map of the normalized increments
                        of the time series data shown in
                        Figure~\ref{\SETLABEL:TF}.}
                    \label{\SETLABEL:TD}
                \end{minipage}
            \end{center}
        \end{figure}

% Local Variables:
% TeX-parse-self: t
% TeX-auto-save: t
% TeX-master: "fractal.tex"
% End:


            Figure~\ref{\SETLABEL:NH} would seem to indicate that the
            time series data for the {\market} represents a cumulative
            sum/integration of a random process that has a Gaussian
            distribution, (ie., satisfies the Gaussian increments
            property of fractional Brownian
            motion~\cite[pp. 250]{Crownover},) tending to justify the
            assumption that the time series data represents fractional
            Brownian motion.

        %
% -----------------------------------------------------------------------------
%
% A license is hereby granted to reproduce this software source code and
% to create executable versions from this source code for personal,
% non-commercial use.  The copyright notice included with the software
% must be maintained in all copies produced.
%
% THIS PROGRAM IS PROVIDED "AS IS". THE AUTHOR PROVIDES NO WARRANTIES
% WHATSOEVER, EXPRESSED OR IMPLIED, INCLUDING WARRANTIES OF
% MERCHANTABILITY, TITLE, OR FITNESS FOR ANY PARTICULAR PURPOSE.  THE
% AUTHOR DOES NOT WARRANT THAT USE OF THIS PROGRAM DOES NOT INFRINGE THE
% INTELLECTUAL PROPERTY RIGHTS OF ANY THIRD PARTY IN ANY COUNTRY.
%
% Copyright (c) 1994-2006, John Conover, All Rights Reserved.
%
% Comments and/or bug reports should be addressed to:
%
%     john@email.johncon.com (John Conover)
%
% -----------------------------------------------------------------------------
%
% Revision: \RCSRevision \\
% Revision Time: \RCSTime UMT \\
% Revision Date: \RCSDate \\
% Revision Id: \RCSId \\
% Revision File: \RCSLog \\
\RCS $Revision: 0.0 $
\RCS $Date: 2006/01/20 04:38:13 $
\RCS $Id: instant.tex,v 0.0 2006/01/20 04:38:13 john Exp $
% $Log: instant.tex,v $
% Revision 0.0  2006/01/20 04:38:13  john
% Initial version
%
%
    \subsection{Instantaneous Analysis of Normalized Increments}
        \label{\SETLABEL:IA}

        \subidx{\market}{instantaneous analysis of normalized increments}
        \idx{average of normalized increments}
        \idx{root mean square of normalized increments}
        \subidx{Shannon probability}{instantaneous computation of}
        \subidx{average of normalized increments}{instantaneous computation of}
        \subidx{root mean square of normalized increments}{instantaneous computation of}
        \subidx{instantaneous computation}{Shannon probability}
        \subidx{instantaneous computation}{average of normalized increments}
        \subidx{instantaneous computation}{root mean square of normalized increments}
        \idx{time series}
        \subidx{time series}{instantaneous analysis}
        \subidx{instantaneous analysis}{time series}
        \subidx{time series}{increments}
        \subidx{time series}{analysis}
        \subidx{Shannon}{probability}
        \subidx{probability}{Shannon}
        \subidx{normalized}{increments}
        \subidx{increments}{normalized}

        The program {\it tsinstant}\/, which is briefly described in
        Appendix~\ref{programs}, is for finding the instantaneous
        fraction of change in a time series. The value of a sample in
        the time series is subtracted from the previous sample in the
        time series, and divided by the value of the previous sample.
        As explained in Chapter~\ref{general},
        Sections~\ref{derivation},~\ref{GA},~\ref{abmfi},~\ref{aftsma}
        and,~\ref{ompl} for Brownian motion, random walk fractals, the
        absolute value of the instantaneous fraction of change is also
        the root mean square of the instantaneous fraction of
        change\footnote{The absolute value of the normalized
        increments, when averaged, is related to the root mean square
        of the increments by a constant. If the normalized increments
        are a fixed increment, the constant is unity. If the
        normalized increments have a Gaussian distribution, the
        constant is $\approx 0.8$ depending on the accuracy of of
        ``fit'' to a Gaussian distribution.}. Squaring this value is
        the average of the instantaneous fraction of change, and
        adding unity to the absolute value of the instantaneous
        fraction of change, and dividing by two, is the Shannon
        probability of the instantaneous fraction of change.

        Figure~\ref{\SETLABEL:IA1} is the instantaneous value of the
        root mean square of the normalized increments for the
        {\market}, and Figure~\ref{\SETLABEL:IA2} is the instantaneous
        Shannon probability for the normalized increments.

        \begin{figure}[ht]
            \begin{center}
                \begin{minipage}[t]{0.45\textwidth}
                    \epsfxsize=1.0\linewidth
                    \epsffile{\directory/data.tsinstant-r.eps}
                    \caption[{\market}, instantaneous value of
                        rms.]{{\market}, instantaneous value of the
                        root mean square of the normalized increments,
                        provided by running the program {\it
                        tsinstant}\/ with the -r option on the data
                        presented in Figure~\ref{\SETLABEL:TS}.}
                    \label{\SETLABEL:IA1}
                    \label{\SETLABELQ:IA1}
                \end{minipage}
                \hfill
                \begin{minipage}[t]{0.45\textwidth}
                    \epsfxsize=1.0\linewidth
                    \epsffile{\directory/data.tsinstant-s.eps}
                    \caption[{\market}, instantaneous value of
                        Shannon probability.]{{\market}, instantaneous
                        value of the Shannon probability of the
                        normalized increments, provided by running the
                        program {\it tsinstant}\/ with the -s option
                        on the data presented in
                        Figure~\ref{\SETLABEL:TS}.}
                    \label{\SETLABEL:IA2}
                    \label{\SETLABELQ:IA2}
                \end{minipage}
            \end{center}
        \end{figure}

% Local Variables:
% TeX-parse-self: t
% TeX-auto-save: t
% TeX-master: "fractal.tex"
% End:


        %
% -----------------------------------------------------------------------------
%
% A license is hereby granted to reproduce this software source code and
% to create executable versions from this source code for personal,
% non-commercial use.  The copyright notice included with the software
% must be maintained in all copies produced.
%
% THIS PROGRAM IS PROVIDED "AS IS". THE AUTHOR PROVIDES NO WARRANTIES
% WHATSOEVER, EXPRESSED OR IMPLIED, INCLUDING WARRANTIES OF
% MERCHANTABILITY, TITLE, OR FITNESS FOR ANY PARTICULAR PURPOSE.  THE
% AUTHOR DOES NOT WARRANT THAT USE OF THIS PROGRAM DOES NOT INFRINGE THE
% INTELLECTUAL PROPERTY RIGHTS OF ANY THIRD PARTY IN ANY COUNTRY.
%
% Copyright (c) 1994-2006, John Conover, All Rights Reserved.
%
% Comments and/or bug reports should be addressed to:
%
%     john@email.johncon.com (John Conover)
%
% -----------------------------------------------------------------------------
%
% Revision: \RCSRevision \\
% Revision Time: \RCSTime UMT \\
% Revision Date: \RCSDate \\
% Revision Id: \RCSId \\
% Revision File: \RCSLog \\
\RCS $Revision: 0.0 $
\RCS $Date: 2006/01/20 04:38:13 $
\RCS $Id: logistic.tex,v 0.0 2006/01/20 04:38:13 john Exp $
% $Log: logistic.tex,v $
% Revision 0.0  2006/01/20 04:38:13  john
% Initial version
%
%
    \subsection{Logistic Analysis}
        \label{\SETLABEL:LA}

        \subidx{\market}{Logistic function analysis}
        \subidx{time series}{logistic function}
        \subidx{logistic function}{time series}
        \subidx{time series}{increments}
        \subidx{time series}{analysis}
        \subidx{cumulative sum}{analysis}
        \subidx{analysis}{cumulative sum}
        \subidx{analysis}{random process}
        \subidx{random process}{analysis}
        The data in this section is presented in tabular form in
        Section~\ref{\SETLABELREF:LAA}.  Figure~\ref{\SETLABEL:LA1} is
        a graph of the logistic function estimates of the time series
        data for the {\market}. The reader is cautioned that these
        graphs are constructed using the method suggested in
        Chapter~\ref{general}, Section~\ref{nlextend} and enormous
        precision is required for adequate prediction of the logistic
        function,~\cite{Modis}. Particularly, the non-linear term will
        usually require intervention to produce a practical fit to the
        data. In addition, there are numerical stability issues with
        logistic function methodologies\footnote{For example, in
        Figures~\ref{\SETLABEL:LA1} and~\ref{\SETLABEL:LA2}, if the
        non-linear term, $b$, was greater than zero, it was set to
        zero to produce the graphs. See Section~\ref{\SETLABELREF:LAA}
        for the actual derived values. In other cases, the magnitude
        of $b$ was too large, resulting in a graph that was decreasing
        as a function of time}.  The methodology should be regarded as
        ``fragile.'' It is included for completeness.

        \idx{least squares approximation}
        Figure~\ref{\SETLABEL:LA1} is a graph of the logistic function
        for the time series data presented in
        Figure~\ref{\SETLABEL:TS}. The data presented was made by
        running the program {\it tsdlogistic}\/, which is described
        briefly in Appendix~\ref{programs}, on the parameters
        extracted from the time series data as suggested in
        Figure~\ref{\SETLABEL:TF}. The program {\it tslsq}\/ was used
        to derive the constant and the slope of the normalized
        increments of the data presented in Figure~\ref{\SETLABEL:TF}.
        Figure~\ref{\SETLABEL:LA2} is the same graph, but with the
        time scale expanded by a factor of two.

        \begin{figure}[ht]
            \begin{center}
                \begin{minipage}[t]{0.45\textwidth}
                    \epsfxsize=1.0\linewidth
                    \epsffile{\directory/data.tsfraction.tslsq-p.tsdlogistic.eps}
                    \caption[{\market}, logistic function
                        estimates.]{{\market}, logistic function
                        estimates, provided by running the {\it
                        tslsq}\/ program on the normalized increments
                        presented in Figure~\ref{\SETLABEL:TF} with
                        the -p option. These parameters were used as
                        arguments to the {\it tsdlogistic}\/ program.}
                    \label{\SETLABEL:LA1}
                    \label{\SETLABELQ:LA1}
                \end{minipage}
                \hfill
                \begin{minipage}[t]{0.45\textwidth}
                    \epsfxsize=1.0\linewidth
                    \epsffile{\directory/data.tsfraction.tslsq-p.tsdlogistic2.eps}
                    \caption[{\market}, logistic function
                        estimates.]{{\market}, logistic function
                        estimates of Figure~\ref{\SETLABEL:LA1} with
                        the time scale expanded by a factor of two.}
                    \label{\SETLABEL:LA2}
                    \label{\SETLABELQ:LA2}
                \end{minipage}
            \end{center}
        \end{figure}

% Local Variables:
% TeX-parse-self: t
% TeX-auto-save: t
% TeX-master: "fractal.tex"
% End:


        %
% -----------------------------------------------------------------------------
%
% A license is hereby granted to reproduce this software source code and
% to create executable versions from this source code for personal,
% non-commercial use.  The copyright notice included with the software
% must be maintained in all copies produced.
%
% THIS PROGRAM IS PROVIDED "AS IS". THE AUTHOR PROVIDES NO WARRANTIES
% WHATSOEVER, EXPRESSED OR IMPLIED, INCLUDING WARRANTIES OF
% MERCHANTABILITY, TITLE, OR FITNESS FOR ANY PARTICULAR PURPOSE.  THE
% AUTHOR DOES NOT WARRANT THAT USE OF THIS PROGRAM DOES NOT INFRINGE THE
% INTELLECTUAL PROPERTY RIGHTS OF ANY THIRD PARTY IN ANY COUNTRY.
%
% Copyright (c) 1994-2006, John Conover, All Rights Reserved.
%
% Comments and/or bug reports should be addressed to:
%
%     john@email.johncon.com (John Conover)
%
% -----------------------------------------------------------------------------
%
% Revision: \RCSRevision \\
% Revision Time: \RCSTime UMT \\
% Revision Date: \RCSDate \\
% Revision Id: \RCSId \\
% Revision File: \RCSLog \\
\RCS $Revision: 0.0 $
\RCS $Date: 2006/01/20 04:38:13 $
\RCS $Id: hurst.tex,v 0.0 2006/01/20 04:38:13 john Exp $
% $Log: hurst.tex,v $
% Revision 0.0  2006/01/20 04:38:13  john
% Initial version
%
%
    \subsection{Hurst Coefficient Analysis}
        \label{\SETLABEL:H}

        \subidx{\market}{Hurst coefficient analysis}
        \subidx{Hurst coefficient}{analysis}
        \subidx{increments}{normalized}
        \subidx{normalized}{increments}
        \subidx{programs}{tshurst}
        \subidx{tshurst}{program}
        The data in this section is presented in tabular form in
        Section~\ref{\SETLABELREF:HCHP}. Figure~\ref{\SETLABEL:HC} is
        a graph of the Hurst coefficient data time series data shown
        in Figure~\ref{\SETLABEL:TS}. The slope of the graph is the
        Hurst coefficient.  The data for this figure was produced by
        the program {\it tshurst}\/, which is described briefly in
        Appendix~\ref{programs}.

        \subidx{\market}{H parameter analysis}
        \subidx{H parameter}{analysis}
        \subidx{programs}{tshcalc}
        \subidx{tshcalc}{program}
        Figure~\ref{\SETLABEL:HP} is a graph of the H parameter data
        for the normalized increments of the time series data shown in
        Figure~\ref{\SETLABEL:TF}. The data for this figure was
        produced by the program {\it tshcalc}\/, which is described
        briefly in Appendix~\ref{programs}.

        \begin{figure}[ht]
            \begin{center}
                \begin{minipage}[t]{0.45\textwidth}
                    \epsfxsize=1.0\linewidth
                    \epsffile{\directory/data.tshurst.eps}
                    \caption[{\market}, Hurst coefficient data]{{\market},
                        Hurst coefficient data for the normalized
                        increments of the time series data shown in
                        Figure~\ref{\SETLABEL:TF}.  The slope of the graph
                        is the Hurst coefficient.}
                    \label{\SETLABEL:HC}
                \end{minipage}
                \hfill
                \begin{minipage}[t]{0.45\textwidth}
                    \epsfxsize=1.0\linewidth
                    \epsffile{\directory/data.tshcalc.eps}
                    \caption[{\market}, H parameter data]{{\market}, H
                        parameter data for the normalized increments of
                        the time series data shown in
                        Figure~\ref{\SETLABEL:TF} The slope of the graph
                        is the H parameter.}
                    \label{\SETLABEL:HP}
                \end{minipage}
            \end{center}
        \end{figure}

        \subidx{revenue}{See, rate of revenue returns}
        \subidx{returns}{See, rate of revenue returns}
        \subidx{\market}{revenues}
        \subidx{Hurst coefficient}{analysis}
        \subidx{\market}{Hurst coefficient analysis}
        \subidx{\market}{rate of change}
        \subidx{\market}{windows of opportunity}
        \subidx{rate of revenue returns}{forecast}
        \subidx{forecast}{rate of revenue returns}
        \idx{windows of opportunity}
        \subidx{programs}{tslsq}
        \subidx{tslsq}{program}

        The approximately linear slope of the graph in
        Figure~\ref{\SETLABEL:HC} implies that the variance of the
        rate of revenue returns, (per {\timescale},) in the {\market},
        $V(t_2 - t_1)$, over a period of time is proportional to the
        period of time raised to twice the Hurst
        coefficient~\cite[pp. 180]{Feder},~\cite[pp. 246]{Crownover}.
        This seems to be a quantitative statement concerning how fast,
        and to what degree, the rate of revenue returns' state of
        affairs can change over a period of time.  An additional
        implication, for Hurst coefficients sufficiently close to 0.5,
        is that the probability of the state of affairs repeating
        sometime in the future goes down with increasing
        time\footnote{It can be shown that the number of expected
        market ``high'' and ``low'' transitions, $N$, scales with the
        square root of time, or $N \propto \sqrt {t}$, meaning that
        the cumulative distribution of the probability, $P$, of the
        duration of a market's ``high'' or ``low'' exceeding a given
        time interval, $t$, is proportional to the reciprocal of the
        square root of the time interval, $P \propto 1 / \sqrt {t}$,
        (or, conversely, that the probability of the duration of a
        market's ``high'' or ``low'' exceeding a given time interval
        is proportional to the reciprocal of the time interval raised
        to the power $3 / 2$, ie., $P \propto 1 / t^{3 /
        2}$,~\cite[pp. 153]{Schroeder}. What this means is that a
        histogram of the ``zero free'' run-lengths of a market being
        ``high'' or ``low,'' over a long time, would have a $1 / t^{3
        / 2}$ characteristic.)}, $t$, $p(t) = erf (1/\sqrt{2t})$ which
        is approximately $1/\sqrt{t}$ for $t \gg
        1$~\cite[pp. 160]{Schroeder}. Figures~\ref{\SETLABEL:FN},
        and,~\ref{\SETLABEL:FF} compare methods of approximation of
        the ``forecastability'' of the rate of revenue returns in the
        {\market} for the near term and far term,
        respectively~\cite[pp. 83-84]{Peters:CAOITCM}\footnote{The
        author is not comfortable with Peters' interpretation. For
        example, if the algorithm explained
        in~\cite[pp. 82]{Peters:CAOITCM} is used on ``white noise''
        which, by definition, never has any correlations, the short
        term Hurst coefficient, and thus the ``forecastability,'' is
        still near unity---a bit of an enigma. This can be verified
        with the {\it tswhite}\/ and {\it tshurst}\/ programs, which
        are briefly described in Appendix~\ref{programs}.}.  This
        seems to be a quantitative statement concerning ``windows of
        opportunity'' in the rate of revenue returns, (per
        {\timescale}.)  The program {\it tslsq}\/ was used on the
        Hurst coefficient data, presented in
        Figure~\ref{\SETLABEL:HC}, to provide a least squares
        approximation to the Hurst coefficient. The superimposed least
        squares approximation with on original Hurst coefficient data
        is presented.  The time series data has a Hurst coefficient of
        {\thurstlow}, so that:

        \subidx{\market}{Hurst coefficient analysis}
        \begin{eqnarray}
            V\left(t_2 - t_1\right) & \propto & \left(t_2 - t_1\right)^{2 \cdot H}\\
            V\left(t_2 - t_1\right) & \propto & \left(t_2 - t_1\right)^{2 \cdot {\thurstlow}}\\
                                    & \propto & \left(t_2 - t_1\right)^{\thurstlowtwo}
            \label{\SETLABEL:V}
        \end{eqnarray}

        \subidx{fractional}{Brownian motion}
        \subidx{Brownian motion}{fractional}
        \idx{fractal}
        \noindent where $V(t_2 - t_1)$ is the variance of the
        increments of the rate of revenue returns, (per {\timescale},)
        over the time interval $t_2 -
        t_1$,~\cite[pp. 177]{Feder},~\cite[pp. 494]{Peitgen}. If $H >
        \frac{1}{2}$, then the time series is termed as being
        characterized by ``fractional Brownian
        motion~\cite[pp. 170]{Feder}.''

        \subidx{rate of revenue returns}{predictability}
        \subidx{rate of revenue returns}{forecastability}
        \subidx{rate of revenue returns}{consistency}
        \subidx{predictability}{rate of revenue returns}
        \subidx{forecastability}{rate of revenue returns}
        \subidx{consistency}{rate of revenue returns}
        \subidx{\market}{rate of revenue returns, predictability}
        \subidx{\market}{rate of revenue returns, forecastability}
        \subidx{\market}{rate of revenue returns, consistency}
        \subidx{Hurst coefficient}{analysis}
        \subidx{\market}{Hurst coefficient analysis}
        \subidx{\market}{rate of change}

        In some sense, the Hurst coefficient is a quantitative
        expression of the ``forecastability'' of the future based on
        the past\footnote{Actually, in general, when summing fractal
        entities, the method used should be a root mean square
        process, dependent on the Hurst Coefficient, $H$, where
        $P_{total}^H = P_1^H + P_2^H + \cdots$, where $P_n$ is the
        fractal entities. For a Brownian motion, or random walk type
        of fractal the Hurst Coefficient is a function of time into
        the future. For the ``near term,'' the Hurst coefficient is
        very near unity, meaning the summation process is linear. For
        the ``long term,'' $H \approx 0.5$, or a standard root mean
        square summation process should be used. If $H$ is $0.5$ then
        the market is termed a Brownian motion, or random walk
        process. If it is larger than 0.5, it is termed fractional
        Brownian motion process. For a random walk process, ``near
        term'' and ``far term'' are quantitatively differentiated on
        the Hurst Coefficient graph where $1 - \ln (t) = 0.5 \cdot \ln
        (t)$, or when $\ln (t) = 2$, or $t = 7.389\ldots$ See
        Section~\ref{\SETLABEL:FS} for the particulars on using Hurst
        Coefficient to sum fractal process' for the {\market}. See
        also~\cite[pp. 67, 83-84]{Peters:CAOITCM} and~\cite[pp. 129,
        159]{Schroeder} for particulars on the implications of the
        Hurst Coefficient and root mean square summation issues.}.  A
        Hurst coefficient of {\thurstlow}, (for the near future, and
        {\thurstall} for the distant future.) implies that the
        likelihood of the rate of revenue returns, (per {\timescale},)
        for any two consecutive {\timescale}s being the same is
        {\thurstlowhundred}\%~\cite[pp. 66]{Peters:CAOITCM} for the
        near future, and {\thurstall} for the distant
        future. Likewise, there is a {\thurstlowhundred}\% chance of
        the rate of revenue returns, (per {\timescale},) movements
        being the same in consecutive time periods---ie., if, in a
        given {\timescale}, the rate of revenue returns, (per
        {\timescale},) is increasing, there is a {\thurstlowhundred}\%
        that the rate of revenue returns, (per {\timescale},) will
        increase in the following period, also. In some sense, this is
        a quantitative statement on how ``predictable,'' or
        ``forecastable'' the rate of revenue returns, (per
        {\timescale},) for the {\market} are over time, since the
        probability of having $n$ many consecutive {\timescale}s of
        the same agenda is $H^n$ where $H$ is the Hurst coefficient,
        or, letting the short term probability of having $n$ many
        {\timescale}s of the same market agenda, $p_a$, is:

        \begin{eqnarray}
            p_a\left(n\right) & = & H^{n}\\
                              & = & {\thurstlow}^{n}
            \label{\SETLABEL:MA}
        \end{eqnarray}

        \subidx{rate of revenue returns}{predictability}
        \subidx{rate of revenue returns}{forecastability}
        \subidx{rate of revenue returns}{consistency}
        \subidx{predictability}{rate of revenue returns}
        \subidx{forecastability}{rate of revenue returns}
        \subidx{consistency}{rate of revenue returns}
        As an interesting interpretation of the normalized increments
        of the time series data presented in
        Figure~\ref{\SETLABEL:TF}, if the vertical axis is multiplied
        by 100, to convert to percent, then the graph represents the
        error, in percent, that would be made by forecasting, month by
        month, that the next {\timescale}'s rate of revenue returns
        would be the same as the current {\timescale}'s revenue
        rate. Interestingly, it is $\datafractionmean \cdot 100$
        percent, on the average, with a standard deviation of
        $\datafractionstddev \cdot 100$ percent, and a root mean
        square error value of $\datafractionrms \cdot 100$
        percent---small values for such a simple forecasting
        mechanism.

        \subidx{\market}{rate of revenue returns, range}
        \subidx{Hurst coefficient}{analysis}
        \subidx{\market}{Hurst coefficient analysis}
        \subidx{\market}{rate of change}

        This is, essentially, a statement of the range of values, in
        the increments of the rate of revenue returns, (per
        {\timescale},) that is to be expected over the time interval,
        $t_2 - t_1$,
        $R_v$,~\cite[pp. 178]{Feder},~\cite[pp. 172]{Cambel}:

        \begin{eqnarray}
            R_v\left(t_2 - t_1\right) & \propto & \left(t_2 - t_1\right)^{H}\\
                                      & \propto & \left(t_2 - t_1\right)^{\thurstlow}
            \label{\SETLABEL:R}
        \end{eqnarray}

        \subidx{\market}{rate of revenue returns, range}
        \subidx{Hurst coefficient}{analysis}
        \subidx{\market}{Hurst coefficient analysis}
        \subidx{\market}{rate of change}
        \subidx{Markov}{statistics}
        \subidx{statistics}{Markov}
        \noindent where $R$ is the range of values in the increments
        of the rate of revenue returns, (per {\timescale}.) A Hurst
        coefficient, $H$, that is much larger than $\frac{1}{2}$, (but
        less than 1,) implies a strongly non-Gaussian distribution in
        the increments of the rate of revenue returns, (per
        {\timescale},)~\cite[pp. 152, 194]{Feder}, and a Hurst
        coefficient near $\frac{1}{2}$ implies that the increments of
        the rate of revenue returns, (per {\timescale}) is
        characteristic of an independent
        process~\cite[pp. 195]{Feder}. Extreme caution should be
        exercised in using Markov statistics in any analysis where the
        Hurst coefficient is not
        $\frac{1}{2}$,~\cite[pp. 124]{Crownover},~\cite[pp. 106]{Peters:CAOITCM}.


        As a useful approximation, if $H$, is approximately
        $\frac{1}{2}$, Equation~\ref{\SETLABEL:R} reduces
        to,~\cite[pp. 129]{Schroeder}:

        \begin{eqnarray}
            R\left(t_2 - t_1\right) & \propto & (t_2 - t_1)^{\frac{1}{2}}\\
                                    & \propto & \sqrt{\left(t_2 - t_1\right)}
        \end{eqnarray}

        \subidx{\market}{rate of revenue returns, range}
        \subidx{\market}{rate of revenue returns, increase and decrease}
        \subidx{Hurst coefficient}{analysis}
        \subidx{\market}{Hurst coefficient analysis}
        \subidx{\market}{rate of change}
        \subidx{Markov}{statistics}
        \subidx{statistics}{Markov}

        In the case where the Hurst coefficient, $H$, is
        $\frac{1}{2}$, the range of values in the increments of the
        rate of revenue returns, (per {\timescale},) divided by the
        standard deviation of these values, $S$, can be anticipated to
        increase over time according to the following
        relation,~\cite[pp. 154]{Feder},~\cite[pp. 129]{Schroeder}:

        \begin{equation}
            \frac{R\left(t_2 - t_1\right)}{S} \propto \left(t_2 - t_1\right)^{\frac{1}{2}}
        \end{equation}

        \subidx{\market}{rate of revenue returns, range}
        \subidx{\market}{rate of revenue returns, increase and decrease}
        \subidx{Hurst coefficient}{analysis}
        \subidx{\market}{Hurst coefficient analysis}
        \subidx{\market}{rate of change}
        \noindent which is a useful conceptual approximation, since it
        involves only the square root function---if the range and the
        standard deviation of the increments of the rate of revenue
        returns, (per {\timescale},) are known, (and $H \approx
        \frac{1}{2}$,) then the expected change in $\frac{R}{S}$, will
        increase with the square root of time\footnote{To be precise,
        it is actually asymptotically proportional to
        $\tau^{\frac{1}{2}}$}.

        Another useful approximation when rescaling processes that are
        characterize by Brownian motion, (ie., when $H \approx
        \frac{1}{2}$,) is that:

        \begin{eqnarray}
            X\left(t\right) & \propto & \frac{X\left(rt\right)}{r^{H}}\\
                            & \propto & \frac{X\left(rt\right)}{r^{\thurstlow}}
        \end{eqnarray}

        \idx{Brownian motion}
        \idx{fractal}
        Where $X(t)$ is the process characterized by Brownian motion,
        and $r$ is a scaling factor,~\cite[pp. 494]{Peitgen}.

        \subidx{programs}{tslsq}
        \subidx{tslsq}{program}
        The program {\it tslsq}\/ was used on the H parameter data,
        presented in Figure~\ref{\SETLABEL:HP}, to provide a least
        squares approximation to the H parameter for the
        {\market}. The superimposed least squares approximation on the
        original H parameter data is presented.  By contrast, the H
        parameter, as derived by the methodology outlined
        in~\cite[pp. 249]{Crownover}, is {\thcalclow} for the near
        future, and {\thcalcall} for the distant future.

        \subidx{\market}{Hurst coefficient analysis}
        \subidx{Hurst coefficient}{analysis}
        \subidx{increments}{normalized}
        \subidx{normalized}{increments}
        \subidx{programs}{tshurst}
        \subidx{tshurst}{program}
        \subidx{\market}{H parameter analysis}
        \subidx{H parameter}{analysis}
        \subidx{programs}{tshcalc}
        \subidx{tshcalc}{program}
        Figures~\ref{\SETLABEL:HC} and~\ref{\SETLABEL:HP} represent
        Hurst coefficient and H parameter data that are derived from
        the normalized increments, shown in
        Figure~\ref{\SETLABEL:TF}. In this case, the data is
        considered a normalized derivative of the time series data
        presented in Figure~\ref{\SETLABEL:TF}, instead of a
        cumulative sum.  The program, {\it tshurst}\/, is described
        briefly in appendix~\ref{programs}, and the data for
        figures~\ref{\SETLABEL:THC} and~\ref{\SETLABEL:THP} was made
        using the -d option.

        \begin{figure}[ht]
            \begin{center}
                \begin{minipage}[t]{0.45\textwidth}
                    \epsfxsize=1.0\linewidth
                    \epsffile{\directory/data.tsfraction.tshurst-d.eps}
                    \caption[{\market}, traditional Hurst coefficient
                        data]{{\market}, traditional Hurst coefficient
                        data for the time series data shown in
                        Figure~\ref{\SETLABEL:TS}.  The slope of the
                        graph is the Hurst coefficient, and is
                        {\hurstlow} for the near term, and
                        {\hurstall} for the far term.}
                    \label{\SETLABEL:THC}
                \end{minipage}
                \hfill
                \begin{minipage}[t]{0.45\textwidth}
                    \epsfxsize=1.0\linewidth
                    \epsffile{\directory/data.tsfraction.tshcalc-d.eps}
                    \caption[{\market}, traditional H parameter
                        data]{{\market}, traditional H parameter data
                        for the time series data shown in
                        Figure~\ref{\SETLABEL:TS} The slope of the
                        graph is the H parameter, and is {\hcalclow}
                        for the near term, and {\hcalcall} for the
                        far term.}
                    \label{\SETLABEL:THP}
                \end{minipage}
            \end{center}
        \end{figure}

% Local Variables:
% TeX-parse-self: t
% TeX-auto-save: t
% TeX-master: "fractal.tex"
% End:


        %
% -----------------------------------------------------------------------------
%
% A license is hereby granted to reproduce this software source code and
% to create executable versions from this source code for personal,
% non-commercial use.  The copyright notice included with the software
% must be maintained in all copies produced.
%
% THIS PROGRAM IS PROVIDED "AS IS". THE AUTHOR PROVIDES NO WARRANTIES
% WHATSOEVER, EXPRESSED OR IMPLIED, INCLUDING WARRANTIES OF
% MERCHANTABILITY, TITLE, OR FITNESS FOR ANY PARTICULAR PURPOSE.  THE
% AUTHOR DOES NOT WARRANT THAT USE OF THIS PROGRAM DOES NOT INFRINGE THE
% INTELLECTUAL PROPERTY RIGHTS OF ANY THIRD PARTY IN ANY COUNTRY.
%
% Copyright (c) 1994-2006, John Conover, All Rights Reserved.
%
% Comments and/or bug reports should be addressed to:
%
%     john@email.johncon.com (John Conover)
%
% -----------------------------------------------------------------------------
%
% Revision: \RCSRevision \\
% Revision Time: \RCSTime UMT \\
% Revision Date: \RCSDate \\
% Revision Id: \RCSId \\
% Revision File: \RCSLog \\
\RCS $Revision: 0.0 $
\RCS $Date: 2006/01/20 04:38:13 $
\RCS $Id: fiscal.tex,v 0.0 2006/01/20 04:38:13 john Exp $
% $Log: fiscal.tex,v $
% Revision 0.0  2006/01/20 04:38:13  john
% Initial version
%
%
    \subsection{Fixed Increment Approximation for Fiscal Strategy}
        \label{\SETLABEL:FS}

        \subidx{\market}{fiscal strategy}
        \subidx{markets}{analysis}
        \subidx{analysis}{markets}
        \subidx{strategy}{fiscal}
        \subidx{fiscal}{strategy}
        The data in this section is presented in tabular form in
        Section~\ref{\SETLABELREF:LR}. This section derives various
        values based on the ``average'' of the normalized increments
        presented in Figure~\ref{\SETLABEL:TFA}. These values are an
        approximation to a, probably, complex process with a
        distribution shown in Figure~\ref{\SETLABEL:TF}. These values
        will be used in a fixed increment Brownian fractal analysis
        and simulation of the {\market}, and may, or may not, provide
        adequate accuracy for projections.

        For an organization operating in the {\market}, the fiscal
        strategy, commensurate with the aggregate environment, can be
        derived as follows~\cite[pp. 128, pp
        151]{Schroeder},~\cite[pp. 450]{Reza},~\cite[pp. 270]{Pierce}:
        \vspace{0.15in}

        \subsubsection{Logarithmic Returns}
            \label{\SETLABEL:LR}

            \subidx{logarithmic}{returns}
            \subidx{returns}{logarithmic}
            \subidx{\market}{logarithmic returns}
            The logarithmic returns can be calculated by various
            means. Four will be presented here, for comparison.

            \subidx{programs}{tsnormal}
            \subidx{tsnormal}{program}
            \subidx{logarithmic}{returns}
            \subidx{returns}{logarithmic}
            The logarithmic returns, in bits, $bits$, as computed from
            the mean, by the program {\it tsnormal}\/, which is
            described in Chapter~\ref{programs}, and is presented in
            Figure~\ref{\SETLABEL:TF}, and Equation~\ref{abits} from
            Section~\ref{ereturns} in Chapter~\ref{general}:

            \begin{equation}
                bits = \frac{\ln \left({\datafractionmean} + 1\right)}{\ln \left(2\right)} = \datafractionmeanbits
            \end{equation}

            \subidx{programs}{tslsq}
            \subidx{tslsq}{program}
            \subidx{logarithmic}{returns}
            \subidx{returns}{logarithmic}
            \noindent By comparison, the logarithmic returns, in bits,
            $bits$, as computed from the constant in the least squares
            approximation, using the program {\it tslsq}\/, which is briefly
            described in Chapter~\ref{programs}, as presented in
            Figure~\ref{\SETLABEL:TF}, and Equation~\ref{abits} from
            Section~\ref{ereturns} in Chapter~\ref{general}:

            \begin{equation}
                bits = \frac{\ln \left({\datafractionconstant} + 1\right)}{\ln \left(2\right)} = \datafractionconstantbits
            \end{equation}

            Note that if the mean is not constant in
            Figure~\ref{\SETLABEL:TF}, this method will not provide
            accurate results.

            \subidx{programs}{tslsq}
            \subidx{tslsq}{program}
            \subidx{logarithmic}{returns}
            \subidx{returns}{logarithmic}
            \noindent And by yet another comparison, using the program
            {\it tslsq}\/, which is briefly described in
            Chapter~\ref{programs}, with the -e -p options, to provide
            a formula for the least squares exponential fit to the
            time series data set presented in
            Figure~\ref{\SETLABEL:TS}:

            \begin{equation}
                bits = {\datatslsqepbits}
            \end{equation}

            \subidx{programs}{tslogreturns}
            \subidx{tslogreturns}{program}
            \subidx{logarithmic}{returns}
            \subidx{returns}{logarithmic}
            \noindent And finally, by comparison, from the
            {\it tslogreturns}\/ program, which is briefly described
            in Chapter~\ref{programs}, with the -p option, to provide
            a formula for the logarithmic returns of the time series
            data set presented in Figure~\ref{\SETLABEL:TS}:

            \begin{equation}
                bits = {\logreturns}
            \end{equation}

        \subsubsection{Calculation of Shannon Probability}
            \label{\SETLABEL:SP}

            \subidx{\market}{Shannon probability}
            Ideally, all of the values presented in
            Section~\ref{\SETLABEL:LR} would be equal. Using the
            logarithmic returns provided by the {\it tslogreturns}\/
            program, to be consistent
            with~\cite[pp. 81]{Peters:CAOITCM}

            \subidx{programs}{tslogreturns}
            \subidx{tslogreturns}{program}
            \begin{equation}
                2^{{\logreturns}t}
            \end{equation}

            \noindent therefore:
            \begin{equation}
                C\left(p\right) = {\logreturns}
            \end{equation}
            \subidx{programs}{tsshannon}
            \subidx{tsshannon}{program}
            \subidx{Shannon}{probability}
            \subidx{probability}{Shannon}
            \noindent and, {\it tsshannon}\/ {\logreturns} gives:
            \begin{equation}
                \label{\SETLABEL:F0}
                C\left({\shannonlogreturns}\right) = {\logreturns}
            \end{equation}
            \noindent therefore:
            \begin{eqnarray}
                2^{C\left({\shannonlogreturns}\right)} & = & 2^{\logreturns}\\
                                                       & = & {\twologreturns}\\
                                                       & = & {\twologreturnshundred}\%
            \end{eqnarray}
            \noindent and:
            \begin{eqnarray}
                2p - 1 & = & \left(2 \cdot {\shannonlogreturns}\right) - 1\\
                       & = & {\twopone}\\
                       \label{\SETLABEL:F1}
                       & = & {\twoponehundred}\%
            \end{eqnarray}

            \subidx{\market}{fiscal strategy}
            \subidx{markets}{analysis}
            \subidx{analysis}{markets}
            \subidx{strategy}{fiscal}
            \subidx{fiscal}{strategy}
            \subidx{\market}{fiscal strategy}
            \subidx{\market}{growth rate}
            Presuming the simplified assumptions outlined in
            Section~\ref{assumptions}, the ``typical'' organization
            operating in the {\market} executes a long term fiscal
            strategy, commensurate with the aggregate environment,
            that is to invest, every {\timescale}, in sufficient
            additional resources and infrastructure, to increase the
            manufacturing of goods and services by {\twoponehundred}\%
            of its rate of revenue returns, (per {\timescale}.) As a
            conceptual model, the remaining {\hundredtwoponehundred}\%
            will be held in ``reserve'' with a
            {\shannonlogreturnshundred}\% chance of making twice the
            {\twoponehundred}\% back, (and a
            {\hundredshannonlogreturnshundred}\% chance of making
            0.0,) in one {\timescale}, on the average, for an average
            growth in its rate of revenue returns, (per {\timescale},)
            of {\twologreturnshundred}\%, or a doubling of its rate of
            revenue returns, (per {\timescale},) in
            {\oneoverlogreturns} {\timescale}s.

        \subsubsection{Example Fixed Increment Approximation Fiscal Strategies}

            \subidx{\market}{fiscal strategy}
            \subidx{markets}{analysis}
            \subidx{analysis}{markets}
            \subidx{strategy}{fiscal}
            \subidx{fiscal}{strategy}
            \subidx{\market}{fiscal strategy}
            \subidx{\market}{growth rate}
            \subidx{\market}{management metric}
            \idx{management metric}
            A possible metric on the effectiveness of long term fiscal
            management could possibly be that if an investment of
            {\twoponehundred}\% per {\timescale} of the rate of
            revenue returns, (per {\timescale},) is made in resources
            and infrastructure, then the rate of revenue returns would
            be expected to increase by {\twologreturnshundred}\%, per
            {\timescale}, on average.

            Note that the metrics presented in this section are
            representative of the {\market} as an aggregate whole, and
            may or may not be accurate representations for any
            particular participant in the environment. Of interest to
            the participants in the environment would be a similar
            analysis of each product or service rendered in the
            marketplace.

            \subidx{\market}{fiscal strategy}
            \subidx{markets}{analysis}
            \subidx{analysis}{markets}
            \subidx{strategy}{fiscal}
            \subidx{fiscal}{strategy}
            \subidx{\market}{fiscal strategy}
            As a simple illustrative example, a company operating in
            this environment might obtain a credit line from a bank
            that is equal to {\twoponehundred}\% of its rate of
            revenue returns, (per {\timescale},) to finance additional
            operations. In this simple scenario, the company would use
            its revenue base as collateral for the loan. Some
            {\timescale}s, depending on the {\market}'s environment,
            the company's rate of revenue returns exceeds what was
            borrowed from the bank, and the loan is repaid in
            full. Other {\timescale}s, the company must default, and
            the bank seizes a portion of the company's revenue base to
            pay the delinquent loan. However, on the average, the
            company will expand its rate of revenue returns at
            {\twologreturnshundred}\% per {\timescale}.

            \subidx{\market}{fiscal strategy}
            \subidx{markets}{analysis}
            \subidx{analysis}{markets}
            \subidx{strategy}{fiscal}
            \subidx{fiscal}{strategy}
            \subidx{\market}{fiscal strategy}
            As another simple example, a company re-invests
            {\twoponehundred}\% of its rate of revenue returns, (per
            {\timescale},) in development, marketing, sales, and
            distribution of new products.  Although some products will
            be successful and the return on the investment will exceed
            the {\twoponehundred}\% per {\timescale} investment,
            others will not. However, on the average, the company will
            expand it gross rate of revenue returns at
            {\twologreturnshundred}\% per {\timescale}.

            \subidx{\market}{fiscal strategy}
            \subidx{markets}{analysis}
            \subidx{analysis}{markets}
            \subidx{strategy}{fiscal}
            \subidx{fiscal}{strategy}
            \subidx{\market}{fiscal strategy}
            \subidx{\market}{product portfolio}
            \subidx{\market}{product diversity}
            \subidx{\market}{product mix}
            \subidx{\market}{optimum number of products}
            \idx{product portfolio}
            \idx{product diversity}
            \idx{optimum number of products}
            \idx{product mix}

            As an example of ``product portfolio'' management, suppose
            a company re-invests {\twoponehundred}\% of its rate of
            revenue returns, (per {\timescale},) in development,
            marketing, sales, and distribution of new products.
            Further suppose that the company has two products, and a
            fractal analysis of the individual product rate of revenue
            return time series indicates that one product has a
            Shannon probability of 0.65, and the other has a Shannon
            probability of 0.55. Then the percentage of re-investment
            in the first product would be $(2 \cdot 0.65 - 1) \cdot
            {\twoponehundred}$, percent of the rate of revenue
            returns, and $(2 \cdot 0.55 - 1) \cdot {\twoponehundred}$
            percent for the second product, implying that the company
            should diversify its product line\footnote{The astute
            reader would note that the linear addition was used to add
            the contribution to development of each product. This is a
            ``near term'' interpretation. Actually, in general, the
            method used should be a root mean square process,
            dependent on the Hurst Coefficient, $H$, where
            $P_{total}^H = P_1^H + P_2^H + \cdots$, where $P_n$ is the
            contribution to each individual product. For a Brownian
            motion, or random walk type of fractal the Hurst
            Coefficient is a function of time into the future. For the
            ``near term,'' the Hurst coefficient is very near unity,
            meaning the summation process is linear. For the ``long
            term,'' $H \approx 0.5$, or a standard root mean square
            summation process should be used. If $H$ is $0.5$ then the
            market is termed a Brownian motion, or random walk
            process. If it is larger than 0.5, it is termed fractional
            Brownian motion process. For a random walk process, ``near
            term'' and ``far term'' are quantitatively differentiated
            on the Hurst Coefficient graph where $1 - \ln (t) = 0.5
            \cdot \ln (t)$, or when $\ln (t) = 2$, or $t =
            7.389\ldots$ See~\cite[pp. 67, 83-84]{Peters:CAOITCM}
            and~\cite[pp. 129, 159]{Schroeder} for particulars on the
            implications of the Hurst Coefficient and root mean square
            summation issues.}.  Note that this is a ``bet hedging''
            metric methodology, and assumes that the products have
            uncorrelated revenue return rates. If this re-investment
            methodology is not feasible, perhaps for strategic
            financial reasons, then the re-investment in both products
            should total the ${\twoponehundred}$\%, and the investment
            in each product should be made at a ratio of $\frac{(2
            \cdot 0.65 - 1)}{(2 \cdot 0.55 - 1)} = 3 : 1$,
            respectively. Note that this ``bet hedging'' can be used
            to define the optimal number of products that can be
            supported on the rate of revenue returns. If it assumed
            that all products are ``typical'' for the {\market}, as a
            standard bench mark, then the optimal number will be
            $\frac{1}{{\twopone}}$. Note that this is a
            ``theoretical'' value, since not all products are
            ``typical,'' and there may be strategic reasons, for
            example product leveraging, that may increase the number
            of products above the optimum. However, most of the
            revenue should come from the optimal number of products,
            since having more products will decrease the amount of the
            potential investment in each product, and having less than
            the optimum number of products will increase the risk that
            many of the products could suffer a ``down market''
            concurrently, impacting the rate of revenue returns.  As
            another interesting interpretation of the optimal
            ``hedging of bets,'' in product portfolio strategy, and
            considering the graph of the normalized increments
            presented in Figure~\ref{\SETLABEL:TF}, if the
            organization is running optimally, then these products
            will generate, at least in principle, one standard
            deviation, approximately $0.8413 = 84.13$\% of the future
            growth in rate of revenue returns. Naturally, these are
            approximations, and the values are an approximation to a,
            probably, complex process, and appropriate scrutiny should
            be exercised before making specific projections.  As yet
            another example of ``product portfolio'' management,
            consider the issue of product mix. In this interpretation,
            {\twoponehundred}\% of the product manufactured should be
            ``proprietary,'' while the rest is ``industry standard.''
            As yet another possibility, {\twoponehundred}\% of the
            product manufactured should be predatory into new markets,
            and the remainder in markets that are ``traditional'' for
            the company.

% Local Variables:
% TeX-parse-self: t
% TeX-auto-save: t
% TeX-master: "fractal.tex"
% End:


        %
% -----------------------------------------------------------------------------
%
% A license is hereby granted to reproduce this software source code and
% to create executable versions from this source code for personal,
% non-commercial use.  The copyright notice included with the software
% must be maintained in all copies produced.
%
% THIS PROGRAM IS PROVIDED "AS IS". THE AUTHOR PROVIDES NO WARRANTIES
% WHATSOEVER, EXPRESSED OR IMPLIED, INCLUDING WARRANTIES OF
% MERCHANTABILITY, TITLE, OR FITNESS FOR ANY PARTICULAR PURPOSE.  THE
% AUTHOR DOES NOT WARRANT THAT USE OF THIS PROGRAM DOES NOT INFRINGE THE
% INTELLECTUAL PROPERTY RIGHTS OF ANY THIRD PARTY IN ANY COUNTRY.
%
% Copyright (c) 1994-2006, John Conover, All Rights Reserved.
%
% Comments and/or bug reports should be addressed to:
%
%     john@email.johncon.com (John Conover)
%
% -----------------------------------------------------------------------------
%
% Revision: \RCSRevision \\
% Revision Time: \RCSTime UMT \\
% Revision Date: \RCSDate \\
% Revision Id: \RCSId \\
% Revision File: \RCSLog \\
\RCS $Revision: 0.0 $
\RCS $Date: 2006/01/20 04:38:13 $
\RCS $Id: companies.tex,v 0.0 2006/01/20 04:38:13 john Exp $
% $Log: companies.tex,v $
% Revision 0.0  2006/01/20 04:38:13  john
% Initial version
%
%
    \subsection{Number of Companies}
        \label{\SETLABEL:QNC}

        \subidx{\market}{number of companies}
        \subidx{number of companies}{analysis}
        \subidx{analysis}{number of companies}
        \subidx{Shannon}{probability}
        \subidx{probability}{Shannon}
        This section evaluates the approximate, or ``average,'' number
        of companies in the {\market}, and uses the method outlined in
        Chapter~\ref{general}, Section~\ref{aftsma}. Since the
        average, $avg_{ind}$, and the root mean square, $rms_{ind}$,
        of the normalized increments of the {\market} time series is
        \datafractionmean, and \datafractionrms respectively, the
        number of companies participating in the market can be
        calculated by Equation~\ref{ncompanies} to be {\ncompanies}.

        If this value seems consistent number of companies in the
        {\market}, within the assumptions outlined in
        Chapter~\ref{general}, Section~\ref{aftsma}, then it would
        seem that there is some circumstantial or indirect evidence
        that the companies participating in the {\market} are
        operating optimally, and the ``average'' Shannon probability,
        $P$ for each participating company would be, using
        Equation~\ref{pncompanies}, {\pncompanies}, which would be the
        value which should be used in Section~\ref{\SETLABEL:FS} for
        each participating company if market expansion was to be
        consistent with the rest of the industry. However, if the
        Shannon probability derived in Section~\ref{\SETLABEL:FS} is
        greater than the average Shannon probability for the companies
        participating in the {\market}, as derived in this section,
        then the market would, possibly, be exploitable with the
        fiscal strategy outlined in Section~\ref{\SETLABEL:FS}. The
        maximum exploitability for the {\market} is derived in
        Section~\ref{\SETLABEL:MAXSHANNON}, but it is probably of
        doubtful practicality.

        Note that these optimizations would maximize a company's
        market growth. Since there are probably many companies
        competing in the market place, this would not necessarily
        maximize a company's P\&L, as described in
        Chapter~\ref{general}, Section~\ref{ompl}. The Shannon
        probability that maximizes market share in the {\market} is
        \pncompanies, with several alternative solutions listed in the
        previous paragraph. However, these should be contrasted to the
        Shannon probability that maximizes a company's P\&L which is
        \avgrms~in the {\market}. In all cases, the fraction of the
        P\&L that should be ``wagered'' on the future, $f$, should be:

        \begin{equation}
            f = 2P - 1
        \end{equation}

        \noindent where $P$ is the particular Shannon probability
        chosen optimize a particular fiscal strategy. Interestingly,
        the measured Shannon probability of the {\market} would tend
        to indicate that the companies participating in the market
        have chosen a fiscal strategy that optimizes market growth, as
        opposed to capital growth.

        \subidx{\market}{increasing returns}
        \subidx{economic increasing returns}{\market}
        As interesting interpretation of these exploitive issues,
        since all three fiscal strategies will result in exponential
        market growth for every company participating in the market,
        is that they may represent, perhaps, an example of
        ``increasing returns.''

% Local Variables:
% TeX-parse-self: t
% TeX-auto-save: t
% TeX-master: "fractal.tex"
% End:


        %
% -----------------------------------------------------------------------------
%
% A license is hereby granted to reproduce this software source code and
% to create executable versions from this source code for personal,
% non-commercial use.  The copyright notice included with the software
% must be maintained in all copies produced.
%
% THIS PROGRAM IS PROVIDED "AS IS". THE AUTHOR PROVIDES NO WARRANTIES
% WHATSOEVER, EXPRESSED OR IMPLIED, INCLUDING WARRANTIES OF
% MERCHANTABILITY, TITLE, OR FITNESS FOR ANY PARTICULAR PURPOSE.  THE
% AUTHOR DOES NOT WARRANT THAT USE OF THIS PROGRAM DOES NOT INFRINGE THE
% INTELLECTUAL PROPERTY RIGHTS OF ANY THIRD PARTY IN ANY COUNTRY.
%
% Copyright (c) 1994-2006, John Conover, All Rights Reserved.
%
% Comments and/or bug reports should be addressed to:
%
%     john@email.johncon.com (John Conover)
%
% -----------------------------------------------------------------------------
%
% Revision: \RCSRevision \\
% Revision Time: \RCSTime UMT \\
% Revision Date: \RCSDate \\
% Revision Id: \RCSId \\
% Revision File: \RCSLog \\
\RCS $Revision: 0.0 $
\RCS $Date: 2006/01/20 04:38:13 $
\RCS $Id: operations.tex,v 0.0 2006/01/20 04:38:13 john Exp $
% $Log: operations.tex,v $
% Revision 0.0  2006/01/20 04:38:13  john
% Initial version
%
%
    \subsection{Fixed Increment Approximation for Operational Strategy}
        \label{\SETLABEL:OPS}.

        This section derives various values based on the ``average''
        of the normalized increments presented in
        Figure~\ref{\SETLABEL:TFA}. These values are an approximation
        to a, probably, complex process with a distribution shown in
        Figure~\ref{\SETLABEL:TF}. These values will be used in a
        fixed increment Brownian fractal analysis and simulation of
        the {\market}, and may, or may not, provide adequate accuracy
        for projections.

        \subidx{\market}{fiscal strategy}
        \subidx{\market}{Shannon probability}
        \subidx{strategy}{fiscal}
        \subidx{fiscal}{strategy}
        \subidx{Shannon}{probability}
        \subidx{probability}{Shannon}
        It should be noted that the analysis of fiscal strategy,
        presented in Section~\ref{\SETLABEL:FS}, is derived from the
        {\market} metrics and may, or may not, be maximally
        optimal. For the optimal fiscal strategy, which may be
        exploitable, see Section~\ref{\SETLABEL:MAXSHANNON}.

        \subidx{strategy}{exploitable}
        \subidx{exploitable}{strategy}
        \subidx{\market}{windows of opportunity}
        \idx{windows of opportunity}
        \subidx{decision}{obsolete}
        \subidx{obsolete}{decision}
        \subidx{decision}{timeliness}
        \subidx{timeliness}{decision}
        \subidx{rate of revenue returns}{forecast}
        \subidx{forecast}{rate of revenue returns}
        An additional exploitable strategy may be time itself.
        Equations~\ref{\SETLABEL:V},~\ref{\SETLABEL:R},
        and,~\ref{\SETLABEL:MA}, are, essentially, metrics on how fast
        a decision, which is based on information concerning the
        current status of the {\market}, becomes obsolete. Obviously,
        how long a decision is expected to remain relevant should be
        addressed as an operational necessity in strategic planning
        and project management. Figures~\ref{\SETLABEL:FN},
        and,~\ref{\SETLABEL:FF} compare methods of approximation of
        the ``forecastability'' of rate of revenue returns in the
        {\market} for the near term and far
        term~\cite[pp. 83-84]{Peters:CAOITCM}, respectively. As a
        general rule, caution must be exercised when making decisions
        that will span a time interval larger than the time interval
        where the ``forecastability'' of rate of revenue returns drops
        below 50\%. Beyond this time interval, the chances increase
        that the competitive and market forces will alter the market
        environment in a possibly detrimental unanticipated
        fashion. Obviously, there is significant advantage in
        ``timeliness'' of development, manufacturing, and distribution
        of products and services that are consistent with this
        temporal agenda. Automation of these processes, if executed
        consistently with this agenda, should be considered a
        competitive advantage.

        \subidx{strategy}{exploitable}
        \subidx{exploitable}{strategy}
        \subidx{rate of revenue returns}{forecast}
        \subidx{forecast}{rate of revenue returns}
        \idx{product life cycle}
        \idx{life cycle, product}
        In some sense, this temporal agenda defines the ``average''
        product or service life cycle in the {\market}. When the
        ``forecastability'' of rate of revenue returns drops below
        50\%, there is an even chance that the rate of revenue returns
        for the product or service will change in a detrimental
        fashion. If it is assumed that a product or service life cycle
        consists of a ramp up, a maintenence interval, and a ramp
        down, then, if all three life cycle intervals are equal, the
        product life cycle will be, approximately, three times the
        time interval where the ``forecastability'' of rate of revenue
        returns drops below 50\%. Although probably not an accurate
        prediction of product or service life cycle, the technique may
        be used as a conceptual approximation to the dynamics of
        ``market windows.\footnote{For example, consider the market
        for table salt. Since it has inelastic supply and demand
        curves, and is a necessary requirement for life, it would be
        expected that the Hurst coefficient would be very near
        unity---ignoring competitive pressures in the market. The
        predictability of the table salt market would, therefore, be
        expected to be relatively good, over time.}''  The conceptual
        approximation will probably predict a ``conservative'' or
        ``pessimistic'' value in relation to actual markets.

        \begin{figure}[ht]
            \begin{center}
                \begin{minipage}[t]{0.45\textwidth}
                    \epsfxsize=1.0\linewidth
                    \epsffile{\directory/datahurstlownear.eps}
                    \caption[{\market}, ``forecastability'' of near
                        term rate of revenue returns]{{\market},
                        ``forecastability'' of near term rate of
                        revenue returns. Although the error function
                        is the most accurate, for the near term,
                        $H^{t} = \thurstlow^{t}$ may be used as a
                        reliable metric of ``forecastability'' of the
                        rate of revenue returns.}
                    \label{\SETLABEL:FN}
                \end{minipage}
                \hfill
                \begin{minipage}[t]{0.45\textwidth}
                    \epsfxsize=1.0\linewidth
                    \epsffile{\directory/datahurstlowfar.eps}
                    \caption[{\market}, ``forecastability'' of far
                        term rate of revenue returns]{{\market},
                        ``forecastability'' of far term rate of
                        revenue returns. Although the error function
                        is the most accurate, for the far term,
                        $\frac{1}{\sqrt{t}}$ may be used as a reliable
                        metric of ``forecastability'' of the rate of
                        revenue returns.}
                    \label{\SETLABEL:FF}
                \end{minipage}
            \end{center}
        \end{figure}

        \idx{operations research}
        As an interesting interpretation of the data presented in
        Figure~\ref{\SETLABEL:FN}, there may be, perhaps, some
        applicability to such operational agendas as inventory
        control. Maintaining too little inventory, obviously, will
        create a situation where the organization can not exploit
        market expansion, and maintaining too much inventory,
        likewise, would over extend the company, creating unnecessary
        losses when the market contracts. The company should maintain
        inventory levels that do not exceed, from
        Equation~\ref{\SETLABEL:MA}, ${\thurstlow}^{n} = 0.5$
        {\timescale}s of operations. Since the optimal amount of
        inventory and, from Equation~\ref{\SETLABEL:V}, the variance
        of change in the rate of revenue returns in the future can be
        calculated, there may, perhaps, be some applicability to a
        forecasting methodology that can be incorporated into other
        areas of operations research, for example the linear algebras
        using simplex methodologies for optimization of manufacturing
        processes. Traditionally, these forecasts are made by the
        sales department, and are subject to various subjective
        biases.

% Local Variables:
% TeX-parse-self: t
% TeX-auto-save: t
% TeX-master: "fractal.tex"
% End:


        %
% -----------------------------------------------------------------------------
%
% A license is hereby granted to reproduce this software source code and
% to create executable versions from this source code for personal,
% non-commercial use.  The copyright notice included with the software
% must be maintained in all copies produced.
%
% THIS PROGRAM IS PROVIDED "AS IS". THE AUTHOR PROVIDES NO WARRANTIES
% WHATSOEVER, EXPRESSED OR IMPLIED, INCLUDING WARRANTIES OF
% MERCHANTABILITY, TITLE, OR FITNESS FOR ANY PARTICULAR PURPOSE.  THE
% AUTHOR DOES NOT WARRANT THAT USE OF THIS PROGRAM DOES NOT INFRINGE THE
% INTELLECTUAL PROPERTY RIGHTS OF ANY THIRD PARTY IN ANY COUNTRY.
%
% Copyright (c) 1994-2006, John Conover, All Rights Reserved.
%
% Comments and/or bug reports should be addressed to:
%
%     john@email.johncon.com (John Conover)
%
% -----------------------------------------------------------------------------
%
% Revision: \RCSRevision \\
% Revision Time: \RCSTime UMT \\
% Revision Date: \RCSDate \\
% Revision Id: \RCSId \\
% Revision File: \RCSLog \\
\RCS $Revision: 0.0 $
\RCS $Date: 2006/01/20 04:38:13 $
\RCS $Id: simulation.tex,v 0.0 2006/01/20 04:38:13 john Exp $
% $Log: simulation.tex,v $
% Revision 0.0  2006/01/20 04:38:13  john
% Initial version
%
%
    \subsection{Simulation of Fixed Increment Approximation for Fiscal Strategy}
        \label{\SETLABEL:TSUNFAIRBROWNIAN}

        \subidx{\market}{market simulation}
        The data in this section is presented in tabular form in
        Section~\ref{\SETLABELREF:SIM}.
        Figure~\ref{\SETLABEL:TSUNFAIRBROWNIAN0} represents a
        constructional simulation of the time series data presented in
        Figure~\ref{\SETLABEL:TS}. The program {\it
        tsunfairbrownian}\/, which is briefly described in
        appendix~\ref{programs}, was used in the reconstruction. The
        reconstructed data is superimposed on the original time series
        data.  The program, {\it tsunfairbrownian}\/, essentially,
        constructs the new time series as a Brownian fractal with
        fixed increments---the value of the fixed increment is derived
        from the root mean square average of the normalized increments
        presented in Figure~\ref{\SETLABEL:TF}. The ``quality'' of
        such a reconstruction should be subject to adequate scepticism
        and scrutiny since, in all probability, the normalized
        increments presented in Figure~\ref{\SETLABEL:TF} represent a
        relatively complex process, that may not be ``modeled'' with
        such a simple methodology.

        As a further comparison of the the constructional simulation
        with the original time series data,
        Figure~\ref{\SETLABEL:TSUNFAIRBROWNIAN1} presents a normalized
        histogram of the normalized increments of the reconstructed
        time series, superimposed on the normalized histogram
        presented in Figure~\ref{\SETLABEL:NH}.

        \subidx{\market}{fiscal strategy, simulation}
        \subidx{markets}{simulation}
        \subidx{simulation}{markets}
        \subidx{strategy}{fiscal, simulation}
        \subidx{fiscal}{strategy, simulation}
        \subidx{programs}{tsunfairbrownian}
        \subidx{tsunfairbrownian}{program}
        \begin{figure}[ht]
            \begin{center}
                \begin{minipage}[t]{0.45\textwidth}
                    \epsfxsize=1.0\linewidth
                    \epsffile{\directory/tsunfairbrownian-f.eps}
                    \caption[{\market}, Time series data, empirical and
                        simulated]{{\market}, Time series data, empirical
                        and simulated, using the program {\it tsunfairbrownian}\/
                        with f = {\datafractionrms}. This data is
                        superimposed on the data presented in
                        Figure~\ref{\SETLABEL:TS}.}
                    \label{\SETLABEL:TSUNFAIRBROWNIAN0}
                \end{minipage}
                \hfill
                \begin{minipage}[t]{0.45\textwidth}
                    \epsfxsize=1.0\linewidth
                    \epsffile{\directory/tsunfairbrownian-f.tsfraction.tsnormal-s30.eps}
                    \caption[{\market}, normalized histogram,
                        empirical and simulated]{{\market}, normalized
                        histogram of the normalized increments of the
                        time series data shown in
                        Figure~\ref{\SETLABEL:TSUNFAIRBROWNIAN0},
                        empirical and simulated.  The empirical data
                        has a mean of {\datafractionmean}, with a
                        standard deviation of {\datafractionstddev}.
                        By comparison, the simulated data has a mean
                        of {\tsunfairbrownianfractionmean} with a
                        standard deviation of
                        {\tsunfairbrownianfractionstddev}. This data
                        is superimposed on the data presented in
                        Figure~\ref{\SETLABEL:NH}. The area under the
                        four curves is identical.}
                    \label{\SETLABEL:TSUNFAIRBROWNIAN1}
                \end{minipage}
            \end{center}
        \end{figure}

% Local Variables:
% TeX-parse-self: t
% TeX-auto-save: t
% TeX-master: "fractal.tex"
% End:


        %
% -----------------------------------------------------------------------------
%
% A license is hereby granted to reproduce this software source code and
% to create executable versions from this source code for personal,
% non-commercial use.  The copyright notice included with the software
% must be maintained in all copies produced.
%
% THIS PROGRAM IS PROVIDED "AS IS". THE AUTHOR PROVIDES NO WARRANTIES
% WHATSOEVER, EXPRESSED OR IMPLIED, INCLUDING WARRANTIES OF
% MERCHANTABILITY, TITLE, OR FITNESS FOR ANY PARTICULAR PURPOSE.  THE
% AUTHOR DOES NOT WARRANT THAT USE OF THIS PROGRAM DOES NOT INFRINGE THE
% INTELLECTUAL PROPERTY RIGHTS OF ANY THIRD PARTY IN ANY COUNTRY.
%
% Copyright (c) 1994-2006, John Conover, All Rights Reserved.
%
% Comments and/or bug reports should be addressed to:
%
%     john@email.johncon.com (John Conover)
%
% -----------------------------------------------------------------------------
%
% Revision: \RCSRevision \\
% Revision Time: \RCSTime UMT \\
% Revision Date: \RCSDate \\
% Revision Id: \RCSId \\
% Revision File: \RCSLog \\
\RCS $Revision: 0.0 $
\RCS $Date: 2006/01/20 04:38:13 $
\RCS $Id: maximum.tex,v 0.0 2006/01/20 04:38:13 john Exp $
% $Log: maximum.tex,v $
% Revision 0.0  2006/01/20 04:38:13  john
% Initial version
%
%
    \subsection{Simulation of Fixed Increment Approximation for Optimally Maximal Fiscal Strategy}
        \label{\SETLABEL:MAXSHANNON}
        \subidx{\market}{fiscal strategy, simulation}
        \subidx{\market}{maximum Shannon probability}
        \subidx{markets}{simulation}
        \subidx{simulation}{markets}
        \subidx{strategy}{optimum fiscal, simulation}
        \subidx{fiscal}{optimum strategy, simulation}
        \subidx{programs}{tsunfairbrownian}
        \subidx{tsunfairbrownian}{program}
        \subidx{Shannon}{probability}
        \subidx{probability}{Shannon}

        \subidx{strategy}{exploitable}
        \subidx{exploitable}{strategy}
        \subidx{programs}{tsshannonmax}
        \subidx{tsshannonmax}{program}
        \subidx{programs}{tsunfairbrownian}
        \subidx{tsunfairbrownian}{program}
        \subidx{strategy}{fiscal}
        \subidx{fiscal}{strategy}
        The data in this section is presented in tabular form in
        Section~\ref{\SETLABELREF:MAXSHANNON}. One of the issues of
        analysis, as mentioned in Section~\ref{\SETLABEL:OPS}, is to
        determine the maximum Shannon probability for the time series
        presented in Figure~\ref{\SETLABEL:TS}. Potentially, this
        could be exploited with an aggressive fiscal
        strategy. Figure~\ref{\SETLABEL:SHANNONMAX0} is a graph of the
        output of the {\it tsshannonmax}\/ program, which is described
        briefly in appendix~\ref{programs}. The maximum of this
        function is the maximum Shannon probability for the time
        series data presented in Figure~\ref{\SETLABEL:TS}.
        Figure~\ref{\SETLABEL:SHANNONMAX1} was constructed using {\it
        tsunfairbrownian}\/ program, which is also described in
        appendix~\ref{programs}, with the maximum Shannon probability,
        and the time series data presented in
        Figure~\ref{\SETLABEL:TS}. This represents a ``what if'' the
        investment strategy was changed from a Shannon probability of
        {\shannonlogreturns}, as derived in Section~\ref{\SETLABEL:SP}
        to {\shannonmax}. This process, essentially, extracts the
        random statistical data from the time series presented in
        Figure~\ref{\SETLABEL:TS}, and constructs a new time series,
        using the random statistical data, with a different investment
        strategy.  The program, {\it tsunfairbrownian}\/, essentially,
        constructs the new time series as a Brownian fractal with
        fixed increments.  The ``quality'' of such a reconstruction
        should be subject to adequate scepticism and scrutiny since,
        in all probability, the increments in the original data
        represent a relatively complex process, that may not be
        ``modeled'' with such a simple methodology.

        \begin{figure}[ht]
            \begin{center}
                \begin{minipage}[t]{0.45\textwidth}
                    \epsfxsize=1.0\linewidth
                    \epsffile{\directory/data.tsshannonmax.eps}
                    \caption[{\market}, maximum rate of revenue
                        returns] {{\market}, maximum rate of revenue
                        returns, per {\timescale}, vs. Shannon
                        probability. The maximum rate of revenue
                        returns, per {\timescale}, occurs at a Shannon
                        probability of {\shannonmax}.}
                    \label{\SETLABEL:SHANNONMAX0}
                \end{minipage}
                \hfill
                \begin{minipage}[t]{0.45\textwidth}
                    \epsfxsize=1.0\linewidth
                    \epsffile{\directory/data.tsshannonmax-p.tsunfairbrownian-p.eps}
                    \caption[{\market}, maximum rate of revenue
                        returns] {{\market}, maximum rate of revenue
                        returns, per {\timescale}, at a Shannon
                        probability, of {\shannonmax}, corresponding
                        to a ``wager'' fraction of {\twoponemax}.}
                    \label{\SETLABEL:SHANNONMAX1}
                \end{minipage}
            \end{center}
        \end{figure}

        \subidx{fractional}{Brownian motion}
        \subidx{Brownian motion}{fractional}
        \subidx{Shannon}{probability}
        \subidx{probability}{Shannon}
        \subidx{programs}{tsshannonmax}
        \subidx{tsshannonmax}{program}
        If it is assumed that the time series data set, presented in
        Figure~\ref{\SETLABEL:TS}, constitutes classical Brownian
        motion, then the Shannon probability can be calculated by
        counting the total number of {\timescale}s that the {\market}
        movement was positive, and dividing by the total number of
        {timescale}s represented in the time series. This quotient is
        {\pmax}, as compared with the predicted value from the program
        {\it tsshannonmax}\/ of {\shannonmax}.

% Local Variables:
% TeX-parse-self: t
% TeX-auto-save: t
% TeX-master: "fractal.tex"
% End:


        %
% -----------------------------------------------------------------------------
%
% A license is hereby granted to reproduce this software source code and
% to create executable versions from this source code for personal,
% non-commercial use.  The copyright notice included with the software
% must be maintained in all copies produced.
%
% THIS PROGRAM IS PROVIDED "AS IS". THE AUTHOR PROVIDES NO WARRANTIES
% WHATSOEVER, EXPRESSED OR IMPLIED, INCLUDING WARRANTIES OF
% MERCHANTABILITY, TITLE, OR FITNESS FOR ANY PARTICULAR PURPOSE.  THE
% AUTHOR DOES NOT WARRANT THAT USE OF THIS PROGRAM DOES NOT INFRINGE THE
% INTELLECTUAL PROPERTY RIGHTS OF ANY THIRD PARTY IN ANY COUNTRY.
%
% Copyright (c) 1994-2006, John Conover, All Rights Reserved.
%
% Comments and/or bug reports should be addressed to:
%
%     john@email.johncon.com (John Conover)
%
% -----------------------------------------------------------------------------
%
% Revision: \RCSRevision \\
% Revision Time: \RCSTime UMT \\
% Revision Date: \RCSDate \\
% Revision Id: \RCSId \\
% Revision File: \RCSLog \\
\RCS $Revision: 0.0 $
\RCS $Date: 2006/01/20 04:38:13 $
\RCS $Id: verification.tex,v 0.0 2006/01/20 04:38:13 john Exp $
% $Log: verification.tex,v $
% Revision 0.0  2006/01/20 04:38:13  john
% Initial version
%
%
    \subsection{Qualitative Verification of Fixed Increment Approximation Analysis}
        \label{\SETLABEL:QVA}

        \subidx{\market}{verification of analysis}
        \subidx{verification}{analysis}
        \subidx{analysis}{verification}
        \subidx{quality}{of analysis}
        \subidx{verification}{of methodology}
        \subidx{methodology}{verification of}
        \subidx{Shannon}{probability}
        \subidx{probability}{Shannon}

        This section evaluates various values based on the ``average''
        of the normalized increments presented in
        Figure~\ref{\SETLABEL:TFA}. These values are an approximation
        to a, probably, complex process with a distribution shown in
        Figure~\ref{\SETLABEL:TF}. These values will be used in a
        fixed increment Brownian fractal analysis of the {\market},
        and may, or may not, provide adequate accuracy for
        projections.

        The data in this section is presented in tabular form in
        sections~\ref{\SETLABELREF:VI1} and~\ref{\SETLABELREF:VI2}.
        As a subjective evaluation of the ``quality'' of the analysis
        of the {\market}, from Chapter~\ref{methodology},
        Equation~\ref{metricvalues1}, and using the mean and root mean
        square values of the normalized increments of the time series
        data presented in Figure~\ref{\SETLABEL:TS} from
        Figure~\ref{\SETLABEL:TF}, and the Shannon probability as
        calculated by counting the total number of {\timescale}s that
        the {\market} movement was positive, as presented in
        Section~\ref{\SETLABEL:MAXSHANNON}:

        \begin{eqnarray}
                  P & \approx & \frac{\frac{avg}{rms} + 1}{2}\\
            {\pmax} & \approx & \frac{\frac{\datafractionmean}{\datafractionrms} + 1}{2}\\
            {\pmax} & \approx & {\avgrms}
            \label{\SETLABEL:AVGS}
        \end{eqnarray}

        \subidx{Shannon}{probability}
        \subidx{probability}{Shannon}
        \noindent and comparing these values to the Shannon
        probability, as found by the {\it tsshannonmax}\/ program, which
        iterates for a maximum:

        \begin{eqnarray}
            {\pmax} \approx {\avgrms} \approx {\shannonmax}
        \end{eqnarray}

        \subidx{logarithmic}{returns}
        \subidx{returns}{logarithmic}
        In addition, the different methods of calculating the
        logarithmic returns, presented in Section~\ref{\SETLABEL:FS},
        should be compared. The four methods used were the mean of
        Figure~\ref{\SETLABEL:TF}, the constant in the least squares
        approximation to Figure~\ref{\SETLABEL:TF}, the least squares
        exponential approximation to Figure~\ref{\SETLABEL:TS}, and
        the logarithmic returns of Figure~\ref{\SETLABEL:TS}, derived
        as the mean of the logarithms of the quotients of the
        increments. The values for each of the methods are,
        respectively:

        \begin{equation}
            \datafractionmeanbits \approx \datafractionconstantbits \approx \datatslsqepbits \approx \logreturns
        \end{equation}

        It is implied in Section~\ref{\SETLABEL:FS},
        Subsection~\ref{\SETLABEL:SP} and in
        Section~\ref{\SETLABEL:TSUNFAIRBROWNIAN} that, a Brownian
        motion with fixed increments fractal may ``model'' the
        {\market}. Using Equation~\ref{stddev9} from
        Chapter~\ref{general}, Section~\ref{abmfi}:

        \begin{eqnarray}
                                    rms \left(2P - 1\right) & \approx & \frac{\sigma \left(2P - 1\right)}{2 \sqrt{P\left(1 - P\right)}}\\
            \datafractionrms \left(2 \cdot \pmax - 1\right) & \approx & \frac{\datafractionstddev \left(2 \cdot \pmax - 1\right)}{2\sqrt{\pmax \left(1 - \pmax\right)}}\\
                       \datafractionrms \cdot \twopminusone & \approx & \datafractionstddev \cdot \twopx\\
                                                      \rmsp & \approx & \sigmap
        \end{eqnarray}

        \noindent and, equating to the mean:

        \begin{equation}
            \datafractionmean \approx \rmsp \approx \sigmap
        \end{equation}

        \subidx{Shannon}{probability}
        \subidx{probability}{Shannon}
        \noindent where, as in Equation~\ref{\SETLABEL:AVGS} using the
        mean, root mean square, and standard deviation values of the
        normalized increments of the time series data presented in
        Figure~\ref{\SETLABEL:TS} from Figure~\ref{\SETLABEL:TF}, and
        the Shannon probability as calculated by counting the total
        number of {\timescale}s that the {\market} movement was
        positive, as presented in Section~\ref{\SETLABEL:MAXSHANNON}.

        As a final qualitative comparison, the absolute value of the
        normalized increments should be the same as the root mean
        square value\footnote{The absolute value of the normalized
        increments, when averaged, is related to the root mean square
        of the increments by a constant. If the normalized increments
        are a fixed increment, the constant is unity. If the
        normalized increments have a Gaussian distribution, the
        constant is $\approx 0.8$ depending on the accuracy of of
        ``fit'' to a Gaussian distribution.}, where the absolute value
        is presented in Figure~\ref{\SETLABEL:TFA}, and the root mean
        square value is presented in Figure~\ref{\SETLABEL:TF}:

        \begin{equation}
            \datafractionabsmean \approx \datafractionrms
        \end{equation}

        Note, that if the {\market} could be ``modeled'' as a Brownian
        motion with fixed increments fractal, then the standard
        deviation of the absolute value of the normalized increments
        of the time series data presented in Figure~\ref{\SETLABEL:TS}
        from Figure~\ref{\SETLABEL:TF} should be zero. It is
        $\datafractionabsstddev$.

% Local Variables:
% TeX-parse-self: t
% TeX-auto-save: t
% TeX-master: "fractal.tex"
% End:


    \renewcommand{\market}{Simulated Industrial Market}
    \renewcommand{\directory}{../markets/tsmarket}
    \renewcommand{\datafractionmean}{0.008052}
\renewcommand{\datafractionmeanbits}{0.011570}
\renewcommand{\datafractionmeanq}{0.002684}
\renewcommand{\datafractionmeanbitsq}{0.003867}
\renewcommand{\datafractionstddev}{0.038579}
\renewcommand{\datafractionrms}{0.039311}
\renewcommand{\avgrms}{0.602414}
\renewcommand{\ncompanies}{5.210454}
\renewcommand{\pncompanies}{0.544866}
\renewcommand{\datafractionabsmean}{0.029745}
\renewcommand{\datafractionabsstddev}{0.025769}
\renewcommand{\datafractionconstant}{0.010041}
\renewcommand{\datafractionconstantbits}{0.014414}
\renewcommand{\datafractionconstantq}{0.003347}
\renewcommand{\datafractionconstantbitsq}{0.004821}
\renewcommand{\datafractionslope}{-0.000021}
\renewcommand{\datafractionabsconstant}{0.035145}
\renewcommand{\datafractionabsslope}{-0.000057}
\renewcommand{\hurstall}{0.659558}
\renewcommand{\hurstlow}{0.707509}
\renewcommand{\hurstlowtwo}{1.415018}
\renewcommand{\hurstlowhundred}{70.750900}
\renewcommand{\hcalcall}{0.184942}
\renewcommand{\hcalclow}{0.102042}
\renewcommand{\shannonmax}{0.604167}
\renewcommand{\twoponemax}{0.208334}
\renewcommand{\logreturns}{0.010456}
\renewcommand{\twologreturns}{1.007274}
\renewcommand{\twologreturnshundred}{0.727387}
\renewcommand{\oneoverlogreturns}{95.638868}
\renewcommand{\pmax}{0.602094}
\renewcommand{\twopminusone}{0.204188}
\renewcommand{\rmsp}{0.008027}
\renewcommand{\twopx}{0.208583}
\renewcommand{\sigmap}{0.008047}
\renewcommand{\tsunfairbrownianfractionmean}{0.007862}
\renewcommand{\tsunfairbrownianfractionstddev}{0.038619}
\renewcommand{\shannonlogreturns}{0.560125}
\renewcommand{\shannonlogreturnshundred}{56.012500}
\renewcommand{\twopone}{0.120250}
\renewcommand{\twoponehundred}{12.025000}
\renewcommand{\hundredtwoponehundred}{87.975000}
\renewcommand{\hundredshannonlogreturnshundred}{43.987500}
\renewcommand{\datatslsqepbits}{0.007623}
\renewcommand{\thurstall}{0.633980}
\renewcommand{\thurstlow}{0.710108}
\renewcommand{\thurstlowtwo}{1.420216}
\renewcommand{\thurstlowhundred}{71.010800}
\renewcommand{\thcalcall}{0.247886}
\renewcommand{\thcalclow}{0.171737}
\renewcommand{\chisquared}{2.862000}
\renewcommand{\critical}{42.557000}

    \renewcommand{\timescale}{month}
    \subidx{market}{\market}
    \idx{\market}

    \section{\market}

        \renewcommand{\SETLABEL}{\LABPRE:SIM}
        \renewcommand{\SETLABELQ}{\LABPRE:SIMQ}
        \label{\SETLABEL}
        \renewcommand{\SETLABELREF}{\LABPREREF:SIM}

        \subidx{tsmarket}{program}
        \subidx{programs}{tsmarket}
        For the analysis, the data was in the directory
        {\directory}\footnote{As a simulation model, the program {\it
        tsmarket}\/ was run to make a time series data file, with the
        following parameters:

        \vspace{0.1in}
        {\noindent}tsmarket -p 0.55 -c 11 300 > data
        \vspace{0.1in}

        \noindent to make a time series of 300 elements, with a
        Shannon probability of 0.55, and 11 companies participating in
        the market, each with equal market share, and operating
        optimally.  The data is by {\timescale}s.}.

        The data in this section is presented in tabular form in
        Section~\ref{\SETLABELREF}. Note that in this analysis, the
        rate of revenue returns means the increase or decrease in the
        cumulative sum of the {\market}. This is included for
        ``theoretical'' comparative purposes.

        %
% -----------------------------------------------------------------------------
%
% A license is hereby granted to reproduce this software source code and
% to create executable versions from this source code for personal,
% non-commercial use.  The copyright notice included with the software
% must be maintained in all copies produced.
%
% THIS PROGRAM IS PROVIDED "AS IS". THE AUTHOR PROVIDES NO WARRANTIES
% WHATSOEVER, EXPRESSED OR IMPLIED, INCLUDING WARRANTIES OF
% MERCHANTABILITY, TITLE, OR FITNESS FOR ANY PARTICULAR PURPOSE.  THE
% AUTHOR DOES NOT WARRANT THAT USE OF THIS PROGRAM DOES NOT INFRINGE THE
% INTELLECTUAL PROPERTY RIGHTS OF ANY THIRD PARTY IN ANY COUNTRY.
%
% Copyright (c) 1994-2006, John Conover, All Rights Reserved.
%
% Comments and/or bug reports should be addressed to:
%
%     john@email.johncon.com (John Conover)
%
% -----------------------------------------------------------------------------
%
% Revision: \RCSRevision \\
% Revision Time: \RCSTime UMT \\
% Revision Date: \RCSDate \\
% Revision Id: \RCSId \\
% Revision File: \RCSLog \\
\RCS $Revision: 0.0 $
\RCS $Date: 2006/01/20 04:38:13 $
\RCS $Id: fraction.tex,v 0.0 2006/01/20 04:38:13 john Exp $
% $Log: fraction.tex,v $
% Revision 0.0  2006/01/20 04:38:13  john
% Initial version
%
%
    \subsection{Time Series Increments Analysis}
        \label{\SETLABEL:TSA}

        \subidx{\market}{Time series analysis}
        \subidx{time series}{increments}
        \subidx{time series}{analysis}
        \subidx{cumulative sum}{analysis}
        \subidx{analysis}{cumulative sum}
        \subidx{analysis}{random process}
        \subidx{random process}{analysis}
        \subidx{Gaussian}{increments}
        \subidx{increments}{Gaussian}
        \subidx{Brownian}{motion, fractional}
        \subidx{fractional}{Brownian motion}
        \subidx{fractal}{Brownian motion}
        The data in this section is presented in tabular form in
        Section~\ref{\SETLABELREF:TSA}.  Figure~\ref{\SETLABEL:TS} is
        a graph of the time series data for the {\market}.

        \subidx{increments}{normalized}
        \subidx{normalized}{increments}
        \subidx{programs}{tsfraction}
        \subidx{tsfraction}{program}
        Figure~\ref{\SETLABEL:TF} is a graph of the normalized
        increments of the time series data presented in
        Figure~\ref{\SETLABEL:TS}. The data presented was made by
        running the program {\it tsfraction}\/ on the time series
        data. The program {\it tsfraction}\/ is described briefly in
        Appendix~\ref{programs}, and subtracts the previous value from
        the next value, dividing this difference by the previous
        value, for each element in the time series data. The new time
        series contains the instantaneous change in the rate of
        revenue returns, divided by the magnitude of the instantaneous
        rate of revenue returns.

        \subidx{mean}{standard deviation}
        \subidx{standard deviation}{mean}
        \idx{root mean square}
        \idx{least squares approximation}
        \begin{figure}[ht]
            \begin{center}
                \begin{minipage}[t]{0.45\textwidth}
                    \epsfxsize=1.0\linewidth
                    \epsffile{\directory/data.eps}
                    \caption{{\market}, time series data.}
                    \label{\SETLABEL:TS}
                    \label{\SETLABELQ:TS}
                \end{minipage}
                \hfill
                \begin{minipage}[t]{0.45\textwidth}
                    \epsfxsize=1.0\linewidth
                    \epsffile{\directory/data.tsfraction.eps}
                    \caption[{\market}, normalized
                        increments]{{\market}, normalized increments
                        of the time series data presented in
                        Figure~\ref{\SETLABEL:TS}. The mean is
                        {\datafractionmean} with a standard deviation
                        of {\datafractionstddev}. The formula for the
                        least squares approximation is
                        ${\datafractionconstant} +
                        {\datafractionslope}t$, and the root mean
                        squared value is {\datafractionrms}. The
                        graph, labeled ``data\-.tsfraction\-.tsrms,''
                        is the running root mean square, and
                        ``data\-.tsfraction\-.tsavg'' is the running
                        average of the normalized increments.  This
                        graph is the fraction of change in the time
                        series, as a function of time. Note that the
                        slope of the mean, {\datafractionslope}, is
                        the coefficient of the nonlinearity term in
                        the normalized increments. See
                        Chapter~\ref{general}, Section~\ref{nlextend}
                        for a possible application of the logistic
                        function to this data set.}
                    \label{\SETLABEL:TF}
                    \label{\SETLABELQ:TF}
                \end{minipage}
            \end{center}
        \end{figure}

        \subidx{absolute value}{increments}
        \subidx{increments}{absolute value}

        Figure~\ref{\SETLABEL:TFA} is a graph of the absolute value of
        the normalized increments of the time series data presented in
        Figure~\ref{\SETLABEL:TF}. The data presented was made by
        running the Unix utility sed(1) on the normalized increments
        time series data to remove the negative signs. This is an
        absolute value procedure.  The resulting time series contains
        the absolute value of the instantaneous change in the rate of
        revenue returns, divided by the magnitude of the instantaneous
        rate of revenue returns\footnote{The absolute value of the
        normalized increments, when averaged, is related to the root
        mean square of the increments by a constant. If the normalized
        increments are a fixed increment, the constant is unity. If
        the normalized increments have a Gaussian distribution, the
        constant is $\approx 0.8$ depending on the accuracy of of
        ``fit'' to a Gaussian distribution.}.

        \subidx{histogram}{normalized}
        \subidx{normalized}{histogram}
        \subidx{programs}{tsnormal}
        \subidx{tsnormal}{program}
        \subidx{mean}{standard deviation}
        \subidx{standard deviation}{mean}
        \idx{root mean square}
        \idx{least squares approximation}
        \subidx{\market}{analysis of increments}
        Figure~\ref{\SETLABEL:NH} is the normalized histogram of the
        normalized increments of the time series data shown in
        Figure~\ref{\SETLABEL:TF}. The abscissa is 3 $\sigma$ limits,
        and the area under the two curves is identical. The data for
        this figure was produced by the program {\it tsnormal}\/,
        which is described briefly in Appendix~\ref{programs}.

        \begin{figure}[ht]
            \begin{center}
                \begin{minipage}[t]{0.45\textwidth}
                    \epsfxsize=1.0\linewidth
                    \epsffile{\directory/data.tsfraction.abs.eps}
                    \caption[{\market}, absolute value of the
                        normalized increments]{{\market}, absolute
                        value of the normalized increments of the time
                        series data presented in
                        Figure~\ref{\SETLABEL:TF}.  The mean is
                        {\datafractionabsmean} with a standard
                        deviation of {\datafractionabsstddev}. The
                        formula for the least squares approximation is
                        ${\datafractionabsconstant} +
                        {\datafractionabsslope}t$, and the root mean
                        square value, from Figure~\ref{\SETLABEL:TF},
                        is {\datafractionrms}.  The graph, labeled
                        ``data\-.tsfraction\-.tsrms,'' is the running
                        root mean square, and
                        ``data\-.tsfraction\-.tsavg'' is the running
                        average of the normalized increments presented
                        in Figure~\ref{\SETLABEL:TF}, superimposed
                        here for convenience. This graph is the
                        absolute value of the fraction of change in
                        the time series, as a function of time.}
                    \label{\SETLABEL:TFA}
                    \label{\SETLABELQ:TFA}
                \end{minipage}
                \hfill
                \begin{minipage}[t]{0.45\textwidth}
                    \epsfxsize=1.0\linewidth
                    \epsffile{\directory/data.tsfraction.tsnormal-s30.eps}
                    \caption[{\market}, normalized histogram of the
                        normalized increments]{{\market}, normalized
                        histogram of the normalized increments of the
                        time series data shown in
                        Figure~\ref{\SETLABEL:TF}.  The data has a
                        mean of {\datafractionmean}, with a standard
                        deviation of {\datafractionstddev}.  The area
                        under the two curves is identical. The
                        $\chi^2$ value of the observed and expected
                        values of the two curves is {\chisquared},
                        with a critical value of {\critical}.}
                    \label{\SETLABEL:NH}
                \end{minipage}
            \end{center}
        \end{figure}

        \subidx{programs}{tsXsquared}
        \subidx{tsXsquared}{program}
        \subidx{\market}{chi-squared values of increments}
        The program {\it tsXsquared}\/, which is briefly described in
        appendix~\ref{programs}, was used to derive the $\chi^2$
        statistics for the data presented in
        Figure~\ref{\SETLABEL:NH}.

        \subidx{programs}{tsstatest}
        \subidx{tsstatest}{program}
        \subidx{\market}{statistical estimates}

        Figure~\ref{\SETLABEL:SE} is the statistical estimate for the
        data presented in Figure~\ref{\SETLABEL:TF}, as derived by the
        program {\it tsstatest}\/, which is briefly described in
        appendix~\ref{programs}.

        \begin{figure}[ht]
            \begin{center}
                \begin{minipage}[t]{\textwidth}
                    \center{\fbox{\parbox{0.9\textwidth}{\XXX{\directory/data.tsstatest-f0.1-c0.9-i.tex}}}}
                    \caption[{\market}, statistical estimates of the
                        normalized increments]{{\market}, statistical
                        estimates of the normalized increments of the
                        time series shown in Figure~\ref{\SETLABEL:TF}.
                        The table was produced with the {\it
                        tsstatest}\/ program, and illustrates the
                        size of the data set required for a confidence
                        level of 90\%, with an error estimate of $\pm$
                        10\%, or alternately, the error estimate on
                        the time series shown in Figure~\ref{\SETLABEL:TF}.}
                    \label{\SETLABEL:SE}
                \end{minipage}
            \end{center}
        \end{figure}

        Note that the data set size estimations, as produced by the
        {\it tsstatest}\/ program, are probably very conservative,
        depending on the magnitude of the Shannon probability, $P =
        \shannonlogreturns$, as derived in
        Section~\ref{\SETLABEL:SP}. See Chapter~\ref{general},
        Section~\ref{serdss} for possible alternative methodologies
        for addressing the analysis of fractal time series with
        limited data set sizes. Depending on the magnitude of the
        Shannon probability, $P$, these estimates can be several
        orders of magnitude too high.

        \subidx{derivative of increments}{normalized}
        \subidx{normalized}{derivative of increments}
        \subidx{programs}{tsderivative}
        \subidx{tsderivative}{program}
        Figure~\ref{\SETLABEL:TF1} is the normalized histogram of the
        first derivative of the normalized increments of the time
        series data shown in Figure~\ref{\SETLABEL:TF}. In principle,
        if the distribution of the normalized increments presented in
        Figure~\ref{\SETLABEL:NH} is Gaussian in nature, this
        distribution would be similar to ``white noise,'' as presented
        in appendix~\ref{programs}, Figure~\ref{whiteexample}. The
        data was generated by the {\it tsderivative}\/ program, which
        is briefly described in
        appendix~\ref{programs}. Figure~\ref{\SETLABEL:TF2} is the
        normalized histogram of the second derivative of the
        normalized increments of the time series data shown in
        Figure~\ref{\SETLABEL:TF}. In principle, if the distribution
        of the normalized increments presented in
        Figure~\ref{\SETLABEL:NH} is an integrated Gaussian
        distribution in nature, this distribution would be similar to
        ``white noise,'' as presented in appendix~\ref{programs},
        Figure~\ref{whiteexample}.

        \begin{figure}[ht]
            \begin{center}
                \begin{minipage}[t]{0.45\textwidth}
                    \epsfxsize=1.0\linewidth
                    \epsffile{\directory/data.tsfraction.tsderivative.tsnormal-s30.eps}
                    \caption[{\market}, histogram of the first
                        derivative of the increments]{{\market},
                        normalized histogram of the first derivative
                        of the normalized increments of the time
                        series data shown in
                        Figure~\ref{\SETLABEL:TF}.}
                    \label{\SETLABEL:TF1}
                \end{minipage}
                \hfill
                \begin{minipage}[t]{0.45\textwidth}
                    \epsfxsize=1.0\linewidth
                    \epsffile{\directory/data.tsfraction.2tsderivative.tsnormal-s30.eps}
                    \caption[{\market}, histogram of the second
                        derivative of the increments]{{\market},
                        normalized histogram of second derivative of
                        the the normalized increments of the time
                        series data shown in
                        Figure~\ref{\SETLABEL:TF}.}
                    \label{\SETLABEL:TF2}
                \end{minipage}
            \end{center}
        \end{figure}

        \subidx{fractal}{range}
        \subidx{fractal}{R/S analysis}
        \subidx{\market}{rate of revenue returns, range}
        \subidx{\market}{deterministic mechanism}
        \subidx{deterministic}{mechanism}
        \subidx{mechanism}{deterministic}
        Figure~\ref{\SETLABEL:TR} is the range of values of the time
        series shown in Figure~\ref{\SETLABEL:TS}. The horizontal axis
        is time into the future. In principle, if the time series was
        characterized as fractional Brownian motion the graph in
        Figure~\ref{\SETLABEL:TR} would be a square root
        function\footnote{Note that the ``roughness,'' or ``sawtooth''
        characteristics of the graph in Figure~\ref{\SETLABEL:TR} are
        a computational artifact---caused by not using the -m option
        to the program {\it tshurst}\/, which is computationally
        inefficient.}. Figure~\ref{\SETLABEL:TD} is the deterministic
        map of the normalized increments of the time series data shown
        in Figure~\ref{\SETLABEL:TF}. The deterministic map is useful
        for determining if a time series was created by a
        deterministic mechanism. This, essentially, maps each element
        in the time series with the previous element in the time
        series.  See,~\cite[pp. 745]{Peitgen}.

        \begin{figure}[ht]
            \begin{center}
                \begin{minipage}[t]{0.45\textwidth}
                    \epsfxsize=1.0\linewidth
                    \epsffile{\directory/data.tshurst-f.eps}
                    \caption[{\market}, range]{{\market}, range of the
                        time series data shown in
                        Figure~\ref{\SETLABEL:TS}.}
                    \label{\SETLABEL:TR}
                \end{minipage}
                \hfill
                \begin{minipage}[t]{0.45\textwidth}
                    \epsfxsize=1.0\linewidth
                    \epsffile{\directory/data.tsfraction.tsdeterministic.eps}
                    \caption[{\market}, deterministic map]{{\market},
                        deterministic map of the normalized increments
                        of the time series data shown in
                        Figure~\ref{\SETLABEL:TF}.}
                    \label{\SETLABEL:TD}
                \end{minipage}
            \end{center}
        \end{figure}

% Local Variables:
% TeX-parse-self: t
% TeX-auto-save: t
% TeX-master: "fractal.tex"
% End:


            Figure~\ref{\SETLABEL:NH} would seem to indicate that the
            time series data for the {\market} represents a cumulative
            sum/integration of a random process that has a Gaussian
            distribution, (ie., satisfies the Gaussian increments
            property of fractional Brownian
            motion~\cite[pp. 250]{Crownover},) tending to justify the
            assumption that the time series data represents fractional
            Brownian motion.

        %
% -----------------------------------------------------------------------------
%
% A license is hereby granted to reproduce this software source code and
% to create executable versions from this source code for personal,
% non-commercial use.  The copyright notice included with the software
% must be maintained in all copies produced.
%
% THIS PROGRAM IS PROVIDED "AS IS". THE AUTHOR PROVIDES NO WARRANTIES
% WHATSOEVER, EXPRESSED OR IMPLIED, INCLUDING WARRANTIES OF
% MERCHANTABILITY, TITLE, OR FITNESS FOR ANY PARTICULAR PURPOSE.  THE
% AUTHOR DOES NOT WARRANT THAT USE OF THIS PROGRAM DOES NOT INFRINGE THE
% INTELLECTUAL PROPERTY RIGHTS OF ANY THIRD PARTY IN ANY COUNTRY.
%
% Copyright (c) 1994-2006, John Conover, All Rights Reserved.
%
% Comments and/or bug reports should be addressed to:
%
%     john@email.johncon.com (John Conover)
%
% -----------------------------------------------------------------------------
%
% Revision: \RCSRevision \\
% Revision Time: \RCSTime UMT \\
% Revision Date: \RCSDate \\
% Revision Id: \RCSId \\
% Revision File: \RCSLog \\
\RCS $Revision: 0.0 $
\RCS $Date: 2006/01/20 04:38:13 $
\RCS $Id: instant.tex,v 0.0 2006/01/20 04:38:13 john Exp $
% $Log: instant.tex,v $
% Revision 0.0  2006/01/20 04:38:13  john
% Initial version
%
%
    \subsection{Instantaneous Analysis of Normalized Increments}
        \label{\SETLABEL:IA}

        \subidx{\market}{instantaneous analysis of normalized increments}
        \idx{average of normalized increments}
        \idx{root mean square of normalized increments}
        \subidx{Shannon probability}{instantaneous computation of}
        \subidx{average of normalized increments}{instantaneous computation of}
        \subidx{root mean square of normalized increments}{instantaneous computation of}
        \subidx{instantaneous computation}{Shannon probability}
        \subidx{instantaneous computation}{average of normalized increments}
        \subidx{instantaneous computation}{root mean square of normalized increments}
        \idx{time series}
        \subidx{time series}{instantaneous analysis}
        \subidx{instantaneous analysis}{time series}
        \subidx{time series}{increments}
        \subidx{time series}{analysis}
        \subidx{Shannon}{probability}
        \subidx{probability}{Shannon}
        \subidx{normalized}{increments}
        \subidx{increments}{normalized}

        The program {\it tsinstant}\/, which is briefly described in
        Appendix~\ref{programs}, is for finding the instantaneous
        fraction of change in a time series. The value of a sample in
        the time series is subtracted from the previous sample in the
        time series, and divided by the value of the previous sample.
        As explained in Chapter~\ref{general},
        Sections~\ref{derivation},~\ref{GA},~\ref{abmfi},~\ref{aftsma}
        and,~\ref{ompl} for Brownian motion, random walk fractals, the
        absolute value of the instantaneous fraction of change is also
        the root mean square of the instantaneous fraction of
        change\footnote{The absolute value of the normalized
        increments, when averaged, is related to the root mean square
        of the increments by a constant. If the normalized increments
        are a fixed increment, the constant is unity. If the
        normalized increments have a Gaussian distribution, the
        constant is $\approx 0.8$ depending on the accuracy of of
        ``fit'' to a Gaussian distribution.}. Squaring this value is
        the average of the instantaneous fraction of change, and
        adding unity to the absolute value of the instantaneous
        fraction of change, and dividing by two, is the Shannon
        probability of the instantaneous fraction of change.

        Figure~\ref{\SETLABEL:IA1} is the instantaneous value of the
        root mean square of the normalized increments for the
        {\market}, and Figure~\ref{\SETLABEL:IA2} is the instantaneous
        Shannon probability for the normalized increments.

        \begin{figure}[ht]
            \begin{center}
                \begin{minipage}[t]{0.45\textwidth}
                    \epsfxsize=1.0\linewidth
                    \epsffile{\directory/data.tsinstant-r.eps}
                    \caption[{\market}, instantaneous value of
                        rms.]{{\market}, instantaneous value of the
                        root mean square of the normalized increments,
                        provided by running the program {\it
                        tsinstant}\/ with the -r option on the data
                        presented in Figure~\ref{\SETLABEL:TS}.}
                    \label{\SETLABEL:IA1}
                    \label{\SETLABELQ:IA1}
                \end{minipage}
                \hfill
                \begin{minipage}[t]{0.45\textwidth}
                    \epsfxsize=1.0\linewidth
                    \epsffile{\directory/data.tsinstant-s.eps}
                    \caption[{\market}, instantaneous value of
                        Shannon probability.]{{\market}, instantaneous
                        value of the Shannon probability of the
                        normalized increments, provided by running the
                        program {\it tsinstant}\/ with the -s option
                        on the data presented in
                        Figure~\ref{\SETLABEL:TS}.}
                    \label{\SETLABEL:IA2}
                    \label{\SETLABELQ:IA2}
                \end{minipage}
            \end{center}
        \end{figure}

% Local Variables:
% TeX-parse-self: t
% TeX-auto-save: t
% TeX-master: "fractal.tex"
% End:


        %
% -----------------------------------------------------------------------------
%
% A license is hereby granted to reproduce this software source code and
% to create executable versions from this source code for personal,
% non-commercial use.  The copyright notice included with the software
% must be maintained in all copies produced.
%
% THIS PROGRAM IS PROVIDED "AS IS". THE AUTHOR PROVIDES NO WARRANTIES
% WHATSOEVER, EXPRESSED OR IMPLIED, INCLUDING WARRANTIES OF
% MERCHANTABILITY, TITLE, OR FITNESS FOR ANY PARTICULAR PURPOSE.  THE
% AUTHOR DOES NOT WARRANT THAT USE OF THIS PROGRAM DOES NOT INFRINGE THE
% INTELLECTUAL PROPERTY RIGHTS OF ANY THIRD PARTY IN ANY COUNTRY.
%
% Copyright (c) 1994-2006, John Conover, All Rights Reserved.
%
% Comments and/or bug reports should be addressed to:
%
%     john@email.johncon.com (John Conover)
%
% -----------------------------------------------------------------------------
%
% Revision: \RCSRevision \\
% Revision Time: \RCSTime UMT \\
% Revision Date: \RCSDate \\
% Revision Id: \RCSId \\
% Revision File: \RCSLog \\
\RCS $Revision: 0.0 $
\RCS $Date: 2006/01/20 04:38:13 $
\RCS $Id: logistic.tex,v 0.0 2006/01/20 04:38:13 john Exp $
% $Log: logistic.tex,v $
% Revision 0.0  2006/01/20 04:38:13  john
% Initial version
%
%
    \subsection{Logistic Analysis}
        \label{\SETLABEL:LA}

        \subidx{\market}{Logistic function analysis}
        \subidx{time series}{logistic function}
        \subidx{logistic function}{time series}
        \subidx{time series}{increments}
        \subidx{time series}{analysis}
        \subidx{cumulative sum}{analysis}
        \subidx{analysis}{cumulative sum}
        \subidx{analysis}{random process}
        \subidx{random process}{analysis}
        The data in this section is presented in tabular form in
        Section~\ref{\SETLABELREF:LAA}.  Figure~\ref{\SETLABEL:LA1} is
        a graph of the logistic function estimates of the time series
        data for the {\market}. The reader is cautioned that these
        graphs are constructed using the method suggested in
        Chapter~\ref{general}, Section~\ref{nlextend} and enormous
        precision is required for adequate prediction of the logistic
        function,~\cite{Modis}. Particularly, the non-linear term will
        usually require intervention to produce a practical fit to the
        data. In addition, there are numerical stability issues with
        logistic function methodologies\footnote{For example, in
        Figures~\ref{\SETLABEL:LA1} and~\ref{\SETLABEL:LA2}, if the
        non-linear term, $b$, was greater than zero, it was set to
        zero to produce the graphs. See Section~\ref{\SETLABELREF:LAA}
        for the actual derived values. In other cases, the magnitude
        of $b$ was too large, resulting in a graph that was decreasing
        as a function of time}.  The methodology should be regarded as
        ``fragile.'' It is included for completeness.

        \idx{least squares approximation}
        Figure~\ref{\SETLABEL:LA1} is a graph of the logistic function
        for the time series data presented in
        Figure~\ref{\SETLABEL:TS}. The data presented was made by
        running the program {\it tsdlogistic}\/, which is described
        briefly in Appendix~\ref{programs}, on the parameters
        extracted from the time series data as suggested in
        Figure~\ref{\SETLABEL:TF}. The program {\it tslsq}\/ was used
        to derive the constant and the slope of the normalized
        increments of the data presented in Figure~\ref{\SETLABEL:TF}.
        Figure~\ref{\SETLABEL:LA2} is the same graph, but with the
        time scale expanded by a factor of two.

        \begin{figure}[ht]
            \begin{center}
                \begin{minipage}[t]{0.45\textwidth}
                    \epsfxsize=1.0\linewidth
                    \epsffile{\directory/data.tsfraction.tslsq-p.tsdlogistic.eps}
                    \caption[{\market}, logistic function
                        estimates.]{{\market}, logistic function
                        estimates, provided by running the {\it
                        tslsq}\/ program on the normalized increments
                        presented in Figure~\ref{\SETLABEL:TF} with
                        the -p option. These parameters were used as
                        arguments to the {\it tsdlogistic}\/ program.}
                    \label{\SETLABEL:LA1}
                    \label{\SETLABELQ:LA1}
                \end{minipage}
                \hfill
                \begin{minipage}[t]{0.45\textwidth}
                    \epsfxsize=1.0\linewidth
                    \epsffile{\directory/data.tsfraction.tslsq-p.tsdlogistic2.eps}
                    \caption[{\market}, logistic function
                        estimates.]{{\market}, logistic function
                        estimates of Figure~\ref{\SETLABEL:LA1} with
                        the time scale expanded by a factor of two.}
                    \label{\SETLABEL:LA2}
                    \label{\SETLABELQ:LA2}
                \end{minipage}
            \end{center}
        \end{figure}

% Local Variables:
% TeX-parse-self: t
% TeX-auto-save: t
% TeX-master: "fractal.tex"
% End:


        %
% -----------------------------------------------------------------------------
%
% A license is hereby granted to reproduce this software source code and
% to create executable versions from this source code for personal,
% non-commercial use.  The copyright notice included with the software
% must be maintained in all copies produced.
%
% THIS PROGRAM IS PROVIDED "AS IS". THE AUTHOR PROVIDES NO WARRANTIES
% WHATSOEVER, EXPRESSED OR IMPLIED, INCLUDING WARRANTIES OF
% MERCHANTABILITY, TITLE, OR FITNESS FOR ANY PARTICULAR PURPOSE.  THE
% AUTHOR DOES NOT WARRANT THAT USE OF THIS PROGRAM DOES NOT INFRINGE THE
% INTELLECTUAL PROPERTY RIGHTS OF ANY THIRD PARTY IN ANY COUNTRY.
%
% Copyright (c) 1994-2006, John Conover, All Rights Reserved.
%
% Comments and/or bug reports should be addressed to:
%
%     john@email.johncon.com (John Conover)
%
% -----------------------------------------------------------------------------
%
% Revision: \RCSRevision \\
% Revision Time: \RCSTime UMT \\
% Revision Date: \RCSDate \\
% Revision Id: \RCSId \\
% Revision File: \RCSLog \\
\RCS $Revision: 0.0 $
\RCS $Date: 2006/01/20 04:38:13 $
\RCS $Id: hurst.tex,v 0.0 2006/01/20 04:38:13 john Exp $
% $Log: hurst.tex,v $
% Revision 0.0  2006/01/20 04:38:13  john
% Initial version
%
%
    \subsection{Hurst Coefficient Analysis}
        \label{\SETLABEL:H}

        \subidx{\market}{Hurst coefficient analysis}
        \subidx{Hurst coefficient}{analysis}
        \subidx{increments}{normalized}
        \subidx{normalized}{increments}
        \subidx{programs}{tshurst}
        \subidx{tshurst}{program}
        The data in this section is presented in tabular form in
        Section~\ref{\SETLABELREF:HCHP}. Figure~\ref{\SETLABEL:HC} is
        a graph of the Hurst coefficient data time series data shown
        in Figure~\ref{\SETLABEL:TS}. The slope of the graph is the
        Hurst coefficient.  The data for this figure was produced by
        the program {\it tshurst}\/, which is described briefly in
        Appendix~\ref{programs}.

        \subidx{\market}{H parameter analysis}
        \subidx{H parameter}{analysis}
        \subidx{programs}{tshcalc}
        \subidx{tshcalc}{program}
        Figure~\ref{\SETLABEL:HP} is a graph of the H parameter data
        for the normalized increments of the time series data shown in
        Figure~\ref{\SETLABEL:TF}. The data for this figure was
        produced by the program {\it tshcalc}\/, which is described
        briefly in Appendix~\ref{programs}.

        \begin{figure}[ht]
            \begin{center}
                \begin{minipage}[t]{0.45\textwidth}
                    \epsfxsize=1.0\linewidth
                    \epsffile{\directory/data.tshurst.eps}
                    \caption[{\market}, Hurst coefficient data]{{\market},
                        Hurst coefficient data for the normalized
                        increments of the time series data shown in
                        Figure~\ref{\SETLABEL:TF}.  The slope of the graph
                        is the Hurst coefficient.}
                    \label{\SETLABEL:HC}
                \end{minipage}
                \hfill
                \begin{minipage}[t]{0.45\textwidth}
                    \epsfxsize=1.0\linewidth
                    \epsffile{\directory/data.tshcalc.eps}
                    \caption[{\market}, H parameter data]{{\market}, H
                        parameter data for the normalized increments of
                        the time series data shown in
                        Figure~\ref{\SETLABEL:TF} The slope of the graph
                        is the H parameter.}
                    \label{\SETLABEL:HP}
                \end{minipage}
            \end{center}
        \end{figure}

        \subidx{revenue}{See, rate of revenue returns}
        \subidx{returns}{See, rate of revenue returns}
        \subidx{\market}{revenues}
        \subidx{Hurst coefficient}{analysis}
        \subidx{\market}{Hurst coefficient analysis}
        \subidx{\market}{rate of change}
        \subidx{\market}{windows of opportunity}
        \subidx{rate of revenue returns}{forecast}
        \subidx{forecast}{rate of revenue returns}
        \idx{windows of opportunity}
        \subidx{programs}{tslsq}
        \subidx{tslsq}{program}

        The approximately linear slope of the graph in
        Figure~\ref{\SETLABEL:HC} implies that the variance of the
        rate of revenue returns, (per {\timescale},) in the {\market},
        $V(t_2 - t_1)$, over a period of time is proportional to the
        period of time raised to twice the Hurst
        coefficient~\cite[pp. 180]{Feder},~\cite[pp. 246]{Crownover}.
        This seems to be a quantitative statement concerning how fast,
        and to what degree, the rate of revenue returns' state of
        affairs can change over a period of time.  An additional
        implication, for Hurst coefficients sufficiently close to 0.5,
        is that the probability of the state of affairs repeating
        sometime in the future goes down with increasing
        time\footnote{It can be shown that the number of expected
        market ``high'' and ``low'' transitions, $N$, scales with the
        square root of time, or $N \propto \sqrt {t}$, meaning that
        the cumulative distribution of the probability, $P$, of the
        duration of a market's ``high'' or ``low'' exceeding a given
        time interval, $t$, is proportional to the reciprocal of the
        square root of the time interval, $P \propto 1 / \sqrt {t}$,
        (or, conversely, that the probability of the duration of a
        market's ``high'' or ``low'' exceeding a given time interval
        is proportional to the reciprocal of the time interval raised
        to the power $3 / 2$, ie., $P \propto 1 / t^{3 /
        2}$,~\cite[pp. 153]{Schroeder}. What this means is that a
        histogram of the ``zero free'' run-lengths of a market being
        ``high'' or ``low,'' over a long time, would have a $1 / t^{3
        / 2}$ characteristic.)}, $t$, $p(t) = erf (1/\sqrt{2t})$ which
        is approximately $1/\sqrt{t}$ for $t \gg
        1$~\cite[pp. 160]{Schroeder}. Figures~\ref{\SETLABEL:FN},
        and,~\ref{\SETLABEL:FF} compare methods of approximation of
        the ``forecastability'' of the rate of revenue returns in the
        {\market} for the near term and far term,
        respectively~\cite[pp. 83-84]{Peters:CAOITCM}\footnote{The
        author is not comfortable with Peters' interpretation. For
        example, if the algorithm explained
        in~\cite[pp. 82]{Peters:CAOITCM} is used on ``white noise''
        which, by definition, never has any correlations, the short
        term Hurst coefficient, and thus the ``forecastability,'' is
        still near unity---a bit of an enigma. This can be verified
        with the {\it tswhite}\/ and {\it tshurst}\/ programs, which
        are briefly described in Appendix~\ref{programs}.}.  This
        seems to be a quantitative statement concerning ``windows of
        opportunity'' in the rate of revenue returns, (per
        {\timescale}.)  The program {\it tslsq}\/ was used on the
        Hurst coefficient data, presented in
        Figure~\ref{\SETLABEL:HC}, to provide a least squares
        approximation to the Hurst coefficient. The superimposed least
        squares approximation with on original Hurst coefficient data
        is presented.  The time series data has a Hurst coefficient of
        {\thurstlow}, so that:

        \subidx{\market}{Hurst coefficient analysis}
        \begin{eqnarray}
            V\left(t_2 - t_1\right) & \propto & \left(t_2 - t_1\right)^{2 \cdot H}\\
            V\left(t_2 - t_1\right) & \propto & \left(t_2 - t_1\right)^{2 \cdot {\thurstlow}}\\
                                    & \propto & \left(t_2 - t_1\right)^{\thurstlowtwo}
            \label{\SETLABEL:V}
        \end{eqnarray}

        \subidx{fractional}{Brownian motion}
        \subidx{Brownian motion}{fractional}
        \idx{fractal}
        \noindent where $V(t_2 - t_1)$ is the variance of the
        increments of the rate of revenue returns, (per {\timescale},)
        over the time interval $t_2 -
        t_1$,~\cite[pp. 177]{Feder},~\cite[pp. 494]{Peitgen}. If $H >
        \frac{1}{2}$, then the time series is termed as being
        characterized by ``fractional Brownian
        motion~\cite[pp. 170]{Feder}.''

        \subidx{rate of revenue returns}{predictability}
        \subidx{rate of revenue returns}{forecastability}
        \subidx{rate of revenue returns}{consistency}
        \subidx{predictability}{rate of revenue returns}
        \subidx{forecastability}{rate of revenue returns}
        \subidx{consistency}{rate of revenue returns}
        \subidx{\market}{rate of revenue returns, predictability}
        \subidx{\market}{rate of revenue returns, forecastability}
        \subidx{\market}{rate of revenue returns, consistency}
        \subidx{Hurst coefficient}{analysis}
        \subidx{\market}{Hurst coefficient analysis}
        \subidx{\market}{rate of change}

        In some sense, the Hurst coefficient is a quantitative
        expression of the ``forecastability'' of the future based on
        the past\footnote{Actually, in general, when summing fractal
        entities, the method used should be a root mean square
        process, dependent on the Hurst Coefficient, $H$, where
        $P_{total}^H = P_1^H + P_2^H + \cdots$, where $P_n$ is the
        fractal entities. For a Brownian motion, or random walk type
        of fractal the Hurst Coefficient is a function of time into
        the future. For the ``near term,'' the Hurst coefficient is
        very near unity, meaning the summation process is linear. For
        the ``long term,'' $H \approx 0.5$, or a standard root mean
        square summation process should be used. If $H$ is $0.5$ then
        the market is termed a Brownian motion, or random walk
        process. If it is larger than 0.5, it is termed fractional
        Brownian motion process. For a random walk process, ``near
        term'' and ``far term'' are quantitatively differentiated on
        the Hurst Coefficient graph where $1 - \ln (t) = 0.5 \cdot \ln
        (t)$, or when $\ln (t) = 2$, or $t = 7.389\ldots$ See
        Section~\ref{\SETLABEL:FS} for the particulars on using Hurst
        Coefficient to sum fractal process' for the {\market}. See
        also~\cite[pp. 67, 83-84]{Peters:CAOITCM} and~\cite[pp. 129,
        159]{Schroeder} for particulars on the implications of the
        Hurst Coefficient and root mean square summation issues.}.  A
        Hurst coefficient of {\thurstlow}, (for the near future, and
        {\thurstall} for the distant future.) implies that the
        likelihood of the rate of revenue returns, (per {\timescale},)
        for any two consecutive {\timescale}s being the same is
        {\thurstlowhundred}\%~\cite[pp. 66]{Peters:CAOITCM} for the
        near future, and {\thurstall} for the distant
        future. Likewise, there is a {\thurstlowhundred}\% chance of
        the rate of revenue returns, (per {\timescale},) movements
        being the same in consecutive time periods---ie., if, in a
        given {\timescale}, the rate of revenue returns, (per
        {\timescale},) is increasing, there is a {\thurstlowhundred}\%
        that the rate of revenue returns, (per {\timescale},) will
        increase in the following period, also. In some sense, this is
        a quantitative statement on how ``predictable,'' or
        ``forecastable'' the rate of revenue returns, (per
        {\timescale},) for the {\market} are over time, since the
        probability of having $n$ many consecutive {\timescale}s of
        the same agenda is $H^n$ where $H$ is the Hurst coefficient,
        or, letting the short term probability of having $n$ many
        {\timescale}s of the same market agenda, $p_a$, is:

        \begin{eqnarray}
            p_a\left(n\right) & = & H^{n}\\
                              & = & {\thurstlow}^{n}
            \label{\SETLABEL:MA}
        \end{eqnarray}

        \subidx{rate of revenue returns}{predictability}
        \subidx{rate of revenue returns}{forecastability}
        \subidx{rate of revenue returns}{consistency}
        \subidx{predictability}{rate of revenue returns}
        \subidx{forecastability}{rate of revenue returns}
        \subidx{consistency}{rate of revenue returns}
        As an interesting interpretation of the normalized increments
        of the time series data presented in
        Figure~\ref{\SETLABEL:TF}, if the vertical axis is multiplied
        by 100, to convert to percent, then the graph represents the
        error, in percent, that would be made by forecasting, month by
        month, that the next {\timescale}'s rate of revenue returns
        would be the same as the current {\timescale}'s revenue
        rate. Interestingly, it is $\datafractionmean \cdot 100$
        percent, on the average, with a standard deviation of
        $\datafractionstddev \cdot 100$ percent, and a root mean
        square error value of $\datafractionrms \cdot 100$
        percent---small values for such a simple forecasting
        mechanism.

        \subidx{\market}{rate of revenue returns, range}
        \subidx{Hurst coefficient}{analysis}
        \subidx{\market}{Hurst coefficient analysis}
        \subidx{\market}{rate of change}

        This is, essentially, a statement of the range of values, in
        the increments of the rate of revenue returns, (per
        {\timescale},) that is to be expected over the time interval,
        $t_2 - t_1$,
        $R_v$,~\cite[pp. 178]{Feder},~\cite[pp. 172]{Cambel}:

        \begin{eqnarray}
            R_v\left(t_2 - t_1\right) & \propto & \left(t_2 - t_1\right)^{H}\\
                                      & \propto & \left(t_2 - t_1\right)^{\thurstlow}
            \label{\SETLABEL:R}
        \end{eqnarray}

        \subidx{\market}{rate of revenue returns, range}
        \subidx{Hurst coefficient}{analysis}
        \subidx{\market}{Hurst coefficient analysis}
        \subidx{\market}{rate of change}
        \subidx{Markov}{statistics}
        \subidx{statistics}{Markov}
        \noindent where $R$ is the range of values in the increments
        of the rate of revenue returns, (per {\timescale}.) A Hurst
        coefficient, $H$, that is much larger than $\frac{1}{2}$, (but
        less than 1,) implies a strongly non-Gaussian distribution in
        the increments of the rate of revenue returns, (per
        {\timescale},)~\cite[pp. 152, 194]{Feder}, and a Hurst
        coefficient near $\frac{1}{2}$ implies that the increments of
        the rate of revenue returns, (per {\timescale}) is
        characteristic of an independent
        process~\cite[pp. 195]{Feder}. Extreme caution should be
        exercised in using Markov statistics in any analysis where the
        Hurst coefficient is not
        $\frac{1}{2}$,~\cite[pp. 124]{Crownover},~\cite[pp. 106]{Peters:CAOITCM}.


        As a useful approximation, if $H$, is approximately
        $\frac{1}{2}$, Equation~\ref{\SETLABEL:R} reduces
        to,~\cite[pp. 129]{Schroeder}:

        \begin{eqnarray}
            R\left(t_2 - t_1\right) & \propto & (t_2 - t_1)^{\frac{1}{2}}\\
                                    & \propto & \sqrt{\left(t_2 - t_1\right)}
        \end{eqnarray}

        \subidx{\market}{rate of revenue returns, range}
        \subidx{\market}{rate of revenue returns, increase and decrease}
        \subidx{Hurst coefficient}{analysis}
        \subidx{\market}{Hurst coefficient analysis}
        \subidx{\market}{rate of change}
        \subidx{Markov}{statistics}
        \subidx{statistics}{Markov}

        In the case where the Hurst coefficient, $H$, is
        $\frac{1}{2}$, the range of values in the increments of the
        rate of revenue returns, (per {\timescale},) divided by the
        standard deviation of these values, $S$, can be anticipated to
        increase over time according to the following
        relation,~\cite[pp. 154]{Feder},~\cite[pp. 129]{Schroeder}:

        \begin{equation}
            \frac{R\left(t_2 - t_1\right)}{S} \propto \left(t_2 - t_1\right)^{\frac{1}{2}}
        \end{equation}

        \subidx{\market}{rate of revenue returns, range}
        \subidx{\market}{rate of revenue returns, increase and decrease}
        \subidx{Hurst coefficient}{analysis}
        \subidx{\market}{Hurst coefficient analysis}
        \subidx{\market}{rate of change}
        \noindent which is a useful conceptual approximation, since it
        involves only the square root function---if the range and the
        standard deviation of the increments of the rate of revenue
        returns, (per {\timescale},) are known, (and $H \approx
        \frac{1}{2}$,) then the expected change in $\frac{R}{S}$, will
        increase with the square root of time\footnote{To be precise,
        it is actually asymptotically proportional to
        $\tau^{\frac{1}{2}}$}.

        Another useful approximation when rescaling processes that are
        characterize by Brownian motion, (ie., when $H \approx
        \frac{1}{2}$,) is that:

        \begin{eqnarray}
            X\left(t\right) & \propto & \frac{X\left(rt\right)}{r^{H}}\\
                            & \propto & \frac{X\left(rt\right)}{r^{\thurstlow}}
        \end{eqnarray}

        \idx{Brownian motion}
        \idx{fractal}
        Where $X(t)$ is the process characterized by Brownian motion,
        and $r$ is a scaling factor,~\cite[pp. 494]{Peitgen}.

        \subidx{programs}{tslsq}
        \subidx{tslsq}{program}
        The program {\it tslsq}\/ was used on the H parameter data,
        presented in Figure~\ref{\SETLABEL:HP}, to provide a least
        squares approximation to the H parameter for the
        {\market}. The superimposed least squares approximation on the
        original H parameter data is presented.  By contrast, the H
        parameter, as derived by the methodology outlined
        in~\cite[pp. 249]{Crownover}, is {\thcalclow} for the near
        future, and {\thcalcall} for the distant future.

        \subidx{\market}{Hurst coefficient analysis}
        \subidx{Hurst coefficient}{analysis}
        \subidx{increments}{normalized}
        \subidx{normalized}{increments}
        \subidx{programs}{tshurst}
        \subidx{tshurst}{program}
        \subidx{\market}{H parameter analysis}
        \subidx{H parameter}{analysis}
        \subidx{programs}{tshcalc}
        \subidx{tshcalc}{program}
        Figures~\ref{\SETLABEL:HC} and~\ref{\SETLABEL:HP} represent
        Hurst coefficient and H parameter data that are derived from
        the normalized increments, shown in
        Figure~\ref{\SETLABEL:TF}. In this case, the data is
        considered a normalized derivative of the time series data
        presented in Figure~\ref{\SETLABEL:TF}, instead of a
        cumulative sum.  The program, {\it tshurst}\/, is described
        briefly in appendix~\ref{programs}, and the data for
        figures~\ref{\SETLABEL:THC} and~\ref{\SETLABEL:THP} was made
        using the -d option.

        \begin{figure}[ht]
            \begin{center}
                \begin{minipage}[t]{0.45\textwidth}
                    \epsfxsize=1.0\linewidth
                    \epsffile{\directory/data.tsfraction.tshurst-d.eps}
                    \caption[{\market}, traditional Hurst coefficient
                        data]{{\market}, traditional Hurst coefficient
                        data for the time series data shown in
                        Figure~\ref{\SETLABEL:TS}.  The slope of the
                        graph is the Hurst coefficient, and is
                        {\hurstlow} for the near term, and
                        {\hurstall} for the far term.}
                    \label{\SETLABEL:THC}
                \end{minipage}
                \hfill
                \begin{minipage}[t]{0.45\textwidth}
                    \epsfxsize=1.0\linewidth
                    \epsffile{\directory/data.tsfraction.tshcalc-d.eps}
                    \caption[{\market}, traditional H parameter
                        data]{{\market}, traditional H parameter data
                        for the time series data shown in
                        Figure~\ref{\SETLABEL:TS} The slope of the
                        graph is the H parameter, and is {\hcalclow}
                        for the near term, and {\hcalcall} for the
                        far term.}
                    \label{\SETLABEL:THP}
                \end{minipage}
            \end{center}
        \end{figure}

% Local Variables:
% TeX-parse-self: t
% TeX-auto-save: t
% TeX-master: "fractal.tex"
% End:


        %
% -----------------------------------------------------------------------------
%
% A license is hereby granted to reproduce this software source code and
% to create executable versions from this source code for personal,
% non-commercial use.  The copyright notice included with the software
% must be maintained in all copies produced.
%
% THIS PROGRAM IS PROVIDED "AS IS". THE AUTHOR PROVIDES NO WARRANTIES
% WHATSOEVER, EXPRESSED OR IMPLIED, INCLUDING WARRANTIES OF
% MERCHANTABILITY, TITLE, OR FITNESS FOR ANY PARTICULAR PURPOSE.  THE
% AUTHOR DOES NOT WARRANT THAT USE OF THIS PROGRAM DOES NOT INFRINGE THE
% INTELLECTUAL PROPERTY RIGHTS OF ANY THIRD PARTY IN ANY COUNTRY.
%
% Copyright (c) 1994-2006, John Conover, All Rights Reserved.
%
% Comments and/or bug reports should be addressed to:
%
%     john@email.johncon.com (John Conover)
%
% -----------------------------------------------------------------------------
%
% Revision: \RCSRevision \\
% Revision Time: \RCSTime UMT \\
% Revision Date: \RCSDate \\
% Revision Id: \RCSId \\
% Revision File: \RCSLog \\
\RCS $Revision: 0.0 $
\RCS $Date: 2006/01/20 04:38:13 $
\RCS $Id: fiscal.tex,v 0.0 2006/01/20 04:38:13 john Exp $
% $Log: fiscal.tex,v $
% Revision 0.0  2006/01/20 04:38:13  john
% Initial version
%
%
    \subsection{Fixed Increment Approximation for Fiscal Strategy}
        \label{\SETLABEL:FS}

        \subidx{\market}{fiscal strategy}
        \subidx{markets}{analysis}
        \subidx{analysis}{markets}
        \subidx{strategy}{fiscal}
        \subidx{fiscal}{strategy}
        The data in this section is presented in tabular form in
        Section~\ref{\SETLABELREF:LR}. This section derives various
        values based on the ``average'' of the normalized increments
        presented in Figure~\ref{\SETLABEL:TFA}. These values are an
        approximation to a, probably, complex process with a
        distribution shown in Figure~\ref{\SETLABEL:TF}. These values
        will be used in a fixed increment Brownian fractal analysis
        and simulation of the {\market}, and may, or may not, provide
        adequate accuracy for projections.

        For an organization operating in the {\market}, the fiscal
        strategy, commensurate with the aggregate environment, can be
        derived as follows~\cite[pp. 128, pp
        151]{Schroeder},~\cite[pp. 450]{Reza},~\cite[pp. 270]{Pierce}:
        \vspace{0.15in}

        \subsubsection{Logarithmic Returns}
            \label{\SETLABEL:LR}

            \subidx{logarithmic}{returns}
            \subidx{returns}{logarithmic}
            \subidx{\market}{logarithmic returns}
            The logarithmic returns can be calculated by various
            means. Four will be presented here, for comparison.

            \subidx{programs}{tsnormal}
            \subidx{tsnormal}{program}
            \subidx{logarithmic}{returns}
            \subidx{returns}{logarithmic}
            The logarithmic returns, in bits, $bits$, as computed from
            the mean, by the program {\it tsnormal}\/, which is
            described in Chapter~\ref{programs}, and is presented in
            Figure~\ref{\SETLABEL:TF}, and Equation~\ref{abits} from
            Section~\ref{ereturns} in Chapter~\ref{general}:

            \begin{equation}
                bits = \frac{\ln \left({\datafractionmean} + 1\right)}{\ln \left(2\right)} = \datafractionmeanbits
            \end{equation}

            \subidx{programs}{tslsq}
            \subidx{tslsq}{program}
            \subidx{logarithmic}{returns}
            \subidx{returns}{logarithmic}
            \noindent By comparison, the logarithmic returns, in bits,
            $bits$, as computed from the constant in the least squares
            approximation, using the program {\it tslsq}\/, which is briefly
            described in Chapter~\ref{programs}, as presented in
            Figure~\ref{\SETLABEL:TF}, and Equation~\ref{abits} from
            Section~\ref{ereturns} in Chapter~\ref{general}:

            \begin{equation}
                bits = \frac{\ln \left({\datafractionconstant} + 1\right)}{\ln \left(2\right)} = \datafractionconstantbits
            \end{equation}

            Note that if the mean is not constant in
            Figure~\ref{\SETLABEL:TF}, this method will not provide
            accurate results.

            \subidx{programs}{tslsq}
            \subidx{tslsq}{program}
            \subidx{logarithmic}{returns}
            \subidx{returns}{logarithmic}
            \noindent And by yet another comparison, using the program
            {\it tslsq}\/, which is briefly described in
            Chapter~\ref{programs}, with the -e -p options, to provide
            a formula for the least squares exponential fit to the
            time series data set presented in
            Figure~\ref{\SETLABEL:TS}:

            \begin{equation}
                bits = {\datatslsqepbits}
            \end{equation}

            \subidx{programs}{tslogreturns}
            \subidx{tslogreturns}{program}
            \subidx{logarithmic}{returns}
            \subidx{returns}{logarithmic}
            \noindent And finally, by comparison, from the
            {\it tslogreturns}\/ program, which is briefly described
            in Chapter~\ref{programs}, with the -p option, to provide
            a formula for the logarithmic returns of the time series
            data set presented in Figure~\ref{\SETLABEL:TS}:

            \begin{equation}
                bits = {\logreturns}
            \end{equation}

        \subsubsection{Calculation of Shannon Probability}
            \label{\SETLABEL:SP}

            \subidx{\market}{Shannon probability}
            Ideally, all of the values presented in
            Section~\ref{\SETLABEL:LR} would be equal. Using the
            logarithmic returns provided by the {\it tslogreturns}\/
            program, to be consistent
            with~\cite[pp. 81]{Peters:CAOITCM}

            \subidx{programs}{tslogreturns}
            \subidx{tslogreturns}{program}
            \begin{equation}
                2^{{\logreturns}t}
            \end{equation}

            \noindent therefore:
            \begin{equation}
                C\left(p\right) = {\logreturns}
            \end{equation}
            \subidx{programs}{tsshannon}
            \subidx{tsshannon}{program}
            \subidx{Shannon}{probability}
            \subidx{probability}{Shannon}
            \noindent and, {\it tsshannon}\/ {\logreturns} gives:
            \begin{equation}
                \label{\SETLABEL:F0}
                C\left({\shannonlogreturns}\right) = {\logreturns}
            \end{equation}
            \noindent therefore:
            \begin{eqnarray}
                2^{C\left({\shannonlogreturns}\right)} & = & 2^{\logreturns}\\
                                                       & = & {\twologreturns}\\
                                                       & = & {\twologreturnshundred}\%
            \end{eqnarray}
            \noindent and:
            \begin{eqnarray}
                2p - 1 & = & \left(2 \cdot {\shannonlogreturns}\right) - 1\\
                       & = & {\twopone}\\
                       \label{\SETLABEL:F1}
                       & = & {\twoponehundred}\%
            \end{eqnarray}

            \subidx{\market}{fiscal strategy}
            \subidx{markets}{analysis}
            \subidx{analysis}{markets}
            \subidx{strategy}{fiscal}
            \subidx{fiscal}{strategy}
            \subidx{\market}{fiscal strategy}
            \subidx{\market}{growth rate}
            Presuming the simplified assumptions outlined in
            Section~\ref{assumptions}, the ``typical'' organization
            operating in the {\market} executes a long term fiscal
            strategy, commensurate with the aggregate environment,
            that is to invest, every {\timescale}, in sufficient
            additional resources and infrastructure, to increase the
            manufacturing of goods and services by {\twoponehundred}\%
            of its rate of revenue returns, (per {\timescale}.) As a
            conceptual model, the remaining {\hundredtwoponehundred}\%
            will be held in ``reserve'' with a
            {\shannonlogreturnshundred}\% chance of making twice the
            {\twoponehundred}\% back, (and a
            {\hundredshannonlogreturnshundred}\% chance of making
            0.0,) in one {\timescale}, on the average, for an average
            growth in its rate of revenue returns, (per {\timescale},)
            of {\twologreturnshundred}\%, or a doubling of its rate of
            revenue returns, (per {\timescale},) in
            {\oneoverlogreturns} {\timescale}s.

        \subsubsection{Example Fixed Increment Approximation Fiscal Strategies}

            \subidx{\market}{fiscal strategy}
            \subidx{markets}{analysis}
            \subidx{analysis}{markets}
            \subidx{strategy}{fiscal}
            \subidx{fiscal}{strategy}
            \subidx{\market}{fiscal strategy}
            \subidx{\market}{growth rate}
            \subidx{\market}{management metric}
            \idx{management metric}
            A possible metric on the effectiveness of long term fiscal
            management could possibly be that if an investment of
            {\twoponehundred}\% per {\timescale} of the rate of
            revenue returns, (per {\timescale},) is made in resources
            and infrastructure, then the rate of revenue returns would
            be expected to increase by {\twologreturnshundred}\%, per
            {\timescale}, on average.

            Note that the metrics presented in this section are
            representative of the {\market} as an aggregate whole, and
            may or may not be accurate representations for any
            particular participant in the environment. Of interest to
            the participants in the environment would be a similar
            analysis of each product or service rendered in the
            marketplace.

            \subidx{\market}{fiscal strategy}
            \subidx{markets}{analysis}
            \subidx{analysis}{markets}
            \subidx{strategy}{fiscal}
            \subidx{fiscal}{strategy}
            \subidx{\market}{fiscal strategy}
            As a simple illustrative example, a company operating in
            this environment might obtain a credit line from a bank
            that is equal to {\twoponehundred}\% of its rate of
            revenue returns, (per {\timescale},) to finance additional
            operations. In this simple scenario, the company would use
            its revenue base as collateral for the loan. Some
            {\timescale}s, depending on the {\market}'s environment,
            the company's rate of revenue returns exceeds what was
            borrowed from the bank, and the loan is repaid in
            full. Other {\timescale}s, the company must default, and
            the bank seizes a portion of the company's revenue base to
            pay the delinquent loan. However, on the average, the
            company will expand its rate of revenue returns at
            {\twologreturnshundred}\% per {\timescale}.

            \subidx{\market}{fiscal strategy}
            \subidx{markets}{analysis}
            \subidx{analysis}{markets}
            \subidx{strategy}{fiscal}
            \subidx{fiscal}{strategy}
            \subidx{\market}{fiscal strategy}
            As another simple example, a company re-invests
            {\twoponehundred}\% of its rate of revenue returns, (per
            {\timescale},) in development, marketing, sales, and
            distribution of new products.  Although some products will
            be successful and the return on the investment will exceed
            the {\twoponehundred}\% per {\timescale} investment,
            others will not. However, on the average, the company will
            expand it gross rate of revenue returns at
            {\twologreturnshundred}\% per {\timescale}.

            \subidx{\market}{fiscal strategy}
            \subidx{markets}{analysis}
            \subidx{analysis}{markets}
            \subidx{strategy}{fiscal}
            \subidx{fiscal}{strategy}
            \subidx{\market}{fiscal strategy}
            \subidx{\market}{product portfolio}
            \subidx{\market}{product diversity}
            \subidx{\market}{product mix}
            \subidx{\market}{optimum number of products}
            \idx{product portfolio}
            \idx{product diversity}
            \idx{optimum number of products}
            \idx{product mix}

            As an example of ``product portfolio'' management, suppose
            a company re-invests {\twoponehundred}\% of its rate of
            revenue returns, (per {\timescale},) in development,
            marketing, sales, and distribution of new products.
            Further suppose that the company has two products, and a
            fractal analysis of the individual product rate of revenue
            return time series indicates that one product has a
            Shannon probability of 0.65, and the other has a Shannon
            probability of 0.55. Then the percentage of re-investment
            in the first product would be $(2 \cdot 0.65 - 1) \cdot
            {\twoponehundred}$, percent of the rate of revenue
            returns, and $(2 \cdot 0.55 - 1) \cdot {\twoponehundred}$
            percent for the second product, implying that the company
            should diversify its product line\footnote{The astute
            reader would note that the linear addition was used to add
            the contribution to development of each product. This is a
            ``near term'' interpretation. Actually, in general, the
            method used should be a root mean square process,
            dependent on the Hurst Coefficient, $H$, where
            $P_{total}^H = P_1^H + P_2^H + \cdots$, where $P_n$ is the
            contribution to each individual product. For a Brownian
            motion, or random walk type of fractal the Hurst
            Coefficient is a function of time into the future. For the
            ``near term,'' the Hurst coefficient is very near unity,
            meaning the summation process is linear. For the ``long
            term,'' $H \approx 0.5$, or a standard root mean square
            summation process should be used. If $H$ is $0.5$ then the
            market is termed a Brownian motion, or random walk
            process. If it is larger than 0.5, it is termed fractional
            Brownian motion process. For a random walk process, ``near
            term'' and ``far term'' are quantitatively differentiated
            on the Hurst Coefficient graph where $1 - \ln (t) = 0.5
            \cdot \ln (t)$, or when $\ln (t) = 2$, or $t =
            7.389\ldots$ See~\cite[pp. 67, 83-84]{Peters:CAOITCM}
            and~\cite[pp. 129, 159]{Schroeder} for particulars on the
            implications of the Hurst Coefficient and root mean square
            summation issues.}.  Note that this is a ``bet hedging''
            metric methodology, and assumes that the products have
            uncorrelated revenue return rates. If this re-investment
            methodology is not feasible, perhaps for strategic
            financial reasons, then the re-investment in both products
            should total the ${\twoponehundred}$\%, and the investment
            in each product should be made at a ratio of $\frac{(2
            \cdot 0.65 - 1)}{(2 \cdot 0.55 - 1)} = 3 : 1$,
            respectively. Note that this ``bet hedging'' can be used
            to define the optimal number of products that can be
            supported on the rate of revenue returns. If it assumed
            that all products are ``typical'' for the {\market}, as a
            standard bench mark, then the optimal number will be
            $\frac{1}{{\twopone}}$. Note that this is a
            ``theoretical'' value, since not all products are
            ``typical,'' and there may be strategic reasons, for
            example product leveraging, that may increase the number
            of products above the optimum. However, most of the
            revenue should come from the optimal number of products,
            since having more products will decrease the amount of the
            potential investment in each product, and having less than
            the optimum number of products will increase the risk that
            many of the products could suffer a ``down market''
            concurrently, impacting the rate of revenue returns.  As
            another interesting interpretation of the optimal
            ``hedging of bets,'' in product portfolio strategy, and
            considering the graph of the normalized increments
            presented in Figure~\ref{\SETLABEL:TF}, if the
            organization is running optimally, then these products
            will generate, at least in principle, one standard
            deviation, approximately $0.8413 = 84.13$\% of the future
            growth in rate of revenue returns. Naturally, these are
            approximations, and the values are an approximation to a,
            probably, complex process, and appropriate scrutiny should
            be exercised before making specific projections.  As yet
            another example of ``product portfolio'' management,
            consider the issue of product mix. In this interpretation,
            {\twoponehundred}\% of the product manufactured should be
            ``proprietary,'' while the rest is ``industry standard.''
            As yet another possibility, {\twoponehundred}\% of the
            product manufactured should be predatory into new markets,
            and the remainder in markets that are ``traditional'' for
            the company.

% Local Variables:
% TeX-parse-self: t
% TeX-auto-save: t
% TeX-master: "fractal.tex"
% End:


        %
% -----------------------------------------------------------------------------
%
% A license is hereby granted to reproduce this software source code and
% to create executable versions from this source code for personal,
% non-commercial use.  The copyright notice included with the software
% must be maintained in all copies produced.
%
% THIS PROGRAM IS PROVIDED "AS IS". THE AUTHOR PROVIDES NO WARRANTIES
% WHATSOEVER, EXPRESSED OR IMPLIED, INCLUDING WARRANTIES OF
% MERCHANTABILITY, TITLE, OR FITNESS FOR ANY PARTICULAR PURPOSE.  THE
% AUTHOR DOES NOT WARRANT THAT USE OF THIS PROGRAM DOES NOT INFRINGE THE
% INTELLECTUAL PROPERTY RIGHTS OF ANY THIRD PARTY IN ANY COUNTRY.
%
% Copyright (c) 1994-2006, John Conover, All Rights Reserved.
%
% Comments and/or bug reports should be addressed to:
%
%     john@email.johncon.com (John Conover)
%
% -----------------------------------------------------------------------------
%
% Revision: \RCSRevision \\
% Revision Time: \RCSTime UMT \\
% Revision Date: \RCSDate \\
% Revision Id: \RCSId \\
% Revision File: \RCSLog \\
\RCS $Revision: 0.0 $
\RCS $Date: 2006/01/20 04:38:13 $
\RCS $Id: companies.tex,v 0.0 2006/01/20 04:38:13 john Exp $
% $Log: companies.tex,v $
% Revision 0.0  2006/01/20 04:38:13  john
% Initial version
%
%
    \subsection{Number of Companies}
        \label{\SETLABEL:QNC}

        \subidx{\market}{number of companies}
        \subidx{number of companies}{analysis}
        \subidx{analysis}{number of companies}
        \subidx{Shannon}{probability}
        \subidx{probability}{Shannon}
        This section evaluates the approximate, or ``average,'' number
        of companies in the {\market}, and uses the method outlined in
        Chapter~\ref{general}, Section~\ref{aftsma}. Since the
        average, $avg_{ind}$, and the root mean square, $rms_{ind}$,
        of the normalized increments of the {\market} time series is
        \datafractionmean, and \datafractionrms respectively, the
        number of companies participating in the market can be
        calculated by Equation~\ref{ncompanies} to be {\ncompanies}.

        If this value seems consistent number of companies in the
        {\market}, within the assumptions outlined in
        Chapter~\ref{general}, Section~\ref{aftsma}, then it would
        seem that there is some circumstantial or indirect evidence
        that the companies participating in the {\market} are
        operating optimally, and the ``average'' Shannon probability,
        $P$ for each participating company would be, using
        Equation~\ref{pncompanies}, {\pncompanies}, which would be the
        value which should be used in Section~\ref{\SETLABEL:FS} for
        each participating company if market expansion was to be
        consistent with the rest of the industry. However, if the
        Shannon probability derived in Section~\ref{\SETLABEL:FS} is
        greater than the average Shannon probability for the companies
        participating in the {\market}, as derived in this section,
        then the market would, possibly, be exploitable with the
        fiscal strategy outlined in Section~\ref{\SETLABEL:FS}. The
        maximum exploitability for the {\market} is derived in
        Section~\ref{\SETLABEL:MAXSHANNON}, but it is probably of
        doubtful practicality.

        Note that these optimizations would maximize a company's
        market growth. Since there are probably many companies
        competing in the market place, this would not necessarily
        maximize a company's P\&L, as described in
        Chapter~\ref{general}, Section~\ref{ompl}. The Shannon
        probability that maximizes market share in the {\market} is
        \pncompanies, with several alternative solutions listed in the
        previous paragraph. However, these should be contrasted to the
        Shannon probability that maximizes a company's P\&L which is
        \avgrms~in the {\market}. In all cases, the fraction of the
        P\&L that should be ``wagered'' on the future, $f$, should be:

        \begin{equation}
            f = 2P - 1
        \end{equation}

        \noindent where $P$ is the particular Shannon probability
        chosen optimize a particular fiscal strategy. Interestingly,
        the measured Shannon probability of the {\market} would tend
        to indicate that the companies participating in the market
        have chosen a fiscal strategy that optimizes market growth, as
        opposed to capital growth.

        \subidx{\market}{increasing returns}
        \subidx{economic increasing returns}{\market}
        As interesting interpretation of these exploitive issues,
        since all three fiscal strategies will result in exponential
        market growth for every company participating in the market,
        is that they may represent, perhaps, an example of
        ``increasing returns.''

% Local Variables:
% TeX-parse-self: t
% TeX-auto-save: t
% TeX-master: "fractal.tex"
% End:


        %
% -----------------------------------------------------------------------------
%
% A license is hereby granted to reproduce this software source code and
% to create executable versions from this source code for personal,
% non-commercial use.  The copyright notice included with the software
% must be maintained in all copies produced.
%
% THIS PROGRAM IS PROVIDED "AS IS". THE AUTHOR PROVIDES NO WARRANTIES
% WHATSOEVER, EXPRESSED OR IMPLIED, INCLUDING WARRANTIES OF
% MERCHANTABILITY, TITLE, OR FITNESS FOR ANY PARTICULAR PURPOSE.  THE
% AUTHOR DOES NOT WARRANT THAT USE OF THIS PROGRAM DOES NOT INFRINGE THE
% INTELLECTUAL PROPERTY RIGHTS OF ANY THIRD PARTY IN ANY COUNTRY.
%
% Copyright (c) 1994-2006, John Conover, All Rights Reserved.
%
% Comments and/or bug reports should be addressed to:
%
%     john@email.johncon.com (John Conover)
%
% -----------------------------------------------------------------------------
%
% Revision: \RCSRevision \\
% Revision Time: \RCSTime UMT \\
% Revision Date: \RCSDate \\
% Revision Id: \RCSId \\
% Revision File: \RCSLog \\
\RCS $Revision: 0.0 $
\RCS $Date: 2006/01/20 04:38:13 $
\RCS $Id: operations.tex,v 0.0 2006/01/20 04:38:13 john Exp $
% $Log: operations.tex,v $
% Revision 0.0  2006/01/20 04:38:13  john
% Initial version
%
%
    \subsection{Fixed Increment Approximation for Operational Strategy}
        \label{\SETLABEL:OPS}.

        This section derives various values based on the ``average''
        of the normalized increments presented in
        Figure~\ref{\SETLABEL:TFA}. These values are an approximation
        to a, probably, complex process with a distribution shown in
        Figure~\ref{\SETLABEL:TF}. These values will be used in a
        fixed increment Brownian fractal analysis and simulation of
        the {\market}, and may, or may not, provide adequate accuracy
        for projections.

        \subidx{\market}{fiscal strategy}
        \subidx{\market}{Shannon probability}
        \subidx{strategy}{fiscal}
        \subidx{fiscal}{strategy}
        \subidx{Shannon}{probability}
        \subidx{probability}{Shannon}
        It should be noted that the analysis of fiscal strategy,
        presented in Section~\ref{\SETLABEL:FS}, is derived from the
        {\market} metrics and may, or may not, be maximally
        optimal. For the optimal fiscal strategy, which may be
        exploitable, see Section~\ref{\SETLABEL:MAXSHANNON}.

        \subidx{strategy}{exploitable}
        \subidx{exploitable}{strategy}
        \subidx{\market}{windows of opportunity}
        \idx{windows of opportunity}
        \subidx{decision}{obsolete}
        \subidx{obsolete}{decision}
        \subidx{decision}{timeliness}
        \subidx{timeliness}{decision}
        \subidx{rate of revenue returns}{forecast}
        \subidx{forecast}{rate of revenue returns}
        An additional exploitable strategy may be time itself.
        Equations~\ref{\SETLABEL:V},~\ref{\SETLABEL:R},
        and,~\ref{\SETLABEL:MA}, are, essentially, metrics on how fast
        a decision, which is based on information concerning the
        current status of the {\market}, becomes obsolete. Obviously,
        how long a decision is expected to remain relevant should be
        addressed as an operational necessity in strategic planning
        and project management. Figures~\ref{\SETLABEL:FN},
        and,~\ref{\SETLABEL:FF} compare methods of approximation of
        the ``forecastability'' of rate of revenue returns in the
        {\market} for the near term and far
        term~\cite[pp. 83-84]{Peters:CAOITCM}, respectively. As a
        general rule, caution must be exercised when making decisions
        that will span a time interval larger than the time interval
        where the ``forecastability'' of rate of revenue returns drops
        below 50\%. Beyond this time interval, the chances increase
        that the competitive and market forces will alter the market
        environment in a possibly detrimental unanticipated
        fashion. Obviously, there is significant advantage in
        ``timeliness'' of development, manufacturing, and distribution
        of products and services that are consistent with this
        temporal agenda. Automation of these processes, if executed
        consistently with this agenda, should be considered a
        competitive advantage.

        \subidx{strategy}{exploitable}
        \subidx{exploitable}{strategy}
        \subidx{rate of revenue returns}{forecast}
        \subidx{forecast}{rate of revenue returns}
        \idx{product life cycle}
        \idx{life cycle, product}
        In some sense, this temporal agenda defines the ``average''
        product or service life cycle in the {\market}. When the
        ``forecastability'' of rate of revenue returns drops below
        50\%, there is an even chance that the rate of revenue returns
        for the product or service will change in a detrimental
        fashion. If it is assumed that a product or service life cycle
        consists of a ramp up, a maintenence interval, and a ramp
        down, then, if all three life cycle intervals are equal, the
        product life cycle will be, approximately, three times the
        time interval where the ``forecastability'' of rate of revenue
        returns drops below 50\%. Although probably not an accurate
        prediction of product or service life cycle, the technique may
        be used as a conceptual approximation to the dynamics of
        ``market windows.\footnote{For example, consider the market
        for table salt. Since it has inelastic supply and demand
        curves, and is a necessary requirement for life, it would be
        expected that the Hurst coefficient would be very near
        unity---ignoring competitive pressures in the market. The
        predictability of the table salt market would, therefore, be
        expected to be relatively good, over time.}''  The conceptual
        approximation will probably predict a ``conservative'' or
        ``pessimistic'' value in relation to actual markets.

        \begin{figure}[ht]
            \begin{center}
                \begin{minipage}[t]{0.45\textwidth}
                    \epsfxsize=1.0\linewidth
                    \epsffile{\directory/datahurstlownear.eps}
                    \caption[{\market}, ``forecastability'' of near
                        term rate of revenue returns]{{\market},
                        ``forecastability'' of near term rate of
                        revenue returns. Although the error function
                        is the most accurate, for the near term,
                        $H^{t} = \thurstlow^{t}$ may be used as a
                        reliable metric of ``forecastability'' of the
                        rate of revenue returns.}
                    \label{\SETLABEL:FN}
                \end{minipage}
                \hfill
                \begin{minipage}[t]{0.45\textwidth}
                    \epsfxsize=1.0\linewidth
                    \epsffile{\directory/datahurstlowfar.eps}
                    \caption[{\market}, ``forecastability'' of far
                        term rate of revenue returns]{{\market},
                        ``forecastability'' of far term rate of
                        revenue returns. Although the error function
                        is the most accurate, for the far term,
                        $\frac{1}{\sqrt{t}}$ may be used as a reliable
                        metric of ``forecastability'' of the rate of
                        revenue returns.}
                    \label{\SETLABEL:FF}
                \end{minipage}
            \end{center}
        \end{figure}

        \idx{operations research}
        As an interesting interpretation of the data presented in
        Figure~\ref{\SETLABEL:FN}, there may be, perhaps, some
        applicability to such operational agendas as inventory
        control. Maintaining too little inventory, obviously, will
        create a situation where the organization can not exploit
        market expansion, and maintaining too much inventory,
        likewise, would over extend the company, creating unnecessary
        losses when the market contracts. The company should maintain
        inventory levels that do not exceed, from
        Equation~\ref{\SETLABEL:MA}, ${\thurstlow}^{n} = 0.5$
        {\timescale}s of operations. Since the optimal amount of
        inventory and, from Equation~\ref{\SETLABEL:V}, the variance
        of change in the rate of revenue returns in the future can be
        calculated, there may, perhaps, be some applicability to a
        forecasting methodology that can be incorporated into other
        areas of operations research, for example the linear algebras
        using simplex methodologies for optimization of manufacturing
        processes. Traditionally, these forecasts are made by the
        sales department, and are subject to various subjective
        biases.

% Local Variables:
% TeX-parse-self: t
% TeX-auto-save: t
% TeX-master: "fractal.tex"
% End:


        %
% -----------------------------------------------------------------------------
%
% A license is hereby granted to reproduce this software source code and
% to create executable versions from this source code for personal,
% non-commercial use.  The copyright notice included with the software
% must be maintained in all copies produced.
%
% THIS PROGRAM IS PROVIDED "AS IS". THE AUTHOR PROVIDES NO WARRANTIES
% WHATSOEVER, EXPRESSED OR IMPLIED, INCLUDING WARRANTIES OF
% MERCHANTABILITY, TITLE, OR FITNESS FOR ANY PARTICULAR PURPOSE.  THE
% AUTHOR DOES NOT WARRANT THAT USE OF THIS PROGRAM DOES NOT INFRINGE THE
% INTELLECTUAL PROPERTY RIGHTS OF ANY THIRD PARTY IN ANY COUNTRY.
%
% Copyright (c) 1994-2006, John Conover, All Rights Reserved.
%
% Comments and/or bug reports should be addressed to:
%
%     john@email.johncon.com (John Conover)
%
% -----------------------------------------------------------------------------
%
% Revision: \RCSRevision \\
% Revision Time: \RCSTime UMT \\
% Revision Date: \RCSDate \\
% Revision Id: \RCSId \\
% Revision File: \RCSLog \\
\RCS $Revision: 0.0 $
\RCS $Date: 2006/01/20 04:38:13 $
\RCS $Id: simulation.tex,v 0.0 2006/01/20 04:38:13 john Exp $
% $Log: simulation.tex,v $
% Revision 0.0  2006/01/20 04:38:13  john
% Initial version
%
%
    \subsection{Simulation of Fixed Increment Approximation for Fiscal Strategy}
        \label{\SETLABEL:TSUNFAIRBROWNIAN}

        \subidx{\market}{market simulation}
        The data in this section is presented in tabular form in
        Section~\ref{\SETLABELREF:SIM}.
        Figure~\ref{\SETLABEL:TSUNFAIRBROWNIAN0} represents a
        constructional simulation of the time series data presented in
        Figure~\ref{\SETLABEL:TS}. The program {\it
        tsunfairbrownian}\/, which is briefly described in
        appendix~\ref{programs}, was used in the reconstruction. The
        reconstructed data is superimposed on the original time series
        data.  The program, {\it tsunfairbrownian}\/, essentially,
        constructs the new time series as a Brownian fractal with
        fixed increments---the value of the fixed increment is derived
        from the root mean square average of the normalized increments
        presented in Figure~\ref{\SETLABEL:TF}. The ``quality'' of
        such a reconstruction should be subject to adequate scepticism
        and scrutiny since, in all probability, the normalized
        increments presented in Figure~\ref{\SETLABEL:TF} represent a
        relatively complex process, that may not be ``modeled'' with
        such a simple methodology.

        As a further comparison of the the constructional simulation
        with the original time series data,
        Figure~\ref{\SETLABEL:TSUNFAIRBROWNIAN1} presents a normalized
        histogram of the normalized increments of the reconstructed
        time series, superimposed on the normalized histogram
        presented in Figure~\ref{\SETLABEL:NH}.

        \subidx{\market}{fiscal strategy, simulation}
        \subidx{markets}{simulation}
        \subidx{simulation}{markets}
        \subidx{strategy}{fiscal, simulation}
        \subidx{fiscal}{strategy, simulation}
        \subidx{programs}{tsunfairbrownian}
        \subidx{tsunfairbrownian}{program}
        \begin{figure}[ht]
            \begin{center}
                \begin{minipage}[t]{0.45\textwidth}
                    \epsfxsize=1.0\linewidth
                    \epsffile{\directory/tsunfairbrownian-f.eps}
                    \caption[{\market}, Time series data, empirical and
                        simulated]{{\market}, Time series data, empirical
                        and simulated, using the program {\it tsunfairbrownian}\/
                        with f = {\datafractionrms}. This data is
                        superimposed on the data presented in
                        Figure~\ref{\SETLABEL:TS}.}
                    \label{\SETLABEL:TSUNFAIRBROWNIAN0}
                \end{minipage}
                \hfill
                \begin{minipage}[t]{0.45\textwidth}
                    \epsfxsize=1.0\linewidth
                    \epsffile{\directory/tsunfairbrownian-f.tsfraction.tsnormal-s30.eps}
                    \caption[{\market}, normalized histogram,
                        empirical and simulated]{{\market}, normalized
                        histogram of the normalized increments of the
                        time series data shown in
                        Figure~\ref{\SETLABEL:TSUNFAIRBROWNIAN0},
                        empirical and simulated.  The empirical data
                        has a mean of {\datafractionmean}, with a
                        standard deviation of {\datafractionstddev}.
                        By comparison, the simulated data has a mean
                        of {\tsunfairbrownianfractionmean} with a
                        standard deviation of
                        {\tsunfairbrownianfractionstddev}. This data
                        is superimposed on the data presented in
                        Figure~\ref{\SETLABEL:NH}. The area under the
                        four curves is identical.}
                    \label{\SETLABEL:TSUNFAIRBROWNIAN1}
                \end{minipage}
            \end{center}
        \end{figure}

% Local Variables:
% TeX-parse-self: t
% TeX-auto-save: t
% TeX-master: "fractal.tex"
% End:


        %
% -----------------------------------------------------------------------------
%
% A license is hereby granted to reproduce this software source code and
% to create executable versions from this source code for personal,
% non-commercial use.  The copyright notice included with the software
% must be maintained in all copies produced.
%
% THIS PROGRAM IS PROVIDED "AS IS". THE AUTHOR PROVIDES NO WARRANTIES
% WHATSOEVER, EXPRESSED OR IMPLIED, INCLUDING WARRANTIES OF
% MERCHANTABILITY, TITLE, OR FITNESS FOR ANY PARTICULAR PURPOSE.  THE
% AUTHOR DOES NOT WARRANT THAT USE OF THIS PROGRAM DOES NOT INFRINGE THE
% INTELLECTUAL PROPERTY RIGHTS OF ANY THIRD PARTY IN ANY COUNTRY.
%
% Copyright (c) 1994-2006, John Conover, All Rights Reserved.
%
% Comments and/or bug reports should be addressed to:
%
%     john@email.johncon.com (John Conover)
%
% -----------------------------------------------------------------------------
%
% Revision: \RCSRevision \\
% Revision Time: \RCSTime UMT \\
% Revision Date: \RCSDate \\
% Revision Id: \RCSId \\
% Revision File: \RCSLog \\
\RCS $Revision: 0.0 $
\RCS $Date: 2006/01/20 04:38:13 $
\RCS $Id: maximum.tex,v 0.0 2006/01/20 04:38:13 john Exp $
% $Log: maximum.tex,v $
% Revision 0.0  2006/01/20 04:38:13  john
% Initial version
%
%
    \subsection{Simulation of Fixed Increment Approximation for Optimally Maximal Fiscal Strategy}
        \label{\SETLABEL:MAXSHANNON}
        \subidx{\market}{fiscal strategy, simulation}
        \subidx{\market}{maximum Shannon probability}
        \subidx{markets}{simulation}
        \subidx{simulation}{markets}
        \subidx{strategy}{optimum fiscal, simulation}
        \subidx{fiscal}{optimum strategy, simulation}
        \subidx{programs}{tsunfairbrownian}
        \subidx{tsunfairbrownian}{program}
        \subidx{Shannon}{probability}
        \subidx{probability}{Shannon}

        \subidx{strategy}{exploitable}
        \subidx{exploitable}{strategy}
        \subidx{programs}{tsshannonmax}
        \subidx{tsshannonmax}{program}
        \subidx{programs}{tsunfairbrownian}
        \subidx{tsunfairbrownian}{program}
        \subidx{strategy}{fiscal}
        \subidx{fiscal}{strategy}
        The data in this section is presented in tabular form in
        Section~\ref{\SETLABELREF:MAXSHANNON}. One of the issues of
        analysis, as mentioned in Section~\ref{\SETLABEL:OPS}, is to
        determine the maximum Shannon probability for the time series
        presented in Figure~\ref{\SETLABEL:TS}. Potentially, this
        could be exploited with an aggressive fiscal
        strategy. Figure~\ref{\SETLABEL:SHANNONMAX0} is a graph of the
        output of the {\it tsshannonmax}\/ program, which is described
        briefly in appendix~\ref{programs}. The maximum of this
        function is the maximum Shannon probability for the time
        series data presented in Figure~\ref{\SETLABEL:TS}.
        Figure~\ref{\SETLABEL:SHANNONMAX1} was constructed using {\it
        tsunfairbrownian}\/ program, which is also described in
        appendix~\ref{programs}, with the maximum Shannon probability,
        and the time series data presented in
        Figure~\ref{\SETLABEL:TS}. This represents a ``what if'' the
        investment strategy was changed from a Shannon probability of
        {\shannonlogreturns}, as derived in Section~\ref{\SETLABEL:SP}
        to {\shannonmax}. This process, essentially, extracts the
        random statistical data from the time series presented in
        Figure~\ref{\SETLABEL:TS}, and constructs a new time series,
        using the random statistical data, with a different investment
        strategy.  The program, {\it tsunfairbrownian}\/, essentially,
        constructs the new time series as a Brownian fractal with
        fixed increments.  The ``quality'' of such a reconstruction
        should be subject to adequate scepticism and scrutiny since,
        in all probability, the increments in the original data
        represent a relatively complex process, that may not be
        ``modeled'' with such a simple methodology.

        \begin{figure}[ht]
            \begin{center}
                \begin{minipage}[t]{0.45\textwidth}
                    \epsfxsize=1.0\linewidth
                    \epsffile{\directory/data.tsshannonmax.eps}
                    \caption[{\market}, maximum rate of revenue
                        returns] {{\market}, maximum rate of revenue
                        returns, per {\timescale}, vs. Shannon
                        probability. The maximum rate of revenue
                        returns, per {\timescale}, occurs at a Shannon
                        probability of {\shannonmax}.}
                    \label{\SETLABEL:SHANNONMAX0}
                \end{minipage}
                \hfill
                \begin{minipage}[t]{0.45\textwidth}
                    \epsfxsize=1.0\linewidth
                    \epsffile{\directory/data.tsshannonmax-p.tsunfairbrownian-p.eps}
                    \caption[{\market}, maximum rate of revenue
                        returns] {{\market}, maximum rate of revenue
                        returns, per {\timescale}, at a Shannon
                        probability, of {\shannonmax}, corresponding
                        to a ``wager'' fraction of {\twoponemax}.}
                    \label{\SETLABEL:SHANNONMAX1}
                \end{minipage}
            \end{center}
        \end{figure}

        \subidx{fractional}{Brownian motion}
        \subidx{Brownian motion}{fractional}
        \subidx{Shannon}{probability}
        \subidx{probability}{Shannon}
        \subidx{programs}{tsshannonmax}
        \subidx{tsshannonmax}{program}
        If it is assumed that the time series data set, presented in
        Figure~\ref{\SETLABEL:TS}, constitutes classical Brownian
        motion, then the Shannon probability can be calculated by
        counting the total number of {\timescale}s that the {\market}
        movement was positive, and dividing by the total number of
        {timescale}s represented in the time series. This quotient is
        {\pmax}, as compared with the predicted value from the program
        {\it tsshannonmax}\/ of {\shannonmax}.

% Local Variables:
% TeX-parse-self: t
% TeX-auto-save: t
% TeX-master: "fractal.tex"
% End:


        %
% -----------------------------------------------------------------------------
%
% A license is hereby granted to reproduce this software source code and
% to create executable versions from this source code for personal,
% non-commercial use.  The copyright notice included with the software
% must be maintained in all copies produced.
%
% THIS PROGRAM IS PROVIDED "AS IS". THE AUTHOR PROVIDES NO WARRANTIES
% WHATSOEVER, EXPRESSED OR IMPLIED, INCLUDING WARRANTIES OF
% MERCHANTABILITY, TITLE, OR FITNESS FOR ANY PARTICULAR PURPOSE.  THE
% AUTHOR DOES NOT WARRANT THAT USE OF THIS PROGRAM DOES NOT INFRINGE THE
% INTELLECTUAL PROPERTY RIGHTS OF ANY THIRD PARTY IN ANY COUNTRY.
%
% Copyright (c) 1994-2006, John Conover, All Rights Reserved.
%
% Comments and/or bug reports should be addressed to:
%
%     john@email.johncon.com (John Conover)
%
% -----------------------------------------------------------------------------
%
% Revision: \RCSRevision \\
% Revision Time: \RCSTime UMT \\
% Revision Date: \RCSDate \\
% Revision Id: \RCSId \\
% Revision File: \RCSLog \\
\RCS $Revision: 0.0 $
\RCS $Date: 2006/01/20 04:38:13 $
\RCS $Id: verification.tex,v 0.0 2006/01/20 04:38:13 john Exp $
% $Log: verification.tex,v $
% Revision 0.0  2006/01/20 04:38:13  john
% Initial version
%
%
    \subsection{Qualitative Verification of Fixed Increment Approximation Analysis}
        \label{\SETLABEL:QVA}

        \subidx{\market}{verification of analysis}
        \subidx{verification}{analysis}
        \subidx{analysis}{verification}
        \subidx{quality}{of analysis}
        \subidx{verification}{of methodology}
        \subidx{methodology}{verification of}
        \subidx{Shannon}{probability}
        \subidx{probability}{Shannon}

        This section evaluates various values based on the ``average''
        of the normalized increments presented in
        Figure~\ref{\SETLABEL:TFA}. These values are an approximation
        to a, probably, complex process with a distribution shown in
        Figure~\ref{\SETLABEL:TF}. These values will be used in a
        fixed increment Brownian fractal analysis of the {\market},
        and may, or may not, provide adequate accuracy for
        projections.

        The data in this section is presented in tabular form in
        sections~\ref{\SETLABELREF:VI1} and~\ref{\SETLABELREF:VI2}.
        As a subjective evaluation of the ``quality'' of the analysis
        of the {\market}, from Chapter~\ref{methodology},
        Equation~\ref{metricvalues1}, and using the mean and root mean
        square values of the normalized increments of the time series
        data presented in Figure~\ref{\SETLABEL:TS} from
        Figure~\ref{\SETLABEL:TF}, and the Shannon probability as
        calculated by counting the total number of {\timescale}s that
        the {\market} movement was positive, as presented in
        Section~\ref{\SETLABEL:MAXSHANNON}:

        \begin{eqnarray}
                  P & \approx & \frac{\frac{avg}{rms} + 1}{2}\\
            {\pmax} & \approx & \frac{\frac{\datafractionmean}{\datafractionrms} + 1}{2}\\
            {\pmax} & \approx & {\avgrms}
            \label{\SETLABEL:AVGS}
        \end{eqnarray}

        \subidx{Shannon}{probability}
        \subidx{probability}{Shannon}
        \noindent and comparing these values to the Shannon
        probability, as found by the {\it tsshannonmax}\/ program, which
        iterates for a maximum:

        \begin{eqnarray}
            {\pmax} \approx {\avgrms} \approx {\shannonmax}
        \end{eqnarray}

        \subidx{logarithmic}{returns}
        \subidx{returns}{logarithmic}
        In addition, the different methods of calculating the
        logarithmic returns, presented in Section~\ref{\SETLABEL:FS},
        should be compared. The four methods used were the mean of
        Figure~\ref{\SETLABEL:TF}, the constant in the least squares
        approximation to Figure~\ref{\SETLABEL:TF}, the least squares
        exponential approximation to Figure~\ref{\SETLABEL:TS}, and
        the logarithmic returns of Figure~\ref{\SETLABEL:TS}, derived
        as the mean of the logarithms of the quotients of the
        increments. The values for each of the methods are,
        respectively:

        \begin{equation}
            \datafractionmeanbits \approx \datafractionconstantbits \approx \datatslsqepbits \approx \logreturns
        \end{equation}

        It is implied in Section~\ref{\SETLABEL:FS},
        Subsection~\ref{\SETLABEL:SP} and in
        Section~\ref{\SETLABEL:TSUNFAIRBROWNIAN} that, a Brownian
        motion with fixed increments fractal may ``model'' the
        {\market}. Using Equation~\ref{stddev9} from
        Chapter~\ref{general}, Section~\ref{abmfi}:

        \begin{eqnarray}
                                    rms \left(2P - 1\right) & \approx & \frac{\sigma \left(2P - 1\right)}{2 \sqrt{P\left(1 - P\right)}}\\
            \datafractionrms \left(2 \cdot \pmax - 1\right) & \approx & \frac{\datafractionstddev \left(2 \cdot \pmax - 1\right)}{2\sqrt{\pmax \left(1 - \pmax\right)}}\\
                       \datafractionrms \cdot \twopminusone & \approx & \datafractionstddev \cdot \twopx\\
                                                      \rmsp & \approx & \sigmap
        \end{eqnarray}

        \noindent and, equating to the mean:

        \begin{equation}
            \datafractionmean \approx \rmsp \approx \sigmap
        \end{equation}

        \subidx{Shannon}{probability}
        \subidx{probability}{Shannon}
        \noindent where, as in Equation~\ref{\SETLABEL:AVGS} using the
        mean, root mean square, and standard deviation values of the
        normalized increments of the time series data presented in
        Figure~\ref{\SETLABEL:TS} from Figure~\ref{\SETLABEL:TF}, and
        the Shannon probability as calculated by counting the total
        number of {\timescale}s that the {\market} movement was
        positive, as presented in Section~\ref{\SETLABEL:MAXSHANNON}.

        As a final qualitative comparison, the absolute value of the
        normalized increments should be the same as the root mean
        square value\footnote{The absolute value of the normalized
        increments, when averaged, is related to the root mean square
        of the increments by a constant. If the normalized increments
        are a fixed increment, the constant is unity. If the
        normalized increments have a Gaussian distribution, the
        constant is $\approx 0.8$ depending on the accuracy of of
        ``fit'' to a Gaussian distribution.}, where the absolute value
        is presented in Figure~\ref{\SETLABEL:TFA}, and the root mean
        square value is presented in Figure~\ref{\SETLABEL:TF}:

        \begin{equation}
            \datafractionabsmean \approx \datafractionrms
        \end{equation}

        Note, that if the {\market} could be ``modeled'' as a Brownian
        motion with fixed increments fractal, then the standard
        deviation of the absolute value of the normalized increments
        of the time series data presented in Figure~\ref{\SETLABEL:TS}
        from Figure~\ref{\SETLABEL:TF} should be zero. It is
        $\datafractionabsstddev$.

% Local Variables:
% TeX-parse-self: t
% TeX-auto-save: t
% TeX-master: "fractal.tex"
% End:


    \renewcommand{\market}{Discreet Logistic Function}
    \renewcommand{\directory}{../markets/tsdlogistic}
    \renewcommand{\datafractionmean}{0.008052}
\renewcommand{\datafractionmeanbits}{0.011570}
\renewcommand{\datafractionmeanq}{0.002684}
\renewcommand{\datafractionmeanbitsq}{0.003867}
\renewcommand{\datafractionstddev}{0.038579}
\renewcommand{\datafractionrms}{0.039311}
\renewcommand{\avgrms}{0.602414}
\renewcommand{\ncompanies}{5.210454}
\renewcommand{\pncompanies}{0.544866}
\renewcommand{\datafractionabsmean}{0.029745}
\renewcommand{\datafractionabsstddev}{0.025769}
\renewcommand{\datafractionconstant}{0.010041}
\renewcommand{\datafractionconstantbits}{0.014414}
\renewcommand{\datafractionconstantq}{0.003347}
\renewcommand{\datafractionconstantbitsq}{0.004821}
\renewcommand{\datafractionslope}{-0.000021}
\renewcommand{\datafractionabsconstant}{0.035145}
\renewcommand{\datafractionabsslope}{-0.000057}
\renewcommand{\hurstall}{0.659558}
\renewcommand{\hurstlow}{0.707509}
\renewcommand{\hurstlowtwo}{1.415018}
\renewcommand{\hurstlowhundred}{70.750900}
\renewcommand{\hcalcall}{0.184942}
\renewcommand{\hcalclow}{0.102042}
\renewcommand{\shannonmax}{0.604167}
\renewcommand{\twoponemax}{0.208334}
\renewcommand{\logreturns}{0.010456}
\renewcommand{\twologreturns}{1.007274}
\renewcommand{\twologreturnshundred}{0.727387}
\renewcommand{\oneoverlogreturns}{95.638868}
\renewcommand{\pmax}{0.602094}
\renewcommand{\twopminusone}{0.204188}
\renewcommand{\rmsp}{0.008027}
\renewcommand{\twopx}{0.208583}
\renewcommand{\sigmap}{0.008047}
\renewcommand{\tsunfairbrownianfractionmean}{0.007862}
\renewcommand{\tsunfairbrownianfractionstddev}{0.038619}
\renewcommand{\shannonlogreturns}{0.560125}
\renewcommand{\shannonlogreturnshundred}{56.012500}
\renewcommand{\twopone}{0.120250}
\renewcommand{\twoponehundred}{12.025000}
\renewcommand{\hundredtwoponehundred}{87.975000}
\renewcommand{\hundredshannonlogreturnshundred}{43.987500}
\renewcommand{\datatslsqepbits}{0.007623}
\renewcommand{\thurstall}{0.633980}
\renewcommand{\thurstlow}{0.710108}
\renewcommand{\thurstlowtwo}{1.420216}
\renewcommand{\thurstlowhundred}{71.010800}
\renewcommand{\thcalcall}{0.247886}
\renewcommand{\thcalclow}{0.171737}
\renewcommand{\chisquared}{2.862000}
\renewcommand{\critical}{42.557000}

    \renewcommand{\timescale}{month}
    \subidx{market}{\market}
    \idx{\market}

    \section{\market}

        \renewcommand{\SETLABEL}{\LABPRE:DLF}
        \renewcommand{\SETLABELQ}{\LABPRE:DLFQ}
        \label{\SETLABEL}
        \renewcommand{\SETLABELREF}{\LABPREREF:DLF}

        \subidx{tsdlogistic}{program}
        \subidx{programs}{tsdlogistic}
        For the analysis, the data was in the directory
        {\directory}\footnote{As a simulation model, the program {\it
        tsdlogistic}\/ was run to make a time series data file, with
        the following parameters:

        \vspace{0.1in}
        {\noindent}tsdlogistic 4 1 315 | awk '{if ($1 > 0.0) print $1}' > data
        \vspace{0.1in}

        \noindent to make a time series of 300 elements, with no
        element equal to zero. The data is by {\timescale}s.}.

        The data in this section is presented in tabular form in
        Section~\ref{\SETLABELREF}. This is included for
        ``theoretical'' comparative purposes---of particular interest
        is the deterministic map in Figure~\ref{\SETLABEL:TD}.

        %
% -----------------------------------------------------------------------------
%
% A license is hereby granted to reproduce this software source code and
% to create executable versions from this source code for personal,
% non-commercial use.  The copyright notice included with the software
% must be maintained in all copies produced.
%
% THIS PROGRAM IS PROVIDED "AS IS". THE AUTHOR PROVIDES NO WARRANTIES
% WHATSOEVER, EXPRESSED OR IMPLIED, INCLUDING WARRANTIES OF
% MERCHANTABILITY, TITLE, OR FITNESS FOR ANY PARTICULAR PURPOSE.  THE
% AUTHOR DOES NOT WARRANT THAT USE OF THIS PROGRAM DOES NOT INFRINGE THE
% INTELLECTUAL PROPERTY RIGHTS OF ANY THIRD PARTY IN ANY COUNTRY.
%
% Copyright (c) 1994-2006, John Conover, All Rights Reserved.
%
% Comments and/or bug reports should be addressed to:
%
%     john@email.johncon.com (John Conover)
%
% -----------------------------------------------------------------------------
%
% Revision: \RCSRevision \\
% Revision Time: \RCSTime UMT \\
% Revision Date: \RCSDate \\
% Revision Id: \RCSId \\
% Revision File: \RCSLog \\
\RCS $Revision: 0.0 $
\RCS $Date: 2006/01/20 04:38:13 $
\RCS $Id: fraction.tex,v 0.0 2006/01/20 04:38:13 john Exp $
% $Log: fraction.tex,v $
% Revision 0.0  2006/01/20 04:38:13  john
% Initial version
%
%
    \subsection{Time Series Increments Analysis}
        \label{\SETLABEL:TSA}

        \subidx{\market}{Time series analysis}
        \subidx{time series}{increments}
        \subidx{time series}{analysis}
        \subidx{cumulative sum}{analysis}
        \subidx{analysis}{cumulative sum}
        \subidx{analysis}{random process}
        \subidx{random process}{analysis}
        \subidx{Gaussian}{increments}
        \subidx{increments}{Gaussian}
        \subidx{Brownian}{motion, fractional}
        \subidx{fractional}{Brownian motion}
        \subidx{fractal}{Brownian motion}
        The data in this section is presented in tabular form in
        Section~\ref{\SETLABELREF:TSA}.  Figure~\ref{\SETLABEL:TS} is
        a graph of the time series data for the {\market}.

        \subidx{increments}{normalized}
        \subidx{normalized}{increments}
        \subidx{programs}{tsfraction}
        \subidx{tsfraction}{program}
        Figure~\ref{\SETLABEL:TF} is a graph of the normalized
        increments of the time series data presented in
        Figure~\ref{\SETLABEL:TS}. The data presented was made by
        running the program {\it tsfraction}\/ on the time series
        data. The program {\it tsfraction}\/ is described briefly in
        Appendix~\ref{programs}, and subtracts the previous value from
        the next value, dividing this difference by the previous
        value, for each element in the time series data. The new time
        series contains the instantaneous change in the rate of
        revenue returns, divided by the magnitude of the instantaneous
        rate of revenue returns.

        \subidx{mean}{standard deviation}
        \subidx{standard deviation}{mean}
        \idx{root mean square}
        \idx{least squares approximation}
        \begin{figure}[ht]
            \begin{center}
                \begin{minipage}[t]{0.45\textwidth}
                    \epsfxsize=1.0\linewidth
                    \epsffile{\directory/data.eps}
                    \caption{{\market}, time series data.}
                    \label{\SETLABEL:TS}
                    \label{\SETLABELQ:TS}
                \end{minipage}
                \hfill
                \begin{minipage}[t]{0.45\textwidth}
                    \epsfxsize=1.0\linewidth
                    \epsffile{\directory/data.tsfraction.eps}
                    \caption[{\market}, normalized
                        increments]{{\market}, normalized increments
                        of the time series data presented in
                        Figure~\ref{\SETLABEL:TS}. The mean is
                        {\datafractionmean} with a standard deviation
                        of {\datafractionstddev}. The formula for the
                        least squares approximation is
                        ${\datafractionconstant} +
                        {\datafractionslope}t$, and the root mean
                        squared value is {\datafractionrms}. The
                        graph, labeled ``data\-.tsfraction\-.tsrms,''
                        is the running root mean square, and
                        ``data\-.tsfraction\-.tsavg'' is the running
                        average of the normalized increments.  This
                        graph is the fraction of change in the time
                        series, as a function of time. Note that the
                        slope of the mean, {\datafractionslope}, is
                        the coefficient of the nonlinearity term in
                        the normalized increments. See
                        Chapter~\ref{general}, Section~\ref{nlextend}
                        for a possible application of the logistic
                        function to this data set.}
                    \label{\SETLABEL:TF}
                    \label{\SETLABELQ:TF}
                \end{minipage}
            \end{center}
        \end{figure}

        \subidx{absolute value}{increments}
        \subidx{increments}{absolute value}

        Figure~\ref{\SETLABEL:TFA} is a graph of the absolute value of
        the normalized increments of the time series data presented in
        Figure~\ref{\SETLABEL:TF}. The data presented was made by
        running the Unix utility sed(1) on the normalized increments
        time series data to remove the negative signs. This is an
        absolute value procedure.  The resulting time series contains
        the absolute value of the instantaneous change in the rate of
        revenue returns, divided by the magnitude of the instantaneous
        rate of revenue returns\footnote{The absolute value of the
        normalized increments, when averaged, is related to the root
        mean square of the increments by a constant. If the normalized
        increments are a fixed increment, the constant is unity. If
        the normalized increments have a Gaussian distribution, the
        constant is $\approx 0.8$ depending on the accuracy of of
        ``fit'' to a Gaussian distribution.}.

        \subidx{histogram}{normalized}
        \subidx{normalized}{histogram}
        \subidx{programs}{tsnormal}
        \subidx{tsnormal}{program}
        \subidx{mean}{standard deviation}
        \subidx{standard deviation}{mean}
        \idx{root mean square}
        \idx{least squares approximation}
        \subidx{\market}{analysis of increments}
        Figure~\ref{\SETLABEL:NH} is the normalized histogram of the
        normalized increments of the time series data shown in
        Figure~\ref{\SETLABEL:TF}. The abscissa is 3 $\sigma$ limits,
        and the area under the two curves is identical. The data for
        this figure was produced by the program {\it tsnormal}\/,
        which is described briefly in Appendix~\ref{programs}.

        \begin{figure}[ht]
            \begin{center}
                \begin{minipage}[t]{0.45\textwidth}
                    \epsfxsize=1.0\linewidth
                    \epsffile{\directory/data.tsfraction.abs.eps}
                    \caption[{\market}, absolute value of the
                        normalized increments]{{\market}, absolute
                        value of the normalized increments of the time
                        series data presented in
                        Figure~\ref{\SETLABEL:TF}.  The mean is
                        {\datafractionabsmean} with a standard
                        deviation of {\datafractionabsstddev}. The
                        formula for the least squares approximation is
                        ${\datafractionabsconstant} +
                        {\datafractionabsslope}t$, and the root mean
                        square value, from Figure~\ref{\SETLABEL:TF},
                        is {\datafractionrms}.  The graph, labeled
                        ``data\-.tsfraction\-.tsrms,'' is the running
                        root mean square, and
                        ``data\-.tsfraction\-.tsavg'' is the running
                        average of the normalized increments presented
                        in Figure~\ref{\SETLABEL:TF}, superimposed
                        here for convenience. This graph is the
                        absolute value of the fraction of change in
                        the time series, as a function of time.}
                    \label{\SETLABEL:TFA}
                    \label{\SETLABELQ:TFA}
                \end{minipage}
                \hfill
                \begin{minipage}[t]{0.45\textwidth}
                    \epsfxsize=1.0\linewidth
                    \epsffile{\directory/data.tsfraction.tsnormal-s30.eps}
                    \caption[{\market}, normalized histogram of the
                        normalized increments]{{\market}, normalized
                        histogram of the normalized increments of the
                        time series data shown in
                        Figure~\ref{\SETLABEL:TF}.  The data has a
                        mean of {\datafractionmean}, with a standard
                        deviation of {\datafractionstddev}.  The area
                        under the two curves is identical. The
                        $\chi^2$ value of the observed and expected
                        values of the two curves is {\chisquared},
                        with a critical value of {\critical}.}
                    \label{\SETLABEL:NH}
                \end{minipage}
            \end{center}
        \end{figure}

        \subidx{programs}{tsXsquared}
        \subidx{tsXsquared}{program}
        \subidx{\market}{chi-squared values of increments}
        The program {\it tsXsquared}\/, which is briefly described in
        appendix~\ref{programs}, was used to derive the $\chi^2$
        statistics for the data presented in
        Figure~\ref{\SETLABEL:NH}.

        \subidx{programs}{tsstatest}
        \subidx{tsstatest}{program}
        \subidx{\market}{statistical estimates}

        Figure~\ref{\SETLABEL:SE} is the statistical estimate for the
        data presented in Figure~\ref{\SETLABEL:TF}, as derived by the
        program {\it tsstatest}\/, which is briefly described in
        appendix~\ref{programs}.

        \begin{figure}[ht]
            \begin{center}
                \begin{minipage}[t]{\textwidth}
                    \center{\fbox{\parbox{0.9\textwidth}{\XXX{\directory/data.tsstatest-f0.1-c0.9-i.tex}}}}
                    \caption[{\market}, statistical estimates of the
                        normalized increments]{{\market}, statistical
                        estimates of the normalized increments of the
                        time series shown in Figure~\ref{\SETLABEL:TF}.
                        The table was produced with the {\it
                        tsstatest}\/ program, and illustrates the
                        size of the data set required for a confidence
                        level of 90\%, with an error estimate of $\pm$
                        10\%, or alternately, the error estimate on
                        the time series shown in Figure~\ref{\SETLABEL:TF}.}
                    \label{\SETLABEL:SE}
                \end{minipage}
            \end{center}
        \end{figure}

        Note that the data set size estimations, as produced by the
        {\it tsstatest}\/ program, are probably very conservative,
        depending on the magnitude of the Shannon probability, $P =
        \shannonlogreturns$, as derived in
        Section~\ref{\SETLABEL:SP}. See Chapter~\ref{general},
        Section~\ref{serdss} for possible alternative methodologies
        for addressing the analysis of fractal time series with
        limited data set sizes. Depending on the magnitude of the
        Shannon probability, $P$, these estimates can be several
        orders of magnitude too high.

        \subidx{derivative of increments}{normalized}
        \subidx{normalized}{derivative of increments}
        \subidx{programs}{tsderivative}
        \subidx{tsderivative}{program}
        Figure~\ref{\SETLABEL:TF1} is the normalized histogram of the
        first derivative of the normalized increments of the time
        series data shown in Figure~\ref{\SETLABEL:TF}. In principle,
        if the distribution of the normalized increments presented in
        Figure~\ref{\SETLABEL:NH} is Gaussian in nature, this
        distribution would be similar to ``white noise,'' as presented
        in appendix~\ref{programs}, Figure~\ref{whiteexample}. The
        data was generated by the {\it tsderivative}\/ program, which
        is briefly described in
        appendix~\ref{programs}. Figure~\ref{\SETLABEL:TF2} is the
        normalized histogram of the second derivative of the
        normalized increments of the time series data shown in
        Figure~\ref{\SETLABEL:TF}. In principle, if the distribution
        of the normalized increments presented in
        Figure~\ref{\SETLABEL:NH} is an integrated Gaussian
        distribution in nature, this distribution would be similar to
        ``white noise,'' as presented in appendix~\ref{programs},
        Figure~\ref{whiteexample}.

        \begin{figure}[ht]
            \begin{center}
                \begin{minipage}[t]{0.45\textwidth}
                    \epsfxsize=1.0\linewidth
                    \epsffile{\directory/data.tsfraction.tsderivative.tsnormal-s30.eps}
                    \caption[{\market}, histogram of the first
                        derivative of the increments]{{\market},
                        normalized histogram of the first derivative
                        of the normalized increments of the time
                        series data shown in
                        Figure~\ref{\SETLABEL:TF}.}
                    \label{\SETLABEL:TF1}
                \end{minipage}
                \hfill
                \begin{minipage}[t]{0.45\textwidth}
                    \epsfxsize=1.0\linewidth
                    \epsffile{\directory/data.tsfraction.2tsderivative.tsnormal-s30.eps}
                    \caption[{\market}, histogram of the second
                        derivative of the increments]{{\market},
                        normalized histogram of second derivative of
                        the the normalized increments of the time
                        series data shown in
                        Figure~\ref{\SETLABEL:TF}.}
                    \label{\SETLABEL:TF2}
                \end{minipage}
            \end{center}
        \end{figure}

        \subidx{fractal}{range}
        \subidx{fractal}{R/S analysis}
        \subidx{\market}{rate of revenue returns, range}
        \subidx{\market}{deterministic mechanism}
        \subidx{deterministic}{mechanism}
        \subidx{mechanism}{deterministic}
        Figure~\ref{\SETLABEL:TR} is the range of values of the time
        series shown in Figure~\ref{\SETLABEL:TS}. The horizontal axis
        is time into the future. In principle, if the time series was
        characterized as fractional Brownian motion the graph in
        Figure~\ref{\SETLABEL:TR} would be a square root
        function\footnote{Note that the ``roughness,'' or ``sawtooth''
        characteristics of the graph in Figure~\ref{\SETLABEL:TR} are
        a computational artifact---caused by not using the -m option
        to the program {\it tshurst}\/, which is computationally
        inefficient.}. Figure~\ref{\SETLABEL:TD} is the deterministic
        map of the normalized increments of the time series data shown
        in Figure~\ref{\SETLABEL:TF}. The deterministic map is useful
        for determining if a time series was created by a
        deterministic mechanism. This, essentially, maps each element
        in the time series with the previous element in the time
        series.  See,~\cite[pp. 745]{Peitgen}.

        \begin{figure}[ht]
            \begin{center}
                \begin{minipage}[t]{0.45\textwidth}
                    \epsfxsize=1.0\linewidth
                    \epsffile{\directory/data.tshurst-f.eps}
                    \caption[{\market}, range]{{\market}, range of the
                        time series data shown in
                        Figure~\ref{\SETLABEL:TS}.}
                    \label{\SETLABEL:TR}
                \end{minipage}
                \hfill
                \begin{minipage}[t]{0.45\textwidth}
                    \epsfxsize=1.0\linewidth
                    \epsffile{\directory/data.tsfraction.tsdeterministic.eps}
                    \caption[{\market}, deterministic map]{{\market},
                        deterministic map of the normalized increments
                        of the time series data shown in
                        Figure~\ref{\SETLABEL:TF}.}
                    \label{\SETLABEL:TD}
                \end{minipage}
            \end{center}
        \end{figure}

% Local Variables:
% TeX-parse-self: t
% TeX-auto-save: t
% TeX-master: "fractal.tex"
% End:


            Figure~\ref{\SETLABEL:NH} would seem to indicate that the
            time series data for the {\market} does not represent a
            cumulative sum/integration of a random process that has a
            Gaussian distribution, (ie., satisfies the Gaussian
            increments property of fractional Brownian
            motion~\cite[pp. 250]{Crownover},) tending to discount the
            assumption that the time series data represents fractional
            Brownian motion.

        %
% -----------------------------------------------------------------------------
%
% A license is hereby granted to reproduce this software source code and
% to create executable versions from this source code for personal,
% non-commercial use.  The copyright notice included with the software
% must be maintained in all copies produced.
%
% THIS PROGRAM IS PROVIDED "AS IS". THE AUTHOR PROVIDES NO WARRANTIES
% WHATSOEVER, EXPRESSED OR IMPLIED, INCLUDING WARRANTIES OF
% MERCHANTABILITY, TITLE, OR FITNESS FOR ANY PARTICULAR PURPOSE.  THE
% AUTHOR DOES NOT WARRANT THAT USE OF THIS PROGRAM DOES NOT INFRINGE THE
% INTELLECTUAL PROPERTY RIGHTS OF ANY THIRD PARTY IN ANY COUNTRY.
%
% Copyright (c) 1994-2006, John Conover, All Rights Reserved.
%
% Comments and/or bug reports should be addressed to:
%
%     john@email.johncon.com (John Conover)
%
% -----------------------------------------------------------------------------
%
% Revision: \RCSRevision \\
% Revision Time: \RCSTime UMT \\
% Revision Date: \RCSDate \\
% Revision Id: \RCSId \\
% Revision File: \RCSLog \\
\RCS $Revision: 0.0 $
\RCS $Date: 2006/01/20 04:38:13 $
\RCS $Id: instant.tex,v 0.0 2006/01/20 04:38:13 john Exp $
% $Log: instant.tex,v $
% Revision 0.0  2006/01/20 04:38:13  john
% Initial version
%
%
    \subsection{Instantaneous Analysis of Normalized Increments}
        \label{\SETLABEL:IA}

        \subidx{\market}{instantaneous analysis of normalized increments}
        \idx{average of normalized increments}
        \idx{root mean square of normalized increments}
        \subidx{Shannon probability}{instantaneous computation of}
        \subidx{average of normalized increments}{instantaneous computation of}
        \subidx{root mean square of normalized increments}{instantaneous computation of}
        \subidx{instantaneous computation}{Shannon probability}
        \subidx{instantaneous computation}{average of normalized increments}
        \subidx{instantaneous computation}{root mean square of normalized increments}
        \idx{time series}
        \subidx{time series}{instantaneous analysis}
        \subidx{instantaneous analysis}{time series}
        \subidx{time series}{increments}
        \subidx{time series}{analysis}
        \subidx{Shannon}{probability}
        \subidx{probability}{Shannon}
        \subidx{normalized}{increments}
        \subidx{increments}{normalized}

        The program {\it tsinstant}\/, which is briefly described in
        Appendix~\ref{programs}, is for finding the instantaneous
        fraction of change in a time series. The value of a sample in
        the time series is subtracted from the previous sample in the
        time series, and divided by the value of the previous sample.
        As explained in Chapter~\ref{general},
        Sections~\ref{derivation},~\ref{GA},~\ref{abmfi},~\ref{aftsma}
        and,~\ref{ompl} for Brownian motion, random walk fractals, the
        absolute value of the instantaneous fraction of change is also
        the root mean square of the instantaneous fraction of
        change\footnote{The absolute value of the normalized
        increments, when averaged, is related to the root mean square
        of the increments by a constant. If the normalized increments
        are a fixed increment, the constant is unity. If the
        normalized increments have a Gaussian distribution, the
        constant is $\approx 0.8$ depending on the accuracy of of
        ``fit'' to a Gaussian distribution.}. Squaring this value is
        the average of the instantaneous fraction of change, and
        adding unity to the absolute value of the instantaneous
        fraction of change, and dividing by two, is the Shannon
        probability of the instantaneous fraction of change.

        Figure~\ref{\SETLABEL:IA1} is the instantaneous value of the
        root mean square of the normalized increments for the
        {\market}, and Figure~\ref{\SETLABEL:IA2} is the instantaneous
        Shannon probability for the normalized increments.

        \begin{figure}[ht]
            \begin{center}
                \begin{minipage}[t]{0.45\textwidth}
                    \epsfxsize=1.0\linewidth
                    \epsffile{\directory/data.tsinstant-r.eps}
                    \caption[{\market}, instantaneous value of
                        rms.]{{\market}, instantaneous value of the
                        root mean square of the normalized increments,
                        provided by running the program {\it
                        tsinstant}\/ with the -r option on the data
                        presented in Figure~\ref{\SETLABEL:TS}.}
                    \label{\SETLABEL:IA1}
                    \label{\SETLABELQ:IA1}
                \end{minipage}
                \hfill
                \begin{minipage}[t]{0.45\textwidth}
                    \epsfxsize=1.0\linewidth
                    \epsffile{\directory/data.tsinstant-s.eps}
                    \caption[{\market}, instantaneous value of
                        Shannon probability.]{{\market}, instantaneous
                        value of the Shannon probability of the
                        normalized increments, provided by running the
                        program {\it tsinstant}\/ with the -s option
                        on the data presented in
                        Figure~\ref{\SETLABEL:TS}.}
                    \label{\SETLABEL:IA2}
                    \label{\SETLABELQ:IA2}
                \end{minipage}
            \end{center}
        \end{figure}

% Local Variables:
% TeX-parse-self: t
% TeX-auto-save: t
% TeX-master: "fractal.tex"
% End:


        %
% -----------------------------------------------------------------------------
%
% A license is hereby granted to reproduce this software source code and
% to create executable versions from this source code for personal,
% non-commercial use.  The copyright notice included with the software
% must be maintained in all copies produced.
%
% THIS PROGRAM IS PROVIDED "AS IS". THE AUTHOR PROVIDES NO WARRANTIES
% WHATSOEVER, EXPRESSED OR IMPLIED, INCLUDING WARRANTIES OF
% MERCHANTABILITY, TITLE, OR FITNESS FOR ANY PARTICULAR PURPOSE.  THE
% AUTHOR DOES NOT WARRANT THAT USE OF THIS PROGRAM DOES NOT INFRINGE THE
% INTELLECTUAL PROPERTY RIGHTS OF ANY THIRD PARTY IN ANY COUNTRY.
%
% Copyright (c) 1994-2006, John Conover, All Rights Reserved.
%
% Comments and/or bug reports should be addressed to:
%
%     john@email.johncon.com (John Conover)
%
% -----------------------------------------------------------------------------
%
% Revision: \RCSRevision \\
% Revision Time: \RCSTime UMT \\
% Revision Date: \RCSDate \\
% Revision Id: \RCSId \\
% Revision File: \RCSLog \\
\RCS $Revision: 0.0 $
\RCS $Date: 2006/01/20 04:38:13 $
\RCS $Id: logistic.tex,v 0.0 2006/01/20 04:38:13 john Exp $
% $Log: logistic.tex,v $
% Revision 0.0  2006/01/20 04:38:13  john
% Initial version
%
%
    \subsection{Logistic Analysis}
        \label{\SETLABEL:LA}

        \subidx{\market}{Logistic function analysis}
        \subidx{time series}{logistic function}
        \subidx{logistic function}{time series}
        \subidx{time series}{increments}
        \subidx{time series}{analysis}
        \subidx{cumulative sum}{analysis}
        \subidx{analysis}{cumulative sum}
        \subidx{analysis}{random process}
        \subidx{random process}{analysis}
        The data in this section is presented in tabular form in
        Section~\ref{\SETLABELREF:LAA}.  Figure~\ref{\SETLABEL:LA1} is
        a graph of the logistic function estimates of the time series
        data for the {\market}. The reader is cautioned that these
        graphs are constructed using the method suggested in
        Chapter~\ref{general}, Section~\ref{nlextend} and enormous
        precision is required for adequate prediction of the logistic
        function,~\cite{Modis}. Particularly, the non-linear term will
        usually require intervention to produce a practical fit to the
        data. In addition, there are numerical stability issues with
        logistic function methodologies\footnote{For example, in
        Figures~\ref{\SETLABEL:LA1} and~\ref{\SETLABEL:LA2}, if the
        non-linear term, $b$, was greater than zero, it was set to
        zero to produce the graphs. See Section~\ref{\SETLABELREF:LAA}
        for the actual derived values. In other cases, the magnitude
        of $b$ was too large, resulting in a graph that was decreasing
        as a function of time}.  The methodology should be regarded as
        ``fragile.'' It is included for completeness.

        \idx{least squares approximation}
        Figure~\ref{\SETLABEL:LA1} is a graph of the logistic function
        for the time series data presented in
        Figure~\ref{\SETLABEL:TS}. The data presented was made by
        running the program {\it tsdlogistic}\/, which is described
        briefly in Appendix~\ref{programs}, on the parameters
        extracted from the time series data as suggested in
        Figure~\ref{\SETLABEL:TF}. The program {\it tslsq}\/ was used
        to derive the constant and the slope of the normalized
        increments of the data presented in Figure~\ref{\SETLABEL:TF}.
        Figure~\ref{\SETLABEL:LA2} is the same graph, but with the
        time scale expanded by a factor of two.

        \begin{figure}[ht]
            \begin{center}
                \begin{minipage}[t]{0.45\textwidth}
                    \epsfxsize=1.0\linewidth
                    \epsffile{\directory/data.tsfraction.tslsq-p.tsdlogistic.eps}
                    \caption[{\market}, logistic function
                        estimates.]{{\market}, logistic function
                        estimates, provided by running the {\it
                        tslsq}\/ program on the normalized increments
                        presented in Figure~\ref{\SETLABEL:TF} with
                        the -p option. These parameters were used as
                        arguments to the {\it tsdlogistic}\/ program.}
                    \label{\SETLABEL:LA1}
                    \label{\SETLABELQ:LA1}
                \end{minipage}
                \hfill
                \begin{minipage}[t]{0.45\textwidth}
                    \epsfxsize=1.0\linewidth
                    \epsffile{\directory/data.tsfraction.tslsq-p.tsdlogistic2.eps}
                    \caption[{\market}, logistic function
                        estimates.]{{\market}, logistic function
                        estimates of Figure~\ref{\SETLABEL:LA1} with
                        the time scale expanded by a factor of two.}
                    \label{\SETLABEL:LA2}
                    \label{\SETLABELQ:LA2}
                \end{minipage}
            \end{center}
        \end{figure}

% Local Variables:
% TeX-parse-self: t
% TeX-auto-save: t
% TeX-master: "fractal.tex"
% End:


        %
% -----------------------------------------------------------------------------
%
% A license is hereby granted to reproduce this software source code and
% to create executable versions from this source code for personal,
% non-commercial use.  The copyright notice included with the software
% must be maintained in all copies produced.
%
% THIS PROGRAM IS PROVIDED "AS IS". THE AUTHOR PROVIDES NO WARRANTIES
% WHATSOEVER, EXPRESSED OR IMPLIED, INCLUDING WARRANTIES OF
% MERCHANTABILITY, TITLE, OR FITNESS FOR ANY PARTICULAR PURPOSE.  THE
% AUTHOR DOES NOT WARRANT THAT USE OF THIS PROGRAM DOES NOT INFRINGE THE
% INTELLECTUAL PROPERTY RIGHTS OF ANY THIRD PARTY IN ANY COUNTRY.
%
% Copyright (c) 1994-2006, John Conover, All Rights Reserved.
%
% Comments and/or bug reports should be addressed to:
%
%     john@email.johncon.com (John Conover)
%
% -----------------------------------------------------------------------------
%
% Revision: \RCSRevision \\
% Revision Time: \RCSTime UMT \\
% Revision Date: \RCSDate \\
% Revision Id: \RCSId \\
% Revision File: \RCSLog \\
\RCS $Revision: 0.0 $
\RCS $Date: 2006/01/20 04:38:13 $
\RCS $Id: hurst.tex,v 0.0 2006/01/20 04:38:13 john Exp $
% $Log: hurst.tex,v $
% Revision 0.0  2006/01/20 04:38:13  john
% Initial version
%
%
    \subsection{Hurst Coefficient Analysis}
        \label{\SETLABEL:H}

        \subidx{\market}{Hurst coefficient analysis}
        \subidx{Hurst coefficient}{analysis}
        \subidx{increments}{normalized}
        \subidx{normalized}{increments}
        \subidx{programs}{tshurst}
        \subidx{tshurst}{program}
        The data in this section is presented in tabular form in
        Section~\ref{\SETLABELREF:HCHP}. Figure~\ref{\SETLABEL:HC} is
        a graph of the Hurst coefficient data time series data shown
        in Figure~\ref{\SETLABEL:TS}. The slope of the graph is the
        Hurst coefficient.  The data for this figure was produced by
        the program {\it tshurst}\/, which is described briefly in
        Appendix~\ref{programs}.

        \subidx{\market}{H parameter analysis}
        \subidx{H parameter}{analysis}
        \subidx{programs}{tshcalc}
        \subidx{tshcalc}{program}
        Figure~\ref{\SETLABEL:HP} is a graph of the H parameter data
        for the normalized increments of the time series data shown in
        Figure~\ref{\SETLABEL:TF}. The data for this figure was
        produced by the program {\it tshcalc}\/, which is described
        briefly in Appendix~\ref{programs}.

        \begin{figure}[ht]
            \begin{center}
                \begin{minipage}[t]{0.45\textwidth}
                    \epsfxsize=1.0\linewidth
                    \epsffile{\directory/data.tshurst.eps}
                    \caption[{\market}, Hurst coefficient data]{{\market},
                        Hurst coefficient data for the normalized
                        increments of the time series data shown in
                        Figure~\ref{\SETLABEL:TF}.  The slope of the graph
                        is the Hurst coefficient.}
                    \label{\SETLABEL:HC}
                \end{minipage}
                \hfill
                \begin{minipage}[t]{0.45\textwidth}
                    \epsfxsize=1.0\linewidth
                    \epsffile{\directory/data.tshcalc.eps}
                    \caption[{\market}, H parameter data]{{\market}, H
                        parameter data for the normalized increments of
                        the time series data shown in
                        Figure~\ref{\SETLABEL:TF} The slope of the graph
                        is the H parameter.}
                    \label{\SETLABEL:HP}
                \end{minipage}
            \end{center}
        \end{figure}

        \subidx{revenue}{See, rate of revenue returns}
        \subidx{returns}{See, rate of revenue returns}
        \subidx{\market}{revenues}
        \subidx{Hurst coefficient}{analysis}
        \subidx{\market}{Hurst coefficient analysis}
        \subidx{\market}{rate of change}
        \subidx{\market}{windows of opportunity}
        \subidx{rate of revenue returns}{forecast}
        \subidx{forecast}{rate of revenue returns}
        \idx{windows of opportunity}
        \subidx{programs}{tslsq}
        \subidx{tslsq}{program}

        The approximately linear slope of the graph in
        Figure~\ref{\SETLABEL:HC} implies that the variance of the
        rate of revenue returns, (per {\timescale},) in the {\market},
        $V(t_2 - t_1)$, over a period of time is proportional to the
        period of time raised to twice the Hurst
        coefficient~\cite[pp. 180]{Feder},~\cite[pp. 246]{Crownover}.
        This seems to be a quantitative statement concerning how fast,
        and to what degree, the rate of revenue returns' state of
        affairs can change over a period of time.  An additional
        implication, for Hurst coefficients sufficiently close to 0.5,
        is that the probability of the state of affairs repeating
        sometime in the future goes down with increasing
        time\footnote{It can be shown that the number of expected
        market ``high'' and ``low'' transitions, $N$, scales with the
        square root of time, or $N \propto \sqrt {t}$, meaning that
        the cumulative distribution of the probability, $P$, of the
        duration of a market's ``high'' or ``low'' exceeding a given
        time interval, $t$, is proportional to the reciprocal of the
        square root of the time interval, $P \propto 1 / \sqrt {t}$,
        (or, conversely, that the probability of the duration of a
        market's ``high'' or ``low'' exceeding a given time interval
        is proportional to the reciprocal of the time interval raised
        to the power $3 / 2$, ie., $P \propto 1 / t^{3 /
        2}$,~\cite[pp. 153]{Schroeder}. What this means is that a
        histogram of the ``zero free'' run-lengths of a market being
        ``high'' or ``low,'' over a long time, would have a $1 / t^{3
        / 2}$ characteristic.)}, $t$, $p(t) = erf (1/\sqrt{2t})$ which
        is approximately $1/\sqrt{t}$ for $t \gg
        1$~\cite[pp. 160]{Schroeder}. Figures~\ref{\SETLABEL:FN},
        and,~\ref{\SETLABEL:FF} compare methods of approximation of
        the ``forecastability'' of the rate of revenue returns in the
        {\market} for the near term and far term,
        respectively~\cite[pp. 83-84]{Peters:CAOITCM}\footnote{The
        author is not comfortable with Peters' interpretation. For
        example, if the algorithm explained
        in~\cite[pp. 82]{Peters:CAOITCM} is used on ``white noise''
        which, by definition, never has any correlations, the short
        term Hurst coefficient, and thus the ``forecastability,'' is
        still near unity---a bit of an enigma. This can be verified
        with the {\it tswhite}\/ and {\it tshurst}\/ programs, which
        are briefly described in Appendix~\ref{programs}.}.  This
        seems to be a quantitative statement concerning ``windows of
        opportunity'' in the rate of revenue returns, (per
        {\timescale}.)  The program {\it tslsq}\/ was used on the
        Hurst coefficient data, presented in
        Figure~\ref{\SETLABEL:HC}, to provide a least squares
        approximation to the Hurst coefficient. The superimposed least
        squares approximation with on original Hurst coefficient data
        is presented.  The time series data has a Hurst coefficient of
        {\thurstlow}, so that:

        \subidx{\market}{Hurst coefficient analysis}
        \begin{eqnarray}
            V\left(t_2 - t_1\right) & \propto & \left(t_2 - t_1\right)^{2 \cdot H}\\
            V\left(t_2 - t_1\right) & \propto & \left(t_2 - t_1\right)^{2 \cdot {\thurstlow}}\\
                                    & \propto & \left(t_2 - t_1\right)^{\thurstlowtwo}
            \label{\SETLABEL:V}
        \end{eqnarray}

        \subidx{fractional}{Brownian motion}
        \subidx{Brownian motion}{fractional}
        \idx{fractal}
        \noindent where $V(t_2 - t_1)$ is the variance of the
        increments of the rate of revenue returns, (per {\timescale},)
        over the time interval $t_2 -
        t_1$,~\cite[pp. 177]{Feder},~\cite[pp. 494]{Peitgen}. If $H >
        \frac{1}{2}$, then the time series is termed as being
        characterized by ``fractional Brownian
        motion~\cite[pp. 170]{Feder}.''

        \subidx{rate of revenue returns}{predictability}
        \subidx{rate of revenue returns}{forecastability}
        \subidx{rate of revenue returns}{consistency}
        \subidx{predictability}{rate of revenue returns}
        \subidx{forecastability}{rate of revenue returns}
        \subidx{consistency}{rate of revenue returns}
        \subidx{\market}{rate of revenue returns, predictability}
        \subidx{\market}{rate of revenue returns, forecastability}
        \subidx{\market}{rate of revenue returns, consistency}
        \subidx{Hurst coefficient}{analysis}
        \subidx{\market}{Hurst coefficient analysis}
        \subidx{\market}{rate of change}

        In some sense, the Hurst coefficient is a quantitative
        expression of the ``forecastability'' of the future based on
        the past\footnote{Actually, in general, when summing fractal
        entities, the method used should be a root mean square
        process, dependent on the Hurst Coefficient, $H$, where
        $P_{total}^H = P_1^H + P_2^H + \cdots$, where $P_n$ is the
        fractal entities. For a Brownian motion, or random walk type
        of fractal the Hurst Coefficient is a function of time into
        the future. For the ``near term,'' the Hurst coefficient is
        very near unity, meaning the summation process is linear. For
        the ``long term,'' $H \approx 0.5$, or a standard root mean
        square summation process should be used. If $H$ is $0.5$ then
        the market is termed a Brownian motion, or random walk
        process. If it is larger than 0.5, it is termed fractional
        Brownian motion process. For a random walk process, ``near
        term'' and ``far term'' are quantitatively differentiated on
        the Hurst Coefficient graph where $1 - \ln (t) = 0.5 \cdot \ln
        (t)$, or when $\ln (t) = 2$, or $t = 7.389\ldots$ See
        Section~\ref{\SETLABEL:FS} for the particulars on using Hurst
        Coefficient to sum fractal process' for the {\market}. See
        also~\cite[pp. 67, 83-84]{Peters:CAOITCM} and~\cite[pp. 129,
        159]{Schroeder} for particulars on the implications of the
        Hurst Coefficient and root mean square summation issues.}.  A
        Hurst coefficient of {\thurstlow}, (for the near future, and
        {\thurstall} for the distant future.) implies that the
        likelihood of the rate of revenue returns, (per {\timescale},)
        for any two consecutive {\timescale}s being the same is
        {\thurstlowhundred}\%~\cite[pp. 66]{Peters:CAOITCM} for the
        near future, and {\thurstall} for the distant
        future. Likewise, there is a {\thurstlowhundred}\% chance of
        the rate of revenue returns, (per {\timescale},) movements
        being the same in consecutive time periods---ie., if, in a
        given {\timescale}, the rate of revenue returns, (per
        {\timescale},) is increasing, there is a {\thurstlowhundred}\%
        that the rate of revenue returns, (per {\timescale},) will
        increase in the following period, also. In some sense, this is
        a quantitative statement on how ``predictable,'' or
        ``forecastable'' the rate of revenue returns, (per
        {\timescale},) for the {\market} are over time, since the
        probability of having $n$ many consecutive {\timescale}s of
        the same agenda is $H^n$ where $H$ is the Hurst coefficient,
        or, letting the short term probability of having $n$ many
        {\timescale}s of the same market agenda, $p_a$, is:

        \begin{eqnarray}
            p_a\left(n\right) & = & H^{n}\\
                              & = & {\thurstlow}^{n}
            \label{\SETLABEL:MA}
        \end{eqnarray}

        \subidx{rate of revenue returns}{predictability}
        \subidx{rate of revenue returns}{forecastability}
        \subidx{rate of revenue returns}{consistency}
        \subidx{predictability}{rate of revenue returns}
        \subidx{forecastability}{rate of revenue returns}
        \subidx{consistency}{rate of revenue returns}
        As an interesting interpretation of the normalized increments
        of the time series data presented in
        Figure~\ref{\SETLABEL:TF}, if the vertical axis is multiplied
        by 100, to convert to percent, then the graph represents the
        error, in percent, that would be made by forecasting, month by
        month, that the next {\timescale}'s rate of revenue returns
        would be the same as the current {\timescale}'s revenue
        rate. Interestingly, it is $\datafractionmean \cdot 100$
        percent, on the average, with a standard deviation of
        $\datafractionstddev \cdot 100$ percent, and a root mean
        square error value of $\datafractionrms \cdot 100$
        percent---small values for such a simple forecasting
        mechanism.

        \subidx{\market}{rate of revenue returns, range}
        \subidx{Hurst coefficient}{analysis}
        \subidx{\market}{Hurst coefficient analysis}
        \subidx{\market}{rate of change}

        This is, essentially, a statement of the range of values, in
        the increments of the rate of revenue returns, (per
        {\timescale},) that is to be expected over the time interval,
        $t_2 - t_1$,
        $R_v$,~\cite[pp. 178]{Feder},~\cite[pp. 172]{Cambel}:

        \begin{eqnarray}
            R_v\left(t_2 - t_1\right) & \propto & \left(t_2 - t_1\right)^{H}\\
                                      & \propto & \left(t_2 - t_1\right)^{\thurstlow}
            \label{\SETLABEL:R}
        \end{eqnarray}

        \subidx{\market}{rate of revenue returns, range}
        \subidx{Hurst coefficient}{analysis}
        \subidx{\market}{Hurst coefficient analysis}
        \subidx{\market}{rate of change}
        \subidx{Markov}{statistics}
        \subidx{statistics}{Markov}
        \noindent where $R$ is the range of values in the increments
        of the rate of revenue returns, (per {\timescale}.) A Hurst
        coefficient, $H$, that is much larger than $\frac{1}{2}$, (but
        less than 1,) implies a strongly non-Gaussian distribution in
        the increments of the rate of revenue returns, (per
        {\timescale},)~\cite[pp. 152, 194]{Feder}, and a Hurst
        coefficient near $\frac{1}{2}$ implies that the increments of
        the rate of revenue returns, (per {\timescale}) is
        characteristic of an independent
        process~\cite[pp. 195]{Feder}. Extreme caution should be
        exercised in using Markov statistics in any analysis where the
        Hurst coefficient is not
        $\frac{1}{2}$,~\cite[pp. 124]{Crownover},~\cite[pp. 106]{Peters:CAOITCM}.


        As a useful approximation, if $H$, is approximately
        $\frac{1}{2}$, Equation~\ref{\SETLABEL:R} reduces
        to,~\cite[pp. 129]{Schroeder}:

        \begin{eqnarray}
            R\left(t_2 - t_1\right) & \propto & (t_2 - t_1)^{\frac{1}{2}}\\
                                    & \propto & \sqrt{\left(t_2 - t_1\right)}
        \end{eqnarray}

        \subidx{\market}{rate of revenue returns, range}
        \subidx{\market}{rate of revenue returns, increase and decrease}
        \subidx{Hurst coefficient}{analysis}
        \subidx{\market}{Hurst coefficient analysis}
        \subidx{\market}{rate of change}
        \subidx{Markov}{statistics}
        \subidx{statistics}{Markov}

        In the case where the Hurst coefficient, $H$, is
        $\frac{1}{2}$, the range of values in the increments of the
        rate of revenue returns, (per {\timescale},) divided by the
        standard deviation of these values, $S$, can be anticipated to
        increase over time according to the following
        relation,~\cite[pp. 154]{Feder},~\cite[pp. 129]{Schroeder}:

        \begin{equation}
            \frac{R\left(t_2 - t_1\right)}{S} \propto \left(t_2 - t_1\right)^{\frac{1}{2}}
        \end{equation}

        \subidx{\market}{rate of revenue returns, range}
        \subidx{\market}{rate of revenue returns, increase and decrease}
        \subidx{Hurst coefficient}{analysis}
        \subidx{\market}{Hurst coefficient analysis}
        \subidx{\market}{rate of change}
        \noindent which is a useful conceptual approximation, since it
        involves only the square root function---if the range and the
        standard deviation of the increments of the rate of revenue
        returns, (per {\timescale},) are known, (and $H \approx
        \frac{1}{2}$,) then the expected change in $\frac{R}{S}$, will
        increase with the square root of time\footnote{To be precise,
        it is actually asymptotically proportional to
        $\tau^{\frac{1}{2}}$}.

        Another useful approximation when rescaling processes that are
        characterize by Brownian motion, (ie., when $H \approx
        \frac{1}{2}$,) is that:

        \begin{eqnarray}
            X\left(t\right) & \propto & \frac{X\left(rt\right)}{r^{H}}\\
                            & \propto & \frac{X\left(rt\right)}{r^{\thurstlow}}
        \end{eqnarray}

        \idx{Brownian motion}
        \idx{fractal}
        Where $X(t)$ is the process characterized by Brownian motion,
        and $r$ is a scaling factor,~\cite[pp. 494]{Peitgen}.

        \subidx{programs}{tslsq}
        \subidx{tslsq}{program}
        The program {\it tslsq}\/ was used on the H parameter data,
        presented in Figure~\ref{\SETLABEL:HP}, to provide a least
        squares approximation to the H parameter for the
        {\market}. The superimposed least squares approximation on the
        original H parameter data is presented.  By contrast, the H
        parameter, as derived by the methodology outlined
        in~\cite[pp. 249]{Crownover}, is {\thcalclow} for the near
        future, and {\thcalcall} for the distant future.

        \subidx{\market}{Hurst coefficient analysis}
        \subidx{Hurst coefficient}{analysis}
        \subidx{increments}{normalized}
        \subidx{normalized}{increments}
        \subidx{programs}{tshurst}
        \subidx{tshurst}{program}
        \subidx{\market}{H parameter analysis}
        \subidx{H parameter}{analysis}
        \subidx{programs}{tshcalc}
        \subidx{tshcalc}{program}
        Figures~\ref{\SETLABEL:HC} and~\ref{\SETLABEL:HP} represent
        Hurst coefficient and H parameter data that are derived from
        the normalized increments, shown in
        Figure~\ref{\SETLABEL:TF}. In this case, the data is
        considered a normalized derivative of the time series data
        presented in Figure~\ref{\SETLABEL:TF}, instead of a
        cumulative sum.  The program, {\it tshurst}\/, is described
        briefly in appendix~\ref{programs}, and the data for
        figures~\ref{\SETLABEL:THC} and~\ref{\SETLABEL:THP} was made
        using the -d option.

        \begin{figure}[ht]
            \begin{center}
                \begin{minipage}[t]{0.45\textwidth}
                    \epsfxsize=1.0\linewidth
                    \epsffile{\directory/data.tsfraction.tshurst-d.eps}
                    \caption[{\market}, traditional Hurst coefficient
                        data]{{\market}, traditional Hurst coefficient
                        data for the time series data shown in
                        Figure~\ref{\SETLABEL:TS}.  The slope of the
                        graph is the Hurst coefficient, and is
                        {\hurstlow} for the near term, and
                        {\hurstall} for the far term.}
                    \label{\SETLABEL:THC}
                \end{minipage}
                \hfill
                \begin{minipage}[t]{0.45\textwidth}
                    \epsfxsize=1.0\linewidth
                    \epsffile{\directory/data.tsfraction.tshcalc-d.eps}
                    \caption[{\market}, traditional H parameter
                        data]{{\market}, traditional H parameter data
                        for the time series data shown in
                        Figure~\ref{\SETLABEL:TS} The slope of the
                        graph is the H parameter, and is {\hcalclow}
                        for the near term, and {\hcalcall} for the
                        far term.}
                    \label{\SETLABEL:THP}
                \end{minipage}
            \end{center}
        \end{figure}

% Local Variables:
% TeX-parse-self: t
% TeX-auto-save: t
% TeX-master: "fractal.tex"
% End:


        %
% -----------------------------------------------------------------------------
%
% A license is hereby granted to reproduce this software source code and
% to create executable versions from this source code for personal,
% non-commercial use.  The copyright notice included with the software
% must be maintained in all copies produced.
%
% THIS PROGRAM IS PROVIDED "AS IS". THE AUTHOR PROVIDES NO WARRANTIES
% WHATSOEVER, EXPRESSED OR IMPLIED, INCLUDING WARRANTIES OF
% MERCHANTABILITY, TITLE, OR FITNESS FOR ANY PARTICULAR PURPOSE.  THE
% AUTHOR DOES NOT WARRANT THAT USE OF THIS PROGRAM DOES NOT INFRINGE THE
% INTELLECTUAL PROPERTY RIGHTS OF ANY THIRD PARTY IN ANY COUNTRY.
%
% Copyright (c) 1994-2006, John Conover, All Rights Reserved.
%
% Comments and/or bug reports should be addressed to:
%
%     john@email.johncon.com (John Conover)
%
% -----------------------------------------------------------------------------
%
% Revision: \RCSRevision \\
% Revision Time: \RCSTime UMT \\
% Revision Date: \RCSDate \\
% Revision Id: \RCSId \\
% Revision File: \RCSLog \\
\RCS $Revision: 0.0 $
\RCS $Date: 2006/01/20 04:38:13 $
\RCS $Id: fiscal.tex,v 0.0 2006/01/20 04:38:13 john Exp $
% $Log: fiscal.tex,v $
% Revision 0.0  2006/01/20 04:38:13  john
% Initial version
%
%
    \subsection{Fixed Increment Approximation for Fiscal Strategy}
        \label{\SETLABEL:FS}

        \subidx{\market}{fiscal strategy}
        \subidx{markets}{analysis}
        \subidx{analysis}{markets}
        \subidx{strategy}{fiscal}
        \subidx{fiscal}{strategy}
        The data in this section is presented in tabular form in
        Section~\ref{\SETLABELREF:LR}. This section derives various
        values based on the ``average'' of the normalized increments
        presented in Figure~\ref{\SETLABEL:TFA}. These values are an
        approximation to a, probably, complex process with a
        distribution shown in Figure~\ref{\SETLABEL:TF}. These values
        will be used in a fixed increment Brownian fractal analysis
        and simulation of the {\market}, and may, or may not, provide
        adequate accuracy for projections.

        For an organization operating in the {\market}, the fiscal
        strategy, commensurate with the aggregate environment, can be
        derived as follows~\cite[pp. 128, pp
        151]{Schroeder},~\cite[pp. 450]{Reza},~\cite[pp. 270]{Pierce}:
        \vspace{0.15in}

        \subsubsection{Logarithmic Returns}
            \label{\SETLABEL:LR}

            \subidx{logarithmic}{returns}
            \subidx{returns}{logarithmic}
            \subidx{\market}{logarithmic returns}
            The logarithmic returns can be calculated by various
            means. Four will be presented here, for comparison.

            \subidx{programs}{tsnormal}
            \subidx{tsnormal}{program}
            \subidx{logarithmic}{returns}
            \subidx{returns}{logarithmic}
            The logarithmic returns, in bits, $bits$, as computed from
            the mean, by the program {\it tsnormal}\/, which is
            described in Chapter~\ref{programs}, and is presented in
            Figure~\ref{\SETLABEL:TF}, and Equation~\ref{abits} from
            Section~\ref{ereturns} in Chapter~\ref{general}:

            \begin{equation}
                bits = \frac{\ln \left({\datafractionmean} + 1\right)}{\ln \left(2\right)} = \datafractionmeanbits
            \end{equation}

            \subidx{programs}{tslsq}
            \subidx{tslsq}{program}
            \subidx{logarithmic}{returns}
            \subidx{returns}{logarithmic}
            \noindent By comparison, the logarithmic returns, in bits,
            $bits$, as computed from the constant in the least squares
            approximation, using the program {\it tslsq}\/, which is briefly
            described in Chapter~\ref{programs}, as presented in
            Figure~\ref{\SETLABEL:TF}, and Equation~\ref{abits} from
            Section~\ref{ereturns} in Chapter~\ref{general}:

            \begin{equation}
                bits = \frac{\ln \left({\datafractionconstant} + 1\right)}{\ln \left(2\right)} = \datafractionconstantbits
            \end{equation}

            Note that if the mean is not constant in
            Figure~\ref{\SETLABEL:TF}, this method will not provide
            accurate results.

            \subidx{programs}{tslsq}
            \subidx{tslsq}{program}
            \subidx{logarithmic}{returns}
            \subidx{returns}{logarithmic}
            \noindent And by yet another comparison, using the program
            {\it tslsq}\/, which is briefly described in
            Chapter~\ref{programs}, with the -e -p options, to provide
            a formula for the least squares exponential fit to the
            time series data set presented in
            Figure~\ref{\SETLABEL:TS}:

            \begin{equation}
                bits = {\datatslsqepbits}
            \end{equation}

            \subidx{programs}{tslogreturns}
            \subidx{tslogreturns}{program}
            \subidx{logarithmic}{returns}
            \subidx{returns}{logarithmic}
            \noindent And finally, by comparison, from the
            {\it tslogreturns}\/ program, which is briefly described
            in Chapter~\ref{programs}, with the -p option, to provide
            a formula for the logarithmic returns of the time series
            data set presented in Figure~\ref{\SETLABEL:TS}:

            \begin{equation}
                bits = {\logreturns}
            \end{equation}

        \subsubsection{Calculation of Shannon Probability}
            \label{\SETLABEL:SP}

            \subidx{\market}{Shannon probability}
            Ideally, all of the values presented in
            Section~\ref{\SETLABEL:LR} would be equal. Using the
            logarithmic returns provided by the {\it tslogreturns}\/
            program, to be consistent
            with~\cite[pp. 81]{Peters:CAOITCM}

            \subidx{programs}{tslogreturns}
            \subidx{tslogreturns}{program}
            \begin{equation}
                2^{{\logreturns}t}
            \end{equation}

            \noindent therefore:
            \begin{equation}
                C\left(p\right) = {\logreturns}
            \end{equation}
            \subidx{programs}{tsshannon}
            \subidx{tsshannon}{program}
            \subidx{Shannon}{probability}
            \subidx{probability}{Shannon}
            \noindent and, {\it tsshannon}\/ {\logreturns} gives:
            \begin{equation}
                \label{\SETLABEL:F0}
                C\left({\shannonlogreturns}\right) = {\logreturns}
            \end{equation}
            \noindent therefore:
            \begin{eqnarray}
                2^{C\left({\shannonlogreturns}\right)} & = & 2^{\logreturns}\\
                                                       & = & {\twologreturns}\\
                                                       & = & {\twologreturnshundred}\%
            \end{eqnarray}
            \noindent and:
            \begin{eqnarray}
                2p - 1 & = & \left(2 \cdot {\shannonlogreturns}\right) - 1\\
                       & = & {\twopone}\\
                       \label{\SETLABEL:F1}
                       & = & {\twoponehundred}\%
            \end{eqnarray}

            \subidx{\market}{fiscal strategy}
            \subidx{markets}{analysis}
            \subidx{analysis}{markets}
            \subidx{strategy}{fiscal}
            \subidx{fiscal}{strategy}
            \subidx{\market}{fiscal strategy}
            \subidx{\market}{growth rate}
            Presuming the simplified assumptions outlined in
            Section~\ref{assumptions}, the ``typical'' organization
            operating in the {\market} executes a long term fiscal
            strategy, commensurate with the aggregate environment,
            that is to invest, every {\timescale}, in sufficient
            additional resources and infrastructure, to increase the
            manufacturing of goods and services by {\twoponehundred}\%
            of its rate of revenue returns, (per {\timescale}.) As a
            conceptual model, the remaining {\hundredtwoponehundred}\%
            will be held in ``reserve'' with a
            {\shannonlogreturnshundred}\% chance of making twice the
            {\twoponehundred}\% back, (and a
            {\hundredshannonlogreturnshundred}\% chance of making
            0.0,) in one {\timescale}, on the average, for an average
            growth in its rate of revenue returns, (per {\timescale},)
            of {\twologreturnshundred}\%, or a doubling of its rate of
            revenue returns, (per {\timescale},) in
            {\oneoverlogreturns} {\timescale}s.

        \subsubsection{Example Fixed Increment Approximation Fiscal Strategies}

            \subidx{\market}{fiscal strategy}
            \subidx{markets}{analysis}
            \subidx{analysis}{markets}
            \subidx{strategy}{fiscal}
            \subidx{fiscal}{strategy}
            \subidx{\market}{fiscal strategy}
            \subidx{\market}{growth rate}
            \subidx{\market}{management metric}
            \idx{management metric}
            A possible metric on the effectiveness of long term fiscal
            management could possibly be that if an investment of
            {\twoponehundred}\% per {\timescale} of the rate of
            revenue returns, (per {\timescale},) is made in resources
            and infrastructure, then the rate of revenue returns would
            be expected to increase by {\twologreturnshundred}\%, per
            {\timescale}, on average.

            Note that the metrics presented in this section are
            representative of the {\market} as an aggregate whole, and
            may or may not be accurate representations for any
            particular participant in the environment. Of interest to
            the participants in the environment would be a similar
            analysis of each product or service rendered in the
            marketplace.

            \subidx{\market}{fiscal strategy}
            \subidx{markets}{analysis}
            \subidx{analysis}{markets}
            \subidx{strategy}{fiscal}
            \subidx{fiscal}{strategy}
            \subidx{\market}{fiscal strategy}
            As a simple illustrative example, a company operating in
            this environment might obtain a credit line from a bank
            that is equal to {\twoponehundred}\% of its rate of
            revenue returns, (per {\timescale},) to finance additional
            operations. In this simple scenario, the company would use
            its revenue base as collateral for the loan. Some
            {\timescale}s, depending on the {\market}'s environment,
            the company's rate of revenue returns exceeds what was
            borrowed from the bank, and the loan is repaid in
            full. Other {\timescale}s, the company must default, and
            the bank seizes a portion of the company's revenue base to
            pay the delinquent loan. However, on the average, the
            company will expand its rate of revenue returns at
            {\twologreturnshundred}\% per {\timescale}.

            \subidx{\market}{fiscal strategy}
            \subidx{markets}{analysis}
            \subidx{analysis}{markets}
            \subidx{strategy}{fiscal}
            \subidx{fiscal}{strategy}
            \subidx{\market}{fiscal strategy}
            As another simple example, a company re-invests
            {\twoponehundred}\% of its rate of revenue returns, (per
            {\timescale},) in development, marketing, sales, and
            distribution of new products.  Although some products will
            be successful and the return on the investment will exceed
            the {\twoponehundred}\% per {\timescale} investment,
            others will not. However, on the average, the company will
            expand it gross rate of revenue returns at
            {\twologreturnshundred}\% per {\timescale}.

            \subidx{\market}{fiscal strategy}
            \subidx{markets}{analysis}
            \subidx{analysis}{markets}
            \subidx{strategy}{fiscal}
            \subidx{fiscal}{strategy}
            \subidx{\market}{fiscal strategy}
            \subidx{\market}{product portfolio}
            \subidx{\market}{product diversity}
            \subidx{\market}{product mix}
            \subidx{\market}{optimum number of products}
            \idx{product portfolio}
            \idx{product diversity}
            \idx{optimum number of products}
            \idx{product mix}

            As an example of ``product portfolio'' management, suppose
            a company re-invests {\twoponehundred}\% of its rate of
            revenue returns, (per {\timescale},) in development,
            marketing, sales, and distribution of new products.
            Further suppose that the company has two products, and a
            fractal analysis of the individual product rate of revenue
            return time series indicates that one product has a
            Shannon probability of 0.65, and the other has a Shannon
            probability of 0.55. Then the percentage of re-investment
            in the first product would be $(2 \cdot 0.65 - 1) \cdot
            {\twoponehundred}$, percent of the rate of revenue
            returns, and $(2 \cdot 0.55 - 1) \cdot {\twoponehundred}$
            percent for the second product, implying that the company
            should diversify its product line\footnote{The astute
            reader would note that the linear addition was used to add
            the contribution to development of each product. This is a
            ``near term'' interpretation. Actually, in general, the
            method used should be a root mean square process,
            dependent on the Hurst Coefficient, $H$, where
            $P_{total}^H = P_1^H + P_2^H + \cdots$, where $P_n$ is the
            contribution to each individual product. For a Brownian
            motion, or random walk type of fractal the Hurst
            Coefficient is a function of time into the future. For the
            ``near term,'' the Hurst coefficient is very near unity,
            meaning the summation process is linear. For the ``long
            term,'' $H \approx 0.5$, or a standard root mean square
            summation process should be used. If $H$ is $0.5$ then the
            market is termed a Brownian motion, or random walk
            process. If it is larger than 0.5, it is termed fractional
            Brownian motion process. For a random walk process, ``near
            term'' and ``far term'' are quantitatively differentiated
            on the Hurst Coefficient graph where $1 - \ln (t) = 0.5
            \cdot \ln (t)$, or when $\ln (t) = 2$, or $t =
            7.389\ldots$ See~\cite[pp. 67, 83-84]{Peters:CAOITCM}
            and~\cite[pp. 129, 159]{Schroeder} for particulars on the
            implications of the Hurst Coefficient and root mean square
            summation issues.}.  Note that this is a ``bet hedging''
            metric methodology, and assumes that the products have
            uncorrelated revenue return rates. If this re-investment
            methodology is not feasible, perhaps for strategic
            financial reasons, then the re-investment in both products
            should total the ${\twoponehundred}$\%, and the investment
            in each product should be made at a ratio of $\frac{(2
            \cdot 0.65 - 1)}{(2 \cdot 0.55 - 1)} = 3 : 1$,
            respectively. Note that this ``bet hedging'' can be used
            to define the optimal number of products that can be
            supported on the rate of revenue returns. If it assumed
            that all products are ``typical'' for the {\market}, as a
            standard bench mark, then the optimal number will be
            $\frac{1}{{\twopone}}$. Note that this is a
            ``theoretical'' value, since not all products are
            ``typical,'' and there may be strategic reasons, for
            example product leveraging, that may increase the number
            of products above the optimum. However, most of the
            revenue should come from the optimal number of products,
            since having more products will decrease the amount of the
            potential investment in each product, and having less than
            the optimum number of products will increase the risk that
            many of the products could suffer a ``down market''
            concurrently, impacting the rate of revenue returns.  As
            another interesting interpretation of the optimal
            ``hedging of bets,'' in product portfolio strategy, and
            considering the graph of the normalized increments
            presented in Figure~\ref{\SETLABEL:TF}, if the
            organization is running optimally, then these products
            will generate, at least in principle, one standard
            deviation, approximately $0.8413 = 84.13$\% of the future
            growth in rate of revenue returns. Naturally, these are
            approximations, and the values are an approximation to a,
            probably, complex process, and appropriate scrutiny should
            be exercised before making specific projections.  As yet
            another example of ``product portfolio'' management,
            consider the issue of product mix. In this interpretation,
            {\twoponehundred}\% of the product manufactured should be
            ``proprietary,'' while the rest is ``industry standard.''
            As yet another possibility, {\twoponehundred}\% of the
            product manufactured should be predatory into new markets,
            and the remainder in markets that are ``traditional'' for
            the company.

% Local Variables:
% TeX-parse-self: t
% TeX-auto-save: t
% TeX-master: "fractal.tex"
% End:


        %
% -----------------------------------------------------------------------------
%
% A license is hereby granted to reproduce this software source code and
% to create executable versions from this source code for personal,
% non-commercial use.  The copyright notice included with the software
% must be maintained in all copies produced.
%
% THIS PROGRAM IS PROVIDED "AS IS". THE AUTHOR PROVIDES NO WARRANTIES
% WHATSOEVER, EXPRESSED OR IMPLIED, INCLUDING WARRANTIES OF
% MERCHANTABILITY, TITLE, OR FITNESS FOR ANY PARTICULAR PURPOSE.  THE
% AUTHOR DOES NOT WARRANT THAT USE OF THIS PROGRAM DOES NOT INFRINGE THE
% INTELLECTUAL PROPERTY RIGHTS OF ANY THIRD PARTY IN ANY COUNTRY.
%
% Copyright (c) 1994-2006, John Conover, All Rights Reserved.
%
% Comments and/or bug reports should be addressed to:
%
%     john@email.johncon.com (John Conover)
%
% -----------------------------------------------------------------------------
%
% Revision: \RCSRevision \\
% Revision Time: \RCSTime UMT \\
% Revision Date: \RCSDate \\
% Revision Id: \RCSId \\
% Revision File: \RCSLog \\
\RCS $Revision: 0.0 $
\RCS $Date: 2006/01/20 04:38:13 $
\RCS $Id: companies.tex,v 0.0 2006/01/20 04:38:13 john Exp $
% $Log: companies.tex,v $
% Revision 0.0  2006/01/20 04:38:13  john
% Initial version
%
%
    \subsection{Number of Companies}
        \label{\SETLABEL:QNC}

        \subidx{\market}{number of companies}
        \subidx{number of companies}{analysis}
        \subidx{analysis}{number of companies}
        \subidx{Shannon}{probability}
        \subidx{probability}{Shannon}
        This section evaluates the approximate, or ``average,'' number
        of companies in the {\market}, and uses the method outlined in
        Chapter~\ref{general}, Section~\ref{aftsma}. Since the
        average, $avg_{ind}$, and the root mean square, $rms_{ind}$,
        of the normalized increments of the {\market} time series is
        \datafractionmean, and \datafractionrms respectively, the
        number of companies participating in the market can be
        calculated by Equation~\ref{ncompanies} to be {\ncompanies}.

        If this value seems consistent number of companies in the
        {\market}, within the assumptions outlined in
        Chapter~\ref{general}, Section~\ref{aftsma}, then it would
        seem that there is some circumstantial or indirect evidence
        that the companies participating in the {\market} are
        operating optimally, and the ``average'' Shannon probability,
        $P$ for each participating company would be, using
        Equation~\ref{pncompanies}, {\pncompanies}, which would be the
        value which should be used in Section~\ref{\SETLABEL:FS} for
        each participating company if market expansion was to be
        consistent with the rest of the industry. However, if the
        Shannon probability derived in Section~\ref{\SETLABEL:FS} is
        greater than the average Shannon probability for the companies
        participating in the {\market}, as derived in this section,
        then the market would, possibly, be exploitable with the
        fiscal strategy outlined in Section~\ref{\SETLABEL:FS}. The
        maximum exploitability for the {\market} is derived in
        Section~\ref{\SETLABEL:MAXSHANNON}, but it is probably of
        doubtful practicality.

        Note that these optimizations would maximize a company's
        market growth. Since there are probably many companies
        competing in the market place, this would not necessarily
        maximize a company's P\&L, as described in
        Chapter~\ref{general}, Section~\ref{ompl}. The Shannon
        probability that maximizes market share in the {\market} is
        \pncompanies, with several alternative solutions listed in the
        previous paragraph. However, these should be contrasted to the
        Shannon probability that maximizes a company's P\&L which is
        \avgrms~in the {\market}. In all cases, the fraction of the
        P\&L that should be ``wagered'' on the future, $f$, should be:

        \begin{equation}
            f = 2P - 1
        \end{equation}

        \noindent where $P$ is the particular Shannon probability
        chosen optimize a particular fiscal strategy. Interestingly,
        the measured Shannon probability of the {\market} would tend
        to indicate that the companies participating in the market
        have chosen a fiscal strategy that optimizes market growth, as
        opposed to capital growth.

        \subidx{\market}{increasing returns}
        \subidx{economic increasing returns}{\market}
        As interesting interpretation of these exploitive issues,
        since all three fiscal strategies will result in exponential
        market growth for every company participating in the market,
        is that they may represent, perhaps, an example of
        ``increasing returns.''

% Local Variables:
% TeX-parse-self: t
% TeX-auto-save: t
% TeX-master: "fractal.tex"
% End:


        %
% -----------------------------------------------------------------------------
%
% A license is hereby granted to reproduce this software source code and
% to create executable versions from this source code for personal,
% non-commercial use.  The copyright notice included with the software
% must be maintained in all copies produced.
%
% THIS PROGRAM IS PROVIDED "AS IS". THE AUTHOR PROVIDES NO WARRANTIES
% WHATSOEVER, EXPRESSED OR IMPLIED, INCLUDING WARRANTIES OF
% MERCHANTABILITY, TITLE, OR FITNESS FOR ANY PARTICULAR PURPOSE.  THE
% AUTHOR DOES NOT WARRANT THAT USE OF THIS PROGRAM DOES NOT INFRINGE THE
% INTELLECTUAL PROPERTY RIGHTS OF ANY THIRD PARTY IN ANY COUNTRY.
%
% Copyright (c) 1994-2006, John Conover, All Rights Reserved.
%
% Comments and/or bug reports should be addressed to:
%
%     john@email.johncon.com (John Conover)
%
% -----------------------------------------------------------------------------
%
% Revision: \RCSRevision \\
% Revision Time: \RCSTime UMT \\
% Revision Date: \RCSDate \\
% Revision Id: \RCSId \\
% Revision File: \RCSLog \\
\RCS $Revision: 0.0 $
\RCS $Date: 2006/01/20 04:38:13 $
\RCS $Id: operations.tex,v 0.0 2006/01/20 04:38:13 john Exp $
% $Log: operations.tex,v $
% Revision 0.0  2006/01/20 04:38:13  john
% Initial version
%
%
    \subsection{Fixed Increment Approximation for Operational Strategy}
        \label{\SETLABEL:OPS}.

        This section derives various values based on the ``average''
        of the normalized increments presented in
        Figure~\ref{\SETLABEL:TFA}. These values are an approximation
        to a, probably, complex process with a distribution shown in
        Figure~\ref{\SETLABEL:TF}. These values will be used in a
        fixed increment Brownian fractal analysis and simulation of
        the {\market}, and may, or may not, provide adequate accuracy
        for projections.

        \subidx{\market}{fiscal strategy}
        \subidx{\market}{Shannon probability}
        \subidx{strategy}{fiscal}
        \subidx{fiscal}{strategy}
        \subidx{Shannon}{probability}
        \subidx{probability}{Shannon}
        It should be noted that the analysis of fiscal strategy,
        presented in Section~\ref{\SETLABEL:FS}, is derived from the
        {\market} metrics and may, or may not, be maximally
        optimal. For the optimal fiscal strategy, which may be
        exploitable, see Section~\ref{\SETLABEL:MAXSHANNON}.

        \subidx{strategy}{exploitable}
        \subidx{exploitable}{strategy}
        \subidx{\market}{windows of opportunity}
        \idx{windows of opportunity}
        \subidx{decision}{obsolete}
        \subidx{obsolete}{decision}
        \subidx{decision}{timeliness}
        \subidx{timeliness}{decision}
        \subidx{rate of revenue returns}{forecast}
        \subidx{forecast}{rate of revenue returns}
        An additional exploitable strategy may be time itself.
        Equations~\ref{\SETLABEL:V},~\ref{\SETLABEL:R},
        and,~\ref{\SETLABEL:MA}, are, essentially, metrics on how fast
        a decision, which is based on information concerning the
        current status of the {\market}, becomes obsolete. Obviously,
        how long a decision is expected to remain relevant should be
        addressed as an operational necessity in strategic planning
        and project management. Figures~\ref{\SETLABEL:FN},
        and,~\ref{\SETLABEL:FF} compare methods of approximation of
        the ``forecastability'' of rate of revenue returns in the
        {\market} for the near term and far
        term~\cite[pp. 83-84]{Peters:CAOITCM}, respectively. As a
        general rule, caution must be exercised when making decisions
        that will span a time interval larger than the time interval
        where the ``forecastability'' of rate of revenue returns drops
        below 50\%. Beyond this time interval, the chances increase
        that the competitive and market forces will alter the market
        environment in a possibly detrimental unanticipated
        fashion. Obviously, there is significant advantage in
        ``timeliness'' of development, manufacturing, and distribution
        of products and services that are consistent with this
        temporal agenda. Automation of these processes, if executed
        consistently with this agenda, should be considered a
        competitive advantage.

        \subidx{strategy}{exploitable}
        \subidx{exploitable}{strategy}
        \subidx{rate of revenue returns}{forecast}
        \subidx{forecast}{rate of revenue returns}
        \idx{product life cycle}
        \idx{life cycle, product}
        In some sense, this temporal agenda defines the ``average''
        product or service life cycle in the {\market}. When the
        ``forecastability'' of rate of revenue returns drops below
        50\%, there is an even chance that the rate of revenue returns
        for the product or service will change in a detrimental
        fashion. If it is assumed that a product or service life cycle
        consists of a ramp up, a maintenence interval, and a ramp
        down, then, if all three life cycle intervals are equal, the
        product life cycle will be, approximately, three times the
        time interval where the ``forecastability'' of rate of revenue
        returns drops below 50\%. Although probably not an accurate
        prediction of product or service life cycle, the technique may
        be used as a conceptual approximation to the dynamics of
        ``market windows.\footnote{For example, consider the market
        for table salt. Since it has inelastic supply and demand
        curves, and is a necessary requirement for life, it would be
        expected that the Hurst coefficient would be very near
        unity---ignoring competitive pressures in the market. The
        predictability of the table salt market would, therefore, be
        expected to be relatively good, over time.}''  The conceptual
        approximation will probably predict a ``conservative'' or
        ``pessimistic'' value in relation to actual markets.

        \begin{figure}[ht]
            \begin{center}
                \begin{minipage}[t]{0.45\textwidth}
                    \epsfxsize=1.0\linewidth
                    \epsffile{\directory/datahurstlownear.eps}
                    \caption[{\market}, ``forecastability'' of near
                        term rate of revenue returns]{{\market},
                        ``forecastability'' of near term rate of
                        revenue returns. Although the error function
                        is the most accurate, for the near term,
                        $H^{t} = \thurstlow^{t}$ may be used as a
                        reliable metric of ``forecastability'' of the
                        rate of revenue returns.}
                    \label{\SETLABEL:FN}
                \end{minipage}
                \hfill
                \begin{minipage}[t]{0.45\textwidth}
                    \epsfxsize=1.0\linewidth
                    \epsffile{\directory/datahurstlowfar.eps}
                    \caption[{\market}, ``forecastability'' of far
                        term rate of revenue returns]{{\market},
                        ``forecastability'' of far term rate of
                        revenue returns. Although the error function
                        is the most accurate, for the far term,
                        $\frac{1}{\sqrt{t}}$ may be used as a reliable
                        metric of ``forecastability'' of the rate of
                        revenue returns.}
                    \label{\SETLABEL:FF}
                \end{minipage}
            \end{center}
        \end{figure}

        \idx{operations research}
        As an interesting interpretation of the data presented in
        Figure~\ref{\SETLABEL:FN}, there may be, perhaps, some
        applicability to such operational agendas as inventory
        control. Maintaining too little inventory, obviously, will
        create a situation where the organization can not exploit
        market expansion, and maintaining too much inventory,
        likewise, would over extend the company, creating unnecessary
        losses when the market contracts. The company should maintain
        inventory levels that do not exceed, from
        Equation~\ref{\SETLABEL:MA}, ${\thurstlow}^{n} = 0.5$
        {\timescale}s of operations. Since the optimal amount of
        inventory and, from Equation~\ref{\SETLABEL:V}, the variance
        of change in the rate of revenue returns in the future can be
        calculated, there may, perhaps, be some applicability to a
        forecasting methodology that can be incorporated into other
        areas of operations research, for example the linear algebras
        using simplex methodologies for optimization of manufacturing
        processes. Traditionally, these forecasts are made by the
        sales department, and are subject to various subjective
        biases.

% Local Variables:
% TeX-parse-self: t
% TeX-auto-save: t
% TeX-master: "fractal.tex"
% End:


        %
% -----------------------------------------------------------------------------
%
% A license is hereby granted to reproduce this software source code and
% to create executable versions from this source code for personal,
% non-commercial use.  The copyright notice included with the software
% must be maintained in all copies produced.
%
% THIS PROGRAM IS PROVIDED "AS IS". THE AUTHOR PROVIDES NO WARRANTIES
% WHATSOEVER, EXPRESSED OR IMPLIED, INCLUDING WARRANTIES OF
% MERCHANTABILITY, TITLE, OR FITNESS FOR ANY PARTICULAR PURPOSE.  THE
% AUTHOR DOES NOT WARRANT THAT USE OF THIS PROGRAM DOES NOT INFRINGE THE
% INTELLECTUAL PROPERTY RIGHTS OF ANY THIRD PARTY IN ANY COUNTRY.
%
% Copyright (c) 1994-2006, John Conover, All Rights Reserved.
%
% Comments and/or bug reports should be addressed to:
%
%     john@email.johncon.com (John Conover)
%
% -----------------------------------------------------------------------------
%
% Revision: \RCSRevision \\
% Revision Time: \RCSTime UMT \\
% Revision Date: \RCSDate \\
% Revision Id: \RCSId \\
% Revision File: \RCSLog \\
\RCS $Revision: 0.0 $
\RCS $Date: 2006/01/20 04:38:13 $
\RCS $Id: simulation.tex,v 0.0 2006/01/20 04:38:13 john Exp $
% $Log: simulation.tex,v $
% Revision 0.0  2006/01/20 04:38:13  john
% Initial version
%
%
    \subsection{Simulation of Fixed Increment Approximation for Fiscal Strategy}
        \label{\SETLABEL:TSUNFAIRBROWNIAN}

        \subidx{\market}{market simulation}
        The data in this section is presented in tabular form in
        Section~\ref{\SETLABELREF:SIM}.
        Figure~\ref{\SETLABEL:TSUNFAIRBROWNIAN0} represents a
        constructional simulation of the time series data presented in
        Figure~\ref{\SETLABEL:TS}. The program {\it
        tsunfairbrownian}\/, which is briefly described in
        appendix~\ref{programs}, was used in the reconstruction. The
        reconstructed data is superimposed on the original time series
        data.  The program, {\it tsunfairbrownian}\/, essentially,
        constructs the new time series as a Brownian fractal with
        fixed increments---the value of the fixed increment is derived
        from the root mean square average of the normalized increments
        presented in Figure~\ref{\SETLABEL:TF}. The ``quality'' of
        such a reconstruction should be subject to adequate scepticism
        and scrutiny since, in all probability, the normalized
        increments presented in Figure~\ref{\SETLABEL:TF} represent a
        relatively complex process, that may not be ``modeled'' with
        such a simple methodology.

        As a further comparison of the the constructional simulation
        with the original time series data,
        Figure~\ref{\SETLABEL:TSUNFAIRBROWNIAN1} presents a normalized
        histogram of the normalized increments of the reconstructed
        time series, superimposed on the normalized histogram
        presented in Figure~\ref{\SETLABEL:NH}.

        \subidx{\market}{fiscal strategy, simulation}
        \subidx{markets}{simulation}
        \subidx{simulation}{markets}
        \subidx{strategy}{fiscal, simulation}
        \subidx{fiscal}{strategy, simulation}
        \subidx{programs}{tsunfairbrownian}
        \subidx{tsunfairbrownian}{program}
        \begin{figure}[ht]
            \begin{center}
                \begin{minipage}[t]{0.45\textwidth}
                    \epsfxsize=1.0\linewidth
                    \epsffile{\directory/tsunfairbrownian-f.eps}
                    \caption[{\market}, Time series data, empirical and
                        simulated]{{\market}, Time series data, empirical
                        and simulated, using the program {\it tsunfairbrownian}\/
                        with f = {\datafractionrms}. This data is
                        superimposed on the data presented in
                        Figure~\ref{\SETLABEL:TS}.}
                    \label{\SETLABEL:TSUNFAIRBROWNIAN0}
                \end{minipage}
                \hfill
                \begin{minipage}[t]{0.45\textwidth}
                    \epsfxsize=1.0\linewidth
                    \epsffile{\directory/tsunfairbrownian-f.tsfraction.tsnormal-s30.eps}
                    \caption[{\market}, normalized histogram,
                        empirical and simulated]{{\market}, normalized
                        histogram of the normalized increments of the
                        time series data shown in
                        Figure~\ref{\SETLABEL:TSUNFAIRBROWNIAN0},
                        empirical and simulated.  The empirical data
                        has a mean of {\datafractionmean}, with a
                        standard deviation of {\datafractionstddev}.
                        By comparison, the simulated data has a mean
                        of {\tsunfairbrownianfractionmean} with a
                        standard deviation of
                        {\tsunfairbrownianfractionstddev}. This data
                        is superimposed on the data presented in
                        Figure~\ref{\SETLABEL:NH}. The area under the
                        four curves is identical.}
                    \label{\SETLABEL:TSUNFAIRBROWNIAN1}
                \end{minipage}
            \end{center}
        \end{figure}

% Local Variables:
% TeX-parse-self: t
% TeX-auto-save: t
% TeX-master: "fractal.tex"
% End:


        %
% -----------------------------------------------------------------------------
%
% A license is hereby granted to reproduce this software source code and
% to create executable versions from this source code for personal,
% non-commercial use.  The copyright notice included with the software
% must be maintained in all copies produced.
%
% THIS PROGRAM IS PROVIDED "AS IS". THE AUTHOR PROVIDES NO WARRANTIES
% WHATSOEVER, EXPRESSED OR IMPLIED, INCLUDING WARRANTIES OF
% MERCHANTABILITY, TITLE, OR FITNESS FOR ANY PARTICULAR PURPOSE.  THE
% AUTHOR DOES NOT WARRANT THAT USE OF THIS PROGRAM DOES NOT INFRINGE THE
% INTELLECTUAL PROPERTY RIGHTS OF ANY THIRD PARTY IN ANY COUNTRY.
%
% Copyright (c) 1994-2006, John Conover, All Rights Reserved.
%
% Comments and/or bug reports should be addressed to:
%
%     john@email.johncon.com (John Conover)
%
% -----------------------------------------------------------------------------
%
% Revision: \RCSRevision \\
% Revision Time: \RCSTime UMT \\
% Revision Date: \RCSDate \\
% Revision Id: \RCSId \\
% Revision File: \RCSLog \\
\RCS $Revision: 0.0 $
\RCS $Date: 2006/01/20 04:38:13 $
\RCS $Id: maximum.tex,v 0.0 2006/01/20 04:38:13 john Exp $
% $Log: maximum.tex,v $
% Revision 0.0  2006/01/20 04:38:13  john
% Initial version
%
%
    \subsection{Simulation of Fixed Increment Approximation for Optimally Maximal Fiscal Strategy}
        \label{\SETLABEL:MAXSHANNON}
        \subidx{\market}{fiscal strategy, simulation}
        \subidx{\market}{maximum Shannon probability}
        \subidx{markets}{simulation}
        \subidx{simulation}{markets}
        \subidx{strategy}{optimum fiscal, simulation}
        \subidx{fiscal}{optimum strategy, simulation}
        \subidx{programs}{tsunfairbrownian}
        \subidx{tsunfairbrownian}{program}
        \subidx{Shannon}{probability}
        \subidx{probability}{Shannon}

        \subidx{strategy}{exploitable}
        \subidx{exploitable}{strategy}
        \subidx{programs}{tsshannonmax}
        \subidx{tsshannonmax}{program}
        \subidx{programs}{tsunfairbrownian}
        \subidx{tsunfairbrownian}{program}
        \subidx{strategy}{fiscal}
        \subidx{fiscal}{strategy}
        The data in this section is presented in tabular form in
        Section~\ref{\SETLABELREF:MAXSHANNON}. One of the issues of
        analysis, as mentioned in Section~\ref{\SETLABEL:OPS}, is to
        determine the maximum Shannon probability for the time series
        presented in Figure~\ref{\SETLABEL:TS}. Potentially, this
        could be exploited with an aggressive fiscal
        strategy. Figure~\ref{\SETLABEL:SHANNONMAX0} is a graph of the
        output of the {\it tsshannonmax}\/ program, which is described
        briefly in appendix~\ref{programs}. The maximum of this
        function is the maximum Shannon probability for the time
        series data presented in Figure~\ref{\SETLABEL:TS}.
        Figure~\ref{\SETLABEL:SHANNONMAX1} was constructed using {\it
        tsunfairbrownian}\/ program, which is also described in
        appendix~\ref{programs}, with the maximum Shannon probability,
        and the time series data presented in
        Figure~\ref{\SETLABEL:TS}. This represents a ``what if'' the
        investment strategy was changed from a Shannon probability of
        {\shannonlogreturns}, as derived in Section~\ref{\SETLABEL:SP}
        to {\shannonmax}. This process, essentially, extracts the
        random statistical data from the time series presented in
        Figure~\ref{\SETLABEL:TS}, and constructs a new time series,
        using the random statistical data, with a different investment
        strategy.  The program, {\it tsunfairbrownian}\/, essentially,
        constructs the new time series as a Brownian fractal with
        fixed increments.  The ``quality'' of such a reconstruction
        should be subject to adequate scepticism and scrutiny since,
        in all probability, the increments in the original data
        represent a relatively complex process, that may not be
        ``modeled'' with such a simple methodology.

        \begin{figure}[ht]
            \begin{center}
                \begin{minipage}[t]{0.45\textwidth}
                    \epsfxsize=1.0\linewidth
                    \epsffile{\directory/data.tsshannonmax.eps}
                    \caption[{\market}, maximum rate of revenue
                        returns] {{\market}, maximum rate of revenue
                        returns, per {\timescale}, vs. Shannon
                        probability. The maximum rate of revenue
                        returns, per {\timescale}, occurs at a Shannon
                        probability of {\shannonmax}.}
                    \label{\SETLABEL:SHANNONMAX0}
                \end{minipage}
                \hfill
                \begin{minipage}[t]{0.45\textwidth}
                    \epsfxsize=1.0\linewidth
                    \epsffile{\directory/data.tsshannonmax-p.tsunfairbrownian-p.eps}
                    \caption[{\market}, maximum rate of revenue
                        returns] {{\market}, maximum rate of revenue
                        returns, per {\timescale}, at a Shannon
                        probability, of {\shannonmax}, corresponding
                        to a ``wager'' fraction of {\twoponemax}.}
                    \label{\SETLABEL:SHANNONMAX1}
                \end{minipage}
            \end{center}
        \end{figure}

        \subidx{fractional}{Brownian motion}
        \subidx{Brownian motion}{fractional}
        \subidx{Shannon}{probability}
        \subidx{probability}{Shannon}
        \subidx{programs}{tsshannonmax}
        \subidx{tsshannonmax}{program}
        If it is assumed that the time series data set, presented in
        Figure~\ref{\SETLABEL:TS}, constitutes classical Brownian
        motion, then the Shannon probability can be calculated by
        counting the total number of {\timescale}s that the {\market}
        movement was positive, and dividing by the total number of
        {timescale}s represented in the time series. This quotient is
        {\pmax}, as compared with the predicted value from the program
        {\it tsshannonmax}\/ of {\shannonmax}.

% Local Variables:
% TeX-parse-self: t
% TeX-auto-save: t
% TeX-master: "fractal.tex"
% End:


        %
% -----------------------------------------------------------------------------
%
% A license is hereby granted to reproduce this software source code and
% to create executable versions from this source code for personal,
% non-commercial use.  The copyright notice included with the software
% must be maintained in all copies produced.
%
% THIS PROGRAM IS PROVIDED "AS IS". THE AUTHOR PROVIDES NO WARRANTIES
% WHATSOEVER, EXPRESSED OR IMPLIED, INCLUDING WARRANTIES OF
% MERCHANTABILITY, TITLE, OR FITNESS FOR ANY PARTICULAR PURPOSE.  THE
% AUTHOR DOES NOT WARRANT THAT USE OF THIS PROGRAM DOES NOT INFRINGE THE
% INTELLECTUAL PROPERTY RIGHTS OF ANY THIRD PARTY IN ANY COUNTRY.
%
% Copyright (c) 1994-2006, John Conover, All Rights Reserved.
%
% Comments and/or bug reports should be addressed to:
%
%     john@email.johncon.com (John Conover)
%
% -----------------------------------------------------------------------------
%
% Revision: \RCSRevision \\
% Revision Time: \RCSTime UMT \\
% Revision Date: \RCSDate \\
% Revision Id: \RCSId \\
% Revision File: \RCSLog \\
\RCS $Revision: 0.0 $
\RCS $Date: 2006/01/20 04:38:13 $
\RCS $Id: verification.tex,v 0.0 2006/01/20 04:38:13 john Exp $
% $Log: verification.tex,v $
% Revision 0.0  2006/01/20 04:38:13  john
% Initial version
%
%
    \subsection{Qualitative Verification of Fixed Increment Approximation Analysis}
        \label{\SETLABEL:QVA}

        \subidx{\market}{verification of analysis}
        \subidx{verification}{analysis}
        \subidx{analysis}{verification}
        \subidx{quality}{of analysis}
        \subidx{verification}{of methodology}
        \subidx{methodology}{verification of}
        \subidx{Shannon}{probability}
        \subidx{probability}{Shannon}

        This section evaluates various values based on the ``average''
        of the normalized increments presented in
        Figure~\ref{\SETLABEL:TFA}. These values are an approximation
        to a, probably, complex process with a distribution shown in
        Figure~\ref{\SETLABEL:TF}. These values will be used in a
        fixed increment Brownian fractal analysis of the {\market},
        and may, or may not, provide adequate accuracy for
        projections.

        The data in this section is presented in tabular form in
        sections~\ref{\SETLABELREF:VI1} and~\ref{\SETLABELREF:VI2}.
        As a subjective evaluation of the ``quality'' of the analysis
        of the {\market}, from Chapter~\ref{methodology},
        Equation~\ref{metricvalues1}, and using the mean and root mean
        square values of the normalized increments of the time series
        data presented in Figure~\ref{\SETLABEL:TS} from
        Figure~\ref{\SETLABEL:TF}, and the Shannon probability as
        calculated by counting the total number of {\timescale}s that
        the {\market} movement was positive, as presented in
        Section~\ref{\SETLABEL:MAXSHANNON}:

        \begin{eqnarray}
                  P & \approx & \frac{\frac{avg}{rms} + 1}{2}\\
            {\pmax} & \approx & \frac{\frac{\datafractionmean}{\datafractionrms} + 1}{2}\\
            {\pmax} & \approx & {\avgrms}
            \label{\SETLABEL:AVGS}
        \end{eqnarray}

        \subidx{Shannon}{probability}
        \subidx{probability}{Shannon}
        \noindent and comparing these values to the Shannon
        probability, as found by the {\it tsshannonmax}\/ program, which
        iterates for a maximum:

        \begin{eqnarray}
            {\pmax} \approx {\avgrms} \approx {\shannonmax}
        \end{eqnarray}

        \subidx{logarithmic}{returns}
        \subidx{returns}{logarithmic}
        In addition, the different methods of calculating the
        logarithmic returns, presented in Section~\ref{\SETLABEL:FS},
        should be compared. The four methods used were the mean of
        Figure~\ref{\SETLABEL:TF}, the constant in the least squares
        approximation to Figure~\ref{\SETLABEL:TF}, the least squares
        exponential approximation to Figure~\ref{\SETLABEL:TS}, and
        the logarithmic returns of Figure~\ref{\SETLABEL:TS}, derived
        as the mean of the logarithms of the quotients of the
        increments. The values for each of the methods are,
        respectively:

        \begin{equation}
            \datafractionmeanbits \approx \datafractionconstantbits \approx \datatslsqepbits \approx \logreturns
        \end{equation}

        It is implied in Section~\ref{\SETLABEL:FS},
        Subsection~\ref{\SETLABEL:SP} and in
        Section~\ref{\SETLABEL:TSUNFAIRBROWNIAN} that, a Brownian
        motion with fixed increments fractal may ``model'' the
        {\market}. Using Equation~\ref{stddev9} from
        Chapter~\ref{general}, Section~\ref{abmfi}:

        \begin{eqnarray}
                                    rms \left(2P - 1\right) & \approx & \frac{\sigma \left(2P - 1\right)}{2 \sqrt{P\left(1 - P\right)}}\\
            \datafractionrms \left(2 \cdot \pmax - 1\right) & \approx & \frac{\datafractionstddev \left(2 \cdot \pmax - 1\right)}{2\sqrt{\pmax \left(1 - \pmax\right)}}\\
                       \datafractionrms \cdot \twopminusone & \approx & \datafractionstddev \cdot \twopx\\
                                                      \rmsp & \approx & \sigmap
        \end{eqnarray}

        \noindent and, equating to the mean:

        \begin{equation}
            \datafractionmean \approx \rmsp \approx \sigmap
        \end{equation}

        \subidx{Shannon}{probability}
        \subidx{probability}{Shannon}
        \noindent where, as in Equation~\ref{\SETLABEL:AVGS} using the
        mean, root mean square, and standard deviation values of the
        normalized increments of the time series data presented in
        Figure~\ref{\SETLABEL:TS} from Figure~\ref{\SETLABEL:TF}, and
        the Shannon probability as calculated by counting the total
        number of {\timescale}s that the {\market} movement was
        positive, as presented in Section~\ref{\SETLABEL:MAXSHANNON}.

        As a final qualitative comparison, the absolute value of the
        normalized increments should be the same as the root mean
        square value\footnote{The absolute value of the normalized
        increments, when averaged, is related to the root mean square
        of the increments by a constant. If the normalized increments
        are a fixed increment, the constant is unity. If the
        normalized increments have a Gaussian distribution, the
        constant is $\approx 0.8$ depending on the accuracy of of
        ``fit'' to a Gaussian distribution.}, where the absolute value
        is presented in Figure~\ref{\SETLABEL:TFA}, and the root mean
        square value is presented in Figure~\ref{\SETLABEL:TF}:

        \begin{equation}
            \datafractionabsmean \approx \datafractionrms
        \end{equation}

        Note, that if the {\market} could be ``modeled'' as a Brownian
        motion with fixed increments fractal, then the standard
        deviation of the absolute value of the normalized increments
        of the time series data presented in Figure~\ref{\SETLABEL:TS}
        from Figure~\ref{\SETLABEL:TF} should be zero. It is
        $\datafractionabsstddev$.

% Local Variables:
% TeX-parse-self: t
% TeX-auto-save: t
% TeX-master: "fractal.tex"
% End:


    \renewcommand{\market}{Simulated Equity Market Index}
    \renewcommand{\directory}{../markets/tsgaussian.tsmath.tsmath.tsunfraction}
    \renewcommand{\datafractionmean}{0.008052}
\renewcommand{\datafractionmeanbits}{0.011570}
\renewcommand{\datafractionmeanq}{0.002684}
\renewcommand{\datafractionmeanbitsq}{0.003867}
\renewcommand{\datafractionstddev}{0.038579}
\renewcommand{\datafractionrms}{0.039311}
\renewcommand{\avgrms}{0.602414}
\renewcommand{\ncompanies}{5.210454}
\renewcommand{\pncompanies}{0.544866}
\renewcommand{\datafractionabsmean}{0.029745}
\renewcommand{\datafractionabsstddev}{0.025769}
\renewcommand{\datafractionconstant}{0.010041}
\renewcommand{\datafractionconstantbits}{0.014414}
\renewcommand{\datafractionconstantq}{0.003347}
\renewcommand{\datafractionconstantbitsq}{0.004821}
\renewcommand{\datafractionslope}{-0.000021}
\renewcommand{\datafractionabsconstant}{0.035145}
\renewcommand{\datafractionabsslope}{-0.000057}
\renewcommand{\hurstall}{0.659558}
\renewcommand{\hurstlow}{0.707509}
\renewcommand{\hurstlowtwo}{1.415018}
\renewcommand{\hurstlowhundred}{70.750900}
\renewcommand{\hcalcall}{0.184942}
\renewcommand{\hcalclow}{0.102042}
\renewcommand{\shannonmax}{0.604167}
\renewcommand{\twoponemax}{0.208334}
\renewcommand{\logreturns}{0.010456}
\renewcommand{\twologreturns}{1.007274}
\renewcommand{\twologreturnshundred}{0.727387}
\renewcommand{\oneoverlogreturns}{95.638868}
\renewcommand{\pmax}{0.602094}
\renewcommand{\twopminusone}{0.204188}
\renewcommand{\rmsp}{0.008027}
\renewcommand{\twopx}{0.208583}
\renewcommand{\sigmap}{0.008047}
\renewcommand{\tsunfairbrownianfractionmean}{0.007862}
\renewcommand{\tsunfairbrownianfractionstddev}{0.038619}
\renewcommand{\shannonlogreturns}{0.560125}
\renewcommand{\shannonlogreturnshundred}{56.012500}
\renewcommand{\twopone}{0.120250}
\renewcommand{\twoponehundred}{12.025000}
\renewcommand{\hundredtwoponehundred}{87.975000}
\renewcommand{\hundredshannonlogreturnshundred}{43.987500}
\renewcommand{\datatslsqepbits}{0.007623}
\renewcommand{\thurstall}{0.633980}
\renewcommand{\thurstlow}{0.710108}
\renewcommand{\thurstlowtwo}{1.420216}
\renewcommand{\thurstlowhundred}{71.010800}
\renewcommand{\thcalcall}{0.247886}
\renewcommand{\thcalclow}{0.171737}
\renewcommand{\chisquared}{2.862000}
\renewcommand{\critical}{42.557000}

    \renewcommand{\timescale}{month}
    \subidx{market}{\market}
    \idx{\market}

    \section{\market}

        \renewcommand{\SETLABEL}{\LABPRE:SEMIX}
        \renewcommand{\SETLABELQ}{\LABPRE:SEMIXQ}
        \label{\SETLABEL}
        \renewcommand{\SETLABELREF}{\LABPREREF:SEMIX}

        \subidx{tsunfraction}{program}
        \subidx{programs}{tsunfraction}
        \subidx{tsgaussian}{program}
        \subidx{programs}{tsgaussian}
        \subidx{tsmath}{program}
        \subidx{programs}{tsmath}

        For the analysis, the data was in the directory
        {\directory}\footnote{As a simulation model, the programs {\it
        tsgaussian}\/, {\it tsmath}\/, and {\it tsunfraction}\/ were
        run to make a time series data file, with the following
        parameters:

        \vspace{0.1in}
            {\noindent}tsgaussian 5000 | tsmath -t -m 0.01 | tsmath -t -a 0.0003 | tsunfraction > data
        \vspace{0.1in}

        \noindent to make a time series of $5000$ elements, with a
        Shannon probability of $0.515$, to demonstrate an alternative
        method of constructing fractal time series. The average of the
        normalized increments is $0.0003$, and the root mean square
        value of the normalize increments is $0.01$, which is
        ``typical'' for an equity market time series.  The data is by
        {\timescale}s.}.

        The program {\it tsunfraction}\/, which is described briefly
        in appendix~\ref{programs}, provides the inverse function of
        the program {\it tsunfraction}\/. This allows a time series
        that contains normalized increments to be constructed, and
        then, cumulative summed into a fractal time series by the
        program {\it tsunfraction}\/.

        The data in this section is presented in tabular form in
        Section~\ref{\SETLABELREF}. Note that in this analysis, the
        rate of revenue returns means the increase or decrease in the
        cumulative sum of the {\market}. This is included for
        ``theoretical'' comparative purposes.

        %
% -----------------------------------------------------------------------------
%
% A license is hereby granted to reproduce this software source code and
% to create executable versions from this source code for personal,
% non-commercial use.  The copyright notice included with the software
% must be maintained in all copies produced.
%
% THIS PROGRAM IS PROVIDED "AS IS". THE AUTHOR PROVIDES NO WARRANTIES
% WHATSOEVER, EXPRESSED OR IMPLIED, INCLUDING WARRANTIES OF
% MERCHANTABILITY, TITLE, OR FITNESS FOR ANY PARTICULAR PURPOSE.  THE
% AUTHOR DOES NOT WARRANT THAT USE OF THIS PROGRAM DOES NOT INFRINGE THE
% INTELLECTUAL PROPERTY RIGHTS OF ANY THIRD PARTY IN ANY COUNTRY.
%
% Copyright (c) 1994-2006, John Conover, All Rights Reserved.
%
% Comments and/or bug reports should be addressed to:
%
%     john@email.johncon.com (John Conover)
%
% -----------------------------------------------------------------------------
%
% Revision: \RCSRevision \\
% Revision Time: \RCSTime UMT \\
% Revision Date: \RCSDate \\
% Revision Id: \RCSId \\
% Revision File: \RCSLog \\
\RCS $Revision: 0.0 $
\RCS $Date: 2006/01/20 04:38:13 $
\RCS $Id: fraction.tex,v 0.0 2006/01/20 04:38:13 john Exp $
% $Log: fraction.tex,v $
% Revision 0.0  2006/01/20 04:38:13  john
% Initial version
%
%
    \subsection{Time Series Increments Analysis}
        \label{\SETLABEL:TSA}

        \subidx{\market}{Time series analysis}
        \subidx{time series}{increments}
        \subidx{time series}{analysis}
        \subidx{cumulative sum}{analysis}
        \subidx{analysis}{cumulative sum}
        \subidx{analysis}{random process}
        \subidx{random process}{analysis}
        \subidx{Gaussian}{increments}
        \subidx{increments}{Gaussian}
        \subidx{Brownian}{motion, fractional}
        \subidx{fractional}{Brownian motion}
        \subidx{fractal}{Brownian motion}
        The data in this section is presented in tabular form in
        Section~\ref{\SETLABELREF:TSA}.  Figure~\ref{\SETLABEL:TS} is
        a graph of the time series data for the {\market}.

        \subidx{increments}{normalized}
        \subidx{normalized}{increments}
        \subidx{programs}{tsfraction}
        \subidx{tsfraction}{program}
        Figure~\ref{\SETLABEL:TF} is a graph of the normalized
        increments of the time series data presented in
        Figure~\ref{\SETLABEL:TS}. The data presented was made by
        running the program {\it tsfraction}\/ on the time series
        data. The program {\it tsfraction}\/ is described briefly in
        Appendix~\ref{programs}, and subtracts the previous value from
        the next value, dividing this difference by the previous
        value, for each element in the time series data. The new time
        series contains the instantaneous change in the rate of
        revenue returns, divided by the magnitude of the instantaneous
        rate of revenue returns.

        \subidx{mean}{standard deviation}
        \subidx{standard deviation}{mean}
        \idx{root mean square}
        \idx{least squares approximation}
        \begin{figure}[ht]
            \begin{center}
                \begin{minipage}[t]{0.45\textwidth}
                    \epsfxsize=1.0\linewidth
                    \epsffile{\directory/data.eps}
                    \caption{{\market}, time series data.}
                    \label{\SETLABEL:TS}
                    \label{\SETLABELQ:TS}
                \end{minipage}
                \hfill
                \begin{minipage}[t]{0.45\textwidth}
                    \epsfxsize=1.0\linewidth
                    \epsffile{\directory/data.tsfraction.eps}
                    \caption[{\market}, normalized
                        increments]{{\market}, normalized increments
                        of the time series data presented in
                        Figure~\ref{\SETLABEL:TS}. The mean is
                        {\datafractionmean} with a standard deviation
                        of {\datafractionstddev}. The formula for the
                        least squares approximation is
                        ${\datafractionconstant} +
                        {\datafractionslope}t$, and the root mean
                        squared value is {\datafractionrms}. The
                        graph, labeled ``data\-.tsfraction\-.tsrms,''
                        is the running root mean square, and
                        ``data\-.tsfraction\-.tsavg'' is the running
                        average of the normalized increments.  This
                        graph is the fraction of change in the time
                        series, as a function of time. Note that the
                        slope of the mean, {\datafractionslope}, is
                        the coefficient of the nonlinearity term in
                        the normalized increments. See
                        Chapter~\ref{general}, Section~\ref{nlextend}
                        for a possible application of the logistic
                        function to this data set.}
                    \label{\SETLABEL:TF}
                    \label{\SETLABELQ:TF}
                \end{minipage}
            \end{center}
        \end{figure}

        \subidx{absolute value}{increments}
        \subidx{increments}{absolute value}

        Figure~\ref{\SETLABEL:TFA} is a graph of the absolute value of
        the normalized increments of the time series data presented in
        Figure~\ref{\SETLABEL:TF}. The data presented was made by
        running the Unix utility sed(1) on the normalized increments
        time series data to remove the negative signs. This is an
        absolute value procedure.  The resulting time series contains
        the absolute value of the instantaneous change in the rate of
        revenue returns, divided by the magnitude of the instantaneous
        rate of revenue returns\footnote{The absolute value of the
        normalized increments, when averaged, is related to the root
        mean square of the increments by a constant. If the normalized
        increments are a fixed increment, the constant is unity. If
        the normalized increments have a Gaussian distribution, the
        constant is $\approx 0.8$ depending on the accuracy of of
        ``fit'' to a Gaussian distribution.}.

        \subidx{histogram}{normalized}
        \subidx{normalized}{histogram}
        \subidx{programs}{tsnormal}
        \subidx{tsnormal}{program}
        \subidx{mean}{standard deviation}
        \subidx{standard deviation}{mean}
        \idx{root mean square}
        \idx{least squares approximation}
        \subidx{\market}{analysis of increments}
        Figure~\ref{\SETLABEL:NH} is the normalized histogram of the
        normalized increments of the time series data shown in
        Figure~\ref{\SETLABEL:TF}. The abscissa is 3 $\sigma$ limits,
        and the area under the two curves is identical. The data for
        this figure was produced by the program {\it tsnormal}\/,
        which is described briefly in Appendix~\ref{programs}.

        \begin{figure}[ht]
            \begin{center}
                \begin{minipage}[t]{0.45\textwidth}
                    \epsfxsize=1.0\linewidth
                    \epsffile{\directory/data.tsfraction.abs.eps}
                    \caption[{\market}, absolute value of the
                        normalized increments]{{\market}, absolute
                        value of the normalized increments of the time
                        series data presented in
                        Figure~\ref{\SETLABEL:TF}.  The mean is
                        {\datafractionabsmean} with a standard
                        deviation of {\datafractionabsstddev}. The
                        formula for the least squares approximation is
                        ${\datafractionabsconstant} +
                        {\datafractionabsslope}t$, and the root mean
                        square value, from Figure~\ref{\SETLABEL:TF},
                        is {\datafractionrms}.  The graph, labeled
                        ``data\-.tsfraction\-.tsrms,'' is the running
                        root mean square, and
                        ``data\-.tsfraction\-.tsavg'' is the running
                        average of the normalized increments presented
                        in Figure~\ref{\SETLABEL:TF}, superimposed
                        here for convenience. This graph is the
                        absolute value of the fraction of change in
                        the time series, as a function of time.}
                    \label{\SETLABEL:TFA}
                    \label{\SETLABELQ:TFA}
                \end{minipage}
                \hfill
                \begin{minipage}[t]{0.45\textwidth}
                    \epsfxsize=1.0\linewidth
                    \epsffile{\directory/data.tsfraction.tsnormal-s30.eps}
                    \caption[{\market}, normalized histogram of the
                        normalized increments]{{\market}, normalized
                        histogram of the normalized increments of the
                        time series data shown in
                        Figure~\ref{\SETLABEL:TF}.  The data has a
                        mean of {\datafractionmean}, with a standard
                        deviation of {\datafractionstddev}.  The area
                        under the two curves is identical. The
                        $\chi^2$ value of the observed and expected
                        values of the two curves is {\chisquared},
                        with a critical value of {\critical}.}
                    \label{\SETLABEL:NH}
                \end{minipage}
            \end{center}
        \end{figure}

        \subidx{programs}{tsXsquared}
        \subidx{tsXsquared}{program}
        \subidx{\market}{chi-squared values of increments}
        The program {\it tsXsquared}\/, which is briefly described in
        appendix~\ref{programs}, was used to derive the $\chi^2$
        statistics for the data presented in
        Figure~\ref{\SETLABEL:NH}.

        \subidx{programs}{tsstatest}
        \subidx{tsstatest}{program}
        \subidx{\market}{statistical estimates}

        Figure~\ref{\SETLABEL:SE} is the statistical estimate for the
        data presented in Figure~\ref{\SETLABEL:TF}, as derived by the
        program {\it tsstatest}\/, which is briefly described in
        appendix~\ref{programs}.

        \begin{figure}[ht]
            \begin{center}
                \begin{minipage}[t]{\textwidth}
                    \center{\fbox{\parbox{0.9\textwidth}{\XXX{\directory/data.tsstatest-f0.1-c0.9-i.tex}}}}
                    \caption[{\market}, statistical estimates of the
                        normalized increments]{{\market}, statistical
                        estimates of the normalized increments of the
                        time series shown in Figure~\ref{\SETLABEL:TF}.
                        The table was produced with the {\it
                        tsstatest}\/ program, and illustrates the
                        size of the data set required for a confidence
                        level of 90\%, with an error estimate of $\pm$
                        10\%, or alternately, the error estimate on
                        the time series shown in Figure~\ref{\SETLABEL:TF}.}
                    \label{\SETLABEL:SE}
                \end{minipage}
            \end{center}
        \end{figure}

        Note that the data set size estimations, as produced by the
        {\it tsstatest}\/ program, are probably very conservative,
        depending on the magnitude of the Shannon probability, $P =
        \shannonlogreturns$, as derived in
        Section~\ref{\SETLABEL:SP}. See Chapter~\ref{general},
        Section~\ref{serdss} for possible alternative methodologies
        for addressing the analysis of fractal time series with
        limited data set sizes. Depending on the magnitude of the
        Shannon probability, $P$, these estimates can be several
        orders of magnitude too high.

        \subidx{derivative of increments}{normalized}
        \subidx{normalized}{derivative of increments}
        \subidx{programs}{tsderivative}
        \subidx{tsderivative}{program}
        Figure~\ref{\SETLABEL:TF1} is the normalized histogram of the
        first derivative of the normalized increments of the time
        series data shown in Figure~\ref{\SETLABEL:TF}. In principle,
        if the distribution of the normalized increments presented in
        Figure~\ref{\SETLABEL:NH} is Gaussian in nature, this
        distribution would be similar to ``white noise,'' as presented
        in appendix~\ref{programs}, Figure~\ref{whiteexample}. The
        data was generated by the {\it tsderivative}\/ program, which
        is briefly described in
        appendix~\ref{programs}. Figure~\ref{\SETLABEL:TF2} is the
        normalized histogram of the second derivative of the
        normalized increments of the time series data shown in
        Figure~\ref{\SETLABEL:TF}. In principle, if the distribution
        of the normalized increments presented in
        Figure~\ref{\SETLABEL:NH} is an integrated Gaussian
        distribution in nature, this distribution would be similar to
        ``white noise,'' as presented in appendix~\ref{programs},
        Figure~\ref{whiteexample}.

        \begin{figure}[ht]
            \begin{center}
                \begin{minipage}[t]{0.45\textwidth}
                    \epsfxsize=1.0\linewidth
                    \epsffile{\directory/data.tsfraction.tsderivative.tsnormal-s30.eps}
                    \caption[{\market}, histogram of the first
                        derivative of the increments]{{\market},
                        normalized histogram of the first derivative
                        of the normalized increments of the time
                        series data shown in
                        Figure~\ref{\SETLABEL:TF}.}
                    \label{\SETLABEL:TF1}
                \end{minipage}
                \hfill
                \begin{minipage}[t]{0.45\textwidth}
                    \epsfxsize=1.0\linewidth
                    \epsffile{\directory/data.tsfraction.2tsderivative.tsnormal-s30.eps}
                    \caption[{\market}, histogram of the second
                        derivative of the increments]{{\market},
                        normalized histogram of second derivative of
                        the the normalized increments of the time
                        series data shown in
                        Figure~\ref{\SETLABEL:TF}.}
                    \label{\SETLABEL:TF2}
                \end{minipage}
            \end{center}
        \end{figure}

        \subidx{fractal}{range}
        \subidx{fractal}{R/S analysis}
        \subidx{\market}{rate of revenue returns, range}
        \subidx{\market}{deterministic mechanism}
        \subidx{deterministic}{mechanism}
        \subidx{mechanism}{deterministic}
        Figure~\ref{\SETLABEL:TR} is the range of values of the time
        series shown in Figure~\ref{\SETLABEL:TS}. The horizontal axis
        is time into the future. In principle, if the time series was
        characterized as fractional Brownian motion the graph in
        Figure~\ref{\SETLABEL:TR} would be a square root
        function\footnote{Note that the ``roughness,'' or ``sawtooth''
        characteristics of the graph in Figure~\ref{\SETLABEL:TR} are
        a computational artifact---caused by not using the -m option
        to the program {\it tshurst}\/, which is computationally
        inefficient.}. Figure~\ref{\SETLABEL:TD} is the deterministic
        map of the normalized increments of the time series data shown
        in Figure~\ref{\SETLABEL:TF}. The deterministic map is useful
        for determining if a time series was created by a
        deterministic mechanism. This, essentially, maps each element
        in the time series with the previous element in the time
        series.  See,~\cite[pp. 745]{Peitgen}.

        \begin{figure}[ht]
            \begin{center}
                \begin{minipage}[t]{0.45\textwidth}
                    \epsfxsize=1.0\linewidth
                    \epsffile{\directory/data.tshurst-f.eps}
                    \caption[{\market}, range]{{\market}, range of the
                        time series data shown in
                        Figure~\ref{\SETLABEL:TS}.}
                    \label{\SETLABEL:TR}
                \end{minipage}
                \hfill
                \begin{minipage}[t]{0.45\textwidth}
                    \epsfxsize=1.0\linewidth
                    \epsffile{\directory/data.tsfraction.tsdeterministic.eps}
                    \caption[{\market}, deterministic map]{{\market},
                        deterministic map of the normalized increments
                        of the time series data shown in
                        Figure~\ref{\SETLABEL:TF}.}
                    \label{\SETLABEL:TD}
                \end{minipage}
            \end{center}
        \end{figure}

% Local Variables:
% TeX-parse-self: t
% TeX-auto-save: t
% TeX-master: "fractal.tex"
% End:


            Figure~\ref{\SETLABEL:NH} would seem to indicate that the
            time series data for the {\market} represents a cumulative
            sum/integration of a random process that has a Gaussian
            distribution, (ie., satisfies the Gaussian increments
            property of fractional Brownian
            motion~\cite[pp. 250]{Crownover},) tending to justify the
            assumption that the time series data represents fractional
            Brownian motion.

        %
% -----------------------------------------------------------------------------
%
% A license is hereby granted to reproduce this software source code and
% to create executable versions from this source code for personal,
% non-commercial use.  The copyright notice included with the software
% must be maintained in all copies produced.
%
% THIS PROGRAM IS PROVIDED "AS IS". THE AUTHOR PROVIDES NO WARRANTIES
% WHATSOEVER, EXPRESSED OR IMPLIED, INCLUDING WARRANTIES OF
% MERCHANTABILITY, TITLE, OR FITNESS FOR ANY PARTICULAR PURPOSE.  THE
% AUTHOR DOES NOT WARRANT THAT USE OF THIS PROGRAM DOES NOT INFRINGE THE
% INTELLECTUAL PROPERTY RIGHTS OF ANY THIRD PARTY IN ANY COUNTRY.
%
% Copyright (c) 1994-2006, John Conover, All Rights Reserved.
%
% Comments and/or bug reports should be addressed to:
%
%     john@email.johncon.com (John Conover)
%
% -----------------------------------------------------------------------------
%
% Revision: \RCSRevision \\
% Revision Time: \RCSTime UMT \\
% Revision Date: \RCSDate \\
% Revision Id: \RCSId \\
% Revision File: \RCSLog \\
\RCS $Revision: 0.0 $
\RCS $Date: 2006/01/20 04:38:13 $
\RCS $Id: instant.tex,v 0.0 2006/01/20 04:38:13 john Exp $
% $Log: instant.tex,v $
% Revision 0.0  2006/01/20 04:38:13  john
% Initial version
%
%
    \subsection{Instantaneous Analysis of Normalized Increments}
        \label{\SETLABEL:IA}

        \subidx{\market}{instantaneous analysis of normalized increments}
        \idx{average of normalized increments}
        \idx{root mean square of normalized increments}
        \subidx{Shannon probability}{instantaneous computation of}
        \subidx{average of normalized increments}{instantaneous computation of}
        \subidx{root mean square of normalized increments}{instantaneous computation of}
        \subidx{instantaneous computation}{Shannon probability}
        \subidx{instantaneous computation}{average of normalized increments}
        \subidx{instantaneous computation}{root mean square of normalized increments}
        \idx{time series}
        \subidx{time series}{instantaneous analysis}
        \subidx{instantaneous analysis}{time series}
        \subidx{time series}{increments}
        \subidx{time series}{analysis}
        \subidx{Shannon}{probability}
        \subidx{probability}{Shannon}
        \subidx{normalized}{increments}
        \subidx{increments}{normalized}

        The program {\it tsinstant}\/, which is briefly described in
        Appendix~\ref{programs}, is for finding the instantaneous
        fraction of change in a time series. The value of a sample in
        the time series is subtracted from the previous sample in the
        time series, and divided by the value of the previous sample.
        As explained in Chapter~\ref{general},
        Sections~\ref{derivation},~\ref{GA},~\ref{abmfi},~\ref{aftsma}
        and,~\ref{ompl} for Brownian motion, random walk fractals, the
        absolute value of the instantaneous fraction of change is also
        the root mean square of the instantaneous fraction of
        change\footnote{The absolute value of the normalized
        increments, when averaged, is related to the root mean square
        of the increments by a constant. If the normalized increments
        are a fixed increment, the constant is unity. If the
        normalized increments have a Gaussian distribution, the
        constant is $\approx 0.8$ depending on the accuracy of of
        ``fit'' to a Gaussian distribution.}. Squaring this value is
        the average of the instantaneous fraction of change, and
        adding unity to the absolute value of the instantaneous
        fraction of change, and dividing by two, is the Shannon
        probability of the instantaneous fraction of change.

        Figure~\ref{\SETLABEL:IA1} is the instantaneous value of the
        root mean square of the normalized increments for the
        {\market}, and Figure~\ref{\SETLABEL:IA2} is the instantaneous
        Shannon probability for the normalized increments.

        \begin{figure}[ht]
            \begin{center}
                \begin{minipage}[t]{0.45\textwidth}
                    \epsfxsize=1.0\linewidth
                    \epsffile{\directory/data.tsinstant-r.eps}
                    \caption[{\market}, instantaneous value of
                        rms.]{{\market}, instantaneous value of the
                        root mean square of the normalized increments,
                        provided by running the program {\it
                        tsinstant}\/ with the -r option on the data
                        presented in Figure~\ref{\SETLABEL:TS}.}
                    \label{\SETLABEL:IA1}
                    \label{\SETLABELQ:IA1}
                \end{minipage}
                \hfill
                \begin{minipage}[t]{0.45\textwidth}
                    \epsfxsize=1.0\linewidth
                    \epsffile{\directory/data.tsinstant-s.eps}
                    \caption[{\market}, instantaneous value of
                        Shannon probability.]{{\market}, instantaneous
                        value of the Shannon probability of the
                        normalized increments, provided by running the
                        program {\it tsinstant}\/ with the -s option
                        on the data presented in
                        Figure~\ref{\SETLABEL:TS}.}
                    \label{\SETLABEL:IA2}
                    \label{\SETLABELQ:IA2}
                \end{minipage}
            \end{center}
        \end{figure}

% Local Variables:
% TeX-parse-self: t
% TeX-auto-save: t
% TeX-master: "fractal.tex"
% End:


        %
% -----------------------------------------------------------------------------
%
% A license is hereby granted to reproduce this software source code and
% to create executable versions from this source code for personal,
% non-commercial use.  The copyright notice included with the software
% must be maintained in all copies produced.
%
% THIS PROGRAM IS PROVIDED "AS IS". THE AUTHOR PROVIDES NO WARRANTIES
% WHATSOEVER, EXPRESSED OR IMPLIED, INCLUDING WARRANTIES OF
% MERCHANTABILITY, TITLE, OR FITNESS FOR ANY PARTICULAR PURPOSE.  THE
% AUTHOR DOES NOT WARRANT THAT USE OF THIS PROGRAM DOES NOT INFRINGE THE
% INTELLECTUAL PROPERTY RIGHTS OF ANY THIRD PARTY IN ANY COUNTRY.
%
% Copyright (c) 1994-2006, John Conover, All Rights Reserved.
%
% Comments and/or bug reports should be addressed to:
%
%     john@email.johncon.com (John Conover)
%
% -----------------------------------------------------------------------------
%
% Revision: \RCSRevision \\
% Revision Time: \RCSTime UMT \\
% Revision Date: \RCSDate \\
% Revision Id: \RCSId \\
% Revision File: \RCSLog \\
\RCS $Revision: 0.0 $
\RCS $Date: 2006/01/20 04:38:13 $
\RCS $Id: logistic.tex,v 0.0 2006/01/20 04:38:13 john Exp $
% $Log: logistic.tex,v $
% Revision 0.0  2006/01/20 04:38:13  john
% Initial version
%
%
    \subsection{Logistic Analysis}
        \label{\SETLABEL:LA}

        \subidx{\market}{Logistic function analysis}
        \subidx{time series}{logistic function}
        \subidx{logistic function}{time series}
        \subidx{time series}{increments}
        \subidx{time series}{analysis}
        \subidx{cumulative sum}{analysis}
        \subidx{analysis}{cumulative sum}
        \subidx{analysis}{random process}
        \subidx{random process}{analysis}
        The data in this section is presented in tabular form in
        Section~\ref{\SETLABELREF:LAA}.  Figure~\ref{\SETLABEL:LA1} is
        a graph of the logistic function estimates of the time series
        data for the {\market}. The reader is cautioned that these
        graphs are constructed using the method suggested in
        Chapter~\ref{general}, Section~\ref{nlextend} and enormous
        precision is required for adequate prediction of the logistic
        function,~\cite{Modis}. Particularly, the non-linear term will
        usually require intervention to produce a practical fit to the
        data. In addition, there are numerical stability issues with
        logistic function methodologies\footnote{For example, in
        Figures~\ref{\SETLABEL:LA1} and~\ref{\SETLABEL:LA2}, if the
        non-linear term, $b$, was greater than zero, it was set to
        zero to produce the graphs. See Section~\ref{\SETLABELREF:LAA}
        for the actual derived values. In other cases, the magnitude
        of $b$ was too large, resulting in a graph that was decreasing
        as a function of time}.  The methodology should be regarded as
        ``fragile.'' It is included for completeness.

        \idx{least squares approximation}
        Figure~\ref{\SETLABEL:LA1} is a graph of the logistic function
        for the time series data presented in
        Figure~\ref{\SETLABEL:TS}. The data presented was made by
        running the program {\it tsdlogistic}\/, which is described
        briefly in Appendix~\ref{programs}, on the parameters
        extracted from the time series data as suggested in
        Figure~\ref{\SETLABEL:TF}. The program {\it tslsq}\/ was used
        to derive the constant and the slope of the normalized
        increments of the data presented in Figure~\ref{\SETLABEL:TF}.
        Figure~\ref{\SETLABEL:LA2} is the same graph, but with the
        time scale expanded by a factor of two.

        \begin{figure}[ht]
            \begin{center}
                \begin{minipage}[t]{0.45\textwidth}
                    \epsfxsize=1.0\linewidth
                    \epsffile{\directory/data.tsfraction.tslsq-p.tsdlogistic.eps}
                    \caption[{\market}, logistic function
                        estimates.]{{\market}, logistic function
                        estimates, provided by running the {\it
                        tslsq}\/ program on the normalized increments
                        presented in Figure~\ref{\SETLABEL:TF} with
                        the -p option. These parameters were used as
                        arguments to the {\it tsdlogistic}\/ program.}
                    \label{\SETLABEL:LA1}
                    \label{\SETLABELQ:LA1}
                \end{minipage}
                \hfill
                \begin{minipage}[t]{0.45\textwidth}
                    \epsfxsize=1.0\linewidth
                    \epsffile{\directory/data.tsfraction.tslsq-p.tsdlogistic2.eps}
                    \caption[{\market}, logistic function
                        estimates.]{{\market}, logistic function
                        estimates of Figure~\ref{\SETLABEL:LA1} with
                        the time scale expanded by a factor of two.}
                    \label{\SETLABEL:LA2}
                    \label{\SETLABELQ:LA2}
                \end{minipage}
            \end{center}
        \end{figure}

% Local Variables:
% TeX-parse-self: t
% TeX-auto-save: t
% TeX-master: "fractal.tex"
% End:


        %
% -----------------------------------------------------------------------------
%
% A license is hereby granted to reproduce this software source code and
% to create executable versions from this source code for personal,
% non-commercial use.  The copyright notice included with the software
% must be maintained in all copies produced.
%
% THIS PROGRAM IS PROVIDED "AS IS". THE AUTHOR PROVIDES NO WARRANTIES
% WHATSOEVER, EXPRESSED OR IMPLIED, INCLUDING WARRANTIES OF
% MERCHANTABILITY, TITLE, OR FITNESS FOR ANY PARTICULAR PURPOSE.  THE
% AUTHOR DOES NOT WARRANT THAT USE OF THIS PROGRAM DOES NOT INFRINGE THE
% INTELLECTUAL PROPERTY RIGHTS OF ANY THIRD PARTY IN ANY COUNTRY.
%
% Copyright (c) 1994-2006, John Conover, All Rights Reserved.
%
% Comments and/or bug reports should be addressed to:
%
%     john@email.johncon.com (John Conover)
%
% -----------------------------------------------------------------------------
%
% Revision: \RCSRevision \\
% Revision Time: \RCSTime UMT \\
% Revision Date: \RCSDate \\
% Revision Id: \RCSId \\
% Revision File: \RCSLog \\
\RCS $Revision: 0.0 $
\RCS $Date: 2006/01/20 04:38:13 $
\RCS $Id: hurst.tex,v 0.0 2006/01/20 04:38:13 john Exp $
% $Log: hurst.tex,v $
% Revision 0.0  2006/01/20 04:38:13  john
% Initial version
%
%
    \subsection{Hurst Coefficient Analysis}
        \label{\SETLABEL:H}

        \subidx{\market}{Hurst coefficient analysis}
        \subidx{Hurst coefficient}{analysis}
        \subidx{increments}{normalized}
        \subidx{normalized}{increments}
        \subidx{programs}{tshurst}
        \subidx{tshurst}{program}
        The data in this section is presented in tabular form in
        Section~\ref{\SETLABELREF:HCHP}. Figure~\ref{\SETLABEL:HC} is
        a graph of the Hurst coefficient data time series data shown
        in Figure~\ref{\SETLABEL:TS}. The slope of the graph is the
        Hurst coefficient.  The data for this figure was produced by
        the program {\it tshurst}\/, which is described briefly in
        Appendix~\ref{programs}.

        \subidx{\market}{H parameter analysis}
        \subidx{H parameter}{analysis}
        \subidx{programs}{tshcalc}
        \subidx{tshcalc}{program}
        Figure~\ref{\SETLABEL:HP} is a graph of the H parameter data
        for the normalized increments of the time series data shown in
        Figure~\ref{\SETLABEL:TF}. The data for this figure was
        produced by the program {\it tshcalc}\/, which is described
        briefly in Appendix~\ref{programs}.

        \begin{figure}[ht]
            \begin{center}
                \begin{minipage}[t]{0.45\textwidth}
                    \epsfxsize=1.0\linewidth
                    \epsffile{\directory/data.tshurst.eps}
                    \caption[{\market}, Hurst coefficient data]{{\market},
                        Hurst coefficient data for the normalized
                        increments of the time series data shown in
                        Figure~\ref{\SETLABEL:TF}.  The slope of the graph
                        is the Hurst coefficient.}
                    \label{\SETLABEL:HC}
                \end{minipage}
                \hfill
                \begin{minipage}[t]{0.45\textwidth}
                    \epsfxsize=1.0\linewidth
                    \epsffile{\directory/data.tshcalc.eps}
                    \caption[{\market}, H parameter data]{{\market}, H
                        parameter data for the normalized increments of
                        the time series data shown in
                        Figure~\ref{\SETLABEL:TF} The slope of the graph
                        is the H parameter.}
                    \label{\SETLABEL:HP}
                \end{minipage}
            \end{center}
        \end{figure}

        \subidx{revenue}{See, rate of revenue returns}
        \subidx{returns}{See, rate of revenue returns}
        \subidx{\market}{revenues}
        \subidx{Hurst coefficient}{analysis}
        \subidx{\market}{Hurst coefficient analysis}
        \subidx{\market}{rate of change}
        \subidx{\market}{windows of opportunity}
        \subidx{rate of revenue returns}{forecast}
        \subidx{forecast}{rate of revenue returns}
        \idx{windows of opportunity}
        \subidx{programs}{tslsq}
        \subidx{tslsq}{program}

        The approximately linear slope of the graph in
        Figure~\ref{\SETLABEL:HC} implies that the variance of the
        rate of revenue returns, (per {\timescale},) in the {\market},
        $V(t_2 - t_1)$, over a period of time is proportional to the
        period of time raised to twice the Hurst
        coefficient~\cite[pp. 180]{Feder},~\cite[pp. 246]{Crownover}.
        This seems to be a quantitative statement concerning how fast,
        and to what degree, the rate of revenue returns' state of
        affairs can change over a period of time.  An additional
        implication, for Hurst coefficients sufficiently close to 0.5,
        is that the probability of the state of affairs repeating
        sometime in the future goes down with increasing
        time\footnote{It can be shown that the number of expected
        market ``high'' and ``low'' transitions, $N$, scales with the
        square root of time, or $N \propto \sqrt {t}$, meaning that
        the cumulative distribution of the probability, $P$, of the
        duration of a market's ``high'' or ``low'' exceeding a given
        time interval, $t$, is proportional to the reciprocal of the
        square root of the time interval, $P \propto 1 / \sqrt {t}$,
        (or, conversely, that the probability of the duration of a
        market's ``high'' or ``low'' exceeding a given time interval
        is proportional to the reciprocal of the time interval raised
        to the power $3 / 2$, ie., $P \propto 1 / t^{3 /
        2}$,~\cite[pp. 153]{Schroeder}. What this means is that a
        histogram of the ``zero free'' run-lengths of a market being
        ``high'' or ``low,'' over a long time, would have a $1 / t^{3
        / 2}$ characteristic.)}, $t$, $p(t) = erf (1/\sqrt{2t})$ which
        is approximately $1/\sqrt{t}$ for $t \gg
        1$~\cite[pp. 160]{Schroeder}. Figures~\ref{\SETLABEL:FN},
        and,~\ref{\SETLABEL:FF} compare methods of approximation of
        the ``forecastability'' of the rate of revenue returns in the
        {\market} for the near term and far term,
        respectively~\cite[pp. 83-84]{Peters:CAOITCM}\footnote{The
        author is not comfortable with Peters' interpretation. For
        example, if the algorithm explained
        in~\cite[pp. 82]{Peters:CAOITCM} is used on ``white noise''
        which, by definition, never has any correlations, the short
        term Hurst coefficient, and thus the ``forecastability,'' is
        still near unity---a bit of an enigma. This can be verified
        with the {\it tswhite}\/ and {\it tshurst}\/ programs, which
        are briefly described in Appendix~\ref{programs}.}.  This
        seems to be a quantitative statement concerning ``windows of
        opportunity'' in the rate of revenue returns, (per
        {\timescale}.)  The program {\it tslsq}\/ was used on the
        Hurst coefficient data, presented in
        Figure~\ref{\SETLABEL:HC}, to provide a least squares
        approximation to the Hurst coefficient. The superimposed least
        squares approximation with on original Hurst coefficient data
        is presented.  The time series data has a Hurst coefficient of
        {\thurstlow}, so that:

        \subidx{\market}{Hurst coefficient analysis}
        \begin{eqnarray}
            V\left(t_2 - t_1\right) & \propto & \left(t_2 - t_1\right)^{2 \cdot H}\\
            V\left(t_2 - t_1\right) & \propto & \left(t_2 - t_1\right)^{2 \cdot {\thurstlow}}\\
                                    & \propto & \left(t_2 - t_1\right)^{\thurstlowtwo}
            \label{\SETLABEL:V}
        \end{eqnarray}

        \subidx{fractional}{Brownian motion}
        \subidx{Brownian motion}{fractional}
        \idx{fractal}
        \noindent where $V(t_2 - t_1)$ is the variance of the
        increments of the rate of revenue returns, (per {\timescale},)
        over the time interval $t_2 -
        t_1$,~\cite[pp. 177]{Feder},~\cite[pp. 494]{Peitgen}. If $H >
        \frac{1}{2}$, then the time series is termed as being
        characterized by ``fractional Brownian
        motion~\cite[pp. 170]{Feder}.''

        \subidx{rate of revenue returns}{predictability}
        \subidx{rate of revenue returns}{forecastability}
        \subidx{rate of revenue returns}{consistency}
        \subidx{predictability}{rate of revenue returns}
        \subidx{forecastability}{rate of revenue returns}
        \subidx{consistency}{rate of revenue returns}
        \subidx{\market}{rate of revenue returns, predictability}
        \subidx{\market}{rate of revenue returns, forecastability}
        \subidx{\market}{rate of revenue returns, consistency}
        \subidx{Hurst coefficient}{analysis}
        \subidx{\market}{Hurst coefficient analysis}
        \subidx{\market}{rate of change}

        In some sense, the Hurst coefficient is a quantitative
        expression of the ``forecastability'' of the future based on
        the past\footnote{Actually, in general, when summing fractal
        entities, the method used should be a root mean square
        process, dependent on the Hurst Coefficient, $H$, where
        $P_{total}^H = P_1^H + P_2^H + \cdots$, where $P_n$ is the
        fractal entities. For a Brownian motion, or random walk type
        of fractal the Hurst Coefficient is a function of time into
        the future. For the ``near term,'' the Hurst coefficient is
        very near unity, meaning the summation process is linear. For
        the ``long term,'' $H \approx 0.5$, or a standard root mean
        square summation process should be used. If $H$ is $0.5$ then
        the market is termed a Brownian motion, or random walk
        process. If it is larger than 0.5, it is termed fractional
        Brownian motion process. For a random walk process, ``near
        term'' and ``far term'' are quantitatively differentiated on
        the Hurst Coefficient graph where $1 - \ln (t) = 0.5 \cdot \ln
        (t)$, or when $\ln (t) = 2$, or $t = 7.389\ldots$ See
        Section~\ref{\SETLABEL:FS} for the particulars on using Hurst
        Coefficient to sum fractal process' for the {\market}. See
        also~\cite[pp. 67, 83-84]{Peters:CAOITCM} and~\cite[pp. 129,
        159]{Schroeder} for particulars on the implications of the
        Hurst Coefficient and root mean square summation issues.}.  A
        Hurst coefficient of {\thurstlow}, (for the near future, and
        {\thurstall} for the distant future.) implies that the
        likelihood of the rate of revenue returns, (per {\timescale},)
        for any two consecutive {\timescale}s being the same is
        {\thurstlowhundred}\%~\cite[pp. 66]{Peters:CAOITCM} for the
        near future, and {\thurstall} for the distant
        future. Likewise, there is a {\thurstlowhundred}\% chance of
        the rate of revenue returns, (per {\timescale},) movements
        being the same in consecutive time periods---ie., if, in a
        given {\timescale}, the rate of revenue returns, (per
        {\timescale},) is increasing, there is a {\thurstlowhundred}\%
        that the rate of revenue returns, (per {\timescale},) will
        increase in the following period, also. In some sense, this is
        a quantitative statement on how ``predictable,'' or
        ``forecastable'' the rate of revenue returns, (per
        {\timescale},) for the {\market} are over time, since the
        probability of having $n$ many consecutive {\timescale}s of
        the same agenda is $H^n$ where $H$ is the Hurst coefficient,
        or, letting the short term probability of having $n$ many
        {\timescale}s of the same market agenda, $p_a$, is:

        \begin{eqnarray}
            p_a\left(n\right) & = & H^{n}\\
                              & = & {\thurstlow}^{n}
            \label{\SETLABEL:MA}
        \end{eqnarray}

        \subidx{rate of revenue returns}{predictability}
        \subidx{rate of revenue returns}{forecastability}
        \subidx{rate of revenue returns}{consistency}
        \subidx{predictability}{rate of revenue returns}
        \subidx{forecastability}{rate of revenue returns}
        \subidx{consistency}{rate of revenue returns}
        As an interesting interpretation of the normalized increments
        of the time series data presented in
        Figure~\ref{\SETLABEL:TF}, if the vertical axis is multiplied
        by 100, to convert to percent, then the graph represents the
        error, in percent, that would be made by forecasting, month by
        month, that the next {\timescale}'s rate of revenue returns
        would be the same as the current {\timescale}'s revenue
        rate. Interestingly, it is $\datafractionmean \cdot 100$
        percent, on the average, with a standard deviation of
        $\datafractionstddev \cdot 100$ percent, and a root mean
        square error value of $\datafractionrms \cdot 100$
        percent---small values for such a simple forecasting
        mechanism.

        \subidx{\market}{rate of revenue returns, range}
        \subidx{Hurst coefficient}{analysis}
        \subidx{\market}{Hurst coefficient analysis}
        \subidx{\market}{rate of change}

        This is, essentially, a statement of the range of values, in
        the increments of the rate of revenue returns, (per
        {\timescale},) that is to be expected over the time interval,
        $t_2 - t_1$,
        $R_v$,~\cite[pp. 178]{Feder},~\cite[pp. 172]{Cambel}:

        \begin{eqnarray}
            R_v\left(t_2 - t_1\right) & \propto & \left(t_2 - t_1\right)^{H}\\
                                      & \propto & \left(t_2 - t_1\right)^{\thurstlow}
            \label{\SETLABEL:R}
        \end{eqnarray}

        \subidx{\market}{rate of revenue returns, range}
        \subidx{Hurst coefficient}{analysis}
        \subidx{\market}{Hurst coefficient analysis}
        \subidx{\market}{rate of change}
        \subidx{Markov}{statistics}
        \subidx{statistics}{Markov}
        \noindent where $R$ is the range of values in the increments
        of the rate of revenue returns, (per {\timescale}.) A Hurst
        coefficient, $H$, that is much larger than $\frac{1}{2}$, (but
        less than 1,) implies a strongly non-Gaussian distribution in
        the increments of the rate of revenue returns, (per
        {\timescale},)~\cite[pp. 152, 194]{Feder}, and a Hurst
        coefficient near $\frac{1}{2}$ implies that the increments of
        the rate of revenue returns, (per {\timescale}) is
        characteristic of an independent
        process~\cite[pp. 195]{Feder}. Extreme caution should be
        exercised in using Markov statistics in any analysis where the
        Hurst coefficient is not
        $\frac{1}{2}$,~\cite[pp. 124]{Crownover},~\cite[pp. 106]{Peters:CAOITCM}.


        As a useful approximation, if $H$, is approximately
        $\frac{1}{2}$, Equation~\ref{\SETLABEL:R} reduces
        to,~\cite[pp. 129]{Schroeder}:

        \begin{eqnarray}
            R\left(t_2 - t_1\right) & \propto & (t_2 - t_1)^{\frac{1}{2}}\\
                                    & \propto & \sqrt{\left(t_2 - t_1\right)}
        \end{eqnarray}

        \subidx{\market}{rate of revenue returns, range}
        \subidx{\market}{rate of revenue returns, increase and decrease}
        \subidx{Hurst coefficient}{analysis}
        \subidx{\market}{Hurst coefficient analysis}
        \subidx{\market}{rate of change}
        \subidx{Markov}{statistics}
        \subidx{statistics}{Markov}

        In the case where the Hurst coefficient, $H$, is
        $\frac{1}{2}$, the range of values in the increments of the
        rate of revenue returns, (per {\timescale},) divided by the
        standard deviation of these values, $S$, can be anticipated to
        increase over time according to the following
        relation,~\cite[pp. 154]{Feder},~\cite[pp. 129]{Schroeder}:

        \begin{equation}
            \frac{R\left(t_2 - t_1\right)}{S} \propto \left(t_2 - t_1\right)^{\frac{1}{2}}
        \end{equation}

        \subidx{\market}{rate of revenue returns, range}
        \subidx{\market}{rate of revenue returns, increase and decrease}
        \subidx{Hurst coefficient}{analysis}
        \subidx{\market}{Hurst coefficient analysis}
        \subidx{\market}{rate of change}
        \noindent which is a useful conceptual approximation, since it
        involves only the square root function---if the range and the
        standard deviation of the increments of the rate of revenue
        returns, (per {\timescale},) are known, (and $H \approx
        \frac{1}{2}$,) then the expected change in $\frac{R}{S}$, will
        increase with the square root of time\footnote{To be precise,
        it is actually asymptotically proportional to
        $\tau^{\frac{1}{2}}$}.

        Another useful approximation when rescaling processes that are
        characterize by Brownian motion, (ie., when $H \approx
        \frac{1}{2}$,) is that:

        \begin{eqnarray}
            X\left(t\right) & \propto & \frac{X\left(rt\right)}{r^{H}}\\
                            & \propto & \frac{X\left(rt\right)}{r^{\thurstlow}}
        \end{eqnarray}

        \idx{Brownian motion}
        \idx{fractal}
        Where $X(t)$ is the process characterized by Brownian motion,
        and $r$ is a scaling factor,~\cite[pp. 494]{Peitgen}.

        \subidx{programs}{tslsq}
        \subidx{tslsq}{program}
        The program {\it tslsq}\/ was used on the H parameter data,
        presented in Figure~\ref{\SETLABEL:HP}, to provide a least
        squares approximation to the H parameter for the
        {\market}. The superimposed least squares approximation on the
        original H parameter data is presented.  By contrast, the H
        parameter, as derived by the methodology outlined
        in~\cite[pp. 249]{Crownover}, is {\thcalclow} for the near
        future, and {\thcalcall} for the distant future.

        \subidx{\market}{Hurst coefficient analysis}
        \subidx{Hurst coefficient}{analysis}
        \subidx{increments}{normalized}
        \subidx{normalized}{increments}
        \subidx{programs}{tshurst}
        \subidx{tshurst}{program}
        \subidx{\market}{H parameter analysis}
        \subidx{H parameter}{analysis}
        \subidx{programs}{tshcalc}
        \subidx{tshcalc}{program}
        Figures~\ref{\SETLABEL:HC} and~\ref{\SETLABEL:HP} represent
        Hurst coefficient and H parameter data that are derived from
        the normalized increments, shown in
        Figure~\ref{\SETLABEL:TF}. In this case, the data is
        considered a normalized derivative of the time series data
        presented in Figure~\ref{\SETLABEL:TF}, instead of a
        cumulative sum.  The program, {\it tshurst}\/, is described
        briefly in appendix~\ref{programs}, and the data for
        figures~\ref{\SETLABEL:THC} and~\ref{\SETLABEL:THP} was made
        using the -d option.

        \begin{figure}[ht]
            \begin{center}
                \begin{minipage}[t]{0.45\textwidth}
                    \epsfxsize=1.0\linewidth
                    \epsffile{\directory/data.tsfraction.tshurst-d.eps}
                    \caption[{\market}, traditional Hurst coefficient
                        data]{{\market}, traditional Hurst coefficient
                        data for the time series data shown in
                        Figure~\ref{\SETLABEL:TS}.  The slope of the
                        graph is the Hurst coefficient, and is
                        {\hurstlow} for the near term, and
                        {\hurstall} for the far term.}
                    \label{\SETLABEL:THC}
                \end{minipage}
                \hfill
                \begin{minipage}[t]{0.45\textwidth}
                    \epsfxsize=1.0\linewidth
                    \epsffile{\directory/data.tsfraction.tshcalc-d.eps}
                    \caption[{\market}, traditional H parameter
                        data]{{\market}, traditional H parameter data
                        for the time series data shown in
                        Figure~\ref{\SETLABEL:TS} The slope of the
                        graph is the H parameter, and is {\hcalclow}
                        for the near term, and {\hcalcall} for the
                        far term.}
                    \label{\SETLABEL:THP}
                \end{minipage}
            \end{center}
        \end{figure}

% Local Variables:
% TeX-parse-self: t
% TeX-auto-save: t
% TeX-master: "fractal.tex"
% End:


        %
% -----------------------------------------------------------------------------
%
% A license is hereby granted to reproduce this software source code and
% to create executable versions from this source code for personal,
% non-commercial use.  The copyright notice included with the software
% must be maintained in all copies produced.
%
% THIS PROGRAM IS PROVIDED "AS IS". THE AUTHOR PROVIDES NO WARRANTIES
% WHATSOEVER, EXPRESSED OR IMPLIED, INCLUDING WARRANTIES OF
% MERCHANTABILITY, TITLE, OR FITNESS FOR ANY PARTICULAR PURPOSE.  THE
% AUTHOR DOES NOT WARRANT THAT USE OF THIS PROGRAM DOES NOT INFRINGE THE
% INTELLECTUAL PROPERTY RIGHTS OF ANY THIRD PARTY IN ANY COUNTRY.
%
% Copyright (c) 1994-2006, John Conover, All Rights Reserved.
%
% Comments and/or bug reports should be addressed to:
%
%     john@email.johncon.com (John Conover)
%
% -----------------------------------------------------------------------------
%
% Revision: \RCSRevision \\
% Revision Time: \RCSTime UMT \\
% Revision Date: \RCSDate \\
% Revision Id: \RCSId \\
% Revision File: \RCSLog \\
\RCS $Revision: 0.0 $
\RCS $Date: 2006/01/20 04:38:13 $
\RCS $Id: fiscal.tex,v 0.0 2006/01/20 04:38:13 john Exp $
% $Log: fiscal.tex,v $
% Revision 0.0  2006/01/20 04:38:13  john
% Initial version
%
%
    \subsection{Fixed Increment Approximation for Fiscal Strategy}
        \label{\SETLABEL:FS}

        \subidx{\market}{fiscal strategy}
        \subidx{markets}{analysis}
        \subidx{analysis}{markets}
        \subidx{strategy}{fiscal}
        \subidx{fiscal}{strategy}
        The data in this section is presented in tabular form in
        Section~\ref{\SETLABELREF:LR}. This section derives various
        values based on the ``average'' of the normalized increments
        presented in Figure~\ref{\SETLABEL:TFA}. These values are an
        approximation to a, probably, complex process with a
        distribution shown in Figure~\ref{\SETLABEL:TF}. These values
        will be used in a fixed increment Brownian fractal analysis
        and simulation of the {\market}, and may, or may not, provide
        adequate accuracy for projections.

        For an organization operating in the {\market}, the fiscal
        strategy, commensurate with the aggregate environment, can be
        derived as follows~\cite[pp. 128, pp
        151]{Schroeder},~\cite[pp. 450]{Reza},~\cite[pp. 270]{Pierce}:
        \vspace{0.15in}

        \subsubsection{Logarithmic Returns}
            \label{\SETLABEL:LR}

            \subidx{logarithmic}{returns}
            \subidx{returns}{logarithmic}
            \subidx{\market}{logarithmic returns}
            The logarithmic returns can be calculated by various
            means. Four will be presented here, for comparison.

            \subidx{programs}{tsnormal}
            \subidx{tsnormal}{program}
            \subidx{logarithmic}{returns}
            \subidx{returns}{logarithmic}
            The logarithmic returns, in bits, $bits$, as computed from
            the mean, by the program {\it tsnormal}\/, which is
            described in Chapter~\ref{programs}, and is presented in
            Figure~\ref{\SETLABEL:TF}, and Equation~\ref{abits} from
            Section~\ref{ereturns} in Chapter~\ref{general}:

            \begin{equation}
                bits = \frac{\ln \left({\datafractionmean} + 1\right)}{\ln \left(2\right)} = \datafractionmeanbits
            \end{equation}

            \subidx{programs}{tslsq}
            \subidx{tslsq}{program}
            \subidx{logarithmic}{returns}
            \subidx{returns}{logarithmic}
            \noindent By comparison, the logarithmic returns, in bits,
            $bits$, as computed from the constant in the least squares
            approximation, using the program {\it tslsq}\/, which is briefly
            described in Chapter~\ref{programs}, as presented in
            Figure~\ref{\SETLABEL:TF}, and Equation~\ref{abits} from
            Section~\ref{ereturns} in Chapter~\ref{general}:

            \begin{equation}
                bits = \frac{\ln \left({\datafractionconstant} + 1\right)}{\ln \left(2\right)} = \datafractionconstantbits
            \end{equation}

            Note that if the mean is not constant in
            Figure~\ref{\SETLABEL:TF}, this method will not provide
            accurate results.

            \subidx{programs}{tslsq}
            \subidx{tslsq}{program}
            \subidx{logarithmic}{returns}
            \subidx{returns}{logarithmic}
            \noindent And by yet another comparison, using the program
            {\it tslsq}\/, which is briefly described in
            Chapter~\ref{programs}, with the -e -p options, to provide
            a formula for the least squares exponential fit to the
            time series data set presented in
            Figure~\ref{\SETLABEL:TS}:

            \begin{equation}
                bits = {\datatslsqepbits}
            \end{equation}

            \subidx{programs}{tslogreturns}
            \subidx{tslogreturns}{program}
            \subidx{logarithmic}{returns}
            \subidx{returns}{logarithmic}
            \noindent And finally, by comparison, from the
            {\it tslogreturns}\/ program, which is briefly described
            in Chapter~\ref{programs}, with the -p option, to provide
            a formula for the logarithmic returns of the time series
            data set presented in Figure~\ref{\SETLABEL:TS}:

            \begin{equation}
                bits = {\logreturns}
            \end{equation}

        \subsubsection{Calculation of Shannon Probability}
            \label{\SETLABEL:SP}

            \subidx{\market}{Shannon probability}
            Ideally, all of the values presented in
            Section~\ref{\SETLABEL:LR} would be equal. Using the
            logarithmic returns provided by the {\it tslogreturns}\/
            program, to be consistent
            with~\cite[pp. 81]{Peters:CAOITCM}

            \subidx{programs}{tslogreturns}
            \subidx{tslogreturns}{program}
            \begin{equation}
                2^{{\logreturns}t}
            \end{equation}

            \noindent therefore:
            \begin{equation}
                C\left(p\right) = {\logreturns}
            \end{equation}
            \subidx{programs}{tsshannon}
            \subidx{tsshannon}{program}
            \subidx{Shannon}{probability}
            \subidx{probability}{Shannon}
            \noindent and, {\it tsshannon}\/ {\logreturns} gives:
            \begin{equation}
                \label{\SETLABEL:F0}
                C\left({\shannonlogreturns}\right) = {\logreturns}
            \end{equation}
            \noindent therefore:
            \begin{eqnarray}
                2^{C\left({\shannonlogreturns}\right)} & = & 2^{\logreturns}\\
                                                       & = & {\twologreturns}\\
                                                       & = & {\twologreturnshundred}\%
            \end{eqnarray}
            \noindent and:
            \begin{eqnarray}
                2p - 1 & = & \left(2 \cdot {\shannonlogreturns}\right) - 1\\
                       & = & {\twopone}\\
                       \label{\SETLABEL:F1}
                       & = & {\twoponehundred}\%
            \end{eqnarray}

            \subidx{\market}{fiscal strategy}
            \subidx{markets}{analysis}
            \subidx{analysis}{markets}
            \subidx{strategy}{fiscal}
            \subidx{fiscal}{strategy}
            \subidx{\market}{fiscal strategy}
            \subidx{\market}{growth rate}
            Presuming the simplified assumptions outlined in
            Section~\ref{assumptions}, the ``typical'' organization
            operating in the {\market} executes a long term fiscal
            strategy, commensurate with the aggregate environment,
            that is to invest, every {\timescale}, in sufficient
            additional resources and infrastructure, to increase the
            manufacturing of goods and services by {\twoponehundred}\%
            of its rate of revenue returns, (per {\timescale}.) As a
            conceptual model, the remaining {\hundredtwoponehundred}\%
            will be held in ``reserve'' with a
            {\shannonlogreturnshundred}\% chance of making twice the
            {\twoponehundred}\% back, (and a
            {\hundredshannonlogreturnshundred}\% chance of making
            0.0,) in one {\timescale}, on the average, for an average
            growth in its rate of revenue returns, (per {\timescale},)
            of {\twologreturnshundred}\%, or a doubling of its rate of
            revenue returns, (per {\timescale},) in
            {\oneoverlogreturns} {\timescale}s.

        \subsubsection{Example Fixed Increment Approximation Fiscal Strategies}

            \subidx{\market}{fiscal strategy}
            \subidx{markets}{analysis}
            \subidx{analysis}{markets}
            \subidx{strategy}{fiscal}
            \subidx{fiscal}{strategy}
            \subidx{\market}{fiscal strategy}
            \subidx{\market}{growth rate}
            \subidx{\market}{management metric}
            \idx{management metric}
            A possible metric on the effectiveness of long term fiscal
            management could possibly be that if an investment of
            {\twoponehundred}\% per {\timescale} of the rate of
            revenue returns, (per {\timescale},) is made in resources
            and infrastructure, then the rate of revenue returns would
            be expected to increase by {\twologreturnshundred}\%, per
            {\timescale}, on average.

            Note that the metrics presented in this section are
            representative of the {\market} as an aggregate whole, and
            may or may not be accurate representations for any
            particular participant in the environment. Of interest to
            the participants in the environment would be a similar
            analysis of each product or service rendered in the
            marketplace.

            \subidx{\market}{fiscal strategy}
            \subidx{markets}{analysis}
            \subidx{analysis}{markets}
            \subidx{strategy}{fiscal}
            \subidx{fiscal}{strategy}
            \subidx{\market}{fiscal strategy}
            As a simple illustrative example, a company operating in
            this environment might obtain a credit line from a bank
            that is equal to {\twoponehundred}\% of its rate of
            revenue returns, (per {\timescale},) to finance additional
            operations. In this simple scenario, the company would use
            its revenue base as collateral for the loan. Some
            {\timescale}s, depending on the {\market}'s environment,
            the company's rate of revenue returns exceeds what was
            borrowed from the bank, and the loan is repaid in
            full. Other {\timescale}s, the company must default, and
            the bank seizes a portion of the company's revenue base to
            pay the delinquent loan. However, on the average, the
            company will expand its rate of revenue returns at
            {\twologreturnshundred}\% per {\timescale}.

            \subidx{\market}{fiscal strategy}
            \subidx{markets}{analysis}
            \subidx{analysis}{markets}
            \subidx{strategy}{fiscal}
            \subidx{fiscal}{strategy}
            \subidx{\market}{fiscal strategy}
            As another simple example, a company re-invests
            {\twoponehundred}\% of its rate of revenue returns, (per
            {\timescale},) in development, marketing, sales, and
            distribution of new products.  Although some products will
            be successful and the return on the investment will exceed
            the {\twoponehundred}\% per {\timescale} investment,
            others will not. However, on the average, the company will
            expand it gross rate of revenue returns at
            {\twologreturnshundred}\% per {\timescale}.

            \subidx{\market}{fiscal strategy}
            \subidx{markets}{analysis}
            \subidx{analysis}{markets}
            \subidx{strategy}{fiscal}
            \subidx{fiscal}{strategy}
            \subidx{\market}{fiscal strategy}
            \subidx{\market}{product portfolio}
            \subidx{\market}{product diversity}
            \subidx{\market}{product mix}
            \subidx{\market}{optimum number of products}
            \idx{product portfolio}
            \idx{product diversity}
            \idx{optimum number of products}
            \idx{product mix}

            As an example of ``product portfolio'' management, suppose
            a company re-invests {\twoponehundred}\% of its rate of
            revenue returns, (per {\timescale},) in development,
            marketing, sales, and distribution of new products.
            Further suppose that the company has two products, and a
            fractal analysis of the individual product rate of revenue
            return time series indicates that one product has a
            Shannon probability of 0.65, and the other has a Shannon
            probability of 0.55. Then the percentage of re-investment
            in the first product would be $(2 \cdot 0.65 - 1) \cdot
            {\twoponehundred}$, percent of the rate of revenue
            returns, and $(2 \cdot 0.55 - 1) \cdot {\twoponehundred}$
            percent for the second product, implying that the company
            should diversify its product line\footnote{The astute
            reader would note that the linear addition was used to add
            the contribution to development of each product. This is a
            ``near term'' interpretation. Actually, in general, the
            method used should be a root mean square process,
            dependent on the Hurst Coefficient, $H$, where
            $P_{total}^H = P_1^H + P_2^H + \cdots$, where $P_n$ is the
            contribution to each individual product. For a Brownian
            motion, or random walk type of fractal the Hurst
            Coefficient is a function of time into the future. For the
            ``near term,'' the Hurst coefficient is very near unity,
            meaning the summation process is linear. For the ``long
            term,'' $H \approx 0.5$, or a standard root mean square
            summation process should be used. If $H$ is $0.5$ then the
            market is termed a Brownian motion, or random walk
            process. If it is larger than 0.5, it is termed fractional
            Brownian motion process. For a random walk process, ``near
            term'' and ``far term'' are quantitatively differentiated
            on the Hurst Coefficient graph where $1 - \ln (t) = 0.5
            \cdot \ln (t)$, or when $\ln (t) = 2$, or $t =
            7.389\ldots$ See~\cite[pp. 67, 83-84]{Peters:CAOITCM}
            and~\cite[pp. 129, 159]{Schroeder} for particulars on the
            implications of the Hurst Coefficient and root mean square
            summation issues.}.  Note that this is a ``bet hedging''
            metric methodology, and assumes that the products have
            uncorrelated revenue return rates. If this re-investment
            methodology is not feasible, perhaps for strategic
            financial reasons, then the re-investment in both products
            should total the ${\twoponehundred}$\%, and the investment
            in each product should be made at a ratio of $\frac{(2
            \cdot 0.65 - 1)}{(2 \cdot 0.55 - 1)} = 3 : 1$,
            respectively. Note that this ``bet hedging'' can be used
            to define the optimal number of products that can be
            supported on the rate of revenue returns. If it assumed
            that all products are ``typical'' for the {\market}, as a
            standard bench mark, then the optimal number will be
            $\frac{1}{{\twopone}}$. Note that this is a
            ``theoretical'' value, since not all products are
            ``typical,'' and there may be strategic reasons, for
            example product leveraging, that may increase the number
            of products above the optimum. However, most of the
            revenue should come from the optimal number of products,
            since having more products will decrease the amount of the
            potential investment in each product, and having less than
            the optimum number of products will increase the risk that
            many of the products could suffer a ``down market''
            concurrently, impacting the rate of revenue returns.  As
            another interesting interpretation of the optimal
            ``hedging of bets,'' in product portfolio strategy, and
            considering the graph of the normalized increments
            presented in Figure~\ref{\SETLABEL:TF}, if the
            organization is running optimally, then these products
            will generate, at least in principle, one standard
            deviation, approximately $0.8413 = 84.13$\% of the future
            growth in rate of revenue returns. Naturally, these are
            approximations, and the values are an approximation to a,
            probably, complex process, and appropriate scrutiny should
            be exercised before making specific projections.  As yet
            another example of ``product portfolio'' management,
            consider the issue of product mix. In this interpretation,
            {\twoponehundred}\% of the product manufactured should be
            ``proprietary,'' while the rest is ``industry standard.''
            As yet another possibility, {\twoponehundred}\% of the
            product manufactured should be predatory into new markets,
            and the remainder in markets that are ``traditional'' for
            the company.

% Local Variables:
% TeX-parse-self: t
% TeX-auto-save: t
% TeX-master: "fractal.tex"
% End:


        %
% -----------------------------------------------------------------------------
%
% A license is hereby granted to reproduce this software source code and
% to create executable versions from this source code for personal,
% non-commercial use.  The copyright notice included with the software
% must be maintained in all copies produced.
%
% THIS PROGRAM IS PROVIDED "AS IS". THE AUTHOR PROVIDES NO WARRANTIES
% WHATSOEVER, EXPRESSED OR IMPLIED, INCLUDING WARRANTIES OF
% MERCHANTABILITY, TITLE, OR FITNESS FOR ANY PARTICULAR PURPOSE.  THE
% AUTHOR DOES NOT WARRANT THAT USE OF THIS PROGRAM DOES NOT INFRINGE THE
% INTELLECTUAL PROPERTY RIGHTS OF ANY THIRD PARTY IN ANY COUNTRY.
%
% Copyright (c) 1994-2006, John Conover, All Rights Reserved.
%
% Comments and/or bug reports should be addressed to:
%
%     john@email.johncon.com (John Conover)
%
% -----------------------------------------------------------------------------
%
% Revision: \RCSRevision \\
% Revision Time: \RCSTime UMT \\
% Revision Date: \RCSDate \\
% Revision Id: \RCSId \\
% Revision File: \RCSLog \\
\RCS $Revision: 0.0 $
\RCS $Date: 2006/01/20 04:38:13 $
\RCS $Id: companies.tex,v 0.0 2006/01/20 04:38:13 john Exp $
% $Log: companies.tex,v $
% Revision 0.0  2006/01/20 04:38:13  john
% Initial version
%
%
    \subsection{Number of Companies}
        \label{\SETLABEL:QNC}

        \subidx{\market}{number of companies}
        \subidx{number of companies}{analysis}
        \subidx{analysis}{number of companies}
        \subidx{Shannon}{probability}
        \subidx{probability}{Shannon}
        This section evaluates the approximate, or ``average,'' number
        of companies in the {\market}, and uses the method outlined in
        Chapter~\ref{general}, Section~\ref{aftsma}. Since the
        average, $avg_{ind}$, and the root mean square, $rms_{ind}$,
        of the normalized increments of the {\market} time series is
        \datafractionmean, and \datafractionrms respectively, the
        number of companies participating in the market can be
        calculated by Equation~\ref{ncompanies} to be {\ncompanies}.

        If this value seems consistent number of companies in the
        {\market}, within the assumptions outlined in
        Chapter~\ref{general}, Section~\ref{aftsma}, then it would
        seem that there is some circumstantial or indirect evidence
        that the companies participating in the {\market} are
        operating optimally, and the ``average'' Shannon probability,
        $P$ for each participating company would be, using
        Equation~\ref{pncompanies}, {\pncompanies}, which would be the
        value which should be used in Section~\ref{\SETLABEL:FS} for
        each participating company if market expansion was to be
        consistent with the rest of the industry. However, if the
        Shannon probability derived in Section~\ref{\SETLABEL:FS} is
        greater than the average Shannon probability for the companies
        participating in the {\market}, as derived in this section,
        then the market would, possibly, be exploitable with the
        fiscal strategy outlined in Section~\ref{\SETLABEL:FS}. The
        maximum exploitability for the {\market} is derived in
        Section~\ref{\SETLABEL:MAXSHANNON}, but it is probably of
        doubtful practicality.

        Note that these optimizations would maximize a company's
        market growth. Since there are probably many companies
        competing in the market place, this would not necessarily
        maximize a company's P\&L, as described in
        Chapter~\ref{general}, Section~\ref{ompl}. The Shannon
        probability that maximizes market share in the {\market} is
        \pncompanies, with several alternative solutions listed in the
        previous paragraph. However, these should be contrasted to the
        Shannon probability that maximizes a company's P\&L which is
        \avgrms~in the {\market}. In all cases, the fraction of the
        P\&L that should be ``wagered'' on the future, $f$, should be:

        \begin{equation}
            f = 2P - 1
        \end{equation}

        \noindent where $P$ is the particular Shannon probability
        chosen optimize a particular fiscal strategy. Interestingly,
        the measured Shannon probability of the {\market} would tend
        to indicate that the companies participating in the market
        have chosen a fiscal strategy that optimizes market growth, as
        opposed to capital growth.

        \subidx{\market}{increasing returns}
        \subidx{economic increasing returns}{\market}
        As interesting interpretation of these exploitive issues,
        since all three fiscal strategies will result in exponential
        market growth for every company participating in the market,
        is that they may represent, perhaps, an example of
        ``increasing returns.''

% Local Variables:
% TeX-parse-self: t
% TeX-auto-save: t
% TeX-master: "fractal.tex"
% End:


        %
% -----------------------------------------------------------------------------
%
% A license is hereby granted to reproduce this software source code and
% to create executable versions from this source code for personal,
% non-commercial use.  The copyright notice included with the software
% must be maintained in all copies produced.
%
% THIS PROGRAM IS PROVIDED "AS IS". THE AUTHOR PROVIDES NO WARRANTIES
% WHATSOEVER, EXPRESSED OR IMPLIED, INCLUDING WARRANTIES OF
% MERCHANTABILITY, TITLE, OR FITNESS FOR ANY PARTICULAR PURPOSE.  THE
% AUTHOR DOES NOT WARRANT THAT USE OF THIS PROGRAM DOES NOT INFRINGE THE
% INTELLECTUAL PROPERTY RIGHTS OF ANY THIRD PARTY IN ANY COUNTRY.
%
% Copyright (c) 1994-2006, John Conover, All Rights Reserved.
%
% Comments and/or bug reports should be addressed to:
%
%     john@email.johncon.com (John Conover)
%
% -----------------------------------------------------------------------------
%
% Revision: \RCSRevision \\
% Revision Time: \RCSTime UMT \\
% Revision Date: \RCSDate \\
% Revision Id: \RCSId \\
% Revision File: \RCSLog \\
\RCS $Revision: 0.0 $
\RCS $Date: 2006/01/20 04:38:13 $
\RCS $Id: operations.tex,v 0.0 2006/01/20 04:38:13 john Exp $
% $Log: operations.tex,v $
% Revision 0.0  2006/01/20 04:38:13  john
% Initial version
%
%
    \subsection{Fixed Increment Approximation for Operational Strategy}
        \label{\SETLABEL:OPS}.

        This section derives various values based on the ``average''
        of the normalized increments presented in
        Figure~\ref{\SETLABEL:TFA}. These values are an approximation
        to a, probably, complex process with a distribution shown in
        Figure~\ref{\SETLABEL:TF}. These values will be used in a
        fixed increment Brownian fractal analysis and simulation of
        the {\market}, and may, or may not, provide adequate accuracy
        for projections.

        \subidx{\market}{fiscal strategy}
        \subidx{\market}{Shannon probability}
        \subidx{strategy}{fiscal}
        \subidx{fiscal}{strategy}
        \subidx{Shannon}{probability}
        \subidx{probability}{Shannon}
        It should be noted that the analysis of fiscal strategy,
        presented in Section~\ref{\SETLABEL:FS}, is derived from the
        {\market} metrics and may, or may not, be maximally
        optimal. For the optimal fiscal strategy, which may be
        exploitable, see Section~\ref{\SETLABEL:MAXSHANNON}.

        \subidx{strategy}{exploitable}
        \subidx{exploitable}{strategy}
        \subidx{\market}{windows of opportunity}
        \idx{windows of opportunity}
        \subidx{decision}{obsolete}
        \subidx{obsolete}{decision}
        \subidx{decision}{timeliness}
        \subidx{timeliness}{decision}
        \subidx{rate of revenue returns}{forecast}
        \subidx{forecast}{rate of revenue returns}
        An additional exploitable strategy may be time itself.
        Equations~\ref{\SETLABEL:V},~\ref{\SETLABEL:R},
        and,~\ref{\SETLABEL:MA}, are, essentially, metrics on how fast
        a decision, which is based on information concerning the
        current status of the {\market}, becomes obsolete. Obviously,
        how long a decision is expected to remain relevant should be
        addressed as an operational necessity in strategic planning
        and project management. Figures~\ref{\SETLABEL:FN},
        and,~\ref{\SETLABEL:FF} compare methods of approximation of
        the ``forecastability'' of rate of revenue returns in the
        {\market} for the near term and far
        term~\cite[pp. 83-84]{Peters:CAOITCM}, respectively. As a
        general rule, caution must be exercised when making decisions
        that will span a time interval larger than the time interval
        where the ``forecastability'' of rate of revenue returns drops
        below 50\%. Beyond this time interval, the chances increase
        that the competitive and market forces will alter the market
        environment in a possibly detrimental unanticipated
        fashion. Obviously, there is significant advantage in
        ``timeliness'' of development, manufacturing, and distribution
        of products and services that are consistent with this
        temporal agenda. Automation of these processes, if executed
        consistently with this agenda, should be considered a
        competitive advantage.

        \subidx{strategy}{exploitable}
        \subidx{exploitable}{strategy}
        \subidx{rate of revenue returns}{forecast}
        \subidx{forecast}{rate of revenue returns}
        \idx{product life cycle}
        \idx{life cycle, product}
        In some sense, this temporal agenda defines the ``average''
        product or service life cycle in the {\market}. When the
        ``forecastability'' of rate of revenue returns drops below
        50\%, there is an even chance that the rate of revenue returns
        for the product or service will change in a detrimental
        fashion. If it is assumed that a product or service life cycle
        consists of a ramp up, a maintenence interval, and a ramp
        down, then, if all three life cycle intervals are equal, the
        product life cycle will be, approximately, three times the
        time interval where the ``forecastability'' of rate of revenue
        returns drops below 50\%. Although probably not an accurate
        prediction of product or service life cycle, the technique may
        be used as a conceptual approximation to the dynamics of
        ``market windows.\footnote{For example, consider the market
        for table salt. Since it has inelastic supply and demand
        curves, and is a necessary requirement for life, it would be
        expected that the Hurst coefficient would be very near
        unity---ignoring competitive pressures in the market. The
        predictability of the table salt market would, therefore, be
        expected to be relatively good, over time.}''  The conceptual
        approximation will probably predict a ``conservative'' or
        ``pessimistic'' value in relation to actual markets.

        \begin{figure}[ht]
            \begin{center}
                \begin{minipage}[t]{0.45\textwidth}
                    \epsfxsize=1.0\linewidth
                    \epsffile{\directory/datahurstlownear.eps}
                    \caption[{\market}, ``forecastability'' of near
                        term rate of revenue returns]{{\market},
                        ``forecastability'' of near term rate of
                        revenue returns. Although the error function
                        is the most accurate, for the near term,
                        $H^{t} = \thurstlow^{t}$ may be used as a
                        reliable metric of ``forecastability'' of the
                        rate of revenue returns.}
                    \label{\SETLABEL:FN}
                \end{minipage}
                \hfill
                \begin{minipage}[t]{0.45\textwidth}
                    \epsfxsize=1.0\linewidth
                    \epsffile{\directory/datahurstlowfar.eps}
                    \caption[{\market}, ``forecastability'' of far
                        term rate of revenue returns]{{\market},
                        ``forecastability'' of far term rate of
                        revenue returns. Although the error function
                        is the most accurate, for the far term,
                        $\frac{1}{\sqrt{t}}$ may be used as a reliable
                        metric of ``forecastability'' of the rate of
                        revenue returns.}
                    \label{\SETLABEL:FF}
                \end{minipage}
            \end{center}
        \end{figure}

        \idx{operations research}
        As an interesting interpretation of the data presented in
        Figure~\ref{\SETLABEL:FN}, there may be, perhaps, some
        applicability to such operational agendas as inventory
        control. Maintaining too little inventory, obviously, will
        create a situation where the organization can not exploit
        market expansion, and maintaining too much inventory,
        likewise, would over extend the company, creating unnecessary
        losses when the market contracts. The company should maintain
        inventory levels that do not exceed, from
        Equation~\ref{\SETLABEL:MA}, ${\thurstlow}^{n} = 0.5$
        {\timescale}s of operations. Since the optimal amount of
        inventory and, from Equation~\ref{\SETLABEL:V}, the variance
        of change in the rate of revenue returns in the future can be
        calculated, there may, perhaps, be some applicability to a
        forecasting methodology that can be incorporated into other
        areas of operations research, for example the linear algebras
        using simplex methodologies for optimization of manufacturing
        processes. Traditionally, these forecasts are made by the
        sales department, and are subject to various subjective
        biases.

% Local Variables:
% TeX-parse-self: t
% TeX-auto-save: t
% TeX-master: "fractal.tex"
% End:


        %
% -----------------------------------------------------------------------------
%
% A license is hereby granted to reproduce this software source code and
% to create executable versions from this source code for personal,
% non-commercial use.  The copyright notice included with the software
% must be maintained in all copies produced.
%
% THIS PROGRAM IS PROVIDED "AS IS". THE AUTHOR PROVIDES NO WARRANTIES
% WHATSOEVER, EXPRESSED OR IMPLIED, INCLUDING WARRANTIES OF
% MERCHANTABILITY, TITLE, OR FITNESS FOR ANY PARTICULAR PURPOSE.  THE
% AUTHOR DOES NOT WARRANT THAT USE OF THIS PROGRAM DOES NOT INFRINGE THE
% INTELLECTUAL PROPERTY RIGHTS OF ANY THIRD PARTY IN ANY COUNTRY.
%
% Copyright (c) 1994-2006, John Conover, All Rights Reserved.
%
% Comments and/or bug reports should be addressed to:
%
%     john@email.johncon.com (John Conover)
%
% -----------------------------------------------------------------------------
%
% Revision: \RCSRevision \\
% Revision Time: \RCSTime UMT \\
% Revision Date: \RCSDate \\
% Revision Id: \RCSId \\
% Revision File: \RCSLog \\
\RCS $Revision: 0.0 $
\RCS $Date: 2006/01/20 04:38:13 $
\RCS $Id: simulation.tex,v 0.0 2006/01/20 04:38:13 john Exp $
% $Log: simulation.tex,v $
% Revision 0.0  2006/01/20 04:38:13  john
% Initial version
%
%
    \subsection{Simulation of Fixed Increment Approximation for Fiscal Strategy}
        \label{\SETLABEL:TSUNFAIRBROWNIAN}

        \subidx{\market}{market simulation}
        The data in this section is presented in tabular form in
        Section~\ref{\SETLABELREF:SIM}.
        Figure~\ref{\SETLABEL:TSUNFAIRBROWNIAN0} represents a
        constructional simulation of the time series data presented in
        Figure~\ref{\SETLABEL:TS}. The program {\it
        tsunfairbrownian}\/, which is briefly described in
        appendix~\ref{programs}, was used in the reconstruction. The
        reconstructed data is superimposed on the original time series
        data.  The program, {\it tsunfairbrownian}\/, essentially,
        constructs the new time series as a Brownian fractal with
        fixed increments---the value of the fixed increment is derived
        from the root mean square average of the normalized increments
        presented in Figure~\ref{\SETLABEL:TF}. The ``quality'' of
        such a reconstruction should be subject to adequate scepticism
        and scrutiny since, in all probability, the normalized
        increments presented in Figure~\ref{\SETLABEL:TF} represent a
        relatively complex process, that may not be ``modeled'' with
        such a simple methodology.

        As a further comparison of the the constructional simulation
        with the original time series data,
        Figure~\ref{\SETLABEL:TSUNFAIRBROWNIAN1} presents a normalized
        histogram of the normalized increments of the reconstructed
        time series, superimposed on the normalized histogram
        presented in Figure~\ref{\SETLABEL:NH}.

        \subidx{\market}{fiscal strategy, simulation}
        \subidx{markets}{simulation}
        \subidx{simulation}{markets}
        \subidx{strategy}{fiscal, simulation}
        \subidx{fiscal}{strategy, simulation}
        \subidx{programs}{tsunfairbrownian}
        \subidx{tsunfairbrownian}{program}
        \begin{figure}[ht]
            \begin{center}
                \begin{minipage}[t]{0.45\textwidth}
                    \epsfxsize=1.0\linewidth
                    \epsffile{\directory/tsunfairbrownian-f.eps}
                    \caption[{\market}, Time series data, empirical and
                        simulated]{{\market}, Time series data, empirical
                        and simulated, using the program {\it tsunfairbrownian}\/
                        with f = {\datafractionrms}. This data is
                        superimposed on the data presented in
                        Figure~\ref{\SETLABEL:TS}.}
                    \label{\SETLABEL:TSUNFAIRBROWNIAN0}
                \end{minipage}
                \hfill
                \begin{minipage}[t]{0.45\textwidth}
                    \epsfxsize=1.0\linewidth
                    \epsffile{\directory/tsunfairbrownian-f.tsfraction.tsnormal-s30.eps}
                    \caption[{\market}, normalized histogram,
                        empirical and simulated]{{\market}, normalized
                        histogram of the normalized increments of the
                        time series data shown in
                        Figure~\ref{\SETLABEL:TSUNFAIRBROWNIAN0},
                        empirical and simulated.  The empirical data
                        has a mean of {\datafractionmean}, with a
                        standard deviation of {\datafractionstddev}.
                        By comparison, the simulated data has a mean
                        of {\tsunfairbrownianfractionmean} with a
                        standard deviation of
                        {\tsunfairbrownianfractionstddev}. This data
                        is superimposed on the data presented in
                        Figure~\ref{\SETLABEL:NH}. The area under the
                        four curves is identical.}
                    \label{\SETLABEL:TSUNFAIRBROWNIAN1}
                \end{minipage}
            \end{center}
        \end{figure}

% Local Variables:
% TeX-parse-self: t
% TeX-auto-save: t
% TeX-master: "fractal.tex"
% End:


        %
% -----------------------------------------------------------------------------
%
% A license is hereby granted to reproduce this software source code and
% to create executable versions from this source code for personal,
% non-commercial use.  The copyright notice included with the software
% must be maintained in all copies produced.
%
% THIS PROGRAM IS PROVIDED "AS IS". THE AUTHOR PROVIDES NO WARRANTIES
% WHATSOEVER, EXPRESSED OR IMPLIED, INCLUDING WARRANTIES OF
% MERCHANTABILITY, TITLE, OR FITNESS FOR ANY PARTICULAR PURPOSE.  THE
% AUTHOR DOES NOT WARRANT THAT USE OF THIS PROGRAM DOES NOT INFRINGE THE
% INTELLECTUAL PROPERTY RIGHTS OF ANY THIRD PARTY IN ANY COUNTRY.
%
% Copyright (c) 1994-2006, John Conover, All Rights Reserved.
%
% Comments and/or bug reports should be addressed to:
%
%     john@email.johncon.com (John Conover)
%
% -----------------------------------------------------------------------------
%
% Revision: \RCSRevision \\
% Revision Time: \RCSTime UMT \\
% Revision Date: \RCSDate \\
% Revision Id: \RCSId \\
% Revision File: \RCSLog \\
\RCS $Revision: 0.0 $
\RCS $Date: 2006/01/20 04:38:13 $
\RCS $Id: maximum.tex,v 0.0 2006/01/20 04:38:13 john Exp $
% $Log: maximum.tex,v $
% Revision 0.0  2006/01/20 04:38:13  john
% Initial version
%
%
    \subsection{Simulation of Fixed Increment Approximation for Optimally Maximal Fiscal Strategy}
        \label{\SETLABEL:MAXSHANNON}
        \subidx{\market}{fiscal strategy, simulation}
        \subidx{\market}{maximum Shannon probability}
        \subidx{markets}{simulation}
        \subidx{simulation}{markets}
        \subidx{strategy}{optimum fiscal, simulation}
        \subidx{fiscal}{optimum strategy, simulation}
        \subidx{programs}{tsunfairbrownian}
        \subidx{tsunfairbrownian}{program}
        \subidx{Shannon}{probability}
        \subidx{probability}{Shannon}

        \subidx{strategy}{exploitable}
        \subidx{exploitable}{strategy}
        \subidx{programs}{tsshannonmax}
        \subidx{tsshannonmax}{program}
        \subidx{programs}{tsunfairbrownian}
        \subidx{tsunfairbrownian}{program}
        \subidx{strategy}{fiscal}
        \subidx{fiscal}{strategy}
        The data in this section is presented in tabular form in
        Section~\ref{\SETLABELREF:MAXSHANNON}. One of the issues of
        analysis, as mentioned in Section~\ref{\SETLABEL:OPS}, is to
        determine the maximum Shannon probability for the time series
        presented in Figure~\ref{\SETLABEL:TS}. Potentially, this
        could be exploited with an aggressive fiscal
        strategy. Figure~\ref{\SETLABEL:SHANNONMAX0} is a graph of the
        output of the {\it tsshannonmax}\/ program, which is described
        briefly in appendix~\ref{programs}. The maximum of this
        function is the maximum Shannon probability for the time
        series data presented in Figure~\ref{\SETLABEL:TS}.
        Figure~\ref{\SETLABEL:SHANNONMAX1} was constructed using {\it
        tsunfairbrownian}\/ program, which is also described in
        appendix~\ref{programs}, with the maximum Shannon probability,
        and the time series data presented in
        Figure~\ref{\SETLABEL:TS}. This represents a ``what if'' the
        investment strategy was changed from a Shannon probability of
        {\shannonlogreturns}, as derived in Section~\ref{\SETLABEL:SP}
        to {\shannonmax}. This process, essentially, extracts the
        random statistical data from the time series presented in
        Figure~\ref{\SETLABEL:TS}, and constructs a new time series,
        using the random statistical data, with a different investment
        strategy.  The program, {\it tsunfairbrownian}\/, essentially,
        constructs the new time series as a Brownian fractal with
        fixed increments.  The ``quality'' of such a reconstruction
        should be subject to adequate scepticism and scrutiny since,
        in all probability, the increments in the original data
        represent a relatively complex process, that may not be
        ``modeled'' with such a simple methodology.

        \begin{figure}[ht]
            \begin{center}
                \begin{minipage}[t]{0.45\textwidth}
                    \epsfxsize=1.0\linewidth
                    \epsffile{\directory/data.tsshannonmax.eps}
                    \caption[{\market}, maximum rate of revenue
                        returns] {{\market}, maximum rate of revenue
                        returns, per {\timescale}, vs. Shannon
                        probability. The maximum rate of revenue
                        returns, per {\timescale}, occurs at a Shannon
                        probability of {\shannonmax}.}
                    \label{\SETLABEL:SHANNONMAX0}
                \end{minipage}
                \hfill
                \begin{minipage}[t]{0.45\textwidth}
                    \epsfxsize=1.0\linewidth
                    \epsffile{\directory/data.tsshannonmax-p.tsunfairbrownian-p.eps}
                    \caption[{\market}, maximum rate of revenue
                        returns] {{\market}, maximum rate of revenue
                        returns, per {\timescale}, at a Shannon
                        probability, of {\shannonmax}, corresponding
                        to a ``wager'' fraction of {\twoponemax}.}
                    \label{\SETLABEL:SHANNONMAX1}
                \end{minipage}
            \end{center}
        \end{figure}

        \subidx{fractional}{Brownian motion}
        \subidx{Brownian motion}{fractional}
        \subidx{Shannon}{probability}
        \subidx{probability}{Shannon}
        \subidx{programs}{tsshannonmax}
        \subidx{tsshannonmax}{program}
        If it is assumed that the time series data set, presented in
        Figure~\ref{\SETLABEL:TS}, constitutes classical Brownian
        motion, then the Shannon probability can be calculated by
        counting the total number of {\timescale}s that the {\market}
        movement was positive, and dividing by the total number of
        {timescale}s represented in the time series. This quotient is
        {\pmax}, as compared with the predicted value from the program
        {\it tsshannonmax}\/ of {\shannonmax}.

% Local Variables:
% TeX-parse-self: t
% TeX-auto-save: t
% TeX-master: "fractal.tex"
% End:


        %
% -----------------------------------------------------------------------------
%
% A license is hereby granted to reproduce this software source code and
% to create executable versions from this source code for personal,
% non-commercial use.  The copyright notice included with the software
% must be maintained in all copies produced.
%
% THIS PROGRAM IS PROVIDED "AS IS". THE AUTHOR PROVIDES NO WARRANTIES
% WHATSOEVER, EXPRESSED OR IMPLIED, INCLUDING WARRANTIES OF
% MERCHANTABILITY, TITLE, OR FITNESS FOR ANY PARTICULAR PURPOSE.  THE
% AUTHOR DOES NOT WARRANT THAT USE OF THIS PROGRAM DOES NOT INFRINGE THE
% INTELLECTUAL PROPERTY RIGHTS OF ANY THIRD PARTY IN ANY COUNTRY.
%
% Copyright (c) 1994-2006, John Conover, All Rights Reserved.
%
% Comments and/or bug reports should be addressed to:
%
%     john@email.johncon.com (John Conover)
%
% -----------------------------------------------------------------------------
%
% Revision: \RCSRevision \\
% Revision Time: \RCSTime UMT \\
% Revision Date: \RCSDate \\
% Revision Id: \RCSId \\
% Revision File: \RCSLog \\
\RCS $Revision: 0.0 $
\RCS $Date: 2006/01/20 04:38:13 $
\RCS $Id: verification.tex,v 0.0 2006/01/20 04:38:13 john Exp $
% $Log: verification.tex,v $
% Revision 0.0  2006/01/20 04:38:13  john
% Initial version
%
%
    \subsection{Qualitative Verification of Fixed Increment Approximation Analysis}
        \label{\SETLABEL:QVA}

        \subidx{\market}{verification of analysis}
        \subidx{verification}{analysis}
        \subidx{analysis}{verification}
        \subidx{quality}{of analysis}
        \subidx{verification}{of methodology}
        \subidx{methodology}{verification of}
        \subidx{Shannon}{probability}
        \subidx{probability}{Shannon}

        This section evaluates various values based on the ``average''
        of the normalized increments presented in
        Figure~\ref{\SETLABEL:TFA}. These values are an approximation
        to a, probably, complex process with a distribution shown in
        Figure~\ref{\SETLABEL:TF}. These values will be used in a
        fixed increment Brownian fractal analysis of the {\market},
        and may, or may not, provide adequate accuracy for
        projections.

        The data in this section is presented in tabular form in
        sections~\ref{\SETLABELREF:VI1} and~\ref{\SETLABELREF:VI2}.
        As a subjective evaluation of the ``quality'' of the analysis
        of the {\market}, from Chapter~\ref{methodology},
        Equation~\ref{metricvalues1}, and using the mean and root mean
        square values of the normalized increments of the time series
        data presented in Figure~\ref{\SETLABEL:TS} from
        Figure~\ref{\SETLABEL:TF}, and the Shannon probability as
        calculated by counting the total number of {\timescale}s that
        the {\market} movement was positive, as presented in
        Section~\ref{\SETLABEL:MAXSHANNON}:

        \begin{eqnarray}
                  P & \approx & \frac{\frac{avg}{rms} + 1}{2}\\
            {\pmax} & \approx & \frac{\frac{\datafractionmean}{\datafractionrms} + 1}{2}\\
            {\pmax} & \approx & {\avgrms}
            \label{\SETLABEL:AVGS}
        \end{eqnarray}

        \subidx{Shannon}{probability}
        \subidx{probability}{Shannon}
        \noindent and comparing these values to the Shannon
        probability, as found by the {\it tsshannonmax}\/ program, which
        iterates for a maximum:

        \begin{eqnarray}
            {\pmax} \approx {\avgrms} \approx {\shannonmax}
        \end{eqnarray}

        \subidx{logarithmic}{returns}
        \subidx{returns}{logarithmic}
        In addition, the different methods of calculating the
        logarithmic returns, presented in Section~\ref{\SETLABEL:FS},
        should be compared. The four methods used were the mean of
        Figure~\ref{\SETLABEL:TF}, the constant in the least squares
        approximation to Figure~\ref{\SETLABEL:TF}, the least squares
        exponential approximation to Figure~\ref{\SETLABEL:TS}, and
        the logarithmic returns of Figure~\ref{\SETLABEL:TS}, derived
        as the mean of the logarithms of the quotients of the
        increments. The values for each of the methods are,
        respectively:

        \begin{equation}
            \datafractionmeanbits \approx \datafractionconstantbits \approx \datatslsqepbits \approx \logreturns
        \end{equation}

        It is implied in Section~\ref{\SETLABEL:FS},
        Subsection~\ref{\SETLABEL:SP} and in
        Section~\ref{\SETLABEL:TSUNFAIRBROWNIAN} that, a Brownian
        motion with fixed increments fractal may ``model'' the
        {\market}. Using Equation~\ref{stddev9} from
        Chapter~\ref{general}, Section~\ref{abmfi}:

        \begin{eqnarray}
                                    rms \left(2P - 1\right) & \approx & \frac{\sigma \left(2P - 1\right)}{2 \sqrt{P\left(1 - P\right)}}\\
            \datafractionrms \left(2 \cdot \pmax - 1\right) & \approx & \frac{\datafractionstddev \left(2 \cdot \pmax - 1\right)}{2\sqrt{\pmax \left(1 - \pmax\right)}}\\
                       \datafractionrms \cdot \twopminusone & \approx & \datafractionstddev \cdot \twopx\\
                                                      \rmsp & \approx & \sigmap
        \end{eqnarray}

        \noindent and, equating to the mean:

        \begin{equation}
            \datafractionmean \approx \rmsp \approx \sigmap
        \end{equation}

        \subidx{Shannon}{probability}
        \subidx{probability}{Shannon}
        \noindent where, as in Equation~\ref{\SETLABEL:AVGS} using the
        mean, root mean square, and standard deviation values of the
        normalized increments of the time series data presented in
        Figure~\ref{\SETLABEL:TS} from Figure~\ref{\SETLABEL:TF}, and
        the Shannon probability as calculated by counting the total
        number of {\timescale}s that the {\market} movement was
        positive, as presented in Section~\ref{\SETLABEL:MAXSHANNON}.

        As a final qualitative comparison, the absolute value of the
        normalized increments should be the same as the root mean
        square value\footnote{The absolute value of the normalized
        increments, when averaged, is related to the root mean square
        of the increments by a constant. If the normalized increments
        are a fixed increment, the constant is unity. If the
        normalized increments have a Gaussian distribution, the
        constant is $\approx 0.8$ depending on the accuracy of of
        ``fit'' to a Gaussian distribution.}, where the absolute value
        is presented in Figure~\ref{\SETLABEL:TFA}, and the root mean
        square value is presented in Figure~\ref{\SETLABEL:TF}:

        \begin{equation}
            \datafractionabsmean \approx \datafractionrms
        \end{equation}

        Note, that if the {\market} could be ``modeled'' as a Brownian
        motion with fixed increments fractal, then the standard
        deviation of the absolute value of the normalized increments
        of the time series data presented in Figure~\ref{\SETLABEL:TS}
        from Figure~\ref{\SETLABEL:TF} should be zero. It is
        $\datafractionabsstddev$.

% Local Variables:
% TeX-parse-self: t
% TeX-auto-save: t
% TeX-master: "fractal.tex"
% End:


% Local Variables:
% TeX-parse-self: t
% TeX-auto-save: t
% TeX-master: "fractal.tex"
% End:
