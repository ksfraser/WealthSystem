%
% -----------------------------------------------------------------------------
%
% A license is hereby granted to reproduce this software source code and
% to create executable versions from this source code for personal,
% non-commercial use.  The copyright notice included with the software
% must be maintained in all copies produced.
%
% THIS PROGRAM IS PROVIDED "AS IS". THE AUTHOR PROVIDES NO WARRANTIES
% WHATSOEVER, EXPRESSED OR IMPLIED, INCLUDING WARRANTIES OF
% MERCHANTABILITY, TITLE, OR FITNESS FOR ANY PARTICULAR PURPOSE.  THE
% AUTHOR DOES NOT WARRANT THAT USE OF THIS PROGRAM DOES NOT INFRINGE THE
% INTELLECTUAL PROPERTY RIGHTS OF ANY THIRD PARTY IN ANY COUNTRY.
%
% Copyright (c) 1994-2006, John Conover, All Rights Reserved.
%
% Comments and/or bug reports should be addressed to:
%
%     john@email.johncon.com (John Conover)
%
% -----------------------------------------------------------------------------
%
% Revision: \RCSRevision \\
% Revision Time: \RCSTime UMT \\
% Revision Date: \RCSDate \\
% Revision Id: \RCSId \\
% Revision File: \RCSLog \\
\RCS $Revision: 0.0 $
\RCS $Date: 2006/01/20 04:38:13 $
\RCS $Id: chap4.tex,v 0.0 2006/01/20 04:38:13 john Exp $
% $Log: chap4.tex,v $
% Revision 0.0  2006/01/20 04:38:13  john
% Initial version
%
%
\chapter{Conclusions and Observations}
    \label{conclusions}

    This chapter presents various qualitative conclusions, as analyzed
    in appendix~\ref{markets} and appendix~\ref{tables}, concerning
    the Fractal Analysis of Various Market Segments in the North
    American Electronics Industry. It should not be concluded that
    these industries are representative of industries in general, and
    is offered in academic perspective, and under no circumstances
    would it be appropriate to consider it financial advice.

    \section{Comparison of Derived Relationships with Industrial Observations}

        \idx{optimization}
        \subidx{investments}{research and development}
        \subidx{research and development}{investments}
        \subidx{venture}{capital}
        \subidx{capital}{venture}
        \idx{inventory}
        \subidx{corporate}{failure rate}
        There are some interesting relationships that were presented,
        and, although they may be coincidental, taken in the larger
        context that many of the relationships are very close to what
        industry analysts have used as ``rules of thumb,'' or ``bench
        marks'' derived through years of experience, it would seem
        that using fractal analysis on industry or market place
        historical data sets may possibly provide an additional
        analytical ``tool'' for optimizing industrial pro forma
        issues. As a partial selection of the relationships:

        \begin{itemize}

            \item Research,development, and infrastructural
            investments seem reasonable at about 12 to 20 percent of
            the rate of revenue returns for the market segments
            analyzed. This seems consistent with the industry.

            \item Venture success rates at 60 months seems reasonable
            at about 1 in 11, which is commensurate with the industry.

            \item Project success rates, of 8 month duration, are
            about 1 in 3, which is consistent with numbers from the
            Application Specific Integrated Circuit business, which
            could be considered as ``representative.''

            \item The ``80/20 rule'' that 80\% of an organization's
            revenue comes from only a few, 3 was shown to be typical,
            products is really, probably, 84.13\%, or one standard
            deviation---which is consistent through the industry.

            \item The ``80/20 rule'' that 80\% of an organization's
            products should be ``industry standard,'' and the
            remainder ``proprietary'' is probably, one standard
            deviation, or 84.13\%.

            \item Although prediction of product life cycle in the
            operations sections proved to be ``pessimistic,'' it was,
            none the less, depending on the reader's point of view,
            reasonable, and was fairly consistent with industry
            averages.

            \item The inventory control dynamics presented in the
            operations section, seem to be consistent with the markets
            analyzed.

            \item The failure rate of Fortune 500 Companies seems
            consistent with predicted failure rate of organizations in
            the markets analyzed, although the rate of failure was
            shown to be ``optimistic,'' when related to re-investment
            strategy.

            \item The calculated number of companies participating in
            the markets analyzed is reasonably close to the industry
            numbers, and there is inferential evidence that they are
            operating optimally---at least in the entropic sense as
            defined in Chapter~\ref{general}---which seems consistent
            with the economic theory that the companies that operate
            the most optimally or efficiently will, eventually,
            dominate the market. (The calculated number of companies
            participating in the various markets varied between 6 and
            28, with an average of 10, and with Shannon probabilities
            for the individual company's market time series varying
            between 0.54 and 0.6, with an average of 0.57, which,
            interestingly, is close, within approximately 5\%, to the
            Shannon probability for the various company's stock price
            time series.)

            \item The variance in the aggregate market time series is
            smaller than the the variance of the time series for any
            company participating in the market, which is consistent
            with the industries analyzed.

            \item It would seem that there is some supporting evidence
            that optimizing a company's fiscal strategy to achieve
            maximum market growth and optimizing a company's fiscal
            strategy to optimize capital growth may be mutually
            exclusive, which has, traditionally, been the case in the
            industries analyzed. Additionally, it would seem that, at
            least in the markets analyzed, the fiscal strategies
            deployed would tend to be optimizing market growth, which
            seems consistent with author's experience in these
            industries.

        \end{itemize}

    \section{Asides and Speculations}

        Several issues that were not addressed are the relationship
        between a company's valuation, perhaps calculated by the value
        of its stock, and company's rate of revenue returns. If their
        is a causality, it would seem that there would be, in general,
        a reason offered as to why the Dow Jones Average is rising
        exponentially. Additionally, it would seem that there would be
        a correlation that could be confirmed in employment figures,
        economic indicators, flow of money, etc. These speculations
        are offered as a suggestion for further investigation into the
        applicability of fractal analysis to industrial markets---as a
        possible means of induction.

        \idx{operations research}
        There is some possibility that fractal analysis can be used in
        conjuction with other contemporary methodologies of operations
        research. For example, possibly, failure analysis could be
        used to optimize the expected life of a company vs.\ the
        growth rate of a company as alluded to in the optimally
        maximal fiscal strategy sections of
        appendix~\ref{markets}. Additionally, perhaps, fractal
        analysis could be used in forecasting market dynamics in
        conjunction with mathematical methods and linear programming
        optimization of corporate operations, specifically inventory
        control.

        \subidx{fractal analysis}{relation to management methodology}
        \subidx{management methodology}{relation to fractal analysis}
        There are remaining issues that, although addressed, were not
        addressed to the author's satisfaction. Specifically, there
        was no reason offered as to why the companies analyzed were
        not operating closer to the maximum Shannon probabilities, as
        presented in the simulation and maximization sections of
        appendix~\ref{markets}. Additionally, it would seem that
        visibility into the future, regarding rate of revenue returns,
        was only a few months, at best. This would seem to be in
        disagreement with the prevailing concept that ``strategic
        planning'' should be ``long term.'' An interesting
        interpretation of this may be that these industries require a
        more dynamic management methodology, perhaps using ``rolling''
        budgets, etc. But this would seem to be inconsistent with
        methodologies where objectives are monitored on an annual
        basis. It would seem that, looking at the graphs of the
        normalized increments in all sections presented in
        appendix~\ref{markets}, that profit and loss issues are very
        dynamic, and, probably, require detailed attention at no more
        than a monthly rate---these graphs show that a lot of dynamic
        changes can occur in a year, or even a quarter.

    \section{Conclusion}

        Overall, taken in context, it would seem, depending on the
        reader's point of view, that fractal analysis could provide
        additional insight into market and industrial
        operations---perhaps offering appropriate optimizations in
        specific circumstances. Granted, this is a controversial usage
        of the methodology, and there are interpretations that were
        made, which may or may not be considered appropriate---the
        text states that, in all cases where such interpretations were
        made, that it was an ``interesting interpretation,'' in an
        attempt not to mislead the reader. The application of fractal
        analysis to the optimization of industrial operations should
        be considered as ``novel,'' and, although there may be
        academic value, it would be inappropriate to accept any
        conclusions presented in this manuscript as ``factual'' at
        this time.

% Local Variables:
% TeX-parse-self: t
% TeX-auto-save: t
% TeX-master: "fractal.tex"
% End:
