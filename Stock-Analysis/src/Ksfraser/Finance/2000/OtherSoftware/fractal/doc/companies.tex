%
% -----------------------------------------------------------------------------
%
% A license is hereby granted to reproduce this software source code and
% to create executable versions from this source code for personal,
% non-commercial use.  The copyright notice included with the software
% must be maintained in all copies produced.
%
% THIS PROGRAM IS PROVIDED "AS IS". THE AUTHOR PROVIDES NO WARRANTIES
% WHATSOEVER, EXPRESSED OR IMPLIED, INCLUDING WARRANTIES OF
% MERCHANTABILITY, TITLE, OR FITNESS FOR ANY PARTICULAR PURPOSE.  THE
% AUTHOR DOES NOT WARRANT THAT USE OF THIS PROGRAM DOES NOT INFRINGE THE
% INTELLECTUAL PROPERTY RIGHTS OF ANY THIRD PARTY IN ANY COUNTRY.
%
% Copyright (c) 1994-2006, John Conover, All Rights Reserved.
%
% Comments and/or bug reports should be addressed to:
%
%     john@email.johncon.com (John Conover)
%
% -----------------------------------------------------------------------------
%
% Revision: \RCSRevision \\
% Revision Time: \RCSTime UMT \\
% Revision Date: \RCSDate \\
% Revision Id: \RCSId \\
% Revision File: \RCSLog \\
\RCS $Revision: 0.0 $
\RCS $Date: 2006/01/20 04:38:13 $
\RCS $Id: companies.tex,v 0.0 2006/01/20 04:38:13 john Exp $
% $Log: companies.tex,v $
% Revision 0.0  2006/01/20 04:38:13  john
% Initial version
%
%
    \subsection{Number of Companies}
        \label{\SETLABEL:QNC}

        \subidx{\market}{number of companies}
        \subidx{number of companies}{analysis}
        \subidx{analysis}{number of companies}
        \subidx{Shannon}{probability}
        \subidx{probability}{Shannon}
        This section evaluates the approximate, or ``average,'' number
        of companies in the {\market}, and uses the method outlined in
        Chapter~\ref{general}, Section~\ref{aftsma}. Since the
        average, $avg_{ind}$, and the root mean square, $rms_{ind}$,
        of the normalized increments of the {\market} time series is
        \datafractionmean, and \datafractionrms respectively, the
        number of companies participating in the market can be
        calculated by Equation~\ref{ncompanies} to be {\ncompanies}.

        If this value seems consistent number of companies in the
        {\market}, within the assumptions outlined in
        Chapter~\ref{general}, Section~\ref{aftsma}, then it would
        seem that there is some circumstantial or indirect evidence
        that the companies participating in the {\market} are
        operating optimally, and the ``average'' Shannon probability,
        $P$ for each participating company would be, using
        Equation~\ref{pncompanies}, {\pncompanies}, which would be the
        value which should be used in Section~\ref{\SETLABEL:FS} for
        each participating company if market expansion was to be
        consistent with the rest of the industry. However, if the
        Shannon probability derived in Section~\ref{\SETLABEL:FS} is
        greater than the average Shannon probability for the companies
        participating in the {\market}, as derived in this section,
        then the market would, possibly, be exploitable with the
        fiscal strategy outlined in Section~\ref{\SETLABEL:FS}. The
        maximum exploitability for the {\market} is derived in
        Section~\ref{\SETLABEL:MAXSHANNON}, but it is probably of
        doubtful practicality.

        Note that these optimizations would maximize a company's
        market growth. Since there are probably many companies
        competing in the market place, this would not necessarily
        maximize a company's P\&L, as described in
        Chapter~\ref{general}, Section~\ref{ompl}. The Shannon
        probability that maximizes market share in the {\market} is
        \pncompanies, with several alternative solutions listed in the
        previous paragraph. However, these should be contrasted to the
        Shannon probability that maximizes a company's P\&L which is
        \avgrms~in the {\market}. In all cases, the fraction of the
        P\&L that should be ``wagered'' on the future, $f$, should be:

        \begin{equation}
            f = 2P - 1
        \end{equation}

        \noindent where $P$ is the particular Shannon probability
        chosen optimize a particular fiscal strategy. Interestingly,
        the measured Shannon probability of the {\market} would tend
        to indicate that the companies participating in the market
        have chosen a fiscal strategy that optimizes market growth, as
        opposed to capital growth.

        \subidx{\market}{increasing returns}
        \subidx{economic increasing returns}{\market}
        As interesting interpretation of these exploitive issues,
        since all three fiscal strategies will result in exponential
        market growth for every company participating in the market,
        is that they may represent, perhaps, an example of
        ``increasing returns.''

% Local Variables:
% TeX-parse-self: t
% TeX-auto-save: t
% TeX-master: "fractal.tex"
% End:
