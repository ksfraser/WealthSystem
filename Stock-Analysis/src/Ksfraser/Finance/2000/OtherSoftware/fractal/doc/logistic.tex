%
% -----------------------------------------------------------------------------
%
% A license is hereby granted to reproduce this software source code and
% to create executable versions from this source code for personal,
% non-commercial use.  The copyright notice included with the software
% must be maintained in all copies produced.
%
% THIS PROGRAM IS PROVIDED "AS IS". THE AUTHOR PROVIDES NO WARRANTIES
% WHATSOEVER, EXPRESSED OR IMPLIED, INCLUDING WARRANTIES OF
% MERCHANTABILITY, TITLE, OR FITNESS FOR ANY PARTICULAR PURPOSE.  THE
% AUTHOR DOES NOT WARRANT THAT USE OF THIS PROGRAM DOES NOT INFRINGE THE
% INTELLECTUAL PROPERTY RIGHTS OF ANY THIRD PARTY IN ANY COUNTRY.
%
% Copyright (c) 1994-2006, John Conover, All Rights Reserved.
%
% Comments and/or bug reports should be addressed to:
%
%     john@email.johncon.com (John Conover)
%
% -----------------------------------------------------------------------------
%
% Revision: \RCSRevision \\
% Revision Time: \RCSTime UMT \\
% Revision Date: \RCSDate \\
% Revision Id: \RCSId \\
% Revision File: \RCSLog \\
\RCS $Revision: 0.0 $
\RCS $Date: 2006/01/20 04:38:13 $
\RCS $Id: logistic.tex,v 0.0 2006/01/20 04:38:13 john Exp $
% $Log: logistic.tex,v $
% Revision 0.0  2006/01/20 04:38:13  john
% Initial version
%
%
    \subsection{Logistic Analysis}
        \label{\SETLABEL:LA}

        \subidx{\market}{Logistic function analysis}
        \subidx{time series}{logistic function}
        \subidx{logistic function}{time series}
        \subidx{time series}{increments}
        \subidx{time series}{analysis}
        \subidx{cumulative sum}{analysis}
        \subidx{analysis}{cumulative sum}
        \subidx{analysis}{random process}
        \subidx{random process}{analysis}
        The data in this section is presented in tabular form in
        Section~\ref{\SETLABELREF:LAA}.  Figure~\ref{\SETLABEL:LA1} is
        a graph of the logistic function estimates of the time series
        data for the {\market}. The reader is cautioned that these
        graphs are constructed using the method suggested in
        Chapter~\ref{general}, Section~\ref{nlextend} and enormous
        precision is required for adequate prediction of the logistic
        function,~\cite{Modis}. Particularly, the non-linear term will
        usually require intervention to produce a practical fit to the
        data. In addition, there are numerical stability issues with
        logistic function methodologies\footnote{For example, in
        Figures~\ref{\SETLABEL:LA1} and~\ref{\SETLABEL:LA2}, if the
        non-linear term, $b$, was greater than zero, it was set to
        zero to produce the graphs. See Section~\ref{\SETLABELREF:LAA}
        for the actual derived values. In other cases, the magnitude
        of $b$ was too large, resulting in a graph that was decreasing
        as a function of time}.  The methodology should be regarded as
        ``fragile.'' It is included for completeness.

        \idx{least squares approximation}
        Figure~\ref{\SETLABEL:LA1} is a graph of the logistic function
        for the time series data presented in
        Figure~\ref{\SETLABEL:TS}. The data presented was made by
        running the program {\it tsdlogistic}\/, which is described
        briefly in Appendix~\ref{programs}, on the parameters
        extracted from the time series data as suggested in
        Figure~\ref{\SETLABEL:TF}. The program {\it tslsq}\/ was used
        to derive the constant and the slope of the normalized
        increments of the data presented in Figure~\ref{\SETLABEL:TF}.
        Figure~\ref{\SETLABEL:LA2} is the same graph, but with the
        time scale expanded by a factor of two.

        \begin{figure}[ht]
            \begin{center}
                \begin{minipage}[t]{0.45\textwidth}
                    \epsfxsize=1.0\linewidth
                    \epsffile{\directory/data.tsfraction.tslsq-p.tsdlogistic.eps}
                    \caption[{\market}, logistic function
                        estimates.]{{\market}, logistic function
                        estimates, provided by running the {\it
                        tslsq}\/ program on the normalized increments
                        presented in Figure~\ref{\SETLABEL:TF} with
                        the -p option. These parameters were used as
                        arguments to the {\it tsdlogistic}\/ program.}
                    \label{\SETLABEL:LA1}
                    \label{\SETLABELQ:LA1}
                \end{minipage}
                \hfill
                \begin{minipage}[t]{0.45\textwidth}
                    \epsfxsize=1.0\linewidth
                    \epsffile{\directory/data.tsfraction.tslsq-p.tsdlogistic2.eps}
                    \caption[{\market}, logistic function
                        estimates.]{{\market}, logistic function
                        estimates of Figure~\ref{\SETLABEL:LA1} with
                        the time scale expanded by a factor of two.}
                    \label{\SETLABEL:LA2}
                    \label{\SETLABELQ:LA2}
                \end{minipage}
            \end{center}
        \end{figure}

% Local Variables:
% TeX-parse-self: t
% TeX-auto-save: t
% TeX-master: "fractal.tex"
% End:
