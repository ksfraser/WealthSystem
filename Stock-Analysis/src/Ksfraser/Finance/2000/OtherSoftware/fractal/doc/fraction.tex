%
% -----------------------------------------------------------------------------
%
% A license is hereby granted to reproduce this software source code and
% to create executable versions from this source code for personal,
% non-commercial use.  The copyright notice included with the software
% must be maintained in all copies produced.
%
% THIS PROGRAM IS PROVIDED "AS IS". THE AUTHOR PROVIDES NO WARRANTIES
% WHATSOEVER, EXPRESSED OR IMPLIED, INCLUDING WARRANTIES OF
% MERCHANTABILITY, TITLE, OR FITNESS FOR ANY PARTICULAR PURPOSE.  THE
% AUTHOR DOES NOT WARRANT THAT USE OF THIS PROGRAM DOES NOT INFRINGE THE
% INTELLECTUAL PROPERTY RIGHTS OF ANY THIRD PARTY IN ANY COUNTRY.
%
% Copyright (c) 1994-2006, John Conover, All Rights Reserved.
%
% Comments and/or bug reports should be addressed to:
%
%     john@email.johncon.com (John Conover)
%
% -----------------------------------------------------------------------------
%
% Revision: \RCSRevision \\
% Revision Time: \RCSTime UMT \\
% Revision Date: \RCSDate \\
% Revision Id: \RCSId \\
% Revision File: \RCSLog \\
\RCS $Revision: 0.0 $
\RCS $Date: 2006/01/20 04:38:13 $
\RCS $Id: fraction.tex,v 0.0 2006/01/20 04:38:13 john Exp $
% $Log: fraction.tex,v $
% Revision 0.0  2006/01/20 04:38:13  john
% Initial version
%
%
    \subsection{Time Series Increments Analysis}
        \label{\SETLABEL:TSA}

        \subidx{\market}{Time series analysis}
        \subidx{time series}{increments}
        \subidx{time series}{analysis}
        \subidx{cumulative sum}{analysis}
        \subidx{analysis}{cumulative sum}
        \subidx{analysis}{random process}
        \subidx{random process}{analysis}
        \subidx{Gaussian}{increments}
        \subidx{increments}{Gaussian}
        \subidx{Brownian}{motion, fractional}
        \subidx{fractional}{Brownian motion}
        \subidx{fractal}{Brownian motion}
        The data in this section is presented in tabular form in
        Section~\ref{\SETLABELREF:TSA}.  Figure~\ref{\SETLABEL:TS} is
        a graph of the time series data for the {\market}.

        \subidx{increments}{normalized}
        \subidx{normalized}{increments}
        \subidx{programs}{tsfraction}
        \subidx{tsfraction}{program}
        Figure~\ref{\SETLABEL:TF} is a graph of the normalized
        increments of the time series data presented in
        Figure~\ref{\SETLABEL:TS}. The data presented was made by
        running the program {\it tsfraction}\/ on the time series
        data. The program {\it tsfraction}\/ is described briefly in
        Appendix~\ref{programs}, and subtracts the previous value from
        the next value, dividing this difference by the previous
        value, for each element in the time series data. The new time
        series contains the instantaneous change in the rate of
        revenue returns, divided by the magnitude of the instantaneous
        rate of revenue returns.

        \subidx{mean}{standard deviation}
        \subidx{standard deviation}{mean}
        \idx{root mean square}
        \idx{least squares approximation}
        \begin{figure}[ht]
            \begin{center}
                \begin{minipage}[t]{0.45\textwidth}
                    \epsfxsize=1.0\linewidth
                    \epsffile{\directory/data.eps}
                    \caption{{\market}, time series data.}
                    \label{\SETLABEL:TS}
                    \label{\SETLABELQ:TS}
                \end{minipage}
                \hfill
                \begin{minipage}[t]{0.45\textwidth}
                    \epsfxsize=1.0\linewidth
                    \epsffile{\directory/data.tsfraction.eps}
                    \caption[{\market}, normalized
                        increments]{{\market}, normalized increments
                        of the time series data presented in
                        Figure~\ref{\SETLABEL:TS}. The mean is
                        {\datafractionmean} with a standard deviation
                        of {\datafractionstddev}. The formula for the
                        least squares approximation is
                        ${\datafractionconstant} +
                        {\datafractionslope}t$, and the root mean
                        squared value is {\datafractionrms}. The
                        graph, labeled ``data\-.tsfraction\-.tsrms,''
                        is the running root mean square, and
                        ``data\-.tsfraction\-.tsavg'' is the running
                        average of the normalized increments.  This
                        graph is the fraction of change in the time
                        series, as a function of time. Note that the
                        slope of the mean, {\datafractionslope}, is
                        the coefficient of the nonlinearity term in
                        the normalized increments. See
                        Chapter~\ref{general}, Section~\ref{nlextend}
                        for a possible application of the logistic
                        function to this data set.}
                    \label{\SETLABEL:TF}
                    \label{\SETLABELQ:TF}
                \end{minipage}
            \end{center}
        \end{figure}

        \subidx{absolute value}{increments}
        \subidx{increments}{absolute value}

        Figure~\ref{\SETLABEL:TFA} is a graph of the absolute value of
        the normalized increments of the time series data presented in
        Figure~\ref{\SETLABEL:TF}. The data presented was made by
        running the Unix utility sed(1) on the normalized increments
        time series data to remove the negative signs. This is an
        absolute value procedure.  The resulting time series contains
        the absolute value of the instantaneous change in the rate of
        revenue returns, divided by the magnitude of the instantaneous
        rate of revenue returns\footnote{The absolute value of the
        normalized increments, when averaged, is related to the root
        mean square of the increments by a constant. If the normalized
        increments are a fixed increment, the constant is unity. If
        the normalized increments have a Gaussian distribution, the
        constant is $\approx 0.8$ depending on the accuracy of of
        ``fit'' to a Gaussian distribution.}.

        \subidx{histogram}{normalized}
        \subidx{normalized}{histogram}
        \subidx{programs}{tsnormal}
        \subidx{tsnormal}{program}
        \subidx{mean}{standard deviation}
        \subidx{standard deviation}{mean}
        \idx{root mean square}
        \idx{least squares approximation}
        \subidx{\market}{analysis of increments}
        Figure~\ref{\SETLABEL:NH} is the normalized histogram of the
        normalized increments of the time series data shown in
        Figure~\ref{\SETLABEL:TF}. The abscissa is 3 $\sigma$ limits,
        and the area under the two curves is identical. The data for
        this figure was produced by the program {\it tsnormal}\/,
        which is described briefly in Appendix~\ref{programs}.

        \begin{figure}[ht]
            \begin{center}
                \begin{minipage}[t]{0.45\textwidth}
                    \epsfxsize=1.0\linewidth
                    \epsffile{\directory/data.tsfraction.abs.eps}
                    \caption[{\market}, absolute value of the
                        normalized increments]{{\market}, absolute
                        value of the normalized increments of the time
                        series data presented in
                        Figure~\ref{\SETLABEL:TF}.  The mean is
                        {\datafractionabsmean} with a standard
                        deviation of {\datafractionabsstddev}. The
                        formula for the least squares approximation is
                        ${\datafractionabsconstant} +
                        {\datafractionabsslope}t$, and the root mean
                        square value, from Figure~\ref{\SETLABEL:TF},
                        is {\datafractionrms}.  The graph, labeled
                        ``data\-.tsfraction\-.tsrms,'' is the running
                        root mean square, and
                        ``data\-.tsfraction\-.tsavg'' is the running
                        average of the normalized increments presented
                        in Figure~\ref{\SETLABEL:TF}, superimposed
                        here for convenience. This graph is the
                        absolute value of the fraction of change in
                        the time series, as a function of time.}
                    \label{\SETLABEL:TFA}
                    \label{\SETLABELQ:TFA}
                \end{minipage}
                \hfill
                \begin{minipage}[t]{0.45\textwidth}
                    \epsfxsize=1.0\linewidth
                    \epsffile{\directory/data.tsfraction.tsnormal-s30.eps}
                    \caption[{\market}, normalized histogram of the
                        normalized increments]{{\market}, normalized
                        histogram of the normalized increments of the
                        time series data shown in
                        Figure~\ref{\SETLABEL:TF}.  The data has a
                        mean of {\datafractionmean}, with a standard
                        deviation of {\datafractionstddev}.  The area
                        under the two curves is identical. The
                        $\chi^2$ value of the observed and expected
                        values of the two curves is {\chisquared},
                        with a critical value of {\critical}.}
                    \label{\SETLABEL:NH}
                \end{minipage}
            \end{center}
        \end{figure}

        \subidx{programs}{tsXsquared}
        \subidx{tsXsquared}{program}
        \subidx{\market}{chi-squared values of increments}
        The program {\it tsXsquared}\/, which is briefly described in
        appendix~\ref{programs}, was used to derive the $\chi^2$
        statistics for the data presented in
        Figure~\ref{\SETLABEL:NH}.

        \subidx{programs}{tsstatest}
        \subidx{tsstatest}{program}
        \subidx{\market}{statistical estimates}

        Figure~\ref{\SETLABEL:SE} is the statistical estimate for the
        data presented in Figure~\ref{\SETLABEL:TF}, as derived by the
        program {\it tsstatest}\/, which is briefly described in
        appendix~\ref{programs}.

        \begin{figure}[ht]
            \begin{center}
                \begin{minipage}[t]{\textwidth}
                    \center{\fbox{\parbox{0.9\textwidth}{\XXX{\directory/data.tsstatest-f0.1-c0.9-i.tex}}}}
                    \caption[{\market}, statistical estimates of the
                        normalized increments]{{\market}, statistical
                        estimates of the normalized increments of the
                        time series shown in Figure~\ref{\SETLABEL:TF}.
                        The table was produced with the {\it
                        tsstatest}\/ program, and illustrates the
                        size of the data set required for a confidence
                        level of 90\%, with an error estimate of $\pm$
                        10\%, or alternately, the error estimate on
                        the time series shown in Figure~\ref{\SETLABEL:TF}.}
                    \label{\SETLABEL:SE}
                \end{minipage}
            \end{center}
        \end{figure}

        Note that the data set size estimations, as produced by the
        {\it tsstatest}\/ program, are probably very conservative,
        depending on the magnitude of the Shannon probability, $P =
        \shannonlogreturns$, as derived in
        Section~\ref{\SETLABEL:SP}. See Chapter~\ref{general},
        Section~\ref{serdss} for possible alternative methodologies
        for addressing the analysis of fractal time series with
        limited data set sizes. Depending on the magnitude of the
        Shannon probability, $P$, these estimates can be several
        orders of magnitude too high.

        \subidx{derivative of increments}{normalized}
        \subidx{normalized}{derivative of increments}
        \subidx{programs}{tsderivative}
        \subidx{tsderivative}{program}
        Figure~\ref{\SETLABEL:TF1} is the normalized histogram of the
        first derivative of the normalized increments of the time
        series data shown in Figure~\ref{\SETLABEL:TF}. In principle,
        if the distribution of the normalized increments presented in
        Figure~\ref{\SETLABEL:NH} is Gaussian in nature, this
        distribution would be similar to ``white noise,'' as presented
        in appendix~\ref{programs}, Figure~\ref{whiteexample}. The
        data was generated by the {\it tsderivative}\/ program, which
        is briefly described in
        appendix~\ref{programs}. Figure~\ref{\SETLABEL:TF2} is the
        normalized histogram of the second derivative of the
        normalized increments of the time series data shown in
        Figure~\ref{\SETLABEL:TF}. In principle, if the distribution
        of the normalized increments presented in
        Figure~\ref{\SETLABEL:NH} is an integrated Gaussian
        distribution in nature, this distribution would be similar to
        ``white noise,'' as presented in appendix~\ref{programs},
        Figure~\ref{whiteexample}.

        \begin{figure}[ht]
            \begin{center}
                \begin{minipage}[t]{0.45\textwidth}
                    \epsfxsize=1.0\linewidth
                    \epsffile{\directory/data.tsfraction.tsderivative.tsnormal-s30.eps}
                    \caption[{\market}, histogram of the first
                        derivative of the increments]{{\market},
                        normalized histogram of the first derivative
                        of the normalized increments of the time
                        series data shown in
                        Figure~\ref{\SETLABEL:TF}.}
                    \label{\SETLABEL:TF1}
                \end{minipage}
                \hfill
                \begin{minipage}[t]{0.45\textwidth}
                    \epsfxsize=1.0\linewidth
                    \epsffile{\directory/data.tsfraction.2tsderivative.tsnormal-s30.eps}
                    \caption[{\market}, histogram of the second
                        derivative of the increments]{{\market},
                        normalized histogram of second derivative of
                        the the normalized increments of the time
                        series data shown in
                        Figure~\ref{\SETLABEL:TF}.}
                    \label{\SETLABEL:TF2}
                \end{minipage}
            \end{center}
        \end{figure}

        \subidx{fractal}{range}
        \subidx{fractal}{R/S analysis}
        \subidx{\market}{rate of revenue returns, range}
        \subidx{\market}{deterministic mechanism}
        \subidx{deterministic}{mechanism}
        \subidx{mechanism}{deterministic}
        Figure~\ref{\SETLABEL:TR} is the range of values of the time
        series shown in Figure~\ref{\SETLABEL:TS}. The horizontal axis
        is time into the future. In principle, if the time series was
        characterized as fractional Brownian motion the graph in
        Figure~\ref{\SETLABEL:TR} would be a square root
        function\footnote{Note that the ``roughness,'' or ``sawtooth''
        characteristics of the graph in Figure~\ref{\SETLABEL:TR} are
        a computational artifact---caused by not using the -m option
        to the program {\it tshurst}\/, which is computationally
        inefficient.}. Figure~\ref{\SETLABEL:TD} is the deterministic
        map of the normalized increments of the time series data shown
        in Figure~\ref{\SETLABEL:TF}. The deterministic map is useful
        for determining if a time series was created by a
        deterministic mechanism. This, essentially, maps each element
        in the time series with the previous element in the time
        series.  See,~\cite[pp. 745]{Peitgen}.

        \begin{figure}[ht]
            \begin{center}
                \begin{minipage}[t]{0.45\textwidth}
                    \epsfxsize=1.0\linewidth
                    \epsffile{\directory/data.tshurst-f.eps}
                    \caption[{\market}, range]{{\market}, range of the
                        time series data shown in
                        Figure~\ref{\SETLABEL:TS}.}
                    \label{\SETLABEL:TR}
                \end{minipage}
                \hfill
                \begin{minipage}[t]{0.45\textwidth}
                    \epsfxsize=1.0\linewidth
                    \epsffile{\directory/data.tsfraction.tsdeterministic.eps}
                    \caption[{\market}, deterministic map]{{\market},
                        deterministic map of the normalized increments
                        of the time series data shown in
                        Figure~\ref{\SETLABEL:TF}.}
                    \label{\SETLABEL:TD}
                \end{minipage}
            \end{center}
        \end{figure}

% Local Variables:
% TeX-parse-self: t
% TeX-auto-save: t
% TeX-master: "fractal.tex"
% End:
