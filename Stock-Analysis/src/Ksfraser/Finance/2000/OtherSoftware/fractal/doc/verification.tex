%
% -----------------------------------------------------------------------------
%
% A license is hereby granted to reproduce this software source code and
% to create executable versions from this source code for personal,
% non-commercial use.  The copyright notice included with the software
% must be maintained in all copies produced.
%
% THIS PROGRAM IS PROVIDED "AS IS". THE AUTHOR PROVIDES NO WARRANTIES
% WHATSOEVER, EXPRESSED OR IMPLIED, INCLUDING WARRANTIES OF
% MERCHANTABILITY, TITLE, OR FITNESS FOR ANY PARTICULAR PURPOSE.  THE
% AUTHOR DOES NOT WARRANT THAT USE OF THIS PROGRAM DOES NOT INFRINGE THE
% INTELLECTUAL PROPERTY RIGHTS OF ANY THIRD PARTY IN ANY COUNTRY.
%
% Copyright (c) 1994-2006, John Conover, All Rights Reserved.
%
% Comments and/or bug reports should be addressed to:
%
%     john@email.johncon.com (John Conover)
%
% -----------------------------------------------------------------------------
%
% Revision: \RCSRevision \\
% Revision Time: \RCSTime UMT \\
% Revision Date: \RCSDate \\
% Revision Id: \RCSId \\
% Revision File: \RCSLog \\
\RCS $Revision: 0.0 $
\RCS $Date: 2006/01/20 04:38:13 $
\RCS $Id: verification.tex,v 0.0 2006/01/20 04:38:13 john Exp $
% $Log: verification.tex,v $
% Revision 0.0  2006/01/20 04:38:13  john
% Initial version
%
%
    \subsection{Qualitative Verification of Fixed Increment Approximation Analysis}
        \label{\SETLABEL:QVA}

        \subidx{\market}{verification of analysis}
        \subidx{verification}{analysis}
        \subidx{analysis}{verification}
        \subidx{quality}{of analysis}
        \subidx{verification}{of methodology}
        \subidx{methodology}{verification of}
        \subidx{Shannon}{probability}
        \subidx{probability}{Shannon}

        This section evaluates various values based on the ``average''
        of the normalized increments presented in
        Figure~\ref{\SETLABEL:TFA}. These values are an approximation
        to a, probably, complex process with a distribution shown in
        Figure~\ref{\SETLABEL:TF}. These values will be used in a
        fixed increment Brownian fractal analysis of the {\market},
        and may, or may not, provide adequate accuracy for
        projections.

        The data in this section is presented in tabular form in
        sections~\ref{\SETLABELREF:VI1} and~\ref{\SETLABELREF:VI2}.
        As a subjective evaluation of the ``quality'' of the analysis
        of the {\market}, from Chapter~\ref{methodology},
        Equation~\ref{metricvalues1}, and using the mean and root mean
        square values of the normalized increments of the time series
        data presented in Figure~\ref{\SETLABEL:TS} from
        Figure~\ref{\SETLABEL:TF}, and the Shannon probability as
        calculated by counting the total number of {\timescale}s that
        the {\market} movement was positive, as presented in
        Section~\ref{\SETLABEL:MAXSHANNON}:

        \begin{eqnarray}
                  P & \approx & \frac{\frac{avg}{rms} + 1}{2}\\
            {\pmax} & \approx & \frac{\frac{\datafractionmean}{\datafractionrms} + 1}{2}\\
            {\pmax} & \approx & {\avgrms}
            \label{\SETLABEL:AVGS}
        \end{eqnarray}

        \subidx{Shannon}{probability}
        \subidx{probability}{Shannon}
        \noindent and comparing these values to the Shannon
        probability, as found by the {\it tsshannonmax}\/ program, which
        iterates for a maximum:

        \begin{eqnarray}
            {\pmax} \approx {\avgrms} \approx {\shannonmax}
        \end{eqnarray}

        \subidx{logarithmic}{returns}
        \subidx{returns}{logarithmic}
        In addition, the different methods of calculating the
        logarithmic returns, presented in Section~\ref{\SETLABEL:FS},
        should be compared. The four methods used were the mean of
        Figure~\ref{\SETLABEL:TF}, the constant in the least squares
        approximation to Figure~\ref{\SETLABEL:TF}, the least squares
        exponential approximation to Figure~\ref{\SETLABEL:TS}, and
        the logarithmic returns of Figure~\ref{\SETLABEL:TS}, derived
        as the mean of the logarithms of the quotients of the
        increments. The values for each of the methods are,
        respectively:

        \begin{equation}
            \datafractionmeanbits \approx \datafractionconstantbits \approx \datatslsqepbits \approx \logreturns
        \end{equation}

        It is implied in Section~\ref{\SETLABEL:FS},
        Subsection~\ref{\SETLABEL:SP} and in
        Section~\ref{\SETLABEL:TSUNFAIRBROWNIAN} that, a Brownian
        motion with fixed increments fractal may ``model'' the
        {\market}. Using Equation~\ref{stddev9} from
        Chapter~\ref{general}, Section~\ref{abmfi}:

        \begin{eqnarray}
                                    rms \left(2P - 1\right) & \approx & \frac{\sigma \left(2P - 1\right)}{2 \sqrt{P\left(1 - P\right)}}\\
            \datafractionrms \left(2 \cdot \pmax - 1\right) & \approx & \frac{\datafractionstddev \left(2 \cdot \pmax - 1\right)}{2\sqrt{\pmax \left(1 - \pmax\right)}}\\
                       \datafractionrms \cdot \twopminusone & \approx & \datafractionstddev \cdot \twopx\\
                                                      \rmsp & \approx & \sigmap
        \end{eqnarray}

        \noindent and, equating to the mean:

        \begin{equation}
            \datafractionmean \approx \rmsp \approx \sigmap
        \end{equation}

        \subidx{Shannon}{probability}
        \subidx{probability}{Shannon}
        \noindent where, as in Equation~\ref{\SETLABEL:AVGS} using the
        mean, root mean square, and standard deviation values of the
        normalized increments of the time series data presented in
        Figure~\ref{\SETLABEL:TS} from Figure~\ref{\SETLABEL:TF}, and
        the Shannon probability as calculated by counting the total
        number of {\timescale}s that the {\market} movement was
        positive, as presented in Section~\ref{\SETLABEL:MAXSHANNON}.

        As a final qualitative comparison, the absolute value of the
        normalized increments should be the same as the root mean
        square value\footnote{The absolute value of the normalized
        increments, when averaged, is related to the root mean square
        of the increments by a constant. If the normalized increments
        are a fixed increment, the constant is unity. If the
        normalized increments have a Gaussian distribution, the
        constant is $\approx 0.8$ depending on the accuracy of of
        ``fit'' to a Gaussian distribution.}, where the absolute value
        is presented in Figure~\ref{\SETLABEL:TFA}, and the root mean
        square value is presented in Figure~\ref{\SETLABEL:TF}:

        \begin{equation}
            \datafractionabsmean \approx \datafractionrms
        \end{equation}

        Note, that if the {\market} could be ``modeled'' as a Brownian
        motion with fixed increments fractal, then the standard
        deviation of the absolute value of the normalized increments
        of the time series data presented in Figure~\ref{\SETLABEL:TS}
        from Figure~\ref{\SETLABEL:TF} should be zero. It is
        $\datafractionabsstddev$.

% Local Variables:
% TeX-parse-self: t
% TeX-auto-save: t
% TeX-master: "fractal.tex"
% End:
