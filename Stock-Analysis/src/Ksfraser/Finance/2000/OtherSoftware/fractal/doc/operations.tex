%
% -----------------------------------------------------------------------------
%
% A license is hereby granted to reproduce this software source code and
% to create executable versions from this source code for personal,
% non-commercial use.  The copyright notice included with the software
% must be maintained in all copies produced.
%
% THIS PROGRAM IS PROVIDED "AS IS". THE AUTHOR PROVIDES NO WARRANTIES
% WHATSOEVER, EXPRESSED OR IMPLIED, INCLUDING WARRANTIES OF
% MERCHANTABILITY, TITLE, OR FITNESS FOR ANY PARTICULAR PURPOSE.  THE
% AUTHOR DOES NOT WARRANT THAT USE OF THIS PROGRAM DOES NOT INFRINGE THE
% INTELLECTUAL PROPERTY RIGHTS OF ANY THIRD PARTY IN ANY COUNTRY.
%
% Copyright (c) 1994-2006, John Conover, All Rights Reserved.
%
% Comments and/or bug reports should be addressed to:
%
%     john@email.johncon.com (John Conover)
%
% -----------------------------------------------------------------------------
%
% Revision: \RCSRevision \\
% Revision Time: \RCSTime UMT \\
% Revision Date: \RCSDate \\
% Revision Id: \RCSId \\
% Revision File: \RCSLog \\
\RCS $Revision: 0.0 $
\RCS $Date: 2006/01/20 04:38:13 $
\RCS $Id: operations.tex,v 0.0 2006/01/20 04:38:13 john Exp $
% $Log: operations.tex,v $
% Revision 0.0  2006/01/20 04:38:13  john
% Initial version
%
%
    \subsection{Fixed Increment Approximation for Operational Strategy}
        \label{\SETLABEL:OPS}.

        This section derives various values based on the ``average''
        of the normalized increments presented in
        Figure~\ref{\SETLABEL:TFA}. These values are an approximation
        to a, probably, complex process with a distribution shown in
        Figure~\ref{\SETLABEL:TF}. These values will be used in a
        fixed increment Brownian fractal analysis and simulation of
        the {\market}, and may, or may not, provide adequate accuracy
        for projections.

        \subidx{\market}{fiscal strategy}
        \subidx{\market}{Shannon probability}
        \subidx{strategy}{fiscal}
        \subidx{fiscal}{strategy}
        \subidx{Shannon}{probability}
        \subidx{probability}{Shannon}
        It should be noted that the analysis of fiscal strategy,
        presented in Section~\ref{\SETLABEL:FS}, is derived from the
        {\market} metrics and may, or may not, be maximally
        optimal. For the optimal fiscal strategy, which may be
        exploitable, see Section~\ref{\SETLABEL:MAXSHANNON}.

        \subidx{strategy}{exploitable}
        \subidx{exploitable}{strategy}
        \subidx{\market}{windows of opportunity}
        \idx{windows of opportunity}
        \subidx{decision}{obsolete}
        \subidx{obsolete}{decision}
        \subidx{decision}{timeliness}
        \subidx{timeliness}{decision}
        \subidx{rate of revenue returns}{forecast}
        \subidx{forecast}{rate of revenue returns}
        An additional exploitable strategy may be time itself.
        Equations~\ref{\SETLABEL:V},~\ref{\SETLABEL:R},
        and,~\ref{\SETLABEL:MA}, are, essentially, metrics on how fast
        a decision, which is based on information concerning the
        current status of the {\market}, becomes obsolete. Obviously,
        how long a decision is expected to remain relevant should be
        addressed as an operational necessity in strategic planning
        and project management. Figures~\ref{\SETLABEL:FN},
        and,~\ref{\SETLABEL:FF} compare methods of approximation of
        the ``forecastability'' of rate of revenue returns in the
        {\market} for the near term and far
        term~\cite[pp. 83-84]{Peters:CAOITCM}, respectively. As a
        general rule, caution must be exercised when making decisions
        that will span a time interval larger than the time interval
        where the ``forecastability'' of rate of revenue returns drops
        below 50\%. Beyond this time interval, the chances increase
        that the competitive and market forces will alter the market
        environment in a possibly detrimental unanticipated
        fashion. Obviously, there is significant advantage in
        ``timeliness'' of development, manufacturing, and distribution
        of products and services that are consistent with this
        temporal agenda. Automation of these processes, if executed
        consistently with this agenda, should be considered a
        competitive advantage.

        \subidx{strategy}{exploitable}
        \subidx{exploitable}{strategy}
        \subidx{rate of revenue returns}{forecast}
        \subidx{forecast}{rate of revenue returns}
        \idx{product life cycle}
        \idx{life cycle, product}
        In some sense, this temporal agenda defines the ``average''
        product or service life cycle in the {\market}. When the
        ``forecastability'' of rate of revenue returns drops below
        50\%, there is an even chance that the rate of revenue returns
        for the product or service will change in a detrimental
        fashion. If it is assumed that a product or service life cycle
        consists of a ramp up, a maintenence interval, and a ramp
        down, then, if all three life cycle intervals are equal, the
        product life cycle will be, approximately, three times the
        time interval where the ``forecastability'' of rate of revenue
        returns drops below 50\%. Although probably not an accurate
        prediction of product or service life cycle, the technique may
        be used as a conceptual approximation to the dynamics of
        ``market windows.\footnote{For example, consider the market
        for table salt. Since it has inelastic supply and demand
        curves, and is a necessary requirement for life, it would be
        expected that the Hurst coefficient would be very near
        unity---ignoring competitive pressures in the market. The
        predictability of the table salt market would, therefore, be
        expected to be relatively good, over time.}''  The conceptual
        approximation will probably predict a ``conservative'' or
        ``pessimistic'' value in relation to actual markets.

        \begin{figure}[ht]
            \begin{center}
                \begin{minipage}[t]{0.45\textwidth}
                    \epsfxsize=1.0\linewidth
                    \epsffile{\directory/datahurstlownear.eps}
                    \caption[{\market}, ``forecastability'' of near
                        term rate of revenue returns]{{\market},
                        ``forecastability'' of near term rate of
                        revenue returns. Although the error function
                        is the most accurate, for the near term,
                        $H^{t} = \thurstlow^{t}$ may be used as a
                        reliable metric of ``forecastability'' of the
                        rate of revenue returns.}
                    \label{\SETLABEL:FN}
                \end{minipage}
                \hfill
                \begin{minipage}[t]{0.45\textwidth}
                    \epsfxsize=1.0\linewidth
                    \epsffile{\directory/datahurstlowfar.eps}
                    \caption[{\market}, ``forecastability'' of far
                        term rate of revenue returns]{{\market},
                        ``forecastability'' of far term rate of
                        revenue returns. Although the error function
                        is the most accurate, for the far term,
                        $\frac{1}{\sqrt{t}}$ may be used as a reliable
                        metric of ``forecastability'' of the rate of
                        revenue returns.}
                    \label{\SETLABEL:FF}
                \end{minipage}
            \end{center}
        \end{figure}

        \idx{operations research}
        As an interesting interpretation of the data presented in
        Figure~\ref{\SETLABEL:FN}, there may be, perhaps, some
        applicability to such operational agendas as inventory
        control. Maintaining too little inventory, obviously, will
        create a situation where the organization can not exploit
        market expansion, and maintaining too much inventory,
        likewise, would over extend the company, creating unnecessary
        losses when the market contracts. The company should maintain
        inventory levels that do not exceed, from
        Equation~\ref{\SETLABEL:MA}, ${\thurstlow}^{n} = 0.5$
        {\timescale}s of operations. Since the optimal amount of
        inventory and, from Equation~\ref{\SETLABEL:V}, the variance
        of change in the rate of revenue returns in the future can be
        calculated, there may, perhaps, be some applicability to a
        forecasting methodology that can be incorporated into other
        areas of operations research, for example the linear algebras
        using simplex methodologies for optimization of manufacturing
        processes. Traditionally, these forecasts are made by the
        sales department, and are subject to various subjective
        biases.

% Local Variables:
% TeX-parse-self: t
% TeX-auto-save: t
% TeX-master: "fractal.tex"
% End:
