%
% -----------------------------------------------------------------------------
%
% A license is hereby granted to reproduce this software source code and
% to create executable versions from this source code for personal,
% non-commercial use.  The copyright notice included with the software
% must be maintained in all copies produced.
%
% THIS PROGRAM IS PROVIDED "AS IS". THE AUTHOR PROVIDES NO WARRANTIES
% WHATSOEVER, EXPRESSED OR IMPLIED, INCLUDING WARRANTIES OF
% MERCHANTABILITY, TITLE, OR FITNESS FOR ANY PARTICULAR PURPOSE.  THE
% AUTHOR DOES NOT WARRANT THAT USE OF THIS PROGRAM DOES NOT INFRINGE THE
% INTELLECTUAL PROPERTY RIGHTS OF ANY THIRD PARTY IN ANY COUNTRY.
%
% Copyright (c) 1994-2006, John Conover, All Rights Reserved.
%
% Comments and/or bug reports should be addressed to:
%
%     john@email.johncon.com (John Conover)
%
% -----------------------------------------------------------------------------
%
% Revision: \RCSRevision \\
% Revision Time: \RCSTime UMT \\
% Revision Date: \RCSDate \\
% Revision Id: \RCSId \\
% Revision File: \RCSLog \\
\RCS $Revision: 0.0 $
\RCS $Date: 2006/01/20 04:38:13 $
\RCS $Id: fiscal.tex,v 0.0 2006/01/20 04:38:13 john Exp $
% $Log: fiscal.tex,v $
% Revision 0.0  2006/01/20 04:38:13  john
% Initial version
%
%
    \subsection{Fixed Increment Approximation for Fiscal Strategy}
        \label{\SETLABEL:FS}

        \subidx{\market}{fiscal strategy}
        \subidx{markets}{analysis}
        \subidx{analysis}{markets}
        \subidx{strategy}{fiscal}
        \subidx{fiscal}{strategy}
        The data in this section is presented in tabular form in
        Section~\ref{\SETLABELREF:LR}. This section derives various
        values based on the ``average'' of the normalized increments
        presented in Figure~\ref{\SETLABEL:TFA}. These values are an
        approximation to a, probably, complex process with a
        distribution shown in Figure~\ref{\SETLABEL:TF}. These values
        will be used in a fixed increment Brownian fractal analysis
        and simulation of the {\market}, and may, or may not, provide
        adequate accuracy for projections.

        For an organization operating in the {\market}, the fiscal
        strategy, commensurate with the aggregate environment, can be
        derived as follows~\cite[pp. 128, pp
        151]{Schroeder},~\cite[pp. 450]{Reza},~\cite[pp. 270]{Pierce}:
        \vspace{0.15in}

        \subsubsection{Logarithmic Returns}
            \label{\SETLABEL:LR}

            \subidx{logarithmic}{returns}
            \subidx{returns}{logarithmic}
            \subidx{\market}{logarithmic returns}
            The logarithmic returns can be calculated by various
            means. Four will be presented here, for comparison.

            \subidx{programs}{tsnormal}
            \subidx{tsnormal}{program}
            \subidx{logarithmic}{returns}
            \subidx{returns}{logarithmic}
            The logarithmic returns, in bits, $bits$, as computed from
            the mean, by the program {\it tsnormal}\/, which is
            described in Chapter~\ref{programs}, and is presented in
            Figure~\ref{\SETLABEL:TF}, and Equation~\ref{abits} from
            Section~\ref{ereturns} in Chapter~\ref{general}:

            \begin{equation}
                bits = \frac{\ln \left({\datafractionmean} + 1\right)}{\ln \left(2\right)} = \datafractionmeanbits
            \end{equation}

            \subidx{programs}{tslsq}
            \subidx{tslsq}{program}
            \subidx{logarithmic}{returns}
            \subidx{returns}{logarithmic}
            \noindent By comparison, the logarithmic returns, in bits,
            $bits$, as computed from the constant in the least squares
            approximation, using the program {\it tslsq}\/, which is briefly
            described in Chapter~\ref{programs}, as presented in
            Figure~\ref{\SETLABEL:TF}, and Equation~\ref{abits} from
            Section~\ref{ereturns} in Chapter~\ref{general}:

            \begin{equation}
                bits = \frac{\ln \left({\datafractionconstant} + 1\right)}{\ln \left(2\right)} = \datafractionconstantbits
            \end{equation}

            Note that if the mean is not constant in
            Figure~\ref{\SETLABEL:TF}, this method will not provide
            accurate results.

            \subidx{programs}{tslsq}
            \subidx{tslsq}{program}
            \subidx{logarithmic}{returns}
            \subidx{returns}{logarithmic}
            \noindent And by yet another comparison, using the program
            {\it tslsq}\/, which is briefly described in
            Chapter~\ref{programs}, with the -e -p options, to provide
            a formula for the least squares exponential fit to the
            time series data set presented in
            Figure~\ref{\SETLABEL:TS}:

            \begin{equation}
                bits = {\datatslsqepbits}
            \end{equation}

            \subidx{programs}{tslogreturns}
            \subidx{tslogreturns}{program}
            \subidx{logarithmic}{returns}
            \subidx{returns}{logarithmic}
            \noindent And finally, by comparison, from the
            {\it tslogreturns}\/ program, which is briefly described
            in Chapter~\ref{programs}, with the -p option, to provide
            a formula for the logarithmic returns of the time series
            data set presented in Figure~\ref{\SETLABEL:TS}:

            \begin{equation}
                bits = {\logreturns}
            \end{equation}

        \subsubsection{Calculation of Shannon Probability}
            \label{\SETLABEL:SP}

            \subidx{\market}{Shannon probability}
            Ideally, all of the values presented in
            Section~\ref{\SETLABEL:LR} would be equal. Using the
            logarithmic returns provided by the {\it tslogreturns}\/
            program, to be consistent
            with~\cite[pp. 81]{Peters:CAOITCM}

            \subidx{programs}{tslogreturns}
            \subidx{tslogreturns}{program}
            \begin{equation}
                2^{{\logreturns}t}
            \end{equation}

            \noindent therefore:
            \begin{equation}
                C\left(p\right) = {\logreturns}
            \end{equation}
            \subidx{programs}{tsshannon}
            \subidx{tsshannon}{program}
            \subidx{Shannon}{probability}
            \subidx{probability}{Shannon}
            \noindent and, {\it tsshannon}\/ {\logreturns} gives:
            \begin{equation}
                \label{\SETLABEL:F0}
                C\left({\shannonlogreturns}\right) = {\logreturns}
            \end{equation}
            \noindent therefore:
            \begin{eqnarray}
                2^{C\left({\shannonlogreturns}\right)} & = & 2^{\logreturns}\\
                                                       & = & {\twologreturns}\\
                                                       & = & {\twologreturnshundred}\%
            \end{eqnarray}
            \noindent and:
            \begin{eqnarray}
                2p - 1 & = & \left(2 \cdot {\shannonlogreturns}\right) - 1\\
                       & = & {\twopone}\\
                       \label{\SETLABEL:F1}
                       & = & {\twoponehundred}\%
            \end{eqnarray}

            \subidx{\market}{fiscal strategy}
            \subidx{markets}{analysis}
            \subidx{analysis}{markets}
            \subidx{strategy}{fiscal}
            \subidx{fiscal}{strategy}
            \subidx{\market}{fiscal strategy}
            \subidx{\market}{growth rate}
            Presuming the simplified assumptions outlined in
            Section~\ref{assumptions}, the ``typical'' organization
            operating in the {\market} executes a long term fiscal
            strategy, commensurate with the aggregate environment,
            that is to invest, every {\timescale}, in sufficient
            additional resources and infrastructure, to increase the
            manufacturing of goods and services by {\twoponehundred}\%
            of its rate of revenue returns, (per {\timescale}.) As a
            conceptual model, the remaining {\hundredtwoponehundred}\%
            will be held in ``reserve'' with a
            {\shannonlogreturnshundred}\% chance of making twice the
            {\twoponehundred}\% back, (and a
            {\hundredshannonlogreturnshundred}\% chance of making
            0.0,) in one {\timescale}, on the average, for an average
            growth in its rate of revenue returns, (per {\timescale},)
            of {\twologreturnshundred}\%, or a doubling of its rate of
            revenue returns, (per {\timescale},) in
            {\oneoverlogreturns} {\timescale}s.

        \subsubsection{Example Fixed Increment Approximation Fiscal Strategies}

            \subidx{\market}{fiscal strategy}
            \subidx{markets}{analysis}
            \subidx{analysis}{markets}
            \subidx{strategy}{fiscal}
            \subidx{fiscal}{strategy}
            \subidx{\market}{fiscal strategy}
            \subidx{\market}{growth rate}
            \subidx{\market}{management metric}
            \idx{management metric}
            A possible metric on the effectiveness of long term fiscal
            management could possibly be that if an investment of
            {\twoponehundred}\% per {\timescale} of the rate of
            revenue returns, (per {\timescale},) is made in resources
            and infrastructure, then the rate of revenue returns would
            be expected to increase by {\twologreturnshundred}\%, per
            {\timescale}, on average.

            Note that the metrics presented in this section are
            representative of the {\market} as an aggregate whole, and
            may or may not be accurate representations for any
            particular participant in the environment. Of interest to
            the participants in the environment would be a similar
            analysis of each product or service rendered in the
            marketplace.

            \subidx{\market}{fiscal strategy}
            \subidx{markets}{analysis}
            \subidx{analysis}{markets}
            \subidx{strategy}{fiscal}
            \subidx{fiscal}{strategy}
            \subidx{\market}{fiscal strategy}
            As a simple illustrative example, a company operating in
            this environment might obtain a credit line from a bank
            that is equal to {\twoponehundred}\% of its rate of
            revenue returns, (per {\timescale},) to finance additional
            operations. In this simple scenario, the company would use
            its revenue base as collateral for the loan. Some
            {\timescale}s, depending on the {\market}'s environment,
            the company's rate of revenue returns exceeds what was
            borrowed from the bank, and the loan is repaid in
            full. Other {\timescale}s, the company must default, and
            the bank seizes a portion of the company's revenue base to
            pay the delinquent loan. However, on the average, the
            company will expand its rate of revenue returns at
            {\twologreturnshundred}\% per {\timescale}.

            \subidx{\market}{fiscal strategy}
            \subidx{markets}{analysis}
            \subidx{analysis}{markets}
            \subidx{strategy}{fiscal}
            \subidx{fiscal}{strategy}
            \subidx{\market}{fiscal strategy}
            As another simple example, a company re-invests
            {\twoponehundred}\% of its rate of revenue returns, (per
            {\timescale},) in development, marketing, sales, and
            distribution of new products.  Although some products will
            be successful and the return on the investment will exceed
            the {\twoponehundred}\% per {\timescale} investment,
            others will not. However, on the average, the company will
            expand it gross rate of revenue returns at
            {\twologreturnshundred}\% per {\timescale}.

            \subidx{\market}{fiscal strategy}
            \subidx{markets}{analysis}
            \subidx{analysis}{markets}
            \subidx{strategy}{fiscal}
            \subidx{fiscal}{strategy}
            \subidx{\market}{fiscal strategy}
            \subidx{\market}{product portfolio}
            \subidx{\market}{product diversity}
            \subidx{\market}{product mix}
            \subidx{\market}{optimum number of products}
            \idx{product portfolio}
            \idx{product diversity}
            \idx{optimum number of products}
            \idx{product mix}

            As an example of ``product portfolio'' management, suppose
            a company re-invests {\twoponehundred}\% of its rate of
            revenue returns, (per {\timescale},) in development,
            marketing, sales, and distribution of new products.
            Further suppose that the company has two products, and a
            fractal analysis of the individual product rate of revenue
            return time series indicates that one product has a
            Shannon probability of 0.65, and the other has a Shannon
            probability of 0.55. Then the percentage of re-investment
            in the first product would be $(2 \cdot 0.65 - 1) \cdot
            {\twoponehundred}$, percent of the rate of revenue
            returns, and $(2 \cdot 0.55 - 1) \cdot {\twoponehundred}$
            percent for the second product, implying that the company
            should diversify its product line\footnote{The astute
            reader would note that the linear addition was used to add
            the contribution to development of each product. This is a
            ``near term'' interpretation. Actually, in general, the
            method used should be a root mean square process,
            dependent on the Hurst Coefficient, $H$, where
            $P_{total}^H = P_1^H + P_2^H + \cdots$, where $P_n$ is the
            contribution to each individual product. For a Brownian
            motion, or random walk type of fractal the Hurst
            Coefficient is a function of time into the future. For the
            ``near term,'' the Hurst coefficient is very near unity,
            meaning the summation process is linear. For the ``long
            term,'' $H \approx 0.5$, or a standard root mean square
            summation process should be used. If $H$ is $0.5$ then the
            market is termed a Brownian motion, or random walk
            process. If it is larger than 0.5, it is termed fractional
            Brownian motion process. For a random walk process, ``near
            term'' and ``far term'' are quantitatively differentiated
            on the Hurst Coefficient graph where $1 - \ln (t) = 0.5
            \cdot \ln (t)$, or when $\ln (t) = 2$, or $t =
            7.389\ldots$ See~\cite[pp. 67, 83-84]{Peters:CAOITCM}
            and~\cite[pp. 129, 159]{Schroeder} for particulars on the
            implications of the Hurst Coefficient and root mean square
            summation issues.}.  Note that this is a ``bet hedging''
            metric methodology, and assumes that the products have
            uncorrelated revenue return rates. If this re-investment
            methodology is not feasible, perhaps for strategic
            financial reasons, then the re-investment in both products
            should total the ${\twoponehundred}$\%, and the investment
            in each product should be made at a ratio of $\frac{(2
            \cdot 0.65 - 1)}{(2 \cdot 0.55 - 1)} = 3 : 1$,
            respectively. Note that this ``bet hedging'' can be used
            to define the optimal number of products that can be
            supported on the rate of revenue returns. If it assumed
            that all products are ``typical'' for the {\market}, as a
            standard bench mark, then the optimal number will be
            $\frac{1}{{\twopone}}$. Note that this is a
            ``theoretical'' value, since not all products are
            ``typical,'' and there may be strategic reasons, for
            example product leveraging, that may increase the number
            of products above the optimum. However, most of the
            revenue should come from the optimal number of products,
            since having more products will decrease the amount of the
            potential investment in each product, and having less than
            the optimum number of products will increase the risk that
            many of the products could suffer a ``down market''
            concurrently, impacting the rate of revenue returns.  As
            another interesting interpretation of the optimal
            ``hedging of bets,'' in product portfolio strategy, and
            considering the graph of the normalized increments
            presented in Figure~\ref{\SETLABEL:TF}, if the
            organization is running optimally, then these products
            will generate, at least in principle, one standard
            deviation, approximately $0.8413 = 84.13$\% of the future
            growth in rate of revenue returns. Naturally, these are
            approximations, and the values are an approximation to a,
            probably, complex process, and appropriate scrutiny should
            be exercised before making specific projections.  As yet
            another example of ``product portfolio'' management,
            consider the issue of product mix. In this interpretation,
            {\twoponehundred}\% of the product manufactured should be
            ``proprietary,'' while the rest is ``industry standard.''
            As yet another possibility, {\twoponehundred}\% of the
            product manufactured should be predatory into new markets,
            and the remainder in markets that are ``traditional'' for
            the company.

% Local Variables:
% TeX-parse-self: t
% TeX-auto-save: t
% TeX-master: "fractal.tex"
% End:
