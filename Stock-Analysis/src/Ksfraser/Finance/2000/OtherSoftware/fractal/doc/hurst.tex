%
% -----------------------------------------------------------------------------
%
% A license is hereby granted to reproduce this software source code and
% to create executable versions from this source code for personal,
% non-commercial use.  The copyright notice included with the software
% must be maintained in all copies produced.
%
% THIS PROGRAM IS PROVIDED "AS IS". THE AUTHOR PROVIDES NO WARRANTIES
% WHATSOEVER, EXPRESSED OR IMPLIED, INCLUDING WARRANTIES OF
% MERCHANTABILITY, TITLE, OR FITNESS FOR ANY PARTICULAR PURPOSE.  THE
% AUTHOR DOES NOT WARRANT THAT USE OF THIS PROGRAM DOES NOT INFRINGE THE
% INTELLECTUAL PROPERTY RIGHTS OF ANY THIRD PARTY IN ANY COUNTRY.
%
% Copyright (c) 1994-2006, John Conover, All Rights Reserved.
%
% Comments and/or bug reports should be addressed to:
%
%     john@email.johncon.com (John Conover)
%
% -----------------------------------------------------------------------------
%
% Revision: \RCSRevision \\
% Revision Time: \RCSTime UMT \\
% Revision Date: \RCSDate \\
% Revision Id: \RCSId \\
% Revision File: \RCSLog \\
\RCS $Revision: 0.0 $
\RCS $Date: 2006/01/20 04:38:13 $
\RCS $Id: hurst.tex,v 0.0 2006/01/20 04:38:13 john Exp $
% $Log: hurst.tex,v $
% Revision 0.0  2006/01/20 04:38:13  john
% Initial version
%
%
    \subsection{Hurst Coefficient Analysis}
        \label{\SETLABEL:H}

        \subidx{\market}{Hurst coefficient analysis}
        \subidx{Hurst coefficient}{analysis}
        \subidx{increments}{normalized}
        \subidx{normalized}{increments}
        \subidx{programs}{tshurst}
        \subidx{tshurst}{program}
        The data in this section is presented in tabular form in
        Section~\ref{\SETLABELREF:HCHP}. Figure~\ref{\SETLABEL:HC} is
        a graph of the Hurst coefficient data time series data shown
        in Figure~\ref{\SETLABEL:TS}. The slope of the graph is the
        Hurst coefficient.  The data for this figure was produced by
        the program {\it tshurst}\/, which is described briefly in
        Appendix~\ref{programs}.

        \subidx{\market}{H parameter analysis}
        \subidx{H parameter}{analysis}
        \subidx{programs}{tshcalc}
        \subidx{tshcalc}{program}
        Figure~\ref{\SETLABEL:HP} is a graph of the H parameter data
        for the normalized increments of the time series data shown in
        Figure~\ref{\SETLABEL:TF}. The data for this figure was
        produced by the program {\it tshcalc}\/, which is described
        briefly in Appendix~\ref{programs}.

        \begin{figure}[ht]
            \begin{center}
                \begin{minipage}[t]{0.45\textwidth}
                    \epsfxsize=1.0\linewidth
                    \epsffile{\directory/data.tshurst.eps}
                    \caption[{\market}, Hurst coefficient data]{{\market},
                        Hurst coefficient data for the normalized
                        increments of the time series data shown in
                        Figure~\ref{\SETLABEL:TF}.  The slope of the graph
                        is the Hurst coefficient.}
                    \label{\SETLABEL:HC}
                \end{minipage}
                \hfill
                \begin{minipage}[t]{0.45\textwidth}
                    \epsfxsize=1.0\linewidth
                    \epsffile{\directory/data.tshcalc.eps}
                    \caption[{\market}, H parameter data]{{\market}, H
                        parameter data for the normalized increments of
                        the time series data shown in
                        Figure~\ref{\SETLABEL:TF} The slope of the graph
                        is the H parameter.}
                    \label{\SETLABEL:HP}
                \end{minipage}
            \end{center}
        \end{figure}

        \subidx{revenue}{See, rate of revenue returns}
        \subidx{returns}{See, rate of revenue returns}
        \subidx{\market}{revenues}
        \subidx{Hurst coefficient}{analysis}
        \subidx{\market}{Hurst coefficient analysis}
        \subidx{\market}{rate of change}
        \subidx{\market}{windows of opportunity}
        \subidx{rate of revenue returns}{forecast}
        \subidx{forecast}{rate of revenue returns}
        \idx{windows of opportunity}
        \subidx{programs}{tslsq}
        \subidx{tslsq}{program}

        The approximately linear slope of the graph in
        Figure~\ref{\SETLABEL:HC} implies that the variance of the
        rate of revenue returns, (per {\timescale},) in the {\market},
        $V(t_2 - t_1)$, over a period of time is proportional to the
        period of time raised to twice the Hurst
        coefficient~\cite[pp. 180]{Feder},~\cite[pp. 246]{Crownover}.
        This seems to be a quantitative statement concerning how fast,
        and to what degree, the rate of revenue returns' state of
        affairs can change over a period of time.  An additional
        implication, for Hurst coefficients sufficiently close to 0.5,
        is that the probability of the state of affairs repeating
        sometime in the future goes down with increasing
        time\footnote{It can be shown that the number of expected
        market ``high'' and ``low'' transitions, $N$, scales with the
        square root of time, or $N \propto \sqrt {t}$, meaning that
        the cumulative distribution of the probability, $P$, of the
        duration of a market's ``high'' or ``low'' exceeding a given
        time interval, $t$, is proportional to the reciprocal of the
        square root of the time interval, $P \propto 1 / \sqrt {t}$,
        (or, conversely, that the probability of the duration of a
        market's ``high'' or ``low'' exceeding a given time interval
        is proportional to the reciprocal of the time interval raised
        to the power $3 / 2$, ie., $P \propto 1 / t^{3 /
        2}$,~\cite[pp. 153]{Schroeder}. What this means is that a
        histogram of the ``zero free'' run-lengths of a market being
        ``high'' or ``low,'' over a long time, would have a $1 / t^{3
        / 2}$ characteristic.)}, $t$, $p(t) = erf (1/\sqrt{2t})$ which
        is approximately $1/\sqrt{t}$ for $t \gg
        1$~\cite[pp. 160]{Schroeder}. Figures~\ref{\SETLABEL:FN},
        and,~\ref{\SETLABEL:FF} compare methods of approximation of
        the ``forecastability'' of the rate of revenue returns in the
        {\market} for the near term and far term,
        respectively~\cite[pp. 83-84]{Peters:CAOITCM}\footnote{The
        author is not comfortable with Peters' interpretation. For
        example, if the algorithm explained
        in~\cite[pp. 82]{Peters:CAOITCM} is used on ``white noise''
        which, by definition, never has any correlations, the short
        term Hurst coefficient, and thus the ``forecastability,'' is
        still near unity---a bit of an enigma. This can be verified
        with the {\it tswhite}\/ and {\it tshurst}\/ programs, which
        are briefly described in Appendix~\ref{programs}.}.  This
        seems to be a quantitative statement concerning ``windows of
        opportunity'' in the rate of revenue returns, (per
        {\timescale}.)  The program {\it tslsq}\/ was used on the
        Hurst coefficient data, presented in
        Figure~\ref{\SETLABEL:HC}, to provide a least squares
        approximation to the Hurst coefficient. The superimposed least
        squares approximation with on original Hurst coefficient data
        is presented.  The time series data has a Hurst coefficient of
        {\thurstlow}, so that:

        \subidx{\market}{Hurst coefficient analysis}
        \begin{eqnarray}
            V\left(t_2 - t_1\right) & \propto & \left(t_2 - t_1\right)^{2 \cdot H}\\
            V\left(t_2 - t_1\right) & \propto & \left(t_2 - t_1\right)^{2 \cdot {\thurstlow}}\\
                                    & \propto & \left(t_2 - t_1\right)^{\thurstlowtwo}
            \label{\SETLABEL:V}
        \end{eqnarray}

        \subidx{fractional}{Brownian motion}
        \subidx{Brownian motion}{fractional}
        \idx{fractal}
        \noindent where $V(t_2 - t_1)$ is the variance of the
        increments of the rate of revenue returns, (per {\timescale},)
        over the time interval $t_2 -
        t_1$,~\cite[pp. 177]{Feder},~\cite[pp. 494]{Peitgen}. If $H >
        \frac{1}{2}$, then the time series is termed as being
        characterized by ``fractional Brownian
        motion~\cite[pp. 170]{Feder}.''

        \subidx{rate of revenue returns}{predictability}
        \subidx{rate of revenue returns}{forecastability}
        \subidx{rate of revenue returns}{consistency}
        \subidx{predictability}{rate of revenue returns}
        \subidx{forecastability}{rate of revenue returns}
        \subidx{consistency}{rate of revenue returns}
        \subidx{\market}{rate of revenue returns, predictability}
        \subidx{\market}{rate of revenue returns, forecastability}
        \subidx{\market}{rate of revenue returns, consistency}
        \subidx{Hurst coefficient}{analysis}
        \subidx{\market}{Hurst coefficient analysis}
        \subidx{\market}{rate of change}

        In some sense, the Hurst coefficient is a quantitative
        expression of the ``forecastability'' of the future based on
        the past\footnote{Actually, in general, when summing fractal
        entities, the method used should be a root mean square
        process, dependent on the Hurst Coefficient, $H$, where
        $P_{total}^H = P_1^H + P_2^H + \cdots$, where $P_n$ is the
        fractal entities. For a Brownian motion, or random walk type
        of fractal the Hurst Coefficient is a function of time into
        the future. For the ``near term,'' the Hurst coefficient is
        very near unity, meaning the summation process is linear. For
        the ``long term,'' $H \approx 0.5$, or a standard root mean
        square summation process should be used. If $H$ is $0.5$ then
        the market is termed a Brownian motion, or random walk
        process. If it is larger than 0.5, it is termed fractional
        Brownian motion process. For a random walk process, ``near
        term'' and ``far term'' are quantitatively differentiated on
        the Hurst Coefficient graph where $1 - \ln (t) = 0.5 \cdot \ln
        (t)$, or when $\ln (t) = 2$, or $t = 7.389\ldots$ See
        Section~\ref{\SETLABEL:FS} for the particulars on using Hurst
        Coefficient to sum fractal process' for the {\market}. See
        also~\cite[pp. 67, 83-84]{Peters:CAOITCM} and~\cite[pp. 129,
        159]{Schroeder} for particulars on the implications of the
        Hurst Coefficient and root mean square summation issues.}.  A
        Hurst coefficient of {\thurstlow}, (for the near future, and
        {\thurstall} for the distant future.) implies that the
        likelihood of the rate of revenue returns, (per {\timescale},)
        for any two consecutive {\timescale}s being the same is
        {\thurstlowhundred}\%~\cite[pp. 66]{Peters:CAOITCM} for the
        near future, and {\thurstall} for the distant
        future. Likewise, there is a {\thurstlowhundred}\% chance of
        the rate of revenue returns, (per {\timescale},) movements
        being the same in consecutive time periods---ie., if, in a
        given {\timescale}, the rate of revenue returns, (per
        {\timescale},) is increasing, there is a {\thurstlowhundred}\%
        that the rate of revenue returns, (per {\timescale},) will
        increase in the following period, also. In some sense, this is
        a quantitative statement on how ``predictable,'' or
        ``forecastable'' the rate of revenue returns, (per
        {\timescale},) for the {\market} are over time, since the
        probability of having $n$ many consecutive {\timescale}s of
        the same agenda is $H^n$ where $H$ is the Hurst coefficient,
        or, letting the short term probability of having $n$ many
        {\timescale}s of the same market agenda, $p_a$, is:

        \begin{eqnarray}
            p_a\left(n\right) & = & H^{n}\\
                              & = & {\thurstlow}^{n}
            \label{\SETLABEL:MA}
        \end{eqnarray}

        \subidx{rate of revenue returns}{predictability}
        \subidx{rate of revenue returns}{forecastability}
        \subidx{rate of revenue returns}{consistency}
        \subidx{predictability}{rate of revenue returns}
        \subidx{forecastability}{rate of revenue returns}
        \subidx{consistency}{rate of revenue returns}
        As an interesting interpretation of the normalized increments
        of the time series data presented in
        Figure~\ref{\SETLABEL:TF}, if the vertical axis is multiplied
        by 100, to convert to percent, then the graph represents the
        error, in percent, that would be made by forecasting, month by
        month, that the next {\timescale}'s rate of revenue returns
        would be the same as the current {\timescale}'s revenue
        rate. Interestingly, it is $\datafractionmean \cdot 100$
        percent, on the average, with a standard deviation of
        $\datafractionstddev \cdot 100$ percent, and a root mean
        square error value of $\datafractionrms \cdot 100$
        percent---small values for such a simple forecasting
        mechanism.

        \subidx{\market}{rate of revenue returns, range}
        \subidx{Hurst coefficient}{analysis}
        \subidx{\market}{Hurst coefficient analysis}
        \subidx{\market}{rate of change}

        This is, essentially, a statement of the range of values, in
        the increments of the rate of revenue returns, (per
        {\timescale},) that is to be expected over the time interval,
        $t_2 - t_1$,
        $R_v$,~\cite[pp. 178]{Feder},~\cite[pp. 172]{Cambel}:

        \begin{eqnarray}
            R_v\left(t_2 - t_1\right) & \propto & \left(t_2 - t_1\right)^{H}\\
                                      & \propto & \left(t_2 - t_1\right)^{\thurstlow}
            \label{\SETLABEL:R}
        \end{eqnarray}

        \subidx{\market}{rate of revenue returns, range}
        \subidx{Hurst coefficient}{analysis}
        \subidx{\market}{Hurst coefficient analysis}
        \subidx{\market}{rate of change}
        \subidx{Markov}{statistics}
        \subidx{statistics}{Markov}
        \noindent where $R$ is the range of values in the increments
        of the rate of revenue returns, (per {\timescale}.) A Hurst
        coefficient, $H$, that is much larger than $\frac{1}{2}$, (but
        less than 1,) implies a strongly non-Gaussian distribution in
        the increments of the rate of revenue returns, (per
        {\timescale},)~\cite[pp. 152, 194]{Feder}, and a Hurst
        coefficient near $\frac{1}{2}$ implies that the increments of
        the rate of revenue returns, (per {\timescale}) is
        characteristic of an independent
        process~\cite[pp. 195]{Feder}. Extreme caution should be
        exercised in using Markov statistics in any analysis where the
        Hurst coefficient is not
        $\frac{1}{2}$,~\cite[pp. 124]{Crownover},~\cite[pp. 106]{Peters:CAOITCM}.


        As a useful approximation, if $H$, is approximately
        $\frac{1}{2}$, Equation~\ref{\SETLABEL:R} reduces
        to,~\cite[pp. 129]{Schroeder}:

        \begin{eqnarray}
            R\left(t_2 - t_1\right) & \propto & (t_2 - t_1)^{\frac{1}{2}}\\
                                    & \propto & \sqrt{\left(t_2 - t_1\right)}
        \end{eqnarray}

        \subidx{\market}{rate of revenue returns, range}
        \subidx{\market}{rate of revenue returns, increase and decrease}
        \subidx{Hurst coefficient}{analysis}
        \subidx{\market}{Hurst coefficient analysis}
        \subidx{\market}{rate of change}
        \subidx{Markov}{statistics}
        \subidx{statistics}{Markov}

        In the case where the Hurst coefficient, $H$, is
        $\frac{1}{2}$, the range of values in the increments of the
        rate of revenue returns, (per {\timescale},) divided by the
        standard deviation of these values, $S$, can be anticipated to
        increase over time according to the following
        relation,~\cite[pp. 154]{Feder},~\cite[pp. 129]{Schroeder}:

        \begin{equation}
            \frac{R\left(t_2 - t_1\right)}{S} \propto \left(t_2 - t_1\right)^{\frac{1}{2}}
        \end{equation}

        \subidx{\market}{rate of revenue returns, range}
        \subidx{\market}{rate of revenue returns, increase and decrease}
        \subidx{Hurst coefficient}{analysis}
        \subidx{\market}{Hurst coefficient analysis}
        \subidx{\market}{rate of change}
        \noindent which is a useful conceptual approximation, since it
        involves only the square root function---if the range and the
        standard deviation of the increments of the rate of revenue
        returns, (per {\timescale},) are known, (and $H \approx
        \frac{1}{2}$,) then the expected change in $\frac{R}{S}$, will
        increase with the square root of time\footnote{To be precise,
        it is actually asymptotically proportional to
        $\tau^{\frac{1}{2}}$}.

        Another useful approximation when rescaling processes that are
        characterize by Brownian motion, (ie., when $H \approx
        \frac{1}{2}$,) is that:

        \begin{eqnarray}
            X\left(t\right) & \propto & \frac{X\left(rt\right)}{r^{H}}\\
                            & \propto & \frac{X\left(rt\right)}{r^{\thurstlow}}
        \end{eqnarray}

        \idx{Brownian motion}
        \idx{fractal}
        Where $X(t)$ is the process characterized by Brownian motion,
        and $r$ is a scaling factor,~\cite[pp. 494]{Peitgen}.

        \subidx{programs}{tslsq}
        \subidx{tslsq}{program}
        The program {\it tslsq}\/ was used on the H parameter data,
        presented in Figure~\ref{\SETLABEL:HP}, to provide a least
        squares approximation to the H parameter for the
        {\market}. The superimposed least squares approximation on the
        original H parameter data is presented.  By contrast, the H
        parameter, as derived by the methodology outlined
        in~\cite[pp. 249]{Crownover}, is {\thcalclow} for the near
        future, and {\thcalcall} for the distant future.

        \subidx{\market}{Hurst coefficient analysis}
        \subidx{Hurst coefficient}{analysis}
        \subidx{increments}{normalized}
        \subidx{normalized}{increments}
        \subidx{programs}{tshurst}
        \subidx{tshurst}{program}
        \subidx{\market}{H parameter analysis}
        \subidx{H parameter}{analysis}
        \subidx{programs}{tshcalc}
        \subidx{tshcalc}{program}
        Figures~\ref{\SETLABEL:HC} and~\ref{\SETLABEL:HP} represent
        Hurst coefficient and H parameter data that are derived from
        the normalized increments, shown in
        Figure~\ref{\SETLABEL:TF}. In this case, the data is
        considered a normalized derivative of the time series data
        presented in Figure~\ref{\SETLABEL:TF}, instead of a
        cumulative sum.  The program, {\it tshurst}\/, is described
        briefly in appendix~\ref{programs}, and the data for
        figures~\ref{\SETLABEL:THC} and~\ref{\SETLABEL:THP} was made
        using the -d option.

        \begin{figure}[ht]
            \begin{center}
                \begin{minipage}[t]{0.45\textwidth}
                    \epsfxsize=1.0\linewidth
                    \epsffile{\directory/data.tsfraction.tshurst-d.eps}
                    \caption[{\market}, traditional Hurst coefficient
                        data]{{\market}, traditional Hurst coefficient
                        data for the time series data shown in
                        Figure~\ref{\SETLABEL:TS}.  The slope of the
                        graph is the Hurst coefficient, and is
                        {\hurstlow} for the near term, and
                        {\hurstall} for the far term.}
                    \label{\SETLABEL:THC}
                \end{minipage}
                \hfill
                \begin{minipage}[t]{0.45\textwidth}
                    \epsfxsize=1.0\linewidth
                    \epsffile{\directory/data.tsfraction.tshcalc-d.eps}
                    \caption[{\market}, traditional H parameter
                        data]{{\market}, traditional H parameter data
                        for the time series data shown in
                        Figure~\ref{\SETLABEL:TS} The slope of the
                        graph is the H parameter, and is {\hcalclow}
                        for the near term, and {\hcalcall} for the
                        far term.}
                    \label{\SETLABEL:THP}
                \end{minipage}
            \end{center}
        \end{figure}

% Local Variables:
% TeX-parse-self: t
% TeX-auto-save: t
% TeX-master: "fractal.tex"
% End:
