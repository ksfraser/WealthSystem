%
% -----------------------------------------------------------------------------
%
% A license is hereby granted to reproduce this software source code and
% to create executable versions from this source code for personal,
% non-commercial use.  The copyright notice included with the software
% must be maintained in all copies produced.
%
% THIS PROGRAM IS PROVIDED "AS IS". THE AUTHOR PROVIDES NO WARRANTIES
% WHATSOEVER, EXPRESSED OR IMPLIED, INCLUDING WARRANTIES OF
% MERCHANTABILITY, TITLE, OR FITNESS FOR ANY PARTICULAR PURPOSE.  THE
% AUTHOR DOES NOT WARRANT THAT USE OF THIS PROGRAM DOES NOT INFRINGE THE
% INTELLECTUAL PROPERTY RIGHTS OF ANY THIRD PARTY IN ANY COUNTRY.
%
% Copyright (c) 1994-2006, John Conover, All Rights Reserved.
%
% Comments and/or bug reports should be addressed to:
%
%     john@email.johncon.com (John Conover)
%
% -----------------------------------------------------------------------------
%
% Revision: \RCSRevision \\
% Revision Time: \RCSTime UMT \\
% Revision Date: \RCSDate \\
% Revision Id: \RCSId \\
% Revision File: \RCSLog \\
\RCS $Revision: 0.0 $
\RCS $Date: 2006/01/20 04:38:13 $
\RCS $Id: abstract.tex,v 0.0 2006/01/20 04:38:13 john Exp $
% $Log: abstract.tex,v $
% Revision 0.0  2006/01/20 04:38:13  john
% Initial version
%
%
\begin{abstract}

    This manuscript presents some personal notes on a fractal analysis
    of various market segments in the North American electronics
    industry. Although a very simple model is presented to analyze the
    dynamics of the industry's markets, there is, probably, reasonable
    evidence presented that the market segments do, indeed, have
    fractal characteristics. Although the model presented does not
    offer significant advantages over other quantitative
    methodologies, the qualitative analysis, without having access to
    any other data other than the time series of the market's rate of
    revenue returns, would seem to predict that:

    \begin{itemize}

        \item Research, development and infrastructural investments
        seem reasonable at about 12 to 20 percent of the rate of
        revenue returns for the market segments analyzed. This seems
        consistent with the industry.

        \item Venture success rates at 60 months seems reasonable at
        about 1 in 12, which is commensurate with the industry.

        \item Project success rates, of 8 month duration, are about 1
        in 3, which is consistent with numbers from the Application
        Specific Integrated Circuit business, which could be
        considered as ``representative.''

        \item The ``80/20 rule'' that 80\% of an organization's
        revenue comes from only a few, 3 was shown to be typical,
        products is really, probably, 84.13\%, or one standard
        deviation---which is consistent through the industry.

        \item The ``80/20 rule'' that 80\% of an organization's
        products should be ``industry standard,'' and the remainder
        ``proprietary'' is probably, one standard deviation, or
        84.13\%.

        \item Although the prediction of product life cycle will be
        shown to be ``pessimistic,'' it is none, the less, depending
        on the reader's point of view, reasonable, and was fairly
        consistent with industry averages.

        \item The inventory control dynamics presented seem to be
        consistent with the markets analyzed.

        \item The failure rate of Fortune 500 Companies seems
        consistent with predicted failure rate of organizations in the
        markets analyzed, although the rate of failure will be shown
        to be ``optimistic,'' when related to re-investment strategy.

        \item The calculated number of companies participating in the
        markets analyzed is reasonably close to the industry numbers,
        and there is inferential evidence that they are operating
        optimally---at least in the entropic sense as defined in
        Chapter~\ref{general}---which seems consistent with the
        economic theory that the companies that operate the most
        optimally or efficiently will, eventually, dominate the
        market. (The calculated number of companies participating in
        the various markets varied between 6 and 28, with an average
        of 10, and with Shannon probabilities for the individual
        company's market time series varying between 0.54 and 0.6,
        with an average of 0.57, which, interestingly, is close,
        within approximately 5\%, to the Shannon probability for the
        various company's stock price time series.)

        \item The variance in the aggregate market time series is
        smaller than the the variance of the time series for any
        company participating in the market, which is consistent with
        the industries analyzed.

        \item It would seem that there is some supporting evidence
        that optimizing a company's fiscal strategy to achieve maximum
        market growth and optimizing a company's fiscal strategy to
        optimize capital growth may be mutually exclusive, which has,
        traditionally, been the case in the industries
        analyzed. Additionally, it would seem that, at least in the
        markets analyzed, the fiscal strategies deployed would tend to
        be optimizing market growth, which seems consistent with
        author's experience in these industries.

    \end{itemize}

        Additionally, it would seem to be shown that visibility into
        the future, regarding rate of revenue returns, was only a few
        months, at best. This would seem to be in disagreement with
        the prevailing concept that ``strategic planning'' should be
        ``long term.'' An interesting interpretation of this may be
        that these industries require a more dynamic management
        methodology, perhaps using ``rolling'' budgets, etc.\ to
        approximate an immediate feedback mechanism. But this would
        seem to be inconsistent with methodologies where objectives
        are monitored on an annual basis---it would seem that profit
        and loss issues are very dynamic, and, probably, require
        detailed attention at no more than a monthly rate, including
        inventory and project management issues.

    \end{abstract}

% Local Variables:
% TeX-parse-self: t
% TeX-auto-save: t
% TeX-master: "fractal.tex"
% End:
