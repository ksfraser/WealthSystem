%
% -----------------------------------------------------------------------------
%
% A license is hereby granted to reproduce this software source code and
% to create executable versions from this source code for personal,
% non-commercial use.  The copyright notice included with the software
% must be maintained in all copies produced.
%
% THIS PROGRAM IS PROVIDED "AS IS". THE AUTHOR PROVIDES NO WARRANTIES
% WHATSOEVER, EXPRESSED OR IMPLIED, INCLUDING WARRANTIES OF
% MERCHANTABILITY, TITLE, OR FITNESS FOR ANY PARTICULAR PURPOSE.  THE
% AUTHOR DOES NOT WARRANT THAT USE OF THIS PROGRAM DOES NOT INFRINGE THE
% INTELLECTUAL PROPERTY RIGHTS OF ANY THIRD PARTY IN ANY COUNTRY.
%
% Copyright (c) 1994-2006, John Conover, All Rights Reserved.
%
% Comments and/or bug reports should be addressed to:
%
%     john@email.johncon.com (John Conover)
%
% -----------------------------------------------------------------------------
%
% Revision: \RCSRevision \\
% Revision Time: \RCSTime UMT \\
% Revision Date: \RCSDate \\
% Revision Id: \RCSId \\
% Revision File: \RCSLog \\
\RCS $Revision: 0.0 $
\RCS $Date: 2006/01/09 04:38:13 $
\RCS $Id: appd.tex,v 0.0 2006/01/09 04:38:13 john Exp $
% $Log: appd.tex,v $
% Revision 0.0  2006/01/09 04:38:13  john
% Initial version
%
%
% The following are numerical data macros-the data is inserted by the
% the input of the parameters.tex files-these files were made during the
% processing of the data for each market place in ../markets, with the
% execution of the maketex.awk awk script, which is as follows:
%
% -----------------------------------------------------------------------------
%
% Make the LaTeX parameters for this market
%
% Input must be successive records that contain:
%
%     0) the mean of the file data.tsfraction, from file
%         "data.tsfraction.tsnormal-p.mean"
%     1) the standard deviation of the file data.tsfraction, from the file,
%         "data.tsfraction.tsnormal-p.stddev"
%     2) the root mean square of the file data.tsfraction, from file
%         "data.tsfraction.tsrms-p"
%     3) the fraction of cumulative returns wagered, from file
%         "data.tsfraction.abs.tsnormal-p.mean"
%     4) the standard deviation of cumulative returns wagered, from the
%         the file "data.tsfraction.abs.tsnormal-p.stddev"
%     5) the constant in the least squares approximation of the file
%         data.tsfraction, from the file "data.tsfraction.tslsq-p.constant"
%     6) the slope in the least squares approximation of the file
%         data.tsfraction, from the file "data.tsfraction.tslsq-p.slope"
%     7) the constant in the least squares approximation of the file
%         data.tsfraction.abs , from the file
%         "data.tsfraction.abs.tslsq-p.constant"
%     8) the slope in the least squares approximation of the file
%         data.tsfraction.abs, from the file
%         "data.tsfraction.abs.tslsq-p.slope"
%     9) the hurst coefficient, from file
%        "data.tsfraction.tshurst-d.tslsq-p.hurstall"
%     10) the hurst coefficient for the lower end of the graph, from file
%         "data.tsfraction.tshurst-d.tslsq-p.low.hurstlow"
%     11) the h parameter, from file
%         "data.tsfraction.tshcalc-d.tslsq-p.hcalcall"
%     12) the h parameter for the lower end end of the graph, from file
%         "data.tsfraction.tshcalc-d.tslsq-p.low.hcalclow"
%     13) the maximum value of Shannon probability, from file
%         "data.tsshannonmax-p.max"
%     14) the logrithmic returns using tslogreturns, from file
%         "data.tslogreturns-p.tsshannon.returns"
%     15) the count of records with negative signes in data.tsfraction, from
%         the file "data.tsfraction.pmaxnumerator"
%     16) the count of records in data.tsfraction, from the file
%         "data.tsfraction.pmaxdenominator"
%     17) the mean of the file tsunfairbrownian-f.tsfraction, from file
%         "tsunfairbrownian-f.fraction.mean"
%     18) the standard deviation of the file tsunfairbrownian-f.tsfraction,
%         from file "tsunfairbrownian-f.fraction.mean"
%     19) the Shannon probability using tslogreturns, from file
%         "data.tslogreturns-p.tsshannon.probability"
%     20) the logrithmic returns in bits from the file
%         "data.tslsq-e-p.bits"
%     21) the traditional hurst coefficient, from file
%        "data.tshurst.tslsq-p.hurstall"
%     22) the traditional hurst coefficient for the lower end of the graph,
%         from file "data.tshurst.tslsq-p.low.hurstlow"
%     23) the traditional h parameter, from file
%         "data.tshcalc.tslsq-p.hcalcall"
%     24) the traditional h parameter for the lower end end of the graph,
%         from file "data.tshcalc.tslsq-p.low.hcalclow"
%     25) the chi-squared value, from file "chisquared"
%     26) the critical value for the chi-squared value, from file
%         "critical"
%
%{
%
%    if (linectr == 0)
%    {
%        datafractionmean = $0
%        printf ("\\renewcommand{\\datafractionmean}{%f}\n", datafractionmean)
%        datafractionmeanbits = log (datafractionmean + 1) / log (2.0)
%        printf ("\\renewcommand{\\datafractionmeanbits}{%f}\n", datafractionmeanbits)
%        datafractionmeanq = datafractionmean / 3.0
%        printf ("\\renewcommand{\\datafractionmeanq}{%f}\n", datafractionmeanq)
%        datafractionmeanbitsq = log (datafractionmeanq + 1) / log (2.0)
%        printf ("\\renewcommand{\\datafractionmeanbitsq}{%f}\n", datafractionmeanbitsq)
%    }
%
%    if (linectr == 1)
%    {
%        datafractionstddev = $0
%        printf ("\\renewcommand{\\datafractionstddev}{%f}\n", datafractionstddev)
%    }
%
%    if (linectr == 2)
%    {
%        datafractionrms = $0
%        printf ("\\renewcommand{\\datafractionrms}{%f}\n", datafractionrms)
%        avgrms = ((datafractionmean / datafractionrms) + 1.0) / 2
%        printf ("\\renewcommand{\\avgrms}{%f}\n", avgrms)
%        ncompanies = datafractionmean / (datafractionrms * datafractionrms)
%        printf ("\\renewcommand{\\ncompanies}{%f}\n", ncompanies)
%        pncompanies = ((datafractionmean / (sqrt (ncompanies) * datafractionrms)) + 1.0) / 2.0
%        printf ("\\renewcommand{\\pncompanies}{%f}\n", pncompanies)
%    }
%
%    if (linectr == 3)
%    {
%        datafractionabsmean = $0
%        printf ("\\renewcommand{\\datafractionabsmean}{%f}\n", datafractionabsmean)
%    }
%
%    if (linectr == 4)
%    {
%        datafractionabsstddev = $0
%        printf ("\\renewcommand{\\datafractionabsstddev}{%f}\n", datafractionabsstddev)
%    }
%
%    if (linectr == 5)
%    {
%        datafractionconstant = $0
%        printf ("\\renewcommand{\\datafractionconstant}{%f}\n", datafractionconstant)
%        datafractionconstantbits = log (datafractionconstant + 1) / log (2.0)
%        printf ("\\renewcommand{\\datafractionconstantbits}{%f}\n", datafractionconstantbits)
%        datafractionconstantq = datafractionconstant / 3.0
%        printf ("\\renewcommand{\\datafractionconstantq}{%f}\n", datafractionconstantq)
%        datafractionconstantbitsq = log (datafractionconstantq + 1) / log (2.0)
%        printf ("\\renewcommand{\\datafractionconstantbitsq}{%f}\n", datafractionconstantbitsq)
%    }
%
%    if (linectr == 6)
%    {
%        datafractionslope = $0
%        printf ("\\renewcommand{\\datafractionslope}{%f}\n", datafractionslope)
%    }
%
%    if (linectr == 7)
%    {
%        datafractionabsconstant = $0
%        printf ("\\renewcommand{\\datafractionabsconstant}{%f}\n", datafractionabsconstant)
%    }
%
%    if (linectr == 8)
%    {
%        datafractionabsslope = $0
%        printf ("\\renewcommand{\\datafractionabsslope}{%f}\n", datafractionabsslope)
%    }
%
%    if (linectr == 9)
%    {
%        hurstall = $0
%        printf ("\\renewcommand{\\hurstall}{%f}\n", hurstall)
%    }
%
%    if (linectr == 10)
%    {
%        hurstlow = $0
%        printf ("\\renewcommand{\\hurstlow}{%f}\n", hurstlow)
%        hurstlowtwo = hurstlow * 2.0
%        printf ("\\renewcommand{\\hurstlowtwo}{%f}\n", hurstlowtwo)
%        hurstlowhundred = hurstlow * 100.0
%        printf ("\\renewcommand{\\hurstlowhundred}{%f}\n", hurstlowhundred)
%    }
%
%    if (linectr == 11)
%    {
%        hcalcall = $0
%        printf ("\\renewcommand{\\hcalcall}{%f}\n", hcalcall)
%    }
%
%    if (linectr == 12)
%    {
%        hcalclow = $0
%        printf ("\\renewcommand{\\hcalclow}{%f}\n", hcalclow)
%    }
%
%    if (linectr == 13)
%    {
%        shannonmax = $0
%        printf ("\\renewcommand{\\shannonmax}{%f}\n", shannonmax)
%        twoponemax = 2.0 * shannonmax - 1
%        printf ("\\renewcommand{\\twoponemax}{%f}\n", twoponemax)
%    }
%
%    if (linectr == 14)
%    {
%        logreturns = $0
%        printf ("\\renewcommand{\\logreturns}{%f}\n", logreturns)
%        twologreturns = exp (logreturns * log (2.0))
%        printf ("\\renewcommand{\\twologreturns}{%f}\n", twologreturns)
%        twologreturnshundred = (twologreturns - 1.0) * 100.0
%        printf ("\\renewcommand{\\twologreturnshundred}{%f}\n", twologreturnshundred)
%        oneoverlogreturns = 1.0 / logreturns
%        printf ("\\renewcommand{\\oneoverlogreturns}{%f}\n", oneoverlogreturns)
%    }
%
%    if (linectr == 15)
%    {
%        pmaxnumerator = $0
%    }
%
%    if (linectr == 16)
%    {
%        pmaxdenominator = $0
%        pmax = (pmaxdenominator - pmaxnumerator) / pmaxdenominator
%        if (pmax == 1)
%        {
%            pmax =0.99999
%        }
%        printf ("\\renewcommand{\\pmax}{%f}\n", pmax)
%        twopminusone = (2 * pmax) - 1
%        printf ("\\renewcommand{\\twopminusone}{%f}\n", twopminusone)
%        rmsp = datafractionrms * twopminusone
%        printf ("\\renewcommand{\\rmsp}{%f}\n", rmsp)
%        twopx = ((2 * pmax) - 1) / (2 * sqrt (pmax * (1 - pmax)))
%        printf ("\\renewcommand{\\twopx}{%f}\n", twopx)
%        sigmap = datafractionstddev * twopx
%        printf ("\\renewcommand{\\sigmap}{%f}\n", sigmap)
%
%    }
%
%    if (linectr == 17)
%    {
%        tsunfairbrownianfractionmean = $0
%        printf ("\\renewcommand{\\tsunfairbrownianfractionmean}{%f}\n", tsunfairbrownianfractionmean)
%    }
%
%    if (linectr == 18)
%    {
%        tsunfairbrownianfractionstddev = $0
%        printf ("\\renewcommand{\\tsunfairbrownianfractionstddev}{%f}\n", tsunfairbrownianfractionstddev)
%    }
%
%    if (linectr == 19)
%    {
%        shannonlogreturns = $0
%        printf ("\\renewcommand{\\shannonlogreturns}{%f}\n", shannonlogreturns)
%        shannonlogreturnshundred = shannonlogreturns * 100.0
%        printf ("\\renewcommand{\\shannonlogreturnshundred}{%f}\n", shannonlogreturnshundred)
%        twopone =  (2.0 * shannonlogreturns) - 1.0
%        printf ("\\renewcommand{\\twopone}{%f}\n", twopone)
%        twoponehundred = twopone * 100.0
%        printf ("\\renewcommand{\\twoponehundred}{%f}\n", twoponehundred)
%        hundredtwoponehundred = 100.0 - twoponehundred
%        printf ("\\renewcommand{\\hundredtwoponehundred}{%f}\n", hundredtwoponehundred)
%        hundredshannonlogreturnshundred = 100.0 - shannonlogreturnshundred
%        printf ("\\renewcommand{\\hundredshannonlogreturnshundred}{%f}\n", hundredshannonlogreturnshundred)
%    }
%
%    if (linectr == 20)
%    {
%        datatslsqepbits = $0
%        printf ("\\renewcommand{\\datatslsqepbits}{%f}\n", datatslsqepbits)
%    }
%
%    if (linectr == 21)
%    {
%        thurstall = $0
%        printf ("\\renewcommand{\\thurstall}{%f}\n", thurstall)
%    }
%
%    if (linectr == 22)
%    {
%        thurstlow = $0
%        printf ("\\renewcommand{\\thurstlow}{%f}\n", thurstlow)
%        thurstlowtwo = thurstlow * 2.0
%        printf ("\\renewcommand{\\thurstlowtwo}{%f}\n", thurstlowtwo)
%        thurstlowhundred = thurstlow * 100.0
%        printf ("\\renewcommand{\\thurstlowhundred}{%f}\n", thurstlowhundred)
%    }
%
%    if (linectr == 23)
%    {
%        thcalcall = $0
%        printf ("\\renewcommand{\\thcalcall}{%f}\n", thcalcall)
%    }
%
%    if (linectr == 24)
%    {
%        thcalclow = $0
%        printf ("\\renewcommand{\\thcalclow}{%f}\n", thcalclow)
%    }
%
%    if (linectr == 25)
%    {
%        chisquared = $0
%        printf ("\\renewcommand{\\chisquared}{%f}\n", chisquared)
%    }
%
%    if (linectr == 26)
%    {
%        critical = $0
%        printf ("\\renewcommand{\\critical}{%f}\n", critical)
%    }
%
%    linectr++
%}
%\newcommand{\LABPRE}{}
%\newcommand{\LABPREREF}{}
%\newcommand{\market}{}
%\newcommand{\directory}{}
%\newcommand{\timescale}{}
%\newcommand{\SETLABEL}{}
%\newcommand{\SETLABELQ}{}
%\newcommand{\SETLABELREF}{}
%\newcommand{\datafractionmean}{0.0}
%\newcommand{\datafractionmeanbits}{0.0}
%\newcommand{\datafractionmeanq}{0.0}
%\newcommand{\datafractionmeanbitsq}{0.0}
%\newcommand{\datafractionstddev}{0.0}
%\newcommand{\datafractionrms}{0.0}
%\newcommand{\avgrms}{0.0}
%\newcommand{\ncompanies}{0.0}
%\newcommand{\pncompanies}{0.0}
%\newcommand{\datafractionabsmean}{0.0}
%\newcommand{\datafractionabsstddev}{0.0}
%\newcommand{\datafractionconstant}{0.0}
%\newcommand{\datafractionconstantbits}{0.0}
%\newcommand{\datafractionconstantq}{0.0}
%\newcommand{\datafractionconstantbitsq}{0.0}
%\newcommand{\datafractionslope}{0.0}
%\newcommand{\datafractionabsconstant}{0.0}
%\newcommand{\datafractionabsslope}{0.0}
%\newcommand{\hurstall}{0.0}
%\newcommand{\hurstlow}{0.0}
%\newcommand{\hurstlowtwo}{0.0}
%\newcommand{\hurstlowhundred}{0.0}
%\newcommand{\hcalcall}{0.0}
%\newcommand{\hcalclow}{0.0}
%\newcommand{\shannonlogreturns}{0.0}
%\newcommand{\shannonlogreturnshundred}{0.0}
%\newcommand{\hundredshannonlogreturnshundred}{0.0}
%\newcommand{\datatslsqepbits}{0.0}
%\newcommand{\twopone}{0.0}
%\newcommand{\twoponehundred}{0.0}
%\newcommand{\hundredtwoponehundred}{0.0}
%\newcommand{\logreturns}{0.0}
%\newcommand{\twologreturns}{0.0}
%\newcommand{\twologreturnshundred}{0.0}
%\newcommand{\oneoverlogreturns}{0.0}
%\newcommand{\shannonmax}{0.0}
%\newcommand{\twoponemax}{0.0}
%\newcommand{\pmax}{0.0}
%\newcommand{\twopminusone}{0.0}
%\newcommand{\rmsp}{0.0}
%\newcommand{\twopx}{0.0}
%\newcommand{\sigmap}{0.0}
%\newcommand{\tsunfairbrownianfractionmean}{0.0}
%\newcommand{\tsunfairbrownianfractionstddev}{0.0}
%\newcommand{\thurstall}{0.0}
%\newcommand{\thurstlow}{0.0}
%\newcommand{\thurstlowtwo}{0.0}
%\newcommand{\thurstlowhundred}{0.0}
%\newcommand{\thcalcall}{0.0}
%\newcommand{\thcalclow}{0.0}
%\newcommand{\chisquared}{0.0}
%\newcommand{\critical}{0.0}
%
\chapter{Condensed Fractal Analysis of Various Market Segments in the North American Electronics Industry}
    \label{tables}

    This appendix presents, in condensed tabular form, the numerical
    metrics that were derived in appendix~\ref{markets}.

    \renewcommand{\LABPRE}{D}
    \renewcommand{\LABPREREF}{C}

    \renewcommand{\market}{North American Integrated Circuit Market}
    \renewcommand{\directory}{../markets/ic.namerica}
    \renewcommand{\datafractionmean}{0.008052}
\renewcommand{\datafractionmeanbits}{0.011570}
\renewcommand{\datafractionmeanq}{0.002684}
\renewcommand{\datafractionmeanbitsq}{0.003867}
\renewcommand{\datafractionstddev}{0.038579}
\renewcommand{\datafractionrms}{0.039311}
\renewcommand{\avgrms}{0.602414}
\renewcommand{\ncompanies}{5.210454}
\renewcommand{\pncompanies}{0.544866}
\renewcommand{\datafractionabsmean}{0.029745}
\renewcommand{\datafractionabsstddev}{0.025769}
\renewcommand{\datafractionconstant}{0.010041}
\renewcommand{\datafractionconstantbits}{0.014414}
\renewcommand{\datafractionconstantq}{0.003347}
\renewcommand{\datafractionconstantbitsq}{0.004821}
\renewcommand{\datafractionslope}{-0.000021}
\renewcommand{\datafractionabsconstant}{0.035145}
\renewcommand{\datafractionabsslope}{-0.000057}
\renewcommand{\hurstall}{0.659558}
\renewcommand{\hurstlow}{0.707509}
\renewcommand{\hurstlowtwo}{1.415018}
\renewcommand{\hurstlowhundred}{70.750900}
\renewcommand{\hcalcall}{0.184942}
\renewcommand{\hcalclow}{0.102042}
\renewcommand{\shannonmax}{0.604167}
\renewcommand{\twoponemax}{0.208334}
\renewcommand{\logreturns}{0.010456}
\renewcommand{\twologreturns}{1.007274}
\renewcommand{\twologreturnshundred}{0.727387}
\renewcommand{\oneoverlogreturns}{95.638868}
\renewcommand{\pmax}{0.602094}
\renewcommand{\twopminusone}{0.204188}
\renewcommand{\rmsp}{0.008027}
\renewcommand{\twopx}{0.208583}
\renewcommand{\sigmap}{0.008047}
\renewcommand{\tsunfairbrownianfractionmean}{0.007862}
\renewcommand{\tsunfairbrownianfractionstddev}{0.038619}
\renewcommand{\shannonlogreturns}{0.560125}
\renewcommand{\shannonlogreturnshundred}{56.012500}
\renewcommand{\twopone}{0.120250}
\renewcommand{\twoponehundred}{12.025000}
\renewcommand{\hundredtwoponehundred}{87.975000}
\renewcommand{\hundredshannonlogreturnshundred}{43.987500}
\renewcommand{\datatslsqepbits}{0.007623}
\renewcommand{\thurstall}{0.633980}
\renewcommand{\thurstlow}{0.710108}
\renewcommand{\thurstlowtwo}{1.420216}
\renewcommand{\thurstlowhundred}{71.010800}
\renewcommand{\thcalcall}{0.247886}
\renewcommand{\thcalclow}{0.171737}
\renewcommand{\chisquared}{2.862000}
\renewcommand{\critical}{42.557000}

    \renewcommand{\timescale}{quarter}
    \subidx{market}{\market}
    \idx{\market}

    \section{\market}

        \renewcommand{\SETLABEL}{\LABPRE:NAICM}
        \renewcommand{\SETLABELREF}{\LABPREREF:NAICM}
        \label{\SETLABEL}

        \idx{Semiconductor Industry Association}
        For the analysis, the data was in the directory
        {\directory}\footnote{Data from the Semiconductor Industry
        Association, 1979---1994, by {\timescale}s, in millions of
        dollars, US.}.

        The data in this section is presented in
        Section~\ref{\SETLABELREF}.

        %
% -----------------------------------------------------------------------------
%
% A license is hereby granted to reproduce this software source code and
% to create executable versions from this source code for personal,
% non-commercial use.  The copyright notice included with the software
% must be maintained in all copies produced.
%
% THIS PROGRAM IS PROVIDED "AS IS". THE AUTHOR PROVIDES NO WARRANTIES
% WHATSOEVER, EXPRESSED OR IMPLIED, INCLUDING WARRANTIES OF
% MERCHANTABILITY, TITLE, OR FITNESS FOR ANY PARTICULAR PURPOSE.  THE
% AUTHOR DOES NOT WARRANT THAT USE OF THIS PROGRAM DOES NOT INFRINGE THE
% INTELLECTUAL PROPERTY RIGHTS OF ANY THIRD PARTY IN ANY COUNTRY.
%
% Copyright (c) 1994-2006, John Conover, All Rights Reserved.
%
% Comments and/or bug reports should be addressed to:
%
%     john@email.johncon.com (John Conover)
%
% -----------------------------------------------------------------------------
%
% Revision: \RCSRevision \\
% Revision Time: \RCSTime UMT \\
% Revision Date: \RCSDate \\
% Revision Id: \RCSId \\
% Revision File: \RCSLog \\
\RCS $Revision: 0.0 $
\RCS $Date: 2006/01/20 04:38:13 $
\RCS $Id: tables.tex,v 0.0 2006/01/20 04:38:13 john Exp $
% $Log: tables.tex,v $
% Revision 0.0  2006/01/20 04:38:13  john
% Initial version
%
%
    \subsection{{\market}, normalized increments}
        \label{\SETLABEL:TSA}

        The data in table~\ref{\SETLABEL:INC} is condensed from
        Section~\ref{\SETLABELREF:TSA}.

        \begin{small}
            \begin{table}[ht]
                \begin{center}
                    \caption[{\market}, normalized increments]
                        {{\market}, normalized increments.}
                    \begin{tabular}{|c|c|c|c|c|c|c|c|c|c|} \hline
                        \multicolumn{5}{|c|}{Normalized}                                                                                  & \multicolumn{5}{|c|}{Normalized Absolute Value}\\ \hline
                        Mean                & Standard              & rms                & \multicolumn{2}{|c|}{Least Squares}            & Mean                   & Standard                 & rms                & \multicolumn{2}{|c|}{Least Squares} \\ \cline{4-5}\cline{9-10}
                        \hspace{0.01in}     & deviation             & \hspace{0.01in}    & Constant                & Slope                & \hspace{0.01in}        & deviation                & \hspace{0.01in}    & Constant                   & Slope \\ \hline\hline
                        {\datafractionmean} & {\datafractionstddev} & {\datafractionrms} & {\datafractionconstant} & {\datafractionslope} & {\datafractionabsmean} & {\datafractionabsstddev} & {\datafractionrms} & {\datafractionabsconstant} & {\datafractionabsslope} \\ \hline
                    \end{tabular}
                    \label{\SETLABEL:INC}
                \end{center}
            \end{table}
        \end{small}

    \subsection{{\market}, Logarithmic Returns, in Bits}
        \label{\SETLABEL:LR}

        The data in table~\ref{\SETLABEL:RET} is condensed from
        Section~\ref{\SETLABELREF:FS}.

        \begin{small}
            \begin{table}[ht]
                \begin{center}
                    \caption[{\market}, Logarithmic Returns, in
                        Bits]{{\market}, Logarithmic Returns, in Bits.}
                    \begin{tabular}{|c|c|c|c|} \hline
                        \multicolumn{2}{|c|}{Calculated from Table~\ref{\SETLABEL:INC}} & \multicolumn{2}{|c|}{From program:}\\ \hline
                        Mean                    & Least squares                       & {\it tslsq}\/              & {\it tslogreturns}\/ \\ \hline\hline
                        {\datafractionmeanbits} & {\datafractionconstantbits} & {\datatslsqepbits} & {\logreturns} \\ \hline
                    \end{tabular}
                    \label{\SETLABEL:RET}
                \end{center}
            \end{table}
        \end{small}

    \subsection{{\market}, Shannon probabilities}
        \label{\SETLABEL:MAXSHANNON}

        The data in table~\ref{\SETLABEL:SHANNON} is condensed from
        sections~\ref{\SETLABELREF:FS}
        and~\ref{\SETLABELREF:MAXSHANNON}.

        \begin{small}
            \begin{table}[ht]
                \begin{center}
                    \caption[{\market}, Shannon
                        probabilities]{{\market}, Shannon
                        probabilities.}
                    \begin{tabular}{|c|c|c|c|} \hline
                        \multicolumn{3}{|c|}{Maximum} & \multicolumn{1}{|c|}{Operational}\\ \hline
                        Fraction of         & $\frac{\frac{\mbox{\scriptsize{mean}}}{\mbox{\scriptsize{rms}}} + 1}{2}$ & \multicolumn{2}{|c|}{From program:}\\ \cline{3-4}
                        positive increments & \hspace{0.01in}                                                          & {\it tsshannonmax}\/    & {\it tsshannon}\/ \\ \hline\hline
                        {\pmax}             & {\avgrms}                                                                & {\shannonmax}   & {\shannonlogreturns} \\ \hline
                    \end{tabular}
                    \label{\SETLABEL:SHANNON}
                \end{center}
            \end{table}
        \end{small}

    \subsection{{\market}, Logistic Analysis}
        \label{\SETLABEL:LAA}

        The data in table~\ref{\SETLABEL:LA} is condensed from
        Section~\ref{\SETLABELREF:LA}\footnote{Note that there are
        numerical stability issues with the methodology used to derive
        the constants---if the non-linear term, $b$, was greater than
        zero, it was set to zero to produce the graphs in
        Section~\ref{\SETLABELREF:LA}.}.

        \begin{small}
            \begin{table}[ht]
                \begin{center}
                    \caption[{\market}, Logistic Analysis.]
                        {{\market}, Logistic Analysis, $x_t = x_{t - 1}\left(a + b \cdot x_{t - 1}\right)$.}
                    \begin{tabular}{|c|c|} \hline
                        $a$ & $b$\\ \hline\hline
                        {\datafractionconstant} & {\datafractionslope}\\ \hline
                    \end{tabular}
                    \label{\SETLABEL:LA}
                \end{center}
            \end{table}
        \end{small}

    \subsection{{\market}, Hurst Coefficients and H  Parameters}
        \label{\SETLABEL:HCHP}

        The data in table~\ref{\SETLABEL:H} is condensed from
        Section~\ref{\SETLABELREF:H}.

        \begin{small}
            \begin{table}[ht]
                \begin{center}
                    \caption[{\market}, Hurst Coefficients and H
                        Parameters]{{\market}, Hurst Coefficients and
                        H Parameters.}
                    \begin{tabular}{|c|c|c|c|} \hline
                        \multicolumn{2}{|c|}{Hurst Coefficients} & \multicolumn{2}{|c|}{H Parameters}\\ \hline
                        Near term   & Far term    & Near term   & Far term \\ \hline\hline
                        {\thurstlow} & {\thurstall} & {\thcalclow} & {\thcalcall} \\ \hline
                    \end{tabular}
                    \label{\SETLABEL:H}
                \end{center}
            \end{table}
        \end{small}

        \begin{small}
            \begin{table}[ht]
                \begin{center}
                    \caption[{\market}, Hurst Coefficients and H
                        Parameters]{{\market}, Hurst Coefficients and
                        H Parameters, as a Derivative.}
                    \begin{tabular}{|c|c|c|c|} \hline
                        \multicolumn{2}{|c|}{Hurst Coefficients} & \multicolumn{2}{|c|}{H Parameters}\\ \hline
                        Near term    & Far term     & Near term    & Far term \\ \hline\hline
                        {\hurstlow} & {\hurstall} & {\hcalclow} & {\hcalcall} \\ \hline
                    \end{tabular}
                    \label{\SETLABEL:TH}
                \end{center}
            \end{table}
        \end{small}

    \subsection{{\market}, verification of the increments}
        \label{\SETLABEL:VI1}

        The data in table~\ref{\SETLABEL:COMP} is condensed from
        Section~\ref{\SETLABELREF:QVA}.

        \begin{small}
            \begin{table}[ht]
                \begin{center}
                    \caption[{\market}, verification of
                        the increments]{{\market}, verification the of
                        the increments, the mean, $\sigma$ is the
                        standard deviation from
                        table~\ref{\SETLABEL:INC},
                        {\datafractionstddev}, and $P$ is the maximum
                        Shannon probability from
                        table~\ref{\SETLABEL:SHANNON}, {\pmax}. In
                        principle, the values should equate.}
                    \begin{tabular}{|c|c|c|} \hline
                        Mean                & $\mbox{rms} (2P - 1)$ & $\frac{{\sigma}(2P - 1)}{2\sqrt{P(P - 1)}} $ \\ \hline\hline
                        {\datafractionmean} & {\rmsp}               & {\sigmap} \\ \hline
                    \end{tabular}
                    \label{\SETLABEL:COMP}
                \end{center}
            \end{table}
        \end{small}

    \subsection{{\market}, verification of the increments}
        \label{\SETLABEL:VI2}

        The data in table~\ref{\SETLABEL:ABS} is condensed from
        Section~\ref{\SETLABELREF:QVA}.

        \begin{small}
            \begin{table}[ht]
                \begin{center}
                    \caption[{\market}, verification of
                        the increments]{{\market}, verification the of
                        increments. In principle, the mean of the
                        absolute value of the increments and the root
                        mean square of the increments should
                        equate\footnote{The absolute value of the
                        normalized increments, when averaged, is
                        related to the root mean square of the
                        increments by a constant. If the normalized
                        increments are a fixed increment, the constant
                        is unity. If the normalized increments have a
                        Gaussian distribution, the constant is
                        $\approx 0.8$ depending on the accuracy of of
                        ``fit'' to a Gaussian distribution.}.}
                    \begin{tabular}{|c|c|} \hline
                        Mean of the               & rms \\
                        absolute value            & \hspace{0.01in} \\ \hline\hline
                        {\datafractionabsmean}    & {\datafractionrms} \\ \hline
                    \end{tabular}
                    \label{\SETLABEL:ABS}
                \end{center}
            \end{table}
        \end{small}

    \subsection{{\market}, $\chi^2$ values of the increments}
        \label{\SETLABEL:XSQ}

        The data in table~\ref{\SETLABEL:XSQT} is condensed from
        Section~\ref{\SETLABELREF:NH}.

        \begin{small}
            \begin{table}[ht]
                \begin{center}
                    \caption[{\market}, $\chi^2$ values of
                        the increments]{{\market}, $\chi^2$ values of
                        the increments. In principle, if the
                        distribution of the normalized increments is a
                        Gaussian distribution, the $\chi^2$ value will
                        be significantly less than the critical
                        value.}
                    \begin{tabular}{|c|c|} \hline
                        $\chi^2$      & Critical Value \\ \hline\hline
                        {\chisquared} & {\critical} \\ \hline
                    \end{tabular}
                    \label{\SETLABEL:XSQT}
                \end{center}
            \end{table}
        \end{small}

    \subsection{{\market}, time series data, empirical and simulated}
        \label{\SETLABEL:SIM}

        The data in table~\ref{\SETLABEL:ES} is condensed from
        Section~\ref{\SETLABELREF:TSUNFAIRBROWNIAN}.

        \begin{small}
            \begin{table}[ht]
                \begin{center}
                    \caption[{\market}, time series data, empirical
                        and simulated]{{\market}, time series data,
                        empirical and simulated, analysis of the
                        normalized increments.}
                    \begin{tabular}{|c|c|c|c|} \hline
                        \multicolumn{2}{|c|}{Empirical} & \multicolumn{2}{|c|}{Simulated}\\ \hline
                        Mean                & Standard              & Mean               & Standard \\
                        \hspace{0.01in}     & deviation             & \hspace{0.01in}    & deviation \\ \hline\hline
                        {\datafractionmean} & {\datafractionstddev} & {\tsunfairbrownianfractionmean} & {\tsunfairbrownianfractionstddev} \\ \hline
                    \end{tabular}
                    \label{\SETLABEL:ES}
                \end{center}
            \end{table}
        \end{small}

    \subsection{{\market}, number of participating companies}
        \label{\SETLABEL:QNC}

        The data in table~\ref{\SETLABEL:NC} is condensed from
        Section~\ref{\SETLABELREF:QNC}.

        \begin{small}
            \begin{table}[ht]
                \begin{center}
                    \caption[{\market}, number of participating
                        companies] {{\market}, number of participating
                        companies.}
                    \begin{tabular}{|c|c|} \hline
                        Number & Shannon probability\\ \hline
                        {\ncompanies} & {\pncompanies}\\ \hline
                    \end{tabular}
                    \label{\SETLABEL:NC}
                \end{center}
            \end{table}
        \end{small}

    \subsection{{\market}, Shannon probability optimizations}
        \label{\SETLABEL:SPO}

        The data in table~\ref{\SETLABEL:SP} is condensed from
        Section~\ref{\SETLABELREF:QNC}.

        \begin{small}
            \begin{table}[ht]
                \begin{center}
                    \caption[{\market}, Shannon probability
                         optimizations] {{\market}, Shannon
                         probability optimization.}
                    \begin{tabular}{|c|c|} \hline
                        optimize capital growth & optimize market growth\\ \hline
                        {\avgrms} & {\pncompanies}\\ \hline
                    \end{tabular}
                    \label{\SETLABEL:SP}
                \end{center}
            \end{table}
        \end{small}

% Local Variables:
% TeX-parse-self: t
% TeX-auto-save: t
% TeX-master: "fractal.tex"
% End:


    \renewcommand{\market}{World Semiconductor Market}
    \renewcommand{\directory}{../markets/semiconductors.world}
    \renewcommand{\datafractionmean}{0.008052}
\renewcommand{\datafractionmeanbits}{0.011570}
\renewcommand{\datafractionmeanq}{0.002684}
\renewcommand{\datafractionmeanbitsq}{0.003867}
\renewcommand{\datafractionstddev}{0.038579}
\renewcommand{\datafractionrms}{0.039311}
\renewcommand{\avgrms}{0.602414}
\renewcommand{\ncompanies}{5.210454}
\renewcommand{\pncompanies}{0.544866}
\renewcommand{\datafractionabsmean}{0.029745}
\renewcommand{\datafractionabsstddev}{0.025769}
\renewcommand{\datafractionconstant}{0.010041}
\renewcommand{\datafractionconstantbits}{0.014414}
\renewcommand{\datafractionconstantq}{0.003347}
\renewcommand{\datafractionconstantbitsq}{0.004821}
\renewcommand{\datafractionslope}{-0.000021}
\renewcommand{\datafractionabsconstant}{0.035145}
\renewcommand{\datafractionabsslope}{-0.000057}
\renewcommand{\hurstall}{0.659558}
\renewcommand{\hurstlow}{0.707509}
\renewcommand{\hurstlowtwo}{1.415018}
\renewcommand{\hurstlowhundred}{70.750900}
\renewcommand{\hcalcall}{0.184942}
\renewcommand{\hcalclow}{0.102042}
\renewcommand{\shannonmax}{0.604167}
\renewcommand{\twoponemax}{0.208334}
\renewcommand{\logreturns}{0.010456}
\renewcommand{\twologreturns}{1.007274}
\renewcommand{\twologreturnshundred}{0.727387}
\renewcommand{\oneoverlogreturns}{95.638868}
\renewcommand{\pmax}{0.602094}
\renewcommand{\twopminusone}{0.204188}
\renewcommand{\rmsp}{0.008027}
\renewcommand{\twopx}{0.208583}
\renewcommand{\sigmap}{0.008047}
\renewcommand{\tsunfairbrownianfractionmean}{0.007862}
\renewcommand{\tsunfairbrownianfractionstddev}{0.038619}
\renewcommand{\shannonlogreturns}{0.560125}
\renewcommand{\shannonlogreturnshundred}{56.012500}
\renewcommand{\twopone}{0.120250}
\renewcommand{\twoponehundred}{12.025000}
\renewcommand{\hundredtwoponehundred}{87.975000}
\renewcommand{\hundredshannonlogreturnshundred}{43.987500}
\renewcommand{\datatslsqepbits}{0.007623}
\renewcommand{\thurstall}{0.633980}
\renewcommand{\thurstlow}{0.710108}
\renewcommand{\thurstlowtwo}{1.420216}
\renewcommand{\thurstlowhundred}{71.010800}
\renewcommand{\thcalcall}{0.247886}
\renewcommand{\thcalclow}{0.171737}
\renewcommand{\chisquared}{2.862000}
\renewcommand{\critical}{42.557000}

    \renewcommand{\timescale}{quarter}
    \subidx{market}{\market}
    \idx{\market}

    \section{\market}

        \renewcommand{\SETLABEL}{\LABPRE:WSM}
        \renewcommand{\SETLABELQ}{\LABPRE:WSMQ}
        \label{\SETLABEL}

        \idx{Semiconductor Industry Association}
        For the analysis, the data was in the directory
        {\directory}\footnote{Data from the Semiconductor Industry
        Association, 1982---1994, by {\timescale}s, in millions of
        dollars, US.}.

        The data in this section is presented in
        Section~\ref{\SETLABELREF}.

        %
% -----------------------------------------------------------------------------
%
% A license is hereby granted to reproduce this software source code and
% to create executable versions from this source code for personal,
% non-commercial use.  The copyright notice included with the software
% must be maintained in all copies produced.
%
% THIS PROGRAM IS PROVIDED "AS IS". THE AUTHOR PROVIDES NO WARRANTIES
% WHATSOEVER, EXPRESSED OR IMPLIED, INCLUDING WARRANTIES OF
% MERCHANTABILITY, TITLE, OR FITNESS FOR ANY PARTICULAR PURPOSE.  THE
% AUTHOR DOES NOT WARRANT THAT USE OF THIS PROGRAM DOES NOT INFRINGE THE
% INTELLECTUAL PROPERTY RIGHTS OF ANY THIRD PARTY IN ANY COUNTRY.
%
% Copyright (c) 1994-2006, John Conover, All Rights Reserved.
%
% Comments and/or bug reports should be addressed to:
%
%     john@email.johncon.com (John Conover)
%
% -----------------------------------------------------------------------------
%
% Revision: \RCSRevision \\
% Revision Time: \RCSTime UMT \\
% Revision Date: \RCSDate \\
% Revision Id: \RCSId \\
% Revision File: \RCSLog \\
\RCS $Revision: 0.0 $
\RCS $Date: 2006/01/20 04:38:13 $
\RCS $Id: tables.tex,v 0.0 2006/01/20 04:38:13 john Exp $
% $Log: tables.tex,v $
% Revision 0.0  2006/01/20 04:38:13  john
% Initial version
%
%
    \subsection{{\market}, normalized increments}
        \label{\SETLABEL:TSA}

        The data in table~\ref{\SETLABEL:INC} is condensed from
        Section~\ref{\SETLABELREF:TSA}.

        \begin{small}
            \begin{table}[ht]
                \begin{center}
                    \caption[{\market}, normalized increments]
                        {{\market}, normalized increments.}
                    \begin{tabular}{|c|c|c|c|c|c|c|c|c|c|} \hline
                        \multicolumn{5}{|c|}{Normalized}                                                                                  & \multicolumn{5}{|c|}{Normalized Absolute Value}\\ \hline
                        Mean                & Standard              & rms                & \multicolumn{2}{|c|}{Least Squares}            & Mean                   & Standard                 & rms                & \multicolumn{2}{|c|}{Least Squares} \\ \cline{4-5}\cline{9-10}
                        \hspace{0.01in}     & deviation             & \hspace{0.01in}    & Constant                & Slope                & \hspace{0.01in}        & deviation                & \hspace{0.01in}    & Constant                   & Slope \\ \hline\hline
                        {\datafractionmean} & {\datafractionstddev} & {\datafractionrms} & {\datafractionconstant} & {\datafractionslope} & {\datafractionabsmean} & {\datafractionabsstddev} & {\datafractionrms} & {\datafractionabsconstant} & {\datafractionabsslope} \\ \hline
                    \end{tabular}
                    \label{\SETLABEL:INC}
                \end{center}
            \end{table}
        \end{small}

    \subsection{{\market}, Logarithmic Returns, in Bits}
        \label{\SETLABEL:LR}

        The data in table~\ref{\SETLABEL:RET} is condensed from
        Section~\ref{\SETLABELREF:FS}.

        \begin{small}
            \begin{table}[ht]
                \begin{center}
                    \caption[{\market}, Logarithmic Returns, in
                        Bits]{{\market}, Logarithmic Returns, in Bits.}
                    \begin{tabular}{|c|c|c|c|} \hline
                        \multicolumn{2}{|c|}{Calculated from Table~\ref{\SETLABEL:INC}} & \multicolumn{2}{|c|}{From program:}\\ \hline
                        Mean                    & Least squares                       & {\it tslsq}\/              & {\it tslogreturns}\/ \\ \hline\hline
                        {\datafractionmeanbits} & {\datafractionconstantbits} & {\datatslsqepbits} & {\logreturns} \\ \hline
                    \end{tabular}
                    \label{\SETLABEL:RET}
                \end{center}
            \end{table}
        \end{small}

    \subsection{{\market}, Shannon probabilities}
        \label{\SETLABEL:MAXSHANNON}

        The data in table~\ref{\SETLABEL:SHANNON} is condensed from
        sections~\ref{\SETLABELREF:FS}
        and~\ref{\SETLABELREF:MAXSHANNON}.

        \begin{small}
            \begin{table}[ht]
                \begin{center}
                    \caption[{\market}, Shannon
                        probabilities]{{\market}, Shannon
                        probabilities.}
                    \begin{tabular}{|c|c|c|c|} \hline
                        \multicolumn{3}{|c|}{Maximum} & \multicolumn{1}{|c|}{Operational}\\ \hline
                        Fraction of         & $\frac{\frac{\mbox{\scriptsize{mean}}}{\mbox{\scriptsize{rms}}} + 1}{2}$ & \multicolumn{2}{|c|}{From program:}\\ \cline{3-4}
                        positive increments & \hspace{0.01in}                                                          & {\it tsshannonmax}\/    & {\it tsshannon}\/ \\ \hline\hline
                        {\pmax}             & {\avgrms}                                                                & {\shannonmax}   & {\shannonlogreturns} \\ \hline
                    \end{tabular}
                    \label{\SETLABEL:SHANNON}
                \end{center}
            \end{table}
        \end{small}

    \subsection{{\market}, Logistic Analysis}
        \label{\SETLABEL:LAA}

        The data in table~\ref{\SETLABEL:LA} is condensed from
        Section~\ref{\SETLABELREF:LA}\footnote{Note that there are
        numerical stability issues with the methodology used to derive
        the constants---if the non-linear term, $b$, was greater than
        zero, it was set to zero to produce the graphs in
        Section~\ref{\SETLABELREF:LA}.}.

        \begin{small}
            \begin{table}[ht]
                \begin{center}
                    \caption[{\market}, Logistic Analysis.]
                        {{\market}, Logistic Analysis, $x_t = x_{t - 1}\left(a + b \cdot x_{t - 1}\right)$.}
                    \begin{tabular}{|c|c|} \hline
                        $a$ & $b$\\ \hline\hline
                        {\datafractionconstant} & {\datafractionslope}\\ \hline
                    \end{tabular}
                    \label{\SETLABEL:LA}
                \end{center}
            \end{table}
        \end{small}

    \subsection{{\market}, Hurst Coefficients and H  Parameters}
        \label{\SETLABEL:HCHP}

        The data in table~\ref{\SETLABEL:H} is condensed from
        Section~\ref{\SETLABELREF:H}.

        \begin{small}
            \begin{table}[ht]
                \begin{center}
                    \caption[{\market}, Hurst Coefficients and H
                        Parameters]{{\market}, Hurst Coefficients and
                        H Parameters.}
                    \begin{tabular}{|c|c|c|c|} \hline
                        \multicolumn{2}{|c|}{Hurst Coefficients} & \multicolumn{2}{|c|}{H Parameters}\\ \hline
                        Near term   & Far term    & Near term   & Far term \\ \hline\hline
                        {\thurstlow} & {\thurstall} & {\thcalclow} & {\thcalcall} \\ \hline
                    \end{tabular}
                    \label{\SETLABEL:H}
                \end{center}
            \end{table}
        \end{small}

        \begin{small}
            \begin{table}[ht]
                \begin{center}
                    \caption[{\market}, Hurst Coefficients and H
                        Parameters]{{\market}, Hurst Coefficients and
                        H Parameters, as a Derivative.}
                    \begin{tabular}{|c|c|c|c|} \hline
                        \multicolumn{2}{|c|}{Hurst Coefficients} & \multicolumn{2}{|c|}{H Parameters}\\ \hline
                        Near term    & Far term     & Near term    & Far term \\ \hline\hline
                        {\hurstlow} & {\hurstall} & {\hcalclow} & {\hcalcall} \\ \hline
                    \end{tabular}
                    \label{\SETLABEL:TH}
                \end{center}
            \end{table}
        \end{small}

    \subsection{{\market}, verification of the increments}
        \label{\SETLABEL:VI1}

        The data in table~\ref{\SETLABEL:COMP} is condensed from
        Section~\ref{\SETLABELREF:QVA}.

        \begin{small}
            \begin{table}[ht]
                \begin{center}
                    \caption[{\market}, verification of
                        the increments]{{\market}, verification the of
                        the increments, the mean, $\sigma$ is the
                        standard deviation from
                        table~\ref{\SETLABEL:INC},
                        {\datafractionstddev}, and $P$ is the maximum
                        Shannon probability from
                        table~\ref{\SETLABEL:SHANNON}, {\pmax}. In
                        principle, the values should equate.}
                    \begin{tabular}{|c|c|c|} \hline
                        Mean                & $\mbox{rms} (2P - 1)$ & $\frac{{\sigma}(2P - 1)}{2\sqrt{P(P - 1)}} $ \\ \hline\hline
                        {\datafractionmean} & {\rmsp}               & {\sigmap} \\ \hline
                    \end{tabular}
                    \label{\SETLABEL:COMP}
                \end{center}
            \end{table}
        \end{small}

    \subsection{{\market}, verification of the increments}
        \label{\SETLABEL:VI2}

        The data in table~\ref{\SETLABEL:ABS} is condensed from
        Section~\ref{\SETLABELREF:QVA}.

        \begin{small}
            \begin{table}[ht]
                \begin{center}
                    \caption[{\market}, verification of
                        the increments]{{\market}, verification the of
                        increments. In principle, the mean of the
                        absolute value of the increments and the root
                        mean square of the increments should
                        equate\footnote{The absolute value of the
                        normalized increments, when averaged, is
                        related to the root mean square of the
                        increments by a constant. If the normalized
                        increments are a fixed increment, the constant
                        is unity. If the normalized increments have a
                        Gaussian distribution, the constant is
                        $\approx 0.8$ depending on the accuracy of of
                        ``fit'' to a Gaussian distribution.}.}
                    \begin{tabular}{|c|c|} \hline
                        Mean of the               & rms \\
                        absolute value            & \hspace{0.01in} \\ \hline\hline
                        {\datafractionabsmean}    & {\datafractionrms} \\ \hline
                    \end{tabular}
                    \label{\SETLABEL:ABS}
                \end{center}
            \end{table}
        \end{small}

    \subsection{{\market}, $\chi^2$ values of the increments}
        \label{\SETLABEL:XSQ}

        The data in table~\ref{\SETLABEL:XSQT} is condensed from
        Section~\ref{\SETLABELREF:NH}.

        \begin{small}
            \begin{table}[ht]
                \begin{center}
                    \caption[{\market}, $\chi^2$ values of
                        the increments]{{\market}, $\chi^2$ values of
                        the increments. In principle, if the
                        distribution of the normalized increments is a
                        Gaussian distribution, the $\chi^2$ value will
                        be significantly less than the critical
                        value.}
                    \begin{tabular}{|c|c|} \hline
                        $\chi^2$      & Critical Value \\ \hline\hline
                        {\chisquared} & {\critical} \\ \hline
                    \end{tabular}
                    \label{\SETLABEL:XSQT}
                \end{center}
            \end{table}
        \end{small}

    \subsection{{\market}, time series data, empirical and simulated}
        \label{\SETLABEL:SIM}

        The data in table~\ref{\SETLABEL:ES} is condensed from
        Section~\ref{\SETLABELREF:TSUNFAIRBROWNIAN}.

        \begin{small}
            \begin{table}[ht]
                \begin{center}
                    \caption[{\market}, time series data, empirical
                        and simulated]{{\market}, time series data,
                        empirical and simulated, analysis of the
                        normalized increments.}
                    \begin{tabular}{|c|c|c|c|} \hline
                        \multicolumn{2}{|c|}{Empirical} & \multicolumn{2}{|c|}{Simulated}\\ \hline
                        Mean                & Standard              & Mean               & Standard \\
                        \hspace{0.01in}     & deviation             & \hspace{0.01in}    & deviation \\ \hline\hline
                        {\datafractionmean} & {\datafractionstddev} & {\tsunfairbrownianfractionmean} & {\tsunfairbrownianfractionstddev} \\ \hline
                    \end{tabular}
                    \label{\SETLABEL:ES}
                \end{center}
            \end{table}
        \end{small}

    \subsection{{\market}, number of participating companies}
        \label{\SETLABEL:QNC}

        The data in table~\ref{\SETLABEL:NC} is condensed from
        Section~\ref{\SETLABELREF:QNC}.

        \begin{small}
            \begin{table}[ht]
                \begin{center}
                    \caption[{\market}, number of participating
                        companies] {{\market}, number of participating
                        companies.}
                    \begin{tabular}{|c|c|} \hline
                        Number & Shannon probability\\ \hline
                        {\ncompanies} & {\pncompanies}\\ \hline
                    \end{tabular}
                    \label{\SETLABEL:NC}
                \end{center}
            \end{table}
        \end{small}

    \subsection{{\market}, Shannon probability optimizations}
        \label{\SETLABEL:SPO}

        The data in table~\ref{\SETLABEL:SP} is condensed from
        Section~\ref{\SETLABELREF:QNC}.

        \begin{small}
            \begin{table}[ht]
                \begin{center}
                    \caption[{\market}, Shannon probability
                         optimizations] {{\market}, Shannon
                         probability optimization.}
                    \begin{tabular}{|c|c|} \hline
                        optimize capital growth & optimize market growth\\ \hline
                        {\avgrms} & {\pncompanies}\\ \hline
                    \end{tabular}
                    \label{\SETLABEL:SP}
                \end{center}
            \end{table}
        \end{small}

% Local Variables:
% TeX-parse-self: t
% TeX-auto-save: t
% TeX-master: "fractal.tex"
% End:


    \renewcommand{\market}{North American Semiconductor Market}
    \renewcommand{\directory}{../markets/semiconductors.namerica}
    \renewcommand{\datafractionmean}{0.008052}
\renewcommand{\datafractionmeanbits}{0.011570}
\renewcommand{\datafractionmeanq}{0.002684}
\renewcommand{\datafractionmeanbitsq}{0.003867}
\renewcommand{\datafractionstddev}{0.038579}
\renewcommand{\datafractionrms}{0.039311}
\renewcommand{\avgrms}{0.602414}
\renewcommand{\ncompanies}{5.210454}
\renewcommand{\pncompanies}{0.544866}
\renewcommand{\datafractionabsmean}{0.029745}
\renewcommand{\datafractionabsstddev}{0.025769}
\renewcommand{\datafractionconstant}{0.010041}
\renewcommand{\datafractionconstantbits}{0.014414}
\renewcommand{\datafractionconstantq}{0.003347}
\renewcommand{\datafractionconstantbitsq}{0.004821}
\renewcommand{\datafractionslope}{-0.000021}
\renewcommand{\datafractionabsconstant}{0.035145}
\renewcommand{\datafractionabsslope}{-0.000057}
\renewcommand{\hurstall}{0.659558}
\renewcommand{\hurstlow}{0.707509}
\renewcommand{\hurstlowtwo}{1.415018}
\renewcommand{\hurstlowhundred}{70.750900}
\renewcommand{\hcalcall}{0.184942}
\renewcommand{\hcalclow}{0.102042}
\renewcommand{\shannonmax}{0.604167}
\renewcommand{\twoponemax}{0.208334}
\renewcommand{\logreturns}{0.010456}
\renewcommand{\twologreturns}{1.007274}
\renewcommand{\twologreturnshundred}{0.727387}
\renewcommand{\oneoverlogreturns}{95.638868}
\renewcommand{\pmax}{0.602094}
\renewcommand{\twopminusone}{0.204188}
\renewcommand{\rmsp}{0.008027}
\renewcommand{\twopx}{0.208583}
\renewcommand{\sigmap}{0.008047}
\renewcommand{\tsunfairbrownianfractionmean}{0.007862}
\renewcommand{\tsunfairbrownianfractionstddev}{0.038619}
\renewcommand{\shannonlogreturns}{0.560125}
\renewcommand{\shannonlogreturnshundred}{56.012500}
\renewcommand{\twopone}{0.120250}
\renewcommand{\twoponehundred}{12.025000}
\renewcommand{\hundredtwoponehundred}{87.975000}
\renewcommand{\hundredshannonlogreturnshundred}{43.987500}
\renewcommand{\datatslsqepbits}{0.007623}
\renewcommand{\thurstall}{0.633980}
\renewcommand{\thurstlow}{0.710108}
\renewcommand{\thurstlowtwo}{1.420216}
\renewcommand{\thurstlowhundred}{71.010800}
\renewcommand{\thcalcall}{0.247886}
\renewcommand{\thcalclow}{0.171737}
\renewcommand{\chisquared}{2.862000}
\renewcommand{\critical}{42.557000}

    \renewcommand{\timescale}{quarter}
    \subidx{market}{\market}
    \idx{\market}

    \section{\market}

        \renewcommand{\SETLABEL}{\LABPRE:NASM}
        \renewcommand{\SETLABELQ}{\LABPRE:NASMQ}
        \label{\SETLABEL}

        \idx{Semiconductor Industry Association}
        For the analysis, the data was in the directory
        {\directory}\footnote{Data from the Semiconductor Industry
        Association, 1979---1994, by {\timescale}s, in millions of
        dollars, US.}.

        The data in this section is presented in
        Section~\ref{\SETLABELREF}.

        %
% -----------------------------------------------------------------------------
%
% A license is hereby granted to reproduce this software source code and
% to create executable versions from this source code for personal,
% non-commercial use.  The copyright notice included with the software
% must be maintained in all copies produced.
%
% THIS PROGRAM IS PROVIDED "AS IS". THE AUTHOR PROVIDES NO WARRANTIES
% WHATSOEVER, EXPRESSED OR IMPLIED, INCLUDING WARRANTIES OF
% MERCHANTABILITY, TITLE, OR FITNESS FOR ANY PARTICULAR PURPOSE.  THE
% AUTHOR DOES NOT WARRANT THAT USE OF THIS PROGRAM DOES NOT INFRINGE THE
% INTELLECTUAL PROPERTY RIGHTS OF ANY THIRD PARTY IN ANY COUNTRY.
%
% Copyright (c) 1994-2006, John Conover, All Rights Reserved.
%
% Comments and/or bug reports should be addressed to:
%
%     john@email.johncon.com (John Conover)
%
% -----------------------------------------------------------------------------
%
% Revision: \RCSRevision \\
% Revision Time: \RCSTime UMT \\
% Revision Date: \RCSDate \\
% Revision Id: \RCSId \\
% Revision File: \RCSLog \\
\RCS $Revision: 0.0 $
\RCS $Date: 2006/01/20 04:38:13 $
\RCS $Id: tables.tex,v 0.0 2006/01/20 04:38:13 john Exp $
% $Log: tables.tex,v $
% Revision 0.0  2006/01/20 04:38:13  john
% Initial version
%
%
    \subsection{{\market}, normalized increments}
        \label{\SETLABEL:TSA}

        The data in table~\ref{\SETLABEL:INC} is condensed from
        Section~\ref{\SETLABELREF:TSA}.

        \begin{small}
            \begin{table}[ht]
                \begin{center}
                    \caption[{\market}, normalized increments]
                        {{\market}, normalized increments.}
                    \begin{tabular}{|c|c|c|c|c|c|c|c|c|c|} \hline
                        \multicolumn{5}{|c|}{Normalized}                                                                                  & \multicolumn{5}{|c|}{Normalized Absolute Value}\\ \hline
                        Mean                & Standard              & rms                & \multicolumn{2}{|c|}{Least Squares}            & Mean                   & Standard                 & rms                & \multicolumn{2}{|c|}{Least Squares} \\ \cline{4-5}\cline{9-10}
                        \hspace{0.01in}     & deviation             & \hspace{0.01in}    & Constant                & Slope                & \hspace{0.01in}        & deviation                & \hspace{0.01in}    & Constant                   & Slope \\ \hline\hline
                        {\datafractionmean} & {\datafractionstddev} & {\datafractionrms} & {\datafractionconstant} & {\datafractionslope} & {\datafractionabsmean} & {\datafractionabsstddev} & {\datafractionrms} & {\datafractionabsconstant} & {\datafractionabsslope} \\ \hline
                    \end{tabular}
                    \label{\SETLABEL:INC}
                \end{center}
            \end{table}
        \end{small}

    \subsection{{\market}, Logarithmic Returns, in Bits}
        \label{\SETLABEL:LR}

        The data in table~\ref{\SETLABEL:RET} is condensed from
        Section~\ref{\SETLABELREF:FS}.

        \begin{small}
            \begin{table}[ht]
                \begin{center}
                    \caption[{\market}, Logarithmic Returns, in
                        Bits]{{\market}, Logarithmic Returns, in Bits.}
                    \begin{tabular}{|c|c|c|c|} \hline
                        \multicolumn{2}{|c|}{Calculated from Table~\ref{\SETLABEL:INC}} & \multicolumn{2}{|c|}{From program:}\\ \hline
                        Mean                    & Least squares                       & {\it tslsq}\/              & {\it tslogreturns}\/ \\ \hline\hline
                        {\datafractionmeanbits} & {\datafractionconstantbits} & {\datatslsqepbits} & {\logreturns} \\ \hline
                    \end{tabular}
                    \label{\SETLABEL:RET}
                \end{center}
            \end{table}
        \end{small}

    \subsection{{\market}, Shannon probabilities}
        \label{\SETLABEL:MAXSHANNON}

        The data in table~\ref{\SETLABEL:SHANNON} is condensed from
        sections~\ref{\SETLABELREF:FS}
        and~\ref{\SETLABELREF:MAXSHANNON}.

        \begin{small}
            \begin{table}[ht]
                \begin{center}
                    \caption[{\market}, Shannon
                        probabilities]{{\market}, Shannon
                        probabilities.}
                    \begin{tabular}{|c|c|c|c|} \hline
                        \multicolumn{3}{|c|}{Maximum} & \multicolumn{1}{|c|}{Operational}\\ \hline
                        Fraction of         & $\frac{\frac{\mbox{\scriptsize{mean}}}{\mbox{\scriptsize{rms}}} + 1}{2}$ & \multicolumn{2}{|c|}{From program:}\\ \cline{3-4}
                        positive increments & \hspace{0.01in}                                                          & {\it tsshannonmax}\/    & {\it tsshannon}\/ \\ \hline\hline
                        {\pmax}             & {\avgrms}                                                                & {\shannonmax}   & {\shannonlogreturns} \\ \hline
                    \end{tabular}
                    \label{\SETLABEL:SHANNON}
                \end{center}
            \end{table}
        \end{small}

    \subsection{{\market}, Logistic Analysis}
        \label{\SETLABEL:LAA}

        The data in table~\ref{\SETLABEL:LA} is condensed from
        Section~\ref{\SETLABELREF:LA}\footnote{Note that there are
        numerical stability issues with the methodology used to derive
        the constants---if the non-linear term, $b$, was greater than
        zero, it was set to zero to produce the graphs in
        Section~\ref{\SETLABELREF:LA}.}.

        \begin{small}
            \begin{table}[ht]
                \begin{center}
                    \caption[{\market}, Logistic Analysis.]
                        {{\market}, Logistic Analysis, $x_t = x_{t - 1}\left(a + b \cdot x_{t - 1}\right)$.}
                    \begin{tabular}{|c|c|} \hline
                        $a$ & $b$\\ \hline\hline
                        {\datafractionconstant} & {\datafractionslope}\\ \hline
                    \end{tabular}
                    \label{\SETLABEL:LA}
                \end{center}
            \end{table}
        \end{small}

    \subsection{{\market}, Hurst Coefficients and H  Parameters}
        \label{\SETLABEL:HCHP}

        The data in table~\ref{\SETLABEL:H} is condensed from
        Section~\ref{\SETLABELREF:H}.

        \begin{small}
            \begin{table}[ht]
                \begin{center}
                    \caption[{\market}, Hurst Coefficients and H
                        Parameters]{{\market}, Hurst Coefficients and
                        H Parameters.}
                    \begin{tabular}{|c|c|c|c|} \hline
                        \multicolumn{2}{|c|}{Hurst Coefficients} & \multicolumn{2}{|c|}{H Parameters}\\ \hline
                        Near term   & Far term    & Near term   & Far term \\ \hline\hline
                        {\thurstlow} & {\thurstall} & {\thcalclow} & {\thcalcall} \\ \hline
                    \end{tabular}
                    \label{\SETLABEL:H}
                \end{center}
            \end{table}
        \end{small}

        \begin{small}
            \begin{table}[ht]
                \begin{center}
                    \caption[{\market}, Hurst Coefficients and H
                        Parameters]{{\market}, Hurst Coefficients and
                        H Parameters, as a Derivative.}
                    \begin{tabular}{|c|c|c|c|} \hline
                        \multicolumn{2}{|c|}{Hurst Coefficients} & \multicolumn{2}{|c|}{H Parameters}\\ \hline
                        Near term    & Far term     & Near term    & Far term \\ \hline\hline
                        {\hurstlow} & {\hurstall} & {\hcalclow} & {\hcalcall} \\ \hline
                    \end{tabular}
                    \label{\SETLABEL:TH}
                \end{center}
            \end{table}
        \end{small}

    \subsection{{\market}, verification of the increments}
        \label{\SETLABEL:VI1}

        The data in table~\ref{\SETLABEL:COMP} is condensed from
        Section~\ref{\SETLABELREF:QVA}.

        \begin{small}
            \begin{table}[ht]
                \begin{center}
                    \caption[{\market}, verification of
                        the increments]{{\market}, verification the of
                        the increments, the mean, $\sigma$ is the
                        standard deviation from
                        table~\ref{\SETLABEL:INC},
                        {\datafractionstddev}, and $P$ is the maximum
                        Shannon probability from
                        table~\ref{\SETLABEL:SHANNON}, {\pmax}. In
                        principle, the values should equate.}
                    \begin{tabular}{|c|c|c|} \hline
                        Mean                & $\mbox{rms} (2P - 1)$ & $\frac{{\sigma}(2P - 1)}{2\sqrt{P(P - 1)}} $ \\ \hline\hline
                        {\datafractionmean} & {\rmsp}               & {\sigmap} \\ \hline
                    \end{tabular}
                    \label{\SETLABEL:COMP}
                \end{center}
            \end{table}
        \end{small}

    \subsection{{\market}, verification of the increments}
        \label{\SETLABEL:VI2}

        The data in table~\ref{\SETLABEL:ABS} is condensed from
        Section~\ref{\SETLABELREF:QVA}.

        \begin{small}
            \begin{table}[ht]
                \begin{center}
                    \caption[{\market}, verification of
                        the increments]{{\market}, verification the of
                        increments. In principle, the mean of the
                        absolute value of the increments and the root
                        mean square of the increments should
                        equate\footnote{The absolute value of the
                        normalized increments, when averaged, is
                        related to the root mean square of the
                        increments by a constant. If the normalized
                        increments are a fixed increment, the constant
                        is unity. If the normalized increments have a
                        Gaussian distribution, the constant is
                        $\approx 0.8$ depending on the accuracy of of
                        ``fit'' to a Gaussian distribution.}.}
                    \begin{tabular}{|c|c|} \hline
                        Mean of the               & rms \\
                        absolute value            & \hspace{0.01in} \\ \hline\hline
                        {\datafractionabsmean}    & {\datafractionrms} \\ \hline
                    \end{tabular}
                    \label{\SETLABEL:ABS}
                \end{center}
            \end{table}
        \end{small}

    \subsection{{\market}, $\chi^2$ values of the increments}
        \label{\SETLABEL:XSQ}

        The data in table~\ref{\SETLABEL:XSQT} is condensed from
        Section~\ref{\SETLABELREF:NH}.

        \begin{small}
            \begin{table}[ht]
                \begin{center}
                    \caption[{\market}, $\chi^2$ values of
                        the increments]{{\market}, $\chi^2$ values of
                        the increments. In principle, if the
                        distribution of the normalized increments is a
                        Gaussian distribution, the $\chi^2$ value will
                        be significantly less than the critical
                        value.}
                    \begin{tabular}{|c|c|} \hline
                        $\chi^2$      & Critical Value \\ \hline\hline
                        {\chisquared} & {\critical} \\ \hline
                    \end{tabular}
                    \label{\SETLABEL:XSQT}
                \end{center}
            \end{table}
        \end{small}

    \subsection{{\market}, time series data, empirical and simulated}
        \label{\SETLABEL:SIM}

        The data in table~\ref{\SETLABEL:ES} is condensed from
        Section~\ref{\SETLABELREF:TSUNFAIRBROWNIAN}.

        \begin{small}
            \begin{table}[ht]
                \begin{center}
                    \caption[{\market}, time series data, empirical
                        and simulated]{{\market}, time series data,
                        empirical and simulated, analysis of the
                        normalized increments.}
                    \begin{tabular}{|c|c|c|c|} \hline
                        \multicolumn{2}{|c|}{Empirical} & \multicolumn{2}{|c|}{Simulated}\\ \hline
                        Mean                & Standard              & Mean               & Standard \\
                        \hspace{0.01in}     & deviation             & \hspace{0.01in}    & deviation \\ \hline\hline
                        {\datafractionmean} & {\datafractionstddev} & {\tsunfairbrownianfractionmean} & {\tsunfairbrownianfractionstddev} \\ \hline
                    \end{tabular}
                    \label{\SETLABEL:ES}
                \end{center}
            \end{table}
        \end{small}

    \subsection{{\market}, number of participating companies}
        \label{\SETLABEL:QNC}

        The data in table~\ref{\SETLABEL:NC} is condensed from
        Section~\ref{\SETLABELREF:QNC}.

        \begin{small}
            \begin{table}[ht]
                \begin{center}
                    \caption[{\market}, number of participating
                        companies] {{\market}, number of participating
                        companies.}
                    \begin{tabular}{|c|c|} \hline
                        Number & Shannon probability\\ \hline
                        {\ncompanies} & {\pncompanies}\\ \hline
                    \end{tabular}
                    \label{\SETLABEL:NC}
                \end{center}
            \end{table}
        \end{small}

    \subsection{{\market}, Shannon probability optimizations}
        \label{\SETLABEL:SPO}

        The data in table~\ref{\SETLABEL:SP} is condensed from
        Section~\ref{\SETLABELREF:QNC}.

        \begin{small}
            \begin{table}[ht]
                \begin{center}
                    \caption[{\market}, Shannon probability
                         optimizations] {{\market}, Shannon
                         probability optimization.}
                    \begin{tabular}{|c|c|} \hline
                        optimize capital growth & optimize market growth\\ \hline
                        {\avgrms} & {\pncompanies}\\ \hline
                    \end{tabular}
                    \label{\SETLABEL:SP}
                \end{center}
            \end{table}
        \end{small}

% Local Variables:
% TeX-parse-self: t
% TeX-auto-save: t
% TeX-master: "fractal.tex"
% End:


    \renewcommand{\market}{United States Electronic Component Shipments}
    \renewcommand{\directory}{../markets/electronic.components.shipments}
    \renewcommand{\datafractionmean}{0.008052}
\renewcommand{\datafractionmeanbits}{0.011570}
\renewcommand{\datafractionmeanq}{0.002684}
\renewcommand{\datafractionmeanbitsq}{0.003867}
\renewcommand{\datafractionstddev}{0.038579}
\renewcommand{\datafractionrms}{0.039311}
\renewcommand{\avgrms}{0.602414}
\renewcommand{\ncompanies}{5.210454}
\renewcommand{\pncompanies}{0.544866}
\renewcommand{\datafractionabsmean}{0.029745}
\renewcommand{\datafractionabsstddev}{0.025769}
\renewcommand{\datafractionconstant}{0.010041}
\renewcommand{\datafractionconstantbits}{0.014414}
\renewcommand{\datafractionconstantq}{0.003347}
\renewcommand{\datafractionconstantbitsq}{0.004821}
\renewcommand{\datafractionslope}{-0.000021}
\renewcommand{\datafractionabsconstant}{0.035145}
\renewcommand{\datafractionabsslope}{-0.000057}
\renewcommand{\hurstall}{0.659558}
\renewcommand{\hurstlow}{0.707509}
\renewcommand{\hurstlowtwo}{1.415018}
\renewcommand{\hurstlowhundred}{70.750900}
\renewcommand{\hcalcall}{0.184942}
\renewcommand{\hcalclow}{0.102042}
\renewcommand{\shannonmax}{0.604167}
\renewcommand{\twoponemax}{0.208334}
\renewcommand{\logreturns}{0.010456}
\renewcommand{\twologreturns}{1.007274}
\renewcommand{\twologreturnshundred}{0.727387}
\renewcommand{\oneoverlogreturns}{95.638868}
\renewcommand{\pmax}{0.602094}
\renewcommand{\twopminusone}{0.204188}
\renewcommand{\rmsp}{0.008027}
\renewcommand{\twopx}{0.208583}
\renewcommand{\sigmap}{0.008047}
\renewcommand{\tsunfairbrownianfractionmean}{0.007862}
\renewcommand{\tsunfairbrownianfractionstddev}{0.038619}
\renewcommand{\shannonlogreturns}{0.560125}
\renewcommand{\shannonlogreturnshundred}{56.012500}
\renewcommand{\twopone}{0.120250}
\renewcommand{\twoponehundred}{12.025000}
\renewcommand{\hundredtwoponehundred}{87.975000}
\renewcommand{\hundredshannonlogreturnshundred}{43.987500}
\renewcommand{\datatslsqepbits}{0.007623}
\renewcommand{\thurstall}{0.633980}
\renewcommand{\thurstlow}{0.710108}
\renewcommand{\thurstlowtwo}{1.420216}
\renewcommand{\thurstlowhundred}{71.010800}
\renewcommand{\thcalcall}{0.247886}
\renewcommand{\thcalclow}{0.171737}
\renewcommand{\chisquared}{2.862000}
\renewcommand{\critical}{42.557000}

    \renewcommand{\timescale}{month}
    \subidx{market}{\market}
    \idx{\market}

    \section{\market}

        \renewcommand{\SETLABEL}{\LABPRE:NAECS}
        \renewcommand{\SETLABELQ}{\LABPRE:NAECSQ}
        \label{\SETLABEL}

        \idx{United States Department of Commerce}
        For the analysis, the data was in the directory
        {\directory}\footnote{Data from the United States Department
        of Commerce, 1979---1994, by {\timescale}s, in millions of
        dollars, US.}.

        The data in this section is presented in
        Section~\ref{\SETLABELREF}.

        %
% -----------------------------------------------------------------------------
%
% A license is hereby granted to reproduce this software source code and
% to create executable versions from this source code for personal,
% non-commercial use.  The copyright notice included with the software
% must be maintained in all copies produced.
%
% THIS PROGRAM IS PROVIDED "AS IS". THE AUTHOR PROVIDES NO WARRANTIES
% WHATSOEVER, EXPRESSED OR IMPLIED, INCLUDING WARRANTIES OF
% MERCHANTABILITY, TITLE, OR FITNESS FOR ANY PARTICULAR PURPOSE.  THE
% AUTHOR DOES NOT WARRANT THAT USE OF THIS PROGRAM DOES NOT INFRINGE THE
% INTELLECTUAL PROPERTY RIGHTS OF ANY THIRD PARTY IN ANY COUNTRY.
%
% Copyright (c) 1994-2006, John Conover, All Rights Reserved.
%
% Comments and/or bug reports should be addressed to:
%
%     john@email.johncon.com (John Conover)
%
% -----------------------------------------------------------------------------
%
% Revision: \RCSRevision \\
% Revision Time: \RCSTime UMT \\
% Revision Date: \RCSDate \\
% Revision Id: \RCSId \\
% Revision File: \RCSLog \\
\RCS $Revision: 0.0 $
\RCS $Date: 2006/01/20 04:38:13 $
\RCS $Id: tables.tex,v 0.0 2006/01/20 04:38:13 john Exp $
% $Log: tables.tex,v $
% Revision 0.0  2006/01/20 04:38:13  john
% Initial version
%
%
    \subsection{{\market}, normalized increments}
        \label{\SETLABEL:TSA}

        The data in table~\ref{\SETLABEL:INC} is condensed from
        Section~\ref{\SETLABELREF:TSA}.

        \begin{small}
            \begin{table}[ht]
                \begin{center}
                    \caption[{\market}, normalized increments]
                        {{\market}, normalized increments.}
                    \begin{tabular}{|c|c|c|c|c|c|c|c|c|c|} \hline
                        \multicolumn{5}{|c|}{Normalized}                                                                                  & \multicolumn{5}{|c|}{Normalized Absolute Value}\\ \hline
                        Mean                & Standard              & rms                & \multicolumn{2}{|c|}{Least Squares}            & Mean                   & Standard                 & rms                & \multicolumn{2}{|c|}{Least Squares} \\ \cline{4-5}\cline{9-10}
                        \hspace{0.01in}     & deviation             & \hspace{0.01in}    & Constant                & Slope                & \hspace{0.01in}        & deviation                & \hspace{0.01in}    & Constant                   & Slope \\ \hline\hline
                        {\datafractionmean} & {\datafractionstddev} & {\datafractionrms} & {\datafractionconstant} & {\datafractionslope} & {\datafractionabsmean} & {\datafractionabsstddev} & {\datafractionrms} & {\datafractionabsconstant} & {\datafractionabsslope} \\ \hline
                    \end{tabular}
                    \label{\SETLABEL:INC}
                \end{center}
            \end{table}
        \end{small}

    \subsection{{\market}, Logarithmic Returns, in Bits}
        \label{\SETLABEL:LR}

        The data in table~\ref{\SETLABEL:RET} is condensed from
        Section~\ref{\SETLABELREF:FS}.

        \begin{small}
            \begin{table}[ht]
                \begin{center}
                    \caption[{\market}, Logarithmic Returns, in
                        Bits]{{\market}, Logarithmic Returns, in Bits.}
                    \begin{tabular}{|c|c|c|c|} \hline
                        \multicolumn{2}{|c|}{Calculated from Table~\ref{\SETLABEL:INC}} & \multicolumn{2}{|c|}{From program:}\\ \hline
                        Mean                    & Least squares                       & {\it tslsq}\/              & {\it tslogreturns}\/ \\ \hline\hline
                        {\datafractionmeanbits} & {\datafractionconstantbits} & {\datatslsqepbits} & {\logreturns} \\ \hline
                    \end{tabular}
                    \label{\SETLABEL:RET}
                \end{center}
            \end{table}
        \end{small}

    \subsection{{\market}, Shannon probabilities}
        \label{\SETLABEL:MAXSHANNON}

        The data in table~\ref{\SETLABEL:SHANNON} is condensed from
        sections~\ref{\SETLABELREF:FS}
        and~\ref{\SETLABELREF:MAXSHANNON}.

        \begin{small}
            \begin{table}[ht]
                \begin{center}
                    \caption[{\market}, Shannon
                        probabilities]{{\market}, Shannon
                        probabilities.}
                    \begin{tabular}{|c|c|c|c|} \hline
                        \multicolumn{3}{|c|}{Maximum} & \multicolumn{1}{|c|}{Operational}\\ \hline
                        Fraction of         & $\frac{\frac{\mbox{\scriptsize{mean}}}{\mbox{\scriptsize{rms}}} + 1}{2}$ & \multicolumn{2}{|c|}{From program:}\\ \cline{3-4}
                        positive increments & \hspace{0.01in}                                                          & {\it tsshannonmax}\/    & {\it tsshannon}\/ \\ \hline\hline
                        {\pmax}             & {\avgrms}                                                                & {\shannonmax}   & {\shannonlogreturns} \\ \hline
                    \end{tabular}
                    \label{\SETLABEL:SHANNON}
                \end{center}
            \end{table}
        \end{small}

    \subsection{{\market}, Logistic Analysis}
        \label{\SETLABEL:LAA}

        The data in table~\ref{\SETLABEL:LA} is condensed from
        Section~\ref{\SETLABELREF:LA}\footnote{Note that there are
        numerical stability issues with the methodology used to derive
        the constants---if the non-linear term, $b$, was greater than
        zero, it was set to zero to produce the graphs in
        Section~\ref{\SETLABELREF:LA}.}.

        \begin{small}
            \begin{table}[ht]
                \begin{center}
                    \caption[{\market}, Logistic Analysis.]
                        {{\market}, Logistic Analysis, $x_t = x_{t - 1}\left(a + b \cdot x_{t - 1}\right)$.}
                    \begin{tabular}{|c|c|} \hline
                        $a$ & $b$\\ \hline\hline
                        {\datafractionconstant} & {\datafractionslope}\\ \hline
                    \end{tabular}
                    \label{\SETLABEL:LA}
                \end{center}
            \end{table}
        \end{small}

    \subsection{{\market}, Hurst Coefficients and H  Parameters}
        \label{\SETLABEL:HCHP}

        The data in table~\ref{\SETLABEL:H} is condensed from
        Section~\ref{\SETLABELREF:H}.

        \begin{small}
            \begin{table}[ht]
                \begin{center}
                    \caption[{\market}, Hurst Coefficients and H
                        Parameters]{{\market}, Hurst Coefficients and
                        H Parameters.}
                    \begin{tabular}{|c|c|c|c|} \hline
                        \multicolumn{2}{|c|}{Hurst Coefficients} & \multicolumn{2}{|c|}{H Parameters}\\ \hline
                        Near term   & Far term    & Near term   & Far term \\ \hline\hline
                        {\thurstlow} & {\thurstall} & {\thcalclow} & {\thcalcall} \\ \hline
                    \end{tabular}
                    \label{\SETLABEL:H}
                \end{center}
            \end{table}
        \end{small}

        \begin{small}
            \begin{table}[ht]
                \begin{center}
                    \caption[{\market}, Hurst Coefficients and H
                        Parameters]{{\market}, Hurst Coefficients and
                        H Parameters, as a Derivative.}
                    \begin{tabular}{|c|c|c|c|} \hline
                        \multicolumn{2}{|c|}{Hurst Coefficients} & \multicolumn{2}{|c|}{H Parameters}\\ \hline
                        Near term    & Far term     & Near term    & Far term \\ \hline\hline
                        {\hurstlow} & {\hurstall} & {\hcalclow} & {\hcalcall} \\ \hline
                    \end{tabular}
                    \label{\SETLABEL:TH}
                \end{center}
            \end{table}
        \end{small}

    \subsection{{\market}, verification of the increments}
        \label{\SETLABEL:VI1}

        The data in table~\ref{\SETLABEL:COMP} is condensed from
        Section~\ref{\SETLABELREF:QVA}.

        \begin{small}
            \begin{table}[ht]
                \begin{center}
                    \caption[{\market}, verification of
                        the increments]{{\market}, verification the of
                        the increments, the mean, $\sigma$ is the
                        standard deviation from
                        table~\ref{\SETLABEL:INC},
                        {\datafractionstddev}, and $P$ is the maximum
                        Shannon probability from
                        table~\ref{\SETLABEL:SHANNON}, {\pmax}. In
                        principle, the values should equate.}
                    \begin{tabular}{|c|c|c|} \hline
                        Mean                & $\mbox{rms} (2P - 1)$ & $\frac{{\sigma}(2P - 1)}{2\sqrt{P(P - 1)}} $ \\ \hline\hline
                        {\datafractionmean} & {\rmsp}               & {\sigmap} \\ \hline
                    \end{tabular}
                    \label{\SETLABEL:COMP}
                \end{center}
            \end{table}
        \end{small}

    \subsection{{\market}, verification of the increments}
        \label{\SETLABEL:VI2}

        The data in table~\ref{\SETLABEL:ABS} is condensed from
        Section~\ref{\SETLABELREF:QVA}.

        \begin{small}
            \begin{table}[ht]
                \begin{center}
                    \caption[{\market}, verification of
                        the increments]{{\market}, verification the of
                        increments. In principle, the mean of the
                        absolute value of the increments and the root
                        mean square of the increments should
                        equate\footnote{The absolute value of the
                        normalized increments, when averaged, is
                        related to the root mean square of the
                        increments by a constant. If the normalized
                        increments are a fixed increment, the constant
                        is unity. If the normalized increments have a
                        Gaussian distribution, the constant is
                        $\approx 0.8$ depending on the accuracy of of
                        ``fit'' to a Gaussian distribution.}.}
                    \begin{tabular}{|c|c|} \hline
                        Mean of the               & rms \\
                        absolute value            & \hspace{0.01in} \\ \hline\hline
                        {\datafractionabsmean}    & {\datafractionrms} \\ \hline
                    \end{tabular}
                    \label{\SETLABEL:ABS}
                \end{center}
            \end{table}
        \end{small}

    \subsection{{\market}, $\chi^2$ values of the increments}
        \label{\SETLABEL:XSQ}

        The data in table~\ref{\SETLABEL:XSQT} is condensed from
        Section~\ref{\SETLABELREF:NH}.

        \begin{small}
            \begin{table}[ht]
                \begin{center}
                    \caption[{\market}, $\chi^2$ values of
                        the increments]{{\market}, $\chi^2$ values of
                        the increments. In principle, if the
                        distribution of the normalized increments is a
                        Gaussian distribution, the $\chi^2$ value will
                        be significantly less than the critical
                        value.}
                    \begin{tabular}{|c|c|} \hline
                        $\chi^2$      & Critical Value \\ \hline\hline
                        {\chisquared} & {\critical} \\ \hline
                    \end{tabular}
                    \label{\SETLABEL:XSQT}
                \end{center}
            \end{table}
        \end{small}

    \subsection{{\market}, time series data, empirical and simulated}
        \label{\SETLABEL:SIM}

        The data in table~\ref{\SETLABEL:ES} is condensed from
        Section~\ref{\SETLABELREF:TSUNFAIRBROWNIAN}.

        \begin{small}
            \begin{table}[ht]
                \begin{center}
                    \caption[{\market}, time series data, empirical
                        and simulated]{{\market}, time series data,
                        empirical and simulated, analysis of the
                        normalized increments.}
                    \begin{tabular}{|c|c|c|c|} \hline
                        \multicolumn{2}{|c|}{Empirical} & \multicolumn{2}{|c|}{Simulated}\\ \hline
                        Mean                & Standard              & Mean               & Standard \\
                        \hspace{0.01in}     & deviation             & \hspace{0.01in}    & deviation \\ \hline\hline
                        {\datafractionmean} & {\datafractionstddev} & {\tsunfairbrownianfractionmean} & {\tsunfairbrownianfractionstddev} \\ \hline
                    \end{tabular}
                    \label{\SETLABEL:ES}
                \end{center}
            \end{table}
        \end{small}

    \subsection{{\market}, number of participating companies}
        \label{\SETLABEL:QNC}

        The data in table~\ref{\SETLABEL:NC} is condensed from
        Section~\ref{\SETLABELREF:QNC}.

        \begin{small}
            \begin{table}[ht]
                \begin{center}
                    \caption[{\market}, number of participating
                        companies] {{\market}, number of participating
                        companies.}
                    \begin{tabular}{|c|c|} \hline
                        Number & Shannon probability\\ \hline
                        {\ncompanies} & {\pncompanies}\\ \hline
                    \end{tabular}
                    \label{\SETLABEL:NC}
                \end{center}
            \end{table}
        \end{small}

    \subsection{{\market}, Shannon probability optimizations}
        \label{\SETLABEL:SPO}

        The data in table~\ref{\SETLABEL:SP} is condensed from
        Section~\ref{\SETLABELREF:QNC}.

        \begin{small}
            \begin{table}[ht]
                \begin{center}
                    \caption[{\market}, Shannon probability
                         optimizations] {{\market}, Shannon
                         probability optimization.}
                    \begin{tabular}{|c|c|} \hline
                        optimize capital growth & optimize market growth\\ \hline
                        {\avgrms} & {\pncompanies}\\ \hline
                    \end{tabular}
                    \label{\SETLABEL:SP}
                \end{center}
            \end{table}
        \end{small}

% Local Variables:
% TeX-parse-self: t
% TeX-auto-save: t
% TeX-master: "fractal.tex"
% End:


    \renewcommand{\market}{United States Electronic Component Production}
    \renewcommand{\directory}{../markets/electronic.components.production}
    \renewcommand{\datafractionmean}{0.008052}
\renewcommand{\datafractionmeanbits}{0.011570}
\renewcommand{\datafractionmeanq}{0.002684}
\renewcommand{\datafractionmeanbitsq}{0.003867}
\renewcommand{\datafractionstddev}{0.038579}
\renewcommand{\datafractionrms}{0.039311}
\renewcommand{\avgrms}{0.602414}
\renewcommand{\ncompanies}{5.210454}
\renewcommand{\pncompanies}{0.544866}
\renewcommand{\datafractionabsmean}{0.029745}
\renewcommand{\datafractionabsstddev}{0.025769}
\renewcommand{\datafractionconstant}{0.010041}
\renewcommand{\datafractionconstantbits}{0.014414}
\renewcommand{\datafractionconstantq}{0.003347}
\renewcommand{\datafractionconstantbitsq}{0.004821}
\renewcommand{\datafractionslope}{-0.000021}
\renewcommand{\datafractionabsconstant}{0.035145}
\renewcommand{\datafractionabsslope}{-0.000057}
\renewcommand{\hurstall}{0.659558}
\renewcommand{\hurstlow}{0.707509}
\renewcommand{\hurstlowtwo}{1.415018}
\renewcommand{\hurstlowhundred}{70.750900}
\renewcommand{\hcalcall}{0.184942}
\renewcommand{\hcalclow}{0.102042}
\renewcommand{\shannonmax}{0.604167}
\renewcommand{\twoponemax}{0.208334}
\renewcommand{\logreturns}{0.010456}
\renewcommand{\twologreturns}{1.007274}
\renewcommand{\twologreturnshundred}{0.727387}
\renewcommand{\oneoverlogreturns}{95.638868}
\renewcommand{\pmax}{0.602094}
\renewcommand{\twopminusone}{0.204188}
\renewcommand{\rmsp}{0.008027}
\renewcommand{\twopx}{0.208583}
\renewcommand{\sigmap}{0.008047}
\renewcommand{\tsunfairbrownianfractionmean}{0.007862}
\renewcommand{\tsunfairbrownianfractionstddev}{0.038619}
\renewcommand{\shannonlogreturns}{0.560125}
\renewcommand{\shannonlogreturnshundred}{56.012500}
\renewcommand{\twopone}{0.120250}
\renewcommand{\twoponehundred}{12.025000}
\renewcommand{\hundredtwoponehundred}{87.975000}
\renewcommand{\hundredshannonlogreturnshundred}{43.987500}
\renewcommand{\datatslsqepbits}{0.007623}
\renewcommand{\thurstall}{0.633980}
\renewcommand{\thurstlow}{0.710108}
\renewcommand{\thurstlowtwo}{1.420216}
\renewcommand{\thurstlowhundred}{71.010800}
\renewcommand{\thcalcall}{0.247886}
\renewcommand{\thcalclow}{0.171737}
\renewcommand{\chisquared}{2.862000}
\renewcommand{\critical}{42.557000}

    \renewcommand{\timescale}{month}
    \subidx{market}{\market}
    \idx{\market}

    \section{\market}

        \renewcommand{\SETLABEL}{\LABPRE:NAECP}
        \renewcommand{\SETLABELQ}{\LABPRE:NAECPQ}
        \label{\SETLABEL}

        \idx{United States Department of Commerce}
        For the analysis, the data was in the directory
        {\directory}\footnote{Data from the United States Department
        of Commerce, 1980---1994, by {\timescale}s, as an index, 1987
        = 100.}.

        The data in this section is presented in
        Section~\ref{\SETLABELREF}.

        %
% -----------------------------------------------------------------------------
%
% A license is hereby granted to reproduce this software source code and
% to create executable versions from this source code for personal,
% non-commercial use.  The copyright notice included with the software
% must be maintained in all copies produced.
%
% THIS PROGRAM IS PROVIDED "AS IS". THE AUTHOR PROVIDES NO WARRANTIES
% WHATSOEVER, EXPRESSED OR IMPLIED, INCLUDING WARRANTIES OF
% MERCHANTABILITY, TITLE, OR FITNESS FOR ANY PARTICULAR PURPOSE.  THE
% AUTHOR DOES NOT WARRANT THAT USE OF THIS PROGRAM DOES NOT INFRINGE THE
% INTELLECTUAL PROPERTY RIGHTS OF ANY THIRD PARTY IN ANY COUNTRY.
%
% Copyright (c) 1994-2006, John Conover, All Rights Reserved.
%
% Comments and/or bug reports should be addressed to:
%
%     john@email.johncon.com (John Conover)
%
% -----------------------------------------------------------------------------
%
% Revision: \RCSRevision \\
% Revision Time: \RCSTime UMT \\
% Revision Date: \RCSDate \\
% Revision Id: \RCSId \\
% Revision File: \RCSLog \\
\RCS $Revision: 0.0 $
\RCS $Date: 2006/01/20 04:38:13 $
\RCS $Id: tables.tex,v 0.0 2006/01/20 04:38:13 john Exp $
% $Log: tables.tex,v $
% Revision 0.0  2006/01/20 04:38:13  john
% Initial version
%
%
    \subsection{{\market}, normalized increments}
        \label{\SETLABEL:TSA}

        The data in table~\ref{\SETLABEL:INC} is condensed from
        Section~\ref{\SETLABELREF:TSA}.

        \begin{small}
            \begin{table}[ht]
                \begin{center}
                    \caption[{\market}, normalized increments]
                        {{\market}, normalized increments.}
                    \begin{tabular}{|c|c|c|c|c|c|c|c|c|c|} \hline
                        \multicolumn{5}{|c|}{Normalized}                                                                                  & \multicolumn{5}{|c|}{Normalized Absolute Value}\\ \hline
                        Mean                & Standard              & rms                & \multicolumn{2}{|c|}{Least Squares}            & Mean                   & Standard                 & rms                & \multicolumn{2}{|c|}{Least Squares} \\ \cline{4-5}\cline{9-10}
                        \hspace{0.01in}     & deviation             & \hspace{0.01in}    & Constant                & Slope                & \hspace{0.01in}        & deviation                & \hspace{0.01in}    & Constant                   & Slope \\ \hline\hline
                        {\datafractionmean} & {\datafractionstddev} & {\datafractionrms} & {\datafractionconstant} & {\datafractionslope} & {\datafractionabsmean} & {\datafractionabsstddev} & {\datafractionrms} & {\datafractionabsconstant} & {\datafractionabsslope} \\ \hline
                    \end{tabular}
                    \label{\SETLABEL:INC}
                \end{center}
            \end{table}
        \end{small}

    \subsection{{\market}, Logarithmic Returns, in Bits}
        \label{\SETLABEL:LR}

        The data in table~\ref{\SETLABEL:RET} is condensed from
        Section~\ref{\SETLABELREF:FS}.

        \begin{small}
            \begin{table}[ht]
                \begin{center}
                    \caption[{\market}, Logarithmic Returns, in
                        Bits]{{\market}, Logarithmic Returns, in Bits.}
                    \begin{tabular}{|c|c|c|c|} \hline
                        \multicolumn{2}{|c|}{Calculated from Table~\ref{\SETLABEL:INC}} & \multicolumn{2}{|c|}{From program:}\\ \hline
                        Mean                    & Least squares                       & {\it tslsq}\/              & {\it tslogreturns}\/ \\ \hline\hline
                        {\datafractionmeanbits} & {\datafractionconstantbits} & {\datatslsqepbits} & {\logreturns} \\ \hline
                    \end{tabular}
                    \label{\SETLABEL:RET}
                \end{center}
            \end{table}
        \end{small}

    \subsection{{\market}, Shannon probabilities}
        \label{\SETLABEL:MAXSHANNON}

        The data in table~\ref{\SETLABEL:SHANNON} is condensed from
        sections~\ref{\SETLABELREF:FS}
        and~\ref{\SETLABELREF:MAXSHANNON}.

        \begin{small}
            \begin{table}[ht]
                \begin{center}
                    \caption[{\market}, Shannon
                        probabilities]{{\market}, Shannon
                        probabilities.}
                    \begin{tabular}{|c|c|c|c|} \hline
                        \multicolumn{3}{|c|}{Maximum} & \multicolumn{1}{|c|}{Operational}\\ \hline
                        Fraction of         & $\frac{\frac{\mbox{\scriptsize{mean}}}{\mbox{\scriptsize{rms}}} + 1}{2}$ & \multicolumn{2}{|c|}{From program:}\\ \cline{3-4}
                        positive increments & \hspace{0.01in}                                                          & {\it tsshannonmax}\/    & {\it tsshannon}\/ \\ \hline\hline
                        {\pmax}             & {\avgrms}                                                                & {\shannonmax}   & {\shannonlogreturns} \\ \hline
                    \end{tabular}
                    \label{\SETLABEL:SHANNON}
                \end{center}
            \end{table}
        \end{small}

    \subsection{{\market}, Logistic Analysis}
        \label{\SETLABEL:LAA}

        The data in table~\ref{\SETLABEL:LA} is condensed from
        Section~\ref{\SETLABELREF:LA}\footnote{Note that there are
        numerical stability issues with the methodology used to derive
        the constants---if the non-linear term, $b$, was greater than
        zero, it was set to zero to produce the graphs in
        Section~\ref{\SETLABELREF:LA}.}.

        \begin{small}
            \begin{table}[ht]
                \begin{center}
                    \caption[{\market}, Logistic Analysis.]
                        {{\market}, Logistic Analysis, $x_t = x_{t - 1}\left(a + b \cdot x_{t - 1}\right)$.}
                    \begin{tabular}{|c|c|} \hline
                        $a$ & $b$\\ \hline\hline
                        {\datafractionconstant} & {\datafractionslope}\\ \hline
                    \end{tabular}
                    \label{\SETLABEL:LA}
                \end{center}
            \end{table}
        \end{small}

    \subsection{{\market}, Hurst Coefficients and H  Parameters}
        \label{\SETLABEL:HCHP}

        The data in table~\ref{\SETLABEL:H} is condensed from
        Section~\ref{\SETLABELREF:H}.

        \begin{small}
            \begin{table}[ht]
                \begin{center}
                    \caption[{\market}, Hurst Coefficients and H
                        Parameters]{{\market}, Hurst Coefficients and
                        H Parameters.}
                    \begin{tabular}{|c|c|c|c|} \hline
                        \multicolumn{2}{|c|}{Hurst Coefficients} & \multicolumn{2}{|c|}{H Parameters}\\ \hline
                        Near term   & Far term    & Near term   & Far term \\ \hline\hline
                        {\thurstlow} & {\thurstall} & {\thcalclow} & {\thcalcall} \\ \hline
                    \end{tabular}
                    \label{\SETLABEL:H}
                \end{center}
            \end{table}
        \end{small}

        \begin{small}
            \begin{table}[ht]
                \begin{center}
                    \caption[{\market}, Hurst Coefficients and H
                        Parameters]{{\market}, Hurst Coefficients and
                        H Parameters, as a Derivative.}
                    \begin{tabular}{|c|c|c|c|} \hline
                        \multicolumn{2}{|c|}{Hurst Coefficients} & \multicolumn{2}{|c|}{H Parameters}\\ \hline
                        Near term    & Far term     & Near term    & Far term \\ \hline\hline
                        {\hurstlow} & {\hurstall} & {\hcalclow} & {\hcalcall} \\ \hline
                    \end{tabular}
                    \label{\SETLABEL:TH}
                \end{center}
            \end{table}
        \end{small}

    \subsection{{\market}, verification of the increments}
        \label{\SETLABEL:VI1}

        The data in table~\ref{\SETLABEL:COMP} is condensed from
        Section~\ref{\SETLABELREF:QVA}.

        \begin{small}
            \begin{table}[ht]
                \begin{center}
                    \caption[{\market}, verification of
                        the increments]{{\market}, verification the of
                        the increments, the mean, $\sigma$ is the
                        standard deviation from
                        table~\ref{\SETLABEL:INC},
                        {\datafractionstddev}, and $P$ is the maximum
                        Shannon probability from
                        table~\ref{\SETLABEL:SHANNON}, {\pmax}. In
                        principle, the values should equate.}
                    \begin{tabular}{|c|c|c|} \hline
                        Mean                & $\mbox{rms} (2P - 1)$ & $\frac{{\sigma}(2P - 1)}{2\sqrt{P(P - 1)}} $ \\ \hline\hline
                        {\datafractionmean} & {\rmsp}               & {\sigmap} \\ \hline
                    \end{tabular}
                    \label{\SETLABEL:COMP}
                \end{center}
            \end{table}
        \end{small}

    \subsection{{\market}, verification of the increments}
        \label{\SETLABEL:VI2}

        The data in table~\ref{\SETLABEL:ABS} is condensed from
        Section~\ref{\SETLABELREF:QVA}.

        \begin{small}
            \begin{table}[ht]
                \begin{center}
                    \caption[{\market}, verification of
                        the increments]{{\market}, verification the of
                        increments. In principle, the mean of the
                        absolute value of the increments and the root
                        mean square of the increments should
                        equate\footnote{The absolute value of the
                        normalized increments, when averaged, is
                        related to the root mean square of the
                        increments by a constant. If the normalized
                        increments are a fixed increment, the constant
                        is unity. If the normalized increments have a
                        Gaussian distribution, the constant is
                        $\approx 0.8$ depending on the accuracy of of
                        ``fit'' to a Gaussian distribution.}.}
                    \begin{tabular}{|c|c|} \hline
                        Mean of the               & rms \\
                        absolute value            & \hspace{0.01in} \\ \hline\hline
                        {\datafractionabsmean}    & {\datafractionrms} \\ \hline
                    \end{tabular}
                    \label{\SETLABEL:ABS}
                \end{center}
            \end{table}
        \end{small}

    \subsection{{\market}, $\chi^2$ values of the increments}
        \label{\SETLABEL:XSQ}

        The data in table~\ref{\SETLABEL:XSQT} is condensed from
        Section~\ref{\SETLABELREF:NH}.

        \begin{small}
            \begin{table}[ht]
                \begin{center}
                    \caption[{\market}, $\chi^2$ values of
                        the increments]{{\market}, $\chi^2$ values of
                        the increments. In principle, if the
                        distribution of the normalized increments is a
                        Gaussian distribution, the $\chi^2$ value will
                        be significantly less than the critical
                        value.}
                    \begin{tabular}{|c|c|} \hline
                        $\chi^2$      & Critical Value \\ \hline\hline
                        {\chisquared} & {\critical} \\ \hline
                    \end{tabular}
                    \label{\SETLABEL:XSQT}
                \end{center}
            \end{table}
        \end{small}

    \subsection{{\market}, time series data, empirical and simulated}
        \label{\SETLABEL:SIM}

        The data in table~\ref{\SETLABEL:ES} is condensed from
        Section~\ref{\SETLABELREF:TSUNFAIRBROWNIAN}.

        \begin{small}
            \begin{table}[ht]
                \begin{center}
                    \caption[{\market}, time series data, empirical
                        and simulated]{{\market}, time series data,
                        empirical and simulated, analysis of the
                        normalized increments.}
                    \begin{tabular}{|c|c|c|c|} \hline
                        \multicolumn{2}{|c|}{Empirical} & \multicolumn{2}{|c|}{Simulated}\\ \hline
                        Mean                & Standard              & Mean               & Standard \\
                        \hspace{0.01in}     & deviation             & \hspace{0.01in}    & deviation \\ \hline\hline
                        {\datafractionmean} & {\datafractionstddev} & {\tsunfairbrownianfractionmean} & {\tsunfairbrownianfractionstddev} \\ \hline
                    \end{tabular}
                    \label{\SETLABEL:ES}
                \end{center}
            \end{table}
        \end{small}

    \subsection{{\market}, number of participating companies}
        \label{\SETLABEL:QNC}

        The data in table~\ref{\SETLABEL:NC} is condensed from
        Section~\ref{\SETLABELREF:QNC}.

        \begin{small}
            \begin{table}[ht]
                \begin{center}
                    \caption[{\market}, number of participating
                        companies] {{\market}, number of participating
                        companies.}
                    \begin{tabular}{|c|c|} \hline
                        Number & Shannon probability\\ \hline
                        {\ncompanies} & {\pncompanies}\\ \hline
                    \end{tabular}
                    \label{\SETLABEL:NC}
                \end{center}
            \end{table}
        \end{small}

    \subsection{{\market}, Shannon probability optimizations}
        \label{\SETLABEL:SPO}

        The data in table~\ref{\SETLABEL:SP} is condensed from
        Section~\ref{\SETLABELREF:QNC}.

        \begin{small}
            \begin{table}[ht]
                \begin{center}
                    \caption[{\market}, Shannon probability
                         optimizations] {{\market}, Shannon
                         probability optimization.}
                    \begin{tabular}{|c|c|} \hline
                        optimize capital growth & optimize market growth\\ \hline
                        {\avgrms} & {\pncompanies}\\ \hline
                    \end{tabular}
                    \label{\SETLABEL:SP}
                \end{center}
            \end{table}
        \end{small}

% Local Variables:
% TeX-parse-self: t
% TeX-auto-save: t
% TeX-master: "fractal.tex"
% End:


    \renewcommand{\market}{United States Electronics Market}
    \renewcommand{\directory}{../markets/electronics}
    \renewcommand{\datafractionmean}{0.008052}
\renewcommand{\datafractionmeanbits}{0.011570}
\renewcommand{\datafractionmeanq}{0.002684}
\renewcommand{\datafractionmeanbitsq}{0.003867}
\renewcommand{\datafractionstddev}{0.038579}
\renewcommand{\datafractionrms}{0.039311}
\renewcommand{\avgrms}{0.602414}
\renewcommand{\ncompanies}{5.210454}
\renewcommand{\pncompanies}{0.544866}
\renewcommand{\datafractionabsmean}{0.029745}
\renewcommand{\datafractionabsstddev}{0.025769}
\renewcommand{\datafractionconstant}{0.010041}
\renewcommand{\datafractionconstantbits}{0.014414}
\renewcommand{\datafractionconstantq}{0.003347}
\renewcommand{\datafractionconstantbitsq}{0.004821}
\renewcommand{\datafractionslope}{-0.000021}
\renewcommand{\datafractionabsconstant}{0.035145}
\renewcommand{\datafractionabsslope}{-0.000057}
\renewcommand{\hurstall}{0.659558}
\renewcommand{\hurstlow}{0.707509}
\renewcommand{\hurstlowtwo}{1.415018}
\renewcommand{\hurstlowhundred}{70.750900}
\renewcommand{\hcalcall}{0.184942}
\renewcommand{\hcalclow}{0.102042}
\renewcommand{\shannonmax}{0.604167}
\renewcommand{\twoponemax}{0.208334}
\renewcommand{\logreturns}{0.010456}
\renewcommand{\twologreturns}{1.007274}
\renewcommand{\twologreturnshundred}{0.727387}
\renewcommand{\oneoverlogreturns}{95.638868}
\renewcommand{\pmax}{0.602094}
\renewcommand{\twopminusone}{0.204188}
\renewcommand{\rmsp}{0.008027}
\renewcommand{\twopx}{0.208583}
\renewcommand{\sigmap}{0.008047}
\renewcommand{\tsunfairbrownianfractionmean}{0.007862}
\renewcommand{\tsunfairbrownianfractionstddev}{0.038619}
\renewcommand{\shannonlogreturns}{0.560125}
\renewcommand{\shannonlogreturnshundred}{56.012500}
\renewcommand{\twopone}{0.120250}
\renewcommand{\twoponehundred}{12.025000}
\renewcommand{\hundredtwoponehundred}{87.975000}
\renewcommand{\hundredshannonlogreturnshundred}{43.987500}
\renewcommand{\datatslsqepbits}{0.007623}
\renewcommand{\thurstall}{0.633980}
\renewcommand{\thurstlow}{0.710108}
\renewcommand{\thurstlowtwo}{1.420216}
\renewcommand{\thurstlowhundred}{71.010800}
\renewcommand{\thcalcall}{0.247886}
\renewcommand{\thcalclow}{0.171737}
\renewcommand{\chisquared}{2.862000}
\renewcommand{\critical}{42.557000}

    \renewcommand{\timescale}{month}
    \subidx{market}{\market}
    \idx{\market}

    \section{\market}

        \renewcommand{\SETLABEL}{\LABPRE:NAEM}
        \renewcommand{\SETLABELQ}{\LABPRE:NAEMQ}
        \label{\SETLABEL}

        \idx{United States Department of Commerce}
        For the analysis, the data was in the directory
        {\directory}\footnote{Data from the United States Department
        of Commerce, 1980---1994, by {\timescale}s, in millions of
        dollars, US.}.

        The data in this section is presented in
        Section~\ref{\SETLABELREF}.

        %
% -----------------------------------------------------------------------------
%
% A license is hereby granted to reproduce this software source code and
% to create executable versions from this source code for personal,
% non-commercial use.  The copyright notice included with the software
% must be maintained in all copies produced.
%
% THIS PROGRAM IS PROVIDED "AS IS". THE AUTHOR PROVIDES NO WARRANTIES
% WHATSOEVER, EXPRESSED OR IMPLIED, INCLUDING WARRANTIES OF
% MERCHANTABILITY, TITLE, OR FITNESS FOR ANY PARTICULAR PURPOSE.  THE
% AUTHOR DOES NOT WARRANT THAT USE OF THIS PROGRAM DOES NOT INFRINGE THE
% INTELLECTUAL PROPERTY RIGHTS OF ANY THIRD PARTY IN ANY COUNTRY.
%
% Copyright (c) 1994-2006, John Conover, All Rights Reserved.
%
% Comments and/or bug reports should be addressed to:
%
%     john@email.johncon.com (John Conover)
%
% -----------------------------------------------------------------------------
%
% Revision: \RCSRevision \\
% Revision Time: \RCSTime UMT \\
% Revision Date: \RCSDate \\
% Revision Id: \RCSId \\
% Revision File: \RCSLog \\
\RCS $Revision: 0.0 $
\RCS $Date: 2006/01/20 04:38:13 $
\RCS $Id: tables.tex,v 0.0 2006/01/20 04:38:13 john Exp $
% $Log: tables.tex,v $
% Revision 0.0  2006/01/20 04:38:13  john
% Initial version
%
%
    \subsection{{\market}, normalized increments}
        \label{\SETLABEL:TSA}

        The data in table~\ref{\SETLABEL:INC} is condensed from
        Section~\ref{\SETLABELREF:TSA}.

        \begin{small}
            \begin{table}[ht]
                \begin{center}
                    \caption[{\market}, normalized increments]
                        {{\market}, normalized increments.}
                    \begin{tabular}{|c|c|c|c|c|c|c|c|c|c|} \hline
                        \multicolumn{5}{|c|}{Normalized}                                                                                  & \multicolumn{5}{|c|}{Normalized Absolute Value}\\ \hline
                        Mean                & Standard              & rms                & \multicolumn{2}{|c|}{Least Squares}            & Mean                   & Standard                 & rms                & \multicolumn{2}{|c|}{Least Squares} \\ \cline{4-5}\cline{9-10}
                        \hspace{0.01in}     & deviation             & \hspace{0.01in}    & Constant                & Slope                & \hspace{0.01in}        & deviation                & \hspace{0.01in}    & Constant                   & Slope \\ \hline\hline
                        {\datafractionmean} & {\datafractionstddev} & {\datafractionrms} & {\datafractionconstant} & {\datafractionslope} & {\datafractionabsmean} & {\datafractionabsstddev} & {\datafractionrms} & {\datafractionabsconstant} & {\datafractionabsslope} \\ \hline
                    \end{tabular}
                    \label{\SETLABEL:INC}
                \end{center}
            \end{table}
        \end{small}

    \subsection{{\market}, Logarithmic Returns, in Bits}
        \label{\SETLABEL:LR}

        The data in table~\ref{\SETLABEL:RET} is condensed from
        Section~\ref{\SETLABELREF:FS}.

        \begin{small}
            \begin{table}[ht]
                \begin{center}
                    \caption[{\market}, Logarithmic Returns, in
                        Bits]{{\market}, Logarithmic Returns, in Bits.}
                    \begin{tabular}{|c|c|c|c|} \hline
                        \multicolumn{2}{|c|}{Calculated from Table~\ref{\SETLABEL:INC}} & \multicolumn{2}{|c|}{From program:}\\ \hline
                        Mean                    & Least squares                       & {\it tslsq}\/              & {\it tslogreturns}\/ \\ \hline\hline
                        {\datafractionmeanbits} & {\datafractionconstantbits} & {\datatslsqepbits} & {\logreturns} \\ \hline
                    \end{tabular}
                    \label{\SETLABEL:RET}
                \end{center}
            \end{table}
        \end{small}

    \subsection{{\market}, Shannon probabilities}
        \label{\SETLABEL:MAXSHANNON}

        The data in table~\ref{\SETLABEL:SHANNON} is condensed from
        sections~\ref{\SETLABELREF:FS}
        and~\ref{\SETLABELREF:MAXSHANNON}.

        \begin{small}
            \begin{table}[ht]
                \begin{center}
                    \caption[{\market}, Shannon
                        probabilities]{{\market}, Shannon
                        probabilities.}
                    \begin{tabular}{|c|c|c|c|} \hline
                        \multicolumn{3}{|c|}{Maximum} & \multicolumn{1}{|c|}{Operational}\\ \hline
                        Fraction of         & $\frac{\frac{\mbox{\scriptsize{mean}}}{\mbox{\scriptsize{rms}}} + 1}{2}$ & \multicolumn{2}{|c|}{From program:}\\ \cline{3-4}
                        positive increments & \hspace{0.01in}                                                          & {\it tsshannonmax}\/    & {\it tsshannon}\/ \\ \hline\hline
                        {\pmax}             & {\avgrms}                                                                & {\shannonmax}   & {\shannonlogreturns} \\ \hline
                    \end{tabular}
                    \label{\SETLABEL:SHANNON}
                \end{center}
            \end{table}
        \end{small}

    \subsection{{\market}, Logistic Analysis}
        \label{\SETLABEL:LAA}

        The data in table~\ref{\SETLABEL:LA} is condensed from
        Section~\ref{\SETLABELREF:LA}\footnote{Note that there are
        numerical stability issues with the methodology used to derive
        the constants---if the non-linear term, $b$, was greater than
        zero, it was set to zero to produce the graphs in
        Section~\ref{\SETLABELREF:LA}.}.

        \begin{small}
            \begin{table}[ht]
                \begin{center}
                    \caption[{\market}, Logistic Analysis.]
                        {{\market}, Logistic Analysis, $x_t = x_{t - 1}\left(a + b \cdot x_{t - 1}\right)$.}
                    \begin{tabular}{|c|c|} \hline
                        $a$ & $b$\\ \hline\hline
                        {\datafractionconstant} & {\datafractionslope}\\ \hline
                    \end{tabular}
                    \label{\SETLABEL:LA}
                \end{center}
            \end{table}
        \end{small}

    \subsection{{\market}, Hurst Coefficients and H  Parameters}
        \label{\SETLABEL:HCHP}

        The data in table~\ref{\SETLABEL:H} is condensed from
        Section~\ref{\SETLABELREF:H}.

        \begin{small}
            \begin{table}[ht]
                \begin{center}
                    \caption[{\market}, Hurst Coefficients and H
                        Parameters]{{\market}, Hurst Coefficients and
                        H Parameters.}
                    \begin{tabular}{|c|c|c|c|} \hline
                        \multicolumn{2}{|c|}{Hurst Coefficients} & \multicolumn{2}{|c|}{H Parameters}\\ \hline
                        Near term   & Far term    & Near term   & Far term \\ \hline\hline
                        {\thurstlow} & {\thurstall} & {\thcalclow} & {\thcalcall} \\ \hline
                    \end{tabular}
                    \label{\SETLABEL:H}
                \end{center}
            \end{table}
        \end{small}

        \begin{small}
            \begin{table}[ht]
                \begin{center}
                    \caption[{\market}, Hurst Coefficients and H
                        Parameters]{{\market}, Hurst Coefficients and
                        H Parameters, as a Derivative.}
                    \begin{tabular}{|c|c|c|c|} \hline
                        \multicolumn{2}{|c|}{Hurst Coefficients} & \multicolumn{2}{|c|}{H Parameters}\\ \hline
                        Near term    & Far term     & Near term    & Far term \\ \hline\hline
                        {\hurstlow} & {\hurstall} & {\hcalclow} & {\hcalcall} \\ \hline
                    \end{tabular}
                    \label{\SETLABEL:TH}
                \end{center}
            \end{table}
        \end{small}

    \subsection{{\market}, verification of the increments}
        \label{\SETLABEL:VI1}

        The data in table~\ref{\SETLABEL:COMP} is condensed from
        Section~\ref{\SETLABELREF:QVA}.

        \begin{small}
            \begin{table}[ht]
                \begin{center}
                    \caption[{\market}, verification of
                        the increments]{{\market}, verification the of
                        the increments, the mean, $\sigma$ is the
                        standard deviation from
                        table~\ref{\SETLABEL:INC},
                        {\datafractionstddev}, and $P$ is the maximum
                        Shannon probability from
                        table~\ref{\SETLABEL:SHANNON}, {\pmax}. In
                        principle, the values should equate.}
                    \begin{tabular}{|c|c|c|} \hline
                        Mean                & $\mbox{rms} (2P - 1)$ & $\frac{{\sigma}(2P - 1)}{2\sqrt{P(P - 1)}} $ \\ \hline\hline
                        {\datafractionmean} & {\rmsp}               & {\sigmap} \\ \hline
                    \end{tabular}
                    \label{\SETLABEL:COMP}
                \end{center}
            \end{table}
        \end{small}

    \subsection{{\market}, verification of the increments}
        \label{\SETLABEL:VI2}

        The data in table~\ref{\SETLABEL:ABS} is condensed from
        Section~\ref{\SETLABELREF:QVA}.

        \begin{small}
            \begin{table}[ht]
                \begin{center}
                    \caption[{\market}, verification of
                        the increments]{{\market}, verification the of
                        increments. In principle, the mean of the
                        absolute value of the increments and the root
                        mean square of the increments should
                        equate\footnote{The absolute value of the
                        normalized increments, when averaged, is
                        related to the root mean square of the
                        increments by a constant. If the normalized
                        increments are a fixed increment, the constant
                        is unity. If the normalized increments have a
                        Gaussian distribution, the constant is
                        $\approx 0.8$ depending on the accuracy of of
                        ``fit'' to a Gaussian distribution.}.}
                    \begin{tabular}{|c|c|} \hline
                        Mean of the               & rms \\
                        absolute value            & \hspace{0.01in} \\ \hline\hline
                        {\datafractionabsmean}    & {\datafractionrms} \\ \hline
                    \end{tabular}
                    \label{\SETLABEL:ABS}
                \end{center}
            \end{table}
        \end{small}

    \subsection{{\market}, $\chi^2$ values of the increments}
        \label{\SETLABEL:XSQ}

        The data in table~\ref{\SETLABEL:XSQT} is condensed from
        Section~\ref{\SETLABELREF:NH}.

        \begin{small}
            \begin{table}[ht]
                \begin{center}
                    \caption[{\market}, $\chi^2$ values of
                        the increments]{{\market}, $\chi^2$ values of
                        the increments. In principle, if the
                        distribution of the normalized increments is a
                        Gaussian distribution, the $\chi^2$ value will
                        be significantly less than the critical
                        value.}
                    \begin{tabular}{|c|c|} \hline
                        $\chi^2$      & Critical Value \\ \hline\hline
                        {\chisquared} & {\critical} \\ \hline
                    \end{tabular}
                    \label{\SETLABEL:XSQT}
                \end{center}
            \end{table}
        \end{small}

    \subsection{{\market}, time series data, empirical and simulated}
        \label{\SETLABEL:SIM}

        The data in table~\ref{\SETLABEL:ES} is condensed from
        Section~\ref{\SETLABELREF:TSUNFAIRBROWNIAN}.

        \begin{small}
            \begin{table}[ht]
                \begin{center}
                    \caption[{\market}, time series data, empirical
                        and simulated]{{\market}, time series data,
                        empirical and simulated, analysis of the
                        normalized increments.}
                    \begin{tabular}{|c|c|c|c|} \hline
                        \multicolumn{2}{|c|}{Empirical} & \multicolumn{2}{|c|}{Simulated}\\ \hline
                        Mean                & Standard              & Mean               & Standard \\
                        \hspace{0.01in}     & deviation             & \hspace{0.01in}    & deviation \\ \hline\hline
                        {\datafractionmean} & {\datafractionstddev} & {\tsunfairbrownianfractionmean} & {\tsunfairbrownianfractionstddev} \\ \hline
                    \end{tabular}
                    \label{\SETLABEL:ES}
                \end{center}
            \end{table}
        \end{small}

    \subsection{{\market}, number of participating companies}
        \label{\SETLABEL:QNC}

        The data in table~\ref{\SETLABEL:NC} is condensed from
        Section~\ref{\SETLABELREF:QNC}.

        \begin{small}
            \begin{table}[ht]
                \begin{center}
                    \caption[{\market}, number of participating
                        companies] {{\market}, number of participating
                        companies.}
                    \begin{tabular}{|c|c|} \hline
                        Number & Shannon probability\\ \hline
                        {\ncompanies} & {\pncompanies}\\ \hline
                    \end{tabular}
                    \label{\SETLABEL:NC}
                \end{center}
            \end{table}
        \end{small}

    \subsection{{\market}, Shannon probability optimizations}
        \label{\SETLABEL:SPO}

        The data in table~\ref{\SETLABEL:SP} is condensed from
        Section~\ref{\SETLABELREF:QNC}.

        \begin{small}
            \begin{table}[ht]
                \begin{center}
                    \caption[{\market}, Shannon probability
                         optimizations] {{\market}, Shannon
                         probability optimization.}
                    \begin{tabular}{|c|c|} \hline
                        optimize capital growth & optimize market growth\\ \hline
                        {\avgrms} & {\pncompanies}\\ \hline
                    \end{tabular}
                    \label{\SETLABEL:SP}
                \end{center}
            \end{table}
        \end{small}

% Local Variables:
% TeX-parse-self: t
% TeX-auto-save: t
% TeX-master: "fractal.tex"
% End:


    \renewcommand{\market}{United States Office Computer Market}
    \renewcommand{\directory}{../markets/computer.office}
    \renewcommand{\datafractionmean}{0.008052}
\renewcommand{\datafractionmeanbits}{0.011570}
\renewcommand{\datafractionmeanq}{0.002684}
\renewcommand{\datafractionmeanbitsq}{0.003867}
\renewcommand{\datafractionstddev}{0.038579}
\renewcommand{\datafractionrms}{0.039311}
\renewcommand{\avgrms}{0.602414}
\renewcommand{\ncompanies}{5.210454}
\renewcommand{\pncompanies}{0.544866}
\renewcommand{\datafractionabsmean}{0.029745}
\renewcommand{\datafractionabsstddev}{0.025769}
\renewcommand{\datafractionconstant}{0.010041}
\renewcommand{\datafractionconstantbits}{0.014414}
\renewcommand{\datafractionconstantq}{0.003347}
\renewcommand{\datafractionconstantbitsq}{0.004821}
\renewcommand{\datafractionslope}{-0.000021}
\renewcommand{\datafractionabsconstant}{0.035145}
\renewcommand{\datafractionabsslope}{-0.000057}
\renewcommand{\hurstall}{0.659558}
\renewcommand{\hurstlow}{0.707509}
\renewcommand{\hurstlowtwo}{1.415018}
\renewcommand{\hurstlowhundred}{70.750900}
\renewcommand{\hcalcall}{0.184942}
\renewcommand{\hcalclow}{0.102042}
\renewcommand{\shannonmax}{0.604167}
\renewcommand{\twoponemax}{0.208334}
\renewcommand{\logreturns}{0.010456}
\renewcommand{\twologreturns}{1.007274}
\renewcommand{\twologreturnshundred}{0.727387}
\renewcommand{\oneoverlogreturns}{95.638868}
\renewcommand{\pmax}{0.602094}
\renewcommand{\twopminusone}{0.204188}
\renewcommand{\rmsp}{0.008027}
\renewcommand{\twopx}{0.208583}
\renewcommand{\sigmap}{0.008047}
\renewcommand{\tsunfairbrownianfractionmean}{0.007862}
\renewcommand{\tsunfairbrownianfractionstddev}{0.038619}
\renewcommand{\shannonlogreturns}{0.560125}
\renewcommand{\shannonlogreturnshundred}{56.012500}
\renewcommand{\twopone}{0.120250}
\renewcommand{\twoponehundred}{12.025000}
\renewcommand{\hundredtwoponehundred}{87.975000}
\renewcommand{\hundredshannonlogreturnshundred}{43.987500}
\renewcommand{\datatslsqepbits}{0.007623}
\renewcommand{\thurstall}{0.633980}
\renewcommand{\thurstlow}{0.710108}
\renewcommand{\thurstlowtwo}{1.420216}
\renewcommand{\thurstlowhundred}{71.010800}
\renewcommand{\thcalcall}{0.247886}
\renewcommand{\thcalclow}{0.171737}
\renewcommand{\chisquared}{2.862000}
\renewcommand{\critical}{42.557000}

    \renewcommand{\timescale}{month}
    \subidx{market}{\market}
    \idx{\market}

    \section{\market}

        \renewcommand{\SETLABEL}{\LABPRE:NAOCM}
        \renewcommand{\SETLABELQ}{\LABPRE:NAOCMQ}
        \label{\SETLABEL}

        \idx{United States Department of Commerce}
        For the analysis, the data was in the directory
        {\directory}\footnote{Data from the United States Department
        of Commerce, 1982---1994, by {\timescale}s, as an index, 1987
        = 100.}.

        The data in this section is presented in
        Section~\ref{\SETLABELREF}.

        %
% -----------------------------------------------------------------------------
%
% A license is hereby granted to reproduce this software source code and
% to create executable versions from this source code for personal,
% non-commercial use.  The copyright notice included with the software
% must be maintained in all copies produced.
%
% THIS PROGRAM IS PROVIDED "AS IS". THE AUTHOR PROVIDES NO WARRANTIES
% WHATSOEVER, EXPRESSED OR IMPLIED, INCLUDING WARRANTIES OF
% MERCHANTABILITY, TITLE, OR FITNESS FOR ANY PARTICULAR PURPOSE.  THE
% AUTHOR DOES NOT WARRANT THAT USE OF THIS PROGRAM DOES NOT INFRINGE THE
% INTELLECTUAL PROPERTY RIGHTS OF ANY THIRD PARTY IN ANY COUNTRY.
%
% Copyright (c) 1994-2006, John Conover, All Rights Reserved.
%
% Comments and/or bug reports should be addressed to:
%
%     john@email.johncon.com (John Conover)
%
% -----------------------------------------------------------------------------
%
% Revision: \RCSRevision \\
% Revision Time: \RCSTime UMT \\
% Revision Date: \RCSDate \\
% Revision Id: \RCSId \\
% Revision File: \RCSLog \\
\RCS $Revision: 0.0 $
\RCS $Date: 2006/01/20 04:38:13 $
\RCS $Id: tables.tex,v 0.0 2006/01/20 04:38:13 john Exp $
% $Log: tables.tex,v $
% Revision 0.0  2006/01/20 04:38:13  john
% Initial version
%
%
    \subsection{{\market}, normalized increments}
        \label{\SETLABEL:TSA}

        The data in table~\ref{\SETLABEL:INC} is condensed from
        Section~\ref{\SETLABELREF:TSA}.

        \begin{small}
            \begin{table}[ht]
                \begin{center}
                    \caption[{\market}, normalized increments]
                        {{\market}, normalized increments.}
                    \begin{tabular}{|c|c|c|c|c|c|c|c|c|c|} \hline
                        \multicolumn{5}{|c|}{Normalized}                                                                                  & \multicolumn{5}{|c|}{Normalized Absolute Value}\\ \hline
                        Mean                & Standard              & rms                & \multicolumn{2}{|c|}{Least Squares}            & Mean                   & Standard                 & rms                & \multicolumn{2}{|c|}{Least Squares} \\ \cline{4-5}\cline{9-10}
                        \hspace{0.01in}     & deviation             & \hspace{0.01in}    & Constant                & Slope                & \hspace{0.01in}        & deviation                & \hspace{0.01in}    & Constant                   & Slope \\ \hline\hline
                        {\datafractionmean} & {\datafractionstddev} & {\datafractionrms} & {\datafractionconstant} & {\datafractionslope} & {\datafractionabsmean} & {\datafractionabsstddev} & {\datafractionrms} & {\datafractionabsconstant} & {\datafractionabsslope} \\ \hline
                    \end{tabular}
                    \label{\SETLABEL:INC}
                \end{center}
            \end{table}
        \end{small}

    \subsection{{\market}, Logarithmic Returns, in Bits}
        \label{\SETLABEL:LR}

        The data in table~\ref{\SETLABEL:RET} is condensed from
        Section~\ref{\SETLABELREF:FS}.

        \begin{small}
            \begin{table}[ht]
                \begin{center}
                    \caption[{\market}, Logarithmic Returns, in
                        Bits]{{\market}, Logarithmic Returns, in Bits.}
                    \begin{tabular}{|c|c|c|c|} \hline
                        \multicolumn{2}{|c|}{Calculated from Table~\ref{\SETLABEL:INC}} & \multicolumn{2}{|c|}{From program:}\\ \hline
                        Mean                    & Least squares                       & {\it tslsq}\/              & {\it tslogreturns}\/ \\ \hline\hline
                        {\datafractionmeanbits} & {\datafractionconstantbits} & {\datatslsqepbits} & {\logreturns} \\ \hline
                    \end{tabular}
                    \label{\SETLABEL:RET}
                \end{center}
            \end{table}
        \end{small}

    \subsection{{\market}, Shannon probabilities}
        \label{\SETLABEL:MAXSHANNON}

        The data in table~\ref{\SETLABEL:SHANNON} is condensed from
        sections~\ref{\SETLABELREF:FS}
        and~\ref{\SETLABELREF:MAXSHANNON}.

        \begin{small}
            \begin{table}[ht]
                \begin{center}
                    \caption[{\market}, Shannon
                        probabilities]{{\market}, Shannon
                        probabilities.}
                    \begin{tabular}{|c|c|c|c|} \hline
                        \multicolumn{3}{|c|}{Maximum} & \multicolumn{1}{|c|}{Operational}\\ \hline
                        Fraction of         & $\frac{\frac{\mbox{\scriptsize{mean}}}{\mbox{\scriptsize{rms}}} + 1}{2}$ & \multicolumn{2}{|c|}{From program:}\\ \cline{3-4}
                        positive increments & \hspace{0.01in}                                                          & {\it tsshannonmax}\/    & {\it tsshannon}\/ \\ \hline\hline
                        {\pmax}             & {\avgrms}                                                                & {\shannonmax}   & {\shannonlogreturns} \\ \hline
                    \end{tabular}
                    \label{\SETLABEL:SHANNON}
                \end{center}
            \end{table}
        \end{small}

    \subsection{{\market}, Logistic Analysis}
        \label{\SETLABEL:LAA}

        The data in table~\ref{\SETLABEL:LA} is condensed from
        Section~\ref{\SETLABELREF:LA}\footnote{Note that there are
        numerical stability issues with the methodology used to derive
        the constants---if the non-linear term, $b$, was greater than
        zero, it was set to zero to produce the graphs in
        Section~\ref{\SETLABELREF:LA}.}.

        \begin{small}
            \begin{table}[ht]
                \begin{center}
                    \caption[{\market}, Logistic Analysis.]
                        {{\market}, Logistic Analysis, $x_t = x_{t - 1}\left(a + b \cdot x_{t - 1}\right)$.}
                    \begin{tabular}{|c|c|} \hline
                        $a$ & $b$\\ \hline\hline
                        {\datafractionconstant} & {\datafractionslope}\\ \hline
                    \end{tabular}
                    \label{\SETLABEL:LA}
                \end{center}
            \end{table}
        \end{small}

    \subsection{{\market}, Hurst Coefficients and H  Parameters}
        \label{\SETLABEL:HCHP}

        The data in table~\ref{\SETLABEL:H} is condensed from
        Section~\ref{\SETLABELREF:H}.

        \begin{small}
            \begin{table}[ht]
                \begin{center}
                    \caption[{\market}, Hurst Coefficients and H
                        Parameters]{{\market}, Hurst Coefficients and
                        H Parameters.}
                    \begin{tabular}{|c|c|c|c|} \hline
                        \multicolumn{2}{|c|}{Hurst Coefficients} & \multicolumn{2}{|c|}{H Parameters}\\ \hline
                        Near term   & Far term    & Near term   & Far term \\ \hline\hline
                        {\thurstlow} & {\thurstall} & {\thcalclow} & {\thcalcall} \\ \hline
                    \end{tabular}
                    \label{\SETLABEL:H}
                \end{center}
            \end{table}
        \end{small}

        \begin{small}
            \begin{table}[ht]
                \begin{center}
                    \caption[{\market}, Hurst Coefficients and H
                        Parameters]{{\market}, Hurst Coefficients and
                        H Parameters, as a Derivative.}
                    \begin{tabular}{|c|c|c|c|} \hline
                        \multicolumn{2}{|c|}{Hurst Coefficients} & \multicolumn{2}{|c|}{H Parameters}\\ \hline
                        Near term    & Far term     & Near term    & Far term \\ \hline\hline
                        {\hurstlow} & {\hurstall} & {\hcalclow} & {\hcalcall} \\ \hline
                    \end{tabular}
                    \label{\SETLABEL:TH}
                \end{center}
            \end{table}
        \end{small}

    \subsection{{\market}, verification of the increments}
        \label{\SETLABEL:VI1}

        The data in table~\ref{\SETLABEL:COMP} is condensed from
        Section~\ref{\SETLABELREF:QVA}.

        \begin{small}
            \begin{table}[ht]
                \begin{center}
                    \caption[{\market}, verification of
                        the increments]{{\market}, verification the of
                        the increments, the mean, $\sigma$ is the
                        standard deviation from
                        table~\ref{\SETLABEL:INC},
                        {\datafractionstddev}, and $P$ is the maximum
                        Shannon probability from
                        table~\ref{\SETLABEL:SHANNON}, {\pmax}. In
                        principle, the values should equate.}
                    \begin{tabular}{|c|c|c|} \hline
                        Mean                & $\mbox{rms} (2P - 1)$ & $\frac{{\sigma}(2P - 1)}{2\sqrt{P(P - 1)}} $ \\ \hline\hline
                        {\datafractionmean} & {\rmsp}               & {\sigmap} \\ \hline
                    \end{tabular}
                    \label{\SETLABEL:COMP}
                \end{center}
            \end{table}
        \end{small}

    \subsection{{\market}, verification of the increments}
        \label{\SETLABEL:VI2}

        The data in table~\ref{\SETLABEL:ABS} is condensed from
        Section~\ref{\SETLABELREF:QVA}.

        \begin{small}
            \begin{table}[ht]
                \begin{center}
                    \caption[{\market}, verification of
                        the increments]{{\market}, verification the of
                        increments. In principle, the mean of the
                        absolute value of the increments and the root
                        mean square of the increments should
                        equate\footnote{The absolute value of the
                        normalized increments, when averaged, is
                        related to the root mean square of the
                        increments by a constant. If the normalized
                        increments are a fixed increment, the constant
                        is unity. If the normalized increments have a
                        Gaussian distribution, the constant is
                        $\approx 0.8$ depending on the accuracy of of
                        ``fit'' to a Gaussian distribution.}.}
                    \begin{tabular}{|c|c|} \hline
                        Mean of the               & rms \\
                        absolute value            & \hspace{0.01in} \\ \hline\hline
                        {\datafractionabsmean}    & {\datafractionrms} \\ \hline
                    \end{tabular}
                    \label{\SETLABEL:ABS}
                \end{center}
            \end{table}
        \end{small}

    \subsection{{\market}, $\chi^2$ values of the increments}
        \label{\SETLABEL:XSQ}

        The data in table~\ref{\SETLABEL:XSQT} is condensed from
        Section~\ref{\SETLABELREF:NH}.

        \begin{small}
            \begin{table}[ht]
                \begin{center}
                    \caption[{\market}, $\chi^2$ values of
                        the increments]{{\market}, $\chi^2$ values of
                        the increments. In principle, if the
                        distribution of the normalized increments is a
                        Gaussian distribution, the $\chi^2$ value will
                        be significantly less than the critical
                        value.}
                    \begin{tabular}{|c|c|} \hline
                        $\chi^2$      & Critical Value \\ \hline\hline
                        {\chisquared} & {\critical} \\ \hline
                    \end{tabular}
                    \label{\SETLABEL:XSQT}
                \end{center}
            \end{table}
        \end{small}

    \subsection{{\market}, time series data, empirical and simulated}
        \label{\SETLABEL:SIM}

        The data in table~\ref{\SETLABEL:ES} is condensed from
        Section~\ref{\SETLABELREF:TSUNFAIRBROWNIAN}.

        \begin{small}
            \begin{table}[ht]
                \begin{center}
                    \caption[{\market}, time series data, empirical
                        and simulated]{{\market}, time series data,
                        empirical and simulated, analysis of the
                        normalized increments.}
                    \begin{tabular}{|c|c|c|c|} \hline
                        \multicolumn{2}{|c|}{Empirical} & \multicolumn{2}{|c|}{Simulated}\\ \hline
                        Mean                & Standard              & Mean               & Standard \\
                        \hspace{0.01in}     & deviation             & \hspace{0.01in}    & deviation \\ \hline\hline
                        {\datafractionmean} & {\datafractionstddev} & {\tsunfairbrownianfractionmean} & {\tsunfairbrownianfractionstddev} \\ \hline
                    \end{tabular}
                    \label{\SETLABEL:ES}
                \end{center}
            \end{table}
        \end{small}

    \subsection{{\market}, number of participating companies}
        \label{\SETLABEL:QNC}

        The data in table~\ref{\SETLABEL:NC} is condensed from
        Section~\ref{\SETLABELREF:QNC}.

        \begin{small}
            \begin{table}[ht]
                \begin{center}
                    \caption[{\market}, number of participating
                        companies] {{\market}, number of participating
                        companies.}
                    \begin{tabular}{|c|c|} \hline
                        Number & Shannon probability\\ \hline
                        {\ncompanies} & {\pncompanies}\\ \hline
                    \end{tabular}
                    \label{\SETLABEL:NC}
                \end{center}
            \end{table}
        \end{small}

    \subsection{{\market}, Shannon probability optimizations}
        \label{\SETLABEL:SPO}

        The data in table~\ref{\SETLABEL:SP} is condensed from
        Section~\ref{\SETLABELREF:QNC}.

        \begin{small}
            \begin{table}[ht]
                \begin{center}
                    \caption[{\market}, Shannon probability
                         optimizations] {{\market}, Shannon
                         probability optimization.}
                    \begin{tabular}{|c|c|} \hline
                        optimize capital growth & optimize market growth\\ \hline
                        {\avgrms} & {\pncompanies}\\ \hline
                    \end{tabular}
                    \label{\SETLABEL:SP}
                \end{center}
            \end{table}
        \end{small}

% Local Variables:
% TeX-parse-self: t
% TeX-auto-save: t
% TeX-master: "fractal.tex"
% End:


    \renewcommand{\market}{United States Information Systems Market}
    \renewcommand{\directory}{../markets/information.systems}
    \renewcommand{\datafractionmean}{0.008052}
\renewcommand{\datafractionmeanbits}{0.011570}
\renewcommand{\datafractionmeanq}{0.002684}
\renewcommand{\datafractionmeanbitsq}{0.003867}
\renewcommand{\datafractionstddev}{0.038579}
\renewcommand{\datafractionrms}{0.039311}
\renewcommand{\avgrms}{0.602414}
\renewcommand{\ncompanies}{5.210454}
\renewcommand{\pncompanies}{0.544866}
\renewcommand{\datafractionabsmean}{0.029745}
\renewcommand{\datafractionabsstddev}{0.025769}
\renewcommand{\datafractionconstant}{0.010041}
\renewcommand{\datafractionconstantbits}{0.014414}
\renewcommand{\datafractionconstantq}{0.003347}
\renewcommand{\datafractionconstantbitsq}{0.004821}
\renewcommand{\datafractionslope}{-0.000021}
\renewcommand{\datafractionabsconstant}{0.035145}
\renewcommand{\datafractionabsslope}{-0.000057}
\renewcommand{\hurstall}{0.659558}
\renewcommand{\hurstlow}{0.707509}
\renewcommand{\hurstlowtwo}{1.415018}
\renewcommand{\hurstlowhundred}{70.750900}
\renewcommand{\hcalcall}{0.184942}
\renewcommand{\hcalclow}{0.102042}
\renewcommand{\shannonmax}{0.604167}
\renewcommand{\twoponemax}{0.208334}
\renewcommand{\logreturns}{0.010456}
\renewcommand{\twologreturns}{1.007274}
\renewcommand{\twologreturnshundred}{0.727387}
\renewcommand{\oneoverlogreturns}{95.638868}
\renewcommand{\pmax}{0.602094}
\renewcommand{\twopminusone}{0.204188}
\renewcommand{\rmsp}{0.008027}
\renewcommand{\twopx}{0.208583}
\renewcommand{\sigmap}{0.008047}
\renewcommand{\tsunfairbrownianfractionmean}{0.007862}
\renewcommand{\tsunfairbrownianfractionstddev}{0.038619}
\renewcommand{\shannonlogreturns}{0.560125}
\renewcommand{\shannonlogreturnshundred}{56.012500}
\renewcommand{\twopone}{0.120250}
\renewcommand{\twoponehundred}{12.025000}
\renewcommand{\hundredtwoponehundred}{87.975000}
\renewcommand{\hundredshannonlogreturnshundred}{43.987500}
\renewcommand{\datatslsqepbits}{0.007623}
\renewcommand{\thurstall}{0.633980}
\renewcommand{\thurstlow}{0.710108}
\renewcommand{\thurstlowtwo}{1.420216}
\renewcommand{\thurstlowhundred}{71.010800}
\renewcommand{\thcalcall}{0.247886}
\renewcommand{\thcalclow}{0.171737}
\renewcommand{\chisquared}{2.862000}
\renewcommand{\critical}{42.557000}

    \renewcommand{\timescale}{month}
    \subidx{market}{\market}
    \idx{\market}

    \section{\market}

        \renewcommand{\SETLABEL}{\LABPRE:NAISM}
        \renewcommand{\SETLABELQ}{\LABPRE:NAISMQ}
        \label{\SETLABEL}

        \idx{United States Department of Commerce}
        For the analysis, the data was in the directory
        {\directory}\footnote{Data from the United States Department
        of Commerce, 1979---1994, by {\timescale}s, in millions of
        dollars, US.}.

        The data in this section is presented in
        Section~\ref{\SETLABELREF}.

        %
% -----------------------------------------------------------------------------
%
% A license is hereby granted to reproduce this software source code and
% to create executable versions from this source code for personal,
% non-commercial use.  The copyright notice included with the software
% must be maintained in all copies produced.
%
% THIS PROGRAM IS PROVIDED "AS IS". THE AUTHOR PROVIDES NO WARRANTIES
% WHATSOEVER, EXPRESSED OR IMPLIED, INCLUDING WARRANTIES OF
% MERCHANTABILITY, TITLE, OR FITNESS FOR ANY PARTICULAR PURPOSE.  THE
% AUTHOR DOES NOT WARRANT THAT USE OF THIS PROGRAM DOES NOT INFRINGE THE
% INTELLECTUAL PROPERTY RIGHTS OF ANY THIRD PARTY IN ANY COUNTRY.
%
% Copyright (c) 1994-2006, John Conover, All Rights Reserved.
%
% Comments and/or bug reports should be addressed to:
%
%     john@email.johncon.com (John Conover)
%
% -----------------------------------------------------------------------------
%
% Revision: \RCSRevision \\
% Revision Time: \RCSTime UMT \\
% Revision Date: \RCSDate \\
% Revision Id: \RCSId \\
% Revision File: \RCSLog \\
\RCS $Revision: 0.0 $
\RCS $Date: 2006/01/20 04:38:13 $
\RCS $Id: tables.tex,v 0.0 2006/01/20 04:38:13 john Exp $
% $Log: tables.tex,v $
% Revision 0.0  2006/01/20 04:38:13  john
% Initial version
%
%
    \subsection{{\market}, normalized increments}
        \label{\SETLABEL:TSA}

        The data in table~\ref{\SETLABEL:INC} is condensed from
        Section~\ref{\SETLABELREF:TSA}.

        \begin{small}
            \begin{table}[ht]
                \begin{center}
                    \caption[{\market}, normalized increments]
                        {{\market}, normalized increments.}
                    \begin{tabular}{|c|c|c|c|c|c|c|c|c|c|} \hline
                        \multicolumn{5}{|c|}{Normalized}                                                                                  & \multicolumn{5}{|c|}{Normalized Absolute Value}\\ \hline
                        Mean                & Standard              & rms                & \multicolumn{2}{|c|}{Least Squares}            & Mean                   & Standard                 & rms                & \multicolumn{2}{|c|}{Least Squares} \\ \cline{4-5}\cline{9-10}
                        \hspace{0.01in}     & deviation             & \hspace{0.01in}    & Constant                & Slope                & \hspace{0.01in}        & deviation                & \hspace{0.01in}    & Constant                   & Slope \\ \hline\hline
                        {\datafractionmean} & {\datafractionstddev} & {\datafractionrms} & {\datafractionconstant} & {\datafractionslope} & {\datafractionabsmean} & {\datafractionabsstddev} & {\datafractionrms} & {\datafractionabsconstant} & {\datafractionabsslope} \\ \hline
                    \end{tabular}
                    \label{\SETLABEL:INC}
                \end{center}
            \end{table}
        \end{small}

    \subsection{{\market}, Logarithmic Returns, in Bits}
        \label{\SETLABEL:LR}

        The data in table~\ref{\SETLABEL:RET} is condensed from
        Section~\ref{\SETLABELREF:FS}.

        \begin{small}
            \begin{table}[ht]
                \begin{center}
                    \caption[{\market}, Logarithmic Returns, in
                        Bits]{{\market}, Logarithmic Returns, in Bits.}
                    \begin{tabular}{|c|c|c|c|} \hline
                        \multicolumn{2}{|c|}{Calculated from Table~\ref{\SETLABEL:INC}} & \multicolumn{2}{|c|}{From program:}\\ \hline
                        Mean                    & Least squares                       & {\it tslsq}\/              & {\it tslogreturns}\/ \\ \hline\hline
                        {\datafractionmeanbits} & {\datafractionconstantbits} & {\datatslsqepbits} & {\logreturns} \\ \hline
                    \end{tabular}
                    \label{\SETLABEL:RET}
                \end{center}
            \end{table}
        \end{small}

    \subsection{{\market}, Shannon probabilities}
        \label{\SETLABEL:MAXSHANNON}

        The data in table~\ref{\SETLABEL:SHANNON} is condensed from
        sections~\ref{\SETLABELREF:FS}
        and~\ref{\SETLABELREF:MAXSHANNON}.

        \begin{small}
            \begin{table}[ht]
                \begin{center}
                    \caption[{\market}, Shannon
                        probabilities]{{\market}, Shannon
                        probabilities.}
                    \begin{tabular}{|c|c|c|c|} \hline
                        \multicolumn{3}{|c|}{Maximum} & \multicolumn{1}{|c|}{Operational}\\ \hline
                        Fraction of         & $\frac{\frac{\mbox{\scriptsize{mean}}}{\mbox{\scriptsize{rms}}} + 1}{2}$ & \multicolumn{2}{|c|}{From program:}\\ \cline{3-4}
                        positive increments & \hspace{0.01in}                                                          & {\it tsshannonmax}\/    & {\it tsshannon}\/ \\ \hline\hline
                        {\pmax}             & {\avgrms}                                                                & {\shannonmax}   & {\shannonlogreturns} \\ \hline
                    \end{tabular}
                    \label{\SETLABEL:SHANNON}
                \end{center}
            \end{table}
        \end{small}

    \subsection{{\market}, Logistic Analysis}
        \label{\SETLABEL:LAA}

        The data in table~\ref{\SETLABEL:LA} is condensed from
        Section~\ref{\SETLABELREF:LA}\footnote{Note that there are
        numerical stability issues with the methodology used to derive
        the constants---if the non-linear term, $b$, was greater than
        zero, it was set to zero to produce the graphs in
        Section~\ref{\SETLABELREF:LA}.}.

        \begin{small}
            \begin{table}[ht]
                \begin{center}
                    \caption[{\market}, Logistic Analysis.]
                        {{\market}, Logistic Analysis, $x_t = x_{t - 1}\left(a + b \cdot x_{t - 1}\right)$.}
                    \begin{tabular}{|c|c|} \hline
                        $a$ & $b$\\ \hline\hline
                        {\datafractionconstant} & {\datafractionslope}\\ \hline
                    \end{tabular}
                    \label{\SETLABEL:LA}
                \end{center}
            \end{table}
        \end{small}

    \subsection{{\market}, Hurst Coefficients and H  Parameters}
        \label{\SETLABEL:HCHP}

        The data in table~\ref{\SETLABEL:H} is condensed from
        Section~\ref{\SETLABELREF:H}.

        \begin{small}
            \begin{table}[ht]
                \begin{center}
                    \caption[{\market}, Hurst Coefficients and H
                        Parameters]{{\market}, Hurst Coefficients and
                        H Parameters.}
                    \begin{tabular}{|c|c|c|c|} \hline
                        \multicolumn{2}{|c|}{Hurst Coefficients} & \multicolumn{2}{|c|}{H Parameters}\\ \hline
                        Near term   & Far term    & Near term   & Far term \\ \hline\hline
                        {\thurstlow} & {\thurstall} & {\thcalclow} & {\thcalcall} \\ \hline
                    \end{tabular}
                    \label{\SETLABEL:H}
                \end{center}
            \end{table}
        \end{small}

        \begin{small}
            \begin{table}[ht]
                \begin{center}
                    \caption[{\market}, Hurst Coefficients and H
                        Parameters]{{\market}, Hurst Coefficients and
                        H Parameters, as a Derivative.}
                    \begin{tabular}{|c|c|c|c|} \hline
                        \multicolumn{2}{|c|}{Hurst Coefficients} & \multicolumn{2}{|c|}{H Parameters}\\ \hline
                        Near term    & Far term     & Near term    & Far term \\ \hline\hline
                        {\hurstlow} & {\hurstall} & {\hcalclow} & {\hcalcall} \\ \hline
                    \end{tabular}
                    \label{\SETLABEL:TH}
                \end{center}
            \end{table}
        \end{small}

    \subsection{{\market}, verification of the increments}
        \label{\SETLABEL:VI1}

        The data in table~\ref{\SETLABEL:COMP} is condensed from
        Section~\ref{\SETLABELREF:QVA}.

        \begin{small}
            \begin{table}[ht]
                \begin{center}
                    \caption[{\market}, verification of
                        the increments]{{\market}, verification the of
                        the increments, the mean, $\sigma$ is the
                        standard deviation from
                        table~\ref{\SETLABEL:INC},
                        {\datafractionstddev}, and $P$ is the maximum
                        Shannon probability from
                        table~\ref{\SETLABEL:SHANNON}, {\pmax}. In
                        principle, the values should equate.}
                    \begin{tabular}{|c|c|c|} \hline
                        Mean                & $\mbox{rms} (2P - 1)$ & $\frac{{\sigma}(2P - 1)}{2\sqrt{P(P - 1)}} $ \\ \hline\hline
                        {\datafractionmean} & {\rmsp}               & {\sigmap} \\ \hline
                    \end{tabular}
                    \label{\SETLABEL:COMP}
                \end{center}
            \end{table}
        \end{small}

    \subsection{{\market}, verification of the increments}
        \label{\SETLABEL:VI2}

        The data in table~\ref{\SETLABEL:ABS} is condensed from
        Section~\ref{\SETLABELREF:QVA}.

        \begin{small}
            \begin{table}[ht]
                \begin{center}
                    \caption[{\market}, verification of
                        the increments]{{\market}, verification the of
                        increments. In principle, the mean of the
                        absolute value of the increments and the root
                        mean square of the increments should
                        equate\footnote{The absolute value of the
                        normalized increments, when averaged, is
                        related to the root mean square of the
                        increments by a constant. If the normalized
                        increments are a fixed increment, the constant
                        is unity. If the normalized increments have a
                        Gaussian distribution, the constant is
                        $\approx 0.8$ depending on the accuracy of of
                        ``fit'' to a Gaussian distribution.}.}
                    \begin{tabular}{|c|c|} \hline
                        Mean of the               & rms \\
                        absolute value            & \hspace{0.01in} \\ \hline\hline
                        {\datafractionabsmean}    & {\datafractionrms} \\ \hline
                    \end{tabular}
                    \label{\SETLABEL:ABS}
                \end{center}
            \end{table}
        \end{small}

    \subsection{{\market}, $\chi^2$ values of the increments}
        \label{\SETLABEL:XSQ}

        The data in table~\ref{\SETLABEL:XSQT} is condensed from
        Section~\ref{\SETLABELREF:NH}.

        \begin{small}
            \begin{table}[ht]
                \begin{center}
                    \caption[{\market}, $\chi^2$ values of
                        the increments]{{\market}, $\chi^2$ values of
                        the increments. In principle, if the
                        distribution of the normalized increments is a
                        Gaussian distribution, the $\chi^2$ value will
                        be significantly less than the critical
                        value.}
                    \begin{tabular}{|c|c|} \hline
                        $\chi^2$      & Critical Value \\ \hline\hline
                        {\chisquared} & {\critical} \\ \hline
                    \end{tabular}
                    \label{\SETLABEL:XSQT}
                \end{center}
            \end{table}
        \end{small}

    \subsection{{\market}, time series data, empirical and simulated}
        \label{\SETLABEL:SIM}

        The data in table~\ref{\SETLABEL:ES} is condensed from
        Section~\ref{\SETLABELREF:TSUNFAIRBROWNIAN}.

        \begin{small}
            \begin{table}[ht]
                \begin{center}
                    \caption[{\market}, time series data, empirical
                        and simulated]{{\market}, time series data,
                        empirical and simulated, analysis of the
                        normalized increments.}
                    \begin{tabular}{|c|c|c|c|} \hline
                        \multicolumn{2}{|c|}{Empirical} & \multicolumn{2}{|c|}{Simulated}\\ \hline
                        Mean                & Standard              & Mean               & Standard \\
                        \hspace{0.01in}     & deviation             & \hspace{0.01in}    & deviation \\ \hline\hline
                        {\datafractionmean} & {\datafractionstddev} & {\tsunfairbrownianfractionmean} & {\tsunfairbrownianfractionstddev} \\ \hline
                    \end{tabular}
                    \label{\SETLABEL:ES}
                \end{center}
            \end{table}
        \end{small}

    \subsection{{\market}, number of participating companies}
        \label{\SETLABEL:QNC}

        The data in table~\ref{\SETLABEL:NC} is condensed from
        Section~\ref{\SETLABELREF:QNC}.

        \begin{small}
            \begin{table}[ht]
                \begin{center}
                    \caption[{\market}, number of participating
                        companies] {{\market}, number of participating
                        companies.}
                    \begin{tabular}{|c|c|} \hline
                        Number & Shannon probability\\ \hline
                        {\ncompanies} & {\pncompanies}\\ \hline
                    \end{tabular}
                    \label{\SETLABEL:NC}
                \end{center}
            \end{table}
        \end{small}

    \subsection{{\market}, Shannon probability optimizations}
        \label{\SETLABEL:SPO}

        The data in table~\ref{\SETLABEL:SP} is condensed from
        Section~\ref{\SETLABELREF:QNC}.

        \begin{small}
            \begin{table}[ht]
                \begin{center}
                    \caption[{\market}, Shannon probability
                         optimizations] {{\market}, Shannon
                         probability optimization.}
                    \begin{tabular}{|c|c|} \hline
                        optimize capital growth & optimize market growth\\ \hline
                        {\avgrms} & {\pncompanies}\\ \hline
                    \end{tabular}
                    \label{\SETLABEL:SP}
                \end{center}
            \end{table}
        \end{small}

% Local Variables:
% TeX-parse-self: t
% TeX-auto-save: t
% TeX-master: "fractal.tex"
% End:


    \renewcommand{\market}{Dow Jones Average}
    \renewcommand{\directory}{../markets/dj}
    \renewcommand{\datafractionmean}{0.008052}
\renewcommand{\datafractionmeanbits}{0.011570}
\renewcommand{\datafractionmeanq}{0.002684}
\renewcommand{\datafractionmeanbitsq}{0.003867}
\renewcommand{\datafractionstddev}{0.038579}
\renewcommand{\datafractionrms}{0.039311}
\renewcommand{\avgrms}{0.602414}
\renewcommand{\ncompanies}{5.210454}
\renewcommand{\pncompanies}{0.544866}
\renewcommand{\datafractionabsmean}{0.029745}
\renewcommand{\datafractionabsstddev}{0.025769}
\renewcommand{\datafractionconstant}{0.010041}
\renewcommand{\datafractionconstantbits}{0.014414}
\renewcommand{\datafractionconstantq}{0.003347}
\renewcommand{\datafractionconstantbitsq}{0.004821}
\renewcommand{\datafractionslope}{-0.000021}
\renewcommand{\datafractionabsconstant}{0.035145}
\renewcommand{\datafractionabsslope}{-0.000057}
\renewcommand{\hurstall}{0.659558}
\renewcommand{\hurstlow}{0.707509}
\renewcommand{\hurstlowtwo}{1.415018}
\renewcommand{\hurstlowhundred}{70.750900}
\renewcommand{\hcalcall}{0.184942}
\renewcommand{\hcalclow}{0.102042}
\renewcommand{\shannonmax}{0.604167}
\renewcommand{\twoponemax}{0.208334}
\renewcommand{\logreturns}{0.010456}
\renewcommand{\twologreturns}{1.007274}
\renewcommand{\twologreturnshundred}{0.727387}
\renewcommand{\oneoverlogreturns}{95.638868}
\renewcommand{\pmax}{0.602094}
\renewcommand{\twopminusone}{0.204188}
\renewcommand{\rmsp}{0.008027}
\renewcommand{\twopx}{0.208583}
\renewcommand{\sigmap}{0.008047}
\renewcommand{\tsunfairbrownianfractionmean}{0.007862}
\renewcommand{\tsunfairbrownianfractionstddev}{0.038619}
\renewcommand{\shannonlogreturns}{0.560125}
\renewcommand{\shannonlogreturnshundred}{56.012500}
\renewcommand{\twopone}{0.120250}
\renewcommand{\twoponehundred}{12.025000}
\renewcommand{\hundredtwoponehundred}{87.975000}
\renewcommand{\hundredshannonlogreturnshundred}{43.987500}
\renewcommand{\datatslsqepbits}{0.007623}
\renewcommand{\thurstall}{0.633980}
\renewcommand{\thurstlow}{0.710108}
\renewcommand{\thurstlowtwo}{1.420216}
\renewcommand{\thurstlowhundred}{71.010800}
\renewcommand{\thcalcall}{0.247886}
\renewcommand{\thcalclow}{0.171737}
\renewcommand{\chisquared}{2.862000}
\renewcommand{\critical}{42.557000}

    \renewcommand{\timescale}{month}
    \subidx{market}{\market}
    \idx{\market}

    \section{\market}

        \renewcommand{\SETLABEL}{\LABPRE:DJA}
        \renewcommand{\SETLABELQ}{\LABPRE:DJAQ}
        \label{\SETLABEL}

        \idx{Dow Jones News Information Retrieval Service}
        For the analysis, the data was in the directory
        {\directory}\footnote{Data from Dow Jones News Information
        Retrieval Service, 1981---1994, by {\timescale}s, as an
        index.}.

        The data in this section is presented in
        Section~\ref{\SETLABELREF}.

        %
% -----------------------------------------------------------------------------
%
% A license is hereby granted to reproduce this software source code and
% to create executable versions from this source code for personal,
% non-commercial use.  The copyright notice included with the software
% must be maintained in all copies produced.
%
% THIS PROGRAM IS PROVIDED "AS IS". THE AUTHOR PROVIDES NO WARRANTIES
% WHATSOEVER, EXPRESSED OR IMPLIED, INCLUDING WARRANTIES OF
% MERCHANTABILITY, TITLE, OR FITNESS FOR ANY PARTICULAR PURPOSE.  THE
% AUTHOR DOES NOT WARRANT THAT USE OF THIS PROGRAM DOES NOT INFRINGE THE
% INTELLECTUAL PROPERTY RIGHTS OF ANY THIRD PARTY IN ANY COUNTRY.
%
% Copyright (c) 1994-2006, John Conover, All Rights Reserved.
%
% Comments and/or bug reports should be addressed to:
%
%     john@email.johncon.com (John Conover)
%
% -----------------------------------------------------------------------------
%
% Revision: \RCSRevision \\
% Revision Time: \RCSTime UMT \\
% Revision Date: \RCSDate \\
% Revision Id: \RCSId \\
% Revision File: \RCSLog \\
\RCS $Revision: 0.0 $
\RCS $Date: 2006/01/20 04:38:13 $
\RCS $Id: tables.tex,v 0.0 2006/01/20 04:38:13 john Exp $
% $Log: tables.tex,v $
% Revision 0.0  2006/01/20 04:38:13  john
% Initial version
%
%
    \subsection{{\market}, normalized increments}
        \label{\SETLABEL:TSA}

        The data in table~\ref{\SETLABEL:INC} is condensed from
        Section~\ref{\SETLABELREF:TSA}.

        \begin{small}
            \begin{table}[ht]
                \begin{center}
                    \caption[{\market}, normalized increments]
                        {{\market}, normalized increments.}
                    \begin{tabular}{|c|c|c|c|c|c|c|c|c|c|} \hline
                        \multicolumn{5}{|c|}{Normalized}                                                                                  & \multicolumn{5}{|c|}{Normalized Absolute Value}\\ \hline
                        Mean                & Standard              & rms                & \multicolumn{2}{|c|}{Least Squares}            & Mean                   & Standard                 & rms                & \multicolumn{2}{|c|}{Least Squares} \\ \cline{4-5}\cline{9-10}
                        \hspace{0.01in}     & deviation             & \hspace{0.01in}    & Constant                & Slope                & \hspace{0.01in}        & deviation                & \hspace{0.01in}    & Constant                   & Slope \\ \hline\hline
                        {\datafractionmean} & {\datafractionstddev} & {\datafractionrms} & {\datafractionconstant} & {\datafractionslope} & {\datafractionabsmean} & {\datafractionabsstddev} & {\datafractionrms} & {\datafractionabsconstant} & {\datafractionabsslope} \\ \hline
                    \end{tabular}
                    \label{\SETLABEL:INC}
                \end{center}
            \end{table}
        \end{small}

    \subsection{{\market}, Logarithmic Returns, in Bits}
        \label{\SETLABEL:LR}

        The data in table~\ref{\SETLABEL:RET} is condensed from
        Section~\ref{\SETLABELREF:FS}.

        \begin{small}
            \begin{table}[ht]
                \begin{center}
                    \caption[{\market}, Logarithmic Returns, in
                        Bits]{{\market}, Logarithmic Returns, in Bits.}
                    \begin{tabular}{|c|c|c|c|} \hline
                        \multicolumn{2}{|c|}{Calculated from Table~\ref{\SETLABEL:INC}} & \multicolumn{2}{|c|}{From program:}\\ \hline
                        Mean                    & Least squares                       & {\it tslsq}\/              & {\it tslogreturns}\/ \\ \hline\hline
                        {\datafractionmeanbits} & {\datafractionconstantbits} & {\datatslsqepbits} & {\logreturns} \\ \hline
                    \end{tabular}
                    \label{\SETLABEL:RET}
                \end{center}
            \end{table}
        \end{small}

    \subsection{{\market}, Shannon probabilities}
        \label{\SETLABEL:MAXSHANNON}

        The data in table~\ref{\SETLABEL:SHANNON} is condensed from
        sections~\ref{\SETLABELREF:FS}
        and~\ref{\SETLABELREF:MAXSHANNON}.

        \begin{small}
            \begin{table}[ht]
                \begin{center}
                    \caption[{\market}, Shannon
                        probabilities]{{\market}, Shannon
                        probabilities.}
                    \begin{tabular}{|c|c|c|c|} \hline
                        \multicolumn{3}{|c|}{Maximum} & \multicolumn{1}{|c|}{Operational}\\ \hline
                        Fraction of         & $\frac{\frac{\mbox{\scriptsize{mean}}}{\mbox{\scriptsize{rms}}} + 1}{2}$ & \multicolumn{2}{|c|}{From program:}\\ \cline{3-4}
                        positive increments & \hspace{0.01in}                                                          & {\it tsshannonmax}\/    & {\it tsshannon}\/ \\ \hline\hline
                        {\pmax}             & {\avgrms}                                                                & {\shannonmax}   & {\shannonlogreturns} \\ \hline
                    \end{tabular}
                    \label{\SETLABEL:SHANNON}
                \end{center}
            \end{table}
        \end{small}

    \subsection{{\market}, Logistic Analysis}
        \label{\SETLABEL:LAA}

        The data in table~\ref{\SETLABEL:LA} is condensed from
        Section~\ref{\SETLABELREF:LA}\footnote{Note that there are
        numerical stability issues with the methodology used to derive
        the constants---if the non-linear term, $b$, was greater than
        zero, it was set to zero to produce the graphs in
        Section~\ref{\SETLABELREF:LA}.}.

        \begin{small}
            \begin{table}[ht]
                \begin{center}
                    \caption[{\market}, Logistic Analysis.]
                        {{\market}, Logistic Analysis, $x_t = x_{t - 1}\left(a + b \cdot x_{t - 1}\right)$.}
                    \begin{tabular}{|c|c|} \hline
                        $a$ & $b$\\ \hline\hline
                        {\datafractionconstant} & {\datafractionslope}\\ \hline
                    \end{tabular}
                    \label{\SETLABEL:LA}
                \end{center}
            \end{table}
        \end{small}

    \subsection{{\market}, Hurst Coefficients and H  Parameters}
        \label{\SETLABEL:HCHP}

        The data in table~\ref{\SETLABEL:H} is condensed from
        Section~\ref{\SETLABELREF:H}.

        \begin{small}
            \begin{table}[ht]
                \begin{center}
                    \caption[{\market}, Hurst Coefficients and H
                        Parameters]{{\market}, Hurst Coefficients and
                        H Parameters.}
                    \begin{tabular}{|c|c|c|c|} \hline
                        \multicolumn{2}{|c|}{Hurst Coefficients} & \multicolumn{2}{|c|}{H Parameters}\\ \hline
                        Near term   & Far term    & Near term   & Far term \\ \hline\hline
                        {\thurstlow} & {\thurstall} & {\thcalclow} & {\thcalcall} \\ \hline
                    \end{tabular}
                    \label{\SETLABEL:H}
                \end{center}
            \end{table}
        \end{small}

        \begin{small}
            \begin{table}[ht]
                \begin{center}
                    \caption[{\market}, Hurst Coefficients and H
                        Parameters]{{\market}, Hurst Coefficients and
                        H Parameters, as a Derivative.}
                    \begin{tabular}{|c|c|c|c|} \hline
                        \multicolumn{2}{|c|}{Hurst Coefficients} & \multicolumn{2}{|c|}{H Parameters}\\ \hline
                        Near term    & Far term     & Near term    & Far term \\ \hline\hline
                        {\hurstlow} & {\hurstall} & {\hcalclow} & {\hcalcall} \\ \hline
                    \end{tabular}
                    \label{\SETLABEL:TH}
                \end{center}
            \end{table}
        \end{small}

    \subsection{{\market}, verification of the increments}
        \label{\SETLABEL:VI1}

        The data in table~\ref{\SETLABEL:COMP} is condensed from
        Section~\ref{\SETLABELREF:QVA}.

        \begin{small}
            \begin{table}[ht]
                \begin{center}
                    \caption[{\market}, verification of
                        the increments]{{\market}, verification the of
                        the increments, the mean, $\sigma$ is the
                        standard deviation from
                        table~\ref{\SETLABEL:INC},
                        {\datafractionstddev}, and $P$ is the maximum
                        Shannon probability from
                        table~\ref{\SETLABEL:SHANNON}, {\pmax}. In
                        principle, the values should equate.}
                    \begin{tabular}{|c|c|c|} \hline
                        Mean                & $\mbox{rms} (2P - 1)$ & $\frac{{\sigma}(2P - 1)}{2\sqrt{P(P - 1)}} $ \\ \hline\hline
                        {\datafractionmean} & {\rmsp}               & {\sigmap} \\ \hline
                    \end{tabular}
                    \label{\SETLABEL:COMP}
                \end{center}
            \end{table}
        \end{small}

    \subsection{{\market}, verification of the increments}
        \label{\SETLABEL:VI2}

        The data in table~\ref{\SETLABEL:ABS} is condensed from
        Section~\ref{\SETLABELREF:QVA}.

        \begin{small}
            \begin{table}[ht]
                \begin{center}
                    \caption[{\market}, verification of
                        the increments]{{\market}, verification the of
                        increments. In principle, the mean of the
                        absolute value of the increments and the root
                        mean square of the increments should
                        equate\footnote{The absolute value of the
                        normalized increments, when averaged, is
                        related to the root mean square of the
                        increments by a constant. If the normalized
                        increments are a fixed increment, the constant
                        is unity. If the normalized increments have a
                        Gaussian distribution, the constant is
                        $\approx 0.8$ depending on the accuracy of of
                        ``fit'' to a Gaussian distribution.}.}
                    \begin{tabular}{|c|c|} \hline
                        Mean of the               & rms \\
                        absolute value            & \hspace{0.01in} \\ \hline\hline
                        {\datafractionabsmean}    & {\datafractionrms} \\ \hline
                    \end{tabular}
                    \label{\SETLABEL:ABS}
                \end{center}
            \end{table}
        \end{small}

    \subsection{{\market}, $\chi^2$ values of the increments}
        \label{\SETLABEL:XSQ}

        The data in table~\ref{\SETLABEL:XSQT} is condensed from
        Section~\ref{\SETLABELREF:NH}.

        \begin{small}
            \begin{table}[ht]
                \begin{center}
                    \caption[{\market}, $\chi^2$ values of
                        the increments]{{\market}, $\chi^2$ values of
                        the increments. In principle, if the
                        distribution of the normalized increments is a
                        Gaussian distribution, the $\chi^2$ value will
                        be significantly less than the critical
                        value.}
                    \begin{tabular}{|c|c|} \hline
                        $\chi^2$      & Critical Value \\ \hline\hline
                        {\chisquared} & {\critical} \\ \hline
                    \end{tabular}
                    \label{\SETLABEL:XSQT}
                \end{center}
            \end{table}
        \end{small}

    \subsection{{\market}, time series data, empirical and simulated}
        \label{\SETLABEL:SIM}

        The data in table~\ref{\SETLABEL:ES} is condensed from
        Section~\ref{\SETLABELREF:TSUNFAIRBROWNIAN}.

        \begin{small}
            \begin{table}[ht]
                \begin{center}
                    \caption[{\market}, time series data, empirical
                        and simulated]{{\market}, time series data,
                        empirical and simulated, analysis of the
                        normalized increments.}
                    \begin{tabular}{|c|c|c|c|} \hline
                        \multicolumn{2}{|c|}{Empirical} & \multicolumn{2}{|c|}{Simulated}\\ \hline
                        Mean                & Standard              & Mean               & Standard \\
                        \hspace{0.01in}     & deviation             & \hspace{0.01in}    & deviation \\ \hline\hline
                        {\datafractionmean} & {\datafractionstddev} & {\tsunfairbrownianfractionmean} & {\tsunfairbrownianfractionstddev} \\ \hline
                    \end{tabular}
                    \label{\SETLABEL:ES}
                \end{center}
            \end{table}
        \end{small}

    \subsection{{\market}, number of participating companies}
        \label{\SETLABEL:QNC}

        The data in table~\ref{\SETLABEL:NC} is condensed from
        Section~\ref{\SETLABELREF:QNC}.

        \begin{small}
            \begin{table}[ht]
                \begin{center}
                    \caption[{\market}, number of participating
                        companies] {{\market}, number of participating
                        companies.}
                    \begin{tabular}{|c|c|} \hline
                        Number & Shannon probability\\ \hline
                        {\ncompanies} & {\pncompanies}\\ \hline
                    \end{tabular}
                    \label{\SETLABEL:NC}
                \end{center}
            \end{table}
        \end{small}

    \subsection{{\market}, Shannon probability optimizations}
        \label{\SETLABEL:SPO}

        The data in table~\ref{\SETLABEL:SP} is condensed from
        Section~\ref{\SETLABELREF:QNC}.

        \begin{small}
            \begin{table}[ht]
                \begin{center}
                    \caption[{\market}, Shannon probability
                         optimizations] {{\market}, Shannon
                         probability optimization.}
                    \begin{tabular}{|c|c|} \hline
                        optimize capital growth & optimize market growth\\ \hline
                        {\avgrms} & {\pncompanies}\\ \hline
                    \end{tabular}
                    \label{\SETLABEL:SP}
                \end{center}
            \end{table}
        \end{small}

% Local Variables:
% TeX-parse-self: t
% TeX-auto-save: t
% TeX-master: "fractal.tex"
% End:


    \renewcommand{\market}{Cirrus Logic Stock}
    \renewcommand{\directory}{../markets/crus}
    \renewcommand{\datafractionmean}{0.008052}
\renewcommand{\datafractionmeanbits}{0.011570}
\renewcommand{\datafractionmeanq}{0.002684}
\renewcommand{\datafractionmeanbitsq}{0.003867}
\renewcommand{\datafractionstddev}{0.038579}
\renewcommand{\datafractionrms}{0.039311}
\renewcommand{\avgrms}{0.602414}
\renewcommand{\ncompanies}{5.210454}
\renewcommand{\pncompanies}{0.544866}
\renewcommand{\datafractionabsmean}{0.029745}
\renewcommand{\datafractionabsstddev}{0.025769}
\renewcommand{\datafractionconstant}{0.010041}
\renewcommand{\datafractionconstantbits}{0.014414}
\renewcommand{\datafractionconstantq}{0.003347}
\renewcommand{\datafractionconstantbitsq}{0.004821}
\renewcommand{\datafractionslope}{-0.000021}
\renewcommand{\datafractionabsconstant}{0.035145}
\renewcommand{\datafractionabsslope}{-0.000057}
\renewcommand{\hurstall}{0.659558}
\renewcommand{\hurstlow}{0.707509}
\renewcommand{\hurstlowtwo}{1.415018}
\renewcommand{\hurstlowhundred}{70.750900}
\renewcommand{\hcalcall}{0.184942}
\renewcommand{\hcalclow}{0.102042}
\renewcommand{\shannonmax}{0.604167}
\renewcommand{\twoponemax}{0.208334}
\renewcommand{\logreturns}{0.010456}
\renewcommand{\twologreturns}{1.007274}
\renewcommand{\twologreturnshundred}{0.727387}
\renewcommand{\oneoverlogreturns}{95.638868}
\renewcommand{\pmax}{0.602094}
\renewcommand{\twopminusone}{0.204188}
\renewcommand{\rmsp}{0.008027}
\renewcommand{\twopx}{0.208583}
\renewcommand{\sigmap}{0.008047}
\renewcommand{\tsunfairbrownianfractionmean}{0.007862}
\renewcommand{\tsunfairbrownianfractionstddev}{0.038619}
\renewcommand{\shannonlogreturns}{0.560125}
\renewcommand{\shannonlogreturnshundred}{56.012500}
\renewcommand{\twopone}{0.120250}
\renewcommand{\twoponehundred}{12.025000}
\renewcommand{\hundredtwoponehundred}{87.975000}
\renewcommand{\hundredshannonlogreturnshundred}{43.987500}
\renewcommand{\datatslsqepbits}{0.007623}
\renewcommand{\thurstall}{0.633980}
\renewcommand{\thurstlow}{0.710108}
\renewcommand{\thurstlowtwo}{1.420216}
\renewcommand{\thurstlowhundred}{71.010800}
\renewcommand{\thcalcall}{0.247886}
\renewcommand{\thcalclow}{0.171737}
\renewcommand{\chisquared}{2.862000}
\renewcommand{\critical}{42.557000}

    \renewcommand{\timescale}{day}
    \subidx{market}{\market}
    \idx{\market}

    \section{\market}

        \renewcommand{\SETLABEL}{\LABPRE:CRUS}
        \renewcommand{\SETLABELQ}{\LABPRE:CRUSQ}
        \label{\SETLABEL}

        \idx{Cirrus Logic Stock}
        \idx{Stock, Cirrus Logic}
        For the analysis, the data was in the directory
        {\directory}\footnote{Data from
        ftp://ftp.ai.mit.edu/pub/stocks/results/, by {\timescale}s, in
        dollars, US.}.

        The data in this section is presented in
        Section~\ref{\SETLABELREF}.

        %
% -----------------------------------------------------------------------------
%
% A license is hereby granted to reproduce this software source code and
% to create executable versions from this source code for personal,
% non-commercial use.  The copyright notice included with the software
% must be maintained in all copies produced.
%
% THIS PROGRAM IS PROVIDED "AS IS". THE AUTHOR PROVIDES NO WARRANTIES
% WHATSOEVER, EXPRESSED OR IMPLIED, INCLUDING WARRANTIES OF
% MERCHANTABILITY, TITLE, OR FITNESS FOR ANY PARTICULAR PURPOSE.  THE
% AUTHOR DOES NOT WARRANT THAT USE OF THIS PROGRAM DOES NOT INFRINGE THE
% INTELLECTUAL PROPERTY RIGHTS OF ANY THIRD PARTY IN ANY COUNTRY.
%
% Copyright (c) 1994-2006, John Conover, All Rights Reserved.
%
% Comments and/or bug reports should be addressed to:
%
%     john@email.johncon.com (John Conover)
%
% -----------------------------------------------------------------------------
%
% Revision: \RCSRevision \\
% Revision Time: \RCSTime UMT \\
% Revision Date: \RCSDate \\
% Revision Id: \RCSId \\
% Revision File: \RCSLog \\
\RCS $Revision: 0.0 $
\RCS $Date: 2006/01/20 04:38:13 $
\RCS $Id: tables.tex,v 0.0 2006/01/20 04:38:13 john Exp $
% $Log: tables.tex,v $
% Revision 0.0  2006/01/20 04:38:13  john
% Initial version
%
%
    \subsection{{\market}, normalized increments}
        \label{\SETLABEL:TSA}

        The data in table~\ref{\SETLABEL:INC} is condensed from
        Section~\ref{\SETLABELREF:TSA}.

        \begin{small}
            \begin{table}[ht]
                \begin{center}
                    \caption[{\market}, normalized increments]
                        {{\market}, normalized increments.}
                    \begin{tabular}{|c|c|c|c|c|c|c|c|c|c|} \hline
                        \multicolumn{5}{|c|}{Normalized}                                                                                  & \multicolumn{5}{|c|}{Normalized Absolute Value}\\ \hline
                        Mean                & Standard              & rms                & \multicolumn{2}{|c|}{Least Squares}            & Mean                   & Standard                 & rms                & \multicolumn{2}{|c|}{Least Squares} \\ \cline{4-5}\cline{9-10}
                        \hspace{0.01in}     & deviation             & \hspace{0.01in}    & Constant                & Slope                & \hspace{0.01in}        & deviation                & \hspace{0.01in}    & Constant                   & Slope \\ \hline\hline
                        {\datafractionmean} & {\datafractionstddev} & {\datafractionrms} & {\datafractionconstant} & {\datafractionslope} & {\datafractionabsmean} & {\datafractionabsstddev} & {\datafractionrms} & {\datafractionabsconstant} & {\datafractionabsslope} \\ \hline
                    \end{tabular}
                    \label{\SETLABEL:INC}
                \end{center}
            \end{table}
        \end{small}

    \subsection{{\market}, Logarithmic Returns, in Bits}
        \label{\SETLABEL:LR}

        The data in table~\ref{\SETLABEL:RET} is condensed from
        Section~\ref{\SETLABELREF:FS}.

        \begin{small}
            \begin{table}[ht]
                \begin{center}
                    \caption[{\market}, Logarithmic Returns, in
                        Bits]{{\market}, Logarithmic Returns, in Bits.}
                    \begin{tabular}{|c|c|c|c|} \hline
                        \multicolumn{2}{|c|}{Calculated from Table~\ref{\SETLABEL:INC}} & \multicolumn{2}{|c|}{From program:}\\ \hline
                        Mean                    & Least squares                       & {\it tslsq}\/              & {\it tslogreturns}\/ \\ \hline\hline
                        {\datafractionmeanbits} & {\datafractionconstantbits} & {\datatslsqepbits} & {\logreturns} \\ \hline
                    \end{tabular}
                    \label{\SETLABEL:RET}
                \end{center}
            \end{table}
        \end{small}

    \subsection{{\market}, Shannon probabilities}
        \label{\SETLABEL:MAXSHANNON}

        The data in table~\ref{\SETLABEL:SHANNON} is condensed from
        sections~\ref{\SETLABELREF:FS}
        and~\ref{\SETLABELREF:MAXSHANNON}.

        \begin{small}
            \begin{table}[ht]
                \begin{center}
                    \caption[{\market}, Shannon
                        probabilities]{{\market}, Shannon
                        probabilities.}
                    \begin{tabular}{|c|c|c|c|} \hline
                        \multicolumn{3}{|c|}{Maximum} & \multicolumn{1}{|c|}{Operational}\\ \hline
                        Fraction of         & $\frac{\frac{\mbox{\scriptsize{mean}}}{\mbox{\scriptsize{rms}}} + 1}{2}$ & \multicolumn{2}{|c|}{From program:}\\ \cline{3-4}
                        positive increments & \hspace{0.01in}                                                          & {\it tsshannonmax}\/    & {\it tsshannon}\/ \\ \hline\hline
                        {\pmax}             & {\avgrms}                                                                & {\shannonmax}   & {\shannonlogreturns} \\ \hline
                    \end{tabular}
                    \label{\SETLABEL:SHANNON}
                \end{center}
            \end{table}
        \end{small}

    \subsection{{\market}, Logistic Analysis}
        \label{\SETLABEL:LAA}

        The data in table~\ref{\SETLABEL:LA} is condensed from
        Section~\ref{\SETLABELREF:LA}\footnote{Note that there are
        numerical stability issues with the methodology used to derive
        the constants---if the non-linear term, $b$, was greater than
        zero, it was set to zero to produce the graphs in
        Section~\ref{\SETLABELREF:LA}.}.

        \begin{small}
            \begin{table}[ht]
                \begin{center}
                    \caption[{\market}, Logistic Analysis.]
                        {{\market}, Logistic Analysis, $x_t = x_{t - 1}\left(a + b \cdot x_{t - 1}\right)$.}
                    \begin{tabular}{|c|c|} \hline
                        $a$ & $b$\\ \hline\hline
                        {\datafractionconstant} & {\datafractionslope}\\ \hline
                    \end{tabular}
                    \label{\SETLABEL:LA}
                \end{center}
            \end{table}
        \end{small}

    \subsection{{\market}, Hurst Coefficients and H  Parameters}
        \label{\SETLABEL:HCHP}

        The data in table~\ref{\SETLABEL:H} is condensed from
        Section~\ref{\SETLABELREF:H}.

        \begin{small}
            \begin{table}[ht]
                \begin{center}
                    \caption[{\market}, Hurst Coefficients and H
                        Parameters]{{\market}, Hurst Coefficients and
                        H Parameters.}
                    \begin{tabular}{|c|c|c|c|} \hline
                        \multicolumn{2}{|c|}{Hurst Coefficients} & \multicolumn{2}{|c|}{H Parameters}\\ \hline
                        Near term   & Far term    & Near term   & Far term \\ \hline\hline
                        {\thurstlow} & {\thurstall} & {\thcalclow} & {\thcalcall} \\ \hline
                    \end{tabular}
                    \label{\SETLABEL:H}
                \end{center}
            \end{table}
        \end{small}

        \begin{small}
            \begin{table}[ht]
                \begin{center}
                    \caption[{\market}, Hurst Coefficients and H
                        Parameters]{{\market}, Hurst Coefficients and
                        H Parameters, as a Derivative.}
                    \begin{tabular}{|c|c|c|c|} \hline
                        \multicolumn{2}{|c|}{Hurst Coefficients} & \multicolumn{2}{|c|}{H Parameters}\\ \hline
                        Near term    & Far term     & Near term    & Far term \\ \hline\hline
                        {\hurstlow} & {\hurstall} & {\hcalclow} & {\hcalcall} \\ \hline
                    \end{tabular}
                    \label{\SETLABEL:TH}
                \end{center}
            \end{table}
        \end{small}

    \subsection{{\market}, verification of the increments}
        \label{\SETLABEL:VI1}

        The data in table~\ref{\SETLABEL:COMP} is condensed from
        Section~\ref{\SETLABELREF:QVA}.

        \begin{small}
            \begin{table}[ht]
                \begin{center}
                    \caption[{\market}, verification of
                        the increments]{{\market}, verification the of
                        the increments, the mean, $\sigma$ is the
                        standard deviation from
                        table~\ref{\SETLABEL:INC},
                        {\datafractionstddev}, and $P$ is the maximum
                        Shannon probability from
                        table~\ref{\SETLABEL:SHANNON}, {\pmax}. In
                        principle, the values should equate.}
                    \begin{tabular}{|c|c|c|} \hline
                        Mean                & $\mbox{rms} (2P - 1)$ & $\frac{{\sigma}(2P - 1)}{2\sqrt{P(P - 1)}} $ \\ \hline\hline
                        {\datafractionmean} & {\rmsp}               & {\sigmap} \\ \hline
                    \end{tabular}
                    \label{\SETLABEL:COMP}
                \end{center}
            \end{table}
        \end{small}

    \subsection{{\market}, verification of the increments}
        \label{\SETLABEL:VI2}

        The data in table~\ref{\SETLABEL:ABS} is condensed from
        Section~\ref{\SETLABELREF:QVA}.

        \begin{small}
            \begin{table}[ht]
                \begin{center}
                    \caption[{\market}, verification of
                        the increments]{{\market}, verification the of
                        increments. In principle, the mean of the
                        absolute value of the increments and the root
                        mean square of the increments should
                        equate\footnote{The absolute value of the
                        normalized increments, when averaged, is
                        related to the root mean square of the
                        increments by a constant. If the normalized
                        increments are a fixed increment, the constant
                        is unity. If the normalized increments have a
                        Gaussian distribution, the constant is
                        $\approx 0.8$ depending on the accuracy of of
                        ``fit'' to a Gaussian distribution.}.}
                    \begin{tabular}{|c|c|} \hline
                        Mean of the               & rms \\
                        absolute value            & \hspace{0.01in} \\ \hline\hline
                        {\datafractionabsmean}    & {\datafractionrms} \\ \hline
                    \end{tabular}
                    \label{\SETLABEL:ABS}
                \end{center}
            \end{table}
        \end{small}

    \subsection{{\market}, $\chi^2$ values of the increments}
        \label{\SETLABEL:XSQ}

        The data in table~\ref{\SETLABEL:XSQT} is condensed from
        Section~\ref{\SETLABELREF:NH}.

        \begin{small}
            \begin{table}[ht]
                \begin{center}
                    \caption[{\market}, $\chi^2$ values of
                        the increments]{{\market}, $\chi^2$ values of
                        the increments. In principle, if the
                        distribution of the normalized increments is a
                        Gaussian distribution, the $\chi^2$ value will
                        be significantly less than the critical
                        value.}
                    \begin{tabular}{|c|c|} \hline
                        $\chi^2$      & Critical Value \\ \hline\hline
                        {\chisquared} & {\critical} \\ \hline
                    \end{tabular}
                    \label{\SETLABEL:XSQT}
                \end{center}
            \end{table}
        \end{small}

    \subsection{{\market}, time series data, empirical and simulated}
        \label{\SETLABEL:SIM}

        The data in table~\ref{\SETLABEL:ES} is condensed from
        Section~\ref{\SETLABELREF:TSUNFAIRBROWNIAN}.

        \begin{small}
            \begin{table}[ht]
                \begin{center}
                    \caption[{\market}, time series data, empirical
                        and simulated]{{\market}, time series data,
                        empirical and simulated, analysis of the
                        normalized increments.}
                    \begin{tabular}{|c|c|c|c|} \hline
                        \multicolumn{2}{|c|}{Empirical} & \multicolumn{2}{|c|}{Simulated}\\ \hline
                        Mean                & Standard              & Mean               & Standard \\
                        \hspace{0.01in}     & deviation             & \hspace{0.01in}    & deviation \\ \hline\hline
                        {\datafractionmean} & {\datafractionstddev} & {\tsunfairbrownianfractionmean} & {\tsunfairbrownianfractionstddev} \\ \hline
                    \end{tabular}
                    \label{\SETLABEL:ES}
                \end{center}
            \end{table}
        \end{small}

    \subsection{{\market}, number of participating companies}
        \label{\SETLABEL:QNC}

        The data in table~\ref{\SETLABEL:NC} is condensed from
        Section~\ref{\SETLABELREF:QNC}.

        \begin{small}
            \begin{table}[ht]
                \begin{center}
                    \caption[{\market}, number of participating
                        companies] {{\market}, number of participating
                        companies.}
                    \begin{tabular}{|c|c|} \hline
                        Number & Shannon probability\\ \hline
                        {\ncompanies} & {\pncompanies}\\ \hline
                    \end{tabular}
                    \label{\SETLABEL:NC}
                \end{center}
            \end{table}
        \end{small}

    \subsection{{\market}, Shannon probability optimizations}
        \label{\SETLABEL:SPO}

        The data in table~\ref{\SETLABEL:SP} is condensed from
        Section~\ref{\SETLABELREF:QNC}.

        \begin{small}
            \begin{table}[ht]
                \begin{center}
                    \caption[{\market}, Shannon probability
                         optimizations] {{\market}, Shannon
                         probability optimization.}
                    \begin{tabular}{|c|c|} \hline
                        optimize capital growth & optimize market growth\\ \hline
                        {\avgrms} & {\pncompanies}\\ \hline
                    \end{tabular}
                    \label{\SETLABEL:SP}
                \end{center}
            \end{table}
        \end{small}

% Local Variables:
% TeX-parse-self: t
% TeX-auto-save: t
% TeX-master: "fractal.tex"
% End:


    \renewcommand{\market}{United States Gross Domestic Product}
    \renewcommand{\directory}{../markets/us.gdp}
    \renewcommand{\datafractionmean}{0.008052}
\renewcommand{\datafractionmeanbits}{0.011570}
\renewcommand{\datafractionmeanq}{0.002684}
\renewcommand{\datafractionmeanbitsq}{0.003867}
\renewcommand{\datafractionstddev}{0.038579}
\renewcommand{\datafractionrms}{0.039311}
\renewcommand{\avgrms}{0.602414}
\renewcommand{\ncompanies}{5.210454}
\renewcommand{\pncompanies}{0.544866}
\renewcommand{\datafractionabsmean}{0.029745}
\renewcommand{\datafractionabsstddev}{0.025769}
\renewcommand{\datafractionconstant}{0.010041}
\renewcommand{\datafractionconstantbits}{0.014414}
\renewcommand{\datafractionconstantq}{0.003347}
\renewcommand{\datafractionconstantbitsq}{0.004821}
\renewcommand{\datafractionslope}{-0.000021}
\renewcommand{\datafractionabsconstant}{0.035145}
\renewcommand{\datafractionabsslope}{-0.000057}
\renewcommand{\hurstall}{0.659558}
\renewcommand{\hurstlow}{0.707509}
\renewcommand{\hurstlowtwo}{1.415018}
\renewcommand{\hurstlowhundred}{70.750900}
\renewcommand{\hcalcall}{0.184942}
\renewcommand{\hcalclow}{0.102042}
\renewcommand{\shannonmax}{0.604167}
\renewcommand{\twoponemax}{0.208334}
\renewcommand{\logreturns}{0.010456}
\renewcommand{\twologreturns}{1.007274}
\renewcommand{\twologreturnshundred}{0.727387}
\renewcommand{\oneoverlogreturns}{95.638868}
\renewcommand{\pmax}{0.602094}
\renewcommand{\twopminusone}{0.204188}
\renewcommand{\rmsp}{0.008027}
\renewcommand{\twopx}{0.208583}
\renewcommand{\sigmap}{0.008047}
\renewcommand{\tsunfairbrownianfractionmean}{0.007862}
\renewcommand{\tsunfairbrownianfractionstddev}{0.038619}
\renewcommand{\shannonlogreturns}{0.560125}
\renewcommand{\shannonlogreturnshundred}{56.012500}
\renewcommand{\twopone}{0.120250}
\renewcommand{\twoponehundred}{12.025000}
\renewcommand{\hundredtwoponehundred}{87.975000}
\renewcommand{\hundredshannonlogreturnshundred}{43.987500}
\renewcommand{\datatslsqepbits}{0.007623}
\renewcommand{\thurstall}{0.633980}
\renewcommand{\thurstlow}{0.710108}
\renewcommand{\thurstlowtwo}{1.420216}
\renewcommand{\thurstlowhundred}{71.010800}
\renewcommand{\thcalcall}{0.247886}
\renewcommand{\thcalclow}{0.171737}
\renewcommand{\chisquared}{2.862000}
\renewcommand{\critical}{42.557000}

    \renewcommand{\timescale}{month}
    \subidx{market}{\market}
    \idx{\market}

    \section{\market}

        \renewcommand{\SETLABEL}{\LABPRE:USGDP}
        \renewcommand{\SETLABELQ}{\LABPRE:USGDPQ}
        \label{\SETLABEL}

        \idx{United States Department of Commerce}
        For the analysis, the data was in the directory
        {\directory}\footnote{Data from the United States Department
        of Commerce, 1979---1994, by {\timescale}s, in billions of
        1987 dollars, US.}.

        The data in this section is presented in
        Section~\ref{\SETLABELREF}.

        %
% -----------------------------------------------------------------------------
%
% A license is hereby granted to reproduce this software source code and
% to create executable versions from this source code for personal,
% non-commercial use.  The copyright notice included with the software
% must be maintained in all copies produced.
%
% THIS PROGRAM IS PROVIDED "AS IS". THE AUTHOR PROVIDES NO WARRANTIES
% WHATSOEVER, EXPRESSED OR IMPLIED, INCLUDING WARRANTIES OF
% MERCHANTABILITY, TITLE, OR FITNESS FOR ANY PARTICULAR PURPOSE.  THE
% AUTHOR DOES NOT WARRANT THAT USE OF THIS PROGRAM DOES NOT INFRINGE THE
% INTELLECTUAL PROPERTY RIGHTS OF ANY THIRD PARTY IN ANY COUNTRY.
%
% Copyright (c) 1994-2006, John Conover, All Rights Reserved.
%
% Comments and/or bug reports should be addressed to:
%
%     john@email.johncon.com (John Conover)
%
% -----------------------------------------------------------------------------
%
% Revision: \RCSRevision \\
% Revision Time: \RCSTime UMT \\
% Revision Date: \RCSDate \\
% Revision Id: \RCSId \\
% Revision File: \RCSLog \\
\RCS $Revision: 0.0 $
\RCS $Date: 2006/01/20 04:38:13 $
\RCS $Id: tables.tex,v 0.0 2006/01/20 04:38:13 john Exp $
% $Log: tables.tex,v $
% Revision 0.0  2006/01/20 04:38:13  john
% Initial version
%
%
    \subsection{{\market}, normalized increments}
        \label{\SETLABEL:TSA}

        The data in table~\ref{\SETLABEL:INC} is condensed from
        Section~\ref{\SETLABELREF:TSA}.

        \begin{small}
            \begin{table}[ht]
                \begin{center}
                    \caption[{\market}, normalized increments]
                        {{\market}, normalized increments.}
                    \begin{tabular}{|c|c|c|c|c|c|c|c|c|c|} \hline
                        \multicolumn{5}{|c|}{Normalized}                                                                                  & \multicolumn{5}{|c|}{Normalized Absolute Value}\\ \hline
                        Mean                & Standard              & rms                & \multicolumn{2}{|c|}{Least Squares}            & Mean                   & Standard                 & rms                & \multicolumn{2}{|c|}{Least Squares} \\ \cline{4-5}\cline{9-10}
                        \hspace{0.01in}     & deviation             & \hspace{0.01in}    & Constant                & Slope                & \hspace{0.01in}        & deviation                & \hspace{0.01in}    & Constant                   & Slope \\ \hline\hline
                        {\datafractionmean} & {\datafractionstddev} & {\datafractionrms} & {\datafractionconstant} & {\datafractionslope} & {\datafractionabsmean} & {\datafractionabsstddev} & {\datafractionrms} & {\datafractionabsconstant} & {\datafractionabsslope} \\ \hline
                    \end{tabular}
                    \label{\SETLABEL:INC}
                \end{center}
            \end{table}
        \end{small}

    \subsection{{\market}, Logarithmic Returns, in Bits}
        \label{\SETLABEL:LR}

        The data in table~\ref{\SETLABEL:RET} is condensed from
        Section~\ref{\SETLABELREF:FS}.

        \begin{small}
            \begin{table}[ht]
                \begin{center}
                    \caption[{\market}, Logarithmic Returns, in
                        Bits]{{\market}, Logarithmic Returns, in Bits.}
                    \begin{tabular}{|c|c|c|c|} \hline
                        \multicolumn{2}{|c|}{Calculated from Table~\ref{\SETLABEL:INC}} & \multicolumn{2}{|c|}{From program:}\\ \hline
                        Mean                    & Least squares                       & {\it tslsq}\/              & {\it tslogreturns}\/ \\ \hline\hline
                        {\datafractionmeanbits} & {\datafractionconstantbits} & {\datatslsqepbits} & {\logreturns} \\ \hline
                    \end{tabular}
                    \label{\SETLABEL:RET}
                \end{center}
            \end{table}
        \end{small}

    \subsection{{\market}, Shannon probabilities}
        \label{\SETLABEL:MAXSHANNON}

        The data in table~\ref{\SETLABEL:SHANNON} is condensed from
        sections~\ref{\SETLABELREF:FS}
        and~\ref{\SETLABELREF:MAXSHANNON}.

        \begin{small}
            \begin{table}[ht]
                \begin{center}
                    \caption[{\market}, Shannon
                        probabilities]{{\market}, Shannon
                        probabilities.}
                    \begin{tabular}{|c|c|c|c|} \hline
                        \multicolumn{3}{|c|}{Maximum} & \multicolumn{1}{|c|}{Operational}\\ \hline
                        Fraction of         & $\frac{\frac{\mbox{\scriptsize{mean}}}{\mbox{\scriptsize{rms}}} + 1}{2}$ & \multicolumn{2}{|c|}{From program:}\\ \cline{3-4}
                        positive increments & \hspace{0.01in}                                                          & {\it tsshannonmax}\/    & {\it tsshannon}\/ \\ \hline\hline
                        {\pmax}             & {\avgrms}                                                                & {\shannonmax}   & {\shannonlogreturns} \\ \hline
                    \end{tabular}
                    \label{\SETLABEL:SHANNON}
                \end{center}
            \end{table}
        \end{small}

    \subsection{{\market}, Logistic Analysis}
        \label{\SETLABEL:LAA}

        The data in table~\ref{\SETLABEL:LA} is condensed from
        Section~\ref{\SETLABELREF:LA}\footnote{Note that there are
        numerical stability issues with the methodology used to derive
        the constants---if the non-linear term, $b$, was greater than
        zero, it was set to zero to produce the graphs in
        Section~\ref{\SETLABELREF:LA}.}.

        \begin{small}
            \begin{table}[ht]
                \begin{center}
                    \caption[{\market}, Logistic Analysis.]
                        {{\market}, Logistic Analysis, $x_t = x_{t - 1}\left(a + b \cdot x_{t - 1}\right)$.}
                    \begin{tabular}{|c|c|} \hline
                        $a$ & $b$\\ \hline\hline
                        {\datafractionconstant} & {\datafractionslope}\\ \hline
                    \end{tabular}
                    \label{\SETLABEL:LA}
                \end{center}
            \end{table}
        \end{small}

    \subsection{{\market}, Hurst Coefficients and H  Parameters}
        \label{\SETLABEL:HCHP}

        The data in table~\ref{\SETLABEL:H} is condensed from
        Section~\ref{\SETLABELREF:H}.

        \begin{small}
            \begin{table}[ht]
                \begin{center}
                    \caption[{\market}, Hurst Coefficients and H
                        Parameters]{{\market}, Hurst Coefficients and
                        H Parameters.}
                    \begin{tabular}{|c|c|c|c|} \hline
                        \multicolumn{2}{|c|}{Hurst Coefficients} & \multicolumn{2}{|c|}{H Parameters}\\ \hline
                        Near term   & Far term    & Near term   & Far term \\ \hline\hline
                        {\thurstlow} & {\thurstall} & {\thcalclow} & {\thcalcall} \\ \hline
                    \end{tabular}
                    \label{\SETLABEL:H}
                \end{center}
            \end{table}
        \end{small}

        \begin{small}
            \begin{table}[ht]
                \begin{center}
                    \caption[{\market}, Hurst Coefficients and H
                        Parameters]{{\market}, Hurst Coefficients and
                        H Parameters, as a Derivative.}
                    \begin{tabular}{|c|c|c|c|} \hline
                        \multicolumn{2}{|c|}{Hurst Coefficients} & \multicolumn{2}{|c|}{H Parameters}\\ \hline
                        Near term    & Far term     & Near term    & Far term \\ \hline\hline
                        {\hurstlow} & {\hurstall} & {\hcalclow} & {\hcalcall} \\ \hline
                    \end{tabular}
                    \label{\SETLABEL:TH}
                \end{center}
            \end{table}
        \end{small}

    \subsection{{\market}, verification of the increments}
        \label{\SETLABEL:VI1}

        The data in table~\ref{\SETLABEL:COMP} is condensed from
        Section~\ref{\SETLABELREF:QVA}.

        \begin{small}
            \begin{table}[ht]
                \begin{center}
                    \caption[{\market}, verification of
                        the increments]{{\market}, verification the of
                        the increments, the mean, $\sigma$ is the
                        standard deviation from
                        table~\ref{\SETLABEL:INC},
                        {\datafractionstddev}, and $P$ is the maximum
                        Shannon probability from
                        table~\ref{\SETLABEL:SHANNON}, {\pmax}. In
                        principle, the values should equate.}
                    \begin{tabular}{|c|c|c|} \hline
                        Mean                & $\mbox{rms} (2P - 1)$ & $\frac{{\sigma}(2P - 1)}{2\sqrt{P(P - 1)}} $ \\ \hline\hline
                        {\datafractionmean} & {\rmsp}               & {\sigmap} \\ \hline
                    \end{tabular}
                    \label{\SETLABEL:COMP}
                \end{center}
            \end{table}
        \end{small}

    \subsection{{\market}, verification of the increments}
        \label{\SETLABEL:VI2}

        The data in table~\ref{\SETLABEL:ABS} is condensed from
        Section~\ref{\SETLABELREF:QVA}.

        \begin{small}
            \begin{table}[ht]
                \begin{center}
                    \caption[{\market}, verification of
                        the increments]{{\market}, verification the of
                        increments. In principle, the mean of the
                        absolute value of the increments and the root
                        mean square of the increments should
                        equate\footnote{The absolute value of the
                        normalized increments, when averaged, is
                        related to the root mean square of the
                        increments by a constant. If the normalized
                        increments are a fixed increment, the constant
                        is unity. If the normalized increments have a
                        Gaussian distribution, the constant is
                        $\approx 0.8$ depending on the accuracy of of
                        ``fit'' to a Gaussian distribution.}.}
                    \begin{tabular}{|c|c|} \hline
                        Mean of the               & rms \\
                        absolute value            & \hspace{0.01in} \\ \hline\hline
                        {\datafractionabsmean}    & {\datafractionrms} \\ \hline
                    \end{tabular}
                    \label{\SETLABEL:ABS}
                \end{center}
            \end{table}
        \end{small}

    \subsection{{\market}, $\chi^2$ values of the increments}
        \label{\SETLABEL:XSQ}

        The data in table~\ref{\SETLABEL:XSQT} is condensed from
        Section~\ref{\SETLABELREF:NH}.

        \begin{small}
            \begin{table}[ht]
                \begin{center}
                    \caption[{\market}, $\chi^2$ values of
                        the increments]{{\market}, $\chi^2$ values of
                        the increments. In principle, if the
                        distribution of the normalized increments is a
                        Gaussian distribution, the $\chi^2$ value will
                        be significantly less than the critical
                        value.}
                    \begin{tabular}{|c|c|} \hline
                        $\chi^2$      & Critical Value \\ \hline\hline
                        {\chisquared} & {\critical} \\ \hline
                    \end{tabular}
                    \label{\SETLABEL:XSQT}
                \end{center}
            \end{table}
        \end{small}

    \subsection{{\market}, time series data, empirical and simulated}
        \label{\SETLABEL:SIM}

        The data in table~\ref{\SETLABEL:ES} is condensed from
        Section~\ref{\SETLABELREF:TSUNFAIRBROWNIAN}.

        \begin{small}
            \begin{table}[ht]
                \begin{center}
                    \caption[{\market}, time series data, empirical
                        and simulated]{{\market}, time series data,
                        empirical and simulated, analysis of the
                        normalized increments.}
                    \begin{tabular}{|c|c|c|c|} \hline
                        \multicolumn{2}{|c|}{Empirical} & \multicolumn{2}{|c|}{Simulated}\\ \hline
                        Mean                & Standard              & Mean               & Standard \\
                        \hspace{0.01in}     & deviation             & \hspace{0.01in}    & deviation \\ \hline\hline
                        {\datafractionmean} & {\datafractionstddev} & {\tsunfairbrownianfractionmean} & {\tsunfairbrownianfractionstddev} \\ \hline
                    \end{tabular}
                    \label{\SETLABEL:ES}
                \end{center}
            \end{table}
        \end{small}

    \subsection{{\market}, number of participating companies}
        \label{\SETLABEL:QNC}

        The data in table~\ref{\SETLABEL:NC} is condensed from
        Section~\ref{\SETLABELREF:QNC}.

        \begin{small}
            \begin{table}[ht]
                \begin{center}
                    \caption[{\market}, number of participating
                        companies] {{\market}, number of participating
                        companies.}
                    \begin{tabular}{|c|c|} \hline
                        Number & Shannon probability\\ \hline
                        {\ncompanies} & {\pncompanies}\\ \hline
                    \end{tabular}
                    \label{\SETLABEL:NC}
                \end{center}
            \end{table}
        \end{small}

    \subsection{{\market}, Shannon probability optimizations}
        \label{\SETLABEL:SPO}

        The data in table~\ref{\SETLABEL:SP} is condensed from
        Section~\ref{\SETLABELREF:QNC}.

        \begin{small}
            \begin{table}[ht]
                \begin{center}
                    \caption[{\market}, Shannon probability
                         optimizations] {{\market}, Shannon
                         probability optimization.}
                    \begin{tabular}{|c|c|} \hline
                        optimize capital growth & optimize market growth\\ \hline
                        {\avgrms} & {\pncompanies}\\ \hline
                    \end{tabular}
                    \label{\SETLABEL:SP}
                \end{center}
            \end{table}
        \end{small}

% Local Variables:
% TeX-parse-self: t
% TeX-auto-save: t
% TeX-master: "fractal.tex"
% End:


    \renewcommand{\market}{United States Employment Figures}
    \renewcommand{\directory}{../markets/us.employment}
    \renewcommand{\datafractionmean}{0.008052}
\renewcommand{\datafractionmeanbits}{0.011570}
\renewcommand{\datafractionmeanq}{0.002684}
\renewcommand{\datafractionmeanbitsq}{0.003867}
\renewcommand{\datafractionstddev}{0.038579}
\renewcommand{\datafractionrms}{0.039311}
\renewcommand{\avgrms}{0.602414}
\renewcommand{\ncompanies}{5.210454}
\renewcommand{\pncompanies}{0.544866}
\renewcommand{\datafractionabsmean}{0.029745}
\renewcommand{\datafractionabsstddev}{0.025769}
\renewcommand{\datafractionconstant}{0.010041}
\renewcommand{\datafractionconstantbits}{0.014414}
\renewcommand{\datafractionconstantq}{0.003347}
\renewcommand{\datafractionconstantbitsq}{0.004821}
\renewcommand{\datafractionslope}{-0.000021}
\renewcommand{\datafractionabsconstant}{0.035145}
\renewcommand{\datafractionabsslope}{-0.000057}
\renewcommand{\hurstall}{0.659558}
\renewcommand{\hurstlow}{0.707509}
\renewcommand{\hurstlowtwo}{1.415018}
\renewcommand{\hurstlowhundred}{70.750900}
\renewcommand{\hcalcall}{0.184942}
\renewcommand{\hcalclow}{0.102042}
\renewcommand{\shannonmax}{0.604167}
\renewcommand{\twoponemax}{0.208334}
\renewcommand{\logreturns}{0.010456}
\renewcommand{\twologreturns}{1.007274}
\renewcommand{\twologreturnshundred}{0.727387}
\renewcommand{\oneoverlogreturns}{95.638868}
\renewcommand{\pmax}{0.602094}
\renewcommand{\twopminusone}{0.204188}
\renewcommand{\rmsp}{0.008027}
\renewcommand{\twopx}{0.208583}
\renewcommand{\sigmap}{0.008047}
\renewcommand{\tsunfairbrownianfractionmean}{0.007862}
\renewcommand{\tsunfairbrownianfractionstddev}{0.038619}
\renewcommand{\shannonlogreturns}{0.560125}
\renewcommand{\shannonlogreturnshundred}{56.012500}
\renewcommand{\twopone}{0.120250}
\renewcommand{\twoponehundred}{12.025000}
\renewcommand{\hundredtwoponehundred}{87.975000}
\renewcommand{\hundredshannonlogreturnshundred}{43.987500}
\renewcommand{\datatslsqepbits}{0.007623}
\renewcommand{\thurstall}{0.633980}
\renewcommand{\thurstlow}{0.710108}
\renewcommand{\thurstlowtwo}{1.420216}
\renewcommand{\thurstlowhundred}{71.010800}
\renewcommand{\thcalcall}{0.247886}
\renewcommand{\thcalclow}{0.171737}
\renewcommand{\chisquared}{2.862000}
\renewcommand{\critical}{42.557000}

    \renewcommand{\timescale}{month}
    \subidx{market}{\market}
    \idx{\market}

    \section{\market}

        \renewcommand{\SETLABEL}{\LABPRE:USEMPLOYMENT}
        \renewcommand{\SETLABELQ}{\LABPRE:USEMPLOYMENTQ}
        \label{\SETLABEL}

        \idx{United States Bureau of Labor and Statistics}
        For the analysis, the data was in the directory
        {\directory}\footnote{Data from the United States Bureau of
        Labor and Statistics, 1980---1994, by {\timescale}s, in
        thousands of persons.}.

        The data in this section is presented in
        Section~\ref{\SETLABELREF}.

        %
% -----------------------------------------------------------------------------
%
% A license is hereby granted to reproduce this software source code and
% to create executable versions from this source code for personal,
% non-commercial use.  The copyright notice included with the software
% must be maintained in all copies produced.
%
% THIS PROGRAM IS PROVIDED "AS IS". THE AUTHOR PROVIDES NO WARRANTIES
% WHATSOEVER, EXPRESSED OR IMPLIED, INCLUDING WARRANTIES OF
% MERCHANTABILITY, TITLE, OR FITNESS FOR ANY PARTICULAR PURPOSE.  THE
% AUTHOR DOES NOT WARRANT THAT USE OF THIS PROGRAM DOES NOT INFRINGE THE
% INTELLECTUAL PROPERTY RIGHTS OF ANY THIRD PARTY IN ANY COUNTRY.
%
% Copyright (c) 1994-2006, John Conover, All Rights Reserved.
%
% Comments and/or bug reports should be addressed to:
%
%     john@email.johncon.com (John Conover)
%
% -----------------------------------------------------------------------------
%
% Revision: \RCSRevision \\
% Revision Time: \RCSTime UMT \\
% Revision Date: \RCSDate \\
% Revision Id: \RCSId \\
% Revision File: \RCSLog \\
\RCS $Revision: 0.0 $
\RCS $Date: 2006/01/20 04:38:13 $
\RCS $Id: tables.tex,v 0.0 2006/01/20 04:38:13 john Exp $
% $Log: tables.tex,v $
% Revision 0.0  2006/01/20 04:38:13  john
% Initial version
%
%
    \subsection{{\market}, normalized increments}
        \label{\SETLABEL:TSA}

        The data in table~\ref{\SETLABEL:INC} is condensed from
        Section~\ref{\SETLABELREF:TSA}.

        \begin{small}
            \begin{table}[ht]
                \begin{center}
                    \caption[{\market}, normalized increments]
                        {{\market}, normalized increments.}
                    \begin{tabular}{|c|c|c|c|c|c|c|c|c|c|} \hline
                        \multicolumn{5}{|c|}{Normalized}                                                                                  & \multicolumn{5}{|c|}{Normalized Absolute Value}\\ \hline
                        Mean                & Standard              & rms                & \multicolumn{2}{|c|}{Least Squares}            & Mean                   & Standard                 & rms                & \multicolumn{2}{|c|}{Least Squares} \\ \cline{4-5}\cline{9-10}
                        \hspace{0.01in}     & deviation             & \hspace{0.01in}    & Constant                & Slope                & \hspace{0.01in}        & deviation                & \hspace{0.01in}    & Constant                   & Slope \\ \hline\hline
                        {\datafractionmean} & {\datafractionstddev} & {\datafractionrms} & {\datafractionconstant} & {\datafractionslope} & {\datafractionabsmean} & {\datafractionabsstddev} & {\datafractionrms} & {\datafractionabsconstant} & {\datafractionabsslope} \\ \hline
                    \end{tabular}
                    \label{\SETLABEL:INC}
                \end{center}
            \end{table}
        \end{small}

    \subsection{{\market}, Logarithmic Returns, in Bits}
        \label{\SETLABEL:LR}

        The data in table~\ref{\SETLABEL:RET} is condensed from
        Section~\ref{\SETLABELREF:FS}.

        \begin{small}
            \begin{table}[ht]
                \begin{center}
                    \caption[{\market}, Logarithmic Returns, in
                        Bits]{{\market}, Logarithmic Returns, in Bits.}
                    \begin{tabular}{|c|c|c|c|} \hline
                        \multicolumn{2}{|c|}{Calculated from Table~\ref{\SETLABEL:INC}} & \multicolumn{2}{|c|}{From program:}\\ \hline
                        Mean                    & Least squares                       & {\it tslsq}\/              & {\it tslogreturns}\/ \\ \hline\hline
                        {\datafractionmeanbits} & {\datafractionconstantbits} & {\datatslsqepbits} & {\logreturns} \\ \hline
                    \end{tabular}
                    \label{\SETLABEL:RET}
                \end{center}
            \end{table}
        \end{small}

    \subsection{{\market}, Shannon probabilities}
        \label{\SETLABEL:MAXSHANNON}

        The data in table~\ref{\SETLABEL:SHANNON} is condensed from
        sections~\ref{\SETLABELREF:FS}
        and~\ref{\SETLABELREF:MAXSHANNON}.

        \begin{small}
            \begin{table}[ht]
                \begin{center}
                    \caption[{\market}, Shannon
                        probabilities]{{\market}, Shannon
                        probabilities.}
                    \begin{tabular}{|c|c|c|c|} \hline
                        \multicolumn{3}{|c|}{Maximum} & \multicolumn{1}{|c|}{Operational}\\ \hline
                        Fraction of         & $\frac{\frac{\mbox{\scriptsize{mean}}}{\mbox{\scriptsize{rms}}} + 1}{2}$ & \multicolumn{2}{|c|}{From program:}\\ \cline{3-4}
                        positive increments & \hspace{0.01in}                                                          & {\it tsshannonmax}\/    & {\it tsshannon}\/ \\ \hline\hline
                        {\pmax}             & {\avgrms}                                                                & {\shannonmax}   & {\shannonlogreturns} \\ \hline
                    \end{tabular}
                    \label{\SETLABEL:SHANNON}
                \end{center}
            \end{table}
        \end{small}

    \subsection{{\market}, Logistic Analysis}
        \label{\SETLABEL:LAA}

        The data in table~\ref{\SETLABEL:LA} is condensed from
        Section~\ref{\SETLABELREF:LA}\footnote{Note that there are
        numerical stability issues with the methodology used to derive
        the constants---if the non-linear term, $b$, was greater than
        zero, it was set to zero to produce the graphs in
        Section~\ref{\SETLABELREF:LA}.}.

        \begin{small}
            \begin{table}[ht]
                \begin{center}
                    \caption[{\market}, Logistic Analysis.]
                        {{\market}, Logistic Analysis, $x_t = x_{t - 1}\left(a + b \cdot x_{t - 1}\right)$.}
                    \begin{tabular}{|c|c|} \hline
                        $a$ & $b$\\ \hline\hline
                        {\datafractionconstant} & {\datafractionslope}\\ \hline
                    \end{tabular}
                    \label{\SETLABEL:LA}
                \end{center}
            \end{table}
        \end{small}

    \subsection{{\market}, Hurst Coefficients and H  Parameters}
        \label{\SETLABEL:HCHP}

        The data in table~\ref{\SETLABEL:H} is condensed from
        Section~\ref{\SETLABELREF:H}.

        \begin{small}
            \begin{table}[ht]
                \begin{center}
                    \caption[{\market}, Hurst Coefficients and H
                        Parameters]{{\market}, Hurst Coefficients and
                        H Parameters.}
                    \begin{tabular}{|c|c|c|c|} \hline
                        \multicolumn{2}{|c|}{Hurst Coefficients} & \multicolumn{2}{|c|}{H Parameters}\\ \hline
                        Near term   & Far term    & Near term   & Far term \\ \hline\hline
                        {\thurstlow} & {\thurstall} & {\thcalclow} & {\thcalcall} \\ \hline
                    \end{tabular}
                    \label{\SETLABEL:H}
                \end{center}
            \end{table}
        \end{small}

        \begin{small}
            \begin{table}[ht]
                \begin{center}
                    \caption[{\market}, Hurst Coefficients and H
                        Parameters]{{\market}, Hurst Coefficients and
                        H Parameters, as a Derivative.}
                    \begin{tabular}{|c|c|c|c|} \hline
                        \multicolumn{2}{|c|}{Hurst Coefficients} & \multicolumn{2}{|c|}{H Parameters}\\ \hline
                        Near term    & Far term     & Near term    & Far term \\ \hline\hline
                        {\hurstlow} & {\hurstall} & {\hcalclow} & {\hcalcall} \\ \hline
                    \end{tabular}
                    \label{\SETLABEL:TH}
                \end{center}
            \end{table}
        \end{small}

    \subsection{{\market}, verification of the increments}
        \label{\SETLABEL:VI1}

        The data in table~\ref{\SETLABEL:COMP} is condensed from
        Section~\ref{\SETLABELREF:QVA}.

        \begin{small}
            \begin{table}[ht]
                \begin{center}
                    \caption[{\market}, verification of
                        the increments]{{\market}, verification the of
                        the increments, the mean, $\sigma$ is the
                        standard deviation from
                        table~\ref{\SETLABEL:INC},
                        {\datafractionstddev}, and $P$ is the maximum
                        Shannon probability from
                        table~\ref{\SETLABEL:SHANNON}, {\pmax}. In
                        principle, the values should equate.}
                    \begin{tabular}{|c|c|c|} \hline
                        Mean                & $\mbox{rms} (2P - 1)$ & $\frac{{\sigma}(2P - 1)}{2\sqrt{P(P - 1)}} $ \\ \hline\hline
                        {\datafractionmean} & {\rmsp}               & {\sigmap} \\ \hline
                    \end{tabular}
                    \label{\SETLABEL:COMP}
                \end{center}
            \end{table}
        \end{small}

    \subsection{{\market}, verification of the increments}
        \label{\SETLABEL:VI2}

        The data in table~\ref{\SETLABEL:ABS} is condensed from
        Section~\ref{\SETLABELREF:QVA}.

        \begin{small}
            \begin{table}[ht]
                \begin{center}
                    \caption[{\market}, verification of
                        the increments]{{\market}, verification the of
                        increments. In principle, the mean of the
                        absolute value of the increments and the root
                        mean square of the increments should
                        equate\footnote{The absolute value of the
                        normalized increments, when averaged, is
                        related to the root mean square of the
                        increments by a constant. If the normalized
                        increments are a fixed increment, the constant
                        is unity. If the normalized increments have a
                        Gaussian distribution, the constant is
                        $\approx 0.8$ depending on the accuracy of of
                        ``fit'' to a Gaussian distribution.}.}
                    \begin{tabular}{|c|c|} \hline
                        Mean of the               & rms \\
                        absolute value            & \hspace{0.01in} \\ \hline\hline
                        {\datafractionabsmean}    & {\datafractionrms} \\ \hline
                    \end{tabular}
                    \label{\SETLABEL:ABS}
                \end{center}
            \end{table}
        \end{small}

    \subsection{{\market}, $\chi^2$ values of the increments}
        \label{\SETLABEL:XSQ}

        The data in table~\ref{\SETLABEL:XSQT} is condensed from
        Section~\ref{\SETLABELREF:NH}.

        \begin{small}
            \begin{table}[ht]
                \begin{center}
                    \caption[{\market}, $\chi^2$ values of
                        the increments]{{\market}, $\chi^2$ values of
                        the increments. In principle, if the
                        distribution of the normalized increments is a
                        Gaussian distribution, the $\chi^2$ value will
                        be significantly less than the critical
                        value.}
                    \begin{tabular}{|c|c|} \hline
                        $\chi^2$      & Critical Value \\ \hline\hline
                        {\chisquared} & {\critical} \\ \hline
                    \end{tabular}
                    \label{\SETLABEL:XSQT}
                \end{center}
            \end{table}
        \end{small}

    \subsection{{\market}, time series data, empirical and simulated}
        \label{\SETLABEL:SIM}

        The data in table~\ref{\SETLABEL:ES} is condensed from
        Section~\ref{\SETLABELREF:TSUNFAIRBROWNIAN}.

        \begin{small}
            \begin{table}[ht]
                \begin{center}
                    \caption[{\market}, time series data, empirical
                        and simulated]{{\market}, time series data,
                        empirical and simulated, analysis of the
                        normalized increments.}
                    \begin{tabular}{|c|c|c|c|} \hline
                        \multicolumn{2}{|c|}{Empirical} & \multicolumn{2}{|c|}{Simulated}\\ \hline
                        Mean                & Standard              & Mean               & Standard \\
                        \hspace{0.01in}     & deviation             & \hspace{0.01in}    & deviation \\ \hline\hline
                        {\datafractionmean} & {\datafractionstddev} & {\tsunfairbrownianfractionmean} & {\tsunfairbrownianfractionstddev} \\ \hline
                    \end{tabular}
                    \label{\SETLABEL:ES}
                \end{center}
            \end{table}
        \end{small}

    \subsection{{\market}, number of participating companies}
        \label{\SETLABEL:QNC}

        The data in table~\ref{\SETLABEL:NC} is condensed from
        Section~\ref{\SETLABELREF:QNC}.

        \begin{small}
            \begin{table}[ht]
                \begin{center}
                    \caption[{\market}, number of participating
                        companies] {{\market}, number of participating
                        companies.}
                    \begin{tabular}{|c|c|} \hline
                        Number & Shannon probability\\ \hline
                        {\ncompanies} & {\pncompanies}\\ \hline
                    \end{tabular}
                    \label{\SETLABEL:NC}
                \end{center}
            \end{table}
        \end{small}

    \subsection{{\market}, Shannon probability optimizations}
        \label{\SETLABEL:SPO}

        The data in table~\ref{\SETLABEL:SP} is condensed from
        Section~\ref{\SETLABELREF:QNC}.

        \begin{small}
            \begin{table}[ht]
                \begin{center}
                    \caption[{\market}, Shannon probability
                         optimizations] {{\market}, Shannon
                         probability optimization.}
                    \begin{tabular}{|c|c|} \hline
                        optimize capital growth & optimize market growth\\ \hline
                        {\avgrms} & {\pncompanies}\\ \hline
                    \end{tabular}
                    \label{\SETLABEL:SP}
                \end{center}
            \end{table}
        \end{small}

% Local Variables:
% TeX-parse-self: t
% TeX-auto-save: t
% TeX-master: "fractal.tex"
% End:


    \renewcommand{\market}{United States Leading Economic Indicators}
    \renewcommand{\directory}{../markets/us.indicators}
    \renewcommand{\datafractionmean}{0.008052}
\renewcommand{\datafractionmeanbits}{0.011570}
\renewcommand{\datafractionmeanq}{0.002684}
\renewcommand{\datafractionmeanbitsq}{0.003867}
\renewcommand{\datafractionstddev}{0.038579}
\renewcommand{\datafractionrms}{0.039311}
\renewcommand{\avgrms}{0.602414}
\renewcommand{\ncompanies}{5.210454}
\renewcommand{\pncompanies}{0.544866}
\renewcommand{\datafractionabsmean}{0.029745}
\renewcommand{\datafractionabsstddev}{0.025769}
\renewcommand{\datafractionconstant}{0.010041}
\renewcommand{\datafractionconstantbits}{0.014414}
\renewcommand{\datafractionconstantq}{0.003347}
\renewcommand{\datafractionconstantbitsq}{0.004821}
\renewcommand{\datafractionslope}{-0.000021}
\renewcommand{\datafractionabsconstant}{0.035145}
\renewcommand{\datafractionabsslope}{-0.000057}
\renewcommand{\hurstall}{0.659558}
\renewcommand{\hurstlow}{0.707509}
\renewcommand{\hurstlowtwo}{1.415018}
\renewcommand{\hurstlowhundred}{70.750900}
\renewcommand{\hcalcall}{0.184942}
\renewcommand{\hcalclow}{0.102042}
\renewcommand{\shannonmax}{0.604167}
\renewcommand{\twoponemax}{0.208334}
\renewcommand{\logreturns}{0.010456}
\renewcommand{\twologreturns}{1.007274}
\renewcommand{\twologreturnshundred}{0.727387}
\renewcommand{\oneoverlogreturns}{95.638868}
\renewcommand{\pmax}{0.602094}
\renewcommand{\twopminusone}{0.204188}
\renewcommand{\rmsp}{0.008027}
\renewcommand{\twopx}{0.208583}
\renewcommand{\sigmap}{0.008047}
\renewcommand{\tsunfairbrownianfractionmean}{0.007862}
\renewcommand{\tsunfairbrownianfractionstddev}{0.038619}
\renewcommand{\shannonlogreturns}{0.560125}
\renewcommand{\shannonlogreturnshundred}{56.012500}
\renewcommand{\twopone}{0.120250}
\renewcommand{\twoponehundred}{12.025000}
\renewcommand{\hundredtwoponehundred}{87.975000}
\renewcommand{\hundredshannonlogreturnshundred}{43.987500}
\renewcommand{\datatslsqepbits}{0.007623}
\renewcommand{\thurstall}{0.633980}
\renewcommand{\thurstlow}{0.710108}
\renewcommand{\thurstlowtwo}{1.420216}
\renewcommand{\thurstlowhundred}{71.010800}
\renewcommand{\thcalcall}{0.247886}
\renewcommand{\thcalclow}{0.171737}
\renewcommand{\chisquared}{2.862000}
\renewcommand{\critical}{42.557000}

    \renewcommand{\timescale}{month}
    \subidx{market}{\market}
    \idx{\market}

    \section{\market}

        \renewcommand{\SETLABEL}{\LABPRE:USINDICATORS}
        \renewcommand{\SETLABELQ}{\LABPRE:USINDICATORSQ}
        \label{\SETLABEL}

        \idx{United States Department of Commerce}
        For the analysis, the data was in the directory
        {\directory}\footnote{Data from the United States Department
        of Commerce, 1980---1994, by {\timescale}s, as an index of
        1987 = 100.}.

        The data in this section is presented in
        Section~\ref{\SETLABELREF}.

        %
% -----------------------------------------------------------------------------
%
% A license is hereby granted to reproduce this software source code and
% to create executable versions from this source code for personal,
% non-commercial use.  The copyright notice included with the software
% must be maintained in all copies produced.
%
% THIS PROGRAM IS PROVIDED "AS IS". THE AUTHOR PROVIDES NO WARRANTIES
% WHATSOEVER, EXPRESSED OR IMPLIED, INCLUDING WARRANTIES OF
% MERCHANTABILITY, TITLE, OR FITNESS FOR ANY PARTICULAR PURPOSE.  THE
% AUTHOR DOES NOT WARRANT THAT USE OF THIS PROGRAM DOES NOT INFRINGE THE
% INTELLECTUAL PROPERTY RIGHTS OF ANY THIRD PARTY IN ANY COUNTRY.
%
% Copyright (c) 1994-2006, John Conover, All Rights Reserved.
%
% Comments and/or bug reports should be addressed to:
%
%     john@email.johncon.com (John Conover)
%
% -----------------------------------------------------------------------------
%
% Revision: \RCSRevision \\
% Revision Time: \RCSTime UMT \\
% Revision Date: \RCSDate \\
% Revision Id: \RCSId \\
% Revision File: \RCSLog \\
\RCS $Revision: 0.0 $
\RCS $Date: 2006/01/20 04:38:13 $
\RCS $Id: tables.tex,v 0.0 2006/01/20 04:38:13 john Exp $
% $Log: tables.tex,v $
% Revision 0.0  2006/01/20 04:38:13  john
% Initial version
%
%
    \subsection{{\market}, normalized increments}
        \label{\SETLABEL:TSA}

        The data in table~\ref{\SETLABEL:INC} is condensed from
        Section~\ref{\SETLABELREF:TSA}.

        \begin{small}
            \begin{table}[ht]
                \begin{center}
                    \caption[{\market}, normalized increments]
                        {{\market}, normalized increments.}
                    \begin{tabular}{|c|c|c|c|c|c|c|c|c|c|} \hline
                        \multicolumn{5}{|c|}{Normalized}                                                                                  & \multicolumn{5}{|c|}{Normalized Absolute Value}\\ \hline
                        Mean                & Standard              & rms                & \multicolumn{2}{|c|}{Least Squares}            & Mean                   & Standard                 & rms                & \multicolumn{2}{|c|}{Least Squares} \\ \cline{4-5}\cline{9-10}
                        \hspace{0.01in}     & deviation             & \hspace{0.01in}    & Constant                & Slope                & \hspace{0.01in}        & deviation                & \hspace{0.01in}    & Constant                   & Slope \\ \hline\hline
                        {\datafractionmean} & {\datafractionstddev} & {\datafractionrms} & {\datafractionconstant} & {\datafractionslope} & {\datafractionabsmean} & {\datafractionabsstddev} & {\datafractionrms} & {\datafractionabsconstant} & {\datafractionabsslope} \\ \hline
                    \end{tabular}
                    \label{\SETLABEL:INC}
                \end{center}
            \end{table}
        \end{small}

    \subsection{{\market}, Logarithmic Returns, in Bits}
        \label{\SETLABEL:LR}

        The data in table~\ref{\SETLABEL:RET} is condensed from
        Section~\ref{\SETLABELREF:FS}.

        \begin{small}
            \begin{table}[ht]
                \begin{center}
                    \caption[{\market}, Logarithmic Returns, in
                        Bits]{{\market}, Logarithmic Returns, in Bits.}
                    \begin{tabular}{|c|c|c|c|} \hline
                        \multicolumn{2}{|c|}{Calculated from Table~\ref{\SETLABEL:INC}} & \multicolumn{2}{|c|}{From program:}\\ \hline
                        Mean                    & Least squares                       & {\it tslsq}\/              & {\it tslogreturns}\/ \\ \hline\hline
                        {\datafractionmeanbits} & {\datafractionconstantbits} & {\datatslsqepbits} & {\logreturns} \\ \hline
                    \end{tabular}
                    \label{\SETLABEL:RET}
                \end{center}
            \end{table}
        \end{small}

    \subsection{{\market}, Shannon probabilities}
        \label{\SETLABEL:MAXSHANNON}

        The data in table~\ref{\SETLABEL:SHANNON} is condensed from
        sections~\ref{\SETLABELREF:FS}
        and~\ref{\SETLABELREF:MAXSHANNON}.

        \begin{small}
            \begin{table}[ht]
                \begin{center}
                    \caption[{\market}, Shannon
                        probabilities]{{\market}, Shannon
                        probabilities.}
                    \begin{tabular}{|c|c|c|c|} \hline
                        \multicolumn{3}{|c|}{Maximum} & \multicolumn{1}{|c|}{Operational}\\ \hline
                        Fraction of         & $\frac{\frac{\mbox{\scriptsize{mean}}}{\mbox{\scriptsize{rms}}} + 1}{2}$ & \multicolumn{2}{|c|}{From program:}\\ \cline{3-4}
                        positive increments & \hspace{0.01in}                                                          & {\it tsshannonmax}\/    & {\it tsshannon}\/ \\ \hline\hline
                        {\pmax}             & {\avgrms}                                                                & {\shannonmax}   & {\shannonlogreturns} \\ \hline
                    \end{tabular}
                    \label{\SETLABEL:SHANNON}
                \end{center}
            \end{table}
        \end{small}

    \subsection{{\market}, Logistic Analysis}
        \label{\SETLABEL:LAA}

        The data in table~\ref{\SETLABEL:LA} is condensed from
        Section~\ref{\SETLABELREF:LA}\footnote{Note that there are
        numerical stability issues with the methodology used to derive
        the constants---if the non-linear term, $b$, was greater than
        zero, it was set to zero to produce the graphs in
        Section~\ref{\SETLABELREF:LA}.}.

        \begin{small}
            \begin{table}[ht]
                \begin{center}
                    \caption[{\market}, Logistic Analysis.]
                        {{\market}, Logistic Analysis, $x_t = x_{t - 1}\left(a + b \cdot x_{t - 1}\right)$.}
                    \begin{tabular}{|c|c|} \hline
                        $a$ & $b$\\ \hline\hline
                        {\datafractionconstant} & {\datafractionslope}\\ \hline
                    \end{tabular}
                    \label{\SETLABEL:LA}
                \end{center}
            \end{table}
        \end{small}

    \subsection{{\market}, Hurst Coefficients and H  Parameters}
        \label{\SETLABEL:HCHP}

        The data in table~\ref{\SETLABEL:H} is condensed from
        Section~\ref{\SETLABELREF:H}.

        \begin{small}
            \begin{table}[ht]
                \begin{center}
                    \caption[{\market}, Hurst Coefficients and H
                        Parameters]{{\market}, Hurst Coefficients and
                        H Parameters.}
                    \begin{tabular}{|c|c|c|c|} \hline
                        \multicolumn{2}{|c|}{Hurst Coefficients} & \multicolumn{2}{|c|}{H Parameters}\\ \hline
                        Near term   & Far term    & Near term   & Far term \\ \hline\hline
                        {\thurstlow} & {\thurstall} & {\thcalclow} & {\thcalcall} \\ \hline
                    \end{tabular}
                    \label{\SETLABEL:H}
                \end{center}
            \end{table}
        \end{small}

        \begin{small}
            \begin{table}[ht]
                \begin{center}
                    \caption[{\market}, Hurst Coefficients and H
                        Parameters]{{\market}, Hurst Coefficients and
                        H Parameters, as a Derivative.}
                    \begin{tabular}{|c|c|c|c|} \hline
                        \multicolumn{2}{|c|}{Hurst Coefficients} & \multicolumn{2}{|c|}{H Parameters}\\ \hline
                        Near term    & Far term     & Near term    & Far term \\ \hline\hline
                        {\hurstlow} & {\hurstall} & {\hcalclow} & {\hcalcall} \\ \hline
                    \end{tabular}
                    \label{\SETLABEL:TH}
                \end{center}
            \end{table}
        \end{small}

    \subsection{{\market}, verification of the increments}
        \label{\SETLABEL:VI1}

        The data in table~\ref{\SETLABEL:COMP} is condensed from
        Section~\ref{\SETLABELREF:QVA}.

        \begin{small}
            \begin{table}[ht]
                \begin{center}
                    \caption[{\market}, verification of
                        the increments]{{\market}, verification the of
                        the increments, the mean, $\sigma$ is the
                        standard deviation from
                        table~\ref{\SETLABEL:INC},
                        {\datafractionstddev}, and $P$ is the maximum
                        Shannon probability from
                        table~\ref{\SETLABEL:SHANNON}, {\pmax}. In
                        principle, the values should equate.}
                    \begin{tabular}{|c|c|c|} \hline
                        Mean                & $\mbox{rms} (2P - 1)$ & $\frac{{\sigma}(2P - 1)}{2\sqrt{P(P - 1)}} $ \\ \hline\hline
                        {\datafractionmean} & {\rmsp}               & {\sigmap} \\ \hline
                    \end{tabular}
                    \label{\SETLABEL:COMP}
                \end{center}
            \end{table}
        \end{small}

    \subsection{{\market}, verification of the increments}
        \label{\SETLABEL:VI2}

        The data in table~\ref{\SETLABEL:ABS} is condensed from
        Section~\ref{\SETLABELREF:QVA}.

        \begin{small}
            \begin{table}[ht]
                \begin{center}
                    \caption[{\market}, verification of
                        the increments]{{\market}, verification the of
                        increments. In principle, the mean of the
                        absolute value of the increments and the root
                        mean square of the increments should
                        equate\footnote{The absolute value of the
                        normalized increments, when averaged, is
                        related to the root mean square of the
                        increments by a constant. If the normalized
                        increments are a fixed increment, the constant
                        is unity. If the normalized increments have a
                        Gaussian distribution, the constant is
                        $\approx 0.8$ depending on the accuracy of of
                        ``fit'' to a Gaussian distribution.}.}
                    \begin{tabular}{|c|c|} \hline
                        Mean of the               & rms \\
                        absolute value            & \hspace{0.01in} \\ \hline\hline
                        {\datafractionabsmean}    & {\datafractionrms} \\ \hline
                    \end{tabular}
                    \label{\SETLABEL:ABS}
                \end{center}
            \end{table}
        \end{small}

    \subsection{{\market}, $\chi^2$ values of the increments}
        \label{\SETLABEL:XSQ}

        The data in table~\ref{\SETLABEL:XSQT} is condensed from
        Section~\ref{\SETLABELREF:NH}.

        \begin{small}
            \begin{table}[ht]
                \begin{center}
                    \caption[{\market}, $\chi^2$ values of
                        the increments]{{\market}, $\chi^2$ values of
                        the increments. In principle, if the
                        distribution of the normalized increments is a
                        Gaussian distribution, the $\chi^2$ value will
                        be significantly less than the critical
                        value.}
                    \begin{tabular}{|c|c|} \hline
                        $\chi^2$      & Critical Value \\ \hline\hline
                        {\chisquared} & {\critical} \\ \hline
                    \end{tabular}
                    \label{\SETLABEL:XSQT}
                \end{center}
            \end{table}
        \end{small}

    \subsection{{\market}, time series data, empirical and simulated}
        \label{\SETLABEL:SIM}

        The data in table~\ref{\SETLABEL:ES} is condensed from
        Section~\ref{\SETLABELREF:TSUNFAIRBROWNIAN}.

        \begin{small}
            \begin{table}[ht]
                \begin{center}
                    \caption[{\market}, time series data, empirical
                        and simulated]{{\market}, time series data,
                        empirical and simulated, analysis of the
                        normalized increments.}
                    \begin{tabular}{|c|c|c|c|} \hline
                        \multicolumn{2}{|c|}{Empirical} & \multicolumn{2}{|c|}{Simulated}\\ \hline
                        Mean                & Standard              & Mean               & Standard \\
                        \hspace{0.01in}     & deviation             & \hspace{0.01in}    & deviation \\ \hline\hline
                        {\datafractionmean} & {\datafractionstddev} & {\tsunfairbrownianfractionmean} & {\tsunfairbrownianfractionstddev} \\ \hline
                    \end{tabular}
                    \label{\SETLABEL:ES}
                \end{center}
            \end{table}
        \end{small}

    \subsection{{\market}, number of participating companies}
        \label{\SETLABEL:QNC}

        The data in table~\ref{\SETLABEL:NC} is condensed from
        Section~\ref{\SETLABELREF:QNC}.

        \begin{small}
            \begin{table}[ht]
                \begin{center}
                    \caption[{\market}, number of participating
                        companies] {{\market}, number of participating
                        companies.}
                    \begin{tabular}{|c|c|} \hline
                        Number & Shannon probability\\ \hline
                        {\ncompanies} & {\pncompanies}\\ \hline
                    \end{tabular}
                    \label{\SETLABEL:NC}
                \end{center}
            \end{table}
        \end{small}

    \subsection{{\market}, Shannon probability optimizations}
        \label{\SETLABEL:SPO}

        The data in table~\ref{\SETLABEL:SP} is condensed from
        Section~\ref{\SETLABELREF:QNC}.

        \begin{small}
            \begin{table}[ht]
                \begin{center}
                    \caption[{\market}, Shannon probability
                         optimizations] {{\market}, Shannon
                         probability optimization.}
                    \begin{tabular}{|c|c|} \hline
                        optimize capital growth & optimize market growth\\ \hline
                        {\avgrms} & {\pncompanies}\\ \hline
                    \end{tabular}
                    \label{\SETLABEL:SP}
                \end{center}
            \end{table}
        \end{small}

% Local Variables:
% TeX-parse-self: t
% TeX-auto-save: t
% TeX-master: "fractal.tex"
% End:


    \renewcommand{\market}{United States M2}
    \renewcommand{\directory}{../markets/us.m2}
    \renewcommand{\datafractionmean}{0.008052}
\renewcommand{\datafractionmeanbits}{0.011570}
\renewcommand{\datafractionmeanq}{0.002684}
\renewcommand{\datafractionmeanbitsq}{0.003867}
\renewcommand{\datafractionstddev}{0.038579}
\renewcommand{\datafractionrms}{0.039311}
\renewcommand{\avgrms}{0.602414}
\renewcommand{\ncompanies}{5.210454}
\renewcommand{\pncompanies}{0.544866}
\renewcommand{\datafractionabsmean}{0.029745}
\renewcommand{\datafractionabsstddev}{0.025769}
\renewcommand{\datafractionconstant}{0.010041}
\renewcommand{\datafractionconstantbits}{0.014414}
\renewcommand{\datafractionconstantq}{0.003347}
\renewcommand{\datafractionconstantbitsq}{0.004821}
\renewcommand{\datafractionslope}{-0.000021}
\renewcommand{\datafractionabsconstant}{0.035145}
\renewcommand{\datafractionabsslope}{-0.000057}
\renewcommand{\hurstall}{0.659558}
\renewcommand{\hurstlow}{0.707509}
\renewcommand{\hurstlowtwo}{1.415018}
\renewcommand{\hurstlowhundred}{70.750900}
\renewcommand{\hcalcall}{0.184942}
\renewcommand{\hcalclow}{0.102042}
\renewcommand{\shannonmax}{0.604167}
\renewcommand{\twoponemax}{0.208334}
\renewcommand{\logreturns}{0.010456}
\renewcommand{\twologreturns}{1.007274}
\renewcommand{\twologreturnshundred}{0.727387}
\renewcommand{\oneoverlogreturns}{95.638868}
\renewcommand{\pmax}{0.602094}
\renewcommand{\twopminusone}{0.204188}
\renewcommand{\rmsp}{0.008027}
\renewcommand{\twopx}{0.208583}
\renewcommand{\sigmap}{0.008047}
\renewcommand{\tsunfairbrownianfractionmean}{0.007862}
\renewcommand{\tsunfairbrownianfractionstddev}{0.038619}
\renewcommand{\shannonlogreturns}{0.560125}
\renewcommand{\shannonlogreturnshundred}{56.012500}
\renewcommand{\twopone}{0.120250}
\renewcommand{\twoponehundred}{12.025000}
\renewcommand{\hundredtwoponehundred}{87.975000}
\renewcommand{\hundredshannonlogreturnshundred}{43.987500}
\renewcommand{\datatslsqepbits}{0.007623}
\renewcommand{\thurstall}{0.633980}
\renewcommand{\thurstlow}{0.710108}
\renewcommand{\thurstlowtwo}{1.420216}
\renewcommand{\thurstlowhundred}{71.010800}
\renewcommand{\thcalcall}{0.247886}
\renewcommand{\thcalclow}{0.171737}
\renewcommand{\chisquared}{2.862000}
\renewcommand{\critical}{42.557000}

    \renewcommand{\timescale}{month}
    \subidx{market}{\market}
    \idx{\market}

    \section{\market}

        \renewcommand{\SETLABEL}{\LABPRE:USM2}
        \renewcommand{\SETLABELQ}{\LABPRE:USM2Q}
        \label{\SETLABEL}

        \idx{United States Federal Reserve Board}
        For the analysis, the data was in the directory
        {\directory}\footnote{Data from the United States Federal
        Reserve Board, 1980---1994, by {\timescale}s, in billions of
        1987 dollars, US.}.

        The data in this section is presented in
        Section~\ref{\SETLABELREF}.

        %
% -----------------------------------------------------------------------------
%
% A license is hereby granted to reproduce this software source code and
% to create executable versions from this source code for personal,
% non-commercial use.  The copyright notice included with the software
% must be maintained in all copies produced.
%
% THIS PROGRAM IS PROVIDED "AS IS". THE AUTHOR PROVIDES NO WARRANTIES
% WHATSOEVER, EXPRESSED OR IMPLIED, INCLUDING WARRANTIES OF
% MERCHANTABILITY, TITLE, OR FITNESS FOR ANY PARTICULAR PURPOSE.  THE
% AUTHOR DOES NOT WARRANT THAT USE OF THIS PROGRAM DOES NOT INFRINGE THE
% INTELLECTUAL PROPERTY RIGHTS OF ANY THIRD PARTY IN ANY COUNTRY.
%
% Copyright (c) 1994-2006, John Conover, All Rights Reserved.
%
% Comments and/or bug reports should be addressed to:
%
%     john@email.johncon.com (John Conover)
%
% -----------------------------------------------------------------------------
%
% Revision: \RCSRevision \\
% Revision Time: \RCSTime UMT \\
% Revision Date: \RCSDate \\
% Revision Id: \RCSId \\
% Revision File: \RCSLog \\
\RCS $Revision: 0.0 $
\RCS $Date: 2006/01/20 04:38:13 $
\RCS $Id: tables.tex,v 0.0 2006/01/20 04:38:13 john Exp $
% $Log: tables.tex,v $
% Revision 0.0  2006/01/20 04:38:13  john
% Initial version
%
%
    \subsection{{\market}, normalized increments}
        \label{\SETLABEL:TSA}

        The data in table~\ref{\SETLABEL:INC} is condensed from
        Section~\ref{\SETLABELREF:TSA}.

        \begin{small}
            \begin{table}[ht]
                \begin{center}
                    \caption[{\market}, normalized increments]
                        {{\market}, normalized increments.}
                    \begin{tabular}{|c|c|c|c|c|c|c|c|c|c|} \hline
                        \multicolumn{5}{|c|}{Normalized}                                                                                  & \multicolumn{5}{|c|}{Normalized Absolute Value}\\ \hline
                        Mean                & Standard              & rms                & \multicolumn{2}{|c|}{Least Squares}            & Mean                   & Standard                 & rms                & \multicolumn{2}{|c|}{Least Squares} \\ \cline{4-5}\cline{9-10}
                        \hspace{0.01in}     & deviation             & \hspace{0.01in}    & Constant                & Slope                & \hspace{0.01in}        & deviation                & \hspace{0.01in}    & Constant                   & Slope \\ \hline\hline
                        {\datafractionmean} & {\datafractionstddev} & {\datafractionrms} & {\datafractionconstant} & {\datafractionslope} & {\datafractionabsmean} & {\datafractionabsstddev} & {\datafractionrms} & {\datafractionabsconstant} & {\datafractionabsslope} \\ \hline
                    \end{tabular}
                    \label{\SETLABEL:INC}
                \end{center}
            \end{table}
        \end{small}

    \subsection{{\market}, Logarithmic Returns, in Bits}
        \label{\SETLABEL:LR}

        The data in table~\ref{\SETLABEL:RET} is condensed from
        Section~\ref{\SETLABELREF:FS}.

        \begin{small}
            \begin{table}[ht]
                \begin{center}
                    \caption[{\market}, Logarithmic Returns, in
                        Bits]{{\market}, Logarithmic Returns, in Bits.}
                    \begin{tabular}{|c|c|c|c|} \hline
                        \multicolumn{2}{|c|}{Calculated from Table~\ref{\SETLABEL:INC}} & \multicolumn{2}{|c|}{From program:}\\ \hline
                        Mean                    & Least squares                       & {\it tslsq}\/              & {\it tslogreturns}\/ \\ \hline\hline
                        {\datafractionmeanbits} & {\datafractionconstantbits} & {\datatslsqepbits} & {\logreturns} \\ \hline
                    \end{tabular}
                    \label{\SETLABEL:RET}
                \end{center}
            \end{table}
        \end{small}

    \subsection{{\market}, Shannon probabilities}
        \label{\SETLABEL:MAXSHANNON}

        The data in table~\ref{\SETLABEL:SHANNON} is condensed from
        sections~\ref{\SETLABELREF:FS}
        and~\ref{\SETLABELREF:MAXSHANNON}.

        \begin{small}
            \begin{table}[ht]
                \begin{center}
                    \caption[{\market}, Shannon
                        probabilities]{{\market}, Shannon
                        probabilities.}
                    \begin{tabular}{|c|c|c|c|} \hline
                        \multicolumn{3}{|c|}{Maximum} & \multicolumn{1}{|c|}{Operational}\\ \hline
                        Fraction of         & $\frac{\frac{\mbox{\scriptsize{mean}}}{\mbox{\scriptsize{rms}}} + 1}{2}$ & \multicolumn{2}{|c|}{From program:}\\ \cline{3-4}
                        positive increments & \hspace{0.01in}                                                          & {\it tsshannonmax}\/    & {\it tsshannon}\/ \\ \hline\hline
                        {\pmax}             & {\avgrms}                                                                & {\shannonmax}   & {\shannonlogreturns} \\ \hline
                    \end{tabular}
                    \label{\SETLABEL:SHANNON}
                \end{center}
            \end{table}
        \end{small}

    \subsection{{\market}, Logistic Analysis}
        \label{\SETLABEL:LAA}

        The data in table~\ref{\SETLABEL:LA} is condensed from
        Section~\ref{\SETLABELREF:LA}\footnote{Note that there are
        numerical stability issues with the methodology used to derive
        the constants---if the non-linear term, $b$, was greater than
        zero, it was set to zero to produce the graphs in
        Section~\ref{\SETLABELREF:LA}.}.

        \begin{small}
            \begin{table}[ht]
                \begin{center}
                    \caption[{\market}, Logistic Analysis.]
                        {{\market}, Logistic Analysis, $x_t = x_{t - 1}\left(a + b \cdot x_{t - 1}\right)$.}
                    \begin{tabular}{|c|c|} \hline
                        $a$ & $b$\\ \hline\hline
                        {\datafractionconstant} & {\datafractionslope}\\ \hline
                    \end{tabular}
                    \label{\SETLABEL:LA}
                \end{center}
            \end{table}
        \end{small}

    \subsection{{\market}, Hurst Coefficients and H  Parameters}
        \label{\SETLABEL:HCHP}

        The data in table~\ref{\SETLABEL:H} is condensed from
        Section~\ref{\SETLABELREF:H}.

        \begin{small}
            \begin{table}[ht]
                \begin{center}
                    \caption[{\market}, Hurst Coefficients and H
                        Parameters]{{\market}, Hurst Coefficients and
                        H Parameters.}
                    \begin{tabular}{|c|c|c|c|} \hline
                        \multicolumn{2}{|c|}{Hurst Coefficients} & \multicolumn{2}{|c|}{H Parameters}\\ \hline
                        Near term   & Far term    & Near term   & Far term \\ \hline\hline
                        {\thurstlow} & {\thurstall} & {\thcalclow} & {\thcalcall} \\ \hline
                    \end{tabular}
                    \label{\SETLABEL:H}
                \end{center}
            \end{table}
        \end{small}

        \begin{small}
            \begin{table}[ht]
                \begin{center}
                    \caption[{\market}, Hurst Coefficients and H
                        Parameters]{{\market}, Hurst Coefficients and
                        H Parameters, as a Derivative.}
                    \begin{tabular}{|c|c|c|c|} \hline
                        \multicolumn{2}{|c|}{Hurst Coefficients} & \multicolumn{2}{|c|}{H Parameters}\\ \hline
                        Near term    & Far term     & Near term    & Far term \\ \hline\hline
                        {\hurstlow} & {\hurstall} & {\hcalclow} & {\hcalcall} \\ \hline
                    \end{tabular}
                    \label{\SETLABEL:TH}
                \end{center}
            \end{table}
        \end{small}

    \subsection{{\market}, verification of the increments}
        \label{\SETLABEL:VI1}

        The data in table~\ref{\SETLABEL:COMP} is condensed from
        Section~\ref{\SETLABELREF:QVA}.

        \begin{small}
            \begin{table}[ht]
                \begin{center}
                    \caption[{\market}, verification of
                        the increments]{{\market}, verification the of
                        the increments, the mean, $\sigma$ is the
                        standard deviation from
                        table~\ref{\SETLABEL:INC},
                        {\datafractionstddev}, and $P$ is the maximum
                        Shannon probability from
                        table~\ref{\SETLABEL:SHANNON}, {\pmax}. In
                        principle, the values should equate.}
                    \begin{tabular}{|c|c|c|} \hline
                        Mean                & $\mbox{rms} (2P - 1)$ & $\frac{{\sigma}(2P - 1)}{2\sqrt{P(P - 1)}} $ \\ \hline\hline
                        {\datafractionmean} & {\rmsp}               & {\sigmap} \\ \hline
                    \end{tabular}
                    \label{\SETLABEL:COMP}
                \end{center}
            \end{table}
        \end{small}

    \subsection{{\market}, verification of the increments}
        \label{\SETLABEL:VI2}

        The data in table~\ref{\SETLABEL:ABS} is condensed from
        Section~\ref{\SETLABELREF:QVA}.

        \begin{small}
            \begin{table}[ht]
                \begin{center}
                    \caption[{\market}, verification of
                        the increments]{{\market}, verification the of
                        increments. In principle, the mean of the
                        absolute value of the increments and the root
                        mean square of the increments should
                        equate\footnote{The absolute value of the
                        normalized increments, when averaged, is
                        related to the root mean square of the
                        increments by a constant. If the normalized
                        increments are a fixed increment, the constant
                        is unity. If the normalized increments have a
                        Gaussian distribution, the constant is
                        $\approx 0.8$ depending on the accuracy of of
                        ``fit'' to a Gaussian distribution.}.}
                    \begin{tabular}{|c|c|} \hline
                        Mean of the               & rms \\
                        absolute value            & \hspace{0.01in} \\ \hline\hline
                        {\datafractionabsmean}    & {\datafractionrms} \\ \hline
                    \end{tabular}
                    \label{\SETLABEL:ABS}
                \end{center}
            \end{table}
        \end{small}

    \subsection{{\market}, $\chi^2$ values of the increments}
        \label{\SETLABEL:XSQ}

        The data in table~\ref{\SETLABEL:XSQT} is condensed from
        Section~\ref{\SETLABELREF:NH}.

        \begin{small}
            \begin{table}[ht]
                \begin{center}
                    \caption[{\market}, $\chi^2$ values of
                        the increments]{{\market}, $\chi^2$ values of
                        the increments. In principle, if the
                        distribution of the normalized increments is a
                        Gaussian distribution, the $\chi^2$ value will
                        be significantly less than the critical
                        value.}
                    \begin{tabular}{|c|c|} \hline
                        $\chi^2$      & Critical Value \\ \hline\hline
                        {\chisquared} & {\critical} \\ \hline
                    \end{tabular}
                    \label{\SETLABEL:XSQT}
                \end{center}
            \end{table}
        \end{small}

    \subsection{{\market}, time series data, empirical and simulated}
        \label{\SETLABEL:SIM}

        The data in table~\ref{\SETLABEL:ES} is condensed from
        Section~\ref{\SETLABELREF:TSUNFAIRBROWNIAN}.

        \begin{small}
            \begin{table}[ht]
                \begin{center}
                    \caption[{\market}, time series data, empirical
                        and simulated]{{\market}, time series data,
                        empirical and simulated, analysis of the
                        normalized increments.}
                    \begin{tabular}{|c|c|c|c|} \hline
                        \multicolumn{2}{|c|}{Empirical} & \multicolumn{2}{|c|}{Simulated}\\ \hline
                        Mean                & Standard              & Mean               & Standard \\
                        \hspace{0.01in}     & deviation             & \hspace{0.01in}    & deviation \\ \hline\hline
                        {\datafractionmean} & {\datafractionstddev} & {\tsunfairbrownianfractionmean} & {\tsunfairbrownianfractionstddev} \\ \hline
                    \end{tabular}
                    \label{\SETLABEL:ES}
                \end{center}
            \end{table}
        \end{small}

    \subsection{{\market}, number of participating companies}
        \label{\SETLABEL:QNC}

        The data in table~\ref{\SETLABEL:NC} is condensed from
        Section~\ref{\SETLABELREF:QNC}.

        \begin{small}
            \begin{table}[ht]
                \begin{center}
                    \caption[{\market}, number of participating
                        companies] {{\market}, number of participating
                        companies.}
                    \begin{tabular}{|c|c|} \hline
                        Number & Shannon probability\\ \hline
                        {\ncompanies} & {\pncompanies}\\ \hline
                    \end{tabular}
                    \label{\SETLABEL:NC}
                \end{center}
            \end{table}
        \end{small}

    \subsection{{\market}, Shannon probability optimizations}
        \label{\SETLABEL:SPO}

        The data in table~\ref{\SETLABEL:SP} is condensed from
        Section~\ref{\SETLABELREF:QNC}.

        \begin{small}
            \begin{table}[ht]
                \begin{center}
                    \caption[{\market}, Shannon probability
                         optimizations] {{\market}, Shannon
                         probability optimization.}
                    \begin{tabular}{|c|c|} \hline
                        optimize capital growth & optimize market growth\\ \hline
                        {\avgrms} & {\pncompanies}\\ \hline
                    \end{tabular}
                    \label{\SETLABEL:SP}
                \end{center}
            \end{table}
        \end{small}

% Local Variables:
% TeX-parse-self: t
% TeX-auto-save: t
% TeX-master: "fractal.tex"
% End:


    \renewcommand{\market}{United States Treasury Bill Returns}
    \renewcommand{\directory}{../markets/us.tbill}
    \renewcommand{\datafractionmean}{0.008052}
\renewcommand{\datafractionmeanbits}{0.011570}
\renewcommand{\datafractionmeanq}{0.002684}
\renewcommand{\datafractionmeanbitsq}{0.003867}
\renewcommand{\datafractionstddev}{0.038579}
\renewcommand{\datafractionrms}{0.039311}
\renewcommand{\avgrms}{0.602414}
\renewcommand{\ncompanies}{5.210454}
\renewcommand{\pncompanies}{0.544866}
\renewcommand{\datafractionabsmean}{0.029745}
\renewcommand{\datafractionabsstddev}{0.025769}
\renewcommand{\datafractionconstant}{0.010041}
\renewcommand{\datafractionconstantbits}{0.014414}
\renewcommand{\datafractionconstantq}{0.003347}
\renewcommand{\datafractionconstantbitsq}{0.004821}
\renewcommand{\datafractionslope}{-0.000021}
\renewcommand{\datafractionabsconstant}{0.035145}
\renewcommand{\datafractionabsslope}{-0.000057}
\renewcommand{\hurstall}{0.659558}
\renewcommand{\hurstlow}{0.707509}
\renewcommand{\hurstlowtwo}{1.415018}
\renewcommand{\hurstlowhundred}{70.750900}
\renewcommand{\hcalcall}{0.184942}
\renewcommand{\hcalclow}{0.102042}
\renewcommand{\shannonmax}{0.604167}
\renewcommand{\twoponemax}{0.208334}
\renewcommand{\logreturns}{0.010456}
\renewcommand{\twologreturns}{1.007274}
\renewcommand{\twologreturnshundred}{0.727387}
\renewcommand{\oneoverlogreturns}{95.638868}
\renewcommand{\pmax}{0.602094}
\renewcommand{\twopminusone}{0.204188}
\renewcommand{\rmsp}{0.008027}
\renewcommand{\twopx}{0.208583}
\renewcommand{\sigmap}{0.008047}
\renewcommand{\tsunfairbrownianfractionmean}{0.007862}
\renewcommand{\tsunfairbrownianfractionstddev}{0.038619}
\renewcommand{\shannonlogreturns}{0.560125}
\renewcommand{\shannonlogreturnshundred}{56.012500}
\renewcommand{\twopone}{0.120250}
\renewcommand{\twoponehundred}{12.025000}
\renewcommand{\hundredtwoponehundred}{87.975000}
\renewcommand{\hundredshannonlogreturnshundred}{43.987500}
\renewcommand{\datatslsqepbits}{0.007623}
\renewcommand{\thurstall}{0.633980}
\renewcommand{\thurstlow}{0.710108}
\renewcommand{\thurstlowtwo}{1.420216}
\renewcommand{\thurstlowhundred}{71.010800}
\renewcommand{\thcalcall}{0.247886}
\renewcommand{\thcalclow}{0.171737}
\renewcommand{\chisquared}{2.862000}
\renewcommand{\critical}{42.557000}

    \renewcommand{\timescale}{month}
    \subidx{market}{\market}
    \idx{\market}

    \section{\market}

        \renewcommand{\SETLABEL}{\LABPRE:USTBILL}
        \renewcommand{\SETLABELQ}{\LABPRE:USTBILLQ}
        \label{\SETLABEL}

        \idx{United States Federal Reserve Board}
        For the analysis, the data was in the directory
        {\directory}\footnote{Data from the United States Federal
        Reserve Board, 1980---1994, by {\timescale}s, in percent. The
        time series, which was Treasury Bill rate of returns, in
        percent per year, was converted to cumulative growth per month
        by converting each element in the time series to a fraction,
        dividing by 12, and adding 1. The previous value of cumulative
        returns was multiplied by this number for the next value of
        cumulative returns.}.

        The data in this section is presented in
        Section~\ref{\SETLABELREF}.

        %
% -----------------------------------------------------------------------------
%
% A license is hereby granted to reproduce this software source code and
% to create executable versions from this source code for personal,
% non-commercial use.  The copyright notice included with the software
% must be maintained in all copies produced.
%
% THIS PROGRAM IS PROVIDED "AS IS". THE AUTHOR PROVIDES NO WARRANTIES
% WHATSOEVER, EXPRESSED OR IMPLIED, INCLUDING WARRANTIES OF
% MERCHANTABILITY, TITLE, OR FITNESS FOR ANY PARTICULAR PURPOSE.  THE
% AUTHOR DOES NOT WARRANT THAT USE OF THIS PROGRAM DOES NOT INFRINGE THE
% INTELLECTUAL PROPERTY RIGHTS OF ANY THIRD PARTY IN ANY COUNTRY.
%
% Copyright (c) 1994-2006, John Conover, All Rights Reserved.
%
% Comments and/or bug reports should be addressed to:
%
%     john@email.johncon.com (John Conover)
%
% -----------------------------------------------------------------------------
%
% Revision: \RCSRevision \\
% Revision Time: \RCSTime UMT \\
% Revision Date: \RCSDate \\
% Revision Id: \RCSId \\
% Revision File: \RCSLog \\
\RCS $Revision: 0.0 $
\RCS $Date: 2006/01/20 04:38:13 $
\RCS $Id: tables.tex,v 0.0 2006/01/20 04:38:13 john Exp $
% $Log: tables.tex,v $
% Revision 0.0  2006/01/20 04:38:13  john
% Initial version
%
%
    \subsection{{\market}, normalized increments}
        \label{\SETLABEL:TSA}

        The data in table~\ref{\SETLABEL:INC} is condensed from
        Section~\ref{\SETLABELREF:TSA}.

        \begin{small}
            \begin{table}[ht]
                \begin{center}
                    \caption[{\market}, normalized increments]
                        {{\market}, normalized increments.}
                    \begin{tabular}{|c|c|c|c|c|c|c|c|c|c|} \hline
                        \multicolumn{5}{|c|}{Normalized}                                                                                  & \multicolumn{5}{|c|}{Normalized Absolute Value}\\ \hline
                        Mean                & Standard              & rms                & \multicolumn{2}{|c|}{Least Squares}            & Mean                   & Standard                 & rms                & \multicolumn{2}{|c|}{Least Squares} \\ \cline{4-5}\cline{9-10}
                        \hspace{0.01in}     & deviation             & \hspace{0.01in}    & Constant                & Slope                & \hspace{0.01in}        & deviation                & \hspace{0.01in}    & Constant                   & Slope \\ \hline\hline
                        {\datafractionmean} & {\datafractionstddev} & {\datafractionrms} & {\datafractionconstant} & {\datafractionslope} & {\datafractionabsmean} & {\datafractionabsstddev} & {\datafractionrms} & {\datafractionabsconstant} & {\datafractionabsslope} \\ \hline
                    \end{tabular}
                    \label{\SETLABEL:INC}
                \end{center}
            \end{table}
        \end{small}

    \subsection{{\market}, Logarithmic Returns, in Bits}
        \label{\SETLABEL:LR}

        The data in table~\ref{\SETLABEL:RET} is condensed from
        Section~\ref{\SETLABELREF:FS}.

        \begin{small}
            \begin{table}[ht]
                \begin{center}
                    \caption[{\market}, Logarithmic Returns, in
                        Bits]{{\market}, Logarithmic Returns, in Bits.}
                    \begin{tabular}{|c|c|c|c|} \hline
                        \multicolumn{2}{|c|}{Calculated from Table~\ref{\SETLABEL:INC}} & \multicolumn{2}{|c|}{From program:}\\ \hline
                        Mean                    & Least squares                       & {\it tslsq}\/              & {\it tslogreturns}\/ \\ \hline\hline
                        {\datafractionmeanbits} & {\datafractionconstantbits} & {\datatslsqepbits} & {\logreturns} \\ \hline
                    \end{tabular}
                    \label{\SETLABEL:RET}
                \end{center}
            \end{table}
        \end{small}

    \subsection{{\market}, Shannon probabilities}
        \label{\SETLABEL:MAXSHANNON}

        The data in table~\ref{\SETLABEL:SHANNON} is condensed from
        sections~\ref{\SETLABELREF:FS}
        and~\ref{\SETLABELREF:MAXSHANNON}.

        \begin{small}
            \begin{table}[ht]
                \begin{center}
                    \caption[{\market}, Shannon
                        probabilities]{{\market}, Shannon
                        probabilities.}
                    \begin{tabular}{|c|c|c|c|} \hline
                        \multicolumn{3}{|c|}{Maximum} & \multicolumn{1}{|c|}{Operational}\\ \hline
                        Fraction of         & $\frac{\frac{\mbox{\scriptsize{mean}}}{\mbox{\scriptsize{rms}}} + 1}{2}$ & \multicolumn{2}{|c|}{From program:}\\ \cline{3-4}
                        positive increments & \hspace{0.01in}                                                          & {\it tsshannonmax}\/    & {\it tsshannon}\/ \\ \hline\hline
                        {\pmax}             & {\avgrms}                                                                & {\shannonmax}   & {\shannonlogreturns} \\ \hline
                    \end{tabular}
                    \label{\SETLABEL:SHANNON}
                \end{center}
            \end{table}
        \end{small}

    \subsection{{\market}, Logistic Analysis}
        \label{\SETLABEL:LAA}

        The data in table~\ref{\SETLABEL:LA} is condensed from
        Section~\ref{\SETLABELREF:LA}\footnote{Note that there are
        numerical stability issues with the methodology used to derive
        the constants---if the non-linear term, $b$, was greater than
        zero, it was set to zero to produce the graphs in
        Section~\ref{\SETLABELREF:LA}.}.

        \begin{small}
            \begin{table}[ht]
                \begin{center}
                    \caption[{\market}, Logistic Analysis.]
                        {{\market}, Logistic Analysis, $x_t = x_{t - 1}\left(a + b \cdot x_{t - 1}\right)$.}
                    \begin{tabular}{|c|c|} \hline
                        $a$ & $b$\\ \hline\hline
                        {\datafractionconstant} & {\datafractionslope}\\ \hline
                    \end{tabular}
                    \label{\SETLABEL:LA}
                \end{center}
            \end{table}
        \end{small}

    \subsection{{\market}, Hurst Coefficients and H  Parameters}
        \label{\SETLABEL:HCHP}

        The data in table~\ref{\SETLABEL:H} is condensed from
        Section~\ref{\SETLABELREF:H}.

        \begin{small}
            \begin{table}[ht]
                \begin{center}
                    \caption[{\market}, Hurst Coefficients and H
                        Parameters]{{\market}, Hurst Coefficients and
                        H Parameters.}
                    \begin{tabular}{|c|c|c|c|} \hline
                        \multicolumn{2}{|c|}{Hurst Coefficients} & \multicolumn{2}{|c|}{H Parameters}\\ \hline
                        Near term   & Far term    & Near term   & Far term \\ \hline\hline
                        {\thurstlow} & {\thurstall} & {\thcalclow} & {\thcalcall} \\ \hline
                    \end{tabular}
                    \label{\SETLABEL:H}
                \end{center}
            \end{table}
        \end{small}

        \begin{small}
            \begin{table}[ht]
                \begin{center}
                    \caption[{\market}, Hurst Coefficients and H
                        Parameters]{{\market}, Hurst Coefficients and
                        H Parameters, as a Derivative.}
                    \begin{tabular}{|c|c|c|c|} \hline
                        \multicolumn{2}{|c|}{Hurst Coefficients} & \multicolumn{2}{|c|}{H Parameters}\\ \hline
                        Near term    & Far term     & Near term    & Far term \\ \hline\hline
                        {\hurstlow} & {\hurstall} & {\hcalclow} & {\hcalcall} \\ \hline
                    \end{tabular}
                    \label{\SETLABEL:TH}
                \end{center}
            \end{table}
        \end{small}

    \subsection{{\market}, verification of the increments}
        \label{\SETLABEL:VI1}

        The data in table~\ref{\SETLABEL:COMP} is condensed from
        Section~\ref{\SETLABELREF:QVA}.

        \begin{small}
            \begin{table}[ht]
                \begin{center}
                    \caption[{\market}, verification of
                        the increments]{{\market}, verification the of
                        the increments, the mean, $\sigma$ is the
                        standard deviation from
                        table~\ref{\SETLABEL:INC},
                        {\datafractionstddev}, and $P$ is the maximum
                        Shannon probability from
                        table~\ref{\SETLABEL:SHANNON}, {\pmax}. In
                        principle, the values should equate.}
                    \begin{tabular}{|c|c|c|} \hline
                        Mean                & $\mbox{rms} (2P - 1)$ & $\frac{{\sigma}(2P - 1)}{2\sqrt{P(P - 1)}} $ \\ \hline\hline
                        {\datafractionmean} & {\rmsp}               & {\sigmap} \\ \hline
                    \end{tabular}
                    \label{\SETLABEL:COMP}
                \end{center}
            \end{table}
        \end{small}

    \subsection{{\market}, verification of the increments}
        \label{\SETLABEL:VI2}

        The data in table~\ref{\SETLABEL:ABS} is condensed from
        Section~\ref{\SETLABELREF:QVA}.

        \begin{small}
            \begin{table}[ht]
                \begin{center}
                    \caption[{\market}, verification of
                        the increments]{{\market}, verification the of
                        increments. In principle, the mean of the
                        absolute value of the increments and the root
                        mean square of the increments should
                        equate\footnote{The absolute value of the
                        normalized increments, when averaged, is
                        related to the root mean square of the
                        increments by a constant. If the normalized
                        increments are a fixed increment, the constant
                        is unity. If the normalized increments have a
                        Gaussian distribution, the constant is
                        $\approx 0.8$ depending on the accuracy of of
                        ``fit'' to a Gaussian distribution.}.}
                    \begin{tabular}{|c|c|} \hline
                        Mean of the               & rms \\
                        absolute value            & \hspace{0.01in} \\ \hline\hline
                        {\datafractionabsmean}    & {\datafractionrms} \\ \hline
                    \end{tabular}
                    \label{\SETLABEL:ABS}
                \end{center}
            \end{table}
        \end{small}

    \subsection{{\market}, $\chi^2$ values of the increments}
        \label{\SETLABEL:XSQ}

        The data in table~\ref{\SETLABEL:XSQT} is condensed from
        Section~\ref{\SETLABELREF:NH}.

        \begin{small}
            \begin{table}[ht]
                \begin{center}
                    \caption[{\market}, $\chi^2$ values of
                        the increments]{{\market}, $\chi^2$ values of
                        the increments. In principle, if the
                        distribution of the normalized increments is a
                        Gaussian distribution, the $\chi^2$ value will
                        be significantly less than the critical
                        value.}
                    \begin{tabular}{|c|c|} \hline
                        $\chi^2$      & Critical Value \\ \hline\hline
                        {\chisquared} & {\critical} \\ \hline
                    \end{tabular}
                    \label{\SETLABEL:XSQT}
                \end{center}
            \end{table}
        \end{small}

    \subsection{{\market}, time series data, empirical and simulated}
        \label{\SETLABEL:SIM}

        The data in table~\ref{\SETLABEL:ES} is condensed from
        Section~\ref{\SETLABELREF:TSUNFAIRBROWNIAN}.

        \begin{small}
            \begin{table}[ht]
                \begin{center}
                    \caption[{\market}, time series data, empirical
                        and simulated]{{\market}, time series data,
                        empirical and simulated, analysis of the
                        normalized increments.}
                    \begin{tabular}{|c|c|c|c|} \hline
                        \multicolumn{2}{|c|}{Empirical} & \multicolumn{2}{|c|}{Simulated}\\ \hline
                        Mean                & Standard              & Mean               & Standard \\
                        \hspace{0.01in}     & deviation             & \hspace{0.01in}    & deviation \\ \hline\hline
                        {\datafractionmean} & {\datafractionstddev} & {\tsunfairbrownianfractionmean} & {\tsunfairbrownianfractionstddev} \\ \hline
                    \end{tabular}
                    \label{\SETLABEL:ES}
                \end{center}
            \end{table}
        \end{small}

    \subsection{{\market}, number of participating companies}
        \label{\SETLABEL:QNC}

        The data in table~\ref{\SETLABEL:NC} is condensed from
        Section~\ref{\SETLABELREF:QNC}.

        \begin{small}
            \begin{table}[ht]
                \begin{center}
                    \caption[{\market}, number of participating
                        companies] {{\market}, number of participating
                        companies.}
                    \begin{tabular}{|c|c|} \hline
                        Number & Shannon probability\\ \hline
                        {\ncompanies} & {\pncompanies}\\ \hline
                    \end{tabular}
                    \label{\SETLABEL:NC}
                \end{center}
            \end{table}
        \end{small}

    \subsection{{\market}, Shannon probability optimizations}
        \label{\SETLABEL:SPO}

        The data in table~\ref{\SETLABEL:SP} is condensed from
        Section~\ref{\SETLABELREF:QNC}.

        \begin{small}
            \begin{table}[ht]
                \begin{center}
                    \caption[{\market}, Shannon probability
                         optimizations] {{\market}, Shannon
                         probability optimization.}
                    \begin{tabular}{|c|c|} \hline
                        optimize capital growth & optimize market growth\\ \hline
                        {\avgrms} & {\pncompanies}\\ \hline
                    \end{tabular}
                    \label{\SETLABEL:SP}
                \end{center}
            \end{table}
        \end{small}

% Local Variables:
% TeX-parse-self: t
% TeX-auto-save: t
% TeX-master: "fractal.tex"
% End:


    \renewcommand{\market}{Coin Tossing Game}
    \renewcommand{\directory}{../markets/tscoin}
    \renewcommand{\datafractionmean}{0.008052}
\renewcommand{\datafractionmeanbits}{0.011570}
\renewcommand{\datafractionmeanq}{0.002684}
\renewcommand{\datafractionmeanbitsq}{0.003867}
\renewcommand{\datafractionstddev}{0.038579}
\renewcommand{\datafractionrms}{0.039311}
\renewcommand{\avgrms}{0.602414}
\renewcommand{\ncompanies}{5.210454}
\renewcommand{\pncompanies}{0.544866}
\renewcommand{\datafractionabsmean}{0.029745}
\renewcommand{\datafractionabsstddev}{0.025769}
\renewcommand{\datafractionconstant}{0.010041}
\renewcommand{\datafractionconstantbits}{0.014414}
\renewcommand{\datafractionconstantq}{0.003347}
\renewcommand{\datafractionconstantbitsq}{0.004821}
\renewcommand{\datafractionslope}{-0.000021}
\renewcommand{\datafractionabsconstant}{0.035145}
\renewcommand{\datafractionabsslope}{-0.000057}
\renewcommand{\hurstall}{0.659558}
\renewcommand{\hurstlow}{0.707509}
\renewcommand{\hurstlowtwo}{1.415018}
\renewcommand{\hurstlowhundred}{70.750900}
\renewcommand{\hcalcall}{0.184942}
\renewcommand{\hcalclow}{0.102042}
\renewcommand{\shannonmax}{0.604167}
\renewcommand{\twoponemax}{0.208334}
\renewcommand{\logreturns}{0.010456}
\renewcommand{\twologreturns}{1.007274}
\renewcommand{\twologreturnshundred}{0.727387}
\renewcommand{\oneoverlogreturns}{95.638868}
\renewcommand{\pmax}{0.602094}
\renewcommand{\twopminusone}{0.204188}
\renewcommand{\rmsp}{0.008027}
\renewcommand{\twopx}{0.208583}
\renewcommand{\sigmap}{0.008047}
\renewcommand{\tsunfairbrownianfractionmean}{0.007862}
\renewcommand{\tsunfairbrownianfractionstddev}{0.038619}
\renewcommand{\shannonlogreturns}{0.560125}
\renewcommand{\shannonlogreturnshundred}{56.012500}
\renewcommand{\twopone}{0.120250}
\renewcommand{\twoponehundred}{12.025000}
\renewcommand{\hundredtwoponehundred}{87.975000}
\renewcommand{\hundredshannonlogreturnshundred}{43.987500}
\renewcommand{\datatslsqepbits}{0.007623}
\renewcommand{\thurstall}{0.633980}
\renewcommand{\thurstlow}{0.710108}
\renewcommand{\thurstlowtwo}{1.420216}
\renewcommand{\thurstlowhundred}{71.010800}
\renewcommand{\thcalcall}{0.247886}
\renewcommand{\thcalclow}{0.171737}
\renewcommand{\chisquared}{2.862000}
\renewcommand{\critical}{42.557000}

    \renewcommand{\timescale}{tosses}
    \subidx{market}{\market}
    \idx{\market}

    \section{\market}

        \renewcommand{\SETLABEL}{\LABPRE:CT}
        \renewcommand{\SETLABELQ}{\LABPRE:CTQ}
        \label{\SETLABEL}

        \subidx{tscoin}{program}
        \subidx{programs}{tscoin}
        For the analysis, the data was in the directory
        {\directory}\footnote{As a simulation model, the program
        {\it tscoin}\/ was run to make a time series data file.  The data is
        by {\timescale}.}.

        The data in this section is presented in
        Section~\ref{\SETLABELREF}.

        %
% -----------------------------------------------------------------------------
%
% A license is hereby granted to reproduce this software source code and
% to create executable versions from this source code for personal,
% non-commercial use.  The copyright notice included with the software
% must be maintained in all copies produced.
%
% THIS PROGRAM IS PROVIDED "AS IS". THE AUTHOR PROVIDES NO WARRANTIES
% WHATSOEVER, EXPRESSED OR IMPLIED, INCLUDING WARRANTIES OF
% MERCHANTABILITY, TITLE, OR FITNESS FOR ANY PARTICULAR PURPOSE.  THE
% AUTHOR DOES NOT WARRANT THAT USE OF THIS PROGRAM DOES NOT INFRINGE THE
% INTELLECTUAL PROPERTY RIGHTS OF ANY THIRD PARTY IN ANY COUNTRY.
%
% Copyright (c) 1994-2006, John Conover, All Rights Reserved.
%
% Comments and/or bug reports should be addressed to:
%
%     john@email.johncon.com (John Conover)
%
% -----------------------------------------------------------------------------
%
% Revision: \RCSRevision \\
% Revision Time: \RCSTime UMT \\
% Revision Date: \RCSDate \\
% Revision Id: \RCSId \\
% Revision File: \RCSLog \\
\RCS $Revision: 0.0 $
\RCS $Date: 2006/01/20 04:38:13 $
\RCS $Id: tables.tex,v 0.0 2006/01/20 04:38:13 john Exp $
% $Log: tables.tex,v $
% Revision 0.0  2006/01/20 04:38:13  john
% Initial version
%
%
    \subsection{{\market}, normalized increments}
        \label{\SETLABEL:TSA}

        The data in table~\ref{\SETLABEL:INC} is condensed from
        Section~\ref{\SETLABELREF:TSA}.

        \begin{small}
            \begin{table}[ht]
                \begin{center}
                    \caption[{\market}, normalized increments]
                        {{\market}, normalized increments.}
                    \begin{tabular}{|c|c|c|c|c|c|c|c|c|c|} \hline
                        \multicolumn{5}{|c|}{Normalized}                                                                                  & \multicolumn{5}{|c|}{Normalized Absolute Value}\\ \hline
                        Mean                & Standard              & rms                & \multicolumn{2}{|c|}{Least Squares}            & Mean                   & Standard                 & rms                & \multicolumn{2}{|c|}{Least Squares} \\ \cline{4-5}\cline{9-10}
                        \hspace{0.01in}     & deviation             & \hspace{0.01in}    & Constant                & Slope                & \hspace{0.01in}        & deviation                & \hspace{0.01in}    & Constant                   & Slope \\ \hline\hline
                        {\datafractionmean} & {\datafractionstddev} & {\datafractionrms} & {\datafractionconstant} & {\datafractionslope} & {\datafractionabsmean} & {\datafractionabsstddev} & {\datafractionrms} & {\datafractionabsconstant} & {\datafractionabsslope} \\ \hline
                    \end{tabular}
                    \label{\SETLABEL:INC}
                \end{center}
            \end{table}
        \end{small}

    \subsection{{\market}, Logarithmic Returns, in Bits}
        \label{\SETLABEL:LR}

        The data in table~\ref{\SETLABEL:RET} is condensed from
        Section~\ref{\SETLABELREF:FS}.

        \begin{small}
            \begin{table}[ht]
                \begin{center}
                    \caption[{\market}, Logarithmic Returns, in
                        Bits]{{\market}, Logarithmic Returns, in Bits.}
                    \begin{tabular}{|c|c|c|c|} \hline
                        \multicolumn{2}{|c|}{Calculated from Table~\ref{\SETLABEL:INC}} & \multicolumn{2}{|c|}{From program:}\\ \hline
                        Mean                    & Least squares                       & {\it tslsq}\/              & {\it tslogreturns}\/ \\ \hline\hline
                        {\datafractionmeanbits} & {\datafractionconstantbits} & {\datatslsqepbits} & {\logreturns} \\ \hline
                    \end{tabular}
                    \label{\SETLABEL:RET}
                \end{center}
            \end{table}
        \end{small}

    \subsection{{\market}, Shannon probabilities}
        \label{\SETLABEL:MAXSHANNON}

        The data in table~\ref{\SETLABEL:SHANNON} is condensed from
        sections~\ref{\SETLABELREF:FS}
        and~\ref{\SETLABELREF:MAXSHANNON}.

        \begin{small}
            \begin{table}[ht]
                \begin{center}
                    \caption[{\market}, Shannon
                        probabilities]{{\market}, Shannon
                        probabilities.}
                    \begin{tabular}{|c|c|c|c|} \hline
                        \multicolumn{3}{|c|}{Maximum} & \multicolumn{1}{|c|}{Operational}\\ \hline
                        Fraction of         & $\frac{\frac{\mbox{\scriptsize{mean}}}{\mbox{\scriptsize{rms}}} + 1}{2}$ & \multicolumn{2}{|c|}{From program:}\\ \cline{3-4}
                        positive increments & \hspace{0.01in}                                                          & {\it tsshannonmax}\/    & {\it tsshannon}\/ \\ \hline\hline
                        {\pmax}             & {\avgrms}                                                                & {\shannonmax}   & {\shannonlogreturns} \\ \hline
                    \end{tabular}
                    \label{\SETLABEL:SHANNON}
                \end{center}
            \end{table}
        \end{small}

    \subsection{{\market}, Logistic Analysis}
        \label{\SETLABEL:LAA}

        The data in table~\ref{\SETLABEL:LA} is condensed from
        Section~\ref{\SETLABELREF:LA}\footnote{Note that there are
        numerical stability issues with the methodology used to derive
        the constants---if the non-linear term, $b$, was greater than
        zero, it was set to zero to produce the graphs in
        Section~\ref{\SETLABELREF:LA}.}.

        \begin{small}
            \begin{table}[ht]
                \begin{center}
                    \caption[{\market}, Logistic Analysis.]
                        {{\market}, Logistic Analysis, $x_t = x_{t - 1}\left(a + b \cdot x_{t - 1}\right)$.}
                    \begin{tabular}{|c|c|} \hline
                        $a$ & $b$\\ \hline\hline
                        {\datafractionconstant} & {\datafractionslope}\\ \hline
                    \end{tabular}
                    \label{\SETLABEL:LA}
                \end{center}
            \end{table}
        \end{small}

    \subsection{{\market}, Hurst Coefficients and H  Parameters}
        \label{\SETLABEL:HCHP}

        The data in table~\ref{\SETLABEL:H} is condensed from
        Section~\ref{\SETLABELREF:H}.

        \begin{small}
            \begin{table}[ht]
                \begin{center}
                    \caption[{\market}, Hurst Coefficients and H
                        Parameters]{{\market}, Hurst Coefficients and
                        H Parameters.}
                    \begin{tabular}{|c|c|c|c|} \hline
                        \multicolumn{2}{|c|}{Hurst Coefficients} & \multicolumn{2}{|c|}{H Parameters}\\ \hline
                        Near term   & Far term    & Near term   & Far term \\ \hline\hline
                        {\thurstlow} & {\thurstall} & {\thcalclow} & {\thcalcall} \\ \hline
                    \end{tabular}
                    \label{\SETLABEL:H}
                \end{center}
            \end{table}
        \end{small}

        \begin{small}
            \begin{table}[ht]
                \begin{center}
                    \caption[{\market}, Hurst Coefficients and H
                        Parameters]{{\market}, Hurst Coefficients and
                        H Parameters, as a Derivative.}
                    \begin{tabular}{|c|c|c|c|} \hline
                        \multicolumn{2}{|c|}{Hurst Coefficients} & \multicolumn{2}{|c|}{H Parameters}\\ \hline
                        Near term    & Far term     & Near term    & Far term \\ \hline\hline
                        {\hurstlow} & {\hurstall} & {\hcalclow} & {\hcalcall} \\ \hline
                    \end{tabular}
                    \label{\SETLABEL:TH}
                \end{center}
            \end{table}
        \end{small}

    \subsection{{\market}, verification of the increments}
        \label{\SETLABEL:VI1}

        The data in table~\ref{\SETLABEL:COMP} is condensed from
        Section~\ref{\SETLABELREF:QVA}.

        \begin{small}
            \begin{table}[ht]
                \begin{center}
                    \caption[{\market}, verification of
                        the increments]{{\market}, verification the of
                        the increments, the mean, $\sigma$ is the
                        standard deviation from
                        table~\ref{\SETLABEL:INC},
                        {\datafractionstddev}, and $P$ is the maximum
                        Shannon probability from
                        table~\ref{\SETLABEL:SHANNON}, {\pmax}. In
                        principle, the values should equate.}
                    \begin{tabular}{|c|c|c|} \hline
                        Mean                & $\mbox{rms} (2P - 1)$ & $\frac{{\sigma}(2P - 1)}{2\sqrt{P(P - 1)}} $ \\ \hline\hline
                        {\datafractionmean} & {\rmsp}               & {\sigmap} \\ \hline
                    \end{tabular}
                    \label{\SETLABEL:COMP}
                \end{center}
            \end{table}
        \end{small}

    \subsection{{\market}, verification of the increments}
        \label{\SETLABEL:VI2}

        The data in table~\ref{\SETLABEL:ABS} is condensed from
        Section~\ref{\SETLABELREF:QVA}.

        \begin{small}
            \begin{table}[ht]
                \begin{center}
                    \caption[{\market}, verification of
                        the increments]{{\market}, verification the of
                        increments. In principle, the mean of the
                        absolute value of the increments and the root
                        mean square of the increments should
                        equate\footnote{The absolute value of the
                        normalized increments, when averaged, is
                        related to the root mean square of the
                        increments by a constant. If the normalized
                        increments are a fixed increment, the constant
                        is unity. If the normalized increments have a
                        Gaussian distribution, the constant is
                        $\approx 0.8$ depending on the accuracy of of
                        ``fit'' to a Gaussian distribution.}.}
                    \begin{tabular}{|c|c|} \hline
                        Mean of the               & rms \\
                        absolute value            & \hspace{0.01in} \\ \hline\hline
                        {\datafractionabsmean}    & {\datafractionrms} \\ \hline
                    \end{tabular}
                    \label{\SETLABEL:ABS}
                \end{center}
            \end{table}
        \end{small}

    \subsection{{\market}, $\chi^2$ values of the increments}
        \label{\SETLABEL:XSQ}

        The data in table~\ref{\SETLABEL:XSQT} is condensed from
        Section~\ref{\SETLABELREF:NH}.

        \begin{small}
            \begin{table}[ht]
                \begin{center}
                    \caption[{\market}, $\chi^2$ values of
                        the increments]{{\market}, $\chi^2$ values of
                        the increments. In principle, if the
                        distribution of the normalized increments is a
                        Gaussian distribution, the $\chi^2$ value will
                        be significantly less than the critical
                        value.}
                    \begin{tabular}{|c|c|} \hline
                        $\chi^2$      & Critical Value \\ \hline\hline
                        {\chisquared} & {\critical} \\ \hline
                    \end{tabular}
                    \label{\SETLABEL:XSQT}
                \end{center}
            \end{table}
        \end{small}

    \subsection{{\market}, time series data, empirical and simulated}
        \label{\SETLABEL:SIM}

        The data in table~\ref{\SETLABEL:ES} is condensed from
        Section~\ref{\SETLABELREF:TSUNFAIRBROWNIAN}.

        \begin{small}
            \begin{table}[ht]
                \begin{center}
                    \caption[{\market}, time series data, empirical
                        and simulated]{{\market}, time series data,
                        empirical and simulated, analysis of the
                        normalized increments.}
                    \begin{tabular}{|c|c|c|c|} \hline
                        \multicolumn{2}{|c|}{Empirical} & \multicolumn{2}{|c|}{Simulated}\\ \hline
                        Mean                & Standard              & Mean               & Standard \\
                        \hspace{0.01in}     & deviation             & \hspace{0.01in}    & deviation \\ \hline\hline
                        {\datafractionmean} & {\datafractionstddev} & {\tsunfairbrownianfractionmean} & {\tsunfairbrownianfractionstddev} \\ \hline
                    \end{tabular}
                    \label{\SETLABEL:ES}
                \end{center}
            \end{table}
        \end{small}

    \subsection{{\market}, number of participating companies}
        \label{\SETLABEL:QNC}

        The data in table~\ref{\SETLABEL:NC} is condensed from
        Section~\ref{\SETLABELREF:QNC}.

        \begin{small}
            \begin{table}[ht]
                \begin{center}
                    \caption[{\market}, number of participating
                        companies] {{\market}, number of participating
                        companies.}
                    \begin{tabular}{|c|c|} \hline
                        Number & Shannon probability\\ \hline
                        {\ncompanies} & {\pncompanies}\\ \hline
                    \end{tabular}
                    \label{\SETLABEL:NC}
                \end{center}
            \end{table}
        \end{small}

    \subsection{{\market}, Shannon probability optimizations}
        \label{\SETLABEL:SPO}

        The data in table~\ref{\SETLABEL:SP} is condensed from
        Section~\ref{\SETLABELREF:QNC}.

        \begin{small}
            \begin{table}[ht]
                \begin{center}
                    \caption[{\market}, Shannon probability
                         optimizations] {{\market}, Shannon
                         probability optimization.}
                    \begin{tabular}{|c|c|} \hline
                        optimize capital growth & optimize market growth\\ \hline
                        {\avgrms} & {\pncompanies}\\ \hline
                    \end{tabular}
                    \label{\SETLABEL:SP}
                \end{center}
            \end{table}
        \end{small}

% Local Variables:
% TeX-parse-self: t
% TeX-auto-save: t
% TeX-master: "fractal.tex"
% End:


    \renewcommand{\market}{Non Optimal Coin Tossing Game}
    \renewcommand{\directory}{../markets/tscoin.tsunfairbrownian}
    \renewcommand{\datafractionmean}{0.008052}
\renewcommand{\datafractionmeanbits}{0.011570}
\renewcommand{\datafractionmeanq}{0.002684}
\renewcommand{\datafractionmeanbitsq}{0.003867}
\renewcommand{\datafractionstddev}{0.038579}
\renewcommand{\datafractionrms}{0.039311}
\renewcommand{\avgrms}{0.602414}
\renewcommand{\ncompanies}{5.210454}
\renewcommand{\pncompanies}{0.544866}
\renewcommand{\datafractionabsmean}{0.029745}
\renewcommand{\datafractionabsstddev}{0.025769}
\renewcommand{\datafractionconstant}{0.010041}
\renewcommand{\datafractionconstantbits}{0.014414}
\renewcommand{\datafractionconstantq}{0.003347}
\renewcommand{\datafractionconstantbitsq}{0.004821}
\renewcommand{\datafractionslope}{-0.000021}
\renewcommand{\datafractionabsconstant}{0.035145}
\renewcommand{\datafractionabsslope}{-0.000057}
\renewcommand{\hurstall}{0.659558}
\renewcommand{\hurstlow}{0.707509}
\renewcommand{\hurstlowtwo}{1.415018}
\renewcommand{\hurstlowhundred}{70.750900}
\renewcommand{\hcalcall}{0.184942}
\renewcommand{\hcalclow}{0.102042}
\renewcommand{\shannonmax}{0.604167}
\renewcommand{\twoponemax}{0.208334}
\renewcommand{\logreturns}{0.010456}
\renewcommand{\twologreturns}{1.007274}
\renewcommand{\twologreturnshundred}{0.727387}
\renewcommand{\oneoverlogreturns}{95.638868}
\renewcommand{\pmax}{0.602094}
\renewcommand{\twopminusone}{0.204188}
\renewcommand{\rmsp}{0.008027}
\renewcommand{\twopx}{0.208583}
\renewcommand{\sigmap}{0.008047}
\renewcommand{\tsunfairbrownianfractionmean}{0.007862}
\renewcommand{\tsunfairbrownianfractionstddev}{0.038619}
\renewcommand{\shannonlogreturns}{0.560125}
\renewcommand{\shannonlogreturnshundred}{56.012500}
\renewcommand{\twopone}{0.120250}
\renewcommand{\twoponehundred}{12.025000}
\renewcommand{\hundredtwoponehundred}{87.975000}
\renewcommand{\hundredshannonlogreturnshundred}{43.987500}
\renewcommand{\datatslsqepbits}{0.007623}
\renewcommand{\thurstall}{0.633980}
\renewcommand{\thurstlow}{0.710108}
\renewcommand{\thurstlowtwo}{1.420216}
\renewcommand{\thurstlowhundred}{71.010800}
\renewcommand{\thcalcall}{0.247886}
\renewcommand{\thcalclow}{0.171737}
\renewcommand{\chisquared}{2.862000}
\renewcommand{\critical}{42.557000}

    \renewcommand{\timescale}{tosses}
    \subidx{market}{\market}
    \idx{\market}

    \section{\market}

        \renewcommand{\SETLABEL}{\LABPRE:NOCT}
        \renewcommand{\SETLABELQ}{\LABPRE:NOCTQ}
        \label{\SETLABEL}

        \idx{tscoin}
        \idx{tsunfairbrownian}
        \subidx{programs}{tscoin}
        \subidx{tscoin}{program}
        \subidx{programs}{tsunfairbrownian}
        \subidx{tsunfairbrownian}{program}
        For the analysis, the data was in the directory
        {\directory}\footnote{As a simulation model, the program
        {\it tscoin}\/ was run to make a time series data file.  The data is
        by {\timescale}s.}.

        The data in this section is presented in
        Section~\ref{\SETLABELREF}.

        %
% -----------------------------------------------------------------------------
%
% A license is hereby granted to reproduce this software source code and
% to create executable versions from this source code for personal,
% non-commercial use.  The copyright notice included with the software
% must be maintained in all copies produced.
%
% THIS PROGRAM IS PROVIDED "AS IS". THE AUTHOR PROVIDES NO WARRANTIES
% WHATSOEVER, EXPRESSED OR IMPLIED, INCLUDING WARRANTIES OF
% MERCHANTABILITY, TITLE, OR FITNESS FOR ANY PARTICULAR PURPOSE.  THE
% AUTHOR DOES NOT WARRANT THAT USE OF THIS PROGRAM DOES NOT INFRINGE THE
% INTELLECTUAL PROPERTY RIGHTS OF ANY THIRD PARTY IN ANY COUNTRY.
%
% Copyright (c) 1994-2006, John Conover, All Rights Reserved.
%
% Comments and/or bug reports should be addressed to:
%
%     john@email.johncon.com (John Conover)
%
% -----------------------------------------------------------------------------
%
% Revision: \RCSRevision \\
% Revision Time: \RCSTime UMT \\
% Revision Date: \RCSDate \\
% Revision Id: \RCSId \\
% Revision File: \RCSLog \\
\RCS $Revision: 0.0 $
\RCS $Date: 2006/01/20 04:38:13 $
\RCS $Id: tables.tex,v 0.0 2006/01/20 04:38:13 john Exp $
% $Log: tables.tex,v $
% Revision 0.0  2006/01/20 04:38:13  john
% Initial version
%
%
    \subsection{{\market}, normalized increments}
        \label{\SETLABEL:TSA}

        The data in table~\ref{\SETLABEL:INC} is condensed from
        Section~\ref{\SETLABELREF:TSA}.

        \begin{small}
            \begin{table}[ht]
                \begin{center}
                    \caption[{\market}, normalized increments]
                        {{\market}, normalized increments.}
                    \begin{tabular}{|c|c|c|c|c|c|c|c|c|c|} \hline
                        \multicolumn{5}{|c|}{Normalized}                                                                                  & \multicolumn{5}{|c|}{Normalized Absolute Value}\\ \hline
                        Mean                & Standard              & rms                & \multicolumn{2}{|c|}{Least Squares}            & Mean                   & Standard                 & rms                & \multicolumn{2}{|c|}{Least Squares} \\ \cline{4-5}\cline{9-10}
                        \hspace{0.01in}     & deviation             & \hspace{0.01in}    & Constant                & Slope                & \hspace{0.01in}        & deviation                & \hspace{0.01in}    & Constant                   & Slope \\ \hline\hline
                        {\datafractionmean} & {\datafractionstddev} & {\datafractionrms} & {\datafractionconstant} & {\datafractionslope} & {\datafractionabsmean} & {\datafractionabsstddev} & {\datafractionrms} & {\datafractionabsconstant} & {\datafractionabsslope} \\ \hline
                    \end{tabular}
                    \label{\SETLABEL:INC}
                \end{center}
            \end{table}
        \end{small}

    \subsection{{\market}, Logarithmic Returns, in Bits}
        \label{\SETLABEL:LR}

        The data in table~\ref{\SETLABEL:RET} is condensed from
        Section~\ref{\SETLABELREF:FS}.

        \begin{small}
            \begin{table}[ht]
                \begin{center}
                    \caption[{\market}, Logarithmic Returns, in
                        Bits]{{\market}, Logarithmic Returns, in Bits.}
                    \begin{tabular}{|c|c|c|c|} \hline
                        \multicolumn{2}{|c|}{Calculated from Table~\ref{\SETLABEL:INC}} & \multicolumn{2}{|c|}{From program:}\\ \hline
                        Mean                    & Least squares                       & {\it tslsq}\/              & {\it tslogreturns}\/ \\ \hline\hline
                        {\datafractionmeanbits} & {\datafractionconstantbits} & {\datatslsqepbits} & {\logreturns} \\ \hline
                    \end{tabular}
                    \label{\SETLABEL:RET}
                \end{center}
            \end{table}
        \end{small}

    \subsection{{\market}, Shannon probabilities}
        \label{\SETLABEL:MAXSHANNON}

        The data in table~\ref{\SETLABEL:SHANNON} is condensed from
        sections~\ref{\SETLABELREF:FS}
        and~\ref{\SETLABELREF:MAXSHANNON}.

        \begin{small}
            \begin{table}[ht]
                \begin{center}
                    \caption[{\market}, Shannon
                        probabilities]{{\market}, Shannon
                        probabilities.}
                    \begin{tabular}{|c|c|c|c|} \hline
                        \multicolumn{3}{|c|}{Maximum} & \multicolumn{1}{|c|}{Operational}\\ \hline
                        Fraction of         & $\frac{\frac{\mbox{\scriptsize{mean}}}{\mbox{\scriptsize{rms}}} + 1}{2}$ & \multicolumn{2}{|c|}{From program:}\\ \cline{3-4}
                        positive increments & \hspace{0.01in}                                                          & {\it tsshannonmax}\/    & {\it tsshannon}\/ \\ \hline\hline
                        {\pmax}             & {\avgrms}                                                                & {\shannonmax}   & {\shannonlogreturns} \\ \hline
                    \end{tabular}
                    \label{\SETLABEL:SHANNON}
                \end{center}
            \end{table}
        \end{small}

    \subsection{{\market}, Logistic Analysis}
        \label{\SETLABEL:LAA}

        The data in table~\ref{\SETLABEL:LA} is condensed from
        Section~\ref{\SETLABELREF:LA}\footnote{Note that there are
        numerical stability issues with the methodology used to derive
        the constants---if the non-linear term, $b$, was greater than
        zero, it was set to zero to produce the graphs in
        Section~\ref{\SETLABELREF:LA}.}.

        \begin{small}
            \begin{table}[ht]
                \begin{center}
                    \caption[{\market}, Logistic Analysis.]
                        {{\market}, Logistic Analysis, $x_t = x_{t - 1}\left(a + b \cdot x_{t - 1}\right)$.}
                    \begin{tabular}{|c|c|} \hline
                        $a$ & $b$\\ \hline\hline
                        {\datafractionconstant} & {\datafractionslope}\\ \hline
                    \end{tabular}
                    \label{\SETLABEL:LA}
                \end{center}
            \end{table}
        \end{small}

    \subsection{{\market}, Hurst Coefficients and H  Parameters}
        \label{\SETLABEL:HCHP}

        The data in table~\ref{\SETLABEL:H} is condensed from
        Section~\ref{\SETLABELREF:H}.

        \begin{small}
            \begin{table}[ht]
                \begin{center}
                    \caption[{\market}, Hurst Coefficients and H
                        Parameters]{{\market}, Hurst Coefficients and
                        H Parameters.}
                    \begin{tabular}{|c|c|c|c|} \hline
                        \multicolumn{2}{|c|}{Hurst Coefficients} & \multicolumn{2}{|c|}{H Parameters}\\ \hline
                        Near term   & Far term    & Near term   & Far term \\ \hline\hline
                        {\thurstlow} & {\thurstall} & {\thcalclow} & {\thcalcall} \\ \hline
                    \end{tabular}
                    \label{\SETLABEL:H}
                \end{center}
            \end{table}
        \end{small}

        \begin{small}
            \begin{table}[ht]
                \begin{center}
                    \caption[{\market}, Hurst Coefficients and H
                        Parameters]{{\market}, Hurst Coefficients and
                        H Parameters, as a Derivative.}
                    \begin{tabular}{|c|c|c|c|} \hline
                        \multicolumn{2}{|c|}{Hurst Coefficients} & \multicolumn{2}{|c|}{H Parameters}\\ \hline
                        Near term    & Far term     & Near term    & Far term \\ \hline\hline
                        {\hurstlow} & {\hurstall} & {\hcalclow} & {\hcalcall} \\ \hline
                    \end{tabular}
                    \label{\SETLABEL:TH}
                \end{center}
            \end{table}
        \end{small}

    \subsection{{\market}, verification of the increments}
        \label{\SETLABEL:VI1}

        The data in table~\ref{\SETLABEL:COMP} is condensed from
        Section~\ref{\SETLABELREF:QVA}.

        \begin{small}
            \begin{table}[ht]
                \begin{center}
                    \caption[{\market}, verification of
                        the increments]{{\market}, verification the of
                        the increments, the mean, $\sigma$ is the
                        standard deviation from
                        table~\ref{\SETLABEL:INC},
                        {\datafractionstddev}, and $P$ is the maximum
                        Shannon probability from
                        table~\ref{\SETLABEL:SHANNON}, {\pmax}. In
                        principle, the values should equate.}
                    \begin{tabular}{|c|c|c|} \hline
                        Mean                & $\mbox{rms} (2P - 1)$ & $\frac{{\sigma}(2P - 1)}{2\sqrt{P(P - 1)}} $ \\ \hline\hline
                        {\datafractionmean} & {\rmsp}               & {\sigmap} \\ \hline
                    \end{tabular}
                    \label{\SETLABEL:COMP}
                \end{center}
            \end{table}
        \end{small}

    \subsection{{\market}, verification of the increments}
        \label{\SETLABEL:VI2}

        The data in table~\ref{\SETLABEL:ABS} is condensed from
        Section~\ref{\SETLABELREF:QVA}.

        \begin{small}
            \begin{table}[ht]
                \begin{center}
                    \caption[{\market}, verification of
                        the increments]{{\market}, verification the of
                        increments. In principle, the mean of the
                        absolute value of the increments and the root
                        mean square of the increments should
                        equate\footnote{The absolute value of the
                        normalized increments, when averaged, is
                        related to the root mean square of the
                        increments by a constant. If the normalized
                        increments are a fixed increment, the constant
                        is unity. If the normalized increments have a
                        Gaussian distribution, the constant is
                        $\approx 0.8$ depending on the accuracy of of
                        ``fit'' to a Gaussian distribution.}.}
                    \begin{tabular}{|c|c|} \hline
                        Mean of the               & rms \\
                        absolute value            & \hspace{0.01in} \\ \hline\hline
                        {\datafractionabsmean}    & {\datafractionrms} \\ \hline
                    \end{tabular}
                    \label{\SETLABEL:ABS}
                \end{center}
            \end{table}
        \end{small}

    \subsection{{\market}, $\chi^2$ values of the increments}
        \label{\SETLABEL:XSQ}

        The data in table~\ref{\SETLABEL:XSQT} is condensed from
        Section~\ref{\SETLABELREF:NH}.

        \begin{small}
            \begin{table}[ht]
                \begin{center}
                    \caption[{\market}, $\chi^2$ values of
                        the increments]{{\market}, $\chi^2$ values of
                        the increments. In principle, if the
                        distribution of the normalized increments is a
                        Gaussian distribution, the $\chi^2$ value will
                        be significantly less than the critical
                        value.}
                    \begin{tabular}{|c|c|} \hline
                        $\chi^2$      & Critical Value \\ \hline\hline
                        {\chisquared} & {\critical} \\ \hline
                    \end{tabular}
                    \label{\SETLABEL:XSQT}
                \end{center}
            \end{table}
        \end{small}

    \subsection{{\market}, time series data, empirical and simulated}
        \label{\SETLABEL:SIM}

        The data in table~\ref{\SETLABEL:ES} is condensed from
        Section~\ref{\SETLABELREF:TSUNFAIRBROWNIAN}.

        \begin{small}
            \begin{table}[ht]
                \begin{center}
                    \caption[{\market}, time series data, empirical
                        and simulated]{{\market}, time series data,
                        empirical and simulated, analysis of the
                        normalized increments.}
                    \begin{tabular}{|c|c|c|c|} \hline
                        \multicolumn{2}{|c|}{Empirical} & \multicolumn{2}{|c|}{Simulated}\\ \hline
                        Mean                & Standard              & Mean               & Standard \\
                        \hspace{0.01in}     & deviation             & \hspace{0.01in}    & deviation \\ \hline\hline
                        {\datafractionmean} & {\datafractionstddev} & {\tsunfairbrownianfractionmean} & {\tsunfairbrownianfractionstddev} \\ \hline
                    \end{tabular}
                    \label{\SETLABEL:ES}
                \end{center}
            \end{table}
        \end{small}

    \subsection{{\market}, number of participating companies}
        \label{\SETLABEL:QNC}

        The data in table~\ref{\SETLABEL:NC} is condensed from
        Section~\ref{\SETLABELREF:QNC}.

        \begin{small}
            \begin{table}[ht]
                \begin{center}
                    \caption[{\market}, number of participating
                        companies] {{\market}, number of participating
                        companies.}
                    \begin{tabular}{|c|c|} \hline
                        Number & Shannon probability\\ \hline
                        {\ncompanies} & {\pncompanies}\\ \hline
                    \end{tabular}
                    \label{\SETLABEL:NC}
                \end{center}
            \end{table}
        \end{small}

    \subsection{{\market}, Shannon probability optimizations}
        \label{\SETLABEL:SPO}

        The data in table~\ref{\SETLABEL:SP} is condensed from
        Section~\ref{\SETLABELREF:QNC}.

        \begin{small}
            \begin{table}[ht]
                \begin{center}
                    \caption[{\market}, Shannon probability
                         optimizations] {{\market}, Shannon
                         probability optimization.}
                    \begin{tabular}{|c|c|} \hline
                        optimize capital growth & optimize market growth\\ \hline
                        {\avgrms} & {\pncompanies}\\ \hline
                    \end{tabular}
                    \label{\SETLABEL:SP}
                \end{center}
            \end{table}
        \end{small}

% Local Variables:
% TeX-parse-self: t
% TeX-auto-save: t
% TeX-master: "fractal.tex"
% End:


    \renewcommand{\market}{Time Sampled Non Optimal Coin Tossing Game}
    \renewcommand{\directory}{../markets/tscoin.tsunfairbrownian.tssample}
    \renewcommand{\datafractionmean}{0.008052}
\renewcommand{\datafractionmeanbits}{0.011570}
\renewcommand{\datafractionmeanq}{0.002684}
\renewcommand{\datafractionmeanbitsq}{0.003867}
\renewcommand{\datafractionstddev}{0.038579}
\renewcommand{\datafractionrms}{0.039311}
\renewcommand{\avgrms}{0.602414}
\renewcommand{\ncompanies}{5.210454}
\renewcommand{\pncompanies}{0.544866}
\renewcommand{\datafractionabsmean}{0.029745}
\renewcommand{\datafractionabsstddev}{0.025769}
\renewcommand{\datafractionconstant}{0.010041}
\renewcommand{\datafractionconstantbits}{0.014414}
\renewcommand{\datafractionconstantq}{0.003347}
\renewcommand{\datafractionconstantbitsq}{0.004821}
\renewcommand{\datafractionslope}{-0.000021}
\renewcommand{\datafractionabsconstant}{0.035145}
\renewcommand{\datafractionabsslope}{-0.000057}
\renewcommand{\hurstall}{0.659558}
\renewcommand{\hurstlow}{0.707509}
\renewcommand{\hurstlowtwo}{1.415018}
\renewcommand{\hurstlowhundred}{70.750900}
\renewcommand{\hcalcall}{0.184942}
\renewcommand{\hcalclow}{0.102042}
\renewcommand{\shannonmax}{0.604167}
\renewcommand{\twoponemax}{0.208334}
\renewcommand{\logreturns}{0.010456}
\renewcommand{\twologreturns}{1.007274}
\renewcommand{\twologreturnshundred}{0.727387}
\renewcommand{\oneoverlogreturns}{95.638868}
\renewcommand{\pmax}{0.602094}
\renewcommand{\twopminusone}{0.204188}
\renewcommand{\rmsp}{0.008027}
\renewcommand{\twopx}{0.208583}
\renewcommand{\sigmap}{0.008047}
\renewcommand{\tsunfairbrownianfractionmean}{0.007862}
\renewcommand{\tsunfairbrownianfractionstddev}{0.038619}
\renewcommand{\shannonlogreturns}{0.560125}
\renewcommand{\shannonlogreturnshundred}{56.012500}
\renewcommand{\twopone}{0.120250}
\renewcommand{\twoponehundred}{12.025000}
\renewcommand{\hundredtwoponehundred}{87.975000}
\renewcommand{\hundredshannonlogreturnshundred}{43.987500}
\renewcommand{\datatslsqepbits}{0.007623}
\renewcommand{\thurstall}{0.633980}
\renewcommand{\thurstlow}{0.710108}
\renewcommand{\thurstlowtwo}{1.420216}
\renewcommand{\thurstlowhundred}{71.010800}
\renewcommand{\thcalcall}{0.247886}
\renewcommand{\thcalclow}{0.171737}
\renewcommand{\chisquared}{2.862000}
\renewcommand{\critical}{42.557000}

    \renewcommand{\timescale}{tosses}
    \subidx{market}{\market}
    \idx{\market}

    \section{\market}

        \renewcommand{\SETLABEL}{\LABPRE:TSNOCT}
        \renewcommand{\SETLABELQ}{\LABPRE:TSNOCTQ}
        \label{\SETLABEL}

        \idx{tscoin}
        \idx{tsunfairbrownian}
        \idx{tssample}
        \subidx{programs}{tscoin}
        \subidx{tscoin}{program}
        \subidx{programs}{tsunfairbrownian}
        \subidx{tsunfairbrownian}{program}
        \subidx{programs}{tssample}
        \subidx{tssample}{program}
        For the analysis, the data was in the directory
        {\directory}\footnote{As a simulation model, the program
        {\it tscoin}\/ was run to make a time series data file.  The data is
        by {\timescale}.}.

        The data in this section is presented in
        Section~\ref{\SETLABELREF}.

        %
% -----------------------------------------------------------------------------
%
% A license is hereby granted to reproduce this software source code and
% to create executable versions from this source code for personal,
% non-commercial use.  The copyright notice included with the software
% must be maintained in all copies produced.
%
% THIS PROGRAM IS PROVIDED "AS IS". THE AUTHOR PROVIDES NO WARRANTIES
% WHATSOEVER, EXPRESSED OR IMPLIED, INCLUDING WARRANTIES OF
% MERCHANTABILITY, TITLE, OR FITNESS FOR ANY PARTICULAR PURPOSE.  THE
% AUTHOR DOES NOT WARRANT THAT USE OF THIS PROGRAM DOES NOT INFRINGE THE
% INTELLECTUAL PROPERTY RIGHTS OF ANY THIRD PARTY IN ANY COUNTRY.
%
% Copyright (c) 1994-2006, John Conover, All Rights Reserved.
%
% Comments and/or bug reports should be addressed to:
%
%     john@email.johncon.com (John Conover)
%
% -----------------------------------------------------------------------------
%
% Revision: \RCSRevision \\
% Revision Time: \RCSTime UMT \\
% Revision Date: \RCSDate \\
% Revision Id: \RCSId \\
% Revision File: \RCSLog \\
\RCS $Revision: 0.0 $
\RCS $Date: 2006/01/20 04:38:13 $
\RCS $Id: tables.tex,v 0.0 2006/01/20 04:38:13 john Exp $
% $Log: tables.tex,v $
% Revision 0.0  2006/01/20 04:38:13  john
% Initial version
%
%
    \subsection{{\market}, normalized increments}
        \label{\SETLABEL:TSA}

        The data in table~\ref{\SETLABEL:INC} is condensed from
        Section~\ref{\SETLABELREF:TSA}.

        \begin{small}
            \begin{table}[ht]
                \begin{center}
                    \caption[{\market}, normalized increments]
                        {{\market}, normalized increments.}
                    \begin{tabular}{|c|c|c|c|c|c|c|c|c|c|} \hline
                        \multicolumn{5}{|c|}{Normalized}                                                                                  & \multicolumn{5}{|c|}{Normalized Absolute Value}\\ \hline
                        Mean                & Standard              & rms                & \multicolumn{2}{|c|}{Least Squares}            & Mean                   & Standard                 & rms                & \multicolumn{2}{|c|}{Least Squares} \\ \cline{4-5}\cline{9-10}
                        \hspace{0.01in}     & deviation             & \hspace{0.01in}    & Constant                & Slope                & \hspace{0.01in}        & deviation                & \hspace{0.01in}    & Constant                   & Slope \\ \hline\hline
                        {\datafractionmean} & {\datafractionstddev} & {\datafractionrms} & {\datafractionconstant} & {\datafractionslope} & {\datafractionabsmean} & {\datafractionabsstddev} & {\datafractionrms} & {\datafractionabsconstant} & {\datafractionabsslope} \\ \hline
                    \end{tabular}
                    \label{\SETLABEL:INC}
                \end{center}
            \end{table}
        \end{small}

    \subsection{{\market}, Logarithmic Returns, in Bits}
        \label{\SETLABEL:LR}

        The data in table~\ref{\SETLABEL:RET} is condensed from
        Section~\ref{\SETLABELREF:FS}.

        \begin{small}
            \begin{table}[ht]
                \begin{center}
                    \caption[{\market}, Logarithmic Returns, in
                        Bits]{{\market}, Logarithmic Returns, in Bits.}
                    \begin{tabular}{|c|c|c|c|} \hline
                        \multicolumn{2}{|c|}{Calculated from Table~\ref{\SETLABEL:INC}} & \multicolumn{2}{|c|}{From program:}\\ \hline
                        Mean                    & Least squares                       & {\it tslsq}\/              & {\it tslogreturns}\/ \\ \hline\hline
                        {\datafractionmeanbits} & {\datafractionconstantbits} & {\datatslsqepbits} & {\logreturns} \\ \hline
                    \end{tabular}
                    \label{\SETLABEL:RET}
                \end{center}
            \end{table}
        \end{small}

    \subsection{{\market}, Shannon probabilities}
        \label{\SETLABEL:MAXSHANNON}

        The data in table~\ref{\SETLABEL:SHANNON} is condensed from
        sections~\ref{\SETLABELREF:FS}
        and~\ref{\SETLABELREF:MAXSHANNON}.

        \begin{small}
            \begin{table}[ht]
                \begin{center}
                    \caption[{\market}, Shannon
                        probabilities]{{\market}, Shannon
                        probabilities.}
                    \begin{tabular}{|c|c|c|c|} \hline
                        \multicolumn{3}{|c|}{Maximum} & \multicolumn{1}{|c|}{Operational}\\ \hline
                        Fraction of         & $\frac{\frac{\mbox{\scriptsize{mean}}}{\mbox{\scriptsize{rms}}} + 1}{2}$ & \multicolumn{2}{|c|}{From program:}\\ \cline{3-4}
                        positive increments & \hspace{0.01in}                                                          & {\it tsshannonmax}\/    & {\it tsshannon}\/ \\ \hline\hline
                        {\pmax}             & {\avgrms}                                                                & {\shannonmax}   & {\shannonlogreturns} \\ \hline
                    \end{tabular}
                    \label{\SETLABEL:SHANNON}
                \end{center}
            \end{table}
        \end{small}

    \subsection{{\market}, Logistic Analysis}
        \label{\SETLABEL:LAA}

        The data in table~\ref{\SETLABEL:LA} is condensed from
        Section~\ref{\SETLABELREF:LA}\footnote{Note that there are
        numerical stability issues with the methodology used to derive
        the constants---if the non-linear term, $b$, was greater than
        zero, it was set to zero to produce the graphs in
        Section~\ref{\SETLABELREF:LA}.}.

        \begin{small}
            \begin{table}[ht]
                \begin{center}
                    \caption[{\market}, Logistic Analysis.]
                        {{\market}, Logistic Analysis, $x_t = x_{t - 1}\left(a + b \cdot x_{t - 1}\right)$.}
                    \begin{tabular}{|c|c|} \hline
                        $a$ & $b$\\ \hline\hline
                        {\datafractionconstant} & {\datafractionslope}\\ \hline
                    \end{tabular}
                    \label{\SETLABEL:LA}
                \end{center}
            \end{table}
        \end{small}

    \subsection{{\market}, Hurst Coefficients and H  Parameters}
        \label{\SETLABEL:HCHP}

        The data in table~\ref{\SETLABEL:H} is condensed from
        Section~\ref{\SETLABELREF:H}.

        \begin{small}
            \begin{table}[ht]
                \begin{center}
                    \caption[{\market}, Hurst Coefficients and H
                        Parameters]{{\market}, Hurst Coefficients and
                        H Parameters.}
                    \begin{tabular}{|c|c|c|c|} \hline
                        \multicolumn{2}{|c|}{Hurst Coefficients} & \multicolumn{2}{|c|}{H Parameters}\\ \hline
                        Near term   & Far term    & Near term   & Far term \\ \hline\hline
                        {\thurstlow} & {\thurstall} & {\thcalclow} & {\thcalcall} \\ \hline
                    \end{tabular}
                    \label{\SETLABEL:H}
                \end{center}
            \end{table}
        \end{small}

        \begin{small}
            \begin{table}[ht]
                \begin{center}
                    \caption[{\market}, Hurst Coefficients and H
                        Parameters]{{\market}, Hurst Coefficients and
                        H Parameters, as a Derivative.}
                    \begin{tabular}{|c|c|c|c|} \hline
                        \multicolumn{2}{|c|}{Hurst Coefficients} & \multicolumn{2}{|c|}{H Parameters}\\ \hline
                        Near term    & Far term     & Near term    & Far term \\ \hline\hline
                        {\hurstlow} & {\hurstall} & {\hcalclow} & {\hcalcall} \\ \hline
                    \end{tabular}
                    \label{\SETLABEL:TH}
                \end{center}
            \end{table}
        \end{small}

    \subsection{{\market}, verification of the increments}
        \label{\SETLABEL:VI1}

        The data in table~\ref{\SETLABEL:COMP} is condensed from
        Section~\ref{\SETLABELREF:QVA}.

        \begin{small}
            \begin{table}[ht]
                \begin{center}
                    \caption[{\market}, verification of
                        the increments]{{\market}, verification the of
                        the increments, the mean, $\sigma$ is the
                        standard deviation from
                        table~\ref{\SETLABEL:INC},
                        {\datafractionstddev}, and $P$ is the maximum
                        Shannon probability from
                        table~\ref{\SETLABEL:SHANNON}, {\pmax}. In
                        principle, the values should equate.}
                    \begin{tabular}{|c|c|c|} \hline
                        Mean                & $\mbox{rms} (2P - 1)$ & $\frac{{\sigma}(2P - 1)}{2\sqrt{P(P - 1)}} $ \\ \hline\hline
                        {\datafractionmean} & {\rmsp}               & {\sigmap} \\ \hline
                    \end{tabular}
                    \label{\SETLABEL:COMP}
                \end{center}
            \end{table}
        \end{small}

    \subsection{{\market}, verification of the increments}
        \label{\SETLABEL:VI2}

        The data in table~\ref{\SETLABEL:ABS} is condensed from
        Section~\ref{\SETLABELREF:QVA}.

        \begin{small}
            \begin{table}[ht]
                \begin{center}
                    \caption[{\market}, verification of
                        the increments]{{\market}, verification the of
                        increments. In principle, the mean of the
                        absolute value of the increments and the root
                        mean square of the increments should
                        equate\footnote{The absolute value of the
                        normalized increments, when averaged, is
                        related to the root mean square of the
                        increments by a constant. If the normalized
                        increments are a fixed increment, the constant
                        is unity. If the normalized increments have a
                        Gaussian distribution, the constant is
                        $\approx 0.8$ depending on the accuracy of of
                        ``fit'' to a Gaussian distribution.}.}
                    \begin{tabular}{|c|c|} \hline
                        Mean of the               & rms \\
                        absolute value            & \hspace{0.01in} \\ \hline\hline
                        {\datafractionabsmean}    & {\datafractionrms} \\ \hline
                    \end{tabular}
                    \label{\SETLABEL:ABS}
                \end{center}
            \end{table}
        \end{small}

    \subsection{{\market}, $\chi^2$ values of the increments}
        \label{\SETLABEL:XSQ}

        The data in table~\ref{\SETLABEL:XSQT} is condensed from
        Section~\ref{\SETLABELREF:NH}.

        \begin{small}
            \begin{table}[ht]
                \begin{center}
                    \caption[{\market}, $\chi^2$ values of
                        the increments]{{\market}, $\chi^2$ values of
                        the increments. In principle, if the
                        distribution of the normalized increments is a
                        Gaussian distribution, the $\chi^2$ value will
                        be significantly less than the critical
                        value.}
                    \begin{tabular}{|c|c|} \hline
                        $\chi^2$      & Critical Value \\ \hline\hline
                        {\chisquared} & {\critical} \\ \hline
                    \end{tabular}
                    \label{\SETLABEL:XSQT}
                \end{center}
            \end{table}
        \end{small}

    \subsection{{\market}, time series data, empirical and simulated}
        \label{\SETLABEL:SIM}

        The data in table~\ref{\SETLABEL:ES} is condensed from
        Section~\ref{\SETLABELREF:TSUNFAIRBROWNIAN}.

        \begin{small}
            \begin{table}[ht]
                \begin{center}
                    \caption[{\market}, time series data, empirical
                        and simulated]{{\market}, time series data,
                        empirical and simulated, analysis of the
                        normalized increments.}
                    \begin{tabular}{|c|c|c|c|} \hline
                        \multicolumn{2}{|c|}{Empirical} & \multicolumn{2}{|c|}{Simulated}\\ \hline
                        Mean                & Standard              & Mean               & Standard \\
                        \hspace{0.01in}     & deviation             & \hspace{0.01in}    & deviation \\ \hline\hline
                        {\datafractionmean} & {\datafractionstddev} & {\tsunfairbrownianfractionmean} & {\tsunfairbrownianfractionstddev} \\ \hline
                    \end{tabular}
                    \label{\SETLABEL:ES}
                \end{center}
            \end{table}
        \end{small}

    \subsection{{\market}, number of participating companies}
        \label{\SETLABEL:QNC}

        The data in table~\ref{\SETLABEL:NC} is condensed from
        Section~\ref{\SETLABELREF:QNC}.

        \begin{small}
            \begin{table}[ht]
                \begin{center}
                    \caption[{\market}, number of participating
                        companies] {{\market}, number of participating
                        companies.}
                    \begin{tabular}{|c|c|} \hline
                        Number & Shannon probability\\ \hline
                        {\ncompanies} & {\pncompanies}\\ \hline
                    \end{tabular}
                    \label{\SETLABEL:NC}
                \end{center}
            \end{table}
        \end{small}

    \subsection{{\market}, Shannon probability optimizations}
        \label{\SETLABEL:SPO}

        The data in table~\ref{\SETLABEL:SP} is condensed from
        Section~\ref{\SETLABELREF:QNC}.

        \begin{small}
            \begin{table}[ht]
                \begin{center}
                    \caption[{\market}, Shannon probability
                         optimizations] {{\market}, Shannon
                         probability optimization.}
                    \begin{tabular}{|c|c|} \hline
                        optimize capital growth & optimize market growth\\ \hline
                        {\avgrms} & {\pncompanies}\\ \hline
                    \end{tabular}
                    \label{\SETLABEL:SP}
                \end{center}
            \end{table}
        \end{small}

% Local Variables:
% TeX-parse-self: t
% TeX-auto-save: t
% TeX-master: "fractal.tex"
% End:


    \renewcommand{\market}{Time Sampled Coin Tossing Game}
    \renewcommand{\directory}{../markets/tscoin.tssample}
    \renewcommand{\datafractionmean}{0.008052}
\renewcommand{\datafractionmeanbits}{0.011570}
\renewcommand{\datafractionmeanq}{0.002684}
\renewcommand{\datafractionmeanbitsq}{0.003867}
\renewcommand{\datafractionstddev}{0.038579}
\renewcommand{\datafractionrms}{0.039311}
\renewcommand{\avgrms}{0.602414}
\renewcommand{\ncompanies}{5.210454}
\renewcommand{\pncompanies}{0.544866}
\renewcommand{\datafractionabsmean}{0.029745}
\renewcommand{\datafractionabsstddev}{0.025769}
\renewcommand{\datafractionconstant}{0.010041}
\renewcommand{\datafractionconstantbits}{0.014414}
\renewcommand{\datafractionconstantq}{0.003347}
\renewcommand{\datafractionconstantbitsq}{0.004821}
\renewcommand{\datafractionslope}{-0.000021}
\renewcommand{\datafractionabsconstant}{0.035145}
\renewcommand{\datafractionabsslope}{-0.000057}
\renewcommand{\hurstall}{0.659558}
\renewcommand{\hurstlow}{0.707509}
\renewcommand{\hurstlowtwo}{1.415018}
\renewcommand{\hurstlowhundred}{70.750900}
\renewcommand{\hcalcall}{0.184942}
\renewcommand{\hcalclow}{0.102042}
\renewcommand{\shannonmax}{0.604167}
\renewcommand{\twoponemax}{0.208334}
\renewcommand{\logreturns}{0.010456}
\renewcommand{\twologreturns}{1.007274}
\renewcommand{\twologreturnshundred}{0.727387}
\renewcommand{\oneoverlogreturns}{95.638868}
\renewcommand{\pmax}{0.602094}
\renewcommand{\twopminusone}{0.204188}
\renewcommand{\rmsp}{0.008027}
\renewcommand{\twopx}{0.208583}
\renewcommand{\sigmap}{0.008047}
\renewcommand{\tsunfairbrownianfractionmean}{0.007862}
\renewcommand{\tsunfairbrownianfractionstddev}{0.038619}
\renewcommand{\shannonlogreturns}{0.560125}
\renewcommand{\shannonlogreturnshundred}{56.012500}
\renewcommand{\twopone}{0.120250}
\renewcommand{\twoponehundred}{12.025000}
\renewcommand{\hundredtwoponehundred}{87.975000}
\renewcommand{\hundredshannonlogreturnshundred}{43.987500}
\renewcommand{\datatslsqepbits}{0.007623}
\renewcommand{\thurstall}{0.633980}
\renewcommand{\thurstlow}{0.710108}
\renewcommand{\thurstlowtwo}{1.420216}
\renewcommand{\thurstlowhundred}{71.010800}
\renewcommand{\thcalcall}{0.247886}
\renewcommand{\thcalclow}{0.171737}
\renewcommand{\chisquared}{2.862000}
\renewcommand{\critical}{42.557000}

    \renewcommand{\timescale}{tosses}
    \subidx{market}{\market}
    \idx{\market}

    \section{\market}

        \renewcommand{\SETLABEL}{\LABPRE:TSCT}
        \renewcommand{\SETLABELQ}{\LABPRE:TSCTQ}
        \label{\SETLABEL}

        \idx{tscoin}
        \idx{tsunfairbrownian}
        \idx{tssample}
        \subidx{programs}{tscoin}
        \subidx{tscoin}{program}
        \subidx{tsunfairbrownian}{program}
        \subidx{programs}{tsunfairbrownian}
        \subidx{programs}{tssample}
        \subidx{tssample}{program}
        For the analysis, the data was in the directory
        {\directory}\footnote{As a simulation model, the program
        {\it tscoin}\/ was run to make a time series data file.  The data is
        by {\timescale}.}.

        The data in this section is presented in
        Section~\ref{\SETLABELREF}.

        %
% -----------------------------------------------------------------------------
%
% A license is hereby granted to reproduce this software source code and
% to create executable versions from this source code for personal,
% non-commercial use.  The copyright notice included with the software
% must be maintained in all copies produced.
%
% THIS PROGRAM IS PROVIDED "AS IS". THE AUTHOR PROVIDES NO WARRANTIES
% WHATSOEVER, EXPRESSED OR IMPLIED, INCLUDING WARRANTIES OF
% MERCHANTABILITY, TITLE, OR FITNESS FOR ANY PARTICULAR PURPOSE.  THE
% AUTHOR DOES NOT WARRANT THAT USE OF THIS PROGRAM DOES NOT INFRINGE THE
% INTELLECTUAL PROPERTY RIGHTS OF ANY THIRD PARTY IN ANY COUNTRY.
%
% Copyright (c) 1994-2006, John Conover, All Rights Reserved.
%
% Comments and/or bug reports should be addressed to:
%
%     john@email.johncon.com (John Conover)
%
% -----------------------------------------------------------------------------
%
% Revision: \RCSRevision \\
% Revision Time: \RCSTime UMT \\
% Revision Date: \RCSDate \\
% Revision Id: \RCSId \\
% Revision File: \RCSLog \\
\RCS $Revision: 0.0 $
\RCS $Date: 2006/01/20 04:38:13 $
\RCS $Id: tables.tex,v 0.0 2006/01/20 04:38:13 john Exp $
% $Log: tables.tex,v $
% Revision 0.0  2006/01/20 04:38:13  john
% Initial version
%
%
    \subsection{{\market}, normalized increments}
        \label{\SETLABEL:TSA}

        The data in table~\ref{\SETLABEL:INC} is condensed from
        Section~\ref{\SETLABELREF:TSA}.

        \begin{small}
            \begin{table}[ht]
                \begin{center}
                    \caption[{\market}, normalized increments]
                        {{\market}, normalized increments.}
                    \begin{tabular}{|c|c|c|c|c|c|c|c|c|c|} \hline
                        \multicolumn{5}{|c|}{Normalized}                                                                                  & \multicolumn{5}{|c|}{Normalized Absolute Value}\\ \hline
                        Mean                & Standard              & rms                & \multicolumn{2}{|c|}{Least Squares}            & Mean                   & Standard                 & rms                & \multicolumn{2}{|c|}{Least Squares} \\ \cline{4-5}\cline{9-10}
                        \hspace{0.01in}     & deviation             & \hspace{0.01in}    & Constant                & Slope                & \hspace{0.01in}        & deviation                & \hspace{0.01in}    & Constant                   & Slope \\ \hline\hline
                        {\datafractionmean} & {\datafractionstddev} & {\datafractionrms} & {\datafractionconstant} & {\datafractionslope} & {\datafractionabsmean} & {\datafractionabsstddev} & {\datafractionrms} & {\datafractionabsconstant} & {\datafractionabsslope} \\ \hline
                    \end{tabular}
                    \label{\SETLABEL:INC}
                \end{center}
            \end{table}
        \end{small}

    \subsection{{\market}, Logarithmic Returns, in Bits}
        \label{\SETLABEL:LR}

        The data in table~\ref{\SETLABEL:RET} is condensed from
        Section~\ref{\SETLABELREF:FS}.

        \begin{small}
            \begin{table}[ht]
                \begin{center}
                    \caption[{\market}, Logarithmic Returns, in
                        Bits]{{\market}, Logarithmic Returns, in Bits.}
                    \begin{tabular}{|c|c|c|c|} \hline
                        \multicolumn{2}{|c|}{Calculated from Table~\ref{\SETLABEL:INC}} & \multicolumn{2}{|c|}{From program:}\\ \hline
                        Mean                    & Least squares                       & {\it tslsq}\/              & {\it tslogreturns}\/ \\ \hline\hline
                        {\datafractionmeanbits} & {\datafractionconstantbits} & {\datatslsqepbits} & {\logreturns} \\ \hline
                    \end{tabular}
                    \label{\SETLABEL:RET}
                \end{center}
            \end{table}
        \end{small}

    \subsection{{\market}, Shannon probabilities}
        \label{\SETLABEL:MAXSHANNON}

        The data in table~\ref{\SETLABEL:SHANNON} is condensed from
        sections~\ref{\SETLABELREF:FS}
        and~\ref{\SETLABELREF:MAXSHANNON}.

        \begin{small}
            \begin{table}[ht]
                \begin{center}
                    \caption[{\market}, Shannon
                        probabilities]{{\market}, Shannon
                        probabilities.}
                    \begin{tabular}{|c|c|c|c|} \hline
                        \multicolumn{3}{|c|}{Maximum} & \multicolumn{1}{|c|}{Operational}\\ \hline
                        Fraction of         & $\frac{\frac{\mbox{\scriptsize{mean}}}{\mbox{\scriptsize{rms}}} + 1}{2}$ & \multicolumn{2}{|c|}{From program:}\\ \cline{3-4}
                        positive increments & \hspace{0.01in}                                                          & {\it tsshannonmax}\/    & {\it tsshannon}\/ \\ \hline\hline
                        {\pmax}             & {\avgrms}                                                                & {\shannonmax}   & {\shannonlogreturns} \\ \hline
                    \end{tabular}
                    \label{\SETLABEL:SHANNON}
                \end{center}
            \end{table}
        \end{small}

    \subsection{{\market}, Logistic Analysis}
        \label{\SETLABEL:LAA}

        The data in table~\ref{\SETLABEL:LA} is condensed from
        Section~\ref{\SETLABELREF:LA}\footnote{Note that there are
        numerical stability issues with the methodology used to derive
        the constants---if the non-linear term, $b$, was greater than
        zero, it was set to zero to produce the graphs in
        Section~\ref{\SETLABELREF:LA}.}.

        \begin{small}
            \begin{table}[ht]
                \begin{center}
                    \caption[{\market}, Logistic Analysis.]
                        {{\market}, Logistic Analysis, $x_t = x_{t - 1}\left(a + b \cdot x_{t - 1}\right)$.}
                    \begin{tabular}{|c|c|} \hline
                        $a$ & $b$\\ \hline\hline
                        {\datafractionconstant} & {\datafractionslope}\\ \hline
                    \end{tabular}
                    \label{\SETLABEL:LA}
                \end{center}
            \end{table}
        \end{small}

    \subsection{{\market}, Hurst Coefficients and H  Parameters}
        \label{\SETLABEL:HCHP}

        The data in table~\ref{\SETLABEL:H} is condensed from
        Section~\ref{\SETLABELREF:H}.

        \begin{small}
            \begin{table}[ht]
                \begin{center}
                    \caption[{\market}, Hurst Coefficients and H
                        Parameters]{{\market}, Hurst Coefficients and
                        H Parameters.}
                    \begin{tabular}{|c|c|c|c|} \hline
                        \multicolumn{2}{|c|}{Hurst Coefficients} & \multicolumn{2}{|c|}{H Parameters}\\ \hline
                        Near term   & Far term    & Near term   & Far term \\ \hline\hline
                        {\thurstlow} & {\thurstall} & {\thcalclow} & {\thcalcall} \\ \hline
                    \end{tabular}
                    \label{\SETLABEL:H}
                \end{center}
            \end{table}
        \end{small}

        \begin{small}
            \begin{table}[ht]
                \begin{center}
                    \caption[{\market}, Hurst Coefficients and H
                        Parameters]{{\market}, Hurst Coefficients and
                        H Parameters, as a Derivative.}
                    \begin{tabular}{|c|c|c|c|} \hline
                        \multicolumn{2}{|c|}{Hurst Coefficients} & \multicolumn{2}{|c|}{H Parameters}\\ \hline
                        Near term    & Far term     & Near term    & Far term \\ \hline\hline
                        {\hurstlow} & {\hurstall} & {\hcalclow} & {\hcalcall} \\ \hline
                    \end{tabular}
                    \label{\SETLABEL:TH}
                \end{center}
            \end{table}
        \end{small}

    \subsection{{\market}, verification of the increments}
        \label{\SETLABEL:VI1}

        The data in table~\ref{\SETLABEL:COMP} is condensed from
        Section~\ref{\SETLABELREF:QVA}.

        \begin{small}
            \begin{table}[ht]
                \begin{center}
                    \caption[{\market}, verification of
                        the increments]{{\market}, verification the of
                        the increments, the mean, $\sigma$ is the
                        standard deviation from
                        table~\ref{\SETLABEL:INC},
                        {\datafractionstddev}, and $P$ is the maximum
                        Shannon probability from
                        table~\ref{\SETLABEL:SHANNON}, {\pmax}. In
                        principle, the values should equate.}
                    \begin{tabular}{|c|c|c|} \hline
                        Mean                & $\mbox{rms} (2P - 1)$ & $\frac{{\sigma}(2P - 1)}{2\sqrt{P(P - 1)}} $ \\ \hline\hline
                        {\datafractionmean} & {\rmsp}               & {\sigmap} \\ \hline
                    \end{tabular}
                    \label{\SETLABEL:COMP}
                \end{center}
            \end{table}
        \end{small}

    \subsection{{\market}, verification of the increments}
        \label{\SETLABEL:VI2}

        The data in table~\ref{\SETLABEL:ABS} is condensed from
        Section~\ref{\SETLABELREF:QVA}.

        \begin{small}
            \begin{table}[ht]
                \begin{center}
                    \caption[{\market}, verification of
                        the increments]{{\market}, verification the of
                        increments. In principle, the mean of the
                        absolute value of the increments and the root
                        mean square of the increments should
                        equate\footnote{The absolute value of the
                        normalized increments, when averaged, is
                        related to the root mean square of the
                        increments by a constant. If the normalized
                        increments are a fixed increment, the constant
                        is unity. If the normalized increments have a
                        Gaussian distribution, the constant is
                        $\approx 0.8$ depending on the accuracy of of
                        ``fit'' to a Gaussian distribution.}.}
                    \begin{tabular}{|c|c|} \hline
                        Mean of the               & rms \\
                        absolute value            & \hspace{0.01in} \\ \hline\hline
                        {\datafractionabsmean}    & {\datafractionrms} \\ \hline
                    \end{tabular}
                    \label{\SETLABEL:ABS}
                \end{center}
            \end{table}
        \end{small}

    \subsection{{\market}, $\chi^2$ values of the increments}
        \label{\SETLABEL:XSQ}

        The data in table~\ref{\SETLABEL:XSQT} is condensed from
        Section~\ref{\SETLABELREF:NH}.

        \begin{small}
            \begin{table}[ht]
                \begin{center}
                    \caption[{\market}, $\chi^2$ values of
                        the increments]{{\market}, $\chi^2$ values of
                        the increments. In principle, if the
                        distribution of the normalized increments is a
                        Gaussian distribution, the $\chi^2$ value will
                        be significantly less than the critical
                        value.}
                    \begin{tabular}{|c|c|} \hline
                        $\chi^2$      & Critical Value \\ \hline\hline
                        {\chisquared} & {\critical} \\ \hline
                    \end{tabular}
                    \label{\SETLABEL:XSQT}
                \end{center}
            \end{table}
        \end{small}

    \subsection{{\market}, time series data, empirical and simulated}
        \label{\SETLABEL:SIM}

        The data in table~\ref{\SETLABEL:ES} is condensed from
        Section~\ref{\SETLABELREF:TSUNFAIRBROWNIAN}.

        \begin{small}
            \begin{table}[ht]
                \begin{center}
                    \caption[{\market}, time series data, empirical
                        and simulated]{{\market}, time series data,
                        empirical and simulated, analysis of the
                        normalized increments.}
                    \begin{tabular}{|c|c|c|c|} \hline
                        \multicolumn{2}{|c|}{Empirical} & \multicolumn{2}{|c|}{Simulated}\\ \hline
                        Mean                & Standard              & Mean               & Standard \\
                        \hspace{0.01in}     & deviation             & \hspace{0.01in}    & deviation \\ \hline\hline
                        {\datafractionmean} & {\datafractionstddev} & {\tsunfairbrownianfractionmean} & {\tsunfairbrownianfractionstddev} \\ \hline
                    \end{tabular}
                    \label{\SETLABEL:ES}
                \end{center}
            \end{table}
        \end{small}

    \subsection{{\market}, number of participating companies}
        \label{\SETLABEL:QNC}

        The data in table~\ref{\SETLABEL:NC} is condensed from
        Section~\ref{\SETLABELREF:QNC}.

        \begin{small}
            \begin{table}[ht]
                \begin{center}
                    \caption[{\market}, number of participating
                        companies] {{\market}, number of participating
                        companies.}
                    \begin{tabular}{|c|c|} \hline
                        Number & Shannon probability\\ \hline
                        {\ncompanies} & {\pncompanies}\\ \hline
                    \end{tabular}
                    \label{\SETLABEL:NC}
                \end{center}
            \end{table}
        \end{small}

    \subsection{{\market}, Shannon probability optimizations}
        \label{\SETLABEL:SPO}

        The data in table~\ref{\SETLABEL:SP} is condensed from
        Section~\ref{\SETLABELREF:QNC}.

        \begin{small}
            \begin{table}[ht]
                \begin{center}
                    \caption[{\market}, Shannon probability
                         optimizations] {{\market}, Shannon
                         probability optimization.}
                    \begin{tabular}{|c|c|} \hline
                        optimize capital growth & optimize market growth\\ \hline
                        {\avgrms} & {\pncompanies}\\ \hline
                    \end{tabular}
                    \label{\SETLABEL:SP}
                \end{center}
            \end{table}
        \end{small}

% Local Variables:
% TeX-parse-self: t
% TeX-auto-save: t
% TeX-master: "fractal.tex"
% End:


    \renewcommand{\market}{Simulated Shannon Probability of 0.6}
    \renewcommand{\directory}{../markets/tsunfairbrownian.exponential}
    \renewcommand{\datafractionmean}{0.008052}
\renewcommand{\datafractionmeanbits}{0.011570}
\renewcommand{\datafractionmeanq}{0.002684}
\renewcommand{\datafractionmeanbitsq}{0.003867}
\renewcommand{\datafractionstddev}{0.038579}
\renewcommand{\datafractionrms}{0.039311}
\renewcommand{\avgrms}{0.602414}
\renewcommand{\ncompanies}{5.210454}
\renewcommand{\pncompanies}{0.544866}
\renewcommand{\datafractionabsmean}{0.029745}
\renewcommand{\datafractionabsstddev}{0.025769}
\renewcommand{\datafractionconstant}{0.010041}
\renewcommand{\datafractionconstantbits}{0.014414}
\renewcommand{\datafractionconstantq}{0.003347}
\renewcommand{\datafractionconstantbitsq}{0.004821}
\renewcommand{\datafractionslope}{-0.000021}
\renewcommand{\datafractionabsconstant}{0.035145}
\renewcommand{\datafractionabsslope}{-0.000057}
\renewcommand{\hurstall}{0.659558}
\renewcommand{\hurstlow}{0.707509}
\renewcommand{\hurstlowtwo}{1.415018}
\renewcommand{\hurstlowhundred}{70.750900}
\renewcommand{\hcalcall}{0.184942}
\renewcommand{\hcalclow}{0.102042}
\renewcommand{\shannonmax}{0.604167}
\renewcommand{\twoponemax}{0.208334}
\renewcommand{\logreturns}{0.010456}
\renewcommand{\twologreturns}{1.007274}
\renewcommand{\twologreturnshundred}{0.727387}
\renewcommand{\oneoverlogreturns}{95.638868}
\renewcommand{\pmax}{0.602094}
\renewcommand{\twopminusone}{0.204188}
\renewcommand{\rmsp}{0.008027}
\renewcommand{\twopx}{0.208583}
\renewcommand{\sigmap}{0.008047}
\renewcommand{\tsunfairbrownianfractionmean}{0.007862}
\renewcommand{\tsunfairbrownianfractionstddev}{0.038619}
\renewcommand{\shannonlogreturns}{0.560125}
\renewcommand{\shannonlogreturnshundred}{56.012500}
\renewcommand{\twopone}{0.120250}
\renewcommand{\twoponehundred}{12.025000}
\renewcommand{\hundredtwoponehundred}{87.975000}
\renewcommand{\hundredshannonlogreturnshundred}{43.987500}
\renewcommand{\datatslsqepbits}{0.007623}
\renewcommand{\thurstall}{0.633980}
\renewcommand{\thurstlow}{0.710108}
\renewcommand{\thurstlowtwo}{1.420216}
\renewcommand{\thurstlowhundred}{71.010800}
\renewcommand{\thcalcall}{0.247886}
\renewcommand{\thcalclow}{0.171737}
\renewcommand{\chisquared}{2.862000}
\renewcommand{\critical}{42.557000}

    \renewcommand{\timescale}{time units}
    \subidx{market}{\market}
    \idx{\market}

    \section{\market}

        \renewcommand{\SETLABEL}{\LABPRE:TSTE}
        \renewcommand{\SETLABELQ}{\LABPRE:TSTEQ}
        \label{\SETLABEL}

        \subidx{tscoin}{program}
        \subidx{programs}{tscoin}
        \subidx{tsunfairbrownian}{program}
        \subidx{programs}{tsunfairbrownian}
        \subidx{programs}{tscoin}
        \subidx{tscoin}{program}
        For the analysis, the data was in the directory
        {\directory}\footnote{As a simulation model, the program
        {\it tsunfairbrownian}\/ was run to make a time series data file.  The
        data is by {\timescale}s.}.

        The data in this section is presented in
        Section~\ref{\SETLABELREF}.

        %
% -----------------------------------------------------------------------------
%
% A license is hereby granted to reproduce this software source code and
% to create executable versions from this source code for personal,
% non-commercial use.  The copyright notice included with the software
% must be maintained in all copies produced.
%
% THIS PROGRAM IS PROVIDED "AS IS". THE AUTHOR PROVIDES NO WARRANTIES
% WHATSOEVER, EXPRESSED OR IMPLIED, INCLUDING WARRANTIES OF
% MERCHANTABILITY, TITLE, OR FITNESS FOR ANY PARTICULAR PURPOSE.  THE
% AUTHOR DOES NOT WARRANT THAT USE OF THIS PROGRAM DOES NOT INFRINGE THE
% INTELLECTUAL PROPERTY RIGHTS OF ANY THIRD PARTY IN ANY COUNTRY.
%
% Copyright (c) 1994-2006, John Conover, All Rights Reserved.
%
% Comments and/or bug reports should be addressed to:
%
%     john@email.johncon.com (John Conover)
%
% -----------------------------------------------------------------------------
%
% Revision: \RCSRevision \\
% Revision Time: \RCSTime UMT \\
% Revision Date: \RCSDate \\
% Revision Id: \RCSId \\
% Revision File: \RCSLog \\
\RCS $Revision: 0.0 $
\RCS $Date: 2006/01/20 04:38:13 $
\RCS $Id: tables.tex,v 0.0 2006/01/20 04:38:13 john Exp $
% $Log: tables.tex,v $
% Revision 0.0  2006/01/20 04:38:13  john
% Initial version
%
%
    \subsection{{\market}, normalized increments}
        \label{\SETLABEL:TSA}

        The data in table~\ref{\SETLABEL:INC} is condensed from
        Section~\ref{\SETLABELREF:TSA}.

        \begin{small}
            \begin{table}[ht]
                \begin{center}
                    \caption[{\market}, normalized increments]
                        {{\market}, normalized increments.}
                    \begin{tabular}{|c|c|c|c|c|c|c|c|c|c|} \hline
                        \multicolumn{5}{|c|}{Normalized}                                                                                  & \multicolumn{5}{|c|}{Normalized Absolute Value}\\ \hline
                        Mean                & Standard              & rms                & \multicolumn{2}{|c|}{Least Squares}            & Mean                   & Standard                 & rms                & \multicolumn{2}{|c|}{Least Squares} \\ \cline{4-5}\cline{9-10}
                        \hspace{0.01in}     & deviation             & \hspace{0.01in}    & Constant                & Slope                & \hspace{0.01in}        & deviation                & \hspace{0.01in}    & Constant                   & Slope \\ \hline\hline
                        {\datafractionmean} & {\datafractionstddev} & {\datafractionrms} & {\datafractionconstant} & {\datafractionslope} & {\datafractionabsmean} & {\datafractionabsstddev} & {\datafractionrms} & {\datafractionabsconstant} & {\datafractionabsslope} \\ \hline
                    \end{tabular}
                    \label{\SETLABEL:INC}
                \end{center}
            \end{table}
        \end{small}

    \subsection{{\market}, Logarithmic Returns, in Bits}
        \label{\SETLABEL:LR}

        The data in table~\ref{\SETLABEL:RET} is condensed from
        Section~\ref{\SETLABELREF:FS}.

        \begin{small}
            \begin{table}[ht]
                \begin{center}
                    \caption[{\market}, Logarithmic Returns, in
                        Bits]{{\market}, Logarithmic Returns, in Bits.}
                    \begin{tabular}{|c|c|c|c|} \hline
                        \multicolumn{2}{|c|}{Calculated from Table~\ref{\SETLABEL:INC}} & \multicolumn{2}{|c|}{From program:}\\ \hline
                        Mean                    & Least squares                       & {\it tslsq}\/              & {\it tslogreturns}\/ \\ \hline\hline
                        {\datafractionmeanbits} & {\datafractionconstantbits} & {\datatslsqepbits} & {\logreturns} \\ \hline
                    \end{tabular}
                    \label{\SETLABEL:RET}
                \end{center}
            \end{table}
        \end{small}

    \subsection{{\market}, Shannon probabilities}
        \label{\SETLABEL:MAXSHANNON}

        The data in table~\ref{\SETLABEL:SHANNON} is condensed from
        sections~\ref{\SETLABELREF:FS}
        and~\ref{\SETLABELREF:MAXSHANNON}.

        \begin{small}
            \begin{table}[ht]
                \begin{center}
                    \caption[{\market}, Shannon
                        probabilities]{{\market}, Shannon
                        probabilities.}
                    \begin{tabular}{|c|c|c|c|} \hline
                        \multicolumn{3}{|c|}{Maximum} & \multicolumn{1}{|c|}{Operational}\\ \hline
                        Fraction of         & $\frac{\frac{\mbox{\scriptsize{mean}}}{\mbox{\scriptsize{rms}}} + 1}{2}$ & \multicolumn{2}{|c|}{From program:}\\ \cline{3-4}
                        positive increments & \hspace{0.01in}                                                          & {\it tsshannonmax}\/    & {\it tsshannon}\/ \\ \hline\hline
                        {\pmax}             & {\avgrms}                                                                & {\shannonmax}   & {\shannonlogreturns} \\ \hline
                    \end{tabular}
                    \label{\SETLABEL:SHANNON}
                \end{center}
            \end{table}
        \end{small}

    \subsection{{\market}, Logistic Analysis}
        \label{\SETLABEL:LAA}

        The data in table~\ref{\SETLABEL:LA} is condensed from
        Section~\ref{\SETLABELREF:LA}\footnote{Note that there are
        numerical stability issues with the methodology used to derive
        the constants---if the non-linear term, $b$, was greater than
        zero, it was set to zero to produce the graphs in
        Section~\ref{\SETLABELREF:LA}.}.

        \begin{small}
            \begin{table}[ht]
                \begin{center}
                    \caption[{\market}, Logistic Analysis.]
                        {{\market}, Logistic Analysis, $x_t = x_{t - 1}\left(a + b \cdot x_{t - 1}\right)$.}
                    \begin{tabular}{|c|c|} \hline
                        $a$ & $b$\\ \hline\hline
                        {\datafractionconstant} & {\datafractionslope}\\ \hline
                    \end{tabular}
                    \label{\SETLABEL:LA}
                \end{center}
            \end{table}
        \end{small}

    \subsection{{\market}, Hurst Coefficients and H  Parameters}
        \label{\SETLABEL:HCHP}

        The data in table~\ref{\SETLABEL:H} is condensed from
        Section~\ref{\SETLABELREF:H}.

        \begin{small}
            \begin{table}[ht]
                \begin{center}
                    \caption[{\market}, Hurst Coefficients and H
                        Parameters]{{\market}, Hurst Coefficients and
                        H Parameters.}
                    \begin{tabular}{|c|c|c|c|} \hline
                        \multicolumn{2}{|c|}{Hurst Coefficients} & \multicolumn{2}{|c|}{H Parameters}\\ \hline
                        Near term   & Far term    & Near term   & Far term \\ \hline\hline
                        {\thurstlow} & {\thurstall} & {\thcalclow} & {\thcalcall} \\ \hline
                    \end{tabular}
                    \label{\SETLABEL:H}
                \end{center}
            \end{table}
        \end{small}

        \begin{small}
            \begin{table}[ht]
                \begin{center}
                    \caption[{\market}, Hurst Coefficients and H
                        Parameters]{{\market}, Hurst Coefficients and
                        H Parameters, as a Derivative.}
                    \begin{tabular}{|c|c|c|c|} \hline
                        \multicolumn{2}{|c|}{Hurst Coefficients} & \multicolumn{2}{|c|}{H Parameters}\\ \hline
                        Near term    & Far term     & Near term    & Far term \\ \hline\hline
                        {\hurstlow} & {\hurstall} & {\hcalclow} & {\hcalcall} \\ \hline
                    \end{tabular}
                    \label{\SETLABEL:TH}
                \end{center}
            \end{table}
        \end{small}

    \subsection{{\market}, verification of the increments}
        \label{\SETLABEL:VI1}

        The data in table~\ref{\SETLABEL:COMP} is condensed from
        Section~\ref{\SETLABELREF:QVA}.

        \begin{small}
            \begin{table}[ht]
                \begin{center}
                    \caption[{\market}, verification of
                        the increments]{{\market}, verification the of
                        the increments, the mean, $\sigma$ is the
                        standard deviation from
                        table~\ref{\SETLABEL:INC},
                        {\datafractionstddev}, and $P$ is the maximum
                        Shannon probability from
                        table~\ref{\SETLABEL:SHANNON}, {\pmax}. In
                        principle, the values should equate.}
                    \begin{tabular}{|c|c|c|} \hline
                        Mean                & $\mbox{rms} (2P - 1)$ & $\frac{{\sigma}(2P - 1)}{2\sqrt{P(P - 1)}} $ \\ \hline\hline
                        {\datafractionmean} & {\rmsp}               & {\sigmap} \\ \hline
                    \end{tabular}
                    \label{\SETLABEL:COMP}
                \end{center}
            \end{table}
        \end{small}

    \subsection{{\market}, verification of the increments}
        \label{\SETLABEL:VI2}

        The data in table~\ref{\SETLABEL:ABS} is condensed from
        Section~\ref{\SETLABELREF:QVA}.

        \begin{small}
            \begin{table}[ht]
                \begin{center}
                    \caption[{\market}, verification of
                        the increments]{{\market}, verification the of
                        increments. In principle, the mean of the
                        absolute value of the increments and the root
                        mean square of the increments should
                        equate\footnote{The absolute value of the
                        normalized increments, when averaged, is
                        related to the root mean square of the
                        increments by a constant. If the normalized
                        increments are a fixed increment, the constant
                        is unity. If the normalized increments have a
                        Gaussian distribution, the constant is
                        $\approx 0.8$ depending on the accuracy of of
                        ``fit'' to a Gaussian distribution.}.}
                    \begin{tabular}{|c|c|} \hline
                        Mean of the               & rms \\
                        absolute value            & \hspace{0.01in} \\ \hline\hline
                        {\datafractionabsmean}    & {\datafractionrms} \\ \hline
                    \end{tabular}
                    \label{\SETLABEL:ABS}
                \end{center}
            \end{table}
        \end{small}

    \subsection{{\market}, $\chi^2$ values of the increments}
        \label{\SETLABEL:XSQ}

        The data in table~\ref{\SETLABEL:XSQT} is condensed from
        Section~\ref{\SETLABELREF:NH}.

        \begin{small}
            \begin{table}[ht]
                \begin{center}
                    \caption[{\market}, $\chi^2$ values of
                        the increments]{{\market}, $\chi^2$ values of
                        the increments. In principle, if the
                        distribution of the normalized increments is a
                        Gaussian distribution, the $\chi^2$ value will
                        be significantly less than the critical
                        value.}
                    \begin{tabular}{|c|c|} \hline
                        $\chi^2$      & Critical Value \\ \hline\hline
                        {\chisquared} & {\critical} \\ \hline
                    \end{tabular}
                    \label{\SETLABEL:XSQT}
                \end{center}
            \end{table}
        \end{small}

    \subsection{{\market}, time series data, empirical and simulated}
        \label{\SETLABEL:SIM}

        The data in table~\ref{\SETLABEL:ES} is condensed from
        Section~\ref{\SETLABELREF:TSUNFAIRBROWNIAN}.

        \begin{small}
            \begin{table}[ht]
                \begin{center}
                    \caption[{\market}, time series data, empirical
                        and simulated]{{\market}, time series data,
                        empirical and simulated, analysis of the
                        normalized increments.}
                    \begin{tabular}{|c|c|c|c|} \hline
                        \multicolumn{2}{|c|}{Empirical} & \multicolumn{2}{|c|}{Simulated}\\ \hline
                        Mean                & Standard              & Mean               & Standard \\
                        \hspace{0.01in}     & deviation             & \hspace{0.01in}    & deviation \\ \hline\hline
                        {\datafractionmean} & {\datafractionstddev} & {\tsunfairbrownianfractionmean} & {\tsunfairbrownianfractionstddev} \\ \hline
                    \end{tabular}
                    \label{\SETLABEL:ES}
                \end{center}
            \end{table}
        \end{small}

    \subsection{{\market}, number of participating companies}
        \label{\SETLABEL:QNC}

        The data in table~\ref{\SETLABEL:NC} is condensed from
        Section~\ref{\SETLABELREF:QNC}.

        \begin{small}
            \begin{table}[ht]
                \begin{center}
                    \caption[{\market}, number of participating
                        companies] {{\market}, number of participating
                        companies.}
                    \begin{tabular}{|c|c|} \hline
                        Number & Shannon probability\\ \hline
                        {\ncompanies} & {\pncompanies}\\ \hline
                    \end{tabular}
                    \label{\SETLABEL:NC}
                \end{center}
            \end{table}
        \end{small}

    \subsection{{\market}, Shannon probability optimizations}
        \label{\SETLABEL:SPO}

        The data in table~\ref{\SETLABEL:SP} is condensed from
        Section~\ref{\SETLABELREF:QNC}.

        \begin{small}
            \begin{table}[ht]
                \begin{center}
                    \caption[{\market}, Shannon probability
                         optimizations] {{\market}, Shannon
                         probability optimization.}
                    \begin{tabular}{|c|c|} \hline
                        optimize capital growth & optimize market growth\\ \hline
                        {\avgrms} & {\pncompanies}\\ \hline
                    \end{tabular}
                    \label{\SETLABEL:SP}
                \end{center}
            \end{table}
        \end{small}

% Local Variables:
% TeX-parse-self: t
% TeX-auto-save: t
% TeX-master: "fractal.tex"
% End:


    \renewcommand{\market}{Coins Tossing Game}
    \renewcommand{\directory}{../markets/tscoins}
    \renewcommand{\datafractionmean}{0.008052}
\renewcommand{\datafractionmeanbits}{0.011570}
\renewcommand{\datafractionmeanq}{0.002684}
\renewcommand{\datafractionmeanbitsq}{0.003867}
\renewcommand{\datafractionstddev}{0.038579}
\renewcommand{\datafractionrms}{0.039311}
\renewcommand{\avgrms}{0.602414}
\renewcommand{\ncompanies}{5.210454}
\renewcommand{\pncompanies}{0.544866}
\renewcommand{\datafractionabsmean}{0.029745}
\renewcommand{\datafractionabsstddev}{0.025769}
\renewcommand{\datafractionconstant}{0.010041}
\renewcommand{\datafractionconstantbits}{0.014414}
\renewcommand{\datafractionconstantq}{0.003347}
\renewcommand{\datafractionconstantbitsq}{0.004821}
\renewcommand{\datafractionslope}{-0.000021}
\renewcommand{\datafractionabsconstant}{0.035145}
\renewcommand{\datafractionabsslope}{-0.000057}
\renewcommand{\hurstall}{0.659558}
\renewcommand{\hurstlow}{0.707509}
\renewcommand{\hurstlowtwo}{1.415018}
\renewcommand{\hurstlowhundred}{70.750900}
\renewcommand{\hcalcall}{0.184942}
\renewcommand{\hcalclow}{0.102042}
\renewcommand{\shannonmax}{0.604167}
\renewcommand{\twoponemax}{0.208334}
\renewcommand{\logreturns}{0.010456}
\renewcommand{\twologreturns}{1.007274}
\renewcommand{\twologreturnshundred}{0.727387}
\renewcommand{\oneoverlogreturns}{95.638868}
\renewcommand{\pmax}{0.602094}
\renewcommand{\twopminusone}{0.204188}
\renewcommand{\rmsp}{0.008027}
\renewcommand{\twopx}{0.208583}
\renewcommand{\sigmap}{0.008047}
\renewcommand{\tsunfairbrownianfractionmean}{0.007862}
\renewcommand{\tsunfairbrownianfractionstddev}{0.038619}
\renewcommand{\shannonlogreturns}{0.560125}
\renewcommand{\shannonlogreturnshundred}{56.012500}
\renewcommand{\twopone}{0.120250}
\renewcommand{\twoponehundred}{12.025000}
\renewcommand{\hundredtwoponehundred}{87.975000}
\renewcommand{\hundredshannonlogreturnshundred}{43.987500}
\renewcommand{\datatslsqepbits}{0.007623}
\renewcommand{\thurstall}{0.633980}
\renewcommand{\thurstlow}{0.710108}
\renewcommand{\thurstlowtwo}{1.420216}
\renewcommand{\thurstlowhundred}{71.010800}
\renewcommand{\thcalcall}{0.247886}
\renewcommand{\thcalclow}{0.171737}
\renewcommand{\chisquared}{2.862000}
\renewcommand{\critical}{42.557000}

    \renewcommand{\timescale}{tosses}
    \subidx{market}{\market}
    \idx{\market}

    \section{\market}

        \renewcommand{\SETLABEL}{\LABPRE:CST}
        \renewcommand{\SETLABELQ}{\LABPRE:CSTQ}
        \label{\SETLABEL}
        \renewcommand{\SETLABELREF}{\LABPREREF:CST}

        \subidx{tscoins}{program}
        \subidx{programs}{tscoins}
        For the analysis, the data was in the directory
        {\directory}\footnote{As a simulation model, the program
        {\it tscoins}\/ was run to make a time series data file.
        The data is by {\timescale}.}.

        The data in this section is presented in
        Section~\ref{\SETLABELREF}.

        %
% -----------------------------------------------------------------------------
%
% A license is hereby granted to reproduce this software source code and
% to create executable versions from this source code for personal,
% non-commercial use.  The copyright notice included with the software
% must be maintained in all copies produced.
%
% THIS PROGRAM IS PROVIDED "AS IS". THE AUTHOR PROVIDES NO WARRANTIES
% WHATSOEVER, EXPRESSED OR IMPLIED, INCLUDING WARRANTIES OF
% MERCHANTABILITY, TITLE, OR FITNESS FOR ANY PARTICULAR PURPOSE.  THE
% AUTHOR DOES NOT WARRANT THAT USE OF THIS PROGRAM DOES NOT INFRINGE THE
% INTELLECTUAL PROPERTY RIGHTS OF ANY THIRD PARTY IN ANY COUNTRY.
%
% Copyright (c) 1994-2006, John Conover, All Rights Reserved.
%
% Comments and/or bug reports should be addressed to:
%
%     john@email.johncon.com (John Conover)
%
% -----------------------------------------------------------------------------
%
% Revision: \RCSRevision \\
% Revision Time: \RCSTime UMT \\
% Revision Date: \RCSDate \\
% Revision Id: \RCSId \\
% Revision File: \RCSLog \\
\RCS $Revision: 0.0 $
\RCS $Date: 2006/01/20 04:38:13 $
\RCS $Id: tables.tex,v 0.0 2006/01/20 04:38:13 john Exp $
% $Log: tables.tex,v $
% Revision 0.0  2006/01/20 04:38:13  john
% Initial version
%
%
    \subsection{{\market}, normalized increments}
        \label{\SETLABEL:TSA}

        The data in table~\ref{\SETLABEL:INC} is condensed from
        Section~\ref{\SETLABELREF:TSA}.

        \begin{small}
            \begin{table}[ht]
                \begin{center}
                    \caption[{\market}, normalized increments]
                        {{\market}, normalized increments.}
                    \begin{tabular}{|c|c|c|c|c|c|c|c|c|c|} \hline
                        \multicolumn{5}{|c|}{Normalized}                                                                                  & \multicolumn{5}{|c|}{Normalized Absolute Value}\\ \hline
                        Mean                & Standard              & rms                & \multicolumn{2}{|c|}{Least Squares}            & Mean                   & Standard                 & rms                & \multicolumn{2}{|c|}{Least Squares} \\ \cline{4-5}\cline{9-10}
                        \hspace{0.01in}     & deviation             & \hspace{0.01in}    & Constant                & Slope                & \hspace{0.01in}        & deviation                & \hspace{0.01in}    & Constant                   & Slope \\ \hline\hline
                        {\datafractionmean} & {\datafractionstddev} & {\datafractionrms} & {\datafractionconstant} & {\datafractionslope} & {\datafractionabsmean} & {\datafractionabsstddev} & {\datafractionrms} & {\datafractionabsconstant} & {\datafractionabsslope} \\ \hline
                    \end{tabular}
                    \label{\SETLABEL:INC}
                \end{center}
            \end{table}
        \end{small}

    \subsection{{\market}, Logarithmic Returns, in Bits}
        \label{\SETLABEL:LR}

        The data in table~\ref{\SETLABEL:RET} is condensed from
        Section~\ref{\SETLABELREF:FS}.

        \begin{small}
            \begin{table}[ht]
                \begin{center}
                    \caption[{\market}, Logarithmic Returns, in
                        Bits]{{\market}, Logarithmic Returns, in Bits.}
                    \begin{tabular}{|c|c|c|c|} \hline
                        \multicolumn{2}{|c|}{Calculated from Table~\ref{\SETLABEL:INC}} & \multicolumn{2}{|c|}{From program:}\\ \hline
                        Mean                    & Least squares                       & {\it tslsq}\/              & {\it tslogreturns}\/ \\ \hline\hline
                        {\datafractionmeanbits} & {\datafractionconstantbits} & {\datatslsqepbits} & {\logreturns} \\ \hline
                    \end{tabular}
                    \label{\SETLABEL:RET}
                \end{center}
            \end{table}
        \end{small}

    \subsection{{\market}, Shannon probabilities}
        \label{\SETLABEL:MAXSHANNON}

        The data in table~\ref{\SETLABEL:SHANNON} is condensed from
        sections~\ref{\SETLABELREF:FS}
        and~\ref{\SETLABELREF:MAXSHANNON}.

        \begin{small}
            \begin{table}[ht]
                \begin{center}
                    \caption[{\market}, Shannon
                        probabilities]{{\market}, Shannon
                        probabilities.}
                    \begin{tabular}{|c|c|c|c|} \hline
                        \multicolumn{3}{|c|}{Maximum} & \multicolumn{1}{|c|}{Operational}\\ \hline
                        Fraction of         & $\frac{\frac{\mbox{\scriptsize{mean}}}{\mbox{\scriptsize{rms}}} + 1}{2}$ & \multicolumn{2}{|c|}{From program:}\\ \cline{3-4}
                        positive increments & \hspace{0.01in}                                                          & {\it tsshannonmax}\/    & {\it tsshannon}\/ \\ \hline\hline
                        {\pmax}             & {\avgrms}                                                                & {\shannonmax}   & {\shannonlogreturns} \\ \hline
                    \end{tabular}
                    \label{\SETLABEL:SHANNON}
                \end{center}
            \end{table}
        \end{small}

    \subsection{{\market}, Logistic Analysis}
        \label{\SETLABEL:LAA}

        The data in table~\ref{\SETLABEL:LA} is condensed from
        Section~\ref{\SETLABELREF:LA}\footnote{Note that there are
        numerical stability issues with the methodology used to derive
        the constants---if the non-linear term, $b$, was greater than
        zero, it was set to zero to produce the graphs in
        Section~\ref{\SETLABELREF:LA}.}.

        \begin{small}
            \begin{table}[ht]
                \begin{center}
                    \caption[{\market}, Logistic Analysis.]
                        {{\market}, Logistic Analysis, $x_t = x_{t - 1}\left(a + b \cdot x_{t - 1}\right)$.}
                    \begin{tabular}{|c|c|} \hline
                        $a$ & $b$\\ \hline\hline
                        {\datafractionconstant} & {\datafractionslope}\\ \hline
                    \end{tabular}
                    \label{\SETLABEL:LA}
                \end{center}
            \end{table}
        \end{small}

    \subsection{{\market}, Hurst Coefficients and H  Parameters}
        \label{\SETLABEL:HCHP}

        The data in table~\ref{\SETLABEL:H} is condensed from
        Section~\ref{\SETLABELREF:H}.

        \begin{small}
            \begin{table}[ht]
                \begin{center}
                    \caption[{\market}, Hurst Coefficients and H
                        Parameters]{{\market}, Hurst Coefficients and
                        H Parameters.}
                    \begin{tabular}{|c|c|c|c|} \hline
                        \multicolumn{2}{|c|}{Hurst Coefficients} & \multicolumn{2}{|c|}{H Parameters}\\ \hline
                        Near term   & Far term    & Near term   & Far term \\ \hline\hline
                        {\thurstlow} & {\thurstall} & {\thcalclow} & {\thcalcall} \\ \hline
                    \end{tabular}
                    \label{\SETLABEL:H}
                \end{center}
            \end{table}
        \end{small}

        \begin{small}
            \begin{table}[ht]
                \begin{center}
                    \caption[{\market}, Hurst Coefficients and H
                        Parameters]{{\market}, Hurst Coefficients and
                        H Parameters, as a Derivative.}
                    \begin{tabular}{|c|c|c|c|} \hline
                        \multicolumn{2}{|c|}{Hurst Coefficients} & \multicolumn{2}{|c|}{H Parameters}\\ \hline
                        Near term    & Far term     & Near term    & Far term \\ \hline\hline
                        {\hurstlow} & {\hurstall} & {\hcalclow} & {\hcalcall} \\ \hline
                    \end{tabular}
                    \label{\SETLABEL:TH}
                \end{center}
            \end{table}
        \end{small}

    \subsection{{\market}, verification of the increments}
        \label{\SETLABEL:VI1}

        The data in table~\ref{\SETLABEL:COMP} is condensed from
        Section~\ref{\SETLABELREF:QVA}.

        \begin{small}
            \begin{table}[ht]
                \begin{center}
                    \caption[{\market}, verification of
                        the increments]{{\market}, verification the of
                        the increments, the mean, $\sigma$ is the
                        standard deviation from
                        table~\ref{\SETLABEL:INC},
                        {\datafractionstddev}, and $P$ is the maximum
                        Shannon probability from
                        table~\ref{\SETLABEL:SHANNON}, {\pmax}. In
                        principle, the values should equate.}
                    \begin{tabular}{|c|c|c|} \hline
                        Mean                & $\mbox{rms} (2P - 1)$ & $\frac{{\sigma}(2P - 1)}{2\sqrt{P(P - 1)}} $ \\ \hline\hline
                        {\datafractionmean} & {\rmsp}               & {\sigmap} \\ \hline
                    \end{tabular}
                    \label{\SETLABEL:COMP}
                \end{center}
            \end{table}
        \end{small}

    \subsection{{\market}, verification of the increments}
        \label{\SETLABEL:VI2}

        The data in table~\ref{\SETLABEL:ABS} is condensed from
        Section~\ref{\SETLABELREF:QVA}.

        \begin{small}
            \begin{table}[ht]
                \begin{center}
                    \caption[{\market}, verification of
                        the increments]{{\market}, verification the of
                        increments. In principle, the mean of the
                        absolute value of the increments and the root
                        mean square of the increments should
                        equate\footnote{The absolute value of the
                        normalized increments, when averaged, is
                        related to the root mean square of the
                        increments by a constant. If the normalized
                        increments are a fixed increment, the constant
                        is unity. If the normalized increments have a
                        Gaussian distribution, the constant is
                        $\approx 0.8$ depending on the accuracy of of
                        ``fit'' to a Gaussian distribution.}.}
                    \begin{tabular}{|c|c|} \hline
                        Mean of the               & rms \\
                        absolute value            & \hspace{0.01in} \\ \hline\hline
                        {\datafractionabsmean}    & {\datafractionrms} \\ \hline
                    \end{tabular}
                    \label{\SETLABEL:ABS}
                \end{center}
            \end{table}
        \end{small}

    \subsection{{\market}, $\chi^2$ values of the increments}
        \label{\SETLABEL:XSQ}

        The data in table~\ref{\SETLABEL:XSQT} is condensed from
        Section~\ref{\SETLABELREF:NH}.

        \begin{small}
            \begin{table}[ht]
                \begin{center}
                    \caption[{\market}, $\chi^2$ values of
                        the increments]{{\market}, $\chi^2$ values of
                        the increments. In principle, if the
                        distribution of the normalized increments is a
                        Gaussian distribution, the $\chi^2$ value will
                        be significantly less than the critical
                        value.}
                    \begin{tabular}{|c|c|} \hline
                        $\chi^2$      & Critical Value \\ \hline\hline
                        {\chisquared} & {\critical} \\ \hline
                    \end{tabular}
                    \label{\SETLABEL:XSQT}
                \end{center}
            \end{table}
        \end{small}

    \subsection{{\market}, time series data, empirical and simulated}
        \label{\SETLABEL:SIM}

        The data in table~\ref{\SETLABEL:ES} is condensed from
        Section~\ref{\SETLABELREF:TSUNFAIRBROWNIAN}.

        \begin{small}
            \begin{table}[ht]
                \begin{center}
                    \caption[{\market}, time series data, empirical
                        and simulated]{{\market}, time series data,
                        empirical and simulated, analysis of the
                        normalized increments.}
                    \begin{tabular}{|c|c|c|c|} \hline
                        \multicolumn{2}{|c|}{Empirical} & \multicolumn{2}{|c|}{Simulated}\\ \hline
                        Mean                & Standard              & Mean               & Standard \\
                        \hspace{0.01in}     & deviation             & \hspace{0.01in}    & deviation \\ \hline\hline
                        {\datafractionmean} & {\datafractionstddev} & {\tsunfairbrownianfractionmean} & {\tsunfairbrownianfractionstddev} \\ \hline
                    \end{tabular}
                    \label{\SETLABEL:ES}
                \end{center}
            \end{table}
        \end{small}

    \subsection{{\market}, number of participating companies}
        \label{\SETLABEL:QNC}

        The data in table~\ref{\SETLABEL:NC} is condensed from
        Section~\ref{\SETLABELREF:QNC}.

        \begin{small}
            \begin{table}[ht]
                \begin{center}
                    \caption[{\market}, number of participating
                        companies] {{\market}, number of participating
                        companies.}
                    \begin{tabular}{|c|c|} \hline
                        Number & Shannon probability\\ \hline
                        {\ncompanies} & {\pncompanies}\\ \hline
                    \end{tabular}
                    \label{\SETLABEL:NC}
                \end{center}
            \end{table}
        \end{small}

    \subsection{{\market}, Shannon probability optimizations}
        \label{\SETLABEL:SPO}

        The data in table~\ref{\SETLABEL:SP} is condensed from
        Section~\ref{\SETLABELREF:QNC}.

        \begin{small}
            \begin{table}[ht]
                \begin{center}
                    \caption[{\market}, Shannon probability
                         optimizations] {{\market}, Shannon
                         probability optimization.}
                    \begin{tabular}{|c|c|} \hline
                        optimize capital growth & optimize market growth\\ \hline
                        {\avgrms} & {\pncompanies}\\ \hline
                    \end{tabular}
                    \label{\SETLABEL:SP}
                \end{center}
            \end{table}
        \end{small}

% Local Variables:
% TeX-parse-self: t
% TeX-auto-save: t
% TeX-master: "fractal.tex"
% End:


    \renewcommand{\market}{Non Optimal Coins Tossing Game}
    \renewcommand{\directory}{../markets/tscoins-f}
    \renewcommand{\datafractionmean}{0.008052}
\renewcommand{\datafractionmeanbits}{0.011570}
\renewcommand{\datafractionmeanq}{0.002684}
\renewcommand{\datafractionmeanbitsq}{0.003867}
\renewcommand{\datafractionstddev}{0.038579}
\renewcommand{\datafractionrms}{0.039311}
\renewcommand{\avgrms}{0.602414}
\renewcommand{\ncompanies}{5.210454}
\renewcommand{\pncompanies}{0.544866}
\renewcommand{\datafractionabsmean}{0.029745}
\renewcommand{\datafractionabsstddev}{0.025769}
\renewcommand{\datafractionconstant}{0.010041}
\renewcommand{\datafractionconstantbits}{0.014414}
\renewcommand{\datafractionconstantq}{0.003347}
\renewcommand{\datafractionconstantbitsq}{0.004821}
\renewcommand{\datafractionslope}{-0.000021}
\renewcommand{\datafractionabsconstant}{0.035145}
\renewcommand{\datafractionabsslope}{-0.000057}
\renewcommand{\hurstall}{0.659558}
\renewcommand{\hurstlow}{0.707509}
\renewcommand{\hurstlowtwo}{1.415018}
\renewcommand{\hurstlowhundred}{70.750900}
\renewcommand{\hcalcall}{0.184942}
\renewcommand{\hcalclow}{0.102042}
\renewcommand{\shannonmax}{0.604167}
\renewcommand{\twoponemax}{0.208334}
\renewcommand{\logreturns}{0.010456}
\renewcommand{\twologreturns}{1.007274}
\renewcommand{\twologreturnshundred}{0.727387}
\renewcommand{\oneoverlogreturns}{95.638868}
\renewcommand{\pmax}{0.602094}
\renewcommand{\twopminusone}{0.204188}
\renewcommand{\rmsp}{0.008027}
\renewcommand{\twopx}{0.208583}
\renewcommand{\sigmap}{0.008047}
\renewcommand{\tsunfairbrownianfractionmean}{0.007862}
\renewcommand{\tsunfairbrownianfractionstddev}{0.038619}
\renewcommand{\shannonlogreturns}{0.560125}
\renewcommand{\shannonlogreturnshundred}{56.012500}
\renewcommand{\twopone}{0.120250}
\renewcommand{\twoponehundred}{12.025000}
\renewcommand{\hundredtwoponehundred}{87.975000}
\renewcommand{\hundredshannonlogreturnshundred}{43.987500}
\renewcommand{\datatslsqepbits}{0.007623}
\renewcommand{\thurstall}{0.633980}
\renewcommand{\thurstlow}{0.710108}
\renewcommand{\thurstlowtwo}{1.420216}
\renewcommand{\thurstlowhundred}{71.010800}
\renewcommand{\thcalcall}{0.247886}
\renewcommand{\thcalclow}{0.171737}
\renewcommand{\chisquared}{2.862000}
\renewcommand{\critical}{42.557000}

    \renewcommand{\timescale}{tosses}
    \subidx{market}{\market}
    \idx{\market}

    \section{\market}

        \renewcommand{\SETLABEL}{\LABPRE:NOCST}
        \renewcommand{\SETLABELQ}{\LABPRE:NOCSTQ}
        \label{\SETLABEL}
        \renewcommand{\SETLABELREF}{\LABPREREF:NOCST}

        \subidx{tscoins}{program}
        \subidx{programs}{tscoins}
        For the analysis, the data was in the directory
        {\directory}\footnote{As a simulation model, the program
        {\it tscoins}\/ was run to make a time series data file.
        The data is by {\timescale}.}.

        The data in this section is presented in
        Section~\ref{\SETLABELREF}.

        %
% -----------------------------------------------------------------------------
%
% A license is hereby granted to reproduce this software source code and
% to create executable versions from this source code for personal,
% non-commercial use.  The copyright notice included with the software
% must be maintained in all copies produced.
%
% THIS PROGRAM IS PROVIDED "AS IS". THE AUTHOR PROVIDES NO WARRANTIES
% WHATSOEVER, EXPRESSED OR IMPLIED, INCLUDING WARRANTIES OF
% MERCHANTABILITY, TITLE, OR FITNESS FOR ANY PARTICULAR PURPOSE.  THE
% AUTHOR DOES NOT WARRANT THAT USE OF THIS PROGRAM DOES NOT INFRINGE THE
% INTELLECTUAL PROPERTY RIGHTS OF ANY THIRD PARTY IN ANY COUNTRY.
%
% Copyright (c) 1994-2006, John Conover, All Rights Reserved.
%
% Comments and/or bug reports should be addressed to:
%
%     john@email.johncon.com (John Conover)
%
% -----------------------------------------------------------------------------
%
% Revision: \RCSRevision \\
% Revision Time: \RCSTime UMT \\
% Revision Date: \RCSDate \\
% Revision Id: \RCSId \\
% Revision File: \RCSLog \\
\RCS $Revision: 0.0 $
\RCS $Date: 2006/01/20 04:38:13 $
\RCS $Id: tables.tex,v 0.0 2006/01/20 04:38:13 john Exp $
% $Log: tables.tex,v $
% Revision 0.0  2006/01/20 04:38:13  john
% Initial version
%
%
    \subsection{{\market}, normalized increments}
        \label{\SETLABEL:TSA}

        The data in table~\ref{\SETLABEL:INC} is condensed from
        Section~\ref{\SETLABELREF:TSA}.

        \begin{small}
            \begin{table}[ht]
                \begin{center}
                    \caption[{\market}, normalized increments]
                        {{\market}, normalized increments.}
                    \begin{tabular}{|c|c|c|c|c|c|c|c|c|c|} \hline
                        \multicolumn{5}{|c|}{Normalized}                                                                                  & \multicolumn{5}{|c|}{Normalized Absolute Value}\\ \hline
                        Mean                & Standard              & rms                & \multicolumn{2}{|c|}{Least Squares}            & Mean                   & Standard                 & rms                & \multicolumn{2}{|c|}{Least Squares} \\ \cline{4-5}\cline{9-10}
                        \hspace{0.01in}     & deviation             & \hspace{0.01in}    & Constant                & Slope                & \hspace{0.01in}        & deviation                & \hspace{0.01in}    & Constant                   & Slope \\ \hline\hline
                        {\datafractionmean} & {\datafractionstddev} & {\datafractionrms} & {\datafractionconstant} & {\datafractionslope} & {\datafractionabsmean} & {\datafractionabsstddev} & {\datafractionrms} & {\datafractionabsconstant} & {\datafractionabsslope} \\ \hline
                    \end{tabular}
                    \label{\SETLABEL:INC}
                \end{center}
            \end{table}
        \end{small}

    \subsection{{\market}, Logarithmic Returns, in Bits}
        \label{\SETLABEL:LR}

        The data in table~\ref{\SETLABEL:RET} is condensed from
        Section~\ref{\SETLABELREF:FS}.

        \begin{small}
            \begin{table}[ht]
                \begin{center}
                    \caption[{\market}, Logarithmic Returns, in
                        Bits]{{\market}, Logarithmic Returns, in Bits.}
                    \begin{tabular}{|c|c|c|c|} \hline
                        \multicolumn{2}{|c|}{Calculated from Table~\ref{\SETLABEL:INC}} & \multicolumn{2}{|c|}{From program:}\\ \hline
                        Mean                    & Least squares                       & {\it tslsq}\/              & {\it tslogreturns}\/ \\ \hline\hline
                        {\datafractionmeanbits} & {\datafractionconstantbits} & {\datatslsqepbits} & {\logreturns} \\ \hline
                    \end{tabular}
                    \label{\SETLABEL:RET}
                \end{center}
            \end{table}
        \end{small}

    \subsection{{\market}, Shannon probabilities}
        \label{\SETLABEL:MAXSHANNON}

        The data in table~\ref{\SETLABEL:SHANNON} is condensed from
        sections~\ref{\SETLABELREF:FS}
        and~\ref{\SETLABELREF:MAXSHANNON}.

        \begin{small}
            \begin{table}[ht]
                \begin{center}
                    \caption[{\market}, Shannon
                        probabilities]{{\market}, Shannon
                        probabilities.}
                    \begin{tabular}{|c|c|c|c|} \hline
                        \multicolumn{3}{|c|}{Maximum} & \multicolumn{1}{|c|}{Operational}\\ \hline
                        Fraction of         & $\frac{\frac{\mbox{\scriptsize{mean}}}{\mbox{\scriptsize{rms}}} + 1}{2}$ & \multicolumn{2}{|c|}{From program:}\\ \cline{3-4}
                        positive increments & \hspace{0.01in}                                                          & {\it tsshannonmax}\/    & {\it tsshannon}\/ \\ \hline\hline
                        {\pmax}             & {\avgrms}                                                                & {\shannonmax}   & {\shannonlogreturns} \\ \hline
                    \end{tabular}
                    \label{\SETLABEL:SHANNON}
                \end{center}
            \end{table}
        \end{small}

    \subsection{{\market}, Logistic Analysis}
        \label{\SETLABEL:LAA}

        The data in table~\ref{\SETLABEL:LA} is condensed from
        Section~\ref{\SETLABELREF:LA}\footnote{Note that there are
        numerical stability issues with the methodology used to derive
        the constants---if the non-linear term, $b$, was greater than
        zero, it was set to zero to produce the graphs in
        Section~\ref{\SETLABELREF:LA}.}.

        \begin{small}
            \begin{table}[ht]
                \begin{center}
                    \caption[{\market}, Logistic Analysis.]
                        {{\market}, Logistic Analysis, $x_t = x_{t - 1}\left(a + b \cdot x_{t - 1}\right)$.}
                    \begin{tabular}{|c|c|} \hline
                        $a$ & $b$\\ \hline\hline
                        {\datafractionconstant} & {\datafractionslope}\\ \hline
                    \end{tabular}
                    \label{\SETLABEL:LA}
                \end{center}
            \end{table}
        \end{small}

    \subsection{{\market}, Hurst Coefficients and H  Parameters}
        \label{\SETLABEL:HCHP}

        The data in table~\ref{\SETLABEL:H} is condensed from
        Section~\ref{\SETLABELREF:H}.

        \begin{small}
            \begin{table}[ht]
                \begin{center}
                    \caption[{\market}, Hurst Coefficients and H
                        Parameters]{{\market}, Hurst Coefficients and
                        H Parameters.}
                    \begin{tabular}{|c|c|c|c|} \hline
                        \multicolumn{2}{|c|}{Hurst Coefficients} & \multicolumn{2}{|c|}{H Parameters}\\ \hline
                        Near term   & Far term    & Near term   & Far term \\ \hline\hline
                        {\thurstlow} & {\thurstall} & {\thcalclow} & {\thcalcall} \\ \hline
                    \end{tabular}
                    \label{\SETLABEL:H}
                \end{center}
            \end{table}
        \end{small}

        \begin{small}
            \begin{table}[ht]
                \begin{center}
                    \caption[{\market}, Hurst Coefficients and H
                        Parameters]{{\market}, Hurst Coefficients and
                        H Parameters, as a Derivative.}
                    \begin{tabular}{|c|c|c|c|} \hline
                        \multicolumn{2}{|c|}{Hurst Coefficients} & \multicolumn{2}{|c|}{H Parameters}\\ \hline
                        Near term    & Far term     & Near term    & Far term \\ \hline\hline
                        {\hurstlow} & {\hurstall} & {\hcalclow} & {\hcalcall} \\ \hline
                    \end{tabular}
                    \label{\SETLABEL:TH}
                \end{center}
            \end{table}
        \end{small}

    \subsection{{\market}, verification of the increments}
        \label{\SETLABEL:VI1}

        The data in table~\ref{\SETLABEL:COMP} is condensed from
        Section~\ref{\SETLABELREF:QVA}.

        \begin{small}
            \begin{table}[ht]
                \begin{center}
                    \caption[{\market}, verification of
                        the increments]{{\market}, verification the of
                        the increments, the mean, $\sigma$ is the
                        standard deviation from
                        table~\ref{\SETLABEL:INC},
                        {\datafractionstddev}, and $P$ is the maximum
                        Shannon probability from
                        table~\ref{\SETLABEL:SHANNON}, {\pmax}. In
                        principle, the values should equate.}
                    \begin{tabular}{|c|c|c|} \hline
                        Mean                & $\mbox{rms} (2P - 1)$ & $\frac{{\sigma}(2P - 1)}{2\sqrt{P(P - 1)}} $ \\ \hline\hline
                        {\datafractionmean} & {\rmsp}               & {\sigmap} \\ \hline
                    \end{tabular}
                    \label{\SETLABEL:COMP}
                \end{center}
            \end{table}
        \end{small}

    \subsection{{\market}, verification of the increments}
        \label{\SETLABEL:VI2}

        The data in table~\ref{\SETLABEL:ABS} is condensed from
        Section~\ref{\SETLABELREF:QVA}.

        \begin{small}
            \begin{table}[ht]
                \begin{center}
                    \caption[{\market}, verification of
                        the increments]{{\market}, verification the of
                        increments. In principle, the mean of the
                        absolute value of the increments and the root
                        mean square of the increments should
                        equate\footnote{The absolute value of the
                        normalized increments, when averaged, is
                        related to the root mean square of the
                        increments by a constant. If the normalized
                        increments are a fixed increment, the constant
                        is unity. If the normalized increments have a
                        Gaussian distribution, the constant is
                        $\approx 0.8$ depending on the accuracy of of
                        ``fit'' to a Gaussian distribution.}.}
                    \begin{tabular}{|c|c|} \hline
                        Mean of the               & rms \\
                        absolute value            & \hspace{0.01in} \\ \hline\hline
                        {\datafractionabsmean}    & {\datafractionrms} \\ \hline
                    \end{tabular}
                    \label{\SETLABEL:ABS}
                \end{center}
            \end{table}
        \end{small}

    \subsection{{\market}, $\chi^2$ values of the increments}
        \label{\SETLABEL:XSQ}

        The data in table~\ref{\SETLABEL:XSQT} is condensed from
        Section~\ref{\SETLABELREF:NH}.

        \begin{small}
            \begin{table}[ht]
                \begin{center}
                    \caption[{\market}, $\chi^2$ values of
                        the increments]{{\market}, $\chi^2$ values of
                        the increments. In principle, if the
                        distribution of the normalized increments is a
                        Gaussian distribution, the $\chi^2$ value will
                        be significantly less than the critical
                        value.}
                    \begin{tabular}{|c|c|} \hline
                        $\chi^2$      & Critical Value \\ \hline\hline
                        {\chisquared} & {\critical} \\ \hline
                    \end{tabular}
                    \label{\SETLABEL:XSQT}
                \end{center}
            \end{table}
        \end{small}

    \subsection{{\market}, time series data, empirical and simulated}
        \label{\SETLABEL:SIM}

        The data in table~\ref{\SETLABEL:ES} is condensed from
        Section~\ref{\SETLABELREF:TSUNFAIRBROWNIAN}.

        \begin{small}
            \begin{table}[ht]
                \begin{center}
                    \caption[{\market}, time series data, empirical
                        and simulated]{{\market}, time series data,
                        empirical and simulated, analysis of the
                        normalized increments.}
                    \begin{tabular}{|c|c|c|c|} \hline
                        \multicolumn{2}{|c|}{Empirical} & \multicolumn{2}{|c|}{Simulated}\\ \hline
                        Mean                & Standard              & Mean               & Standard \\
                        \hspace{0.01in}     & deviation             & \hspace{0.01in}    & deviation \\ \hline\hline
                        {\datafractionmean} & {\datafractionstddev} & {\tsunfairbrownianfractionmean} & {\tsunfairbrownianfractionstddev} \\ \hline
                    \end{tabular}
                    \label{\SETLABEL:ES}
                \end{center}
            \end{table}
        \end{small}

    \subsection{{\market}, number of participating companies}
        \label{\SETLABEL:QNC}

        The data in table~\ref{\SETLABEL:NC} is condensed from
        Section~\ref{\SETLABELREF:QNC}.

        \begin{small}
            \begin{table}[ht]
                \begin{center}
                    \caption[{\market}, number of participating
                        companies] {{\market}, number of participating
                        companies.}
                    \begin{tabular}{|c|c|} \hline
                        Number & Shannon probability\\ \hline
                        {\ncompanies} & {\pncompanies}\\ \hline
                    \end{tabular}
                    \label{\SETLABEL:NC}
                \end{center}
            \end{table}
        \end{small}

    \subsection{{\market}, Shannon probability optimizations}
        \label{\SETLABEL:SPO}

        The data in table~\ref{\SETLABEL:SP} is condensed from
        Section~\ref{\SETLABELREF:QNC}.

        \begin{small}
            \begin{table}[ht]
                \begin{center}
                    \caption[{\market}, Shannon probability
                         optimizations] {{\market}, Shannon
                         probability optimization.}
                    \begin{tabular}{|c|c|} \hline
                        optimize capital growth & optimize market growth\\ \hline
                        {\avgrms} & {\pncompanies}\\ \hline
                    \end{tabular}
                    \label{\SETLABEL:SP}
                \end{center}
            \end{table}
        \end{small}

% Local Variables:
% TeX-parse-self: t
% TeX-auto-save: t
% TeX-master: "fractal.tex"
% End:


    \renewcommand{\market}{Non Optimal Logistic Coins Tossing Game}
    \renewcommand{\directory}{../markets/tscoins-b}
    \renewcommand{\datafractionmean}{0.008052}
\renewcommand{\datafractionmeanbits}{0.011570}
\renewcommand{\datafractionmeanq}{0.002684}
\renewcommand{\datafractionmeanbitsq}{0.003867}
\renewcommand{\datafractionstddev}{0.038579}
\renewcommand{\datafractionrms}{0.039311}
\renewcommand{\avgrms}{0.602414}
\renewcommand{\ncompanies}{5.210454}
\renewcommand{\pncompanies}{0.544866}
\renewcommand{\datafractionabsmean}{0.029745}
\renewcommand{\datafractionabsstddev}{0.025769}
\renewcommand{\datafractionconstant}{0.010041}
\renewcommand{\datafractionconstantbits}{0.014414}
\renewcommand{\datafractionconstantq}{0.003347}
\renewcommand{\datafractionconstantbitsq}{0.004821}
\renewcommand{\datafractionslope}{-0.000021}
\renewcommand{\datafractionabsconstant}{0.035145}
\renewcommand{\datafractionabsslope}{-0.000057}
\renewcommand{\hurstall}{0.659558}
\renewcommand{\hurstlow}{0.707509}
\renewcommand{\hurstlowtwo}{1.415018}
\renewcommand{\hurstlowhundred}{70.750900}
\renewcommand{\hcalcall}{0.184942}
\renewcommand{\hcalclow}{0.102042}
\renewcommand{\shannonmax}{0.604167}
\renewcommand{\twoponemax}{0.208334}
\renewcommand{\logreturns}{0.010456}
\renewcommand{\twologreturns}{1.007274}
\renewcommand{\twologreturnshundred}{0.727387}
\renewcommand{\oneoverlogreturns}{95.638868}
\renewcommand{\pmax}{0.602094}
\renewcommand{\twopminusone}{0.204188}
\renewcommand{\rmsp}{0.008027}
\renewcommand{\twopx}{0.208583}
\renewcommand{\sigmap}{0.008047}
\renewcommand{\tsunfairbrownianfractionmean}{0.007862}
\renewcommand{\tsunfairbrownianfractionstddev}{0.038619}
\renewcommand{\shannonlogreturns}{0.560125}
\renewcommand{\shannonlogreturnshundred}{56.012500}
\renewcommand{\twopone}{0.120250}
\renewcommand{\twoponehundred}{12.025000}
\renewcommand{\hundredtwoponehundred}{87.975000}
\renewcommand{\hundredshannonlogreturnshundred}{43.987500}
\renewcommand{\datatslsqepbits}{0.007623}
\renewcommand{\thurstall}{0.633980}
\renewcommand{\thurstlow}{0.710108}
\renewcommand{\thurstlowtwo}{1.420216}
\renewcommand{\thurstlowhundred}{71.010800}
\renewcommand{\thcalcall}{0.247886}
\renewcommand{\thcalclow}{0.171737}
\renewcommand{\chisquared}{2.862000}
\renewcommand{\critical}{42.557000}

    \renewcommand{\timescale}{tosses}
    \subidx{market}{\market}
    \idx{\market}

    \section{\market}
        \subidx{market}{non-linearity}
        \subidx{non-linearity}{market}
        \subidx{logistic}{function}

        \renewcommand{\SETLABEL}{\LABPRE:NOLCST}
        \renewcommand{\SETLABELQ}{\LABPRE:NOLCSTQ}
        \label{\SETLABEL}
        \renewcommand{\SETLABELREF}{\LABPREREF:NOLCST}

        \subidx{tscoins}{program}
        \subidx{programs}{tscoins}
        For the analysis, the data was in the directory
        {\directory}\footnote{As a simulation model, the program
        {\it tscoins}\/ was run to make a time series data file.
        The data is by {\timescale}.}.

        The data in this section is presented in
        Section~\ref{\SETLABELREF}.

        %
% -----------------------------------------------------------------------------
%
% A license is hereby granted to reproduce this software source code and
% to create executable versions from this source code for personal,
% non-commercial use.  The copyright notice included with the software
% must be maintained in all copies produced.
%
% THIS PROGRAM IS PROVIDED "AS IS". THE AUTHOR PROVIDES NO WARRANTIES
% WHATSOEVER, EXPRESSED OR IMPLIED, INCLUDING WARRANTIES OF
% MERCHANTABILITY, TITLE, OR FITNESS FOR ANY PARTICULAR PURPOSE.  THE
% AUTHOR DOES NOT WARRANT THAT USE OF THIS PROGRAM DOES NOT INFRINGE THE
% INTELLECTUAL PROPERTY RIGHTS OF ANY THIRD PARTY IN ANY COUNTRY.
%
% Copyright (c) 1994-2006, John Conover, All Rights Reserved.
%
% Comments and/or bug reports should be addressed to:
%
%     john@email.johncon.com (John Conover)
%
% -----------------------------------------------------------------------------
%
% Revision: \RCSRevision \\
% Revision Time: \RCSTime UMT \\
% Revision Date: \RCSDate \\
% Revision Id: \RCSId \\
% Revision File: \RCSLog \\
\RCS $Revision: 0.0 $
\RCS $Date: 2006/01/20 04:38:13 $
\RCS $Id: tables.tex,v 0.0 2006/01/20 04:38:13 john Exp $
% $Log: tables.tex,v $
% Revision 0.0  2006/01/20 04:38:13  john
% Initial version
%
%
    \subsection{{\market}, normalized increments}
        \label{\SETLABEL:TSA}

        The data in table~\ref{\SETLABEL:INC} is condensed from
        Section~\ref{\SETLABELREF:TSA}.

        \begin{small}
            \begin{table}[ht]
                \begin{center}
                    \caption[{\market}, normalized increments]
                        {{\market}, normalized increments.}
                    \begin{tabular}{|c|c|c|c|c|c|c|c|c|c|} \hline
                        \multicolumn{5}{|c|}{Normalized}                                                                                  & \multicolumn{5}{|c|}{Normalized Absolute Value}\\ \hline
                        Mean                & Standard              & rms                & \multicolumn{2}{|c|}{Least Squares}            & Mean                   & Standard                 & rms                & \multicolumn{2}{|c|}{Least Squares} \\ \cline{4-5}\cline{9-10}
                        \hspace{0.01in}     & deviation             & \hspace{0.01in}    & Constant                & Slope                & \hspace{0.01in}        & deviation                & \hspace{0.01in}    & Constant                   & Slope \\ \hline\hline
                        {\datafractionmean} & {\datafractionstddev} & {\datafractionrms} & {\datafractionconstant} & {\datafractionslope} & {\datafractionabsmean} & {\datafractionabsstddev} & {\datafractionrms} & {\datafractionabsconstant} & {\datafractionabsslope} \\ \hline
                    \end{tabular}
                    \label{\SETLABEL:INC}
                \end{center}
            \end{table}
        \end{small}

    \subsection{{\market}, Logarithmic Returns, in Bits}
        \label{\SETLABEL:LR}

        The data in table~\ref{\SETLABEL:RET} is condensed from
        Section~\ref{\SETLABELREF:FS}.

        \begin{small}
            \begin{table}[ht]
                \begin{center}
                    \caption[{\market}, Logarithmic Returns, in
                        Bits]{{\market}, Logarithmic Returns, in Bits.}
                    \begin{tabular}{|c|c|c|c|} \hline
                        \multicolumn{2}{|c|}{Calculated from Table~\ref{\SETLABEL:INC}} & \multicolumn{2}{|c|}{From program:}\\ \hline
                        Mean                    & Least squares                       & {\it tslsq}\/              & {\it tslogreturns}\/ \\ \hline\hline
                        {\datafractionmeanbits} & {\datafractionconstantbits} & {\datatslsqepbits} & {\logreturns} \\ \hline
                    \end{tabular}
                    \label{\SETLABEL:RET}
                \end{center}
            \end{table}
        \end{small}

    \subsection{{\market}, Shannon probabilities}
        \label{\SETLABEL:MAXSHANNON}

        The data in table~\ref{\SETLABEL:SHANNON} is condensed from
        sections~\ref{\SETLABELREF:FS}
        and~\ref{\SETLABELREF:MAXSHANNON}.

        \begin{small}
            \begin{table}[ht]
                \begin{center}
                    \caption[{\market}, Shannon
                        probabilities]{{\market}, Shannon
                        probabilities.}
                    \begin{tabular}{|c|c|c|c|} \hline
                        \multicolumn{3}{|c|}{Maximum} & \multicolumn{1}{|c|}{Operational}\\ \hline
                        Fraction of         & $\frac{\frac{\mbox{\scriptsize{mean}}}{\mbox{\scriptsize{rms}}} + 1}{2}$ & \multicolumn{2}{|c|}{From program:}\\ \cline{3-4}
                        positive increments & \hspace{0.01in}                                                          & {\it tsshannonmax}\/    & {\it tsshannon}\/ \\ \hline\hline
                        {\pmax}             & {\avgrms}                                                                & {\shannonmax}   & {\shannonlogreturns} \\ \hline
                    \end{tabular}
                    \label{\SETLABEL:SHANNON}
                \end{center}
            \end{table}
        \end{small}

    \subsection{{\market}, Logistic Analysis}
        \label{\SETLABEL:LAA}

        The data in table~\ref{\SETLABEL:LA} is condensed from
        Section~\ref{\SETLABELREF:LA}\footnote{Note that there are
        numerical stability issues with the methodology used to derive
        the constants---if the non-linear term, $b$, was greater than
        zero, it was set to zero to produce the graphs in
        Section~\ref{\SETLABELREF:LA}.}.

        \begin{small}
            \begin{table}[ht]
                \begin{center}
                    \caption[{\market}, Logistic Analysis.]
                        {{\market}, Logistic Analysis, $x_t = x_{t - 1}\left(a + b \cdot x_{t - 1}\right)$.}
                    \begin{tabular}{|c|c|} \hline
                        $a$ & $b$\\ \hline\hline
                        {\datafractionconstant} & {\datafractionslope}\\ \hline
                    \end{tabular}
                    \label{\SETLABEL:LA}
                \end{center}
            \end{table}
        \end{small}

    \subsection{{\market}, Hurst Coefficients and H  Parameters}
        \label{\SETLABEL:HCHP}

        The data in table~\ref{\SETLABEL:H} is condensed from
        Section~\ref{\SETLABELREF:H}.

        \begin{small}
            \begin{table}[ht]
                \begin{center}
                    \caption[{\market}, Hurst Coefficients and H
                        Parameters]{{\market}, Hurst Coefficients and
                        H Parameters.}
                    \begin{tabular}{|c|c|c|c|} \hline
                        \multicolumn{2}{|c|}{Hurst Coefficients} & \multicolumn{2}{|c|}{H Parameters}\\ \hline
                        Near term   & Far term    & Near term   & Far term \\ \hline\hline
                        {\thurstlow} & {\thurstall} & {\thcalclow} & {\thcalcall} \\ \hline
                    \end{tabular}
                    \label{\SETLABEL:H}
                \end{center}
            \end{table}
        \end{small}

        \begin{small}
            \begin{table}[ht]
                \begin{center}
                    \caption[{\market}, Hurst Coefficients and H
                        Parameters]{{\market}, Hurst Coefficients and
                        H Parameters, as a Derivative.}
                    \begin{tabular}{|c|c|c|c|} \hline
                        \multicolumn{2}{|c|}{Hurst Coefficients} & \multicolumn{2}{|c|}{H Parameters}\\ \hline
                        Near term    & Far term     & Near term    & Far term \\ \hline\hline
                        {\hurstlow} & {\hurstall} & {\hcalclow} & {\hcalcall} \\ \hline
                    \end{tabular}
                    \label{\SETLABEL:TH}
                \end{center}
            \end{table}
        \end{small}

    \subsection{{\market}, verification of the increments}
        \label{\SETLABEL:VI1}

        The data in table~\ref{\SETLABEL:COMP} is condensed from
        Section~\ref{\SETLABELREF:QVA}.

        \begin{small}
            \begin{table}[ht]
                \begin{center}
                    \caption[{\market}, verification of
                        the increments]{{\market}, verification the of
                        the increments, the mean, $\sigma$ is the
                        standard deviation from
                        table~\ref{\SETLABEL:INC},
                        {\datafractionstddev}, and $P$ is the maximum
                        Shannon probability from
                        table~\ref{\SETLABEL:SHANNON}, {\pmax}. In
                        principle, the values should equate.}
                    \begin{tabular}{|c|c|c|} \hline
                        Mean                & $\mbox{rms} (2P - 1)$ & $\frac{{\sigma}(2P - 1)}{2\sqrt{P(P - 1)}} $ \\ \hline\hline
                        {\datafractionmean} & {\rmsp}               & {\sigmap} \\ \hline
                    \end{tabular}
                    \label{\SETLABEL:COMP}
                \end{center}
            \end{table}
        \end{small}

    \subsection{{\market}, verification of the increments}
        \label{\SETLABEL:VI2}

        The data in table~\ref{\SETLABEL:ABS} is condensed from
        Section~\ref{\SETLABELREF:QVA}.

        \begin{small}
            \begin{table}[ht]
                \begin{center}
                    \caption[{\market}, verification of
                        the increments]{{\market}, verification the of
                        increments. In principle, the mean of the
                        absolute value of the increments and the root
                        mean square of the increments should
                        equate\footnote{The absolute value of the
                        normalized increments, when averaged, is
                        related to the root mean square of the
                        increments by a constant. If the normalized
                        increments are a fixed increment, the constant
                        is unity. If the normalized increments have a
                        Gaussian distribution, the constant is
                        $\approx 0.8$ depending on the accuracy of of
                        ``fit'' to a Gaussian distribution.}.}
                    \begin{tabular}{|c|c|} \hline
                        Mean of the               & rms \\
                        absolute value            & \hspace{0.01in} \\ \hline\hline
                        {\datafractionabsmean}    & {\datafractionrms} \\ \hline
                    \end{tabular}
                    \label{\SETLABEL:ABS}
                \end{center}
            \end{table}
        \end{small}

    \subsection{{\market}, $\chi^2$ values of the increments}
        \label{\SETLABEL:XSQ}

        The data in table~\ref{\SETLABEL:XSQT} is condensed from
        Section~\ref{\SETLABELREF:NH}.

        \begin{small}
            \begin{table}[ht]
                \begin{center}
                    \caption[{\market}, $\chi^2$ values of
                        the increments]{{\market}, $\chi^2$ values of
                        the increments. In principle, if the
                        distribution of the normalized increments is a
                        Gaussian distribution, the $\chi^2$ value will
                        be significantly less than the critical
                        value.}
                    \begin{tabular}{|c|c|} \hline
                        $\chi^2$      & Critical Value \\ \hline\hline
                        {\chisquared} & {\critical} \\ \hline
                    \end{tabular}
                    \label{\SETLABEL:XSQT}
                \end{center}
            \end{table}
        \end{small}

    \subsection{{\market}, time series data, empirical and simulated}
        \label{\SETLABEL:SIM}

        The data in table~\ref{\SETLABEL:ES} is condensed from
        Section~\ref{\SETLABELREF:TSUNFAIRBROWNIAN}.

        \begin{small}
            \begin{table}[ht]
                \begin{center}
                    \caption[{\market}, time series data, empirical
                        and simulated]{{\market}, time series data,
                        empirical and simulated, analysis of the
                        normalized increments.}
                    \begin{tabular}{|c|c|c|c|} \hline
                        \multicolumn{2}{|c|}{Empirical} & \multicolumn{2}{|c|}{Simulated}\\ \hline
                        Mean                & Standard              & Mean               & Standard \\
                        \hspace{0.01in}     & deviation             & \hspace{0.01in}    & deviation \\ \hline\hline
                        {\datafractionmean} & {\datafractionstddev} & {\tsunfairbrownianfractionmean} & {\tsunfairbrownianfractionstddev} \\ \hline
                    \end{tabular}
                    \label{\SETLABEL:ES}
                \end{center}
            \end{table}
        \end{small}

    \subsection{{\market}, number of participating companies}
        \label{\SETLABEL:QNC}

        The data in table~\ref{\SETLABEL:NC} is condensed from
        Section~\ref{\SETLABELREF:QNC}.

        \begin{small}
            \begin{table}[ht]
                \begin{center}
                    \caption[{\market}, number of participating
                        companies] {{\market}, number of participating
                        companies.}
                    \begin{tabular}{|c|c|} \hline
                        Number & Shannon probability\\ \hline
                        {\ncompanies} & {\pncompanies}\\ \hline
                    \end{tabular}
                    \label{\SETLABEL:NC}
                \end{center}
            \end{table}
        \end{small}

    \subsection{{\market}, Shannon probability optimizations}
        \label{\SETLABEL:SPO}

        The data in table~\ref{\SETLABEL:SP} is condensed from
        Section~\ref{\SETLABELREF:QNC}.

        \begin{small}
            \begin{table}[ht]
                \begin{center}
                    \caption[{\market}, Shannon probability
                         optimizations] {{\market}, Shannon
                         probability optimization.}
                    \begin{tabular}{|c|c|} \hline
                        optimize capital growth & optimize market growth\\ \hline
                        {\avgrms} & {\pncompanies}\\ \hline
                    \end{tabular}
                    \label{\SETLABEL:SP}
                \end{center}
            \end{table}
        \end{small}

% Local Variables:
% TeX-parse-self: t
% TeX-auto-save: t
% TeX-master: "fractal.tex"
% End:


    \renewcommand{\market}{Simulated Industrial Market}
    \renewcommand{\directory}{../markets/tsmarket}
    \renewcommand{\datafractionmean}{0.008052}
\renewcommand{\datafractionmeanbits}{0.011570}
\renewcommand{\datafractionmeanq}{0.002684}
\renewcommand{\datafractionmeanbitsq}{0.003867}
\renewcommand{\datafractionstddev}{0.038579}
\renewcommand{\datafractionrms}{0.039311}
\renewcommand{\avgrms}{0.602414}
\renewcommand{\ncompanies}{5.210454}
\renewcommand{\pncompanies}{0.544866}
\renewcommand{\datafractionabsmean}{0.029745}
\renewcommand{\datafractionabsstddev}{0.025769}
\renewcommand{\datafractionconstant}{0.010041}
\renewcommand{\datafractionconstantbits}{0.014414}
\renewcommand{\datafractionconstantq}{0.003347}
\renewcommand{\datafractionconstantbitsq}{0.004821}
\renewcommand{\datafractionslope}{-0.000021}
\renewcommand{\datafractionabsconstant}{0.035145}
\renewcommand{\datafractionabsslope}{-0.000057}
\renewcommand{\hurstall}{0.659558}
\renewcommand{\hurstlow}{0.707509}
\renewcommand{\hurstlowtwo}{1.415018}
\renewcommand{\hurstlowhundred}{70.750900}
\renewcommand{\hcalcall}{0.184942}
\renewcommand{\hcalclow}{0.102042}
\renewcommand{\shannonmax}{0.604167}
\renewcommand{\twoponemax}{0.208334}
\renewcommand{\logreturns}{0.010456}
\renewcommand{\twologreturns}{1.007274}
\renewcommand{\twologreturnshundred}{0.727387}
\renewcommand{\oneoverlogreturns}{95.638868}
\renewcommand{\pmax}{0.602094}
\renewcommand{\twopminusone}{0.204188}
\renewcommand{\rmsp}{0.008027}
\renewcommand{\twopx}{0.208583}
\renewcommand{\sigmap}{0.008047}
\renewcommand{\tsunfairbrownianfractionmean}{0.007862}
\renewcommand{\tsunfairbrownianfractionstddev}{0.038619}
\renewcommand{\shannonlogreturns}{0.560125}
\renewcommand{\shannonlogreturnshundred}{56.012500}
\renewcommand{\twopone}{0.120250}
\renewcommand{\twoponehundred}{12.025000}
\renewcommand{\hundredtwoponehundred}{87.975000}
\renewcommand{\hundredshannonlogreturnshundred}{43.987500}
\renewcommand{\datatslsqepbits}{0.007623}
\renewcommand{\thurstall}{0.633980}
\renewcommand{\thurstlow}{0.710108}
\renewcommand{\thurstlowtwo}{1.420216}
\renewcommand{\thurstlowhundred}{71.010800}
\renewcommand{\thcalcall}{0.247886}
\renewcommand{\thcalclow}{0.171737}
\renewcommand{\chisquared}{2.862000}
\renewcommand{\critical}{42.557000}

    \renewcommand{\timescale}{month}
    \subidx{market}{\market}
    \idx{\market}

    \section{\market}

        \renewcommand{\SETLABEL}{\LABPRE:SIM}
        \renewcommand{\SETLABELQ}{\LABPRE:SIMQ}
        \label{\SETLABEL}
        \renewcommand{\SETLABELREF}{\LABPREREF:SIM}

        \subidx{tsmarket}{program}
        \subidx{programs}{tsmarket}
        For the analysis, the data was in the directory
        {\directory}\footnote{As a simulation model, the program
        {\it tsmarket}\/ was run to make a time series data file.
        The data is by {\timescale}s.}.

        The data in this section is presented in
        Section~\ref{\SETLABELREF}.

        %
% -----------------------------------------------------------------------------
%
% A license is hereby granted to reproduce this software source code and
% to create executable versions from this source code for personal,
% non-commercial use.  The copyright notice included with the software
% must be maintained in all copies produced.
%
% THIS PROGRAM IS PROVIDED "AS IS". THE AUTHOR PROVIDES NO WARRANTIES
% WHATSOEVER, EXPRESSED OR IMPLIED, INCLUDING WARRANTIES OF
% MERCHANTABILITY, TITLE, OR FITNESS FOR ANY PARTICULAR PURPOSE.  THE
% AUTHOR DOES NOT WARRANT THAT USE OF THIS PROGRAM DOES NOT INFRINGE THE
% INTELLECTUAL PROPERTY RIGHTS OF ANY THIRD PARTY IN ANY COUNTRY.
%
% Copyright (c) 1994-2006, John Conover, All Rights Reserved.
%
% Comments and/or bug reports should be addressed to:
%
%     john@email.johncon.com (John Conover)
%
% -----------------------------------------------------------------------------
%
% Revision: \RCSRevision \\
% Revision Time: \RCSTime UMT \\
% Revision Date: \RCSDate \\
% Revision Id: \RCSId \\
% Revision File: \RCSLog \\
\RCS $Revision: 0.0 $
\RCS $Date: 2006/01/20 04:38:13 $
\RCS $Id: tables.tex,v 0.0 2006/01/20 04:38:13 john Exp $
% $Log: tables.tex,v $
% Revision 0.0  2006/01/20 04:38:13  john
% Initial version
%
%
    \subsection{{\market}, normalized increments}
        \label{\SETLABEL:TSA}

        The data in table~\ref{\SETLABEL:INC} is condensed from
        Section~\ref{\SETLABELREF:TSA}.

        \begin{small}
            \begin{table}[ht]
                \begin{center}
                    \caption[{\market}, normalized increments]
                        {{\market}, normalized increments.}
                    \begin{tabular}{|c|c|c|c|c|c|c|c|c|c|} \hline
                        \multicolumn{5}{|c|}{Normalized}                                                                                  & \multicolumn{5}{|c|}{Normalized Absolute Value}\\ \hline
                        Mean                & Standard              & rms                & \multicolumn{2}{|c|}{Least Squares}            & Mean                   & Standard                 & rms                & \multicolumn{2}{|c|}{Least Squares} \\ \cline{4-5}\cline{9-10}
                        \hspace{0.01in}     & deviation             & \hspace{0.01in}    & Constant                & Slope                & \hspace{0.01in}        & deviation                & \hspace{0.01in}    & Constant                   & Slope \\ \hline\hline
                        {\datafractionmean} & {\datafractionstddev} & {\datafractionrms} & {\datafractionconstant} & {\datafractionslope} & {\datafractionabsmean} & {\datafractionabsstddev} & {\datafractionrms} & {\datafractionabsconstant} & {\datafractionabsslope} \\ \hline
                    \end{tabular}
                    \label{\SETLABEL:INC}
                \end{center}
            \end{table}
        \end{small}

    \subsection{{\market}, Logarithmic Returns, in Bits}
        \label{\SETLABEL:LR}

        The data in table~\ref{\SETLABEL:RET} is condensed from
        Section~\ref{\SETLABELREF:FS}.

        \begin{small}
            \begin{table}[ht]
                \begin{center}
                    \caption[{\market}, Logarithmic Returns, in
                        Bits]{{\market}, Logarithmic Returns, in Bits.}
                    \begin{tabular}{|c|c|c|c|} \hline
                        \multicolumn{2}{|c|}{Calculated from Table~\ref{\SETLABEL:INC}} & \multicolumn{2}{|c|}{From program:}\\ \hline
                        Mean                    & Least squares                       & {\it tslsq}\/              & {\it tslogreturns}\/ \\ \hline\hline
                        {\datafractionmeanbits} & {\datafractionconstantbits} & {\datatslsqepbits} & {\logreturns} \\ \hline
                    \end{tabular}
                    \label{\SETLABEL:RET}
                \end{center}
            \end{table}
        \end{small}

    \subsection{{\market}, Shannon probabilities}
        \label{\SETLABEL:MAXSHANNON}

        The data in table~\ref{\SETLABEL:SHANNON} is condensed from
        sections~\ref{\SETLABELREF:FS}
        and~\ref{\SETLABELREF:MAXSHANNON}.

        \begin{small}
            \begin{table}[ht]
                \begin{center}
                    \caption[{\market}, Shannon
                        probabilities]{{\market}, Shannon
                        probabilities.}
                    \begin{tabular}{|c|c|c|c|} \hline
                        \multicolumn{3}{|c|}{Maximum} & \multicolumn{1}{|c|}{Operational}\\ \hline
                        Fraction of         & $\frac{\frac{\mbox{\scriptsize{mean}}}{\mbox{\scriptsize{rms}}} + 1}{2}$ & \multicolumn{2}{|c|}{From program:}\\ \cline{3-4}
                        positive increments & \hspace{0.01in}                                                          & {\it tsshannonmax}\/    & {\it tsshannon}\/ \\ \hline\hline
                        {\pmax}             & {\avgrms}                                                                & {\shannonmax}   & {\shannonlogreturns} \\ \hline
                    \end{tabular}
                    \label{\SETLABEL:SHANNON}
                \end{center}
            \end{table}
        \end{small}

    \subsection{{\market}, Logistic Analysis}
        \label{\SETLABEL:LAA}

        The data in table~\ref{\SETLABEL:LA} is condensed from
        Section~\ref{\SETLABELREF:LA}\footnote{Note that there are
        numerical stability issues with the methodology used to derive
        the constants---if the non-linear term, $b$, was greater than
        zero, it was set to zero to produce the graphs in
        Section~\ref{\SETLABELREF:LA}.}.

        \begin{small}
            \begin{table}[ht]
                \begin{center}
                    \caption[{\market}, Logistic Analysis.]
                        {{\market}, Logistic Analysis, $x_t = x_{t - 1}\left(a + b \cdot x_{t - 1}\right)$.}
                    \begin{tabular}{|c|c|} \hline
                        $a$ & $b$\\ \hline\hline
                        {\datafractionconstant} & {\datafractionslope}\\ \hline
                    \end{tabular}
                    \label{\SETLABEL:LA}
                \end{center}
            \end{table}
        \end{small}

    \subsection{{\market}, Hurst Coefficients and H  Parameters}
        \label{\SETLABEL:HCHP}

        The data in table~\ref{\SETLABEL:H} is condensed from
        Section~\ref{\SETLABELREF:H}.

        \begin{small}
            \begin{table}[ht]
                \begin{center}
                    \caption[{\market}, Hurst Coefficients and H
                        Parameters]{{\market}, Hurst Coefficients and
                        H Parameters.}
                    \begin{tabular}{|c|c|c|c|} \hline
                        \multicolumn{2}{|c|}{Hurst Coefficients} & \multicolumn{2}{|c|}{H Parameters}\\ \hline
                        Near term   & Far term    & Near term   & Far term \\ \hline\hline
                        {\thurstlow} & {\thurstall} & {\thcalclow} & {\thcalcall} \\ \hline
                    \end{tabular}
                    \label{\SETLABEL:H}
                \end{center}
            \end{table}
        \end{small}

        \begin{small}
            \begin{table}[ht]
                \begin{center}
                    \caption[{\market}, Hurst Coefficients and H
                        Parameters]{{\market}, Hurst Coefficients and
                        H Parameters, as a Derivative.}
                    \begin{tabular}{|c|c|c|c|} \hline
                        \multicolumn{2}{|c|}{Hurst Coefficients} & \multicolumn{2}{|c|}{H Parameters}\\ \hline
                        Near term    & Far term     & Near term    & Far term \\ \hline\hline
                        {\hurstlow} & {\hurstall} & {\hcalclow} & {\hcalcall} \\ \hline
                    \end{tabular}
                    \label{\SETLABEL:TH}
                \end{center}
            \end{table}
        \end{small}

    \subsection{{\market}, verification of the increments}
        \label{\SETLABEL:VI1}

        The data in table~\ref{\SETLABEL:COMP} is condensed from
        Section~\ref{\SETLABELREF:QVA}.

        \begin{small}
            \begin{table}[ht]
                \begin{center}
                    \caption[{\market}, verification of
                        the increments]{{\market}, verification the of
                        the increments, the mean, $\sigma$ is the
                        standard deviation from
                        table~\ref{\SETLABEL:INC},
                        {\datafractionstddev}, and $P$ is the maximum
                        Shannon probability from
                        table~\ref{\SETLABEL:SHANNON}, {\pmax}. In
                        principle, the values should equate.}
                    \begin{tabular}{|c|c|c|} \hline
                        Mean                & $\mbox{rms} (2P - 1)$ & $\frac{{\sigma}(2P - 1)}{2\sqrt{P(P - 1)}} $ \\ \hline\hline
                        {\datafractionmean} & {\rmsp}               & {\sigmap} \\ \hline
                    \end{tabular}
                    \label{\SETLABEL:COMP}
                \end{center}
            \end{table}
        \end{small}

    \subsection{{\market}, verification of the increments}
        \label{\SETLABEL:VI2}

        The data in table~\ref{\SETLABEL:ABS} is condensed from
        Section~\ref{\SETLABELREF:QVA}.

        \begin{small}
            \begin{table}[ht]
                \begin{center}
                    \caption[{\market}, verification of
                        the increments]{{\market}, verification the of
                        increments. In principle, the mean of the
                        absolute value of the increments and the root
                        mean square of the increments should
                        equate\footnote{The absolute value of the
                        normalized increments, when averaged, is
                        related to the root mean square of the
                        increments by a constant. If the normalized
                        increments are a fixed increment, the constant
                        is unity. If the normalized increments have a
                        Gaussian distribution, the constant is
                        $\approx 0.8$ depending on the accuracy of of
                        ``fit'' to a Gaussian distribution.}.}
                    \begin{tabular}{|c|c|} \hline
                        Mean of the               & rms \\
                        absolute value            & \hspace{0.01in} \\ \hline\hline
                        {\datafractionabsmean}    & {\datafractionrms} \\ \hline
                    \end{tabular}
                    \label{\SETLABEL:ABS}
                \end{center}
            \end{table}
        \end{small}

    \subsection{{\market}, $\chi^2$ values of the increments}
        \label{\SETLABEL:XSQ}

        The data in table~\ref{\SETLABEL:XSQT} is condensed from
        Section~\ref{\SETLABELREF:NH}.

        \begin{small}
            \begin{table}[ht]
                \begin{center}
                    \caption[{\market}, $\chi^2$ values of
                        the increments]{{\market}, $\chi^2$ values of
                        the increments. In principle, if the
                        distribution of the normalized increments is a
                        Gaussian distribution, the $\chi^2$ value will
                        be significantly less than the critical
                        value.}
                    \begin{tabular}{|c|c|} \hline
                        $\chi^2$      & Critical Value \\ \hline\hline
                        {\chisquared} & {\critical} \\ \hline
                    \end{tabular}
                    \label{\SETLABEL:XSQT}
                \end{center}
            \end{table}
        \end{small}

    \subsection{{\market}, time series data, empirical and simulated}
        \label{\SETLABEL:SIM}

        The data in table~\ref{\SETLABEL:ES} is condensed from
        Section~\ref{\SETLABELREF:TSUNFAIRBROWNIAN}.

        \begin{small}
            \begin{table}[ht]
                \begin{center}
                    \caption[{\market}, time series data, empirical
                        and simulated]{{\market}, time series data,
                        empirical and simulated, analysis of the
                        normalized increments.}
                    \begin{tabular}{|c|c|c|c|} \hline
                        \multicolumn{2}{|c|}{Empirical} & \multicolumn{2}{|c|}{Simulated}\\ \hline
                        Mean                & Standard              & Mean               & Standard \\
                        \hspace{0.01in}     & deviation             & \hspace{0.01in}    & deviation \\ \hline\hline
                        {\datafractionmean} & {\datafractionstddev} & {\tsunfairbrownianfractionmean} & {\tsunfairbrownianfractionstddev} \\ \hline
                    \end{tabular}
                    \label{\SETLABEL:ES}
                \end{center}
            \end{table}
        \end{small}

    \subsection{{\market}, number of participating companies}
        \label{\SETLABEL:QNC}

        The data in table~\ref{\SETLABEL:NC} is condensed from
        Section~\ref{\SETLABELREF:QNC}.

        \begin{small}
            \begin{table}[ht]
                \begin{center}
                    \caption[{\market}, number of participating
                        companies] {{\market}, number of participating
                        companies.}
                    \begin{tabular}{|c|c|} \hline
                        Number & Shannon probability\\ \hline
                        {\ncompanies} & {\pncompanies}\\ \hline
                    \end{tabular}
                    \label{\SETLABEL:NC}
                \end{center}
            \end{table}
        \end{small}

    \subsection{{\market}, Shannon probability optimizations}
        \label{\SETLABEL:SPO}

        The data in table~\ref{\SETLABEL:SP} is condensed from
        Section~\ref{\SETLABELREF:QNC}.

        \begin{small}
            \begin{table}[ht]
                \begin{center}
                    \caption[{\market}, Shannon probability
                         optimizations] {{\market}, Shannon
                         probability optimization.}
                    \begin{tabular}{|c|c|} \hline
                        optimize capital growth & optimize market growth\\ \hline
                        {\avgrms} & {\pncompanies}\\ \hline
                    \end{tabular}
                    \label{\SETLABEL:SP}
                \end{center}
            \end{table}
        \end{small}

% Local Variables:
% TeX-parse-self: t
% TeX-auto-save: t
% TeX-master: "fractal.tex"
% End:


    \renewcommand{\market}{Discreet Logistic Function}
    \renewcommand{\directory}{../markets/tsdlogistic}
    \renewcommand{\datafractionmean}{0.008052}
\renewcommand{\datafractionmeanbits}{0.011570}
\renewcommand{\datafractionmeanq}{0.002684}
\renewcommand{\datafractionmeanbitsq}{0.003867}
\renewcommand{\datafractionstddev}{0.038579}
\renewcommand{\datafractionrms}{0.039311}
\renewcommand{\avgrms}{0.602414}
\renewcommand{\ncompanies}{5.210454}
\renewcommand{\pncompanies}{0.544866}
\renewcommand{\datafractionabsmean}{0.029745}
\renewcommand{\datafractionabsstddev}{0.025769}
\renewcommand{\datafractionconstant}{0.010041}
\renewcommand{\datafractionconstantbits}{0.014414}
\renewcommand{\datafractionconstantq}{0.003347}
\renewcommand{\datafractionconstantbitsq}{0.004821}
\renewcommand{\datafractionslope}{-0.000021}
\renewcommand{\datafractionabsconstant}{0.035145}
\renewcommand{\datafractionabsslope}{-0.000057}
\renewcommand{\hurstall}{0.659558}
\renewcommand{\hurstlow}{0.707509}
\renewcommand{\hurstlowtwo}{1.415018}
\renewcommand{\hurstlowhundred}{70.750900}
\renewcommand{\hcalcall}{0.184942}
\renewcommand{\hcalclow}{0.102042}
\renewcommand{\shannonmax}{0.604167}
\renewcommand{\twoponemax}{0.208334}
\renewcommand{\logreturns}{0.010456}
\renewcommand{\twologreturns}{1.007274}
\renewcommand{\twologreturnshundred}{0.727387}
\renewcommand{\oneoverlogreturns}{95.638868}
\renewcommand{\pmax}{0.602094}
\renewcommand{\twopminusone}{0.204188}
\renewcommand{\rmsp}{0.008027}
\renewcommand{\twopx}{0.208583}
\renewcommand{\sigmap}{0.008047}
\renewcommand{\tsunfairbrownianfractionmean}{0.007862}
\renewcommand{\tsunfairbrownianfractionstddev}{0.038619}
\renewcommand{\shannonlogreturns}{0.560125}
\renewcommand{\shannonlogreturnshundred}{56.012500}
\renewcommand{\twopone}{0.120250}
\renewcommand{\twoponehundred}{12.025000}
\renewcommand{\hundredtwoponehundred}{87.975000}
\renewcommand{\hundredshannonlogreturnshundred}{43.987500}
\renewcommand{\datatslsqepbits}{0.007623}
\renewcommand{\thurstall}{0.633980}
\renewcommand{\thurstlow}{0.710108}
\renewcommand{\thurstlowtwo}{1.420216}
\renewcommand{\thurstlowhundred}{71.010800}
\renewcommand{\thcalcall}{0.247886}
\renewcommand{\thcalclow}{0.171737}
\renewcommand{\chisquared}{2.862000}
\renewcommand{\critical}{42.557000}

    \renewcommand{\timescale}{month}
    \subidx{market}{\market}
    \idx{\market}

    \section{\market}

        \renewcommand{\SETLABEL}{\LABPRE:DLF}
        \renewcommand{\SETLABELQ}{\LABPRE:DLFQ}
        \label{\SETLABEL}
        \renewcommand{\SETLABELREF}{\LABPREREF:DLF}

        \subidx{tsdlogistic}{program}
        \subidx{programs}{tsdlogistic}
        For the analysis, the data was in the directory
        {\directory}\footnote{As a simulation model, the program
        {\it tsdlogistic}\/ was run to make a time series data file.
        The data is by {\timescale}s.}.

        The data in this section is presented in
        Section~\ref{\SETLABELREF}.

        %
% -----------------------------------------------------------------------------
%
% A license is hereby granted to reproduce this software source code and
% to create executable versions from this source code for personal,
% non-commercial use.  The copyright notice included with the software
% must be maintained in all copies produced.
%
% THIS PROGRAM IS PROVIDED "AS IS". THE AUTHOR PROVIDES NO WARRANTIES
% WHATSOEVER, EXPRESSED OR IMPLIED, INCLUDING WARRANTIES OF
% MERCHANTABILITY, TITLE, OR FITNESS FOR ANY PARTICULAR PURPOSE.  THE
% AUTHOR DOES NOT WARRANT THAT USE OF THIS PROGRAM DOES NOT INFRINGE THE
% INTELLECTUAL PROPERTY RIGHTS OF ANY THIRD PARTY IN ANY COUNTRY.
%
% Copyright (c) 1994-2006, John Conover, All Rights Reserved.
%
% Comments and/or bug reports should be addressed to:
%
%     john@email.johncon.com (John Conover)
%
% -----------------------------------------------------------------------------
%
% Revision: \RCSRevision \\
% Revision Time: \RCSTime UMT \\
% Revision Date: \RCSDate \\
% Revision Id: \RCSId \\
% Revision File: \RCSLog \\
\RCS $Revision: 0.0 $
\RCS $Date: 2006/01/20 04:38:13 $
\RCS $Id: tables.tex,v 0.0 2006/01/20 04:38:13 john Exp $
% $Log: tables.tex,v $
% Revision 0.0  2006/01/20 04:38:13  john
% Initial version
%
%
    \subsection{{\market}, normalized increments}
        \label{\SETLABEL:TSA}

        The data in table~\ref{\SETLABEL:INC} is condensed from
        Section~\ref{\SETLABELREF:TSA}.

        \begin{small}
            \begin{table}[ht]
                \begin{center}
                    \caption[{\market}, normalized increments]
                        {{\market}, normalized increments.}
                    \begin{tabular}{|c|c|c|c|c|c|c|c|c|c|} \hline
                        \multicolumn{5}{|c|}{Normalized}                                                                                  & \multicolumn{5}{|c|}{Normalized Absolute Value}\\ \hline
                        Mean                & Standard              & rms                & \multicolumn{2}{|c|}{Least Squares}            & Mean                   & Standard                 & rms                & \multicolumn{2}{|c|}{Least Squares} \\ \cline{4-5}\cline{9-10}
                        \hspace{0.01in}     & deviation             & \hspace{0.01in}    & Constant                & Slope                & \hspace{0.01in}        & deviation                & \hspace{0.01in}    & Constant                   & Slope \\ \hline\hline
                        {\datafractionmean} & {\datafractionstddev} & {\datafractionrms} & {\datafractionconstant} & {\datafractionslope} & {\datafractionabsmean} & {\datafractionabsstddev} & {\datafractionrms} & {\datafractionabsconstant} & {\datafractionabsslope} \\ \hline
                    \end{tabular}
                    \label{\SETLABEL:INC}
                \end{center}
            \end{table}
        \end{small}

    \subsection{{\market}, Logarithmic Returns, in Bits}
        \label{\SETLABEL:LR}

        The data in table~\ref{\SETLABEL:RET} is condensed from
        Section~\ref{\SETLABELREF:FS}.

        \begin{small}
            \begin{table}[ht]
                \begin{center}
                    \caption[{\market}, Logarithmic Returns, in
                        Bits]{{\market}, Logarithmic Returns, in Bits.}
                    \begin{tabular}{|c|c|c|c|} \hline
                        \multicolumn{2}{|c|}{Calculated from Table~\ref{\SETLABEL:INC}} & \multicolumn{2}{|c|}{From program:}\\ \hline
                        Mean                    & Least squares                       & {\it tslsq}\/              & {\it tslogreturns}\/ \\ \hline\hline
                        {\datafractionmeanbits} & {\datafractionconstantbits} & {\datatslsqepbits} & {\logreturns} \\ \hline
                    \end{tabular}
                    \label{\SETLABEL:RET}
                \end{center}
            \end{table}
        \end{small}

    \subsection{{\market}, Shannon probabilities}
        \label{\SETLABEL:MAXSHANNON}

        The data in table~\ref{\SETLABEL:SHANNON} is condensed from
        sections~\ref{\SETLABELREF:FS}
        and~\ref{\SETLABELREF:MAXSHANNON}.

        \begin{small}
            \begin{table}[ht]
                \begin{center}
                    \caption[{\market}, Shannon
                        probabilities]{{\market}, Shannon
                        probabilities.}
                    \begin{tabular}{|c|c|c|c|} \hline
                        \multicolumn{3}{|c|}{Maximum} & \multicolumn{1}{|c|}{Operational}\\ \hline
                        Fraction of         & $\frac{\frac{\mbox{\scriptsize{mean}}}{\mbox{\scriptsize{rms}}} + 1}{2}$ & \multicolumn{2}{|c|}{From program:}\\ \cline{3-4}
                        positive increments & \hspace{0.01in}                                                          & {\it tsshannonmax}\/    & {\it tsshannon}\/ \\ \hline\hline
                        {\pmax}             & {\avgrms}                                                                & {\shannonmax}   & {\shannonlogreturns} \\ \hline
                    \end{tabular}
                    \label{\SETLABEL:SHANNON}
                \end{center}
            \end{table}
        \end{small}

    \subsection{{\market}, Logistic Analysis}
        \label{\SETLABEL:LAA}

        The data in table~\ref{\SETLABEL:LA} is condensed from
        Section~\ref{\SETLABELREF:LA}\footnote{Note that there are
        numerical stability issues with the methodology used to derive
        the constants---if the non-linear term, $b$, was greater than
        zero, it was set to zero to produce the graphs in
        Section~\ref{\SETLABELREF:LA}.}.

        \begin{small}
            \begin{table}[ht]
                \begin{center}
                    \caption[{\market}, Logistic Analysis.]
                        {{\market}, Logistic Analysis, $x_t = x_{t - 1}\left(a + b \cdot x_{t - 1}\right)$.}
                    \begin{tabular}{|c|c|} \hline
                        $a$ & $b$\\ \hline\hline
                        {\datafractionconstant} & {\datafractionslope}\\ \hline
                    \end{tabular}
                    \label{\SETLABEL:LA}
                \end{center}
            \end{table}
        \end{small}

    \subsection{{\market}, Hurst Coefficients and H  Parameters}
        \label{\SETLABEL:HCHP}

        The data in table~\ref{\SETLABEL:H} is condensed from
        Section~\ref{\SETLABELREF:H}.

        \begin{small}
            \begin{table}[ht]
                \begin{center}
                    \caption[{\market}, Hurst Coefficients and H
                        Parameters]{{\market}, Hurst Coefficients and
                        H Parameters.}
                    \begin{tabular}{|c|c|c|c|} \hline
                        \multicolumn{2}{|c|}{Hurst Coefficients} & \multicolumn{2}{|c|}{H Parameters}\\ \hline
                        Near term   & Far term    & Near term   & Far term \\ \hline\hline
                        {\thurstlow} & {\thurstall} & {\thcalclow} & {\thcalcall} \\ \hline
                    \end{tabular}
                    \label{\SETLABEL:H}
                \end{center}
            \end{table}
        \end{small}

        \begin{small}
            \begin{table}[ht]
                \begin{center}
                    \caption[{\market}, Hurst Coefficients and H
                        Parameters]{{\market}, Hurst Coefficients and
                        H Parameters, as a Derivative.}
                    \begin{tabular}{|c|c|c|c|} \hline
                        \multicolumn{2}{|c|}{Hurst Coefficients} & \multicolumn{2}{|c|}{H Parameters}\\ \hline
                        Near term    & Far term     & Near term    & Far term \\ \hline\hline
                        {\hurstlow} & {\hurstall} & {\hcalclow} & {\hcalcall} \\ \hline
                    \end{tabular}
                    \label{\SETLABEL:TH}
                \end{center}
            \end{table}
        \end{small}

    \subsection{{\market}, verification of the increments}
        \label{\SETLABEL:VI1}

        The data in table~\ref{\SETLABEL:COMP} is condensed from
        Section~\ref{\SETLABELREF:QVA}.

        \begin{small}
            \begin{table}[ht]
                \begin{center}
                    \caption[{\market}, verification of
                        the increments]{{\market}, verification the of
                        the increments, the mean, $\sigma$ is the
                        standard deviation from
                        table~\ref{\SETLABEL:INC},
                        {\datafractionstddev}, and $P$ is the maximum
                        Shannon probability from
                        table~\ref{\SETLABEL:SHANNON}, {\pmax}. In
                        principle, the values should equate.}
                    \begin{tabular}{|c|c|c|} \hline
                        Mean                & $\mbox{rms} (2P - 1)$ & $\frac{{\sigma}(2P - 1)}{2\sqrt{P(P - 1)}} $ \\ \hline\hline
                        {\datafractionmean} & {\rmsp}               & {\sigmap} \\ \hline
                    \end{tabular}
                    \label{\SETLABEL:COMP}
                \end{center}
            \end{table}
        \end{small}

    \subsection{{\market}, verification of the increments}
        \label{\SETLABEL:VI2}

        The data in table~\ref{\SETLABEL:ABS} is condensed from
        Section~\ref{\SETLABELREF:QVA}.

        \begin{small}
            \begin{table}[ht]
                \begin{center}
                    \caption[{\market}, verification of
                        the increments]{{\market}, verification the of
                        increments. In principle, the mean of the
                        absolute value of the increments and the root
                        mean square of the increments should
                        equate\footnote{The absolute value of the
                        normalized increments, when averaged, is
                        related to the root mean square of the
                        increments by a constant. If the normalized
                        increments are a fixed increment, the constant
                        is unity. If the normalized increments have a
                        Gaussian distribution, the constant is
                        $\approx 0.8$ depending on the accuracy of of
                        ``fit'' to a Gaussian distribution.}.}
                    \begin{tabular}{|c|c|} \hline
                        Mean of the               & rms \\
                        absolute value            & \hspace{0.01in} \\ \hline\hline
                        {\datafractionabsmean}    & {\datafractionrms} \\ \hline
                    \end{tabular}
                    \label{\SETLABEL:ABS}
                \end{center}
            \end{table}
        \end{small}

    \subsection{{\market}, $\chi^2$ values of the increments}
        \label{\SETLABEL:XSQ}

        The data in table~\ref{\SETLABEL:XSQT} is condensed from
        Section~\ref{\SETLABELREF:NH}.

        \begin{small}
            \begin{table}[ht]
                \begin{center}
                    \caption[{\market}, $\chi^2$ values of
                        the increments]{{\market}, $\chi^2$ values of
                        the increments. In principle, if the
                        distribution of the normalized increments is a
                        Gaussian distribution, the $\chi^2$ value will
                        be significantly less than the critical
                        value.}
                    \begin{tabular}{|c|c|} \hline
                        $\chi^2$      & Critical Value \\ \hline\hline
                        {\chisquared} & {\critical} \\ \hline
                    \end{tabular}
                    \label{\SETLABEL:XSQT}
                \end{center}
            \end{table}
        \end{small}

    \subsection{{\market}, time series data, empirical and simulated}
        \label{\SETLABEL:SIM}

        The data in table~\ref{\SETLABEL:ES} is condensed from
        Section~\ref{\SETLABELREF:TSUNFAIRBROWNIAN}.

        \begin{small}
            \begin{table}[ht]
                \begin{center}
                    \caption[{\market}, time series data, empirical
                        and simulated]{{\market}, time series data,
                        empirical and simulated, analysis of the
                        normalized increments.}
                    \begin{tabular}{|c|c|c|c|} \hline
                        \multicolumn{2}{|c|}{Empirical} & \multicolumn{2}{|c|}{Simulated}\\ \hline
                        Mean                & Standard              & Mean               & Standard \\
                        \hspace{0.01in}     & deviation             & \hspace{0.01in}    & deviation \\ \hline\hline
                        {\datafractionmean} & {\datafractionstddev} & {\tsunfairbrownianfractionmean} & {\tsunfairbrownianfractionstddev} \\ \hline
                    \end{tabular}
                    \label{\SETLABEL:ES}
                \end{center}
            \end{table}
        \end{small}

    \subsection{{\market}, number of participating companies}
        \label{\SETLABEL:QNC}

        The data in table~\ref{\SETLABEL:NC} is condensed from
        Section~\ref{\SETLABELREF:QNC}.

        \begin{small}
            \begin{table}[ht]
                \begin{center}
                    \caption[{\market}, number of participating
                        companies] {{\market}, number of participating
                        companies.}
                    \begin{tabular}{|c|c|} \hline
                        Number & Shannon probability\\ \hline
                        {\ncompanies} & {\pncompanies}\\ \hline
                    \end{tabular}
                    \label{\SETLABEL:NC}
                \end{center}
            \end{table}
        \end{small}

    \subsection{{\market}, Shannon probability optimizations}
        \label{\SETLABEL:SPO}

        The data in table~\ref{\SETLABEL:SP} is condensed from
        Section~\ref{\SETLABELREF:QNC}.

        \begin{small}
            \begin{table}[ht]
                \begin{center}
                    \caption[{\market}, Shannon probability
                         optimizations] {{\market}, Shannon
                         probability optimization.}
                    \begin{tabular}{|c|c|} \hline
                        optimize capital growth & optimize market growth\\ \hline
                        {\avgrms} & {\pncompanies}\\ \hline
                    \end{tabular}
                    \label{\SETLABEL:SP}
                \end{center}
            \end{table}
        \end{small}

% Local Variables:
% TeX-parse-self: t
% TeX-auto-save: t
% TeX-master: "fractal.tex"
% End:


    \renewcommand{\market}{Simulated Equity Market Index}
    \renewcommand{\directory}{../markets/tsgaussian.tsmath.tsmath.tsunfraction}
    \renewcommand{\datafractionmean}{0.008052}
\renewcommand{\datafractionmeanbits}{0.011570}
\renewcommand{\datafractionmeanq}{0.002684}
\renewcommand{\datafractionmeanbitsq}{0.003867}
\renewcommand{\datafractionstddev}{0.038579}
\renewcommand{\datafractionrms}{0.039311}
\renewcommand{\avgrms}{0.602414}
\renewcommand{\ncompanies}{5.210454}
\renewcommand{\pncompanies}{0.544866}
\renewcommand{\datafractionabsmean}{0.029745}
\renewcommand{\datafractionabsstddev}{0.025769}
\renewcommand{\datafractionconstant}{0.010041}
\renewcommand{\datafractionconstantbits}{0.014414}
\renewcommand{\datafractionconstantq}{0.003347}
\renewcommand{\datafractionconstantbitsq}{0.004821}
\renewcommand{\datafractionslope}{-0.000021}
\renewcommand{\datafractionabsconstant}{0.035145}
\renewcommand{\datafractionabsslope}{-0.000057}
\renewcommand{\hurstall}{0.659558}
\renewcommand{\hurstlow}{0.707509}
\renewcommand{\hurstlowtwo}{1.415018}
\renewcommand{\hurstlowhundred}{70.750900}
\renewcommand{\hcalcall}{0.184942}
\renewcommand{\hcalclow}{0.102042}
\renewcommand{\shannonmax}{0.604167}
\renewcommand{\twoponemax}{0.208334}
\renewcommand{\logreturns}{0.010456}
\renewcommand{\twologreturns}{1.007274}
\renewcommand{\twologreturnshundred}{0.727387}
\renewcommand{\oneoverlogreturns}{95.638868}
\renewcommand{\pmax}{0.602094}
\renewcommand{\twopminusone}{0.204188}
\renewcommand{\rmsp}{0.008027}
\renewcommand{\twopx}{0.208583}
\renewcommand{\sigmap}{0.008047}
\renewcommand{\tsunfairbrownianfractionmean}{0.007862}
\renewcommand{\tsunfairbrownianfractionstddev}{0.038619}
\renewcommand{\shannonlogreturns}{0.560125}
\renewcommand{\shannonlogreturnshundred}{56.012500}
\renewcommand{\twopone}{0.120250}
\renewcommand{\twoponehundred}{12.025000}
\renewcommand{\hundredtwoponehundred}{87.975000}
\renewcommand{\hundredshannonlogreturnshundred}{43.987500}
\renewcommand{\datatslsqepbits}{0.007623}
\renewcommand{\thurstall}{0.633980}
\renewcommand{\thurstlow}{0.710108}
\renewcommand{\thurstlowtwo}{1.420216}
\renewcommand{\thurstlowhundred}{71.010800}
\renewcommand{\thcalcall}{0.247886}
\renewcommand{\thcalclow}{0.171737}
\renewcommand{\chisquared}{2.862000}
\renewcommand{\critical}{42.557000}

    \renewcommand{\timescale}{month}
    \subidx{market}{\market}
    \idx{\market}

    \section{\market}

        \renewcommand{\SETLABEL}{\LABPRE:SEMIX}
        \renewcommand{\SETLABELQ}{\LABPRE:SEMIXQ}
        \label{\SETLABEL}
        \renewcommand{\SETLABELREF}{\LABPREREF:SEMIX}

        \subidx{tsunfraction}{program}
        \subidx{programs}{tsunfraction}
        \subidx{tsgaussian}{program}
        \subidx{programs}{tsgaussian}
        \subidx{tsmath}{program}
        \subidx{programs}{tsmath}
        For the analysis, the data was in the directory
        {\directory}\footnote{As a simulation model, the programs {\it
        tsgaussian}\/, {\it tsmath}\/, and {\it tsunfraction}\/ were
        run to make a time series data file.  The data is by
        {\timescale}s.}.

        The data in this section is presented in
        Section~\ref{\SETLABELREF}.

        %
% -----------------------------------------------------------------------------
%
% A license is hereby granted to reproduce this software source code and
% to create executable versions from this source code for personal,
% non-commercial use.  The copyright notice included with the software
% must be maintained in all copies produced.
%
% THIS PROGRAM IS PROVIDED "AS IS". THE AUTHOR PROVIDES NO WARRANTIES
% WHATSOEVER, EXPRESSED OR IMPLIED, INCLUDING WARRANTIES OF
% MERCHANTABILITY, TITLE, OR FITNESS FOR ANY PARTICULAR PURPOSE.  THE
% AUTHOR DOES NOT WARRANT THAT USE OF THIS PROGRAM DOES NOT INFRINGE THE
% INTELLECTUAL PROPERTY RIGHTS OF ANY THIRD PARTY IN ANY COUNTRY.
%
% Copyright (c) 1994-2006, John Conover, All Rights Reserved.
%
% Comments and/or bug reports should be addressed to:
%
%     john@email.johncon.com (John Conover)
%
% -----------------------------------------------------------------------------
%
% Revision: \RCSRevision \\
% Revision Time: \RCSTime UMT \\
% Revision Date: \RCSDate \\
% Revision Id: \RCSId \\
% Revision File: \RCSLog \\
\RCS $Revision: 0.0 $
\RCS $Date: 2006/01/20 04:38:13 $
\RCS $Id: tables.tex,v 0.0 2006/01/20 04:38:13 john Exp $
% $Log: tables.tex,v $
% Revision 0.0  2006/01/20 04:38:13  john
% Initial version
%
%
    \subsection{{\market}, normalized increments}
        \label{\SETLABEL:TSA}

        The data in table~\ref{\SETLABEL:INC} is condensed from
        Section~\ref{\SETLABELREF:TSA}.

        \begin{small}
            \begin{table}[ht]
                \begin{center}
                    \caption[{\market}, normalized increments]
                        {{\market}, normalized increments.}
                    \begin{tabular}{|c|c|c|c|c|c|c|c|c|c|} \hline
                        \multicolumn{5}{|c|}{Normalized}                                                                                  & \multicolumn{5}{|c|}{Normalized Absolute Value}\\ \hline
                        Mean                & Standard              & rms                & \multicolumn{2}{|c|}{Least Squares}            & Mean                   & Standard                 & rms                & \multicolumn{2}{|c|}{Least Squares} \\ \cline{4-5}\cline{9-10}
                        \hspace{0.01in}     & deviation             & \hspace{0.01in}    & Constant                & Slope                & \hspace{0.01in}        & deviation                & \hspace{0.01in}    & Constant                   & Slope \\ \hline\hline
                        {\datafractionmean} & {\datafractionstddev} & {\datafractionrms} & {\datafractionconstant} & {\datafractionslope} & {\datafractionabsmean} & {\datafractionabsstddev} & {\datafractionrms} & {\datafractionabsconstant} & {\datafractionabsslope} \\ \hline
                    \end{tabular}
                    \label{\SETLABEL:INC}
                \end{center}
            \end{table}
        \end{small}

    \subsection{{\market}, Logarithmic Returns, in Bits}
        \label{\SETLABEL:LR}

        The data in table~\ref{\SETLABEL:RET} is condensed from
        Section~\ref{\SETLABELREF:FS}.

        \begin{small}
            \begin{table}[ht]
                \begin{center}
                    \caption[{\market}, Logarithmic Returns, in
                        Bits]{{\market}, Logarithmic Returns, in Bits.}
                    \begin{tabular}{|c|c|c|c|} \hline
                        \multicolumn{2}{|c|}{Calculated from Table~\ref{\SETLABEL:INC}} & \multicolumn{2}{|c|}{From program:}\\ \hline
                        Mean                    & Least squares                       & {\it tslsq}\/              & {\it tslogreturns}\/ \\ \hline\hline
                        {\datafractionmeanbits} & {\datafractionconstantbits} & {\datatslsqepbits} & {\logreturns} \\ \hline
                    \end{tabular}
                    \label{\SETLABEL:RET}
                \end{center}
            \end{table}
        \end{small}

    \subsection{{\market}, Shannon probabilities}
        \label{\SETLABEL:MAXSHANNON}

        The data in table~\ref{\SETLABEL:SHANNON} is condensed from
        sections~\ref{\SETLABELREF:FS}
        and~\ref{\SETLABELREF:MAXSHANNON}.

        \begin{small}
            \begin{table}[ht]
                \begin{center}
                    \caption[{\market}, Shannon
                        probabilities]{{\market}, Shannon
                        probabilities.}
                    \begin{tabular}{|c|c|c|c|} \hline
                        \multicolumn{3}{|c|}{Maximum} & \multicolumn{1}{|c|}{Operational}\\ \hline
                        Fraction of         & $\frac{\frac{\mbox{\scriptsize{mean}}}{\mbox{\scriptsize{rms}}} + 1}{2}$ & \multicolumn{2}{|c|}{From program:}\\ \cline{3-4}
                        positive increments & \hspace{0.01in}                                                          & {\it tsshannonmax}\/    & {\it tsshannon}\/ \\ \hline\hline
                        {\pmax}             & {\avgrms}                                                                & {\shannonmax}   & {\shannonlogreturns} \\ \hline
                    \end{tabular}
                    \label{\SETLABEL:SHANNON}
                \end{center}
            \end{table}
        \end{small}

    \subsection{{\market}, Logistic Analysis}
        \label{\SETLABEL:LAA}

        The data in table~\ref{\SETLABEL:LA} is condensed from
        Section~\ref{\SETLABELREF:LA}\footnote{Note that there are
        numerical stability issues with the methodology used to derive
        the constants---if the non-linear term, $b$, was greater than
        zero, it was set to zero to produce the graphs in
        Section~\ref{\SETLABELREF:LA}.}.

        \begin{small}
            \begin{table}[ht]
                \begin{center}
                    \caption[{\market}, Logistic Analysis.]
                        {{\market}, Logistic Analysis, $x_t = x_{t - 1}\left(a + b \cdot x_{t - 1}\right)$.}
                    \begin{tabular}{|c|c|} \hline
                        $a$ & $b$\\ \hline\hline
                        {\datafractionconstant} & {\datafractionslope}\\ \hline
                    \end{tabular}
                    \label{\SETLABEL:LA}
                \end{center}
            \end{table}
        \end{small}

    \subsection{{\market}, Hurst Coefficients and H  Parameters}
        \label{\SETLABEL:HCHP}

        The data in table~\ref{\SETLABEL:H} is condensed from
        Section~\ref{\SETLABELREF:H}.

        \begin{small}
            \begin{table}[ht]
                \begin{center}
                    \caption[{\market}, Hurst Coefficients and H
                        Parameters]{{\market}, Hurst Coefficients and
                        H Parameters.}
                    \begin{tabular}{|c|c|c|c|} \hline
                        \multicolumn{2}{|c|}{Hurst Coefficients} & \multicolumn{2}{|c|}{H Parameters}\\ \hline
                        Near term   & Far term    & Near term   & Far term \\ \hline\hline
                        {\thurstlow} & {\thurstall} & {\thcalclow} & {\thcalcall} \\ \hline
                    \end{tabular}
                    \label{\SETLABEL:H}
                \end{center}
            \end{table}
        \end{small}

        \begin{small}
            \begin{table}[ht]
                \begin{center}
                    \caption[{\market}, Hurst Coefficients and H
                        Parameters]{{\market}, Hurst Coefficients and
                        H Parameters, as a Derivative.}
                    \begin{tabular}{|c|c|c|c|} \hline
                        \multicolumn{2}{|c|}{Hurst Coefficients} & \multicolumn{2}{|c|}{H Parameters}\\ \hline
                        Near term    & Far term     & Near term    & Far term \\ \hline\hline
                        {\hurstlow} & {\hurstall} & {\hcalclow} & {\hcalcall} \\ \hline
                    \end{tabular}
                    \label{\SETLABEL:TH}
                \end{center}
            \end{table}
        \end{small}

    \subsection{{\market}, verification of the increments}
        \label{\SETLABEL:VI1}

        The data in table~\ref{\SETLABEL:COMP} is condensed from
        Section~\ref{\SETLABELREF:QVA}.

        \begin{small}
            \begin{table}[ht]
                \begin{center}
                    \caption[{\market}, verification of
                        the increments]{{\market}, verification the of
                        the increments, the mean, $\sigma$ is the
                        standard deviation from
                        table~\ref{\SETLABEL:INC},
                        {\datafractionstddev}, and $P$ is the maximum
                        Shannon probability from
                        table~\ref{\SETLABEL:SHANNON}, {\pmax}. In
                        principle, the values should equate.}
                    \begin{tabular}{|c|c|c|} \hline
                        Mean                & $\mbox{rms} (2P - 1)$ & $\frac{{\sigma}(2P - 1)}{2\sqrt{P(P - 1)}} $ \\ \hline\hline
                        {\datafractionmean} & {\rmsp}               & {\sigmap} \\ \hline
                    \end{tabular}
                    \label{\SETLABEL:COMP}
                \end{center}
            \end{table}
        \end{small}

    \subsection{{\market}, verification of the increments}
        \label{\SETLABEL:VI2}

        The data in table~\ref{\SETLABEL:ABS} is condensed from
        Section~\ref{\SETLABELREF:QVA}.

        \begin{small}
            \begin{table}[ht]
                \begin{center}
                    \caption[{\market}, verification of
                        the increments]{{\market}, verification the of
                        increments. In principle, the mean of the
                        absolute value of the increments and the root
                        mean square of the increments should
                        equate\footnote{The absolute value of the
                        normalized increments, when averaged, is
                        related to the root mean square of the
                        increments by a constant. If the normalized
                        increments are a fixed increment, the constant
                        is unity. If the normalized increments have a
                        Gaussian distribution, the constant is
                        $\approx 0.8$ depending on the accuracy of of
                        ``fit'' to a Gaussian distribution.}.}
                    \begin{tabular}{|c|c|} \hline
                        Mean of the               & rms \\
                        absolute value            & \hspace{0.01in} \\ \hline\hline
                        {\datafractionabsmean}    & {\datafractionrms} \\ \hline
                    \end{tabular}
                    \label{\SETLABEL:ABS}
                \end{center}
            \end{table}
        \end{small}

    \subsection{{\market}, $\chi^2$ values of the increments}
        \label{\SETLABEL:XSQ}

        The data in table~\ref{\SETLABEL:XSQT} is condensed from
        Section~\ref{\SETLABELREF:NH}.

        \begin{small}
            \begin{table}[ht]
                \begin{center}
                    \caption[{\market}, $\chi^2$ values of
                        the increments]{{\market}, $\chi^2$ values of
                        the increments. In principle, if the
                        distribution of the normalized increments is a
                        Gaussian distribution, the $\chi^2$ value will
                        be significantly less than the critical
                        value.}
                    \begin{tabular}{|c|c|} \hline
                        $\chi^2$      & Critical Value \\ \hline\hline
                        {\chisquared} & {\critical} \\ \hline
                    \end{tabular}
                    \label{\SETLABEL:XSQT}
                \end{center}
            \end{table}
        \end{small}

    \subsection{{\market}, time series data, empirical and simulated}
        \label{\SETLABEL:SIM}

        The data in table~\ref{\SETLABEL:ES} is condensed from
        Section~\ref{\SETLABELREF:TSUNFAIRBROWNIAN}.

        \begin{small}
            \begin{table}[ht]
                \begin{center}
                    \caption[{\market}, time series data, empirical
                        and simulated]{{\market}, time series data,
                        empirical and simulated, analysis of the
                        normalized increments.}
                    \begin{tabular}{|c|c|c|c|} \hline
                        \multicolumn{2}{|c|}{Empirical} & \multicolumn{2}{|c|}{Simulated}\\ \hline
                        Mean                & Standard              & Mean               & Standard \\
                        \hspace{0.01in}     & deviation             & \hspace{0.01in}    & deviation \\ \hline\hline
                        {\datafractionmean} & {\datafractionstddev} & {\tsunfairbrownianfractionmean} & {\tsunfairbrownianfractionstddev} \\ \hline
                    \end{tabular}
                    \label{\SETLABEL:ES}
                \end{center}
            \end{table}
        \end{small}

    \subsection{{\market}, number of participating companies}
        \label{\SETLABEL:QNC}

        The data in table~\ref{\SETLABEL:NC} is condensed from
        Section~\ref{\SETLABELREF:QNC}.

        \begin{small}
            \begin{table}[ht]
                \begin{center}
                    \caption[{\market}, number of participating
                        companies] {{\market}, number of participating
                        companies.}
                    \begin{tabular}{|c|c|} \hline
                        Number & Shannon probability\\ \hline
                        {\ncompanies} & {\pncompanies}\\ \hline
                    \end{tabular}
                    \label{\SETLABEL:NC}
                \end{center}
            \end{table}
        \end{small}

    \subsection{{\market}, Shannon probability optimizations}
        \label{\SETLABEL:SPO}

        The data in table~\ref{\SETLABEL:SP} is condensed from
        Section~\ref{\SETLABELREF:QNC}.

        \begin{small}
            \begin{table}[ht]
                \begin{center}
                    \caption[{\market}, Shannon probability
                         optimizations] {{\market}, Shannon
                         probability optimization.}
                    \begin{tabular}{|c|c|} \hline
                        optimize capital growth & optimize market growth\\ \hline
                        {\avgrms} & {\pncompanies}\\ \hline
                    \end{tabular}
                    \label{\SETLABEL:SP}
                \end{center}
            \end{table}
        \end{small}

% Local Variables:
% TeX-parse-self: t
% TeX-auto-save: t
% TeX-master: "fractal.tex"
% End:


% Local Variables:
% TeX-parse-self: t
% TeX-auto-save: t
% TeX-master: "fractal.tex"
% End:
