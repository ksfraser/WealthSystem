%
% -----------------------------------------------------------------------------
%
% A license is hereby granted to reproduce this software source code and
% to create executable versions from this source code for personal,
% non-commercial use.  The copyright notice included with the software
% must be maintained in all copies produced.
%
% THIS PROGRAM IS PROVIDED "AS IS". THE AUTHOR PROVIDES NO WARRANTIES
% WHATSOEVER, EXPRESSED OR IMPLIED, INCLUDING WARRANTIES OF
% MERCHANTABILITY, TITLE, OR FITNESS FOR ANY PARTICULAR PURPOSE.  THE
% AUTHOR DOES NOT WARRANT THAT USE OF THIS PROGRAM DOES NOT INFRINGE THE
% INTELLECTUAL PROPERTY RIGHTS OF ANY THIRD PARTY IN ANY COUNTRY.
%
% Copyright (c) 1994-2006, John Conover, All Rights Reserved.
%
% Comments and/or bug reports should be addressed to:
%
%     john@email.johncon.com (John Conover)
%
% -----------------------------------------------------------------------------
%
% Revision: \RCSRevision \\
% Revision Time: \RCSTime UMT \\
% Revision Date: \RCSDate \\
% Revision Id: \RCSId \\
% Revision File: \RCSLog \\
\RCS $Revision: 0.0 $
\RCS $Date: 2006/01/20 04:38:13 $
\RCS $Id: chap1.tex,v 0.0 2006/01/20 04:38:13 john Exp $
% $Log: chap1.tex,v $
% Revision 0.0  2006/01/20 04:38:13  john
% Initial version
%
%
\chapter{A Simple Industrial Market Model}
    \label{model}

    This chapter presents a simple industrial market model, and
    assumptions, that should, in principle, be applicable to analysis
    by fractal methodologies. It should be advised that the model
    and assumptions are very simple, and probably not adequately
    sophisticated for accurate or precision analysis of industrial
    markets. The attempt is to present a methodology that is
    applicable to analyzing market dynamics---which makes fractal
    methodologies the method of choice. Because of the simplistic
    models, it would be inappropriate to appropriate to consider this
    presentation financial advice.

    \section{Simplified Assumptions}
        \label{assumptions}

        The simple model is outlined as follows:

        \begin{itemize}

            \item The paradigm is that unless an organization updates
            goods and services rendered to an industrial market place,
            to contemporary standards, the organization's rate of
            revenue returns will decrease.

            \item However, deciding what must be done to update the
            goods and services rendered is a speculative process,
            requiring speculation on what the market place will favor
            in the future. It is assumed that the desires of the
            market place in the future is not predictable,  and
            has, at least to some extent, a degree of ``randomness.''

            \item It is assumed that the updating of the goods and
            services rendered is a continuing, iterated process that
            will constitute a significant portion of the
            organization's resources.

            \item The objective is presumed to be to maximize the rate
            of revenue returns.

        \end{itemize}

        Note that this ``model,'' albeit simple, is a ``prescription''
        for a fractal process. For a brief tutorial on fractal
        processes and analysis, see appendix~\ref{tutorial}.

        If it can be shown, and that remains to be seen, that
        industrial markets exhibit fractal characteristics, then there
        is a large existing mathematical and economic infrastructure
        that can be exploited to, perhaps, optimize industrial
        operations using entropic methodologies. The purpose of this
        manuscript is to offer a suggestion that this may, indeed, be
        the case---although with the un-sophisticated models and
        methodologies that are presented, there may be no intrinsic
        qualitative value.

    \section{Comparison of the Simplified Model with Previous Works}

        \cite[pp. 450]{Reza} citing J. L. Kelly,
        Jr\footnote{J. L. Kelly, Jr., ``{\it A New Interpretation of
        Information Rate,}\/'' Bell System Tech. J., vol. 35,
        pp. 917-926, 1956.} suggests an interesting model which
        presents the problem of the rate of transmission of
        information in a different way:

        \begin{quotation}

            Consider the case of a gambler with a private wire who
            places bets on the outcomes of a game of chance. We assume
            that the side information which he receives has a
            probability $p$ of being true and $1 - p$ of being
            false. Let the capital of the gambler be $V_0$ and $V_k$
            his capital after the $Kth$ betting. Since the gambler is
            not certain that the side information is entirely
            reliable, he places only a fraction $e$ of his capital on
            each bet. Thus, subsequent to $n$ bettings, assuming the
            independence of the successive tips \dots The problem with
            which the gambler is faced is the determination of $e$
            leading to the maximum of the average exponential rate of
            growth of his capital \dots Thus, under these rather
            natural hypotheses, the maximum possible average
            exponential gain of the gambler coincides with the
            numerical value of the channel capacity. If the channel
            were noiseless, the gambler would obviously risk all his
            capital at each betting \dots Also, if he knew the value
            of $p$ beforehand, he would be able to use this knowledge
            to his advantage and bet all his capital (or none). But
            the reliability of the tip is not known to him.

            According to Kelly, here we have an example of a real-life
            situation where considerations similar to the concept of
            source, channel, rate of transinformation, and channel
            capacity are valid. In the above reference, Kelly extends
            these results to more general cases of a gambler placing
            bets on outcomes of several games of chance. The gambler
            receives independent tips on each game conditional on the
            result of another game. The situation is analogous to a
            discrete independent source driving a discrete memoryless
            noisy channel.

            In conclusion, our acquired knowledge of information
            theory, which was based primarily on Shannon's
            communication model, can well be applied to other
            mathematical models arising from real-life problems.

        \end{quotation}

        By similar reasoning, the simple model outlined in
        Section~\ref{assumptions} has the same mechanism suggested by
        Kelly, however, the gambler's capital is market size, measured
        in revenue returns, and the ``tips'' come from the market
        itself with a probability, $p$ which can be computed from the
        fluctuations in revenue returns over time.

        \subsection{Compatability with more Sophisticated Economic Models}

            The proposed model is shown to be consistent as a first
            order approximation to the logistic function used in
            modeling industrial markets in the literature~\cite{Modis}
            in Chapter~\ref{general}, Section~\ref{nlextend}.

            More recently,~\cite[pp. 8]{Arthur:IRABR} argues that
            these types of scenarios are created by multiple agents
            acting on inadequate information in the marketplace. These
            agents form an inductive reasoning system that consists of
            a multitude of ``elements'' in the form of belief-modes or
            hypotheses that adapt to the aggregate economic ecological
            system. This concept is further developed in
            Chapter~\ref{general}, Section~\ref{GEN}.  Thus, it
            qualifies as an {\it adaptive complex}\/ system. These
            systems can be characterized by economic
            increasing-returns~\cite{Arthur:PFITE} of a dynamic
            process with random events---an economy based on positive
            feedback~\cite{Arthur:PFITE}. For general implications to
            the business environment, see~\cite{Arthur:IRATTWOB}. A
            system that consists of dynamic processes, with random
            events can be characterized as a random
            walk~\cite[pp. 6]{Arthur:CIEAFM},~\cite[pp. 6, 10,
            16]{Arthur:CTIRALIBHE},~\cite[pp. 29, 42]{Brock}
            fractal---identically as above. See Chapter~\ref{general},
            Section~\ref{nlextend} for additional comments.

    \section{Examples of Simplified Assumptions}

        As a simplified example of the assumptions presented in
        Section~\ref{assumptions}, consider a shoe manufacturer. The
        shoe company manufactures blue shoes, and red shoes. The
        operational agenda would require speculation on how many pairs
        of red shoes, and how many pairs of blue shoes should be
        manufactured each month, for sales, marketing and distribution
        next month. If the ``forecast'' for blue shoes is incorrect,
        say, perhaps, too large, then not only does the company loose
        the money invested in the manufacture of the blue shoes, but
        also looses market share to the companies that forecasted the
        demand for blue shoes correctly---and the company's rate of
        revenue returns would be expected to decrease next month. It
        is important to note that, at least as far as this simplified
        model is concerned, that the rate of revenue returns is
        actually a cumulative sum of all past decisions and
        investments in infrastructure, made by the company, in
        catering to the market place\footnote{The concept is
        subtile---that the derivative of the cumulative revenue
        returns, the rate of revenue returns, is actually a cumulative
        sum, or, in some sense, an integral of past market pro forma
        of the company. Note that this is reasonable, in some abstract
        sense, because if many bad forecasts were made in the past, we
        would expect the company's rate of revenue returns to be much
        smaller than a company that made more fortunate decisions. The
        attempt there is to establish an isomorphism between a
        company's rate of revenue returns and a cumulative sum process
        of a random variable---ie., a fractal. At least in principle,
        it is possible, using fractal analysis, to have some degree of
        confidence that a market's characteristics are
        fractal. See~\cite[pp. 244]{Crownover}.}

        As a related example, a company attempts a development project
        for a new product, and if the development is successful and
        well received in the market place, the company's rate of
        revenue returns increase.  If not, the company looses the
        money on the development investment, and in addition, the
        company's rate of revenue returns decrease.

        It should be pointed out that these examples illustrate the
        usage of a very simple ``model,'' to describe the operations
        of very complex industries, which is probably not be adequate
        for commercial purposes. Additionally, the model may have the
        disadvantage of abstracting the operational environment in
        such a manner that causality may be difficult to establish,
        and relating the model's parameters to accounting metrics may
        be difficult. However, there is some possibility that the
        methodology outlined in this manuscript can be used with
        Operations Research, as a forecasting methodology.

    \section{General Concepts in the Industrial Market Modeling Methodology}

        As a simple conceptual model, the industrial markets will be
        modeled as a simple tossed unfair coin game, which has
        characteristics of classical Brownian motion, as presented in
        appendix~\ref{tutorial}. The analytical derivation is
        presented in Chapter~\ref{general}, and the analytical fractal
        methodology used will is presented in
        Chapter~\ref{methodology}. A brief description of the software
        programs used in the implementation of the analytical fractal
        methodology is presented in appendix~\ref{programs}. The
        analysis of the markets is presented in
        appendix~\ref{markets}, and the concept of modeling the
        industrial market pro forma with classical Brownian motion
        expanded to models using fractional Brownian motion.

        The concept of using classical Brownian motion to model
        markets was chosen because it is the simplest of all fractal
        systems that can be used to analyze speculative
        markets. Naturally, the quality of such an analysis must be
        subject to appropriate scrutiny, and there are issues which
        can not be handled with such simple models. However, possibly,
        the simple models can be used to approximate industrial market
        process with accuracies that are adequate, hopefully, for
        months, and, possibly, but not likely, perhaps years.

        The classical Brownian motion model assumes that fixed
        increments will approximate the, probably Gaussian distributed
        increments, and simulations will be presented to offer a
        qualitative view of the accuracies, and inaccuracies, of such
        a simple models. For more in depth methods of addressing
        optimizations of markets that exhibit Gaussian distributed
        increments, the reader is referred to the bibliography.

        Chapter~\ref{conclusions} will offer several conclusions
        concerning the analysis presented in appendix~{markets}, and
        is offered in the context that there may be, perhaps, some
        value in using fractal analysis as a methodology, probably, at
        least, in a qualitative sense, in analyzing industrial
        markets. There is no insinuation that the methodology could be
        used for analyzing industrial markets in a qualitative sense
        at this time.

% Local Variables:
% TeX-parse-self: t
% TeX-auto-save: t
% TeX-master: "fractal.tex"
% End:
